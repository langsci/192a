\documentclass[output=paper,newtxmath,modfonts,nonflat,hidelinks]{langsci/langscibook} 

\title{Non-canonical switch-reference in Serer}

\author{Viktoria Apel\affiliation{Humboldt-Universität zu Berlin}}

\IfFileExists{../localcommands.tex}{%hack to check whether this is being compiled as part of a collection or standalone
  \usepackage{pifont}
\usepackage{savesym}

\savesymbol{downingtriple}
\savesymbol{downingdouble}
\savesymbol{downingquad}
\savesymbol{downingquint}
\savesymbol{suph}
\savesymbol{supj}
\savesymbol{supw}
\savesymbol{sups}
\savesymbol{ts}
\savesymbol{tS}
\savesymbol{devi}
\savesymbol{devu}
\savesymbol{devy}
\savesymbol{deva}
\savesymbol{N}
\savesymbol{Z}
\savesymbol{circled}
\savesymbol{sem}
\savesymbol{row}
\savesymbol{tipa}
\savesymbol{tableauxcounter}
\savesymbol{tabhead}
\savesymbol{inp}
\savesymbol{inpno}
\savesymbol{g}
\savesymbol{hanl}
\savesymbol{hanr}
\savesymbol{kuku}
\savesymbol{ip}
\savesymbol{lipm}
\savesymbol{ripm}
\savesymbol{lipn}
\savesymbol{ripn} 
% \usepackage{amsmath} 
% \usepackage{multicol}
\usepackage{qtree} 
\usepackage{tikz-qtree,tikz-qtree-compat}
% \usepackage{tikz}
\usepackage{upgreek}


%%%%%%%%%%%%%%%%%%%%%%%%%%%%%%%%%%%%%%%%%%%%%%%%%%%%
%%%                                              %%%
%%%           Examples                           %%%
%%%                                              %%%
%%%%%%%%%%%%%%%%%%%%%%%%%%%%%%%%%%%%%%%%%%%%%%%%%%%%
% remove the percentage signs in the following lines
% if your book makes use of linguistic examples
\usepackage{tipa}  
\usepackage{pstricks,pst-xkey,pst-asr}

%for sande et al
\usepackage{pst-jtree}
\usepackage{pst-node}
%\usepackage{savesym}


% \usepackage{subcaption}
\usepackage{multirow}  
\usepackage{./langsci/styles/langsci-optional} 
\usepackage{./langsci/styles/langsci-lgr} 
\usepackage{./langsci/styles/langsci-glyphs} 
\usepackage[normalem]{ulem}
%% if you want the source line of examples to be in italics, uncomment the following line
% \def\exfont{\it}
\usetikzlibrary{arrows.meta,topaths,trees}
\usepackage[linguistics]{forest}
\forestset{
	fairly nice empty nodes/.style={
		delay={where content={}{shape=coordinate,for parent={
					for children={anchor=north}}}{}}
}}
\usepackage{soul}
\usepackage{arydshln}
% \usepackage{subfloat}
\usepackage{langsci/styles/langsci-gb4e} 
   
% \usepackage{linguex}
\usepackage{vowel}

\usepackage{pifont}% http://ctan.org/pkg/pifont
\newcommand{\cmark}{\ding{51}}%
\newcommand{\xmark}{\ding{55}}%
 
 
 %Lamont
 \makeatletter
\g@addto@macro\@floatboxreset\centering
\makeatother

\usepackage{newfloat} 
\DeclareFloatingEnvironment[fileext=tbx,name=Tableau]{tableau}
  %add all your local new commands to this file
\newcommand{\downingquad}[4]{\parbox{2.5cm}{#1}\parbox{3.5cm}{#2}\parbox{2.5cm}{#3}\parbox{3.5cm}{#4}}
\newcommand{\downingtriple}[3]{\parbox{4.5cm}{#1}\parbox{3cm}{#2}\parbox{3cm}{#3}}
\newcommand{\downingdouble}[2]{\parbox{4.5cm}{#1}\parbox{6cm}{#2}}
\newcommand{\downingquint}[5]{\parbox{1.75cm}{#1}\parbox{2.25cm}{#2}\parbox{2cm}{#3}\parbox{3cm}{#4}\parbox{2cm}{#5}}
\newcolumntype{Y}{>{\centering\arraybackslash}X}
\newcolumntype{T}{>{\centering\arraybackslash}m{2cm}}

%commands for Kusmer paper below
\newcommand{\ip}{$\upiota$}
\newcommand{\lipm}{(\_{\ip-Max}}
\newcommand{\ripm}{)\_{\ip-Max}}
\newcommand{\lipn}{(\_{\ip}}
\newcommand{\ripn}{)\_{\ip}}
\renewcommand{\_}[1]{\textsubscript{#1}}


%commands for Pillion paper below
\newcommand{\suph}{\textipa{\super h}}
\newcommand{\supj}{\textipa{\super j}}
\newcommand{\supw}{\textipa{\super w}}
\newcommand{\ts}{\textipa{\t{ts}}}
\newcommand{\tS}{\textipa{\t{tS}}}
\newcommand{\devi}{\textipa{\r*i}}
\newcommand{\devu}{\textipa{\r*u}}
\newcommand{\devy}{\textipa{\r*y}}
\newcommand{\deva}{\textipa{\r*a}}
\renewcommand{\N}{\textipa{N}}
\newcommand{\Z}{\textipa{Z}}
% 

%commands for Diercks paper below
\newcommand{\circled}[1]{\begin{tikzpicture}[baseline=(word.base)]
\node[draw, rounded corners, text height=8pt, text depth=2pt, inner sep=2pt, outer sep=0pt, use as bounding box] (word) {#1};
\end{tikzpicture}
}

%commands for Pesetsky paper below
% \newcommand{\sem}[2][]{\mbox{$[\![ $\textbf{#2}$ ]\!]^{#1}$}}
\newcommand{\sem}[2][]{\mbox{$[[ $\textbf{#2}$ ]]^{#1}$}}

% \newcommand{\ripn}{{\color{red}ripn}}%this is used but never defined. Please update the definition



%commands for Lamont paper below
\newcommand{\row}[4]{
	#1. & 
    /{#2}/ & 
    [{#3}] & 
    `#4' \\ 
}
%\newcounter{tableauxcounter}
\newcommand{\tabhead}[2]{
%     \captionsetup{labelformat=empty}
%     \stepcounter{tableauxcounter}
%     \addtocounter{table}{-1}
% 	\centering
% 	\caption{Tableau \thetableauxcounter: #1}
	\caption{#1}
	\label{#2}
}
\newcommand{\candref}[2]{{(\ref{#1}#2)}}
\newcommand{\tableauref}[1]{{Tableau~\ref{#1}}}
% tableaux
\newcommand{\inp}[1]{\multicolumn{2}{|l||}{{#1}}}
\newcommand{\inpno}[1]{\multicolumn{2}{|l||}{#1}}
\newcommand{\g}{\cellcolor{lightgray}}
\newcommand{\hanl}{\HandLeft}
\newcommand{\hanr}{\HandRight}
\newcommand{\kuku}{Kuk\'{u}}

% \newcommand{\nocaption}[1]{{\color{red} Please provide a caption}}

% \providecommand{\biberror}[1]{{\color{red}#1}}

\definecolor{RED}{cmyk}{0.05,1,0.8,0}


\newfontfamily\amharicfont[Script = Ethiopic, Scale = 1.0]{AbyssinicaSIL}
\newcommand{\amh}[1]{{\amharicfont #1}}

% 
% %Gjersoe
\usepackage{textgreek}
% 
\newcommand{\viol}{\fontfamily{MinionPro-OsF}\selectfont\rotatebox{60}{$\star$}}
\newcommand{\myscalex}{0.45}
\newcommand{\myscaley}{0.65}
%\newcommand{\red}[1]{\textcolor{red}{#1}}
%\newcommand{\blue}[1]{\textcolor{blue}{#1}}
\newcommand{\epen}[1]{\colorbox{jgray}{#1}}
\newcommand{\hand}{{\normalsize \ding{43}}}
\definecolor{jgray}{gray}{0.8} 
\usetikzlibrary{positioning}
\usetikzlibrary{matrix}
\newcommand{\mora}{\textmu\xspace}
\newcommand{\si}{\textsigma\xspace}
\newcommand{\ft}{\textPhi\xspace}
\newcommand{\tone}{\texttau\xspace}
\newcommand{\word}{\textomega\xspace}
% \newcommand{\ts}{\texttslig}
\newcommand{\fns}{\footnotesize}
\newcommand{\ns}{\normalsize}
\newcommand{\vs}{\vspace{1em}}
\newcommand{\bs}{\textbackslash}   % backslash
\newcommand{\cmd}[1]{{\bf \color{red}#1}}   % highlights command
\newcommand{\scell}[2][l]{\begin{tabular}[#1]{@{}c@{}}#2\end{tabular}}
% \interfootnotelinepenalty=10000

% --- Snider Representations --- %

\newcommand{\RepLevelHh}{
\begin{minipage}{0.10\textwidth}
\begin{tikzpicture}[xscale=\myscalex,yscale=\myscaley]
%\node (syl) at (0,0) {Hi};
\node (Rt) at (0,1) {o};
\node (H) at (-0.5,2) {H};
\node (R) at (0.5,3) {h};
%\draw [thick] (syl.north) -- (Rt.south) ;
\draw [thick] (Rt.north) -- (H.south) ;
\draw [thick] (Rt.north) -- (R.south) ;
\end{tikzpicture}
\end{minipage}
}

\newcommand{\RepLevelLh}{
\begin{minipage}{0.10\textwidth}
\begin{tikzpicture}[xscale=\myscalex,yscale=\myscaley]
%\node (syl) at (0,0) {Mid2};
\node (Rt) at (0,1) {o};
\node (H) at (-0.5,2) {L};
\node (R) at (0.5,3) {h};
%\draw [thick] (syl.north) -- (Rt.south) ;
\draw [thick] (Rt.north) -- (H.south) ;
\draw [thick] (Rt.north) -- (R.south) ;
\end{tikzpicture}
\end{minipage}
}

\newcommand{\RepLevelHl}{
\begin{minipage}{0.10\textwidth}
\begin{tikzpicture}[xscale=\myscalex,yscale=\myscaley]
%\node (syl) at (0,0) {Mid1};
\node (Rt) at (0,1) {o};
\node (H) at (-0.5,2) {H};
\node (R) at (0.5,3) {l};
%\draw [thick] (syl.north) -- (Rt.south) ;
\draw [thick] (Rt.north) -- (H.south) ;
\draw [thick] (Rt.north) -- (R.south) ;
\end{tikzpicture}
\end{minipage}
}

\newcommand{\RepLevelLl}{
\begin{minipage}{0.10\textwidth}
\begin{tikzpicture}[xscale=\myscalex,yscale=\myscaley]
%\node (syl) at (0,0) {Lo};
\node (Rt) at (0,1) {o};
\node (H) at (-0.5,2) {L};
\node (R) at (0.5,3) {l};
%\draw [thick] (syl.north) -- (Rt.south) ;
\draw [thick] (Rt.north) -- (H.south) ;
\draw [thick] (Rt.north) -- (R.south) ;
\end{tikzpicture}
\end{minipage}
}

% --- Representations --- %

\newcommand{\RepLevel}{
\begin{minipage}{0.10\textwidth}
\begin{tikzpicture}[xscale=\myscalex,yscale=\myscaley]
\node (syl) at (0,0) {\textsigma};
\node (Rt) at (0,1) {o};
\node (H) at (-0.5,2) {\texttau};
\node (R) at (0.5,3) {\textrho};
\draw [thick] (syl.north) -- (Rt.south) ;
\draw [thick] (Rt.north) -- (H.south) ;
\draw [thick] (Rt.north) -- (R.south) ;
\end{tikzpicture}
\end{minipage}
}

\newcommand{\RepContour}{
\begin{minipage}{0.10\textwidth}
\begin{tikzpicture}[xscale=\myscalex,yscale=\myscaley]
\node (syl) at (0,0) {\textsigma};
\node (Rt) at (0,1) {o};
\node (H) at (-0.5,2) {\texttau};
\node (R) at (0.5,3) {\textrho};
\node (Rt2) at (1.5,1.0) {o};
%\node (H2) at (1.0,2) {$\tau$};
%\node (R2) at (2.0,2.5) {R};
\draw [thick] (syl.north) -- (Rt.south) ;
\draw [thick] (Rt.north) -- (H.south) ;
\draw [thick] (Rt.north) -- (R.south) ;
\draw [thick] (syl.north) -- (Rt2.south) ;
%\draw [thick] (Rt2.north) -- (H2.south) ;
%\draw [thick] (Rt2.north) -- (R2.south) ;
\end{tikzpicture}
\end{minipage}
}


% --- OT constraints --- %

\newcommand{\IllustrationDown}{
\begin{minipage}{0.09\textwidth}
\begin{tikzpicture}[xscale=0.7,yscale=0.45]
\node (reg) at (0,0.75) {{\small \textalpha}};
\node (arrow) at (0,0) {{\fns $\downarrow$}};
\node (Rt) at (0,-0.75) {{\small \textbeta}};
\end{tikzpicture}
\end{minipage}
}

\newcommand{\IllustrationUp}{
\begin{minipage}{0.09\textwidth}
\begin{tikzpicture}[xscale=0.7,yscale=0.45]
\node (reg) at (0,0.75) {{\small \textalpha}};
\node (arrow) at (0,0) {{\fns $\uparrow$}};
\node (Rt) at (0,-0.75) {{\small \textbeta}};
\end{tikzpicture}
\end{minipage}
}

\newcommand{\MaxAB}{
\begin{minipage}{0.09\textwidth}
\begin{tikzpicture}[xscale=0.6,yscale=0.4]
\node (max) at (0,0) {{\small \textsc{Max}}};
\node (reg) at (0.75,0.5) {{\fns \textalpha}};
\node (arrow) at (0.75,0) {{\tiny $\downarrow$}};
\node (Rt) at (0.75,-0.5) {{\fns \textbeta}};
\end{tikzpicture}
\end{minipage}
}

\newcommand{\DepAB}{
\begin{minipage}{0.09\textwidth}
\begin{tikzpicture}[xscale=0.6,yscale=0.4]
\node (max) at (0,0) {{\small \textsc{Dep}}};
\node (reg) at (0.75,0.5) {{\fns \textalpha}};
\node (arrow) at (0.75,0) {{\tiny $\downarrow$}};
\node (Rt) at (0.75,-0.5) {{\fns \textbeta}};
\end{tikzpicture}
\end{minipage}
}

\newcommand{\DepHReg}{
\begin{minipage}{0.055\textwidth}
\begin{tikzpicture}[xscale=0.6,yscale=0.4]
\node (dep) at (0,0) {{\small \textsc{Dep}}};
\node (reg) at (0,-1.0) {{\small h}};
\end{tikzpicture}
\end{minipage}
}

\newcommand{\DepLReg}{
\begin{minipage}{0.055\textwidth}
\begin{tikzpicture}[xscale=0.6,yscale=0.4]
\node (dep) at (0,0) {{\small \textsc{Dep}}};
\node (reg) at (0,-1.0) {{\small l}};
\end{tikzpicture}
\end{minipage}
}

\newcommand{\DepReg}{
\begin{minipage}{0.055\textwidth}
\begin{tikzpicture}[xscale=0.6,yscale=0.4]
\node (dep) at (0,0) {{\small \textsc{Dep}}};
\node (reg) at (0,-1.0) {{\small \textrho}};
\end{tikzpicture}
\end{minipage}
}

\newcommand{\DepTRt}{
\begin{minipage}{0.1\textwidth}
\begin{tikzpicture}[xscale=0.6,yscale=0.4]
\node (dep) at (0,0) {{\small \textsc{Dep}}};
\node (t) at (0.75,0.5) {{\fns \texttau}};
\node (arrow) at (0.75,0) {{\tiny $\downarrow$}};
\node (Rt) at (0.75,-0.5) {{\fns o}};
\end{tikzpicture}
\end{minipage}
}

\newcommand{\MaxRegRt}{
\begin{minipage}{0.1\textwidth}
\begin{tikzpicture}[xscale=0.6,yscale=0.4]
\node (max) at (0,0) {{\small \textsc{Max}}};
\node (arrow) at (0.75,0) {{\tiny $\downarrow$}};
\node (Rt) at (0.75,-0.5) {{\fns o}};
\node (reg) at (0.75,0.5) {{\fns \textrho}};
\end{tikzpicture}
\end{minipage}
}

\newcommand{\RegToneByRt}{
\begin{minipage}{0.06\textwidth}
\begin{tikzpicture}[xscale=0.6,yscale=0.5]
\node[rotate=20] (arrow1) at (-0.15,0) {{\fns $\uparrow$}};
\node[rotate=340] (arrow2) at (0.15,0) {{\fns $\uparrow$}};
\node (Rt) at (0,-0.55) {{\small o}};
\node (reg) at (0.4,0.55) {{\small \textrho}};
\node (tone) at (-0.4,0.55) {{\small \texttau}};
\end{tikzpicture}
\end{minipage}
}

\newcommand{\RegToneBySyl}{
\begin{minipage}{0.06\textwidth}
\begin{tikzpicture}[xscale=0.6,yscale=0.5]
\node[rotate=20] (arrow1) at (-0.15,0) {{\fns $\uparrow$}};
\node[rotate=340] (arrow2) at (0.15,0) {{\fns $\uparrow$}};
\node (Rt) at (0,-0.55) {{\small \textsigma}};
\node (reg) at (0.4,0.55) {{\small \textrho}};
\node (tone) at (-0.4,0.55) {{\small \texttau}};
\end{tikzpicture}
\end{minipage}
}

\newcommand{\DepTone}{
\begin{minipage}{0.055\textwidth}
\begin{tikzpicture}[xscale=0.6,yscale=0.4]
\node (dep) at (0,0) {{\small \textsc{Dep}}};
\node (tone) at (0,-1.0) {{\small \texttau}};
\end{tikzpicture}
\end{minipage}
}

\newcommand{\DepTonalRt}{
\begin{minipage}{0.055\textwidth}
\begin{tikzpicture}[xscale=0.6,yscale=0.4]
\node (dep) at (0,0) {{\small \textsc{Dep}}};
\node (tone) at (0,-1.0) {{\small o}};
\end{tikzpicture}
\end{minipage}
}

\newcommand{\DepL}{
\begin{minipage}{0.055\textwidth}
\begin{tikzpicture}[xscale=0.6,yscale=0.4]
\node (dep) at (0,0) {{\small \textsc{Dep}}};
\node (tone) at (0,-1.0) {{\small L}};
\end{tikzpicture}
\end{minipage}
}

\newcommand{\DepH}{
\begin{minipage}{0.055\textwidth}
\begin{tikzpicture}[xscale=0.6,yscale=0.4]
\node (dep) at (0,0) {{\small \textsc{Dep}}};
\node (tone) at (0,-1.0) {{\small H}};
\end{tikzpicture}
\end{minipage}
}

\newcommand{\NoMultDiff}{{\small *loh}}
\newcommand{\Alt}{{\small \textsc{Alt}}}
\newcommand{\NoSkip}{{\small \scell{\textsc{No}\\\textsc{Skip}}}}


\newcommand{\RegDomRt}{
\begin{minipage}{0.030\textwidth}
\begin{tikzpicture}[xscale=0.6,yscale=0.5]
\node (arrow) at (0,0) {{\fns $\downarrow$}};
\node (Rt) at (0,-0.55) {{\small o}};
\node (reg) at (0,0.55) {{\small \textrho}};
\end{tikzpicture}
\end{minipage}
}

\newcommand{\DepRegRt}{
\begin{minipage}{0.1\textwidth}
\begin{tikzpicture}[xscale=0.6,yscale=0.4]
\node (dep) at (0,0) {{\small \textsc{Dep}}};
\node (arrow) at (0.75,0) {{\tiny $\downarrow$}};
\node (Rt) at (0.75,-0.5) {{\fns o}};
\node (reg) at (0.75,0.5) {{\fns \textrho}};
\end{tikzpicture}
\end{minipage}
}

% unused

\newcommand{\ToneByRt}{
\begin{minipage}{0.05\textwidth}
\begin{tikzpicture}[xscale=0.6,yscale=0.5]
\node (arrow) at (0,0) {{\fns $\uparrow$}};
\node (Rt) at (0,-0.55) {{\small o}};
\node (tone) at (0,0.55) {{\small \texttau}};
\end{tikzpicture}
\end{minipage}
}

\newcommand{\RegByRt}{
\begin{minipage}{0.05\textwidth}
\begin{tikzpicture}[xscale=0.6,yscale=0.5]
\node (arrow) at (0,0) {{\fns $\uparrow$}};
\node (Rt) at (0,-0.55) {{\small o}};
\node (reg) at (0,0.55) {{\small \textrho}};
\end{tikzpicture}
\end{minipage}
}

\newcommand{\ToneDomRt}{
\begin{minipage}{0.05\textwidth}
\begin{tikzpicture}[xscale=0.6,yscale=0.5]
\node (arrow) at (0,0) {{\fns $\downarrow$}};
\node (Rt) at (0,-0.55) {{\small o}};
\node (tone) at (0,0.55) {{\small \texttau}};
\end{tikzpicture}
\end{minipage}
}

% --- OT tableaus --- %

% Sec. 3.2, first tabl.

\newcommand{\OTHLInput}{
\begin{minipage}{0.17\textwidth}
\begin{tikzpicture}[xscale=\myscalex,yscale=\myscaley]
\node (tone) at (2,0) {(= H)};
\node (syl) at (0,0) {\textsigma};
\node (Rt) at (0,1) {o};
\node (H) at (-0.5,2) {H};
\node (R) at (0.5,3) {h};
\node (Rt2) at (1.5,1.0) {o};
%\node (H2) at (1.0,2) {\epen{L}};
\node (R2) at (2.0,3) {\blue{l}};
\draw [thick] (syl.north) -- (Rt.south) ;
\draw [thick] (Rt.north) -- (H.south) ;
\draw [thick] (Rt.north) -- (R.south) ;
\draw [thick] (syl.north) -- (Rt2.south) ;
%\draw [dashed] (Rt2.north) -- (H2.south) ;
%\draw [dashed] (Rt2.north) -- (R2.south) ;
\end{tikzpicture}
\end{minipage}
}

\newcommand{\OTHLWinner}{
\begin{minipage}{0.17\textwidth}
\begin{tikzpicture}[xscale=\myscalex,yscale=\myscaley]
\node (tone) at (2,0) {(= HL)};
\node (syl) at (0,0) {\textsigma};
\node (Rt) at (0,1) {o};
\node (H) at (-0.5,2) {H};
\node (R) at (0.5,3) {h};
\node (Rt2) at (1.5,1.0) {o};
\node (H2) at (1.0,2) {\epen{L}};
\node (R2) at (2.0,3) {\blue{l}};
\draw [thick] (syl.north) -- (Rt.south) ;
\draw [thick] (Rt.north) -- (H.south) ;
\draw [thick] (Rt.north) -- (R.south) ;
\draw [thick] (syl.north) -- (Rt2.south) ;
\draw [dashed] (Rt2.north) -- (H2.south) ;
\draw [dashed] (Rt2.north) -- (R2.south) ;
\end{tikzpicture}
\end{minipage}
}

\newcommand{\OTHLSpreadingHOnly}{
\begin{minipage}{0.17\textwidth}
\begin{tikzpicture}[xscale=\myscalex,yscale=\myscaley]
\node (tone) at (2,0) {(= HM)};
\node (syl) at (0,0) {\textsigma};
\node (Rt) at (0,1) {o};
\node (H) at (-0.5,2) {H};
\node (R) at (0.5,3) {h};
\node (Rt2) at (1.5,1.0) {o};
%\node (H2) at (1.0,2) {\epen{L}};
\node (R2) at (2.0,3) {\blue{l}};
\draw [thick] (syl.north) -- (Rt.south) ;
\draw [thick] (Rt.north) -- (H.south) ;
\draw [thick] (Rt.north) -- (R.south) ;
\draw [thick] (syl.north) -- (Rt2.south) ;
\draw [dashed] (Rt2.north) -- (R2.south) ;
\draw [dashed] (Rt2.north) -- (H.south) ;
\end{tikzpicture}
\end{minipage}
}

\newcommand{\OTHLInsertH}{
\begin{minipage}{0.17\textwidth}
\begin{tikzpicture}[xscale=\myscalex,yscale=\myscaley]
\node (tone) at (2,0) {(= HM)};
\node (syl) at (0,0) {\textsigma};
\node (Rt) at (0,1) {o};
\node (H) at (-0.5,2) {H};
\node (R) at (0.5,3) {h};
\node (Rt2) at (1.5,1.0) {o};
\node (H2) at (1.0,2) {\epen{H}};
\node (R2) at (2.0,3) {\blue{l}};
\draw [thick] (syl.north) -- (Rt.south) ;
\draw [thick] (Rt.north) -- (H.south) ;
\draw [thick] (Rt.north) -- (R.south) ;
\draw [thick] (syl.north) -- (Rt2.south) ;
\draw [dashed] (Rt2.north) -- (H2.south) ;
\draw [dashed] (Rt2.north) -- (R2.south) ;
\end{tikzpicture}
\end{minipage}
}

\newcommand{\OTHLOverwriting}{
\begin{minipage}{0.17\textwidth}
\begin{tikzpicture}[xscale=\myscalex,yscale=\myscaley]
\node (syl) at (0,0) {\textsigma};
\node (Rt) at (0,1) {o};
\node (H) at (-0.5,2) {H};
\node (R) at (0.5,3) {h};
\node (Rt2) at (1.5,1.0) {o};
%\node (H2) at (1.0,2) {\epen{L}};
\node (R2) at (2.0,3) {\blue{l}};
\draw [thick] (syl.north) -- (Rt.south) ;
\draw [thick] (Rt.north) -- (H.south) ;
\draw [thick] (Rt.north) -- (R.south) ;
\draw [thick] (syl.north) -- (Rt2.south) ;
%\draw [dashed] (Rt2.north) -- (H2.south) ;
\draw [dashed] (Rt.north) -- (R2.south) ;
\node (del) at (0.3,1.9) {\textbf{=}};
\end{tikzpicture}
\end{minipage}
}

\newcommand{\OTHLSpreading}{
\begin{minipage}{0.17\textwidth}
\begin{tikzpicture}[xscale=\myscalex,yscale=\myscaley]
\node (syl) at (0,0) {\textsigma};
\node (Rt) at (0,1) {o};
\node (H) at (-0.5,2) {H};
\node (R) at (0.5,3) {h};
\node (Rt2) at (1.5,1.0) {o};
%\node (H2) at (1.0,2) {\epen{L}};
\node (R2) at (2.0,3) {\blue{l}};
\draw [thick] (syl.north) -- (Rt.south) ;
\draw [thick] (Rt.north) -- (H.south) ;
\draw [thick] (Rt.north) -- (R.south) ;
\draw [thick] (syl.north) -- (Rt2.south) ;
%\draw [dashed] (Rt2.north) -- (H2.south) ;
\draw [dashed] (Rt2.north) -- (H.south) ;
\draw [dashed] (Rt2.north) -- (R.south) ;
\end{tikzpicture}
\end{minipage}
}

% Sec. 4.2, second tabl.: phrase-medial position

\newcommand{\OTHnoLInput}{
\begin{minipage}{0.17\textwidth}
\begin{tikzpicture}[xscale=\myscalex,yscale=\myscaley]
\node (tone) at (2,0) {(= H)};
\node (syl) at (0,0) {\textsigma};
\node (Rt) at (0,1) {o};
\node (H) at (-0.5,2) {H};
\node (R) at (0.5,3) {h};
\node (Rt2) at (1.5,1.0) {o};
%\node (H2) at (1.0,2) {\epen{L}};
%\node (R2) at (2.0,3) {\blue{l}};
\draw [thick] (syl.north) -- (Rt.south) ;
\draw [thick] (Rt.north) -- (H.south) ;
\draw [thick] (Rt.north) -- (R.south) ;
\draw [thick] (syl.north) -- (Rt2.south) ;
\end{tikzpicture}
\end{minipage}
}

\newcommand{\OTHnoLEpenth}{
\begin{minipage}{0.17\textwidth}
\begin{tikzpicture}[xscale=\myscalex,yscale=\myscaley]
\node (tone) at (2,0) {(= HM)};
\node (syl) at (0,0) {\textsigma};
\node (Rt) at (0,1) {o};
\node (H) at (-0.5,2) {H};
\node (R) at (0.5,3) {h};
\node (Rt2) at (1.5,1.0) {o};
\node (H2) at (1.0,2) {\epen{L}};
\node (R2) at (2.0,3) {\epen{h}};
\draw [thick] (syl.north) -- (Rt.south) ;
\draw [thick] (Rt.north) -- (H.south) ;
\draw [thick] (Rt.north) -- (R.south) ;
\draw [thick] (syl.north) -- (Rt2.south) ;
\draw [dashed] (Rt2.north) -- (H2.south) ;
\draw [dashed] (Rt2.north) -- (R2.south) ;
\end{tikzpicture}
\end{minipage}
}

\newcommand{\OTHnoLSpreading}{
\begin{minipage}{0.17\textwidth}
\begin{tikzpicture}[xscale=\myscalex,yscale=\myscaley]
\node (tone) at (2,0) {(= HH)};
\node (syl) at (0,0) {\textsigma};
\node (Rt) at (0,1) {o};
\node (H) at (-0.5,2) {H};
\node (R) at (0.5,3) {h};
\node (Rt2) at (1.5,1.0) {o};
%\node (H2) at (1.0,2) {\epen{L}};
%\node (R2) at (2.0,3) {\blue{l}};
\draw [thick] (syl.north) -- (Rt.south) ;
\draw [thick] (Rt.north) -- (H.south) ;
\draw [thick] (Rt.north) -- (R.south) ;
\draw [thick] (syl.north) -- (Rt2.south) ;
\draw [dashed] (Rt2.north) -- (H.south) ;
\draw [dashed] (Rt2.north) -- (R.south) ;
\end{tikzpicture}
\end{minipage}
}

% Sec. 4.2, third tabl., LM is unaffected by L\%

\newcommand{\OTLMInput}{
\begin{minipage}{0.2\textwidth}
\begin{tikzpicture}[xscale=\myscalex,yscale=\myscaley]
\node (tone) at (2,0) {(= LM)};
\node (syl) at (0,0) {\textsigma};
\node (Rt) at (0,1) {o};
\node (H) at (-0.5,2) {L};
\node (R) at (0.5,3) {l};
\node (Rt2) at (1.5,1.0) {o};
\node (H2) at (1.0,2) {L};
\node (R2) at (2.0,3) {h};
\node (R3) at (3.0,3) {\blue{l}};
\draw [thick] (syl.north) -- (Rt.south) ;
\draw [thick] (Rt.north) -- (H.south) ;
\draw [thick] (Rt.north) -- (R.south) ;
\draw [thick] (syl.north) -- (Rt2.south) ;
\draw [thick] (Rt2.north) -- (H2.south) ;
\draw [thick] (Rt2.north) -- (R2.south) ;
\end{tikzpicture}
\end{minipage}
}

\newcommand{\OTLMReplace}{
\begin{minipage}{0.2\textwidth}
\begin{tikzpicture}[xscale=\myscalex,yscale=\myscaley]
\node (tone) at (2,0) {(= LL)};
\node (syl) at (0,0) {\textsigma};
\node (Rt) at (0,1) {o};
\node (H) at (-0.5,2) {L};
\node (R) at (0.5,3) {l};
\node (Rt2) at (1.5,1.0) {o};
\node (H2) at (1.0,2) {L};
\node (R2) at (2.0,3) {h};
\node (R3) at (3.0,3) {\blue{l}};
\draw [thick] (syl.north) -- (Rt.south) ;
\draw [thick] (Rt.north) -- (H.south) ;
\draw [thick] (Rt.north) -- (R.south) ;
\draw [thick] (syl.north) -- (Rt2.south) ;
\draw [thick] (Rt2.north) -- (H2.south) ;
\draw [thick] (Rt2.north) -- (R2.south) ;
\draw [dashed] (Rt2.north) -- (R3.south) ;
\node (del) at (1.8,2.1) {\textbf{=}};
\end{tikzpicture}
\end{minipage}
}

\newcommand{\OTLMTwoReg}{
\begin{minipage}{0.2\textwidth}
\begin{tikzpicture}[xscale=\myscalex,yscale=\myscaley]
\node (tone) at (2,0) {(= LML)};
\node (syl) at (0,0) {\textsigma};
\node (Rt) at (0,1) {o};
\node (H) at (-0.5,2) {L};
\node (R) at (0.5,3) {l};
\node (Rt2) at (1.5,1.0) {o};
\node (H2) at (1.0,2) {L};
\node (R2) at (2.0,3) {h};
\node (R3) at (3.0,3) {\blue{l}};
\draw [thick] (syl.north) -- (Rt.south) ;
\draw [thick] (Rt.north) -- (H.south) ;
\draw [thick] (Rt.north) -- (R.south) ;
\draw [thick] (syl.north) -- (Rt2.south) ;
\draw [thick] (Rt2.north) -- (H2.south) ;
\draw [thick] (Rt2.north) -- (R2.south) ;
\draw [dashed] (Rt2.north) -- (R3.south) ;
\end{tikzpicture}
\end{minipage}
}

% Sec. 4.2, fourth tabl., L is affected by L\% but M is not

\newcommand{\OTLInput}{
\begin{minipage}{0.17\textwidth}
\begin{tikzpicture}[xscale=\myscalex,yscale=\myscaley]
\node (tone) at (2,0) {(= L)};
\node (syl) at (0,0) {\textsigma};
\node (Rt) at (0,1) {o};
\node (H) at (-0.5,2) {L};
\node (R) at (0.5,3) {l};
\node (R2) at (2,3) {\blue{l}};
\draw [thick] (syl.north) -- (Rt.south) ;
\draw [thick] (Rt.north) -- (H.south) ;
\draw [thick] (Rt.north) -- (R.south) ;
\end{tikzpicture}
\end{minipage}
}

\newcommand{\OTLLowered}{
\begin{minipage}{0.17\textwidth}
\begin{tikzpicture}[xscale=\myscalex,yscale=\myscaley]
\node (tone) at (2,0) {(= LL)};
\node (syl) at (0,0) {\textsigma};
\node (Rt) at (0,1) {o};
\node (H) at (-0.5,2) {L};
\node (R) at (0.5,3) {l};
\node (R2) at (2,3) {\blue{l}};
\draw [thick] (syl.north) -- (Rt.south) ;
\draw [thick] (Rt.north) -- (H.south) ;
\draw [thick] (Rt.north) -- (R.south) ;
\draw [dashed] (Rt.north) -- (R2.south) ;
\end{tikzpicture}
\end{minipage}
}

\newcommand{\OTMInput}{
\begin{minipage}{0.17\textwidth}
\begin{tikzpicture}[xscale=\myscalex,yscale=\myscaley]
\node (tone) at (2,0) {(= M)};
\node (syl) at (0,0) {\textsigma};
\node (Rt) at (0,1) {o};
\node (H) at (-0.5,2) {L};
\node (R) at (0.5,3) {h};
\node (R2) at (2,3) {\blue{l}};
\draw [thick] (syl.north) -- (Rt.south) ;
\draw [thick] (Rt.north) -- (H.south) ;
\draw [thick] (Rt.north) -- (R.south) ;
\end{tikzpicture}
\end{minipage}
}

\newcommand{\OTMLowered}{
\begin{minipage}{0.17\textwidth}
\begin{tikzpicture}[xscale=\myscalex,yscale=\myscaley]
\node (tone) at (2,0) {(= ML)};
\node (syl) at (0,0) {\textsigma};
\node (Rt) at (0,1) {o};
\node (H) at (-0.5,2) {L};
\node (R) at (0.5,3) {h};
\node (R2) at (2,3) {\blue{l}};
\draw [thick] (syl.north) -- (Rt.south) ;
\draw [thick] (Rt.north) -- (H.south) ;
\draw [thick] (Rt.north) -- (R.south) ;
\draw [dashed] (Rt.north) -- (R2.south) ;
\end{tikzpicture}
\end{minipage}
}

% Sec. 4.2, fifth tableau, polar questions with level tones

\newcommand{\OTLPolIn}{
\begin{minipage}{0.20\textwidth}
\begin{tikzpicture}[xscale=\myscalex-0.05,yscale=\myscaley-0.05]
\node (tone) at (3.5,0) {(= L)};
\node (syl) at (0,0) {\textsigma};
\node (syl2) at (2,0) {\red{\textsigma}};
\node (Rt) at (0,1) {o};
\node (H) at (-0.5,2) {L};
\node (R) at (0.5,3) {l};
\node (Rt2) at (2,1) {\red{o}};
\draw [thick] (syl.north) -- (Rt.south) ;
\draw [thick,red] (syl2.north) -- (Rt2.south) ;
\draw [thick] (Rt.north) -- (H.south) ;
\draw [thick] (Rt.north) -- (R.south) ;
\end{tikzpicture}
\end{minipage}
}

\newcommand{\OTLPolDef}{
\begin{minipage}{0.20\textwidth}
\begin{tikzpicture}[xscale=\myscalex-0.05,yscale=\myscaley-0.05]
\node (tone) at (3.5,0) {(= L.M)};
\node (syl) at (0,0) {\textsigma};
\node (syl2) at (2,0) {\red{\textsigma}};
\node (Rt) at (0,1) {o};
\node (H) at (-0.5,2) {L};
\node (R) at (0.5,3) {l};
\node (H2) at (1.5,2) {\epen{L}};
\node (R2) at (2.5,3) {\epen{h}};
\node (Rt2) at (2,1) {\red{o}};
\draw [thick] (syl.north) -- (Rt.south) ;
\draw [thick,red] (syl2.north) -- (Rt2.south) ;
\draw [thick] (Rt.north) -- (H.south) ;
\draw [thick] (Rt.north) -- (R.south) ;
\draw [semithick,dashed] (Rt2.north) -- (H2.south) ;
\draw [semithick,dashed] (Rt2.north) -- (R2.south) ;
\end{tikzpicture}
\end{minipage}
}

\newcommand{\OTLPolAlt}{
\begin{minipage}{0.20\textwidth}
\begin{tikzpicture}[xscale=\myscalex-0.05,yscale=\myscaley-0.05]
\node (tone) at (3.5,0) {(= L.L)};
\node (syl) at (0,0) {\textsigma};
\node (syl2) at (2,0) {\red{\textsigma}};
\node (Rt) at (0,1) {o};
\node (H) at (-0.5,2) {L};
\node (R) at (0.5,3) {l};
\node (Rt2) at (2,1) {\red{o}};
\draw [thick] (syl.north) -- (Rt.south) ;
\draw [thick,red] (syl2.north) -- (Rt2.south) ;
\draw [thick] (Rt.north) -- (H.south) ;
\draw [thick] (Rt.north) -- (R.south) ;
\draw [semithick,dashed] (Rt2.north) -- (H.south) ;
\draw [semithick,dashed] (Rt2.north) -- (R.south) ;
\end{tikzpicture}
\end{minipage}
}

% Sec. 4.2, sixth tableau, polar questions with contour tones

\newcommand{\OTLLPolIn}{
\begin{minipage}{0.23\textwidth}
\begin{tikzpicture}[xscale=\myscalex-0.05,yscale=\myscaley-0.05]
\node (tone) at (5.2,0) {(= L)};
\node (syl) at (0,0) {\textsigma};
\node (syl3) at (3.4,0) {\red{\textsigma}};
\node (Rt) at (0,1) {o};
\node (Rt2) at (1.7,1) {o};
\node (Rt3) at (3.4,1) {\red{o}};
\node (H) at (-0.5,2) {L};
\node (R) at (0.5,3) {l};
\draw [thick] (syl.north) -- (Rt.south) ;
\draw [thick] (syl.north) -- (Rt2.south) ;
\draw [thick,red] (syl3.north) -- (Rt3.south) ;
\draw [thick] (Rt.north) -- (H.south) ;
\draw [thick] (Rt.north) -- (R.south) ;
\end{tikzpicture}
\end{minipage}
}

\newcommand{\OTLLPolDef}{
\begin{minipage}{0.23\textwidth}
\begin{tikzpicture}[xscale=\myscalex-0.05,yscale=\myscaley-0.05]
\node (tone) at (5.2,0) {(= L.M)};
\node (syl) at (0,0) {\textsigma};
\node (syl3) at (3.4,0) {\red{\textsigma}};
\node (Rt) at (0,1) {o};
\node (Rt2) at (1.7,1) {o};
\node (Rt3) at (3.4,1) {\red{o}};
\node (H) at (-0.5,2) {L};
\node (R) at (0.5,3) {l};
\node (H3) at (2.9,2) {\epen{L}};
\node (R3) at (3.9,3) {\epen{h}};
\draw [thick] (syl.north) -- (Rt.south) ;
\draw [thick] (syl.north) -- (Rt2.south) ;
\draw [thick,red] (syl3.north) -- (Rt3.south) ;
\draw [thick] (Rt.north) -- (H.south) ;
\draw [thick] (Rt.north) -- (R.south) ;
\draw [dashed] (Rt3.north) -- (H3.south) ;
\draw [dashed] (Rt3.north) -- (R3.south) ;
\end{tikzpicture}
\end{minipage}
}

\newcommand{\OTLLPolSkip}{
\begin{minipage}{0.23\textwidth}
\begin{tikzpicture}[xscale=\myscalex-0.05,yscale=\myscaley-0.05]
\node (tone) at (5.2,0) {(= L.L)};
\node (syl) at (0,0) {\textsigma};
\node (syl3) at (3.4,0) {\red{\textsigma}};
\node (Rt) at (0,1) {o};
\node (Rt2) at (1.7,1) {o};
\node (Rt3) at (3.4,1) {\red{o}};
\node (H) at (-0.5,2) {L};
\node (R) at (0.5,3) {l};
\draw [thick] (syl.north) -- (Rt.south) ;
\draw [thick] (syl.north) -- (Rt2.south) ;
\draw [thick,red] (syl3.north) -- (Rt3.south) ;
\draw [thick] (Rt.north) -- (H.south) ;
\draw [thick] (Rt.north) -- (R.south) ;
\draw [dashed] (Rt3.north) -- (H.south) ;
\draw [dashed] (Rt3.north) -- (R.south) ;
\end{tikzpicture}
\end{minipage}
}  
  
\newcommand{\ilit}[1]{#1\il{#1}}    
\newcommand{\isit}[1]{#1\is{#1}}  

\makeatletter
\let\thetitle\@title
\let\theauthor\@author 
\makeatother

\newcommand{\togglepaper}[1][0]{ 
  \bibliography{../localbibliography}
  %% hyphenation points for line breaks
%% Normally, automatic hyphenation in LaTeX is very good
%% If a word is mis-hyphenated, add it to this file
%%
%% add information to TeX file before \begin{document} with:
%% %% hyphenation points for line breaks
%% Normally, automatic hyphenation in LaTeX is very good
%% If a word is mis-hyphenated, add it to this file
%%
%% add information to TeX file before \begin{document} with:
%% \include{localhyphenation}
\hyphenation{
affri-ca-te
affri-ca-tes
com-ple-ments
par-a-digm
Sha-ron
Kings-ton
phe-nom-e-non
Daul-ton
Abu-ba-ka-ri
Ngo-nya-ni
Clem-ents 
King-ston
Tru-cken-brodt
Tab-leau
cophono-logies
mark-edness
Ti-gri-nya
a-mong
Car-stens
Lu-bu-ku-su
}
\hyphenation{
affri-ca-te
affri-ca-tes
com-ple-ments
par-a-digm
Sha-ron
Kings-ton
phe-nom-e-non
Daul-ton
Abu-ba-ka-ri
Ngo-nya-ni
Clem-ents 
King-ston
Tru-cken-brodt
Tab-leau
cophono-logies
mark-edness
Ti-gri-nya
a-mong
Car-stens
Lu-bu-ku-su
}
  \papernote{\scriptsize\normalfont
    \theauthor.
    \thetitle. 
    To appear in: 
    Emily Clem,   Peter Jenks \& Hannah Sande.
    Theory and description in African Linguistics: Selected papers from the 47th Annual Conference on African Linguistics.
    Berlin: Language Science Press. [preliminary page numbering]
  }
  \pagenumbering{roman}
  \setcounter{chapter}{#1}
  \addtocounter{chapter}{-1}
}

\newcommand{\upstep}{\textupstep}


% \newcounter{tableauxcounter}

\renewcommand{\textltailn}{ɲ}
\renewcommand{\textbardotlessj}{ɟ}

\newcommand{\emphkh}[1]{\textit{#1}} %originally \textbf, banned by the guidelines



\definecolor{lsDOIGray}{cmyk}{0,0,0,0.45}


\newcommand{\xuparrow}[1]{%
  {\left\uparrow\vbox to #1{}\right.\kern-\nulldelimiterspace}
}
\renewcommand \textupstep[1]{\char"A71B#1}
\renewcommand \textdownstep[1]{\char"A71C#1}
 
 \newcommand{\ꜛ}{\textsf{ꜛ}}
 
\def\biberror{\undefined}


\newcommand{\OTbox}[1]{\resizebox{.88\textwidth}{!}{#1}}
 
  \togglepaper[19]
}{}


\abstract{This paper takes a closer look at third-person pronouns in the Atlantic language Serer. In canonical affirmative clauses, the language disposes of two sets of non-locative subject pronouns. Previous descriptions of the language relate their distribution to conjugation paradigms on the one hand and/or to construction types on the other. However, an analysis of corpus data clearly contradicts these claims. The data rather provide evidence for a functional account of these pronouns relating their distribution to non-canonical switch-reference -- in the sense that it deviates from the definition of prototypical instances of the latter. This finding contributes to the description of variations of switch-reference systems in general as well as to a more accurate typological profile of Serer.}

\begin{document}

\maketitle

\section{Introduction}

\ili{Serer} is a North-Atlantic language of the Niger-Congo phylum \citep{Segerer2016} and is spoken by about 1.4 million people in Senegal and North-Western Gambia \citep{SimonsFenning2017}. As summarised in \citet[4]{Renaudier2012}, five dialects of \ili{Serer} can be distinguished: \ili{Serer}-\ili{Sine}, \ili{Serer} A’ool, \ili{Serer} Jegem, \ili{Serer} of Fadiouth and Palmarin, and \ili{Serer} \ili{Nyomiñka}. Of these five varieties, \ili{Serer}-\ili{Sine} and \ili{Serer} \ili{Nyomiñka} (Saloum region) are the most thoroughly described ones.\footnote{The data used in this paper are mostly taken from \citet{Faye1979} (\ili{Sine}) and \citet{Renaudier2012} (\ili{Nyomiñka}). In addition, examples were judged and provided with contexts by Papa Saliou Sarr who is a mother tongue speaker from the town Bambey (A’ool variety).}

One of the most prominent features of \ili{Serer}’s nominal morphosyntax is its \isi{noun class} system, which shows slight variation between dialects \citep[see][]{Renaudier2015}. Nouns are marked by a class prefix which in turn can trigger consonant mutation on the noun root \citep{Faye05,McLaughlin94,McLaughlin00,Merrill14,Pozdniakov2006}.

Noun class is indexed on a number of agreement targets such as determiner stems, adjectives, relative pronouns, and numerals up to five \citep[493]{Renaudier2015}.

Turning to the verb system, there are five slots for the composition of verb stems (see \citealt[][90]{FayeMous06}):

\ea\label{ex:apel:1}
\glt \textbf{root} – (derivational suffix(es)) – \textbf{conjugation} \textbf{suffix(es)} –   (\isi{pronoun}) – (relative perfective suffix \textit{-(ii)na})
\z

Finite verbs consist minimally of a root and one or more conjugation suffixes. Roots can hereby exhibit consonant mutation in order to distinguish singular from plural grammatical subjects \citep{McLaughlin94,McLaughlin00}. Conjugation suffixes are commonly divided into perfective and imperfective paradigms. For the sake of convenience, only the suffixes of perfective \textit{-a} \REF{ex:apel:2a} and imperfective -\textit{aa} \REF{ex:apel:2b} are distinguished in this paper.\footnote{All examples are unified in orthography and morpheme breaks. Regardless of the source language, glosses and translations are given uniquely in \ili{English}. Information which is irrelevant for this discussion is removed from the glosses. Singular/plural noun and verb roots are not distinguished. The numbering of noun classes follows \citet[118]{Faye1979}. Note that verb stems without any conjugation suffix are used as narrative perfectives.} 

\ea\label{ex:apel:2} 
\ea\label{ex:apel:2a}{\ili{Serer}-\ili{Sine} \citep[193]{Faye1979}}\\
\gll I pir\textbf{-a} ɓil le. \\
     \textsc{1pl} hit-\textsc{pfv} 5.stone 5.\textsc{def}\\
\glt ‘We hit against the stone.’ 

\ex\label{ex:apel:2b} {\ili{Serer}-\ili{Sine} \citep[217]{Faye1979}}\\
\gll I mbad\textbf{-aa} \\
     \textsc{1pl} beat\textsc{-}\textsc{ipfv}   \\
\glt ‘We beat [someone].’
\z
\z

In the examples in \REF{ex:apel:2} above, all information related to the finite verb is expressed on the verb. I refer to such verbs as “simple” verb forms. These can be differentiated from “complex” verb forms which are defined by the presence of an additional preverbal marker \REF{ex:apel:3a} or by a periphrastic construction involving a locative \isi{subject pronoun} \REF{ex:apel:3b}.

\ea\label{ex:apel:3} 
\ea\label{ex:apel:3a}{\ili{Serer}-\ili{Sine} \citep[217]{Faye1979}}\\
\gll \textbf{Ba} nu mbad.\\
     \textsc{imp.neg} \textsc{2pl} beat\\
\glt ‘Do not beat [someone]!’ 

\ex\label{ex:apel:3b} {\ili{Serer}-\ili{Sine} \citep[248]{Faye1979}}\\
\gll \textbf{Inwe} ngum-aa a-ndok.\\
     1\textsc{pl:loc} build-\textsc{ipfv} 3-hut   \\
\glt  ‘We are building a hut.’
\z
\z

Turning to the pronominal system, first and \isi{second person} \isi{subject} pronouns are either preverbal -- as in examples \REF{ex:apel:2} and \REF{ex:apel:3} -- or appear as enclitics on the verb stem. The enclitic vs. preverbal distribution depends on the person, number, and conjugation paradigm involved. The \isi{third-person} \isi{subject} pronouns are always preverbal. In combination with affirmative verb forms, \ili{Serer} has three \isi{third-person} \isi{subject} pronouns: \textit{a}, \textit{ta/te} and \textit{da/de}. \textit{Ta} and \textit{da} are the variants in the \ili{Sine} dialect. In \ili{Nyomiñka} they are realised as \textit{te} and \textit{de}. \textit{A} is used in both varieties. Whilst \textit{ta/te} and \textit{da/de} uniquely correspond to a singular or plural nouns respectively, \textit{a} is insensitive to number, as shown by \REF{ex:apel:4} for the \ili{Sine} variety:

\ea\label{ex:apel:4}
\ea\label{ex:apel:4a}\ili{Serer}-\ili{Sine} (\citealt{Faye1979}: 283; Papa Saliou Sarr, p.c.)\\
\gll a/ta/*da ret\\
     \textsc{pro/sg:pro/pl:pro} go \\
\glt ‘he/she/it went’ 

\ex\label{ex:apel:4b}{\ili{Serer}-\ili{Sine} (\citealt{Faye1979}: 277, 291; Papa Saliou Sarr, p.c.)}\\
\gll a/*ta/da ndet \\
     \textsc{pro/sg:pro/pl:pro} go  \\
\glt ‘they went’
\z
\z

Many authors relate the distribution of these three pronominal forms to conjugation paradigms and/or to construction types. In affirmative clauses with a non-focal \isi{subject}, the imperfective suffix -\textit{aa} is said to appear with \textit{ta/da} or \textit{te/de} only (\citealt{Faye1979}: 234; \citealt{Renaudier2012}: 347), as illustrated by \REF{ex:apel:5} for \textit{ta}.

\ea\label{ex:apel:5}{\ili{Serer}-\ili{Sine} \citep[283]{Faye1979}}\\
\gll \textbf{ta} ñaam\textbf{-aa}\\
     \textsc{sg:pro} eat\textsc{-ipfv}\\
\glt ‘she ate’
\z

However, this analysis is contradicted by data from the same text (a folktale), as shown in \REF{ex:apel:6} which is the next clause following example \REF{ex:apel:5}. Here, it is even the same verb stem that is preceded by the \isi{pronoun} \textit{a}.
\pagebreak

\ea\label{ex:apel:6}{\ili{Serer}-\ili{Sine} \citep[283]{Faye1979}}\\
\gll \textbf{a} ñaam\textbf{-aa}\\
     \textsc{pro} eat\textsc{-ipfv}\\
\glt ‘she ate’
\z

A similar pronominal distribution is asserted for the complex verb form involving the preverbal marker \textit{kaa} (example \REF{ex:apel:7} below) (\citealt[234]{Faye1979}; \citealt[91f]{FayeMous06}; \citealt[348]{Renaudier2012}). \textit{Kaa} appears in contexts where either the verb or the entire verb phrase is pragmatically in \isi{focus}. The interpretation of any type of term \isi{focus} -- such as \isi{subject}, object, adjunct, etc. -- is excluded.

\ea\label{ex:apel:7}{\ili{Serer}-\ili{Sine} (\citealt{Faye1979}: 196; context by Papa Saliou Sarr,   p.c.)}\\
{\-\hspace{0cm}\{Yoro bought a pagne.\}}\\
\gll   \textbf{Kaa} \textbf{ta} riw pay.\\
     \textsc{non.t.foc} \textsc{sg:pro} weave \textsc{6.}pagne\\
\glt ‘He WOVE a pagne.’ 
\z

However, natural discourse data, as in example \REF{ex:apel:8}, reveal that the \isi{pronoun} \textit{a} is grammatical in this construction type, too:

% Edwin - removed: from line 108 {Faye1979}]

\ea\label{ex:apel:8}
{\ili{Serer}-\ili{Sine} \citep[276]{Faye1979}}\\
{\-\hspace{0cm}\{The habitants of a village have to hide from soldiers under a bush. One woman betrays their shelter by not entering into the bush fast enough.\}}\\
\gll   \textbf{Kaa} \textbf{a} moof.\\
     \textsc{non.t.foc} \textsc{pro} sit.down \\
\glt ‘She SAT DOWN.’
\z

Examples \REF{ex:apel:5} to \REF{ex:apel:8} above show that conjugation paradigms and construction types are obviously not a decisive factor for the distribution of the non-locative preverbal third-person \isi{subject} pronouns. In the remainder of this paper I take a closer look at this phenomenon and argue for a new analysis. The argumentation is based on corpus data provided in the appendices of \posscitet{Faye05} and \posscitet{Renaudier2012} works. I start by examining the \isi{third-person} pronouns in \ili{Serer} in \sectref{sec:apel:2}. In addition to the description of form and function (\sectref{sec:apel:2.1}), I also present a hypothesis for the emergence of \textit{ta/da} and \textit{te/de} (\sectref{sec:apel:2.2}). I then turn to the distribution of \textit{a}, \textit{ta/te}, and \textit{da/de} in discourse (\sectref{sec:apel:2.3}). \sectref{sec:apel:3} deals with the theoretical classification of the phenomenon (\sectref{sec:apel:3.1}) as well as with the \isi{scope} and limits thereof (\sectref{sec:apel:3.2}). My findings are summarised in \sectref{sec:apel:4}.

\section{The third-person pronouns: A closer look}\label{sec:apel:2}

\subsection{Form and function of pronouns}\label{sec:apel:2.1}

As aforementioned, \ili{Serer} possesses three preverbal \isi{subject} pronouns for the third-person in combination with affirmative verb forms: \textit{a}, \textit{ta/te} and \textit{da/de}. Whilst the \isi{pronoun} \textit{a} is insensitive to number and substitutes nouns of all classes, \textit{ta/te} and \textit{da/de} differentiate between singular and plural referents. \textit{Ta/te} and \textit{da/de} share this property with other \isi{third-person} pronouns such as locative, object, possessive, and emphatic pronouns (see \tabref{tab:apel:1}).


\begin{table}
\begin{tabularx}{\textwidth}{Xllllll}
\lsptoprule
\bfseries Number & \multicolumn{3}{c}{ \bfseries Subject} & \bfseries Object & \bfseries Possessive & \bfseries Emphatic \\
& \multicolumn{2}{l}{\bfseries Non-locative} & \multicolumn{1}{l}{\bfseries Locative} &  & \\
\midrule
\small Sg. & & \textit{ta} (S) & \textit{oxe} & \textit{=(i)n/ne} (S) &  \textit{ten/um} (S) & \textit{ten}(S) \normalsize \\
 	& & \textit{te} (N) &   & \textit{=in/ten} (N) &   \textit{ten/=um} (N) & \textit{(o) ten} (N) \\
 & \textit{a} & & & & & \\
Pl. &  & \textit{da} (S) &   \textit{owe} & \textit{(a) den} & \textit{den} & \textit{den} \\
 & & \textit{de} (N) & 	 &	 	\\
\lspbottomrule
\end{tabularx}
\normalsize
\caption{third-person subject, object, possessive, and emphatic pronouns in Serer (\citealt{Faye1979}; \citealt{Renaudier2012}) (S=Serer-Sine, N=Serer Nyomiñka).}
\label{tab:apel:1}
\end{table}



Apart from the bipartite split in number that concerns all pronouns except \textit{a}, \tabref{tab:apel:1} shows that the pronouns \textit{ten} and \textit{den} are polyfunctional and appear as emphatic, possessive, and -- in the \ili{Nyomiñka} variety -- also as object pronouns. This degree of functional conflation reflects a general trend in \ili{Serer}’s nominal system, especially when compared to its closest linguistic relative \ili{Fula}. Not only does \ili{Serer} have fewer noun classes than Proto-\ili{Fula}-\ili{Serer} and present-day \ili{Fula} \citep{Merrill14}, its nouns display also less frequently an overt morphological affix for head noun marking than those in \ili{Fula}. Furthermore, \ili{Fula} has distinct pronouns in the third-person for each \isi{noun class}. Hence, compared to its closest relative, \ili{Serer} exhibits significant reductions in these domains. 

Turning again to the pronouns in \tabref{tab:apel:1}, their occurrence in the \isi{clause structure} is of course well determined. Object pronouns appear either as enclitics to the finite verb or they are simply postverbal.\footnote{The plural object \isi{pronoun} \textit{den} seems only to be preceded by the \isi{object marker} \textit{a} when the \isi{pronoun} refers to humans \citep[112-116]{Renaudier2012}.} Possessive pronouns are part of the \isi{noun phrase} and occur after their head.

The \isi{subject} and emphatic pronouns, on the other hand, can be differentiated with respect to the clausal field in which they occur. Within the field-based approach -- which provides a useful cross-linguistic (abstract) template for syntactic fields that are relevant for \isi{information structure} \citep[see][]{Good2010,Güldemanninprep,Apeletal15} -- the central field is the clause, as schematised in \REF{ex:apel:9}. It hosts the finite verb.

\ea\label{ex:apel:9} [Clause]
\z

Clause-internal constructions (as presented in all examples above) can be defined as single clauses. On the information-structural level, the canonical single clause has a topic-comment pattern. The grammatical \isi{subject} is interpreted as topic.\footnote{In this paper \textsc{topic} is defined as that entity in a sentence about which something is predicated (following \citealt{Strawson1964}; \citealt{Hornby1971}; \citealt{Dik1997}; \citealt{Reinhart1982}; \citealt{Lambrecht1994}).} The verb phrase represents the comment and hosts the \isi{focus} information.\footnote{Applying the functional framework, \textsc{focus} is defined as “that information which is relatively the most important or salient in the given communicative setting” \citep[326]{Dik1997}.}

The clause can be preceded by a topic field (see scheme 10 below). The topic field might host topical entities in contexts in which the topic shall be emphasised, i.e. for contrast or for signalling topic shift \citep[see][153]{Givon76}.

\ea\label{ex:apel:10}
[Topic] [Clause]
\z

One way of exploiting the topic field consists in placing a pragmatic argument therein via \isi{left-dislocation}. Dislocation involves a resumptive \isi{pronoun} in the thematic clause-internal position; this \isi{pronoun} is cross-referential with the dislocated entity \citep{GregoryMichaelis01,Lambrecht2001}. An example for \isi{left-dislocation} in \ili{Serer} is given in \REF{ex:apel:11} below. The emphatic \isi{pronoun} \textit{ten} is resumed by the preverbal \isi{subject pronoun} \textit{a}.\footnote{Note that \citet{Renaudier2012} analyses \textit{a} as an affix (see \sectref{sec:apel:2.2}).} 

\ea\label{ex:apel:11}{\ili{Serer} \ili{Nyomiñka} \citep[53]{Renaudier2012}} \\
\gll   [Ten]\textsubscript{Topic} a-ñaam-a maalo.\\
     \textsc{sg:emph} \textsc{pro-}eat\textsc{-pfv} \textsc{7.}rice \\
\glt ‘[As for] him, he ate rice.’
\z

In the next section I argue that this construction is the grammatical source of the pronouns \textit{ta/da} and \textit{te/de}.

\subsection{Emergence of ta/da and te/de}\label{sec:apel:2.2}

Before turning to pronominal \isi{subject} topics, it might be useful to review nominal \isi{subject} topics in \ili{Serer} first. Within the single clause, nominal grammatical \isi{subject} topics appear in a preverbal position. When the verb is conjugated in an affirmative paradigm, nominal subjects of all noun classes are obligatorily followed by the \isi{pronoun} \textit{a}, as illustrated by the two examples in \REF{ex:apel:12} below.\footnote{There is an asymmetry between affirmative and negative paradigms: with negative paradigms nominal grammatical subjects are not followed by \textit{a}. Note that focal pragmatic \isi{subject} noun phrases do not trigger the presence of \textit{a} either. The same is true for thetic statements in \posscitet{Sasse87} sense in which \textit{a} is ungrammatical, too.}

\ea\label{ex:apel:12}{\ili{Serer}-\ili{Sine} \citep[289]{Faye1979}}\\
\{The Tukulors, the \ili{Serer}, and the Juula are related.\}\\
\ea\label{ex:apel:12a}
\gll Dukloor we a ndef siriiñ.\\
     \textsc{2.}Tukulor \textsc{2.def} \textsc{pro} be \textsc{2.}Muslim\\
\glt ‘The Tukulors are Muslims.’

\ex\label{ex:apel:12b}
\gll \ili{Sereer} ke a yer-aa.\\
     \textsc{9.}\ili{Serer} \textsc{9.def} \textsc{pro} drink\textsc{-ipfv}\\
\glt ‘The \ili{Serer} are animists [i.e. not Muslims].’, lit. ‘The \ili{Serer} drink [alcohol].’
\z
\z

I assume that this canonical marking of clause-internal nominal \isi{subject} topics in \ili{Serer} is the result of the grammaticalisation of a \isi{left-dislocation} construction. The respective grammaticalisation path is schematised in \REF{ex:apel:13} below. In the \isi{left-dislocation} construction, the dislocated \isi{noun phrase} in the preclausal topic field -- which might also be set off prosodically by a pause (indicated by \#) -- is resumed clause-internally by an anaphoric \isi{subject pronoun}. After grammaticalisation the nominal topic is reinterpreted as a clause-internal grammatical \isi{subject}. Now the former \isi{subject pronoun} no longer functions as a \isi{pronoun} but expresses rather some sort of agreement with the (true) grammatical \isi{subject}.\footnote{This grammaticalisation path is cross-linguistically well attested; a similar development has been described for the \isi{subject} markers in \ili{Bantu} languages (\ili{Benue-Congo}) (see, e.g., \citealt{Morimoto2008}).}

\ea\label{ex:apel:13}
{Grammaticalisation path for nominal \isi{subject} topics \citep[adapted from][155]{Givon76}}\\
\glt [the man]\textsubscript{Topic} \# [he came]\textsubscript{Clause} > [the man he(-)came]\textsubscript{Clause}
\z

Taking the grammaticalisation path in \REF{ex:apel:13} above as a basis, the question arises as to the status of \ili{Serer}’s preverbal \textit{a} after grammaticalisation.\footnote{Thanks to the two anonymous reviewers for pointing out this question. The problem of distinguishing free from bound pronominal morphemes in African languages in general is discussed by \citet{Creissels2005typology}.} It is plausible to assume that in the presence of a nominal \isi{subject}, \textit{a} is a bound morpheme being part of the verb stem. Accordingly, the free \isi{pronoun} \textit{a} underwent grammaticalisation resulting in a bound (agreement) prefix. This analysis is adopted, i.a., by \citet{Renaudier2012}, \citet{Neely2013}, and \citet{Heath2014} who describe the \ili{Nyomiñka} variety. Interestingly, \citet{Faye1979} who provides a morpho-syntactic study of the \ili{Sine} dialect treats \textit{a} as a free weak \isi{pronoun} \citep[also][]{FayeMous06}. The different analysis of \textit{a} seems to reflect in fact its different stage of grammaticalisation in the dialects. Nevertheless, historically, it has most likely been a free morpheme in both language varieties.    

Departing from the presumption as sketched in \REF{ex:apel:13} above, the emergence of the \isi{subject} pronouns \textit{ta}/\textit{da} and \textit{te/de}, respectively, proceed along similar lines. In \ili{Serer}-\ili{Sine} \textit{ta} is probably the contracted form of the singular emphatic \isi{pronoun} \textit{ten} and the clause-internal \isi{pronoun} \textit{a} within the \isi{left-dislocation} construction; \textit{da} is the contracted form of the plural emphatic \isi{pronoun} \textit{den} and \textit{a}.\footnote{Special thanks to Lee Pratchett for this observation.} This path is illustrated in \REF{ex:apel:14} for \textit{ta}.

\ea\label{ex:apel:14}
{Emergence of \textit{ta} in \ili{Serer}-Sine}\\
\gll   [Ten]\textsubscript{Topic} [a ñaam-a maalo.]\textsubscript{Clause} > [Ta ñaam-a maalo.]\textsubscript{Clause}\\
     \textsc{sg:emph} \textsc{pro} eat\textsc{-pfv} \textsc{7.}rice {} \textsc{sg:pro} eat\textsc{-pfv} \textsc{7.}rice\\
\glt ‘[As for] him, he ate rice.’ > ‘He ate rice.’
\z

This hypothesis is supported by the observation that \textit{ta} and \textit{da} do not co-occur with \textit{a} in the \ili{Sine} dialect.

In \ili{Serer} \ili{Nyomiñka} the grammaticalisation seems to have led to the (probably optional) drop of the preverbal \textit{a} in conjunction with a phonological reduction of the emphatic \isi{pronoun}, resulting in \textit{te} and \textit{de} respectively. This development is sketched for \textit{te} in the next scheme.

\ea\label{ex:apel:15}
{Emergence of \textit{te} in \ili{Serer} Nyomiñka}\\
\gll   [Ten]\textsubscript{Topic} [a ñaam-a maalo.]\textsubscript{Clause} > [Te (a-)ñaam-a maalo.]\textsubscript{Clause}\\
     \textsc{sg:emph} \textsc{pro} eat\textsc{-pfv} \textsc{7.}rice {} \textsc{sg:pro} \textsc{pro-}eat\textsc{-pfv} \textsc{7.}rice\\
\glt ‘[As for] him, he ate rice.’ > ‘He ate rice.’
\z

The co-occurrence of \textit{te} and \textit{a} in this variety is recorded by \citet{Renaudier2012} and John Merrill (p.c.) and illustrates the further grammaticalisation of \textit{a} as a bound morpheme that functions as pure \isi{agreement marker}.\footnote{At the same time, the co-occurrence provides evidence for the analysis of \textit{ta/te} and \textit{da/de} as free morphemes which are unlikely additionally bound to the verb stem. In fact, the large majority of authors analyse \textit{ta/te} and \textit{da/de} as free pronouns.} Nevertheless, the historical account for the emergence of \textit{ta/da}, \textit{te/de}, and \textit{a} as sketched in \REF{ex:apel:14} and \REF{ex:apel:15} above is supported by their functional role which is \isi{subject} of the next section. 

\subsection{Distribution of non-locative third-person subject pronouns in discourse}\label{sec:apel:2.3}

This section investigates two examples from the corpora of \citet{Faye1979} and \citet{Renaudier2012} in order to exemplify the distribution of the pronouns \textit{a} and \textit{ta/da} or \textit{a} and \textit{te/de}, respectively. Starting with \REF{ex:apel:16} below from \citet{Faye1979} for \ili{Serer}-\ili{Sine}, this example consists of eleven clauses. It is taken from a folk tale in which a woman tries to kill her co-wife’s daughter by burying her alive. Luckily an eagle observes the woman’s actions. It digs out the child and raises her as its own.

\ea\label{ex:apel:16}
{\ili{Serer}-\ili{Sine} \citep[283]{Faye1979}}\\
{\{After she buried the child, she walked away and\}}
\ea\label{ex:apel:16a}
\gll a-qawooƈ ale a gar.\\
     \textsc{3-}eagle \textsc{3.def} \textsc{pro} come \\
\glt ‘the eagle came.’

\ex\label{ex:apel:16b}
\gll \textbf{A}  ut=in.\\
     \textsc{pro} dig.out\textsc{=sg.pro} \\
\glt ‘It [=the eagle] dug her [=the child] out.’

\ex\label{ex:apel:16c}
\gll \textbf{A} ret no mbuday ne no nqel ne.         \\
     \textsc{pro} go \textsc{prep} \textsc{6.}tree \textsc{6.def} \textsc{prep} 6.public.place \textsc{6.def} \\
\glt ‘It went to the tree [species] at the public place.’   

\ex\label{ex:apel:16d}
\gll \textbf{A} rang m-aaga.  \\
     \textsc{pro} build.nest \textsc{loc-}there \\
\glt ‘It built a nest there.’

\ex\label{ex:apel:16e}
\gll \textbf{A} geek m-aaga o-ƥiy onqa.  \\
     \textsc{pro} keep \textsc{loc-}there \textsc{12-}child \textsc{12.def}\\
\glt ‘It kept the child there.’


\ex\label{ex:apel:16f}
\gll \textbf{A} coox-a=n.\\
     \textsc{pro} give\textsc{-pfv=sg.pro}\\
\glt ‘It gave her [food].’


\ex\label{ex:apel:16g}
\gll \textbf{Ta} ñaam-aa.\\
     \textsc{sg:pro} eat\textsc{-ipfv} \\
\glt ‘She [=the girl] ate.’


\ex\label{ex:apel:16h}
\gll \textbf{A} ñaam-aa. \\
     \textsc{pro} eat\textsc{-ipfv}\\
\glt ‘She ate.’


\ex\label{ex:apel:16i}
\gll \textbf{A} ñaam-aa\\
     \textsc{pro} eat\textsc{-ipfv}\\
\glt ‘She ate’


\ex\label{ex:apel:16j}
\gll bo \textbf{a} maak.\\
     until \textsc{pro} grow\\
\glt ‘until she was big.’


\ex\label{ex:apel:16k}
\gll \textbf{Ta} waaƭ-aa wurus iin (…)\\
     \textsc{sg:pro} search.for\textsc{-ipfv} \textsc{7.}gold \textsc{1pl.poss}\\
\glt ‘It [=the eagle] looked for our gold (and our silver, everything that increases us).’
\z
\z

The first clause in \REF{ex:apel:16a} is a single \isi{main clause} with the nominal grammatical \isi{subject} topic \textit{aqawooƈ ale} ‘the eagle’. \textit{The eagle} has been introduced as a referent a couple of clauses before and is therefore definite. In clauses (\ref{ex:apel:16b}-\ref{ex:apel:16f}), \textit{the eagle} is substituted by the \isi{pronoun} \textit{a}. In clause \REF{ex:apel:16g} the singular \isi{subject pronoun} \textit{ta} appears. Pragmatically it refers to \textit{the girl} which is the topic of this clause. In (\ref{ex:apel:16h}-\ref{ex:apel:16j}) the \isi{subject pronoun} is again \textit{a} (still replacing \textit{the girl}). Finally, in \REF{ex:apel:16k} the \isi{pronoun} \textit{ta} is used which again substitutes \textit{the eagle}.

Before interpreting the example from \ili{Sere-Sine} above, it might be useful to also take a look at the \ili{Nyomiñka} variety. The six clauses of \REF{ex:apel:17} are part of a narrative on the relationship between the \ili{Nyomiñka} people and fishing.

% Edwin - removed: {Renaudier2012}]

\ea\label{ex:apel:17}
{\ili{Serer} \ili{Nyomiñka} \citep[356]{Renaudier2012}}
\ea\label{ex:apel:17a}
\gll   Na jamaano paap ke in a-mbaal-eeg-a mbaal.\\
     \textsc{prep} 7.epoch 9.father 9.\textsc{def} \textsc{1pl.poss} \textsc{pro}-fish-\textsc{pret-ipfv} fish\\
\glt ‘At this epoch, our fathers were fishing.’
 

\ex\label{ex:apel:17b}
\gll \textbf{A}-njeg suk.\\
     \textsc{pro}-have \textsc{9.}boat \\
\glt ‘They had boats.’
 

\ex\label{ex:apel:17c}
\gll \textbf{A}-ngaad-oox-a.         \\
     \textsc{pro-}leave-\textsc{midd-pfv} \\
\glt ‘They were nomads.’   
 
\ex\label{ex:apel:17d}
\gll Gi-ndiig a-joot-ang-a,  \\
     \textsc{6-}rainy.season \textsc{pro-}cross\textsc{-hyp-pfv}    \\
\glt ‘When the rainy season passed,’
 

\ex\label{ex:apel:17e}
\gll \textbf{de} iid-ik.  \\
     \textsc{pl:pro} leave.at.dry.season-\textsc{dir}\\
\glt ‘they went during the dry season.’
 

\ex\label{ex:apel:17f}
\gll \textbf{A-}njeg laalaf.\\
     \textsc{pro-}have ambition\\
\glt ‘They had ambition.’
\z
\z

In the first clause in \REF{ex:apel:17a}, the \isi{noun phrase} \textit{paap ke in} ‘our fathers’ is the grammatical \isi{subject} of the verb \textit{mbaaleega} \textit{mbaal} ‘were fishing’.\footnote{Reduplication in \ili{Serer} is discussed by \citet{Heath2014}.} The presence of the prefixed \isi{pronoun} \textit{a} signals the topical status of that \isi{noun phrase} (see \sectref{sec:apel:2.2}). In the next two clauses in \REF{ex:apel:17b} and \REF{ex:apel:17c}, the \isi{pronoun} \textit{a} both times substitutes \textit{our fathers}. In the subsequent \isi{subordinate clause} in \REF{ex:apel:17d}, the noun \textit{gindiig} ‘rainy season’ represents the topical \isi{subject}. Then in \REF{ex:apel:17e} the plural \isi{subject pronoun} \textit{de} occurs which again substitutes \textit{our fathers}. The same \isi{noun phrase} is referred to by \textit{a} in the final clause in \REF{ex:apel:17f}.

The examples \REF{ex:apel:16} and \REF{ex:apel:17} above suggest that the distribution of the \isi{subject} pronouns \textit{a} and \textit{ta/da} or \textit{te/de}, respectively, is linked to the nominal referent that the \isi{pronoun} substitutes. \textit{A} is used whenever it is coreferential with the \isi{subject} of the preceding clause, i.e. when there is topic continuity on the information-structural level. If the two subjects have \isi{disjoint} referents -- i.e. in case of topic change -- in the second clause \textit{ta} or \textit{te} in the singular or \textit{da} or \textit{de} in the plural are used.\footnote{This distribution demonstrates that topic and \isi{subject} are overlapping concepts. Whilst topics operate on the information-structural level, subjects operate on the syntactic level. In an unmarked sentence, the grammatical \isi{subject} is by default the sentence topic.} In the next section I relate these findings on the pragmatic and information-structural level to the grammatical device \isi{switch-reference} which is used for reference tracking.

\section{Non-canonical switch-reference}\label{sec:apel:3}

\subsection{Theoretical classification of the phenomenon in Serer}\label{sec:apel:3.1}

In the past, canonical \textsc{switch-reference} has been described mainly in American, Australian, and Papuan languages \citep{HaimanMunro83}. Recent research, however, shows that \isi{switch-reference} is also found on the African continent.\footnote{Prototypical \isi{switch-reference} is for instance described by \citet{Treis2012} for \ili{Omotic} and \ili{Cushitic} languages (\ili{Afro-Asiatic}) in South-Western Ethiopia.}  Prototypically, it defines constructions in which “a marker on the verb of one clause is used to indicate whether its \isi{subject} has the same or different reference from the \isi{subject} of an adjacent, syntactically related clause” \citep[1]{Stirling1993}. On the functional level, it is “a device for referential tracking” in order to avoid ambiguity \citep[xi]{HaimanMunro83}. An often-cited example from \ili{Mojave}, a Cochimí-Yuman language spoken in the South West of the United States is given in \REF{ex:apel:18} below. In \REF{ex:apel:18a} the subjects in the main and subordinate clause have both the same referent (same subject, SS). This is signalled by the suffix \textit{-k} which replaces the tense marking on the first verb. In \REF{ex:apel:18b} the referents of the two subjects differ (different subject, DS). This is indicated by the suffix \textit{-m} on the first verb.

\ea\label{ex:apel:18}
{\ili{Mojave} (\citealt{Munro1980}: 145, in \citealt{Stirling1993}: 3)}
\ea\label{ex:apel:18a}
\gll   Nya-isvar-\textbf{k}, iima-k.\\
     when\textsc{-}sing\textsc{-ss} dance\textsc{-tns} \\
\glt ‘When he\textsubscript{i} sang, he\textsubscript{i} danced.’

\ex\label{ex:apel:18b}
\gll Nya-isvar-\textbf{m}, iima-k.\\
     when\textsc{-}sing\textsc{-ds} dance\textsc{-tns} \\
\glt ‘When he\textsubscript{i} sang, he\textsubscript{j} danced.’
\z
\z

Cross-linguistically, \isi{switch-reference} marking is more likely to be found with \isi{third-person} subjects than with first or second persons; in some languages \isi{switch-reference} is even limited to the \isi{third-person} \citep[xi]{HaimanMunro83}. As the data in \sectref{sec:apel:2.3} suggest, \ili{Serer} can be aligned with such languages.

However, \ili{Serer} does not have a canonical \isi{switch-reference} system because switch between referents is not marked by verb morphology but by free pronouns. In the literature, pronominal marking in relation to \isi{switch-reference} is discussed under the term \textsc{logophoricity}.\footnote{A full discussion of the differences between the two reference tracking devices \isi{switch-reference} and logophoricity is provided by \citet[50-56]{Stirling1993}.} It is defined by \citet[1]{Stirling1993} as follows: “in central cases of logophoricity, a special \isi{pronoun} form is used within a reported speech context, to indicate coreference with the source of the reported speech”. In contrast to canonical \isi{switch-reference}, logophoric systems have been described for various West-African languages, e.g. \ili{Ewe} (\ili{Gbe}) in Ghana and Togo, \ili{Kera} (Chadic) in Chad and Cameroon, or \ili{Igbo} (\ili{Benue-Congo}) in Nigeria (\textit{ibid}.: 311). Logophoricity in \ili{Igbo} is illustrated in \REF{ex:apel:19} below. The \isi{third-person} \isi{pronoun} in the \isi{complement clause} is \textit{yá} when it has the same referent as the \isi{pronoun} in the \isi{main clause}. When it has a different referent, the \isi{pronoun} in the \isi{complement clause} is \textit{ọ}.

\ea\label{ex:apel:19}
{\ili{Igbo} \citep[19]{HymanComrie81}}
\ea\label{ex:apel:19a}
\gll Ọ́ s\`ịr\`i   nà  \textbf{yá} byàrà. \\
     he said that he.\textsc{ss} came \\
\glt ‘He\textsubscript{i} said that he\textsubscript{i} came.’


\ex\label{ex:apel:19b}
\gll Ọ́ sị̀rì  nà  \textbf{ọ́} byàrà.\\
     he said that he.\textsc{ds} came \\
\glt ‘He\textsubscript{i} said that he\textsubscript{j} came.’
\z
\z

Thus two main characteristics distinguish prototypical \isi{switch-reference} from prototypical logophoricity: 

\begin{enumerate}
\item the location of marking, i.e. verb vs. \isi{pronoun}, and 

\item the syntactic and semantic context of marking, i.e. unspecified adjacent clause vs. \isi{embedded clause} in a reported speech context.
\end{enumerate}

Applying the two definitions above to the non-locative \isi{third-person} \isi{subject} pronouns in \ili{Serer}, it becomes evident that these pronouns are in between the two. On the one hand, they resemble logophoric pronouns because they are pronominal. On the other hand, their occurrence is open to different types of adjacent clauses and is not restricted to contexts of reported speech. Because of the non-restriction of syntactic and semantic context, I relate these pronouns to \textsc{non-canonical switch-reference} -- in the sense that the system under discussion deviates from the definition of archetypal \isi{switch-reference}.\footnote{The term \textsc{switch-reference} in relation to the pronouns \textit{te/de} has been firstly mentioned by \citet{Neely2013}: “Kaa shares this paradigm [=incl. the \isi{third-person} pronouns \textit{te} and \textit{de}, VA] with certain types of subordinate clauses (particularly relative clauses), and clauses where \isi{switch-reference} is indicated.”} 

Non-canonical systems are also found in languages that mark \isi{switch-reference} by clausal coordinators, such as in \ili{Fon} (\ili{Gbe}) from Benin and Nigeria \citep[113f]{LefebvreBrousseau02} or \ili{Supyire} (\ili{Senufo}) from Mali \citep[602ff]{Carlson1994}. On the other hand, there are also languages that mark logophoricity by affixes on the verb, e.g. Gokana (\ili{Benue-Congo}) from Nigeria \citep{HymanComrie81}. As a consequence, cross-linguistically there might be a lot of variation that operates in between these two reference tracking categories.

However, to my knowledge, \isi{switch-reference} pronouns are cross-linguistically uncommon and have only been described for a few languages, amongst which are \ili{Bafut} (Grassfields) from Cameroon \citep[53]{Wiesemann82}, \ili{Kaulong} (Oceanic) from Papua New Guinea \citep[391]{Crowleyetal11}, and \ili{Yiddish} (\ili{Germanic}) \citep[311]{Prince2006}. Whilst in \ili{Bafut} the \isi{switch-reference} marking of subjects is restricted to \isi{consecutive} clauses, in \ili{Kaulong} it is restricted to the marking of the possessive \isi{pronoun}. The data from \ili{Yiddish} show a situation somewhat comparable to the one in \ili{Serer} because \isi{switch-reference} operates across \isi{main clause} boundaries. As the two examples in \REF{ex:apel:20} below reveal, “\ili{Yiddish} has a pronominal form for \isi{switch-reference}, \textit{yener} ‘that [one]’ which is used to refer to something other than the Cp [preferred centre; here: topic of the preceding clause, VA] of the previous utterance” \citep[311]{Prince2006}. Thus, in \REF{ex:apel:20a}, the \isi{subject pronoun} is \textit{er} when it is coreferential with the \isi{subject} of the preceding clause. When the two subjects have a \isi{disjoint} referent, the \isi{pronoun} \textit{yener} is used in the second clause \REF{ex:apel:20b}.

\ea\label{ex:apel:20}
{\ili{Yiddish} \citep[311]{Prince2006}}\\
\ea\label{ex:apel:20a}
{\{A guy\textsubscript{i} had to meet a certain Rubinstein\textsubscript{j} on the train.\}}\\
\gll Iz \textbf{er} arumgegangen oyfn peron. “[...]”.\\
     is he.\textsc{ss} went.around on:the platform \\
\glt ‘So he\textsubscript{i} walked around on the platform “[...]”.’

\ex\label{ex:apel:20b}
{\{A guy\textsubscript{i} once asked a friend\textsubscript{j} of his: “[...]”.\}}\\
\gll Makht \textbf{yener} “[...]”.\\
     makes that.one.\textsc{ds} \\
\glt ‘That one\textsubscript{j} says: “[...]”.’, lit. ‘That one\textsubscript{j} makes: “[...]”.’
\z
\z

At a first glance, \textit{er} and \textit{yener} in \ili{Yiddish} have a similar distribution as \textit{a} and \textit{ta/te/da/de} and in \ili{Serer}. However, the \ili{Yiddish} pronouns differ in (at least) two aspects. Firstly, it is unclear whether \textit{yener} consistently marks \isi{switch-reference} over a longer string of text as is the case for \textit{ta/te/da/de}. Secondly, \textit{yener} has a deictic semantic content. Naturally, pronouns expressing special deixis ‘this one, that one’ or ‘the other’ are associated with referent switch (or topic change) because of their potential contrastive implicature. Although the respective pronouns in \ili{Serer} do not have such a specific semantic content, they are also related to contrast. This is demonstrated in \sectref{sec:apel:2.2} where I suggest that these pronouns arose from emphatic pronouns in a \isi{left-dislocation} construction which is inherently associated with contrast \citep[153]{Givon76}.

In the next section, I define the \isi{scope} of the non-canonical \isi{switch-reference} system in \ili{Serer} and present some puzzling cases, before summarising the results in \sectref{sec:apel:4}. 

\subsection{Scope and limits}\label{sec:apel:3.2}

The analysis of the available corpus data reveals the following: 

\begin{itemize}
\item \isi{switch-reference} in \ili{Serer} is restricted to non-locative \isi{third-person} \isi{subject} pronouns and affirmative clauses;

\item these pronouns are the grammatical \isi{subject} and represent the pragmatic topic of the clause;

\item \isi{switch-reference} operates across sequential clause boundaries -- such as in a sequence of pragmatic dependent clauses in narratives.
\end{itemize}

“Same \isi{subject}” is expressed pronominally by the \isi{pronoun} \textit{a}.\footnote{Rarely a zero \isi{pronoun} is recorded, too.}  “Different \isi{subject}” is either expressed by the use of the lexical noun or by the \isi{pronoun} \textit{ta/te} in the singular and \textit{da/de} in the plural.

In \ili{Serer}-\ili{Sine}, \isi{switch-reference} marking is also extended to the \isi{third-person} markers \textit{tee} (sg.) and \textit{dee} (pl.). \textit{Tee} and \textit{dee} are contracted forms of the pronouns \textit{ta} and \textit{da} and the complementiser \textit{ee}. One of the functions of this complementiser is to introduce \isi{direct speech}. An example for the use of \textit{tee} is given in \REF{ex:apel:21} where \textit{tee} signals \isi{switch-reference} with respect to the \isi{subject} of the preceding affirmative clause. 

\ea\label{ex:apel:21}
{\ili{Serer}-\ili{Sine} \citep[285]{Faye1979}}\\
{\-\hspace{0cm}\{He\textsubscript{i} said: “Is this one your mother?”\}}\\
\gll \textbf{Tee} “haʔa”.\\
     \textsc{sg:comp.ds} no \\
\glt ‘She\textsubscript{j} said: “No.”’
\z

When \isi{direct speech} is announced without referent switch, the expected \isi{pronoun} \textit{a} is used, as illustrated in \REF{ex:apel:22}. 

\ea\label{ex:apel:22}
{\ili{Serer}-\ili{Sine} \citep[284]{Faye1979}}\\
{\-\hspace{0cm}\{He\textsubscript{i} shaved her skull.\}}\\
\gll \textbf{A} lay=in ee: “Gayk-i kellem ke fa xa-paam axe!”\\
     \textsc{pro.ss} say\textsc{=sg.pro} \textsc{comp} herd\textsc{-sg.imp} \textsc{9.}camel \textsc{9.def} and \textsc{11-}donkey \textsc{11.def} \\
\glt ‘He\textsubscript{i} said to her: “Herd the camels and donkeys!”’
\z

Nevertheless, there are some puzzling exceptional instances of unexpected “same \isi{subject}” or “different \isi{subject}” marking in the corpus. An example of the latter is given in \REF{ex:apel:23} below. Although there is no referent switch across the clause boundary, the “different \isi{subject}” \isi{pronoun} \textit{ta} occurs instead of the expected “same \isi{subject}” \isi{pronoun} \textit{a}.

\ea\label{ex:apel:23}
{\ili{Serer}-\ili{Sine} \citep[284]{Faye1979}}\\
{\-\hspace{0cm}\{He\textsubscript{i} spent the day at the public place.\}}\\
\gll   \textbf{Ta} lay=in: “[...]”. \\
     \textsc{sg:pro} say\textsc{=sg.pro} \\
\glt ‘He\textsubscript{i} said to her: “[...]”.’
\z

\citet[98-114]{Stirling1993} discusses such striking cases in different languages and argues that different \isi{subject} marking might also express discontinuity on a pragmatic or semantic discourse level. Despite this appealing explanation, this does not seem to hold in example \REF{ex:apel:23} above because this clause is both syntactically and pragmatically dependent within the narrative. Thus, there is no interruption or discontinuity from a pragmatic perspective here. For this and other reasons, further research is necessary to shed light on these exceptional cases.

Another domain which would benefit from deeper analysis is \isi{impersonal} constructions. Here, the data provide no clear picture with respect to the use of the \isi{subject} pronouns.

Last but not least, more investigation is needed on clausal co- and \isi{subordination}. This applies to complement and adverbial clauses in particular as the present corpus was insufficient to draw meaningful conclusions on \isi{switch-reference} in such contexts. Relative clauses are an exception because they show a clear restriction. Here, only the “different \isi{subject}” pronouns \textit{ta/te} and \textit{da/de} are grammatical, as illustrated in \REF{ex:apel:24} below for the singular in combination with the perfective relative \textit{-na}. The referential status of the \isi{subject pronoun} is disregarded.

\newpage 
\ea\label{ex:apel:24}
{\ili{Serer} \ili{Nyomiñka} \citep[350]{Renaudier2012}}\\
{\-\hspace{0cm}\{The same antelope\textsubscript{i} fell into the ocean. She\textsubscript{i} landed here.\}} \\
\gll   Ye \textbf{te} jees-iid-\textbf{na} m-eeke it, “(...)“.\\
     when \textsc{sg:pro.ds} arrive\textsc{-dir-rel} \textsc{loc-}there also \\
\glt ‘When she\textsubscript{i} arrived here, (they waited until the next day).’
\z

\section{Summary}\label{sec:apel:4}

In this paper, I have presented and discussed evidence of a non-canonical \isi{switch-reference} system in the domain of non-locative \isi{third-person} \isi{subject} pronouns in two varieties of the Atlantic language \ili{Serer}. When such a grammatical \isi{subject pronoun} represents the topic of an affirmative clause, it indicates whether or not it has the same referent as the \isi{subject} of the immediately preceding clause.

Amongst the Atlantic languages, \ili{Serer} is thus the first language for which \isi{switch-reference} has been attested. Furthermore, to my knowledge, its specific type of non-canonical \isi{switch-reference} has not been described for other languages as yet, neither on the African continent -- where \isi{switch-reference} is already a rare phenomenon \citep[3]{Treis2012} -- nor elsewhere.

\section*{Acknowledgments}

Prior to ACAL 47, parts of this paper were presented at the African Linguistics Research Colloquium at Humboldt-Universität zu Berlin (27 October, 2015). I would like to thank my doctoral supervisor Tom Güldemann, my colleagues Ines Fiedler and Lee Pratchett, Nicholas Baier, John Merrill, as well as the audience members and two anonymous reviewers and for their useful comments and suggestions on earlier versions of this paper. In particular my thanks go to Papa Saliou Sarr for his enduring patience in judging and commenting examples of his mother tongue.

\section*{Abbreviations}

\begin{tabularx}{.45\textwidth}{>{\scshape}lQ}
comp & complementiser\\
 def & definite article\\
 dir & directional\\
 ds & different {subject}\\
\end{tabularx}
\begin{tabularx}{.45\textwidth}{>{\scshape}lQ}
 emph & emphatic {pronoun}\\
 foc & {focus}\\
 hyp & hypothetical\\
 imp & imperative\\ 
\end{tabularx}

\begin{tabularx}{.45\textwidth}{>{\scshape}lQ}
 ipfv & imperfective\\ 
 loc & locative\\
 midd & middle voice\\
 neg & negative\\
 non & non\\
 pl & plural\\
 pfv & perfective\\
 poss & possessive {pronoun}\\
\end{tabularx}
\begin{tabularx}{.45\textwidth}{>{\scshape}lQ} prep & preposition\\
 pret & preterite\\
 pro & {pronoun}\\
 rel & relative\\
 sg & singular\\
 ss & same {subject}\\
 t & term\\
 tns & tense
\end{tabularx}


\sloppy
\printbibliography[heading=subbibliography,notkeyword=this]

\end{document}
