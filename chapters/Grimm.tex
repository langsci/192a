\documentclass[output=paper]{LSP/langsci} 
\author{Nadine Grimm\affiliation{University of Rochester}  
}
\title{Implosives in {B}antu {A}80? {T}he case of {G}yeli} 

\abstract{Implosive consonants in Bantu A80 languages are widely attested in the literature. The status that specific authors assign to them, however, differ significantly, ranging from mere phonetic contrasts to phonemic status or even absence in certain languages. Given this variety of language analyses, along with a controversy about necessary and sufficient features of \isi{implosive} sounds, this paper aims at re-assessing the range of \isi{implosive}s and non-\isi{implosive}s within A80 and especially Gyeli (A801). I show that though \isi{implosive}s are expected in Gyeli from previous literature, these sounds are better described as pre-glottalized stops with a relatively long prevoicing time. That raises the question whether this analysis might be more appropriate for other A80 languages as well. While this paper cannot provide any conclusive answer on the latter question, it hopes to raise awareness of the methodological problems associated with the present description of A80 \isi{implosive}s, encouraging a systematic re-evaluation of the data. It also encourages a discussion on how the general fieldworker should go about describing \isi{implosive}(-like) sounds.}
\maketitle

\begin{document}

\section{Introduction} 
\label{sec:grimm:1}

 The occurrence of \isi{implosive}s is areally expected in northwestern \ili{Bantu}, as shown by \citet[58]{Clements2008}. Implosives have also been reported for several \ili{Bantu} A80 languages, including \ili{Mpiemo}, \ili{Shiwa}, Kola, and \ili{Bekwel}. Most authors agree that \isi{implosive}s in A80 languages have phonetic rather than phonemic status, but differ in how they view the relation between \isi{implosive}s and voiced stops, e.g., whether /ɓ/ is an allophone of /b/ or whether a language lacks /b/ altogether. There are also cases where different authors do not agree on the presence or absence of \isi{implosive} sounds in the same language, namely in \ili{Gyeli} and \ili{Shiwa}. This differing treatment of \isi{implosive}s in the A80 literature raises the question whether these consonants really are \isi{implosive}s in the first place in all of these languages.

Data from \ili{Gyeli}, an endangered and under-studied \ili{Bantu} A80 language spoken by ``Pygmy'' hunter-gatherers in southern Cameroon, suggests that consonants which could be taken to be \isi{implosive}s are better described as phonemic voiced plosives that are phonetically realized with pre-glottalization and relatively long prevoicing, typically in stem-initial position. During prevoicing, speakers expand their cheeks, increasing both the vocal tract size and amplitude before release of the voiced plosives /b, d/.   The effects of this realization can easily be mistaken for an \isi{implosive}, given that both \isi{implosive}s and pre-glottalized stops involve the manipulation of the larynx and the resulting waveform looks in many cases like that of a typical \isi{implosive}. The \isi{cheek expansion} clearly indicates, however, that the \isi{airstream mechanism} in \ili{Gyeli} is egressive. The case of pre-glottalized voiced stops in \ili{Gyeli} may serve as a starting point to reconsider special voiced stops in A80 languages and clarify the status of \isi{implosive}s, at least in some languages.

In the remainder of the Introduction, I will critically review definitions of \isi{implosive}s provided by the literature and introduce the \ili{Gyeli} language. In \sectref{sec:grimm:2}, I present the distribution of \isi{implosive}s and their phonetic/phonemic status in \ili{Bantu} A80 languages. \sectref{sec:grimm:3} provides a detailed discussion of voiced stops in \ili{Gyeli}, while \sectref{sec:grimm:4} concludes this paper and gives an outlook on future work that is needed.






\subsection{Definitions of `Implosives' in the literature} 
\label{sec:grimm:1.1}

The average linguist venturing out into the field to describe an under-studied language has to be knowledgeable in all parts of grammar they intend to describe. More often than not, they are not necessarily expert phoneticians, though, and describing phenomena such as \isi{implosive}s, which have long been a source of controversy, can be very challenging. This is due to i) an apparently different \isi{airstream mechanism} that was hard to perceive by some early linguists and ii) the nature of phonetic variation ascribed to \isi{implosive}s. \citet{Xi2009}, who gives an excellent overview of the historical development of \isi{implosive} studies, points out that many linguists have had difficulties in accurately describing \isi{implosive}s because they were perceptually used to a pulmonic \isi{airstream mechanism}. According to her, prior to the recognition of a glottalic airstream, these sounds were often described as pre-glottalized, laryngealized, or pre-nasalized stops which had a long-lasting impact, especially on descriptive linguists.  

In order to analyze and name encountered phenomena as best as they can, descriptivist fieldworkers try to have a good understanding of at least the essential literature on specific topics. 
Textbook definitions often seem to come in handy, especially in terms of terminological issues and definitions. 
Textbook definitions typically summarize core features that are widely agreed upon in defining \isi{implosive} sounds. Generally speaking, \isi{implosive}s seem to be plosives which are produced with an \isi{ingressive airstream} due to larynx lowering. This view is represented, for instance, by \citet[228]{Crystal2008}, who states in his {\it Dictionary of Linguistics and Phonetics} that,
``[the term \isi{implosive}s] refers to the series of \textsc{plosive} sounds it is possible to make using an \isi{airstream mechanism} involving an inwards \isi{movement} of air in the mouth (an \textsc{ingressive airstream}).''  Also general introductions to linguistics emphasize the \isi{ingressive airstream} as a defining feature of \isi{implosive}s, for example by \citet[41]{McGregor2015}: ``Implosives are produced by pulling the larynx downwards during oral closure, and releasing the oral closure, resulting in an audible inrush of air.'' In earlier classic textbooks, another assumed property of \isi{implosive}s was included in the definition, namely a glottalic \isi{airstream mechanism}, as in, for instance, \citet{Fromkin1998}.

The realization of phonemic segments are variable, however, and not every sound that is classified as an \isi{implosive} is realized the same way, which has been noted already by, for instance, \citet{Greenberg1970}. This becomes very clear when looking at the phonetics literature where each of the defining core criteria for \isi{implosive}s have been challenged. Especially for sounds that seem to be at the fringe of an abstract \isi{implosive} category, authors tend to give much wider definitions or, at least, question the relevance of any seemingly defining feature.  There is controversy about categorizing `unusual' \isi{implosive}s, encompassing all core features, namely 
i) airflow mechanism, which could be ingressive vs.\ potentially egressive and glottalic vs.\ not necessarily glottalic, 
ii) manner of articulation, which has been described as plosive vs.\ sonorant vs.\ non-obstruent, and 
iii) larynx lowering, which does not seem to be sufficiently defining, but a matter of degree.

In the {\it World Atlas of Language Structures}, a reference for typology and cross-linguistic comparison,  \citet{Maddieson2013} describes \isi{implosive}s as stops produced with a downward \isi{movement} of the larynx, including the possibility of an inward airflow. Thus, an ingressive airflow is not a necessary, but an optional feature. Also \citet[82]{Ladefoged1996} stress that the presence or absence of negative intra-oral pressure is a variable phonetic feature, proposing ``a gradient between one form of voiced plosive and what may be called a true \isi{implosive}.''
\citet{Lindau1984} states that \isi{implosive}s may be non-glottalized, involving no glottal closure. \citet[56]{Clements2008} support this view, stating that ``\isi{implosive}s cannot be neatly distinguished from non-\isi{implosive} sounds in terms of an alleged glottalic \isi{airstream mechanism}.''

Even the manner of articulation in \isi{implosive}s has been challenged. \citet{Clements2000} views \isi{implosive}s as sonorants rather than stops. Later on, \citet{Clements2002} define \isi{implosive}s rather as non-obstruent (non-explosive) stops which lack a build-up of air pressure, resulting in a weak burst at release.

Finally, a lowering of the larynx appears in many definitions of \isi{implosive}s which might then seem to be the only criterion left in defining \isi{implosive}s. \citet{Ewan1974}, however, hold that larynx lowering is not unique to \isi{implosive}s, but also found in certain voiced stops of \ili{English} or \ili{French}. As such, larynx lowering is not a sufficient feature. As with all other proposed phonetic properties of \isi{implosive}s, larynx lowering is also {subject} to variation, involving more or less lowering which, in turn, may have different effects on the airstream and blur the lines between voiced stops and \isi{implosive}s. Thus, \citet[11]{Xi2009} explains that, ``if  the  degree  of  lowering  the  larynx  is  attenuated,  \isi{implosive}s  are  likely to  change  to
  voiced  stops.  Alternatively,  for  voiced  stops,  if  the  pre-voicing  is  prolonged  by  enlarging  the  supra-glottal  cavity,  it  would  drive  the  voiced  stops  change  to  \isi{implosive}s.''

This controversy reflects a larger issue pertaining to the nature of categories: to what degree can the phonetic details of a category in one language be assumed to hold for the phonetic details of the same category in other languages?  The short answer is that it can be assumed that there are likely to be differences.  Even closely related languages such as \ili{Bantu} A80 display different realization rules for the same segment, as is evident from the literature (see \sectref{sec:grimm:2}). What we do not know is the extent to which phonetic details of e.g., , plosives or \isi{implosive}s differ in terms of voicing details, energy of burst, or aspiration because the relevant literature does not give any information on this. Differences are, however, expected, as are similarities.

Knowing about the phonetic details of a segment in one language can serve as a starting point to investigate and/or re-evaluate categories and their extension across (related) languages, provided that their phonetic details become known as well. Ultimately, this will help answer questions on how we can establish categories for cross-linguistic comparison, given the wide range of phonetic variation, and how telling these categories are.

This brings us back to the practical issues of the descriptive fieldworker. How does one know, given all the within-category variation, that one is dealing with a realization of that category or something  different? In this paper, I explore this question with a class of sounds in \ili{Gyeli} that resemble \isi{implosive}s, but which I argue are pre-voiced stops, based on phonetic analysis rather than on perceptual intuitions only. Assuming the generally agreed-upon core features of \isi{implosive}s--\isi{ingressive airstream}, larynx lowering, and plosive manner of articulation--I will show that \ili{Gyeli} prevoiced stops do not meet the criteria of \isi{ingressive airstream} and larynx lowering, but that auditory effects similar to \isi{implosive}s are achieved through glottalization, prevoicing, and \isi{cheek expansion}.






\subsection{The Gyeli language and data}
\label{sec:grimm:1.2}

While I discuss \isi{implosive} sounds across \ili{Bantu} A80 languages in this paper, \ili{Gyeli} is the main language of analysis and the only language for which I have first-hand data. In this section, I briefly provide some basic information on the language and my methodology.
   
\ili{Gyeli} is a \ili{Bantu} A80 language (A801, following \citealt{Maho2009}) spoken in southern Cameroon by so-called ``Pygmy'' hunter-gatherers. The language is known under a variety of names, including \ili{Bakola}, Bagyeli, and \ili{Bajele}.  There are about 4000--5000 speakers who currently still transmit the language to their children. Nevertheless, \ili{Gyeli} is classified as an endangered language due to a rapidly changing environment that forces speakers to give up their traditional foraging subsistance strategy, adopting farming practices from neighboring agriculturalist \ili{Bantu} groups. In total, \ili{Gyeli} has eight contact languages, the most prominent of which are \ili{Kwasio} (A80) as \ili{Gyeli}'s closest relative, \ili{Bulu} (A70), and \ili{Basaa}  (A40).  Currently, several \ili{Gyeli} dialects are emerging, depending on the main contact language of regional \ili{Gyeli} group.

Previous literature on \ili{Gyeli} comprises a few grammatical descriptions of different \ili{Gyeli} varieties which also differ in terms of their degree of coverage. The most substantial work comes from \citet{Grimm2015} who provides a complete grammar of the variety spoken in Ngolo, i.e., the \ili{Bulu} contact region. An earlier description of `Bajɛle' by \citet{Renaud1976} investigates the phonology and nominal morphology of the \ili{Gyeli} variety spoken around Bipindi, i.e., in the \ili{Kwasio} area. There is also an unpublished manuscript on the dialect of \ili{Lebdjom}, i.e., the \ili{Basaa} contact region, by \citet{NgueUm2012}. Other linguistic work on \ili{Gyeli} include an ethnobotanic study of tree names by \citet{Letouzey1995} and a study of color category innovation in \isi{language contact} by \citet{Grimm2014}. There are no previous phonetic studies of \ili{Gyeli} other than \posscitet{Renaud1976} observations in his phonological description.

Data on the \ili{Gyeli} language stems from my own fieldwork conducted in Cam\-er\-oon between 2010 and 2014. The analysis of the relevant sounds (voiced plosives which are potential candidates for \isi{implosive}s) was done including both tokens from carefully pronounced word list recordings and tokens from natural text.







\section{Implosives in Bantu A80}
\label{sec:grimm:2}

When describing a language, related and neighboring languages can give valuable hints as to what one might expect to find. In the case of \ili{Gyeli}, one might expect to find \isi{implosive} sounds.
Implosives are attested in \ili{Bantu} A80 languages as well as more broadly in northwestern, eastern coastal, and southeastern \ili{Bantu} languages. \citet[28]{Maddieson2003} states that these languages often have at least one \isi{implosive}, which is most frequently a bilabial. According to him, \ili{Bantu} \isi{implosive}s have certain phonetic features in common.  First, they are typically produced without \isi{glottal constriction}. And second, lowering of the larynx is crucial in \ili{Bantu} \isi{implosive} production, having a double effect. On the one hand, the lowering increases the amplitude of vocal fold vibration during closure, resulting in a strong voicing at the release. On the other hand, the larynx lowering during production causes an \isi{ingressive airstream}.

Taking these diagnostics into account, when analyzing \isi{implosive} sounds in spectrograms and waveforms, there are a few things one would expect to find, and also a few that one would {\it not} expect to find. In terms of the absence of glottalization, there should be no indication of a glottal closure. A glottal closure might be visible through a higher amplitude in the waveform  or signs of `noise' in the spectrogram. A glottal closure can, however, also be indicated by the absence of a visible stop closure  altogether when it accompanies another stop, since overlapping gestures of glottal and other stop closures might result in the ``suppression of  any audible burst or frication when it is released,'' as \citet[73]{Ladefoged1996} explain. 
Regarding the effects of larynx lowering, one would expect to see the increasing amplitude of vocal fold vibration in a typical cone shape that occurs in the waveform right before the release as well as an increase in F0. The release, in turn, should have a comparatively stronger voicing than potential voiced plosive counterparts.
The diagnostic of an \isi{ingressive airstream} that is attributed to \ili{Bantu} \isi{implosive}s cannot be inferred from spectrogram or waveform analyses; instead, special techniques for airflow and air pressure need to be used (see, for instance, \citealt{Demolin2011} for a discussion on aerodynamic techniques for phonetic fieldwork.) There might be other cues to airflow though, for instance observing the \isi{movement} of both the larynx and the cheeks. I will return to these diagnostics in \sectref{sec:grimm:3}.

While \isi{implosive}s have been widely reported for \ili{Bantu} A80 languages, there is only one phonetic study of these sounds by \citet{Nagano2012} on \ili{Mpiemo}. Therefore, the following discussion cannot provide a comparison of phonetic features, but rather outlines differing phonemic status and possibly distribution of \isi{implosive}s in those A80 languages for which data on \isi{implosive}s (or their absence) is available. What becomes apparent in this comparison is that \isi{implosive} sounds in A80 receive a very different treatment in terms of their phonemic vs.\ phonetic status. This differing treatment seems puzzling, especially when accounts differ substantially on even the same language. It first brings us back to the issue of deciding what sounds should be labelled as \isi{implosive}s. Beyond this,  is also raises the questions of how much phonetic variation or similarity there really is in A80 `\isi{implosive}s' and in how far this phonetic variation is played out on the phonological level.



\tabref{tab:grimm:1} summarizes the status of potential \isi{implosive}s\footnote{Square brackets indicate phonetic status while slashes / / indicate phonemic status.} within the phonemic plosive series in a representative sample of A80 languages.\footnote{There are, of course, more A80 languages, as classified by \citet{Maho2009}. Also \citet{Cheucle2014} gives an excellent overview of A80 languages and the existing literature. Sufficient description for comparison, however, is mainly restricted to the languages listed in \tabref{tab:grimm:1} which almost cover the major languages, with the exceptions of A82 (\ili{So}) and A87 (\ili{Bomwali}) for which there is no data.}
Most authors agree that \isi{implosive}s in A80 languages, if present, have phonetic rather than phonemic status. \citet[461]{Cheucle2014} even reconstructs voiced stops in Proto-A80 as \isi{implosive}s.
 Despite this tendency, there is still a lot of variation in the description of voiced plosives and/or \isi{implosive}s in several respects, including i) their general presence or absence, ii) the type of voiced plosive/\isi{implosive} (e.g.,  bilabial, alveolar, palatal, velar), and iii) their phonemic status.\footnote{Obviously, there are differences across languages pertaining to the phoneme inventory and realization rules. \ili{Bantu} A80 languages differ most noticeably in the presence or absence of palatal stops and labio-velars. Some languages also lack the voiceless bilabial stop. There are also some commonalities though, including bilabial, alveolar, and velar places of stop articulation, and voicing contrast as a distinctive feature. For reasons of space, I refrain from discussing prenasalized plosives and affricates.  Realization rules, if not involving \isi{implosive} allophones, are not described here. It should only be noted that they may differ across languages and/or authors' descriptions.}

 \todo{check palatal stopl in Ngolo in \tabref{tab:grimm:1}}
\begin{table}
\caption{Status of voiced stops/\isi{implosive}s in A80 languages}
\label{tab:grimm:1}
\resizebox{12cm}{!} {
\begin{tabular}{ll lll l}
  \lsptoprule
\multicolumn{2}{l}{Language} & Implosives & Restrictions  & Plosive series &  Source \\
\midrule
\multicolumn{2}{l}{\ili{Gyeli} (A801)} &		& 		& 			& 		\\ 
 & \ili{Gyeli} (Ngolo)                          & no &  & /p, b, t, d, k, ɟ, g, ʔ/ &  \citet{Grimm2015} \\
 & \ili{Bajele} (Bipindi)                      & \textipa{[\!b]} & \isi{free variation} with [b] & /p, b, t, d, ɟ, k/ &  \citet{Renaud1976} \\ %p.49
 & \ili{Bakola} (Lepdjom)	               & /ɓ, ɗ, ʄ/ &  stem-initial & /p, ɓ, t, ɗ, ʄ, k, kp/ &  \citet{NgueUm2012} \\ % p.3

\tablevspace
\multicolumn{2}{l}{\ili{Shiwa} (A803)} & 	no	& 	&  /p, b, t, d, k, g/	& \citet{Ollomo2013}		\\ 
&  & 	[ɓ, ɗ, ʄ]	& 	none	& /p, b, t, d, k, g/ &  \citet{Dougere2007}		\\ 
\tablevspace
\multicolumn{2}{l}{\ili{Kwasio} (A81)} & 	no	&		& /p, b, t, d, c, ɟ, k/  &  \citet{Lemb1974}		\\ 
\tablevspace
\multicolumn{2}{l}{\ili{Makaa} (A83)} & 	no	&  & /b, t, d, c, ɟ, k, g, kp/			& \citet{Heath2003} 		\\ 
\tablevspace
\multicolumn{2}{l}{\ili{Bekol} (A832)} & 	no	&  &  /(p), b, t, d, c, ɟ, k, g, kp/ & \citet{Henson2007}	\\ 
\tablevspace
\multicolumn{2}{l}{\ili{Njem} (A84)} & 	no	& 			&  /p, b, t, d, c, ɟ, k, g, kp, gb/	& \citet{Beavon2006} \\ 
\tablevspace
\multicolumn{2}{l}{\ili{Konzime} (A842)} & 	no	&  & /p, b, t, d, c, ɟ, k, g, kp, gb/	& \citet{Beavon1983} 		\\ 
\tablevspace
\multicolumn{2}{l}{\ili{Bekwel} (A85b)} & 	[ɓ, ɗ, ʄ, ɠ]	& in C\textsubscript{1} & /p, \textsubdot{b}, b, t, \textsubdot{d}, d, c, \textsubdot{ɟ}, j, k, \textsubdot{g}, g, (kp), (gb)/	& 	\citet{Cheucle2014}	\\ 
\tablevspace
\multicolumn{2}{l}{\ili{Mpiemo} (A86c)} & \textipa{[\!b}, \textipa{\!d]} & before low vowels in C\textsubscript{1} & /p, b, t, d, c, ɟ, k, g, kp, gb/	& \citet{Thornell2004}		\\ 
  & & /\textipa{\!b}, \textipa{\!d}/ & in C\textsubscript{1}, not before /i, u/ & no information &  \citet{Beavon1978} \\ 
 \lspbottomrule
\end{tabular}}
\end{table}





There are three accounts of \ili{Gyeli} (A801), describing different varieties of the language. Each account differs in its assessment of voiced plosives/\isi{implosive}s. In \posscitet{Grimm2015} analysis,  the \ili{Gyeli} variety spoken in Ngolo (\ili{Bulu} contact area) has no \isi{implosive}s at all. Voiced plosives /b/ and /d/ in stem-initial position are realized with preglottalization and relatively long prevoicing. This account is explained in detail in \sectref{sec:grimm:3}. In comparison, \citet[49]{Renaud1976} suggests the presence of a bilabial \isi{implosive} in the \ili{Gyeli} variety spoken around Bipindi (\ili{Kwasio} and \ili{Basaa} contact area). The \isi{implosive} is, however, only a phonetic variant of [b] occuring before the vowels /u, o, õ, ɔ, ɔ̃, a, ã/ in both C\textsubscript{1} and C\textsubscript{2} position. The \isi{implosive} realization is, according to \citet{Renaud1976}, in \isi{free variation} with an egressive glottalized stop. Preceding the vowels /i, e, ɛ, ẽ/, /b/ is realized as a modal voiced stop with a particularly strong burst, including inflating the cheeks and a {\it battement}  (beat) of the lips.  The third account of \ili{Gyeli} concerns the variety spoken in \ili{Lebdjom} (\ili{Basaa} contact area). \citet[3]{NgueUm2012} assigns phonemic status to bilabial, alveolar, and palatal \isi{implosive}s whose occurrence is restricted to the stem-initial position. According to him, there are no voiced plosives, but only voiceless ones. This seems typologically unexpected.

\ili{Shiwa} (A803),\footnote{\citet[51]{Ollomo2013} classifies \ili{Shiwa} as A833 rather than A803, but I stick with \posscitet{Maho2009} classification.} represents another controversial case as to the presence or absence of \isi{implosive}s. According to \citet{Ollomo2013} and \citet{Puech1989}, \ili{Shiwa}  has no \isi{implosive}s, neither phonologically nor phonetically, but a plain plosive series of bilabial, alveolar, and velar plosives, all distinguished by a voicing contrast.\footnote{In addition to \posscitet{Ollomo2013} plosive series, \citet{Puech1989} also posits a phonemic voiced palatal stop.} In contrast to their analysis, \citet[56]{Dougere2007} asserts that all voiced stops in \ili{Shiwa} are generally realized as \isi{implosive}s in all environments, i.e., word/stem initially and intervocalically.

For \ili{Kwasio} (A81), \ili{Makaa} (A83), \ili{Bekol} (A832), \ili{Njem} (A84), and \ili{Konzime} (A842), no \isi{implosive}s are reported, neither phonemic nor phonetic.   As to \ili{Kwasio}, all principal authors -- \citet{Lemb1974}, \citet{Dieu1976}, and \citet{Yemmene2004} -- describing the phonology agree that there is a voicing opposition between at least bilabial and alveolar plosives, but no indication of a phonetic realization of \isi{implosive}s for any of these obstruents.  For \ili{Makaa}, \citet{Heath2003} does not report any \isi{implosive}s either, but states that the phoneme /b/ lacks a voiceless counterpart /p/. The same holds for \ili{Bekol} as described by \citet{Henson2007} who reports that instances of [p] are so rare and only found in loan words that it might not be a phoneme in the language. For \ili{Njem}, \citet{Beavon2006} outlines the phonetic realization of the entire stop series (bilabial, alveolar, palatal, and velar), but \isi{implosive}s are not among the variants. In \ili{Konzime}, labial and alveolar stops are ``released with oral cavity friction'' before high vowels, according to \citet[134]{Beavon1983}, but do not exhibit \isi{implosive} features.


\citet[147]{Cheucle2014} describes all voiced stops -- bilabial, alveolar, palatal, and velar -- as having an \isi{implosive} realization in C\textsubscript{1} position in \ili{Bekwel}. She treats this feature as phonetic rather than phonemic and remarks that the degree of implosion varies across speakers.

Finally, \ili{Mpiemo} receives a different treatment of \isi{implosive}s by different authors.  \citet{Beavon1978} views bilabial and alveolar \isi{implosive}s as having phonemic status which are opposed to their voiced stop counterparts. According to him, they are restricted to C\textsubscript{1} position and precede all vowels except for /i/ and /u/. In contrast to this, \citet{Thornell2004} assign phonetic status to bilabial and alveolar \isi{implosive}s in \ili{Mpiemo}, categorizing them as allophones of /b/ and /d/. They also observe the same distribution of voiced stops and implsoives as Beavon: voiced stops occur before /i/ and /u/ and nasals, in all other stem-initial environments, they are realized as \isi{implosive}s. \figref{fig:grimm:1} shows a bilabial \isi{implosive} of \ili{Mpiemo} as presented by \citet[172]{Thornell2004}.

\begin{figure}
\caption{Bilabial \isi{implosive} in Mpiemo} 
\label{fig:grimm:1}
\includegraphics[width=\textwidth]{figures/mpiemoB.jpg}
\end{figure}

% Note that caption is not on top of figure, why??

The \isi{implosive} exhibits a typical cone-shape \isi{amplitude increase} during closure. In fact, \citet{Nagano2012}, in their detailed phonetic study of \ili{Mpiemo} \isi{implosive}s, state that this \isi{amplitude increase} during closure is a strong acoustic correlate of \isi{implosive}s in \ili{Mpiemo}. In contrast, their egressive counterparts show a decreasing \isi{voicing amplitude}. Other characteristics of \ili{Mpiemo} \isi{implosive}s, according to the authors, include a glottalic ingressive aistream, full voicing (which also holds for egressive plosives), an increased F0 during occlusion (while F0 decreases in voiced plosives), and a \isi{closure duration} for \isi{implosive}s which is generally longer than that for voiced stops. Implosion at release, however, is not a consistent phonetic feature. Keeping the phonetic \ili{Mpiemo} \isi{implosive} features in mind as well as \posscitet{Maddieson2003} general remarks about \ili{Bantu} \isi{implosive}s, I now turn to describing the phonetic features of voiced stops in \ili{Gyeli}. 









\section{Prevoiced stops in Gyeli}
\label{sec:grimm:3}

Despite expectations inherited from the literature on other \ili{Gyeli} dialects and comparison to related languages, I argue that the \ili{Gyeli} variety spoken in Ngolo (\ili{Bulu} contact region) does not have \isi{implosive}s, neither on a phonemic nor on a phonetic level. According to \posscitet{Grimm2015} description, the phonemic distinction the language makes is between voiced and voiceless stops. Bilabial voiced plosives occur word- and stem-initially, and in medial position they are realized as [β]. Alveolar voiced stops are found in word-medial position, but I am concentrating my analysis on those in initial position since it is not to be assumed that a medial position would host \isi{implosive}s if initial positions do not. Velar voiced stops are almost exclusively limited to word-medial positions, so they do not qualify as potential \isi{implosive}s.

\ili{Gyeli} bilabial and alveolar voiced stops in word- and stem-initial position are realized with \isi{glottal constriction} and prevoicing before the burst. At the same time, speakers inflate their cheeks to varying degrees before release.   As such, these sounds have a few phonetic/acoustic features in common with what are typically taken as features of \isi{implosive}s, including glottalization, \isi{amplitude increase} before release, and often a strong burst at release. Especially the cone-shape \isi{amplitude increase} before release, as observed in the waveform in \figref{fig:grimm:3}, makes \ili{Gyeli} prevoiced stops look like typical \isi{implosive}s so that one might be inclined to analyze them as \isi{implosive}s at least phonetically. There is, however, good evidence to assume that these sounds are produced with an egressive airstream. The key argument that also explains the cone-shape \isi{amplitude increase} is the speaker's expansion of the cheeks which goes against assuming an \isi{ingressive airstream}. At the same time, variation in the degree of \isi{cheek expansion} within the same and across different speakers suggests that \isi{implosive}-like phonetic features are not stable enough to label \ili{Gyeli} voiced stops as \isi{implosive}s. In the following, I will compare \ili{Gyeli} voiced stops to \ili{Bantu} and \ili{Mpiemo} \isi{implosive}s, showing that they are not the same class of sounds. I will also provide a more detailed analysis of \ili{Gyeli} voiced stops along a variety of parameters, including voicing, amplitude, intensity, and \isi{closure duration}. I am restricting my illustrations to bilabial voiced plosives due to space limitation. It should be noted though that the same features apply to stem-initial alveolar voiced stops.
% \footnote{While several authors report other types of voiced stops and/or \isi{implosive}s for some A80 languages, as reviewed in \sectref{sec:grimm:2}, this does not apply to \ili{Gyeli}. As \citet{Grimm2015} shows, \ili{Gyeli} has no palatal, but only velar stops. The voiced velar plosive does not occur stem-initially though and therefore does not share the same phonetic features of the other voiced plosives.} %%deleted as requested

\begin{description}
 \item [Glottalization}]


 What \citet[28]{Maddieson2003} generally says about \ili{Bantu} \isi{implosive}s, namely that they are produced without any \isi{glottal constriction}, does not apply to \ili{Gyeli} voiced stops. There is \isi{glottal constriction} throughout, accompanying the entire bilabial or alveolar closure. This might be visible as `noise' in the spectrogram in the circled area of \figref{fig:grimm:2}.\footnote{Glottalization effects might not be as obvious in every token; in \figref{fig:grimm:3}, for instance, it is not.} This could mean two things. On the one hand, one might want to say that \ili{Gyeli} voiced stops could still be \isi{implosive}s which just exhibit different acoustic features than the majority of \ili{Bantu} \isi{implosive}s. On the other hand, one could take this as a cue that \ili{Gyeli} voiced stops are indeed different from \isi{implosive}s found in other \ili{Bantu} languages. The criterion of glottalization alone is, as also discussed in \sectref{sec:grimm:1.1}, inconclusive. Data from \ili{Mpiemo} also illustrates that the degree of vocal fold constriction might be \isi{subject} to variation across speakers \citep[75]{Nagano2012}. 

\begin{figure}
\caption{Production of [b] in {\it bɛ̀ɛ̀} `shoulder', speaker 1}
\label{fig:grimm:2}
\includegraphics[width=\textwidth]{figures/GyeliB-Ada-mini}
\end{figure}

\item[Voicing] As can be seen in both \figref{fig:grimm:2} and \figref{fig:grimm:3}, voiced stops in \ili{Gyeli} are fully voiced, from the onset through the offset of the closure. This is a feature they have in common with voiced stops as well as \isi{implosive}s in \ili{Mpiemo} \citep[74]{Nagano2012}.


\item[Voicing amplitude] While \citet{Nagano2012} convincingly show for \ili{Mpiemo} that \isi{implosive}s are correlated with an increasing \isi{voicing amplitude} during closure and voiced stops with a decreasing one, this distribution does not map onto \ili{Gyeli} stops in any way. Rather, what one finds is a high degree of amplitude variation both speaker-internally and across different speakers which correlates with the degree of cheek inflation.  For instance, [b] in the lexeme {\it bɛ̀ɛ̀} `shoulder' might differ significantly in its \isi{voicing amplitude}. In \figref{fig:grimm:2},\footnote{Both \figref{fig:grimm:2} and \figref{fig:grimm:3} have been produced in Praat.} the \isi{voicing amplitude} is neither increasing or decreasing, but remains level throughout the closure because \isi{cheek expansion} is minimal in this token. In contrast, the same lexeme in \figref{fig:grimm:3}\footnote{The noisy part around 0.1sec into the recording seen both in the waveform and the spectrogram is some background noise and not part of the human speech production. Unfortunately, background noise cannot be completely avoided in fieldwork. I nevertheless choose to present this token since it has the sharpest \isi{amplitude increase} while representing the same lexeme which makes it comparable.} is produced with a steadily rising amplitude. Though this token looks suspiciously like an \isi{implosive}, it is not. The \isi{amplitude increase} is explained by an extreme case of \isi{cheek expansion}.   This distribution does not seem to depend on variability between speakers, but even the same speaker produces tokens with a \isi{voicing amplitude} more on the level side of the spectrum and other tokens with \isi{amplitude increase}.

\begin{figure}
\caption{ Production of [b] in {\it bɛ̀ɛ̀} `shoulder', speaker 2}
\label{fig:grimm:3}
\begin{center}
\includegraphics[width=\textwidth]{figures/GyeliB-Mama-mini.jpg}
\end{center}
\end{figure}

Cheek expansion during stop prevoicing, even if minimal, is a feature of every initial voiced stop in \ili{Gyeli} and does not depend on the phonetic environment. Thus, in contrast to \posscitet{Renaud1976} analysis of the Bipindi variety of \ili{Gyeli}, either realization similar to \figref{fig:grimm:2} or \figref{fig:grimm:3}, or even an \isi{amplitude increase} in between these two extremes, is found before any of the seven vowels /i, u, e, o, ɛ, ɔ, a/.\footnote{Video recordings of natural \ili{Gyeli} text, that may show \isi{cheek expansion}, are available in the DoBeS archive, found under the language name `\ili{Bakola}'. In this paper, I rely on my long familiarity with the language and speakers. Systematic video recordings of voiced stop production are a future project.} 

\item[Intensity] \citet[75]{Nagano2012} state for \ili{Mpiemo} that ``Intensity showed a good correlation with \isi{voicing amplitude} and F0 and it is higher/ increasing for \isi{implosive}s than for plosives.'' In comparison, there does not seem to be a general difference in average F0 between those tokens of [b] which show a level or an increasing amplitude. Average F0 for the tokens in \figref{fig:grimm:2} and \figref{fig:grimm:3},  for example, are both within the range of 135 to 145Hz. There is, however, a difference in the intensity curve which raises steadily in tokens with increasing \isi{voicing amplitude} while the intensity in level amplitude tokens is first relatively low and then shows a sudden and sharp increase towards the offset of the closure.



\item[Closure Duration] Closure durations of voiced plosives vary a lot depending on speaking rate (careful vs.\ fast speech), the lexical vs.\ grammatical function of a morpheme or stem, and the environment (intonation phrase initial vs.\ medial).
200 tokens of [b]\footnote{These measurements comprise tokens of various prevoicing amplitude patterns, i.e., those that are more similar to \figref{fig:grimm:2} and those that are more similar to \figref{fig:grimm:3}. The reason for this is that there is no binary distinction, but rather a scale which, however, does not seem to affect \isi{closure duration}. Thus, VOT is the same for low amplitude and \isi{amplitude increase} tokens.} have been measured for \isi{closure duration} in different environments, covering accompaniment by different vowels and different functional environments (grammatical morpheme vs.\ lexical stem). 

Generally, \isi{closure duration} does not seem to depend on the quality of the following vowel, as shown for lexical and word-initial occurrences in \tabref{tab:grimm:2}.\footnote{Only a few tokens were available for [b] before /o/; this might have skewed the results.} Closure durations are rather similar and no distinction can be made between, for example, high and low vowels.

\begin{table}
\caption{Closure durations of voiced bilabial plosives}
\label{tab:grimm:2}
\begin{tabularx}{\textwidth}{lXll}
  \lsptoprule
\_V & Average duration & Lexical example &  Duration \\
  \midrule
i & [b] = 108ms & {\textipa{b\`ijO}} `hit' &  [b] = 130ms\\
u & [b] = 108ms & {\textipa{b\'ulO}}  `fish (v.)' &  [b] = 130ms \\
e & [b] = 105ms & {\textipa{b\'e}} `pit' &   [b] = 81ms\\
o & [b] = 120ms & {\textipa{b\'ogEsE}} `enlarge' &   [b] = 157ms\\
ɛ & [b] = 115ms & {\textipa{b\`E}} `sow' &   [b] = 145ms\\
ɔ & [b] = 103ms & {\textipa{b\`Ond\`i}} `black colobus monkey' &   [b] = 137ms \\
a & [b] = 100ms & {\textipa{b\'aB\`E}} `disease' &   [b] = 151ms\\
 \lspbottomrule
\end{tabularx}
\end{table}

Occurrences of [b] in grammatical morphemes tend to be much shorter than those occuring in lexical stems. While the \isi{noun class} prefix {\it be}- has an average duration of about 50ms (unless produced very carefully), [b] in {\it bénó} `buttock' measures around 160ms. Both tokens are word-initial. Tokens that are lexical, but not word or phrase inital (e.g.,  preceded by a \isi{noun class} prefix or a \isi{subject marker}) tend to have a shorter duration than their word-initial counterparts. Thus, the second occurrence of [b] in {\it be-bénó} `buttock' only has a closure length of around 80ms, which is still longer that [b] in the prefix which is 30ms in this instance. Closure durations are also longer in very careful speech or to emphasize a particular word. In these cases, the \isi{voicing amplitude} is not necessarily higher, but \isi{closure duration} is relatively longer. 
In any case,  longer closure times might correlate with the percept of \isi{implosive}s while shorter closure times sound more like modally voiced stops.


\item[Airstream Mechanism] A final consideration in terms of phonetic features concerns the \isi{airstream mechanism} involved in the production of plosives. While no aerodynamic data were collected for \ili{Gyeli} so far (and also \citet{Nagano2012} base their phonetic analysis of \ili{Mpiemo} \isi{implosive}s on data that does not include airflow mechanisms or laryngographic measurements), statements about the airflow can be made with some certainty by observing speakers. Especially for voiced stop tokens that involve an increasing \isi{voicing amplitude}, \ili{Gyeli} speakers tend to achieve an increase of the vocal tract size by expanding the cheeks. This has already been noted by \citet{Renaud1976} and confirmed by \citet{Grimm2015}. To expand the cheeks, the airflow has to be egressive. At the same time, this gesture excludes a significant lowering of the larynx. I take this as the key argument not to consider \ili{Gyeli} voiced stops as \isi{implosive} realizations.




\end{description}




\section{Conclusion and outlook}
\label{sec:grimm:4}

The findings in \ili{Gyeli}, as well as the treatment of \isi{implosive}s and their relation to voiced plosives in the A80 literature, have several implications. First, it seems that a fundamental issue in the description of A80 \isi{implosive}s is a terminological question. In the absence of any decisive criteria to clearly identify \isi{implosive}s, scholars may categorize a range of sounds as \isi{implosive}s which, in fact, might be very different from one another. 

This leads to methodological implications. On the one hand, it shows how important it is to provide (basic) phonetic information in grammatical descriptions. These are, however, often insufficient or absent altogether. On the other hand, the phonetic description of sounds in a language might seem daunting to fieldworkers whose expertise lies in other areas of grammar. It might be useful for expert phonetician fieldworkers to develop some general guidelines for descriptive linguists, comparable to the many questionnaires on, for instance, \isi{information structure} or object marking.

Multiple theoretical implications are at stake.  On a micro-areal level, a better understanding of \isi{implosive}(-like) sounds in \ili{Gyeli} and other A80 languages enables us to clarify whether these consonants indeed display a high degree of variation or whether they are more uniform than currently suggested by the literature. Since all languages in the area are closely related and in intense contact with one another, one might expect to find significant similarities also in the phonetic realization of sounds.  This does not mean that the phonetic features of a particular phoneme in one language hold for other languages in the area as well. But given that authors have differing treatment of \isi{implosive}s vs.\ voiced stops in the same language in several cases of A80, it is possible that  these languages share certain features which are interpreted in different ways. Thus, important questions still need to be answered:  what phonetic features do these sounds in A80 have in common, if anything, and  in which respects do they differ? A possible parameter of variation could be, for instance, an oropharyngeal expansion which, according to \citet[55]{Ladefoged1996}, may constitute ``a continuum that links modally voiced stops to \isi{implosive}s.'' Obviously, more phonetic analyses are needed to answer these questions, which then help to answer yet others, for instance about their phonemic or allophonic status and their alleged \isi{free variation}. For future work it would also be desireable to include a more systematic data comparison of different A80 languages, using aerodynamic techniques as well as measuring larynx \isi{movement}.

Implosive(-like) sounds in A80 may also provide an interesting window onto \isi{language contact} phenomena. In this area of intense \isi{language contact} and a high degree of multilingualism among speakers of all languages, it would be fascinating to investigate to what degree \isi{implosive}s or some acoustic features of them are borrowed. \ili{Gyeli} speakers, for example, are known to imitate their linguistic  neighbors deliberately in order to increase their prestige. While the closest related language, \ili{Kwasio}, does not seem to have \isi{implosive}s, other neighboring languages such as \ili{Basaa} do. One could hypothesize that \ili{Gyeli} voiced stops are a partial imitations of \isi{implosive}s found in other languages, just without borrowing larynx lowering and an \isi{ingressive airstream}, which are acoustically replaced by glottalization and a \isi{voicing amplitude} increasing through expanding the cheeks.

On a  broader level,  it is, of course, important for fields such as typology, historical linguistics, or language classification to know whether one is comparing conceptually the same or different sounds. Clarifying whether certain sounds in some \ili{Bantu} sub-families are really \isi{implosive}s might change the extension of assumed linguistic areas and might better our understanding of language relations in respect to their genealogical classification.



 \section*{Acknowledgments}

I would like to thank two anonymous reviewers and Joyce McDonough for their careful reading of this paper and their helpful and constructive comments they gave me. I also thank the ACAL participants, especially Bonny Sands, Didier Demolin, and Firmin Ahoua.


\sloppy
\printbibliography[heading=subbibliography,notkeyword=this]


\end{document}