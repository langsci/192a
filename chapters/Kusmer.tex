\documentclass[output=paper,modfonts,nonflat,hidelinks]{langsci/langscibook}  
\title{Prosody and the conjoint\slash disjoint alternation in {T}shivenḓa} 
\author{Leland Paul Kusmer\affiliation{University of Massachusetts at Amherst}}
 

\abstract{Tshivenḓa (Guthrie S21) shares with other Southern Bantu languages a distinctive alternation in the form of the verb, termed the conjoint\slash disjoint alternation. I will present data from original fieldwork showing that, in contrast to other related languages, the Tshivenḓa conjoint and disjoint forms are not in complementary distribution by syntactic context, and instead show a distinctive three-way split in acceptability. I will also show that the same three-way split obtains in the frequency of utterance-internal penultimate lengthening. I discuss two possible analyses of this correlation, one in which the disjoint is a purely prosodic phenomenon and one in which the correlation is due to the influence of some third factor such as information structure. }

\IfFileExists{../localcommands.tex}{%hack to check whether this is being compiled as part of a collection or standalone
  \usepackage{pifont}
\usepackage{savesym}

\savesymbol{downingtriple}
\savesymbol{downingdouble}
\savesymbol{downingquad}
\savesymbol{downingquint}
\savesymbol{suph}
\savesymbol{supj}
\savesymbol{supw}
\savesymbol{sups}
\savesymbol{ts}
\savesymbol{tS}
\savesymbol{devi}
\savesymbol{devu}
\savesymbol{devy}
\savesymbol{deva}
\savesymbol{N}
\savesymbol{Z}
\savesymbol{circled}
\savesymbol{sem}
\savesymbol{row}
\savesymbol{tipa}
\savesymbol{tableauxcounter}
\savesymbol{tabhead}
\savesymbol{inp}
\savesymbol{inpno}
\savesymbol{g}
\savesymbol{hanl}
\savesymbol{hanr}
\savesymbol{kuku}
\savesymbol{ip}
\savesymbol{lipm}
\savesymbol{ripm}
\savesymbol{lipn}
\savesymbol{ripn} 
% \usepackage{amsmath} 
% \usepackage{multicol}
\usepackage{qtree} 
\usepackage{tikz-qtree,tikz-qtree-compat}
% \usepackage{tikz}
\usepackage{upgreek}


%%%%%%%%%%%%%%%%%%%%%%%%%%%%%%%%%%%%%%%%%%%%%%%%%%%%
%%%                                              %%%
%%%           Examples                           %%%
%%%                                              %%%
%%%%%%%%%%%%%%%%%%%%%%%%%%%%%%%%%%%%%%%%%%%%%%%%%%%%
% remove the percentage signs in the following lines
% if your book makes use of linguistic examples
\usepackage{tipa}  
\usepackage{pstricks,pst-xkey,pst-asr}

%for sande et al
\usepackage{pst-jtree}
\usepackage{pst-node}
%\usepackage{savesym}


% \usepackage{subcaption}
\usepackage{multirow}  
\usepackage{./langsci/styles/langsci-optional} 
\usepackage{./langsci/styles/langsci-lgr} 
\usepackage{./langsci/styles/langsci-glyphs} 
\usepackage[normalem]{ulem}
%% if you want the source line of examples to be in italics, uncomment the following line
% \def\exfont{\it}
\usetikzlibrary{arrows.meta,topaths,trees}
\usepackage[linguistics]{forest}
\forestset{
	fairly nice empty nodes/.style={
		delay={where content={}{shape=coordinate,for parent={
					for children={anchor=north}}}{}}
}}
\usepackage{soul}
\usepackage{arydshln}
% \usepackage{subfloat}
\usepackage{langsci/styles/langsci-gb4e} 
   
% \usepackage{linguex}
\usepackage{vowel}

\usepackage{pifont}% http://ctan.org/pkg/pifont
\newcommand{\cmark}{\ding{51}}%
\newcommand{\xmark}{\ding{55}}%
 
 
 %Lamont
 \makeatletter
\g@addto@macro\@floatboxreset\centering
\makeatother

\usepackage{newfloat} 
\DeclareFloatingEnvironment[fileext=tbx,name=Tableau]{tableau}
  %add all your local new commands to this file
\newcommand{\downingquad}[4]{\parbox{2.5cm}{#1}\parbox{3.5cm}{#2}\parbox{2.5cm}{#3}\parbox{3.5cm}{#4}}
\newcommand{\downingtriple}[3]{\parbox{4.5cm}{#1}\parbox{3cm}{#2}\parbox{3cm}{#3}}
\newcommand{\downingdouble}[2]{\parbox{4.5cm}{#1}\parbox{6cm}{#2}}
\newcommand{\downingquint}[5]{\parbox{1.75cm}{#1}\parbox{2.25cm}{#2}\parbox{2cm}{#3}\parbox{3cm}{#4}\parbox{2cm}{#5}}
\newcolumntype{Y}{>{\centering\arraybackslash}X}
\newcolumntype{T}{>{\centering\arraybackslash}m{2cm}}

%commands for Kusmer paper below
\newcommand{\ip}{$\upiota$}
\newcommand{\lipm}{(\_{\ip-Max}}
\newcommand{\ripm}{)\_{\ip-Max}}
\newcommand{\lipn}{(\_{\ip}}
\newcommand{\ripn}{)\_{\ip}}
\renewcommand{\_}[1]{\textsubscript{#1}}


%commands for Pillion paper below
\newcommand{\suph}{\textipa{\super h}}
\newcommand{\supj}{\textipa{\super j}}
\newcommand{\supw}{\textipa{\super w}}
\newcommand{\ts}{\textipa{\t{ts}}}
\newcommand{\tS}{\textipa{\t{tS}}}
\newcommand{\devi}{\textipa{\r*i}}
\newcommand{\devu}{\textipa{\r*u}}
\newcommand{\devy}{\textipa{\r*y}}
\newcommand{\deva}{\textipa{\r*a}}
\renewcommand{\N}{\textipa{N}}
\newcommand{\Z}{\textipa{Z}}
% 

%commands for Diercks paper below
\newcommand{\circled}[1]{\begin{tikzpicture}[baseline=(word.base)]
\node[draw, rounded corners, text height=8pt, text depth=2pt, inner sep=2pt, outer sep=0pt, use as bounding box] (word) {#1};
\end{tikzpicture}
}

%commands for Pesetsky paper below
% \newcommand{\sem}[2][]{\mbox{$[\![ $\textbf{#2}$ ]\!]^{#1}$}}
\newcommand{\sem}[2][]{\mbox{$[[ $\textbf{#2}$ ]]^{#1}$}}

% \newcommand{\ripn}{{\color{red}ripn}}%this is used but never defined. Please update the definition



%commands for Lamont paper below
\newcommand{\row}[4]{
	#1. & 
    /{#2}/ & 
    [{#3}] & 
    `#4' \\ 
}
%\newcounter{tableauxcounter}
\newcommand{\tabhead}[2]{
%     \captionsetup{labelformat=empty}
%     \stepcounter{tableauxcounter}
%     \addtocounter{table}{-1}
% 	\centering
% 	\caption{Tableau \thetableauxcounter: #1}
	\caption{#1}
	\label{#2}
}
\newcommand{\candref}[2]{{(\ref{#1}#2)}}
\newcommand{\tableauref}[1]{{Tableau~\ref{#1}}}
% tableaux
\newcommand{\inp}[1]{\multicolumn{2}{|l||}{{#1}}}
\newcommand{\inpno}[1]{\multicolumn{2}{|l||}{#1}}
\newcommand{\g}{\cellcolor{lightgray}}
\newcommand{\hanl}{\HandLeft}
\newcommand{\hanr}{\HandRight}
\newcommand{\kuku}{Kuk\'{u}}

% \newcommand{\nocaption}[1]{{\color{red} Please provide a caption}}

% \providecommand{\biberror}[1]{{\color{red}#1}}

\definecolor{RED}{cmyk}{0.05,1,0.8,0}


\newfontfamily\amharicfont[Script = Ethiopic, Scale = 1.0]{AbyssinicaSIL}
\newcommand{\amh}[1]{{\amharicfont #1}}

% 
% %Gjersoe
\usepackage{textgreek}
% 
\newcommand{\viol}{\fontfamily{MinionPro-OsF}\selectfont\rotatebox{60}{$\star$}}
\newcommand{\myscalex}{0.45}
\newcommand{\myscaley}{0.65}
%\newcommand{\red}[1]{\textcolor{red}{#1}}
%\newcommand{\blue}[1]{\textcolor{blue}{#1}}
\newcommand{\epen}[1]{\colorbox{jgray}{#1}}
\newcommand{\hand}{{\normalsize \ding{43}}}
\definecolor{jgray}{gray}{0.8} 
\usetikzlibrary{positioning}
\usetikzlibrary{matrix}
\newcommand{\mora}{\textmu\xspace}
\newcommand{\si}{\textsigma\xspace}
\newcommand{\ft}{\textPhi\xspace}
\newcommand{\tone}{\texttau\xspace}
\newcommand{\word}{\textomega\xspace}
% \newcommand{\ts}{\texttslig}
\newcommand{\fns}{\footnotesize}
\newcommand{\ns}{\normalsize}
\newcommand{\vs}{\vspace{1em}}
\newcommand{\bs}{\textbackslash}   % backslash
\newcommand{\cmd}[1]{{\bf \color{red}#1}}   % highlights command
\newcommand{\scell}[2][l]{\begin{tabular}[#1]{@{}c@{}}#2\end{tabular}}
% \interfootnotelinepenalty=10000

% --- Snider Representations --- %

\newcommand{\RepLevelHh}{
\begin{minipage}{0.10\textwidth}
\begin{tikzpicture}[xscale=\myscalex,yscale=\myscaley]
%\node (syl) at (0,0) {Hi};
\node (Rt) at (0,1) {o};
\node (H) at (-0.5,2) {H};
\node (R) at (0.5,3) {h};
%\draw [thick] (syl.north) -- (Rt.south) ;
\draw [thick] (Rt.north) -- (H.south) ;
\draw [thick] (Rt.north) -- (R.south) ;
\end{tikzpicture}
\end{minipage}
}

\newcommand{\RepLevelLh}{
\begin{minipage}{0.10\textwidth}
\begin{tikzpicture}[xscale=\myscalex,yscale=\myscaley]
%\node (syl) at (0,0) {Mid2};
\node (Rt) at (0,1) {o};
\node (H) at (-0.5,2) {L};
\node (R) at (0.5,3) {h};
%\draw [thick] (syl.north) -- (Rt.south) ;
\draw [thick] (Rt.north) -- (H.south) ;
\draw [thick] (Rt.north) -- (R.south) ;
\end{tikzpicture}
\end{minipage}
}

\newcommand{\RepLevelHl}{
\begin{minipage}{0.10\textwidth}
\begin{tikzpicture}[xscale=\myscalex,yscale=\myscaley]
%\node (syl) at (0,0) {Mid1};
\node (Rt) at (0,1) {o};
\node (H) at (-0.5,2) {H};
\node (R) at (0.5,3) {l};
%\draw [thick] (syl.north) -- (Rt.south) ;
\draw [thick] (Rt.north) -- (H.south) ;
\draw [thick] (Rt.north) -- (R.south) ;
\end{tikzpicture}
\end{minipage}
}

\newcommand{\RepLevelLl}{
\begin{minipage}{0.10\textwidth}
\begin{tikzpicture}[xscale=\myscalex,yscale=\myscaley]
%\node (syl) at (0,0) {Lo};
\node (Rt) at (0,1) {o};
\node (H) at (-0.5,2) {L};
\node (R) at (0.5,3) {l};
%\draw [thick] (syl.north) -- (Rt.south) ;
\draw [thick] (Rt.north) -- (H.south) ;
\draw [thick] (Rt.north) -- (R.south) ;
\end{tikzpicture}
\end{minipage}
}

% --- Representations --- %

\newcommand{\RepLevel}{
\begin{minipage}{0.10\textwidth}
\begin{tikzpicture}[xscale=\myscalex,yscale=\myscaley]
\node (syl) at (0,0) {\textsigma};
\node (Rt) at (0,1) {o};
\node (H) at (-0.5,2) {\texttau};
\node (R) at (0.5,3) {\textrho};
\draw [thick] (syl.north) -- (Rt.south) ;
\draw [thick] (Rt.north) -- (H.south) ;
\draw [thick] (Rt.north) -- (R.south) ;
\end{tikzpicture}
\end{minipage}
}

\newcommand{\RepContour}{
\begin{minipage}{0.10\textwidth}
\begin{tikzpicture}[xscale=\myscalex,yscale=\myscaley]
\node (syl) at (0,0) {\textsigma};
\node (Rt) at (0,1) {o};
\node (H) at (-0.5,2) {\texttau};
\node (R) at (0.5,3) {\textrho};
\node (Rt2) at (1.5,1.0) {o};
%\node (H2) at (1.0,2) {$\tau$};
%\node (R2) at (2.0,2.5) {R};
\draw [thick] (syl.north) -- (Rt.south) ;
\draw [thick] (Rt.north) -- (H.south) ;
\draw [thick] (Rt.north) -- (R.south) ;
\draw [thick] (syl.north) -- (Rt2.south) ;
%\draw [thick] (Rt2.north) -- (H2.south) ;
%\draw [thick] (Rt2.north) -- (R2.south) ;
\end{tikzpicture}
\end{minipage}
}


% --- OT constraints --- %

\newcommand{\IllustrationDown}{
\begin{minipage}{0.09\textwidth}
\begin{tikzpicture}[xscale=0.7,yscale=0.45]
\node (reg) at (0,0.75) {{\small \textalpha}};
\node (arrow) at (0,0) {{\fns $\downarrow$}};
\node (Rt) at (0,-0.75) {{\small \textbeta}};
\end{tikzpicture}
\end{minipage}
}

\newcommand{\IllustrationUp}{
\begin{minipage}{0.09\textwidth}
\begin{tikzpicture}[xscale=0.7,yscale=0.45]
\node (reg) at (0,0.75) {{\small \textalpha}};
\node (arrow) at (0,0) {{\fns $\uparrow$}};
\node (Rt) at (0,-0.75) {{\small \textbeta}};
\end{tikzpicture}
\end{minipage}
}

\newcommand{\MaxAB}{
\begin{minipage}{0.09\textwidth}
\begin{tikzpicture}[xscale=0.6,yscale=0.4]
\node (max) at (0,0) {{\small \textsc{Max}}};
\node (reg) at (0.75,0.5) {{\fns \textalpha}};
\node (arrow) at (0.75,0) {{\tiny $\downarrow$}};
\node (Rt) at (0.75,-0.5) {{\fns \textbeta}};
\end{tikzpicture}
\end{minipage}
}

\newcommand{\DepAB}{
\begin{minipage}{0.09\textwidth}
\begin{tikzpicture}[xscale=0.6,yscale=0.4]
\node (max) at (0,0) {{\small \textsc{Dep}}};
\node (reg) at (0.75,0.5) {{\fns \textalpha}};
\node (arrow) at (0.75,0) {{\tiny $\downarrow$}};
\node (Rt) at (0.75,-0.5) {{\fns \textbeta}};
\end{tikzpicture}
\end{minipage}
}

\newcommand{\DepHReg}{
\begin{minipage}{0.055\textwidth}
\begin{tikzpicture}[xscale=0.6,yscale=0.4]
\node (dep) at (0,0) {{\small \textsc{Dep}}};
\node (reg) at (0,-1.0) {{\small h}};
\end{tikzpicture}
\end{minipage}
}

\newcommand{\DepLReg}{
\begin{minipage}{0.055\textwidth}
\begin{tikzpicture}[xscale=0.6,yscale=0.4]
\node (dep) at (0,0) {{\small \textsc{Dep}}};
\node (reg) at (0,-1.0) {{\small l}};
\end{tikzpicture}
\end{minipage}
}

\newcommand{\DepReg}{
\begin{minipage}{0.055\textwidth}
\begin{tikzpicture}[xscale=0.6,yscale=0.4]
\node (dep) at (0,0) {{\small \textsc{Dep}}};
\node (reg) at (0,-1.0) {{\small \textrho}};
\end{tikzpicture}
\end{minipage}
}

\newcommand{\DepTRt}{
\begin{minipage}{0.1\textwidth}
\begin{tikzpicture}[xscale=0.6,yscale=0.4]
\node (dep) at (0,0) {{\small \textsc{Dep}}};
\node (t) at (0.75,0.5) {{\fns \texttau}};
\node (arrow) at (0.75,0) {{\tiny $\downarrow$}};
\node (Rt) at (0.75,-0.5) {{\fns o}};
\end{tikzpicture}
\end{minipage}
}

\newcommand{\MaxRegRt}{
\begin{minipage}{0.1\textwidth}
\begin{tikzpicture}[xscale=0.6,yscale=0.4]
\node (max) at (0,0) {{\small \textsc{Max}}};
\node (arrow) at (0.75,0) {{\tiny $\downarrow$}};
\node (Rt) at (0.75,-0.5) {{\fns o}};
\node (reg) at (0.75,0.5) {{\fns \textrho}};
\end{tikzpicture}
\end{minipage}
}

\newcommand{\RegToneByRt}{
\begin{minipage}{0.06\textwidth}
\begin{tikzpicture}[xscale=0.6,yscale=0.5]
\node[rotate=20] (arrow1) at (-0.15,0) {{\fns $\uparrow$}};
\node[rotate=340] (arrow2) at (0.15,0) {{\fns $\uparrow$}};
\node (Rt) at (0,-0.55) {{\small o}};
\node (reg) at (0.4,0.55) {{\small \textrho}};
\node (tone) at (-0.4,0.55) {{\small \texttau}};
\end{tikzpicture}
\end{minipage}
}

\newcommand{\RegToneBySyl}{
\begin{minipage}{0.06\textwidth}
\begin{tikzpicture}[xscale=0.6,yscale=0.5]
\node[rotate=20] (arrow1) at (-0.15,0) {{\fns $\uparrow$}};
\node[rotate=340] (arrow2) at (0.15,0) {{\fns $\uparrow$}};
\node (Rt) at (0,-0.55) {{\small \textsigma}};
\node (reg) at (0.4,0.55) {{\small \textrho}};
\node (tone) at (-0.4,0.55) {{\small \texttau}};
\end{tikzpicture}
\end{minipage}
}

\newcommand{\DepTone}{
\begin{minipage}{0.055\textwidth}
\begin{tikzpicture}[xscale=0.6,yscale=0.4]
\node (dep) at (0,0) {{\small \textsc{Dep}}};
\node (tone) at (0,-1.0) {{\small \texttau}};
\end{tikzpicture}
\end{minipage}
}

\newcommand{\DepTonalRt}{
\begin{minipage}{0.055\textwidth}
\begin{tikzpicture}[xscale=0.6,yscale=0.4]
\node (dep) at (0,0) {{\small \textsc{Dep}}};
\node (tone) at (0,-1.0) {{\small o}};
\end{tikzpicture}
\end{minipage}
}

\newcommand{\DepL}{
\begin{minipage}{0.055\textwidth}
\begin{tikzpicture}[xscale=0.6,yscale=0.4]
\node (dep) at (0,0) {{\small \textsc{Dep}}};
\node (tone) at (0,-1.0) {{\small L}};
\end{tikzpicture}
\end{minipage}
}

\newcommand{\DepH}{
\begin{minipage}{0.055\textwidth}
\begin{tikzpicture}[xscale=0.6,yscale=0.4]
\node (dep) at (0,0) {{\small \textsc{Dep}}};
\node (tone) at (0,-1.0) {{\small H}};
\end{tikzpicture}
\end{minipage}
}

\newcommand{\NoMultDiff}{{\small *loh}}
\newcommand{\Alt}{{\small \textsc{Alt}}}
\newcommand{\NoSkip}{{\small \scell{\textsc{No}\\\textsc{Skip}}}}


\newcommand{\RegDomRt}{
\begin{minipage}{0.030\textwidth}
\begin{tikzpicture}[xscale=0.6,yscale=0.5]
\node (arrow) at (0,0) {{\fns $\downarrow$}};
\node (Rt) at (0,-0.55) {{\small o}};
\node (reg) at (0,0.55) {{\small \textrho}};
\end{tikzpicture}
\end{minipage}
}

\newcommand{\DepRegRt}{
\begin{minipage}{0.1\textwidth}
\begin{tikzpicture}[xscale=0.6,yscale=0.4]
\node (dep) at (0,0) {{\small \textsc{Dep}}};
\node (arrow) at (0.75,0) {{\tiny $\downarrow$}};
\node (Rt) at (0.75,-0.5) {{\fns o}};
\node (reg) at (0.75,0.5) {{\fns \textrho}};
\end{tikzpicture}
\end{minipage}
}

% unused

\newcommand{\ToneByRt}{
\begin{minipage}{0.05\textwidth}
\begin{tikzpicture}[xscale=0.6,yscale=0.5]
\node (arrow) at (0,0) {{\fns $\uparrow$}};
\node (Rt) at (0,-0.55) {{\small o}};
\node (tone) at (0,0.55) {{\small \texttau}};
\end{tikzpicture}
\end{minipage}
}

\newcommand{\RegByRt}{
\begin{minipage}{0.05\textwidth}
\begin{tikzpicture}[xscale=0.6,yscale=0.5]
\node (arrow) at (0,0) {{\fns $\uparrow$}};
\node (Rt) at (0,-0.55) {{\small o}};
\node (reg) at (0,0.55) {{\small \textrho}};
\end{tikzpicture}
\end{minipage}
}

\newcommand{\ToneDomRt}{
\begin{minipage}{0.05\textwidth}
\begin{tikzpicture}[xscale=0.6,yscale=0.5]
\node (arrow) at (0,0) {{\fns $\downarrow$}};
\node (Rt) at (0,-0.55) {{\small o}};
\node (tone) at (0,0.55) {{\small \texttau}};
\end{tikzpicture}
\end{minipage}
}

% --- OT tableaus --- %

% Sec. 3.2, first tabl.

\newcommand{\OTHLInput}{
\begin{minipage}{0.17\textwidth}
\begin{tikzpicture}[xscale=\myscalex,yscale=\myscaley]
\node (tone) at (2,0) {(= H)};
\node (syl) at (0,0) {\textsigma};
\node (Rt) at (0,1) {o};
\node (H) at (-0.5,2) {H};
\node (R) at (0.5,3) {h};
\node (Rt2) at (1.5,1.0) {o};
%\node (H2) at (1.0,2) {\epen{L}};
\node (R2) at (2.0,3) {\blue{l}};
\draw [thick] (syl.north) -- (Rt.south) ;
\draw [thick] (Rt.north) -- (H.south) ;
\draw [thick] (Rt.north) -- (R.south) ;
\draw [thick] (syl.north) -- (Rt2.south) ;
%\draw [dashed] (Rt2.north) -- (H2.south) ;
%\draw [dashed] (Rt2.north) -- (R2.south) ;
\end{tikzpicture}
\end{minipage}
}

\newcommand{\OTHLWinner}{
\begin{minipage}{0.17\textwidth}
\begin{tikzpicture}[xscale=\myscalex,yscale=\myscaley]
\node (tone) at (2,0) {(= HL)};
\node (syl) at (0,0) {\textsigma};
\node (Rt) at (0,1) {o};
\node (H) at (-0.5,2) {H};
\node (R) at (0.5,3) {h};
\node (Rt2) at (1.5,1.0) {o};
\node (H2) at (1.0,2) {\epen{L}};
\node (R2) at (2.0,3) {\blue{l}};
\draw [thick] (syl.north) -- (Rt.south) ;
\draw [thick] (Rt.north) -- (H.south) ;
\draw [thick] (Rt.north) -- (R.south) ;
\draw [thick] (syl.north) -- (Rt2.south) ;
\draw [dashed] (Rt2.north) -- (H2.south) ;
\draw [dashed] (Rt2.north) -- (R2.south) ;
\end{tikzpicture}
\end{minipage}
}

\newcommand{\OTHLSpreadingHOnly}{
\begin{minipage}{0.17\textwidth}
\begin{tikzpicture}[xscale=\myscalex,yscale=\myscaley]
\node (tone) at (2,0) {(= HM)};
\node (syl) at (0,0) {\textsigma};
\node (Rt) at (0,1) {o};
\node (H) at (-0.5,2) {H};
\node (R) at (0.5,3) {h};
\node (Rt2) at (1.5,1.0) {o};
%\node (H2) at (1.0,2) {\epen{L}};
\node (R2) at (2.0,3) {\blue{l}};
\draw [thick] (syl.north) -- (Rt.south) ;
\draw [thick] (Rt.north) -- (H.south) ;
\draw [thick] (Rt.north) -- (R.south) ;
\draw [thick] (syl.north) -- (Rt2.south) ;
\draw [dashed] (Rt2.north) -- (R2.south) ;
\draw [dashed] (Rt2.north) -- (H.south) ;
\end{tikzpicture}
\end{minipage}
}

\newcommand{\OTHLInsertH}{
\begin{minipage}{0.17\textwidth}
\begin{tikzpicture}[xscale=\myscalex,yscale=\myscaley]
\node (tone) at (2,0) {(= HM)};
\node (syl) at (0,0) {\textsigma};
\node (Rt) at (0,1) {o};
\node (H) at (-0.5,2) {H};
\node (R) at (0.5,3) {h};
\node (Rt2) at (1.5,1.0) {o};
\node (H2) at (1.0,2) {\epen{H}};
\node (R2) at (2.0,3) {\blue{l}};
\draw [thick] (syl.north) -- (Rt.south) ;
\draw [thick] (Rt.north) -- (H.south) ;
\draw [thick] (Rt.north) -- (R.south) ;
\draw [thick] (syl.north) -- (Rt2.south) ;
\draw [dashed] (Rt2.north) -- (H2.south) ;
\draw [dashed] (Rt2.north) -- (R2.south) ;
\end{tikzpicture}
\end{minipage}
}

\newcommand{\OTHLOverwriting}{
\begin{minipage}{0.17\textwidth}
\begin{tikzpicture}[xscale=\myscalex,yscale=\myscaley]
\node (syl) at (0,0) {\textsigma};
\node (Rt) at (0,1) {o};
\node (H) at (-0.5,2) {H};
\node (R) at (0.5,3) {h};
\node (Rt2) at (1.5,1.0) {o};
%\node (H2) at (1.0,2) {\epen{L}};
\node (R2) at (2.0,3) {\blue{l}};
\draw [thick] (syl.north) -- (Rt.south) ;
\draw [thick] (Rt.north) -- (H.south) ;
\draw [thick] (Rt.north) -- (R.south) ;
\draw [thick] (syl.north) -- (Rt2.south) ;
%\draw [dashed] (Rt2.north) -- (H2.south) ;
\draw [dashed] (Rt.north) -- (R2.south) ;
\node (del) at (0.3,1.9) {\textbf{=}};
\end{tikzpicture}
\end{minipage}
}

\newcommand{\OTHLSpreading}{
\begin{minipage}{0.17\textwidth}
\begin{tikzpicture}[xscale=\myscalex,yscale=\myscaley]
\node (syl) at (0,0) {\textsigma};
\node (Rt) at (0,1) {o};
\node (H) at (-0.5,2) {H};
\node (R) at (0.5,3) {h};
\node (Rt2) at (1.5,1.0) {o};
%\node (H2) at (1.0,2) {\epen{L}};
\node (R2) at (2.0,3) {\blue{l}};
\draw [thick] (syl.north) -- (Rt.south) ;
\draw [thick] (Rt.north) -- (H.south) ;
\draw [thick] (Rt.north) -- (R.south) ;
\draw [thick] (syl.north) -- (Rt2.south) ;
%\draw [dashed] (Rt2.north) -- (H2.south) ;
\draw [dashed] (Rt2.north) -- (H.south) ;
\draw [dashed] (Rt2.north) -- (R.south) ;
\end{tikzpicture}
\end{minipage}
}

% Sec. 4.2, second tabl.: phrase-medial position

\newcommand{\OTHnoLInput}{
\begin{minipage}{0.17\textwidth}
\begin{tikzpicture}[xscale=\myscalex,yscale=\myscaley]
\node (tone) at (2,0) {(= H)};
\node (syl) at (0,0) {\textsigma};
\node (Rt) at (0,1) {o};
\node (H) at (-0.5,2) {H};
\node (R) at (0.5,3) {h};
\node (Rt2) at (1.5,1.0) {o};
%\node (H2) at (1.0,2) {\epen{L}};
%\node (R2) at (2.0,3) {\blue{l}};
\draw [thick] (syl.north) -- (Rt.south) ;
\draw [thick] (Rt.north) -- (H.south) ;
\draw [thick] (Rt.north) -- (R.south) ;
\draw [thick] (syl.north) -- (Rt2.south) ;
\end{tikzpicture}
\end{minipage}
}

\newcommand{\OTHnoLEpenth}{
\begin{minipage}{0.17\textwidth}
\begin{tikzpicture}[xscale=\myscalex,yscale=\myscaley]
\node (tone) at (2,0) {(= HM)};
\node (syl) at (0,0) {\textsigma};
\node (Rt) at (0,1) {o};
\node (H) at (-0.5,2) {H};
\node (R) at (0.5,3) {h};
\node (Rt2) at (1.5,1.0) {o};
\node (H2) at (1.0,2) {\epen{L}};
\node (R2) at (2.0,3) {\epen{h}};
\draw [thick] (syl.north) -- (Rt.south) ;
\draw [thick] (Rt.north) -- (H.south) ;
\draw [thick] (Rt.north) -- (R.south) ;
\draw [thick] (syl.north) -- (Rt2.south) ;
\draw [dashed] (Rt2.north) -- (H2.south) ;
\draw [dashed] (Rt2.north) -- (R2.south) ;
\end{tikzpicture}
\end{minipage}
}

\newcommand{\OTHnoLSpreading}{
\begin{minipage}{0.17\textwidth}
\begin{tikzpicture}[xscale=\myscalex,yscale=\myscaley]
\node (tone) at (2,0) {(= HH)};
\node (syl) at (0,0) {\textsigma};
\node (Rt) at (0,1) {o};
\node (H) at (-0.5,2) {H};
\node (R) at (0.5,3) {h};
\node (Rt2) at (1.5,1.0) {o};
%\node (H2) at (1.0,2) {\epen{L}};
%\node (R2) at (2.0,3) {\blue{l}};
\draw [thick] (syl.north) -- (Rt.south) ;
\draw [thick] (Rt.north) -- (H.south) ;
\draw [thick] (Rt.north) -- (R.south) ;
\draw [thick] (syl.north) -- (Rt2.south) ;
\draw [dashed] (Rt2.north) -- (H.south) ;
\draw [dashed] (Rt2.north) -- (R.south) ;
\end{tikzpicture}
\end{minipage}
}

% Sec. 4.2, third tabl., LM is unaffected by L\%

\newcommand{\OTLMInput}{
\begin{minipage}{0.2\textwidth}
\begin{tikzpicture}[xscale=\myscalex,yscale=\myscaley]
\node (tone) at (2,0) {(= LM)};
\node (syl) at (0,0) {\textsigma};
\node (Rt) at (0,1) {o};
\node (H) at (-0.5,2) {L};
\node (R) at (0.5,3) {l};
\node (Rt2) at (1.5,1.0) {o};
\node (H2) at (1.0,2) {L};
\node (R2) at (2.0,3) {h};
\node (R3) at (3.0,3) {\blue{l}};
\draw [thick] (syl.north) -- (Rt.south) ;
\draw [thick] (Rt.north) -- (H.south) ;
\draw [thick] (Rt.north) -- (R.south) ;
\draw [thick] (syl.north) -- (Rt2.south) ;
\draw [thick] (Rt2.north) -- (H2.south) ;
\draw [thick] (Rt2.north) -- (R2.south) ;
\end{tikzpicture}
\end{minipage}
}

\newcommand{\OTLMReplace}{
\begin{minipage}{0.2\textwidth}
\begin{tikzpicture}[xscale=\myscalex,yscale=\myscaley]
\node (tone) at (2,0) {(= LL)};
\node (syl) at (0,0) {\textsigma};
\node (Rt) at (0,1) {o};
\node (H) at (-0.5,2) {L};
\node (R) at (0.5,3) {l};
\node (Rt2) at (1.5,1.0) {o};
\node (H2) at (1.0,2) {L};
\node (R2) at (2.0,3) {h};
\node (R3) at (3.0,3) {\blue{l}};
\draw [thick] (syl.north) -- (Rt.south) ;
\draw [thick] (Rt.north) -- (H.south) ;
\draw [thick] (Rt.north) -- (R.south) ;
\draw [thick] (syl.north) -- (Rt2.south) ;
\draw [thick] (Rt2.north) -- (H2.south) ;
\draw [thick] (Rt2.north) -- (R2.south) ;
\draw [dashed] (Rt2.north) -- (R3.south) ;
\node (del) at (1.8,2.1) {\textbf{=}};
\end{tikzpicture}
\end{minipage}
}

\newcommand{\OTLMTwoReg}{
\begin{minipage}{0.2\textwidth}
\begin{tikzpicture}[xscale=\myscalex,yscale=\myscaley]
\node (tone) at (2,0) {(= LML)};
\node (syl) at (0,0) {\textsigma};
\node (Rt) at (0,1) {o};
\node (H) at (-0.5,2) {L};
\node (R) at (0.5,3) {l};
\node (Rt2) at (1.5,1.0) {o};
\node (H2) at (1.0,2) {L};
\node (R2) at (2.0,3) {h};
\node (R3) at (3.0,3) {\blue{l}};
\draw [thick] (syl.north) -- (Rt.south) ;
\draw [thick] (Rt.north) -- (H.south) ;
\draw [thick] (Rt.north) -- (R.south) ;
\draw [thick] (syl.north) -- (Rt2.south) ;
\draw [thick] (Rt2.north) -- (H2.south) ;
\draw [thick] (Rt2.north) -- (R2.south) ;
\draw [dashed] (Rt2.north) -- (R3.south) ;
\end{tikzpicture}
\end{minipage}
}

% Sec. 4.2, fourth tabl., L is affected by L\% but M is not

\newcommand{\OTLInput}{
\begin{minipage}{0.17\textwidth}
\begin{tikzpicture}[xscale=\myscalex,yscale=\myscaley]
\node (tone) at (2,0) {(= L)};
\node (syl) at (0,0) {\textsigma};
\node (Rt) at (0,1) {o};
\node (H) at (-0.5,2) {L};
\node (R) at (0.5,3) {l};
\node (R2) at (2,3) {\blue{l}};
\draw [thick] (syl.north) -- (Rt.south) ;
\draw [thick] (Rt.north) -- (H.south) ;
\draw [thick] (Rt.north) -- (R.south) ;
\end{tikzpicture}
\end{minipage}
}

\newcommand{\OTLLowered}{
\begin{minipage}{0.17\textwidth}
\begin{tikzpicture}[xscale=\myscalex,yscale=\myscaley]
\node (tone) at (2,0) {(= LL)};
\node (syl) at (0,0) {\textsigma};
\node (Rt) at (0,1) {o};
\node (H) at (-0.5,2) {L};
\node (R) at (0.5,3) {l};
\node (R2) at (2,3) {\blue{l}};
\draw [thick] (syl.north) -- (Rt.south) ;
\draw [thick] (Rt.north) -- (H.south) ;
\draw [thick] (Rt.north) -- (R.south) ;
\draw [dashed] (Rt.north) -- (R2.south) ;
\end{tikzpicture}
\end{minipage}
}

\newcommand{\OTMInput}{
\begin{minipage}{0.17\textwidth}
\begin{tikzpicture}[xscale=\myscalex,yscale=\myscaley]
\node (tone) at (2,0) {(= M)};
\node (syl) at (0,0) {\textsigma};
\node (Rt) at (0,1) {o};
\node (H) at (-0.5,2) {L};
\node (R) at (0.5,3) {h};
\node (R2) at (2,3) {\blue{l}};
\draw [thick] (syl.north) -- (Rt.south) ;
\draw [thick] (Rt.north) -- (H.south) ;
\draw [thick] (Rt.north) -- (R.south) ;
\end{tikzpicture}
\end{minipage}
}

\newcommand{\OTMLowered}{
\begin{minipage}{0.17\textwidth}
\begin{tikzpicture}[xscale=\myscalex,yscale=\myscaley]
\node (tone) at (2,0) {(= ML)};
\node (syl) at (0,0) {\textsigma};
\node (Rt) at (0,1) {o};
\node (H) at (-0.5,2) {L};
\node (R) at (0.5,3) {h};
\node (R2) at (2,3) {\blue{l}};
\draw [thick] (syl.north) -- (Rt.south) ;
\draw [thick] (Rt.north) -- (H.south) ;
\draw [thick] (Rt.north) -- (R.south) ;
\draw [dashed] (Rt.north) -- (R2.south) ;
\end{tikzpicture}
\end{minipage}
}

% Sec. 4.2, fifth tableau, polar questions with level tones

\newcommand{\OTLPolIn}{
\begin{minipage}{0.20\textwidth}
\begin{tikzpicture}[xscale=\myscalex-0.05,yscale=\myscaley-0.05]
\node (tone) at (3.5,0) {(= L)};
\node (syl) at (0,0) {\textsigma};
\node (syl2) at (2,0) {\red{\textsigma}};
\node (Rt) at (0,1) {o};
\node (H) at (-0.5,2) {L};
\node (R) at (0.5,3) {l};
\node (Rt2) at (2,1) {\red{o}};
\draw [thick] (syl.north) -- (Rt.south) ;
\draw [thick,red] (syl2.north) -- (Rt2.south) ;
\draw [thick] (Rt.north) -- (H.south) ;
\draw [thick] (Rt.north) -- (R.south) ;
\end{tikzpicture}
\end{minipage}
}

\newcommand{\OTLPolDef}{
\begin{minipage}{0.20\textwidth}
\begin{tikzpicture}[xscale=\myscalex-0.05,yscale=\myscaley-0.05]
\node (tone) at (3.5,0) {(= L.M)};
\node (syl) at (0,0) {\textsigma};
\node (syl2) at (2,0) {\red{\textsigma}};
\node (Rt) at (0,1) {o};
\node (H) at (-0.5,2) {L};
\node (R) at (0.5,3) {l};
\node (H2) at (1.5,2) {\epen{L}};
\node (R2) at (2.5,3) {\epen{h}};
\node (Rt2) at (2,1) {\red{o}};
\draw [thick] (syl.north) -- (Rt.south) ;
\draw [thick,red] (syl2.north) -- (Rt2.south) ;
\draw [thick] (Rt.north) -- (H.south) ;
\draw [thick] (Rt.north) -- (R.south) ;
\draw [semithick,dashed] (Rt2.north) -- (H2.south) ;
\draw [semithick,dashed] (Rt2.north) -- (R2.south) ;
\end{tikzpicture}
\end{minipage}
}

\newcommand{\OTLPolAlt}{
\begin{minipage}{0.20\textwidth}
\begin{tikzpicture}[xscale=\myscalex-0.05,yscale=\myscaley-0.05]
\node (tone) at (3.5,0) {(= L.L)};
\node (syl) at (0,0) {\textsigma};
\node (syl2) at (2,0) {\red{\textsigma}};
\node (Rt) at (0,1) {o};
\node (H) at (-0.5,2) {L};
\node (R) at (0.5,3) {l};
\node (Rt2) at (2,1) {\red{o}};
\draw [thick] (syl.north) -- (Rt.south) ;
\draw [thick,red] (syl2.north) -- (Rt2.south) ;
\draw [thick] (Rt.north) -- (H.south) ;
\draw [thick] (Rt.north) -- (R.south) ;
\draw [semithick,dashed] (Rt2.north) -- (H.south) ;
\draw [semithick,dashed] (Rt2.north) -- (R.south) ;
\end{tikzpicture}
\end{minipage}
}

% Sec. 4.2, sixth tableau, polar questions with contour tones

\newcommand{\OTLLPolIn}{
\begin{minipage}{0.23\textwidth}
\begin{tikzpicture}[xscale=\myscalex-0.05,yscale=\myscaley-0.05]
\node (tone) at (5.2,0) {(= L)};
\node (syl) at (0,0) {\textsigma};
\node (syl3) at (3.4,0) {\red{\textsigma}};
\node (Rt) at (0,1) {o};
\node (Rt2) at (1.7,1) {o};
\node (Rt3) at (3.4,1) {\red{o}};
\node (H) at (-0.5,2) {L};
\node (R) at (0.5,3) {l};
\draw [thick] (syl.north) -- (Rt.south) ;
\draw [thick] (syl.north) -- (Rt2.south) ;
\draw [thick,red] (syl3.north) -- (Rt3.south) ;
\draw [thick] (Rt.north) -- (H.south) ;
\draw [thick] (Rt.north) -- (R.south) ;
\end{tikzpicture}
\end{minipage}
}

\newcommand{\OTLLPolDef}{
\begin{minipage}{0.23\textwidth}
\begin{tikzpicture}[xscale=\myscalex-0.05,yscale=\myscaley-0.05]
\node (tone) at (5.2,0) {(= L.M)};
\node (syl) at (0,0) {\textsigma};
\node (syl3) at (3.4,0) {\red{\textsigma}};
\node (Rt) at (0,1) {o};
\node (Rt2) at (1.7,1) {o};
\node (Rt3) at (3.4,1) {\red{o}};
\node (H) at (-0.5,2) {L};
\node (R) at (0.5,3) {l};
\node (H3) at (2.9,2) {\epen{L}};
\node (R3) at (3.9,3) {\epen{h}};
\draw [thick] (syl.north) -- (Rt.south) ;
\draw [thick] (syl.north) -- (Rt2.south) ;
\draw [thick,red] (syl3.north) -- (Rt3.south) ;
\draw [thick] (Rt.north) -- (H.south) ;
\draw [thick] (Rt.north) -- (R.south) ;
\draw [dashed] (Rt3.north) -- (H3.south) ;
\draw [dashed] (Rt3.north) -- (R3.south) ;
\end{tikzpicture}
\end{minipage}
}

\newcommand{\OTLLPolSkip}{
\begin{minipage}{0.23\textwidth}
\begin{tikzpicture}[xscale=\myscalex-0.05,yscale=\myscaley-0.05]
\node (tone) at (5.2,0) {(= L.L)};
\node (syl) at (0,0) {\textsigma};
\node (syl3) at (3.4,0) {\red{\textsigma}};
\node (Rt) at (0,1) {o};
\node (Rt2) at (1.7,1) {o};
\node (Rt3) at (3.4,1) {\red{o}};
\node (H) at (-0.5,2) {L};
\node (R) at (0.5,3) {l};
\draw [thick] (syl.north) -- (Rt.south) ;
\draw [thick] (syl.north) -- (Rt2.south) ;
\draw [thick,red] (syl3.north) -- (Rt3.south) ;
\draw [thick] (Rt.north) -- (H.south) ;
\draw [thick] (Rt.north) -- (R.south) ;
\draw [dashed] (Rt3.north) -- (H.south) ;
\draw [dashed] (Rt3.north) -- (R.south) ;
\end{tikzpicture}
\end{minipage}
}  
  
\newcommand{\ilit}[1]{#1\il{#1}}    
\newcommand{\isit}[1]{#1\is{#1}}  

\makeatletter
\let\thetitle\@title
\let\theauthor\@author 
\makeatother

\newcommand{\togglepaper}[1][0]{ 
  \bibliography{../localbibliography}
  %% hyphenation points for line breaks
%% Normally, automatic hyphenation in LaTeX is very good
%% If a word is mis-hyphenated, add it to this file
%%
%% add information to TeX file before \begin{document} with:
%% %% hyphenation points for line breaks
%% Normally, automatic hyphenation in LaTeX is very good
%% If a word is mis-hyphenated, add it to this file
%%
%% add information to TeX file before \begin{document} with:
%% \include{localhyphenation}
\hyphenation{
affri-ca-te
affri-ca-tes
com-ple-ments
par-a-digm
Sha-ron
Kings-ton
phe-nom-e-non
Daul-ton
Abu-ba-ka-ri
Ngo-nya-ni
Clem-ents 
King-ston
Tru-cken-brodt
Tab-leau
cophono-logies
mark-edness
Ti-gri-nya
a-mong
Car-stens
Lu-bu-ku-su
}
\hyphenation{
affri-ca-te
affri-ca-tes
com-ple-ments
par-a-digm
Sha-ron
Kings-ton
phe-nom-e-non
Daul-ton
Abu-ba-ka-ri
Ngo-nya-ni
Clem-ents 
King-ston
Tru-cken-brodt
Tab-leau
cophono-logies
mark-edness
Ti-gri-nya
a-mong
Car-stens
Lu-bu-ku-su
}
  \papernote{\scriptsize\normalfont
    \theauthor.
    \thetitle. 
    To appear in: 
    Emily Clem,   Peter Jenks \& Hannah Sande.
    Theory and description in African Linguistics: Selected papers from the 47th Annual Conference on African Linguistics.
    Berlin: Language Science Press. [preliminary page numbering]
  }
  \pagenumbering{roman}
  \setcounter{chapter}{#1}
  \addtocounter{chapter}{-1}
}

\newcommand{\upstep}{\textupstep}


% \newcounter{tableauxcounter}

\renewcommand{\textltailn}{ɲ}
\renewcommand{\textbardotlessj}{ɟ}

\newcommand{\emphkh}[1]{\textit{#1}} %originally \textbf, banned by the guidelines



\definecolor{lsDOIGray}{cmyk}{0,0,0,0.45}


\newcommand{\xuparrow}[1]{%
  {\left\uparrow\vbox to #1{}\right.\kern-\nulldelimiterspace}
}
\renewcommand \textupstep[1]{\char"A71B#1}
\renewcommand \textdownstep[1]{\char"A71C#1}
 
 \newcommand{\ꜛ}{\textsf{ꜛ}}
 
\def\biberror{\undefined}


\newcommand{\OTbox}[1]{\resizebox{.88\textwidth}{!}{#1}}
 
  \togglepaper[12]
}{}


\begin{document}
\maketitle 

\section{Introduction}\label{sec:kusmer:intro}

\ili{Tshivenḓa}\footnote{Guthrie S21; \textasciitilde 1.3m speakers in South
Africa (Limpopo Province) \& Zimbabwe.} shares with other Southern \ili{Bantu}
	languages a distinctive morphological alternation in the form of the
	present tense prefix, commonly termed the \textsc{\isi{conjoint}\slash \isi{disjoint}
	alternation}. As shown below, the simple present is expressed either by the
	prefix /a-/ (termed the \isi{disjoint} form) or /\o-/ (termed the \isi{conjoint}).

%\ex. \ag. ndi (a) ḽa ṋemeṋeme\\
		  %1s {\sc a} eat termite\\
		  %``I eat termite.''
	%\bg. ndi *(a) ḽa\\
		  %1s {\sc a} eat\\
		  %``I eat.''
 

\ea \ili{Tshivenḓa} (\ili{Bantu})\label{ex:kusmer:intro_disj}\footnote{Unless otherwise noted, all
examples are from my own fieldwork on \ili{Tshivenḓa}.}\\

	\ea\gll ndi (a) ḽá ṋemeṋeme\\
		 1\textsc{sg} \textsc{dsj} eat termite\\
		\glt `I eat termite.'
	
  
	\ex\gll ndi *(a) ḽá\\
		 1\textsc{sg} \textsc{dsj} eat\\
		 \glt`I eat.'
	\z

\z
  


In this paper, I will present new data from original fieldwork on \ili{Tshivenḓa}
which shows that the distribution of the \isi{disjoint} prefix in that language shows
a three-way distribution: It's obligatory in some contexts, impossible in
others, and optional elsewhere. This contrasts with other languages with this
alternation, e.g. isi\ili{Zulu} (\citealt{Halpert2015}), where the \isi{conjoint} and
\isi{disjoint} forms are typically in \isi{complementary distribution}, i.e. no optionality
is possible. 

I will also present new data data on the prosody of \ili{Tshivenḓa}, which strikingly
shows the same three-way distribution. The prosodic phenomenon in question,
penultimate lengthening, is common to many \ili{Bantu} languages and applies to some
large prosodic unit (typically taken to be the \isi{intonational phrase}). In
\ili{Tshivenḓa}, the penultimate syllable of the utterance is always lengthened, but
some utterance-internal penults may also be lengthened. I will demonstrate that
the same contexts conditioning the three-way split in the \isi{disjoint} prefix
condition a similar split in penultimate lengthening: In those contexts in
which the \isi{disjoint} prefix is required, penultimate lengthening is frequent; in
those contexts in which the prefix is impossible, penultimate lengthening is
vanishingly rare; and in those contexts in which the prefix is optional seem to
allow an intermediate frequency of lengthening.


I will argue that any analysis of these phenomena must capture the close
relation between the \isi{conjoint}\slash \isi{disjoint} alternation and prosody. I will then
present two possible analyses. In one, the \isi{disjoint} prefix is a purely prosodic
phenomenon in the sense that it is conditioned solely by the location of the
verb within an \isi{intonational phrase}.\footnote{This first proposal closely
mirrors one made in \citet{ChengDowning2009} for isi\ili{Zulu}. However,
\citet{Halpert2015} has convincingly argued that the isi\ili{Zulu} case cannot be
prosodic in nature and must have a deeply syntactic origin. The present study
cannot currently decide between these two possibilities; it may be the case
that a similar argument may be made for \ili{Tshivenḓa}.} In the other analysis,
\isi{information structure} plays the role of a ``third factor'' conditioning both the
\isi{disjoint} prefix and the \isi{prosodic structure}. I will discuss the consequences of
each of these analyses and propose further research to help decide between
these two options.

%\ex. \textbf{Conjoint / \isi{disjoint} generalization (Tshiven:
	%\a. \textbf{Disjoint (/a-/)}: appears when the verb is last in an $\upiota$ P.
	%\b. \textbf{Conjoint (/\O/)}: appears elsewhere.

% \protectedex{
% \ea \textbf{Conjoint / \isi{disjoint} generalization (\ili{Tshivenḓa}):}\\
% 	\begin{xlist}
% 	\ex \textbf{Disjoint (/a-/)}: appears when the verb is last in an
% 	\isi{intonational phrase}\\
% 	\ex \textbf{Conjoint (/\o-/)}: appears elsewhere
% \end{xlist}
% \z
% }

% In addition, I will argue that this phenomenon lends support to the model of
% prosody argued for in \citet{SelkirkLee2015}, in which \isi{prosodic structure}
% formation is a distinct process from structure-sensitive phonology. This model
% allows for the possibility of \isi{prosodic structure} which goes entirely unmarked
% by any particular phonological process. Something like this must be happening
% in \ili{Tshivenda}: While the distributions of utterance-internal penultimate lengthening
% and the \isi{disjoint} prefix allow us to diagnose the presence of an intonational
% phrase boundary in certain syntactic structures, in any given utterance that
% boundary may be marked by neither lengthening (which is a variable process) nor
% the \isi{disjoint} prefix (which only appears in the present tense). This view
% supports an indirect reference theory of prosody, in which structure-sensitive
% phonology makes reference not to syntactic structure but to some distinct
% \isi{prosodic structure}; in fact, the idea that prosodic phrases may be present but
% unmarked is a natural consequence of combining an indirect reference theory
% with variable phonological processes.

The structure of this paper is as follows. In \sectref{sec:kusmer:background},
I will discuss the \isi{disjoint} alternation in \ili{Tshivenḓa}, comparing and contrasting
it with other Southern \ili{Bantu} languages. I will then present in
\sectref{sec:kusmer:survey} the results of a survey on the acceptability of
\isi{conjoint} and \isi{disjoint} verb forms in different syntactic contexts, showing that
there is a three-way split in the acceptability of this prefix by syntactic
context. In \sectref{sec:kusmer:lengthening}, I'll go on to discuss the results of a
study on sentence-internal penultimate lengthening across a variety of
syntactic contexts, showing that the same three-way split in the distribution
emerges. In \sectref{sec:kusmer:analysis} I will present two possible models of the
relationship between \isi{disjoint} marking and prosody which can account for this
data. Finally, in \sectref{sec:kusmer:conclusion} I will discuss the advantages and
disadvantages of these models and propose possible future work.

\section{The conjoint\slash disjoint alternation}\label{sec:kusmer:background}

Southern \ili{Bantu} languages frequently show an alternation in the form of the verb
under certain tenses. For instance, in isi\ili{Zulu}, the simple present takes a
prefix /ya-/ in some contexts, but is /\o-/ elsewhere:

%\ex. \ag. uMlungisi u-  pheka iqanda\\
		  %M.	3s- cook  egg\\
		  %``Mlungisi is cooking an egg.''
		  %\bg. * uMlungisi u-  \textbf{ya-} pheka iqanda\\
		 %M.	3s- {\sc ya}- cook  egg\\

\ea \label{ex:kusmer:zulu_trans}
isi\ili{Zulu} \citep{Halpert2015}\\
	\ea[]{ \gll uMlungisi u-  pheka iqanda\\
		  M.	3s- cook  egg\\
		  \glt `Mlungisi is cooking an egg.'}
	  \ex[*]{\gll uMlungisi u-  \textbf{ya-} pheka iqanda\\
		 M.	3\textsc{s-} \textsc{ya}- cook  egg\\}
		\z
\z

		 %\ex. \ag. * uMlungisi u- pheka\\
			%M. 3s- cook\\
			%{~}
			%\bg. uMlungisi u- \textbf{ya-} pheka\\
		 %M.	3s- {\sc ya}- cook\\
		  %``Mlungisi is cooking.''

\ea \label{ex:kusmer:zulu_intrans}
	\ea[*]{\gll uMlungisi u- pheka\\
		M. 3\textsc{s-} cook\\}
\ex[]{\gll uMlungisi u- \textbf{ya-} pheka\\
	 M.	3\textsc{s-} \textsc{ya}- cook\\
  \glt `Mlungisi is cooking.'}
  \z
\z

% % \pagebreak

The short form of the verb (/\o-/) is traditionally termed the ``\isi{conjoint}'' form;
the long form (/ya-/) is called the ``\isi{disjoint}''. \citet{Halpert2015} gives the
following generalization for the distribution of these forms:

%\ex. \textbf{Conjoint-\isi{disjoint} generalization (isi\ili{Zulu})}:
	%\a. \textbf{Conjoint (\o)}: appears when vP contains material (after A
	%\isi{movement})
	%\b. \textbf{Disjoint (\textit{ya})}: appears when vP does not contain
	%material (after A \isi{movement})

\ea\label{ex:kusmer:zulu_gen}
{Conjoint-\isi{disjoint} generalization (isi\ili{Zulu})}:\\
	\ea {Conjoint (\o)}: appears when vP contains material (after A
	\isi{movement})
	\ex {Disjoint (\textit{ya})}: appears when vP does not contain
	material (after A \isi{movement})
	\z
\z

\newpage 
Note two key properties of this generalization:

\begin{enumerate}
	\item The forms of the verb are in complementary
		distribution.
	\item The distribution is predictable based on syntactic context.
\end{enumerate}

This seems to be the norm across Southern \ili{Bantu}: The \isi{disjoint} alternation is a
deeply (morpho-)syntactic fact. In fact, in isi\ili{Zulu} and other languages the
alternation appears in several different tense\slash \isi{aspect}\slash polarity combinations
with different morphological realizations, but with the same structural
generalization governing which form is realized.  In \ili{Tshivenḓa}, by contrast,
the \isi{disjoint} alternation appears only in the simple present tense -- all other
tense\slash \isi{aspect}\slash polarity combinations do not
alternate.\footnote{\citet{Creissels1996} shows that \ili{Setswana}, a
closely-related language, shows tonal reflexes of the \isi{conjoint}\slash \isi{disjoint}
alternation in some tenses. While I can confirm that no such alternation occurs
in the present tense, I currently lack detailed tonal data on other tenses.
However, \citet{Cassimjee1992} does not note any anomalous tonal alternations,
though she does note the present tense \isi{conjoint}\slash \isi{disjoint} distinction; while
this is not conclusive, it supports the hypothesis that \ili{Tshivenḓa} only shows
this alternation in the present tense.} \citet{Poulos1990}  gives the following
generalization about the distribution of the \isi{disjoint} prefix:

%\ex. \textbf{Conjoint-\isi{disjoint} generalization (\ili{Tshivenḓa}, to be revised)}: 
%\a. The \isi{disjoint} is available everywhere.
	%\b. The \isi{conjoint} is ungrammatical when the verb is last in the sentence.

\ea\label{ex:kusmer:poulos_gen}  {Conjoint-\isi{disjoint} generalization (\ili{Tshivenḓa},
after Poulos)}: \\
\ea The \isi{disjoint} is available everywhere.
	\ex The \isi{conjoint} is ungrammatical when the matrix verb is last in the
	sentence.\footnote{Poulos' original generalization ignores the distinction
	between matrix and embedded verbs; in other Southern \ili{Bantu} languages, the
	verb in a \isi{relative clause} may take \isi{conjoint} even when sentence-final. I
	lack detailed data on \ili{Tshivenḓa} relative clauses; however, see
	\sectref{sec:kusmer:conclusion} for further discussion.}
	\z

\z

In contrast to isi\ili{Zulu}, this generalization does not place the \isi{conjoint} \&
\isi{disjoint} forms in \isi{complementary distribution} -- rather, it seems to suggest
that the \isi{disjoint} is the default form, with a specialized \isi{conjoint} form
required only in certain contexts. It also makes no reference to anything
deeply syntactic in nature, but instead refers to the linear order of
constituents. I will show that while the details of this generalization are
inadequate -- the \isi{disjoint} is not in fact available everywhere, and the
\isi{conjoint} is ungrammatical in some cases where the verb is not last in the
sentence -- the underlying nature of this generalization is correct: The
\ili{Tshivenḓa} \isi{conjoint} \& \isi{disjoint} forms are not in \isi{complementary distribution}, and
their distribution seems to be based on post-syntactic conditions.

\section{Survey design and results}\label{sec:kusmer:survey}

I conducted a pilot study on the \isi{conjoint}\slash \isi{disjoint} alternation at the
University of Venḓa in Thohoyandou, Limpopo Province, South Africa. The study
consisted of a short questionnaire asking for grammaticality ratings on a
variety of sentences. The design of the survey was as follows:

\begin{itemize}

	\item 8 conditions, varying what kind of material followed the verb.

	\item Each sentence was presented twice: once in the \isi{conjoint}, once in the
	\isi{disjoint}.

	\item A total of 56 test items were presented, plus 44 fillers
		(grammatical) / controls (ungrammatical) = 100 questions

	\item 12 native speakers of \ili{Tshivenḓa} were asked to rate items from 1
		(``mistaken or incomplete'') to 5 (``natural and complete'').

\end{itemize}

\noindent
The conditions varied based on what material followed the verb:

\begin{enumerate}
\item{}\textbf{final} the verb was sentence final.
\item{}\textbf{temporal} the verb was followed by a \isi{temporal adverb} (`today',now').
\item{}\textbf{locative} followed by a locative adverb (`at home', `in the forest').
\item{}\textbf{manner} followed by a manner adverb (`well', `badly').
\item{}\textbf{fhedzi} followed by the focus-sensitive operator \textit{fhedzi} (`only').
\item{}\textbf{secondary} followed by a \isi{secondary predicate} (`go to the tree').
\item{}\textbf{object} transitive verb + in situ object.
\item{}\textbf{dislocated} transitive verb + right-dislocated object.
\end{enumerate}

A few of these conditions merit some further explanation. First, the
\textsc{dislocated} condition included sentences in which the \isi{direct object} was
coreferenced by an \isi{object marker} on the verb. In many \ili{Bantu} languages,
including \ili{Tshivenḓa}, objects coreferenced in this manner are generally not
in their base position inside the vP \citep{Buell2005}. For instance, as shown in
\REF{ex:kusmer:insitu-demonstration}, it is possible to separate a coreferenced object from the verb with an
adverb; this is not possible with a non-coreferenced object.

%\ex. \ag. Tshiṋoni tshi a dzhia (*zwino) thanga\\
		  %7.bird  \textsc{s.7} \textsc{dsj} now 9.seed\\
		  %``The bird takes (*now) a seed.''
	%\bg. Tshiṋoni tshi a i dzhi zwino thanga\\
		  %7.bird  \textsc{s.7} \textsc{a} \textsc{o.9} now 9.seed\\
		  %``The bird takes it now, the seed.

\ea \label{ex:kusmer:insitu-demonstration}
\ea \gll Tshiṋoni tshi a dzhia (*zwíno) thanga\\
		  7.bird  \textsc{s.}7 \textsc{dsj} take now 9.seed\\
		\glt  `The bird takes (*now) a seed.'
  \ex \gll Tshiṋoni tshi a í dzhi zwíno thanga\\
		  7.bird  \textsc{s.}7 \textsc{dsj} 9\textsc{.obj} take now 9.seed\\
	 \glt `The bird takes it now, the seed.'
	 \z
 \z

The \textsc{secondary} block included sentences in which the verb was followed
by a clausal adjunct marked with the dependent prefix \textit{tshi-}
(\citealt{Warmelo1989}):

%\exg. nḓou i gidima i tshi ya dakani\\
%9.elephant \textsc{s.9} ({\sc a}) run \textsc{s.9} \textsc{dep} go forest.{\sc loc}\\
	 %``The elephant runs into the forest.''

\ea \label{ex:kusmer:secondary}
\gll  nḓou í (a) gidima í tshi ya daka -ni\\
	9.elephant 9.\textsc{subj} (\textsc{dsj}) run 9.\textsc{subj} \textsc{dep}
	go forest \textsc{loc}\\
	\glt  `The elephant runs into the forest.'
 \z


Finally, in the \textsc{fhedzi} condition the verb was followed by the
focus-sensitive operator \textit{fhedzi}, which may be roughly glossed as
`only'. The intention was for this to narrowly \isi{scope} over the VP. However, the
results show that speakers mostly rejected these sentences (regardless of which
form the verb took), indicating that perhaps this narrow \isi{scope} is difficult to
arrive at pragmatically. This condition will be discarded in the
analysis here.


\subsection{Results and analysis}

\figref{fig:kusmer:rawscores} shows the mean ratings per speaker for each
condition, including controls and fillers.\footnote{This box-and-whisker plot
	should be read as follows: The dark horizontal mark indicates the median
	overall rating. The box extends out on either side to the edges of the
	1\textsuperscript{st} and 3\textsuperscript{rd} quartiles, while the
	``whiskers'' extend out to 1.5 times the interquartile range; if no box or
	whisker is drawn, this indicates that the quartiles are at the median
itself, i.e.  that most responses are at the median.  Speakers whose
average response in that condition fell  outside of the extent of the whiskers
are regarded as outliers and plotted as individual points.} The dashed lines
separate out conditions into groups with similar behavior.

% % % \pagebreak


\begin{figure}[p]
	\caption{Raw ratings of conjoint\slash disjoint forms, by condition}
	\label{fig:kusmer:rawscores}
\includegraphics[width=\textwidth]{figures/proc-raw.png}
\end{figure}

Within each condition, I calculated a by-speaker mean difference score between
ratings given to the \isi{disjoint} and to the \isi{conjoint} sentences. In the resulting
score, a positive value indicates that the speaker preferred the \isi{disjoint} form
of the verb, and a negative score that they preferred the \isi{conjoint}. If the score
is not significantly different from zero, then no preference can be assessed.
In \figref{fig:kusmer:meanscores}, error bars indicate 95\% confidence intervals.


\begin{figure}[p]
	\caption{Conjoint / disjoint preferences, by condition}
	\label{fig:kusmer:meanscores}
\includegraphics[width=\textwidth]{figures/proc-results.png}
\end{figure}

% % \pagebreak

From \figref{fig:kusmer:meanscores}, it can be seen that the \textsc{final} and
\textsc{dislocated} conditions show a significant\footnote{Significance was
assessed at the 0.05 level using the Holm-Bonferroni correction for multiple
comparisons.} preference for the \isi{disjoint}; the \textsc{adverb} and
\textsc{object} conditions show no significant difference from zero; and only
the \textsc{secondary} condition shows a significant preference for the
\isi{conjoint}. Together with the fact that the \textsc{adverb} and \textsc{object}
conditions generally received ratings at ceiling, these results show clearly
that there is a three-way split in the grammaticality of the \isi{conjoint} and
\isi{disjoint} forms of the verb, summarized in \tabref{tab:kusmer:avail}.

%\ex. \textbf{Conjoint / \isi{disjoint} availability by context:}

\begin{table}
	\caption{Conjoint/disjoint availability by context}
	\label{tab:kusmer:avail}
\begin{tabular}{ll}
	\lsptoprule
	\textsc{Final} & Disjoint \\
	\textsc{Dislocated object} & Disjoint \\
	\midrule
	\textsc{Adverb} & Either\\
	\textsc{In situ object} & Either\\
	\midrule
	\textsc{Secondary predicate} & Conjoint\\
	\lspbottomrule
\end{tabular}
\end{table}

\bigskip

Compare this distribution with the generalization stated in \citet{Poulos1990}.
This generalization is proven false on two counts: First, the \isi{disjoint} form is
not in fact available everywhere -- in particular, when a \isi{secondary predicate}
follows the verb, the \isi{disjoint} is ungrammatical. Second, the \isi{conjoint} is
ungrammatical in some situations where the verb is not last in the sentence.
However, in at least some contexts, it is true that the \isi{conjoint} and \isi{disjoint}
forms are equally acceptable. This contrasts with the situation in most other
southern \ili{Bantu} languages, particularly isi\ili{Zulu}, where the availability of the
two forms is strictly determined by the syntactic context. I take this as
evidence that the \isi{disjoint} alternation in \ili{Tshivenḓa} is a different class of
phenomenon from the other \ili{Bantu} languages. In particular, in the sections that
follow, I will present evidence that the alternation is prosodically
conditioned in \ili{Tshivenḓa}, and that the optionality of the \isi{disjoint} prefix
corresponds precisely to optionality in the \isi{prosodic phrasing}.


\section{Penultimate lengthening}\label{sec:kusmer:lengthening}

The same syntactic contexts which condition the availability of the \isi{conjoint}
and \isi{disjoint} forms also differ systematically in their prosodic properties,
specifically in the distribution of penultimate lengthening. \ili{Tshivenḓa} does not
have lexically contrastive \isi{vowel length}, but lengthens the penultimate syllable
of intonational phrases:

%\ex. \ag. ndo mbindimedza {\bf ludambwa:na}\\
	  %1s.{\sc pst} destroy 11.dam\\
	  %``I destroyed the dam.''
	  %\bg. ndo mbindimedza ludambwana {\bf namu:si}\\
		   %1s.{\sc pst} destroy 11.dam today\\
		   %``I destroyed the dam today.''

\ea \label{ex:kusmer:lengthening}
	\ea \gll ndó mbíndímédza \textbf{ludambwa:na}\\
	  1sg.\textsc{pst} destroy 11.dam\\
	  \glt `I destroyed the dam.'
	  \ex
	  \gll ndó mbíndímédza ludambwana \textbf{namú:si}\\
	  1sg.\textsc{pst} destroy 11.dam today\\
	\glt `I destroyed the dam today.'
		\z
\z

Penultimate lengthening is common across the \ili{Bantu} family
\citep{Hyman2013penultimate}. It is typically regarded as a phonological (rather than
phonetic) lengthening on the grounds that it may have other effects on the
suprasegmental phonology of the utterance, in particular on \isi{tone}. \ili{Tshivenḓa}
shares with many other \ili{Bantu} languages the property that contour tones may only
occur on lengthened penults, which is typically taken to indicate that the
lengthening adds a tone-bearing unit (e.g. a mora) to the target syllable.

The penult of the entire (declarative) utterance is always lengthened. However,
there may be utterance-internal lengthening, as well. For example, in
\REF{ex:kusmer:internal_lengthening} \textit{ludambwa:na} shows penultimate
lengthening despite not being utterance-final.

%\exg. ndo mbindimedza {\bf ludambwa:na} {\bf namu:si}\\
		   %1s.{\sc pst} destroy 11.dam today\\
		   %``I destroyed the dam today.''

\ea\label{ex:kusmer:internal_lengthening}
\gll ndó mbíndímédza \textbf{ludambwa:na} \textbf{namú:si}\\
		   1sg.\textsc{pst} destroy 11.dam today\\
		\glt   `I destroyed the dam today.'
\z

Comparing \REF{ex:kusmer:internal_lengthening} and
(\ref{ex:kusmer:lengthening}b), it can be seen that internal lengthening in
this syntactic context is apparently variable.  However, there is room for
uncertainty about the source of this variability: If penultimate lengthening is
associated with the \isi{intonational phrase} level of \isi{prosodic structure}, then the
contrast between \REF{ex:kusmer:internal_lengthening} and (\ref{ex:kusmer:lengthening}b)
may indicate a contrast in intonational phrasing.  Alternatively, one might
propose that \REF{ex:kusmer:lengthening} still has an \isi{intonational phrase} boundary
after the verb, and what is variable is not the structure but the lengthening
itself. If the variability lies in the \isi{prosodic structure} formation, then one
might expect to find some syntactic contexts in which the \isi{prosodic structure} is
not variable and internal lengthening happens 100\% of the time. By contrast,
if variability lies in the structure-sensitive phonological lengthening only,
then even in syntactic contexts where the \isi{prosodic structure} was fixed, one
might expect lengthening to be variable. In fact, I will show below that the
distribution of utterance-internal lengthening shows a complicated three-way
distribution that indicates variability in both structure-sensitive phonology
and \isi{prosodic structure} formation.

I conducted a production study to determine the distribution of
sentence-in\-ter\-nal penultimate lengthening. The study comprised four syntactic
contexts which varied in what material followed the verb: \textsc{\textit{in
situ} direct objects}, \textbf{dislocated direct objects}, intransitive verbs
followed by \textsc{adverbs} (balanced across temporal, manner, and locative
adverbials), and \textsc{secondary predicate} clauses. Several other syntactic
contexts were also included and acted as controls for this study.  Within each
syntactic condition, sentences were balanced for other prosodic factors such as
the length and lexical \isi{tone} on the verb. 12 native speakers of \ili{Tshivenḓa} were
recorded with 3 repetitions per sentence; I'm reporting here on a subset of the
data including only 5 speakers and 1 repetition.

%I initially hand-coded all syllables as long or short. I then ran two tests to
%confirm the validity of my transcriptions. First, I ran a mixed-effects linear
%model to determine whether the two classes of syllables identified by my
%transcriptions really showed different distributions of length. Only the penult
%of the verb was considered, and these were compared across all four
%study conditions in addition to control conditions in which I believed the
%targets to be all short or all long. In order to control for both speech rate
%and the varying lengths of the words in question, I took as my dependent
%variable ratio of observed syllable duration to a naive `expected' duration in
%which the syllables divide the length of the word precisely evenly. That is:

%%\ex. \textbf{Syllable duration ratio}\ = \texttt{Duration\_{target} *
%%n\_{syllables}\ /\ Duration\_{word}}

%\ea \textbf{Syllable observed / expected ratio}\ = \texttt{Duration\_{target} *
%n\_{syllables}\ /\ Duration\_{word}}
%\z

%The mixed-effects model looked for a fixed effect of the transcription on the
%syllable duration ratio, with a random effect of the specific word used. The
%random effect is intended to control for phonetic properties specific to that
%word, such as inherent \isi{vowel length} or adjacent segments. That is, the model
%run was:

%%\ex. \texttt{Duration ratio \textasciitilde\ Transcription + (1 | Word)}

%\ea \texttt{Duration ratio \textasciitilde\ Transcription + (1 | Word)}
%\z

%This model showed a significant main effect of transcription at p < .001,
%confirming that the transcription does indeed reflect a real difference in
%duration.

%As a second confirmation, I trained a logistic classifier on 75\% of the data
%and tested on the remaining 25\%. The model was constructed as follows:

%%\ex. \texttt{Transcription \textasciitilde\ Duration\_{target} +
%%Duration\_{word}}

%\ea \texttt{Transcription \textasciitilde\ Duration ratio}
%\z

%This classifier was able to predict the transcription of the test data with
%85\% accuracy, again confirming that transcriptions reflect measurable
%differences in syllable duration.

%\subsection{Internal lengthening distribution}

%Once the accuracy of the transcriptions had been confirmed, I tabulated the
%percentage of tokens displaying utterance-internal penultimate lengthening on
%the verb within each syntactic condition:

After hand-coding all the syllables as long or short, I tabulated the
percentage of tokens displaying utterance-internal penultimate lengthening on
the verb within each syntactic condition. The results are shown in
\tabref{tab:kusmer:lengthening_percentages}.

%\ex. \textbf{Percentage of tokens with internal penultimate lengthening:}

\begin{table}
	\caption{Percentage of tokens with internal penultimate lengthening}
	\label{tab:kusmer:lengthening_percentages}
\begin{tabular}{lr}
	\lsptoprule
	\textsc{(Sentence-final)} & (100\%)\\
	\textsc{Dislocated object} & 60\%\\
	\midrule
	\textsc{Adverb} & 25\%\\
	\textsc{In situ object} & 15\%\\
	\midrule
	\textsc{Secondary predicate} & 5\%\\
\lspbottomrule
\end{tabular}
\end{table}

\largerpage
Strikingly, the distributions also show a three-way split: Utterance-internal
lengthening is common when only a dislocated object follows the verb; when an
 {in situ} object or an adverb follows the verb, lengthening is less
common; and when only a \isi{secondary predicate} follows the verb, lengthening is
vanishingly rare.\footnote{All but one of the \isi{secondary predicate} cases showing
internal lengthening come from the same speaker, who shows many signs of list
intonation in general.} Notably, the syntactic conditions on this distribution
are the same as for the \isi{conjoint}\slash \isi{disjoint} alternation: That is, verbs
followed by dislocated objects pattern the same as sentence-final verbs;
 {in situ} objects and adverbs pattern together, and secondary predicates
pattern a third way.\footnote{Such a correlation between prosody and \isi{disjoint}
marking has been noted before; see, for instance: \cite{vanderSpuy1993,
Buell2005, chengdowning2012} on \ili{Zulu}; \cite{Devos2008} on \ili{Makwe}. I'm grateful
to an anonymous reviewer for bringing these references to my attention.} This
overlap suggests a common origin for both phenomena; in the next section, I
will outline a model of \ili{Tshivena} prosody that explains the commonalities.

\section{Analysis}\label{sec:kusmer:analysis}

We have seen that both the \isi{conjoint}\slash \isi{disjoint} alternation and
sentence-internal penultimate lengthening show a three-way split in their
distributions, and that the syntactic conditions underlying this split pattern
alike between the two phenomena. I will first develop a model that can account
for the three-way split in penultimate lengthening. I will then discuss two
possible ways that the correlation between the prosody and the \isi{disjoint} prefix
can be explained. In one, the \isi{disjoint} prefix is directly conditioned by the
\isi{prosodic structure}; in the other, a ``third factor'' is introduced which
accounts for the variability in both \isi{prosodic phrasing} and \isi{disjoint}
marking.

\subsection{Penultimate lengthening and prosodic variability}

This distribution is challenging to explain under a model of prosody in which
the structure-sensitive phonological marking is in one-to-one correspondence
with the \isi{prosodic structure}. There are two challenging aspects to this
distribution: The first is that the internal marking is sometimes categorically
\textit{absent} (the \isi{secondary predicate} case), but is never categorically
\textit{present}. The second is that some contexts seem to show an intermediate
frequency of lengthening. This first property can be captured by proposing that
\isi{intonational phrase} is \textit{variably} marked by penultimate lengthening, so
that, even in contexts where the verb is always final in an intonational
phrase, the lengthening will not always be present. This second property can be
captured by specifying that these contexts are not actually uniform, but that
differences in the interpretation of  {in situ} objects and adverbs
changes whether they are prosodically grouped with the verb or not. Information
structure (e.g. \isi{focus} or givenness) is the most likely factor at play; since the
present study did not control \isi{information structure}, these differences might
appear as apparently random variation depending on what implicit context
subjects assign to the sentence.

To spell out this proposal in more detail:

\begin{itemize}

	\item I will assume an indirect reference theory of prosody
		(\citealt{Selkirk11}),
		in which prosody is split into two pieces: \isi{prosodic structure} building
		and structure-sensitive phonology.

	\item In particular, I will assume that each utterance has an abstract
		\isi{prosodic structure} which may or may not be marked in the phonology by
		e.g. penultimate lengthening. That is, it is the likelihood of marking,
		not the presence or absence, that indicates a boundary.
		(\citealt{Elfner2016})

	\item I will further assume that recursive prosodic structures are possible
		and that structure-sensitive phonology can make reference to maximal
		and non-maximal recursive phrases (\citealt{ItoMester12}). 
\end{itemize}

\noindent
I propose that penultimate lengthening is controlled by two rules:

%\ex. \textbf{penultimate lengthening rules:}
	%\a. Always lengthen the penultimate syllable of a maximal $\upiota$ P.
	 %\b. Variably lengthen the penultimate syllable of a non-maximal $\upiota$ P.

\ea \label{ex:kusmer:lengthening_rules} {Penultimate lengthening rules:}
\ea Always lengthen the penultimate syllable of a maximal $\upiota$ P.
\ex Variably lengthen the penultimate syllable of a non-maximal $\upiota$ P.
\z
\z

Consider the dislocated object case. I propose that these sentences have a
\isi{prosodic structure} like the following:\footnote{Space does not permit me to
include a full analysis of how the prosodic structures here are generated, but I
assume a constraint-based analysis along the lines of  ``{Match} Theory''
\citep{Selkirk11}.}


%\exg. {\lipm} {\lipn} ndo lu {\bf mbindime(:)dza} \ripn\ {\bf ludambwa:na} \ripm\\
%{} {} 1s.{\sc pst}  \textsc{o.11} destroy {} 11.dam {}\\
	  %``I destroyed the dam.''

\ea \label{ex:kusmer:disloc_pros} 
\gll {\lipm} {\lipn} ndó lú \textbf{mbíndímé(:)dza} {\ripn} \textbf{ludambwa:na} {\ripm}\\
{} {} 1sg.\textsc{pst}  \textsc{11.obj} destroy {} 11.dam {}\\
	\glt  `I destroyed the dam.'
\z

\begin{itemize}
	\item The object \textit{ludambwana} is final in a maximal $\upiota$ P and so is always
		lengthened.
	\item The verb \textit{mbíndímédza} is final in a non-maximal $\upiota$ P and so is
		variably lengthened.
	\item[$\rightarrow$] In my data: The verb is lengthened >50\% of the time.
\end{itemize}

\noindent
Consider next the \isi{secondary predicate} case. I propose that these sentences
have a \isi{prosodic structure} like the following:

%\exg. {\lipm} ndi gidima {\lipn} ndi tshi ya haya:ni {\ripn} {\ripm}\\
%{} 1s run {} 1s {\sc dep} go home.{\sc loc} {} {}\\
	%``I run home.''

\ea \label{ex:kusmer:sec_pros}
\gll {\lipm} ndi gidima {\lipn} ndi tshi ya háyá:ni {\ripn} {\ripm}\\
{} 1sg run {} 1sg \textsc{dep} go home.\textsc{loc} {} {}\\
	\glt `I run home.'
	\z

\begin{itemize}

	\item The goal \textit{hayani} is final in a maximal  $\upiota$ P and so is always
		lengthened.

	\item The main verb \textit{gidima} isn't final in any $\upiota$ P, and so is never lengthened.
	\item[$\rightarrow$] In my data: The verb is lengthened <5\% of the time.
\end{itemize}

Finally, consider the other cases -- adverbs and  {in situ} objects.
Here, I will propose that these sentences may be assigned on of two possible
structures. While I will remain neutral on what conditions each of these
structures, information structural factors such as \isi{focus} or givenness seems
likely; the experiment presented here did not control for these factors, and so
I will treat the choice between the two structures as essentially variable.


%\ex. \a. {\lipm} ndo ṅamaila ṋamu:si {\ripm}
%\bg. {\lipm} {\lipn} ndo ṅamai(:)la {\ripn} ṋamu:si \ripm\\
		 %{} {} 1s.{\sc pst} stagger {} today {}\\
		  %``I staggered today.''

\ea \label{ex:kusmer:var_pros} 
\glll a. {\hspace{1em}} {\lipm} {\lipn} ndo ṅamai(:)la {\ripn} ṋamu:si {\ripm}\\
b. {} {\lipm} {} ndó ṅámáíla {} ṋamú:si {\ripm}\\
{} {} {} {} 1sg.\textsc{pst} stagger {} today {}\\
		\glt  `I staggered today.'
\z

		  \begin{itemize}
	\item Under both prosodic structures, the adverb \textit{ṋamusi} is final in a
		maximal $\upiota$ P and is lengthened.
	\item Under (\ref{ex:kusmer:var_pros}a) there is no non-maximal $\upiota$ P and so no variable
		lengthening.
	\item Under (\ref{ex:kusmer:var_pros}b) the verb is final in a non-maximal $\upiota$ P and is
		variably lengthened.
	\item One thus expects sentence-internal lengthening to occur less frequently
		than with dislocated objects, but more frequently than with secondary
		predicates.
	\item[$\rightarrow$] In my data: The verb is lengthened \textasciitilde 20\%
		of the time.

\end{itemize}

Thus, one can understand the three-way split in penultimate lengthening as
arising from the combination of variation in \isi{prosodic structure} (probably
conditioned by information structural factors) with a variable
structure-sensitive phonology rule.\footnote{If this analysis is correct, we
should see corresponding tonal effects; space constraints will not permit a
discussion of \ili{Tshivenḓa} tone-spreading phenomena here.}

\subsection{Explaining the conjoint\slash disjoint alternation}

If the prosodic structures proposed above are correct, then the following
relationship between intonational phrases, lengthening, and \isi{disjoint} marking
obtains:

\begin{table}
	\caption{Summary of prosody and verb form relationship}
	\label{tab:kusmer:summary_ip_disj}
\begin{tabular}{llll}
\lsptoprule
\textsc{Condition} & \textsc{Last in $\upiota$ P?} & \textsc{Lengthened?} &
	\textsc{Form?}\\
	\midrule
{Dislocated obj}       & Always    & Frequently & Disjoint\\
{Adverb, in situ obj} & Sometimes & Sometimes  & Variable\\
{Secondary predicate}     & Never     & Rarely     & Conjoint\\
\lspbottomrule
\end{tabular}
\end{table}

It seems desirable to explain why \isi{disjoint} marking should track the prosodic
structure so closely. There are at least two possible analyses compatible with
the data presented here. The first is what I will term the \textsc{prosodic
disjoint} analysis, in which \isi{disjoint} marking is taken to be a direct
consequence of the \isi{prosodic structure}. More specifically, \ili{Tshivenḓa} \isi{disjoint}
marking would obey the following generalization:

%\ex. \textbf{Conjoint / \isi{disjoint} generalization (Tshiven:
	%\a. \textbf{Disjoint (/a-/)}: appears when the verb is last in an $\upiota$ P.
	%\b. \textbf{Conjoint (/\o/)}: appears elsewhere.

\ea \label{ex:kusmer:venda_gen_fin} {Conjoint / \isi{disjoint} generalization (\ili{Tshivenḓa}):}
	\ea {Disjoint (/a-/)}: appears when the verb is last in an $\upiota$
	P.\footnote{I remain agnostic as to how this distribution is achieved. The
	most likely option seems to be delention of the /a-/ prefix in the
	elsewhere case, which is somehow bled by the prosody. The alternative, that
	the prefix is actually inserted by the prosody, seems highly unusual based
	on previously-studied prosodic phenomena.}
	\ex {Conjoint (/\o-/)}: appears elsewhere.
	\z
\z

\largerpage
The prosodic \isi{disjoint analysis} represents a significant break from previous
scholarship on Southern \ili{Bantu} languages (see, for instance,
\citealt{Buell2005,ChengDowning2009}), which have typically analyzed \isi{disjoint}
marking as resulting from a combination of syntactic- and
information-structural factors. The \textsc{structural disjoint} analysis,
then, would propose that the correlations reported in
\tabref{tab:kusmer:summary_ip_disj} are the result of a ``third factor'': Insofar
as syntax and \isi{information structure} are capable of influencing both the prosody
and the verb form, we should expect these factors to be correlated with each
other. In this analysis, there is no direct link between \isi{disjoint} marking and
\isi{prosodic structure} at all.

The present study is not capable of distinguishing between these options. In
the next section, I will discuss some of the predictions of each of these
analyses.

\section{Conclusions}\label{sec:kusmer:conclusion}


I have shown using experimental methods that the \isi{conjoint}\slash \isi{disjoint} in
\ili{Tshivenḓa} behaves differently from the reported generalizations given for the
parallel alternation in other Southern \ili{Bantu} languages. In particular, while
other Southern \ili{Bantu} languages typically show the \isi{disjoint} and \isi{conjoint} forms
in \isi{complementary distribution}, in \ili{Tshivenḓa} there is a class of syntactic
contexts in which the \isi{disjoint} prefix is apparently optional. Furthermore, I've
shown that the three-way split one see in the \isi{conjoint}\slash \isi{disjoint} alternation
precisely mirrors a similar three-way split in the distribution of penultimate
lengthening. I've proposed two possible analyses that can capture this
parallel: One in which \isi{disjoint} marking is directly determined by the prosody,
and one in which it is indirectly linked to prosody by way of some other common
factor which influences both.

Both analyses presented here make at least one strong language-internal
predictions which I do not yet have the data to test. First, it predicts that
conjoint-form verbs should never be lengthened, regardless of syntactic
context. This prediction remains to be tested.


The prosodic \isi{disjoint analysis} allows for a parsimonious description of the
\ili{Tshivenḓa} \isi{conjoint}\slash \isi{disjoint} facts: Instead of a three-way split based on the
syntax, we can state the generalization in terms of a two-way split based on
the prosody. This analysis seems particularly appropriate for \ili{Tshivenḓa}, in
comparison to the other Southern \ili{Bantu} languages, in that the \isi{disjoint} prefix
is much more limited in distribution in \ili{Tshivenḓa} than elsewhere: The
alternation occurs only in the simple present (/ habitual) tense, and is only
ever between /a-/ and /\o-/, rather than between two contentful morphemes. One
might imagine, then, that the \ili{Tshivenḓa} /a-/ prefix is really just the present
tense morpheme, and that this morpheme undergoes a deletion process in some
contexts. This would help us understand why no /a-/ prefix appears when any
other overt tense morphology is present. More work will be required to
determine if this specific analysis is the correct one.

The structural \isi{disjoint analysis}, by contrast, requires that we understand
dislocated objects, some in situ objects, and some adverbs to form a
natural class, in opposition to secondary predicates. As noted above, the most
likely factor at play here is \isi{information structure}; furthermore, in order to
explain the prosodic facts, we need this factor regardless of which analysis of
the \isi{disjoint} we pursue. If the determining factor is indeed related to
\isi{information structure}, then we predict that dislocated objects will pattern
uniformly in this respect; this is perhaps unsurprising, given that dislocation
itself is an information-structural process related to backgrounding the object
(see \citealt{Buell2005}, among others). We would then predict that \textit{in
situ} objects and adverbs pattern variably with respect to this factor --- that
is, \ili{Tshivenḓa} apparently allows for such elements to be backgrounded without
overt syntactic dislocation, or at least with a short-distance string-vacuous
\isi{movement}. Finally, we predict that secondary predicates will all pattern
uniformly differently from dislocated objects in this respect --- presumably
meaning that they can never be backgrounded or otherwise marked as ``given''.
This is perhaps the most surprising prediction of this analysis, and yet still
seems well within the range of possibility.

Deciding between these two analyses, then, will require considerable further
work. In particular, the studies presented here did not treat information
structure as a factor in any way; it will be essential to control for
this in future studies. Optimally, this would involve both a judgment task and
a production task, each of which carefully controlled the discourse context for
each test item. I leave such a study for future research.


\section*{Abbreviations}
\begin{tabularx}{.6\textwidth}{lQ}
\textsc{dep} & dependent predicate marker\\ 
\textsc{dsj} & {disjoint} prefix\\
\textsc{loc} & locative suffix\\
\textsc{obj} & object\\
\end{tabularx}
\begin{tabularx}{.35\textwidth}{lQ}
\textsc{pst} & past\\
\textsc{sg} & singular\\ 
\textsc{subj} & {subject}\\
\\
\end{tabularx}

\section*{Acknowledgements}

I'd like to thank Professor N.\,C.\ Netshisaulu at the University of Venḓa,
along with his students Abednico Nyoni, Tshivhase N., and Maduwa Besley, and
the entire UniVen Linguistics department, for all their gracious help with my
\ili{Tshivenḓa} research. Further thanks go to Ramafamba Lindelani,
Ratshimbvumo Perseverance, Munzhedzi Fhatuwani, Mukwevho Robert, Mahandana
Mashudu, Netshiavha Fulufhelo, Mudau Precious, and Ndiambani P.T.\ for their
patience with reading silly sentences over and over again while wearing an
uncomfortable microphone. Seunghun J.\ Lee introduced me to everyone at UniVen
and made this project possible in the first place; Professor M.\ Crous Hlungwani
was a great friend to me during my stay there.

This material is based upon work supported by the National Science Foundation
Graduate Research Fellowship under Grant No. 1451512. Any opinion, findings,
and conclusions or recommendations expressed in this material are those of the
author and do not necessarily reflect the views of the National Science
Foundation.



{\sloppy
\printbibliography[heading=subbibliography,notkeyword=this]}

\end{document}
