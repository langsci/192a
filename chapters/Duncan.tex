\documentclass[output=paper,modfonts,nonflat,
% colorlinks, citecolor=brown,
% draftmode
hidelinks
]{langsci/langscibook} 

\IfFileExists{../localpackages.tex}{%
  \setcounter{chapter}{20}
  \usepackage{pifont}
\usepackage{savesym}

\savesymbol{downingtriple}
\savesymbol{downingdouble}
\savesymbol{downingquad}
\savesymbol{downingquint}
\savesymbol{suph}
\savesymbol{supj}
\savesymbol{supw}
\savesymbol{sups}
\savesymbol{ts}
\savesymbol{tS}
\savesymbol{devi}
\savesymbol{devu}
\savesymbol{devy}
\savesymbol{deva}
\savesymbol{N}
\savesymbol{Z}
\savesymbol{circled}
\savesymbol{sem}
\savesymbol{row}
\savesymbol{tipa}
\savesymbol{tableauxcounter}
\savesymbol{tabhead}
\savesymbol{inp}
\savesymbol{inpno}
\savesymbol{g}
\savesymbol{hanl}
\savesymbol{hanr}
\savesymbol{kuku}
\savesymbol{ip}
\savesymbol{lipm}
\savesymbol{ripm}
\savesymbol{lipn}
\savesymbol{ripn} 
% \usepackage{amsmath} 
% \usepackage{multicol}
\usepackage{qtree} 
\usepackage{tikz-qtree,tikz-qtree-compat}
% \usepackage{tikz}
\usepackage{upgreek}


%%%%%%%%%%%%%%%%%%%%%%%%%%%%%%%%%%%%%%%%%%%%%%%%%%%%
%%%                                              %%%
%%%           Examples                           %%%
%%%                                              %%%
%%%%%%%%%%%%%%%%%%%%%%%%%%%%%%%%%%%%%%%%%%%%%%%%%%%%
% remove the percentage signs in the following lines
% if your book makes use of linguistic examples
\usepackage{tipa}  
\usepackage{pstricks,pst-xkey,pst-asr}

%for sande et al
\usepackage{pst-jtree}
\usepackage{pst-node}
%\usepackage{savesym}


% \usepackage{subcaption}
\usepackage{multirow}  
\usepackage{./langsci/styles/langsci-optional} 
\usepackage{./langsci/styles/langsci-lgr} 
\usepackage{./langsci/styles/langsci-glyphs} 
\usepackage[normalem]{ulem}
%% if you want the source line of examples to be in italics, uncomment the following line
% \def\exfont{\it}
\usetikzlibrary{arrows.meta,topaths,trees}
\usepackage[linguistics]{forest}
\forestset{
	fairly nice empty nodes/.style={
		delay={where content={}{shape=coordinate,for parent={
					for children={anchor=north}}}{}}
}}
\usepackage{soul}
\usepackage{arydshln}
% \usepackage{subfloat}
\usepackage{langsci/styles/langsci-gb4e} 
   
% \usepackage{linguex}
\usepackage{vowel}

\usepackage{pifont}% http://ctan.org/pkg/pifont
\newcommand{\cmark}{\ding{51}}%
\newcommand{\xmark}{\ding{55}}%
 
 
 %Lamont
 \makeatletter
\g@addto@macro\@floatboxreset\centering
\makeatother

\usepackage{newfloat} 
\DeclareFloatingEnvironment[fileext=tbx,name=Tableau]{tableau}
  %add all your local new commands to this file
\newcommand{\downingquad}[4]{\parbox{2.5cm}{#1}\parbox{3.5cm}{#2}\parbox{2.5cm}{#3}\parbox{3.5cm}{#4}}
\newcommand{\downingtriple}[3]{\parbox{4.5cm}{#1}\parbox{3cm}{#2}\parbox{3cm}{#3}}
\newcommand{\downingdouble}[2]{\parbox{4.5cm}{#1}\parbox{6cm}{#2}}
\newcommand{\downingquint}[5]{\parbox{1.75cm}{#1}\parbox{2.25cm}{#2}\parbox{2cm}{#3}\parbox{3cm}{#4}\parbox{2cm}{#5}}
\newcolumntype{Y}{>{\centering\arraybackslash}X}
\newcolumntype{T}{>{\centering\arraybackslash}m{2cm}}

%commands for Kusmer paper below
\newcommand{\ip}{$\upiota$}
\newcommand{\lipm}{(\_{\ip-Max}}
\newcommand{\ripm}{)\_{\ip-Max}}
\newcommand{\lipn}{(\_{\ip}}
\newcommand{\ripn}{)\_{\ip}}
\renewcommand{\_}[1]{\textsubscript{#1}}


%commands for Pillion paper below
\newcommand{\suph}{\textipa{\super h}}
\newcommand{\supj}{\textipa{\super j}}
\newcommand{\supw}{\textipa{\super w}}
\newcommand{\ts}{\textipa{\t{ts}}}
\newcommand{\tS}{\textipa{\t{tS}}}
\newcommand{\devi}{\textipa{\r*i}}
\newcommand{\devu}{\textipa{\r*u}}
\newcommand{\devy}{\textipa{\r*y}}
\newcommand{\deva}{\textipa{\r*a}}
\renewcommand{\N}{\textipa{N}}
\newcommand{\Z}{\textipa{Z}}
% 

%commands for Diercks paper below
\newcommand{\circled}[1]{\begin{tikzpicture}[baseline=(word.base)]
\node[draw, rounded corners, text height=8pt, text depth=2pt, inner sep=2pt, outer sep=0pt, use as bounding box] (word) {#1};
\end{tikzpicture}
}

%commands for Pesetsky paper below
% \newcommand{\sem}[2][]{\mbox{$[\![ $\textbf{#2}$ ]\!]^{#1}$}}
\newcommand{\sem}[2][]{\mbox{$[[ $\textbf{#2}$ ]]^{#1}$}}

% \newcommand{\ripn}{{\color{red}ripn}}%this is used but never defined. Please update the definition



%commands for Lamont paper below
\newcommand{\row}[4]{
	#1. & 
    /{#2}/ & 
    [{#3}] & 
    `#4' \\ 
}
%\newcounter{tableauxcounter}
\newcommand{\tabhead}[2]{
%     \captionsetup{labelformat=empty}
%     \stepcounter{tableauxcounter}
%     \addtocounter{table}{-1}
% 	\centering
% 	\caption{Tableau \thetableauxcounter: #1}
	\caption{#1}
	\label{#2}
}
\newcommand{\candref}[2]{{(\ref{#1}#2)}}
\newcommand{\tableauref}[1]{{Tableau~\ref{#1}}}
% tableaux
\newcommand{\inp}[1]{\multicolumn{2}{|l||}{{#1}}}
\newcommand{\inpno}[1]{\multicolumn{2}{|l||}{#1}}
\newcommand{\g}{\cellcolor{lightgray}}
\newcommand{\hanl}{\HandLeft}
\newcommand{\hanr}{\HandRight}
\newcommand{\kuku}{Kuk\'{u}}

% \newcommand{\nocaption}[1]{{\color{red} Please provide a caption}}

% \providecommand{\biberror}[1]{{\color{red}#1}}

\definecolor{RED}{cmyk}{0.05,1,0.8,0}


\newfontfamily\amharicfont[Script = Ethiopic, Scale = 1.0]{AbyssinicaSIL}
\newcommand{\amh}[1]{{\amharicfont #1}}

% 
% %Gjersoe
\usepackage{textgreek}
% 
\newcommand{\viol}{\fontfamily{MinionPro-OsF}\selectfont\rotatebox{60}{$\star$}}
\newcommand{\myscalex}{0.45}
\newcommand{\myscaley}{0.65}
%\newcommand{\red}[1]{\textcolor{red}{#1}}
%\newcommand{\blue}[1]{\textcolor{blue}{#1}}
\newcommand{\epen}[1]{\colorbox{jgray}{#1}}
\newcommand{\hand}{{\normalsize \ding{43}}}
\definecolor{jgray}{gray}{0.8} 
\usetikzlibrary{positioning}
\usetikzlibrary{matrix}
\newcommand{\mora}{\textmu\xspace}
\newcommand{\si}{\textsigma\xspace}
\newcommand{\ft}{\textPhi\xspace}
\newcommand{\tone}{\texttau\xspace}
\newcommand{\word}{\textomega\xspace}
% \newcommand{\ts}{\texttslig}
\newcommand{\fns}{\footnotesize}
\newcommand{\ns}{\normalsize}
\newcommand{\vs}{\vspace{1em}}
\newcommand{\bs}{\textbackslash}   % backslash
\newcommand{\cmd}[1]{{\bf \color{red}#1}}   % highlights command
\newcommand{\scell}[2][l]{\begin{tabular}[#1]{@{}c@{}}#2\end{tabular}}
% \interfootnotelinepenalty=10000

% --- Snider Representations --- %

\newcommand{\RepLevelHh}{
\begin{minipage}{0.10\textwidth}
\begin{tikzpicture}[xscale=\myscalex,yscale=\myscaley]
%\node (syl) at (0,0) {Hi};
\node (Rt) at (0,1) {o};
\node (H) at (-0.5,2) {H};
\node (R) at (0.5,3) {h};
%\draw [thick] (syl.north) -- (Rt.south) ;
\draw [thick] (Rt.north) -- (H.south) ;
\draw [thick] (Rt.north) -- (R.south) ;
\end{tikzpicture}
\end{minipage}
}

\newcommand{\RepLevelLh}{
\begin{minipage}{0.10\textwidth}
\begin{tikzpicture}[xscale=\myscalex,yscale=\myscaley]
%\node (syl) at (0,0) {Mid2};
\node (Rt) at (0,1) {o};
\node (H) at (-0.5,2) {L};
\node (R) at (0.5,3) {h};
%\draw [thick] (syl.north) -- (Rt.south) ;
\draw [thick] (Rt.north) -- (H.south) ;
\draw [thick] (Rt.north) -- (R.south) ;
\end{tikzpicture}
\end{minipage}
}

\newcommand{\RepLevelHl}{
\begin{minipage}{0.10\textwidth}
\begin{tikzpicture}[xscale=\myscalex,yscale=\myscaley]
%\node (syl) at (0,0) {Mid1};
\node (Rt) at (0,1) {o};
\node (H) at (-0.5,2) {H};
\node (R) at (0.5,3) {l};
%\draw [thick] (syl.north) -- (Rt.south) ;
\draw [thick] (Rt.north) -- (H.south) ;
\draw [thick] (Rt.north) -- (R.south) ;
\end{tikzpicture}
\end{minipage}
}

\newcommand{\RepLevelLl}{
\begin{minipage}{0.10\textwidth}
\begin{tikzpicture}[xscale=\myscalex,yscale=\myscaley]
%\node (syl) at (0,0) {Lo};
\node (Rt) at (0,1) {o};
\node (H) at (-0.5,2) {L};
\node (R) at (0.5,3) {l};
%\draw [thick] (syl.north) -- (Rt.south) ;
\draw [thick] (Rt.north) -- (H.south) ;
\draw [thick] (Rt.north) -- (R.south) ;
\end{tikzpicture}
\end{minipage}
}

% --- Representations --- %

\newcommand{\RepLevel}{
\begin{minipage}{0.10\textwidth}
\begin{tikzpicture}[xscale=\myscalex,yscale=\myscaley]
\node (syl) at (0,0) {\textsigma};
\node (Rt) at (0,1) {o};
\node (H) at (-0.5,2) {\texttau};
\node (R) at (0.5,3) {\textrho};
\draw [thick] (syl.north) -- (Rt.south) ;
\draw [thick] (Rt.north) -- (H.south) ;
\draw [thick] (Rt.north) -- (R.south) ;
\end{tikzpicture}
\end{minipage}
}

\newcommand{\RepContour}{
\begin{minipage}{0.10\textwidth}
\begin{tikzpicture}[xscale=\myscalex,yscale=\myscaley]
\node (syl) at (0,0) {\textsigma};
\node (Rt) at (0,1) {o};
\node (H) at (-0.5,2) {\texttau};
\node (R) at (0.5,3) {\textrho};
\node (Rt2) at (1.5,1.0) {o};
%\node (H2) at (1.0,2) {$\tau$};
%\node (R2) at (2.0,2.5) {R};
\draw [thick] (syl.north) -- (Rt.south) ;
\draw [thick] (Rt.north) -- (H.south) ;
\draw [thick] (Rt.north) -- (R.south) ;
\draw [thick] (syl.north) -- (Rt2.south) ;
%\draw [thick] (Rt2.north) -- (H2.south) ;
%\draw [thick] (Rt2.north) -- (R2.south) ;
\end{tikzpicture}
\end{minipage}
}


% --- OT constraints --- %

\newcommand{\IllustrationDown}{
\begin{minipage}{0.09\textwidth}
\begin{tikzpicture}[xscale=0.7,yscale=0.45]
\node (reg) at (0,0.75) {{\small \textalpha}};
\node (arrow) at (0,0) {{\fns $\downarrow$}};
\node (Rt) at (0,-0.75) {{\small \textbeta}};
\end{tikzpicture}
\end{minipage}
}

\newcommand{\IllustrationUp}{
\begin{minipage}{0.09\textwidth}
\begin{tikzpicture}[xscale=0.7,yscale=0.45]
\node (reg) at (0,0.75) {{\small \textalpha}};
\node (arrow) at (0,0) {{\fns $\uparrow$}};
\node (Rt) at (0,-0.75) {{\small \textbeta}};
\end{tikzpicture}
\end{minipage}
}

\newcommand{\MaxAB}{
\begin{minipage}{0.09\textwidth}
\begin{tikzpicture}[xscale=0.6,yscale=0.4]
\node (max) at (0,0) {{\small \textsc{Max}}};
\node (reg) at (0.75,0.5) {{\fns \textalpha}};
\node (arrow) at (0.75,0) {{\tiny $\downarrow$}};
\node (Rt) at (0.75,-0.5) {{\fns \textbeta}};
\end{tikzpicture}
\end{minipage}
}

\newcommand{\DepAB}{
\begin{minipage}{0.09\textwidth}
\begin{tikzpicture}[xscale=0.6,yscale=0.4]
\node (max) at (0,0) {{\small \textsc{Dep}}};
\node (reg) at (0.75,0.5) {{\fns \textalpha}};
\node (arrow) at (0.75,0) {{\tiny $\downarrow$}};
\node (Rt) at (0.75,-0.5) {{\fns \textbeta}};
\end{tikzpicture}
\end{minipage}
}

\newcommand{\DepHReg}{
\begin{minipage}{0.055\textwidth}
\begin{tikzpicture}[xscale=0.6,yscale=0.4]
\node (dep) at (0,0) {{\small \textsc{Dep}}};
\node (reg) at (0,-1.0) {{\small h}};
\end{tikzpicture}
\end{minipage}
}

\newcommand{\DepLReg}{
\begin{minipage}{0.055\textwidth}
\begin{tikzpicture}[xscale=0.6,yscale=0.4]
\node (dep) at (0,0) {{\small \textsc{Dep}}};
\node (reg) at (0,-1.0) {{\small l}};
\end{tikzpicture}
\end{minipage}
}

\newcommand{\DepReg}{
\begin{minipage}{0.055\textwidth}
\begin{tikzpicture}[xscale=0.6,yscale=0.4]
\node (dep) at (0,0) {{\small \textsc{Dep}}};
\node (reg) at (0,-1.0) {{\small \textrho}};
\end{tikzpicture}
\end{minipage}
}

\newcommand{\DepTRt}{
\begin{minipage}{0.1\textwidth}
\begin{tikzpicture}[xscale=0.6,yscale=0.4]
\node (dep) at (0,0) {{\small \textsc{Dep}}};
\node (t) at (0.75,0.5) {{\fns \texttau}};
\node (arrow) at (0.75,0) {{\tiny $\downarrow$}};
\node (Rt) at (0.75,-0.5) {{\fns o}};
\end{tikzpicture}
\end{minipage}
}

\newcommand{\MaxRegRt}{
\begin{minipage}{0.1\textwidth}
\begin{tikzpicture}[xscale=0.6,yscale=0.4]
\node (max) at (0,0) {{\small \textsc{Max}}};
\node (arrow) at (0.75,0) {{\tiny $\downarrow$}};
\node (Rt) at (0.75,-0.5) {{\fns o}};
\node (reg) at (0.75,0.5) {{\fns \textrho}};
\end{tikzpicture}
\end{minipage}
}

\newcommand{\RegToneByRt}{
\begin{minipage}{0.06\textwidth}
\begin{tikzpicture}[xscale=0.6,yscale=0.5]
\node[rotate=20] (arrow1) at (-0.15,0) {{\fns $\uparrow$}};
\node[rotate=340] (arrow2) at (0.15,0) {{\fns $\uparrow$}};
\node (Rt) at (0,-0.55) {{\small o}};
\node (reg) at (0.4,0.55) {{\small \textrho}};
\node (tone) at (-0.4,0.55) {{\small \texttau}};
\end{tikzpicture}
\end{minipage}
}

\newcommand{\RegToneBySyl}{
\begin{minipage}{0.06\textwidth}
\begin{tikzpicture}[xscale=0.6,yscale=0.5]
\node[rotate=20] (arrow1) at (-0.15,0) {{\fns $\uparrow$}};
\node[rotate=340] (arrow2) at (0.15,0) {{\fns $\uparrow$}};
\node (Rt) at (0,-0.55) {{\small \textsigma}};
\node (reg) at (0.4,0.55) {{\small \textrho}};
\node (tone) at (-0.4,0.55) {{\small \texttau}};
\end{tikzpicture}
\end{minipage}
}

\newcommand{\DepTone}{
\begin{minipage}{0.055\textwidth}
\begin{tikzpicture}[xscale=0.6,yscale=0.4]
\node (dep) at (0,0) {{\small \textsc{Dep}}};
\node (tone) at (0,-1.0) {{\small \texttau}};
\end{tikzpicture}
\end{minipage}
}

\newcommand{\DepTonalRt}{
\begin{minipage}{0.055\textwidth}
\begin{tikzpicture}[xscale=0.6,yscale=0.4]
\node (dep) at (0,0) {{\small \textsc{Dep}}};
\node (tone) at (0,-1.0) {{\small o}};
\end{tikzpicture}
\end{minipage}
}

\newcommand{\DepL}{
\begin{minipage}{0.055\textwidth}
\begin{tikzpicture}[xscale=0.6,yscale=0.4]
\node (dep) at (0,0) {{\small \textsc{Dep}}};
\node (tone) at (0,-1.0) {{\small L}};
\end{tikzpicture}
\end{minipage}
}

\newcommand{\DepH}{
\begin{minipage}{0.055\textwidth}
\begin{tikzpicture}[xscale=0.6,yscale=0.4]
\node (dep) at (0,0) {{\small \textsc{Dep}}};
\node (tone) at (0,-1.0) {{\small H}};
\end{tikzpicture}
\end{minipage}
}

\newcommand{\NoMultDiff}{{\small *loh}}
\newcommand{\Alt}{{\small \textsc{Alt}}}
\newcommand{\NoSkip}{{\small \scell{\textsc{No}\\\textsc{Skip}}}}


\newcommand{\RegDomRt}{
\begin{minipage}{0.030\textwidth}
\begin{tikzpicture}[xscale=0.6,yscale=0.5]
\node (arrow) at (0,0) {{\fns $\downarrow$}};
\node (Rt) at (0,-0.55) {{\small o}};
\node (reg) at (0,0.55) {{\small \textrho}};
\end{tikzpicture}
\end{minipage}
}

\newcommand{\DepRegRt}{
\begin{minipage}{0.1\textwidth}
\begin{tikzpicture}[xscale=0.6,yscale=0.4]
\node (dep) at (0,0) {{\small \textsc{Dep}}};
\node (arrow) at (0.75,0) {{\tiny $\downarrow$}};
\node (Rt) at (0.75,-0.5) {{\fns o}};
\node (reg) at (0.75,0.5) {{\fns \textrho}};
\end{tikzpicture}
\end{minipage}
}

% unused

\newcommand{\ToneByRt}{
\begin{minipage}{0.05\textwidth}
\begin{tikzpicture}[xscale=0.6,yscale=0.5]
\node (arrow) at (0,0) {{\fns $\uparrow$}};
\node (Rt) at (0,-0.55) {{\small o}};
\node (tone) at (0,0.55) {{\small \texttau}};
\end{tikzpicture}
\end{minipage}
}

\newcommand{\RegByRt}{
\begin{minipage}{0.05\textwidth}
\begin{tikzpicture}[xscale=0.6,yscale=0.5]
\node (arrow) at (0,0) {{\fns $\uparrow$}};
\node (Rt) at (0,-0.55) {{\small o}};
\node (reg) at (0,0.55) {{\small \textrho}};
\end{tikzpicture}
\end{minipage}
}

\newcommand{\ToneDomRt}{
\begin{minipage}{0.05\textwidth}
\begin{tikzpicture}[xscale=0.6,yscale=0.5]
\node (arrow) at (0,0) {{\fns $\downarrow$}};
\node (Rt) at (0,-0.55) {{\small o}};
\node (tone) at (0,0.55) {{\small \texttau}};
\end{tikzpicture}
\end{minipage}
}

% --- OT tableaus --- %

% Sec. 3.2, first tabl.

\newcommand{\OTHLInput}{
\begin{minipage}{0.17\textwidth}
\begin{tikzpicture}[xscale=\myscalex,yscale=\myscaley]
\node (tone) at (2,0) {(= H)};
\node (syl) at (0,0) {\textsigma};
\node (Rt) at (0,1) {o};
\node (H) at (-0.5,2) {H};
\node (R) at (0.5,3) {h};
\node (Rt2) at (1.5,1.0) {o};
%\node (H2) at (1.0,2) {\epen{L}};
\node (R2) at (2.0,3) {\blue{l}};
\draw [thick] (syl.north) -- (Rt.south) ;
\draw [thick] (Rt.north) -- (H.south) ;
\draw [thick] (Rt.north) -- (R.south) ;
\draw [thick] (syl.north) -- (Rt2.south) ;
%\draw [dashed] (Rt2.north) -- (H2.south) ;
%\draw [dashed] (Rt2.north) -- (R2.south) ;
\end{tikzpicture}
\end{minipage}
}

\newcommand{\OTHLWinner}{
\begin{minipage}{0.17\textwidth}
\begin{tikzpicture}[xscale=\myscalex,yscale=\myscaley]
\node (tone) at (2,0) {(= HL)};
\node (syl) at (0,0) {\textsigma};
\node (Rt) at (0,1) {o};
\node (H) at (-0.5,2) {H};
\node (R) at (0.5,3) {h};
\node (Rt2) at (1.5,1.0) {o};
\node (H2) at (1.0,2) {\epen{L}};
\node (R2) at (2.0,3) {\blue{l}};
\draw [thick] (syl.north) -- (Rt.south) ;
\draw [thick] (Rt.north) -- (H.south) ;
\draw [thick] (Rt.north) -- (R.south) ;
\draw [thick] (syl.north) -- (Rt2.south) ;
\draw [dashed] (Rt2.north) -- (H2.south) ;
\draw [dashed] (Rt2.north) -- (R2.south) ;
\end{tikzpicture}
\end{minipage}
}

\newcommand{\OTHLSpreadingHOnly}{
\begin{minipage}{0.17\textwidth}
\begin{tikzpicture}[xscale=\myscalex,yscale=\myscaley]
\node (tone) at (2,0) {(= HM)};
\node (syl) at (0,0) {\textsigma};
\node (Rt) at (0,1) {o};
\node (H) at (-0.5,2) {H};
\node (R) at (0.5,3) {h};
\node (Rt2) at (1.5,1.0) {o};
%\node (H2) at (1.0,2) {\epen{L}};
\node (R2) at (2.0,3) {\blue{l}};
\draw [thick] (syl.north) -- (Rt.south) ;
\draw [thick] (Rt.north) -- (H.south) ;
\draw [thick] (Rt.north) -- (R.south) ;
\draw [thick] (syl.north) -- (Rt2.south) ;
\draw [dashed] (Rt2.north) -- (R2.south) ;
\draw [dashed] (Rt2.north) -- (H.south) ;
\end{tikzpicture}
\end{minipage}
}

\newcommand{\OTHLInsertH}{
\begin{minipage}{0.17\textwidth}
\begin{tikzpicture}[xscale=\myscalex,yscale=\myscaley]
\node (tone) at (2,0) {(= HM)};
\node (syl) at (0,0) {\textsigma};
\node (Rt) at (0,1) {o};
\node (H) at (-0.5,2) {H};
\node (R) at (0.5,3) {h};
\node (Rt2) at (1.5,1.0) {o};
\node (H2) at (1.0,2) {\epen{H}};
\node (R2) at (2.0,3) {\blue{l}};
\draw [thick] (syl.north) -- (Rt.south) ;
\draw [thick] (Rt.north) -- (H.south) ;
\draw [thick] (Rt.north) -- (R.south) ;
\draw [thick] (syl.north) -- (Rt2.south) ;
\draw [dashed] (Rt2.north) -- (H2.south) ;
\draw [dashed] (Rt2.north) -- (R2.south) ;
\end{tikzpicture}
\end{minipage}
}

\newcommand{\OTHLOverwriting}{
\begin{minipage}{0.17\textwidth}
\begin{tikzpicture}[xscale=\myscalex,yscale=\myscaley]
\node (syl) at (0,0) {\textsigma};
\node (Rt) at (0,1) {o};
\node (H) at (-0.5,2) {H};
\node (R) at (0.5,3) {h};
\node (Rt2) at (1.5,1.0) {o};
%\node (H2) at (1.0,2) {\epen{L}};
\node (R2) at (2.0,3) {\blue{l}};
\draw [thick] (syl.north) -- (Rt.south) ;
\draw [thick] (Rt.north) -- (H.south) ;
\draw [thick] (Rt.north) -- (R.south) ;
\draw [thick] (syl.north) -- (Rt2.south) ;
%\draw [dashed] (Rt2.north) -- (H2.south) ;
\draw [dashed] (Rt.north) -- (R2.south) ;
\node (del) at (0.3,1.9) {\textbf{=}};
\end{tikzpicture}
\end{minipage}
}

\newcommand{\OTHLSpreading}{
\begin{minipage}{0.17\textwidth}
\begin{tikzpicture}[xscale=\myscalex,yscale=\myscaley]
\node (syl) at (0,0) {\textsigma};
\node (Rt) at (0,1) {o};
\node (H) at (-0.5,2) {H};
\node (R) at (0.5,3) {h};
\node (Rt2) at (1.5,1.0) {o};
%\node (H2) at (1.0,2) {\epen{L}};
\node (R2) at (2.0,3) {\blue{l}};
\draw [thick] (syl.north) -- (Rt.south) ;
\draw [thick] (Rt.north) -- (H.south) ;
\draw [thick] (Rt.north) -- (R.south) ;
\draw [thick] (syl.north) -- (Rt2.south) ;
%\draw [dashed] (Rt2.north) -- (H2.south) ;
\draw [dashed] (Rt2.north) -- (H.south) ;
\draw [dashed] (Rt2.north) -- (R.south) ;
\end{tikzpicture}
\end{minipage}
}

% Sec. 4.2, second tabl.: phrase-medial position

\newcommand{\OTHnoLInput}{
\begin{minipage}{0.17\textwidth}
\begin{tikzpicture}[xscale=\myscalex,yscale=\myscaley]
\node (tone) at (2,0) {(= H)};
\node (syl) at (0,0) {\textsigma};
\node (Rt) at (0,1) {o};
\node (H) at (-0.5,2) {H};
\node (R) at (0.5,3) {h};
\node (Rt2) at (1.5,1.0) {o};
%\node (H2) at (1.0,2) {\epen{L}};
%\node (R2) at (2.0,3) {\blue{l}};
\draw [thick] (syl.north) -- (Rt.south) ;
\draw [thick] (Rt.north) -- (H.south) ;
\draw [thick] (Rt.north) -- (R.south) ;
\draw [thick] (syl.north) -- (Rt2.south) ;
\end{tikzpicture}
\end{minipage}
}

\newcommand{\OTHnoLEpenth}{
\begin{minipage}{0.17\textwidth}
\begin{tikzpicture}[xscale=\myscalex,yscale=\myscaley]
\node (tone) at (2,0) {(= HM)};
\node (syl) at (0,0) {\textsigma};
\node (Rt) at (0,1) {o};
\node (H) at (-0.5,2) {H};
\node (R) at (0.5,3) {h};
\node (Rt2) at (1.5,1.0) {o};
\node (H2) at (1.0,2) {\epen{L}};
\node (R2) at (2.0,3) {\epen{h}};
\draw [thick] (syl.north) -- (Rt.south) ;
\draw [thick] (Rt.north) -- (H.south) ;
\draw [thick] (Rt.north) -- (R.south) ;
\draw [thick] (syl.north) -- (Rt2.south) ;
\draw [dashed] (Rt2.north) -- (H2.south) ;
\draw [dashed] (Rt2.north) -- (R2.south) ;
\end{tikzpicture}
\end{minipage}
}

\newcommand{\OTHnoLSpreading}{
\begin{minipage}{0.17\textwidth}
\begin{tikzpicture}[xscale=\myscalex,yscale=\myscaley]
\node (tone) at (2,0) {(= HH)};
\node (syl) at (0,0) {\textsigma};
\node (Rt) at (0,1) {o};
\node (H) at (-0.5,2) {H};
\node (R) at (0.5,3) {h};
\node (Rt2) at (1.5,1.0) {o};
%\node (H2) at (1.0,2) {\epen{L}};
%\node (R2) at (2.0,3) {\blue{l}};
\draw [thick] (syl.north) -- (Rt.south) ;
\draw [thick] (Rt.north) -- (H.south) ;
\draw [thick] (Rt.north) -- (R.south) ;
\draw [thick] (syl.north) -- (Rt2.south) ;
\draw [dashed] (Rt2.north) -- (H.south) ;
\draw [dashed] (Rt2.north) -- (R.south) ;
\end{tikzpicture}
\end{minipage}
}

% Sec. 4.2, third tabl., LM is unaffected by L\%

\newcommand{\OTLMInput}{
\begin{minipage}{0.2\textwidth}
\begin{tikzpicture}[xscale=\myscalex,yscale=\myscaley]
\node (tone) at (2,0) {(= LM)};
\node (syl) at (0,0) {\textsigma};
\node (Rt) at (0,1) {o};
\node (H) at (-0.5,2) {L};
\node (R) at (0.5,3) {l};
\node (Rt2) at (1.5,1.0) {o};
\node (H2) at (1.0,2) {L};
\node (R2) at (2.0,3) {h};
\node (R3) at (3.0,3) {\blue{l}};
\draw [thick] (syl.north) -- (Rt.south) ;
\draw [thick] (Rt.north) -- (H.south) ;
\draw [thick] (Rt.north) -- (R.south) ;
\draw [thick] (syl.north) -- (Rt2.south) ;
\draw [thick] (Rt2.north) -- (H2.south) ;
\draw [thick] (Rt2.north) -- (R2.south) ;
\end{tikzpicture}
\end{minipage}
}

\newcommand{\OTLMReplace}{
\begin{minipage}{0.2\textwidth}
\begin{tikzpicture}[xscale=\myscalex,yscale=\myscaley]
\node (tone) at (2,0) {(= LL)};
\node (syl) at (0,0) {\textsigma};
\node (Rt) at (0,1) {o};
\node (H) at (-0.5,2) {L};
\node (R) at (0.5,3) {l};
\node (Rt2) at (1.5,1.0) {o};
\node (H2) at (1.0,2) {L};
\node (R2) at (2.0,3) {h};
\node (R3) at (3.0,3) {\blue{l}};
\draw [thick] (syl.north) -- (Rt.south) ;
\draw [thick] (Rt.north) -- (H.south) ;
\draw [thick] (Rt.north) -- (R.south) ;
\draw [thick] (syl.north) -- (Rt2.south) ;
\draw [thick] (Rt2.north) -- (H2.south) ;
\draw [thick] (Rt2.north) -- (R2.south) ;
\draw [dashed] (Rt2.north) -- (R3.south) ;
\node (del) at (1.8,2.1) {\textbf{=}};
\end{tikzpicture}
\end{minipage}
}

\newcommand{\OTLMTwoReg}{
\begin{minipage}{0.2\textwidth}
\begin{tikzpicture}[xscale=\myscalex,yscale=\myscaley]
\node (tone) at (2,0) {(= LML)};
\node (syl) at (0,0) {\textsigma};
\node (Rt) at (0,1) {o};
\node (H) at (-0.5,2) {L};
\node (R) at (0.5,3) {l};
\node (Rt2) at (1.5,1.0) {o};
\node (H2) at (1.0,2) {L};
\node (R2) at (2.0,3) {h};
\node (R3) at (3.0,3) {\blue{l}};
\draw [thick] (syl.north) -- (Rt.south) ;
\draw [thick] (Rt.north) -- (H.south) ;
\draw [thick] (Rt.north) -- (R.south) ;
\draw [thick] (syl.north) -- (Rt2.south) ;
\draw [thick] (Rt2.north) -- (H2.south) ;
\draw [thick] (Rt2.north) -- (R2.south) ;
\draw [dashed] (Rt2.north) -- (R3.south) ;
\end{tikzpicture}
\end{minipage}
}

% Sec. 4.2, fourth tabl., L is affected by L\% but M is not

\newcommand{\OTLInput}{
\begin{minipage}{0.17\textwidth}
\begin{tikzpicture}[xscale=\myscalex,yscale=\myscaley]
\node (tone) at (2,0) {(= L)};
\node (syl) at (0,0) {\textsigma};
\node (Rt) at (0,1) {o};
\node (H) at (-0.5,2) {L};
\node (R) at (0.5,3) {l};
\node (R2) at (2,3) {\blue{l}};
\draw [thick] (syl.north) -- (Rt.south) ;
\draw [thick] (Rt.north) -- (H.south) ;
\draw [thick] (Rt.north) -- (R.south) ;
\end{tikzpicture}
\end{minipage}
}

\newcommand{\OTLLowered}{
\begin{minipage}{0.17\textwidth}
\begin{tikzpicture}[xscale=\myscalex,yscale=\myscaley]
\node (tone) at (2,0) {(= LL)};
\node (syl) at (0,0) {\textsigma};
\node (Rt) at (0,1) {o};
\node (H) at (-0.5,2) {L};
\node (R) at (0.5,3) {l};
\node (R2) at (2,3) {\blue{l}};
\draw [thick] (syl.north) -- (Rt.south) ;
\draw [thick] (Rt.north) -- (H.south) ;
\draw [thick] (Rt.north) -- (R.south) ;
\draw [dashed] (Rt.north) -- (R2.south) ;
\end{tikzpicture}
\end{minipage}
}

\newcommand{\OTMInput}{
\begin{minipage}{0.17\textwidth}
\begin{tikzpicture}[xscale=\myscalex,yscale=\myscaley]
\node (tone) at (2,0) {(= M)};
\node (syl) at (0,0) {\textsigma};
\node (Rt) at (0,1) {o};
\node (H) at (-0.5,2) {L};
\node (R) at (0.5,3) {h};
\node (R2) at (2,3) {\blue{l}};
\draw [thick] (syl.north) -- (Rt.south) ;
\draw [thick] (Rt.north) -- (H.south) ;
\draw [thick] (Rt.north) -- (R.south) ;
\end{tikzpicture}
\end{minipage}
}

\newcommand{\OTMLowered}{
\begin{minipage}{0.17\textwidth}
\begin{tikzpicture}[xscale=\myscalex,yscale=\myscaley]
\node (tone) at (2,0) {(= ML)};
\node (syl) at (0,0) {\textsigma};
\node (Rt) at (0,1) {o};
\node (H) at (-0.5,2) {L};
\node (R) at (0.5,3) {h};
\node (R2) at (2,3) {\blue{l}};
\draw [thick] (syl.north) -- (Rt.south) ;
\draw [thick] (Rt.north) -- (H.south) ;
\draw [thick] (Rt.north) -- (R.south) ;
\draw [dashed] (Rt.north) -- (R2.south) ;
\end{tikzpicture}
\end{minipage}
}

% Sec. 4.2, fifth tableau, polar questions with level tones

\newcommand{\OTLPolIn}{
\begin{minipage}{0.20\textwidth}
\begin{tikzpicture}[xscale=\myscalex-0.05,yscale=\myscaley-0.05]
\node (tone) at (3.5,0) {(= L)};
\node (syl) at (0,0) {\textsigma};
\node (syl2) at (2,0) {\red{\textsigma}};
\node (Rt) at (0,1) {o};
\node (H) at (-0.5,2) {L};
\node (R) at (0.5,3) {l};
\node (Rt2) at (2,1) {\red{o}};
\draw [thick] (syl.north) -- (Rt.south) ;
\draw [thick,red] (syl2.north) -- (Rt2.south) ;
\draw [thick] (Rt.north) -- (H.south) ;
\draw [thick] (Rt.north) -- (R.south) ;
\end{tikzpicture}
\end{minipage}
}

\newcommand{\OTLPolDef}{
\begin{minipage}{0.20\textwidth}
\begin{tikzpicture}[xscale=\myscalex-0.05,yscale=\myscaley-0.05]
\node (tone) at (3.5,0) {(= L.M)};
\node (syl) at (0,0) {\textsigma};
\node (syl2) at (2,0) {\red{\textsigma}};
\node (Rt) at (0,1) {o};
\node (H) at (-0.5,2) {L};
\node (R) at (0.5,3) {l};
\node (H2) at (1.5,2) {\epen{L}};
\node (R2) at (2.5,3) {\epen{h}};
\node (Rt2) at (2,1) {\red{o}};
\draw [thick] (syl.north) -- (Rt.south) ;
\draw [thick,red] (syl2.north) -- (Rt2.south) ;
\draw [thick] (Rt.north) -- (H.south) ;
\draw [thick] (Rt.north) -- (R.south) ;
\draw [semithick,dashed] (Rt2.north) -- (H2.south) ;
\draw [semithick,dashed] (Rt2.north) -- (R2.south) ;
\end{tikzpicture}
\end{minipage}
}

\newcommand{\OTLPolAlt}{
\begin{minipage}{0.20\textwidth}
\begin{tikzpicture}[xscale=\myscalex-0.05,yscale=\myscaley-0.05]
\node (tone) at (3.5,0) {(= L.L)};
\node (syl) at (0,0) {\textsigma};
\node (syl2) at (2,0) {\red{\textsigma}};
\node (Rt) at (0,1) {o};
\node (H) at (-0.5,2) {L};
\node (R) at (0.5,3) {l};
\node (Rt2) at (2,1) {\red{o}};
\draw [thick] (syl.north) -- (Rt.south) ;
\draw [thick,red] (syl2.north) -- (Rt2.south) ;
\draw [thick] (Rt.north) -- (H.south) ;
\draw [thick] (Rt.north) -- (R.south) ;
\draw [semithick,dashed] (Rt2.north) -- (H.south) ;
\draw [semithick,dashed] (Rt2.north) -- (R.south) ;
\end{tikzpicture}
\end{minipage}
}

% Sec. 4.2, sixth tableau, polar questions with contour tones

\newcommand{\OTLLPolIn}{
\begin{minipage}{0.23\textwidth}
\begin{tikzpicture}[xscale=\myscalex-0.05,yscale=\myscaley-0.05]
\node (tone) at (5.2,0) {(= L)};
\node (syl) at (0,0) {\textsigma};
\node (syl3) at (3.4,0) {\red{\textsigma}};
\node (Rt) at (0,1) {o};
\node (Rt2) at (1.7,1) {o};
\node (Rt3) at (3.4,1) {\red{o}};
\node (H) at (-0.5,2) {L};
\node (R) at (0.5,3) {l};
\draw [thick] (syl.north) -- (Rt.south) ;
\draw [thick] (syl.north) -- (Rt2.south) ;
\draw [thick,red] (syl3.north) -- (Rt3.south) ;
\draw [thick] (Rt.north) -- (H.south) ;
\draw [thick] (Rt.north) -- (R.south) ;
\end{tikzpicture}
\end{minipage}
}

\newcommand{\OTLLPolDef}{
\begin{minipage}{0.23\textwidth}
\begin{tikzpicture}[xscale=\myscalex-0.05,yscale=\myscaley-0.05]
\node (tone) at (5.2,0) {(= L.M)};
\node (syl) at (0,0) {\textsigma};
\node (syl3) at (3.4,0) {\red{\textsigma}};
\node (Rt) at (0,1) {o};
\node (Rt2) at (1.7,1) {o};
\node (Rt3) at (3.4,1) {\red{o}};
\node (H) at (-0.5,2) {L};
\node (R) at (0.5,3) {l};
\node (H3) at (2.9,2) {\epen{L}};
\node (R3) at (3.9,3) {\epen{h}};
\draw [thick] (syl.north) -- (Rt.south) ;
\draw [thick] (syl.north) -- (Rt2.south) ;
\draw [thick,red] (syl3.north) -- (Rt3.south) ;
\draw [thick] (Rt.north) -- (H.south) ;
\draw [thick] (Rt.north) -- (R.south) ;
\draw [dashed] (Rt3.north) -- (H3.south) ;
\draw [dashed] (Rt3.north) -- (R3.south) ;
\end{tikzpicture}
\end{minipage}
}

\newcommand{\OTLLPolSkip}{
\begin{minipage}{0.23\textwidth}
\begin{tikzpicture}[xscale=\myscalex-0.05,yscale=\myscaley-0.05]
\node (tone) at (5.2,0) {(= L.L)};
\node (syl) at (0,0) {\textsigma};
\node (syl3) at (3.4,0) {\red{\textsigma}};
\node (Rt) at (0,1) {o};
\node (Rt2) at (1.7,1) {o};
\node (Rt3) at (3.4,1) {\red{o}};
\node (H) at (-0.5,2) {L};
\node (R) at (0.5,3) {l};
\draw [thick] (syl.north) -- (Rt.south) ;
\draw [thick] (syl.north) -- (Rt2.south) ;
\draw [thick,red] (syl3.north) -- (Rt3.south) ;
\draw [thick] (Rt.north) -- (H.south) ;
\draw [thick] (Rt.north) -- (R.south) ;
\draw [dashed] (Rt3.north) -- (H.south) ;
\draw [dashed] (Rt3.north) -- (R.south) ;
\end{tikzpicture}
\end{minipage}
}  
  
\newcommand{\ilit}[1]{#1\il{#1}}    
\newcommand{\isit}[1]{#1\is{#1}}  

\makeatletter
\let\thetitle\@title
\let\theauthor\@author 
\makeatother

\newcommand{\togglepaper}[1][0]{ 
  \bibliography{../localbibliography}
  %% hyphenation points for line breaks
%% Normally, automatic hyphenation in LaTeX is very good
%% If a word is mis-hyphenated, add it to this file
%%
%% add information to TeX file before \begin{document} with:
%% %% hyphenation points for line breaks
%% Normally, automatic hyphenation in LaTeX is very good
%% If a word is mis-hyphenated, add it to this file
%%
%% add information to TeX file before \begin{document} with:
%% \include{localhyphenation}
\hyphenation{
affri-ca-te
affri-ca-tes
com-ple-ments
par-a-digm
Sha-ron
Kings-ton
phe-nom-e-non
Daul-ton
Abu-ba-ka-ri
Ngo-nya-ni
Clem-ents 
King-ston
Tru-cken-brodt
Tab-leau
cophono-logies
mark-edness
Ti-gri-nya
a-mong
Car-stens
Lu-bu-ku-su
}
\hyphenation{
affri-ca-te
affri-ca-tes
com-ple-ments
par-a-digm
Sha-ron
Kings-ton
phe-nom-e-non
Daul-ton
Abu-ba-ka-ri
Ngo-nya-ni
Clem-ents 
King-ston
Tru-cken-brodt
Tab-leau
cophono-logies
mark-edness
Ti-gri-nya
a-mong
Car-stens
Lu-bu-ku-su
}
  \papernote{\scriptsize\normalfont
    \theauthor.
    \thetitle. 
    To appear in: 
    Emily Clem,   Peter Jenks \& Hannah Sande.
    Theory and description in African Linguistics: Selected papers from the 47th Annual Conference on African Linguistics.
    Berlin: Language Science Press. [preliminary page numbering]
  }
  \pagenumbering{roman}
  \setcounter{chapter}{#1}
  \addtocounter{chapter}{-1}
}

\newcommand{\upstep}{\textupstep}


% \newcounter{tableauxcounter}

\renewcommand{\textltailn}{ɲ}
\renewcommand{\textbardotlessj}{ɟ}

\newcommand{\emphkh}[1]{\textit{#1}} %originally \textbf, banned by the guidelines



\definecolor{lsDOIGray}{cmyk}{0,0,0,0.45}


\newcommand{\xuparrow}[1]{%
  {\left\uparrow\vbox to #1{}\right.\kern-\nulldelimiterspace}
}
\renewcommand \textupstep[1]{\char"A71B#1}
\renewcommand \textdownstep[1]{\char"A71C#1}
 
 \newcommand{\ꜛ}{\textsf{ꜛ}}
 
\def\biberror{\undefined}


\newcommand{\OTbox}[1]{\resizebox{.88\textwidth}{!}{#1}}

  \bibliography{../localbibliography}
  \papernote{\scriptsize\normalfont
  Philip T. Duncan, Travis Major  \&  Mfon Udoinyang.
  Verb and predicate coordination in Ibibio. 
  To appear in: 
  Emily Clem,   Peter Jenks \& Hannah Sande.
  Theory and description in African Linguistics: Selected papers from the 47th Annual Conference on African Linguistics.
  Berlin: Language Science Press. [preliminary page numbering]
  }
  \pagenumbering{roman}
}{}
 

 


\author{Philip T. Duncan\affiliation{University of Kansas}\and Travis Major\affiliation{University of California, Los Angeles}\lastand Mfon Udoinyang\affiliation{University of Kansas}}
\title{Verb and predicate coordination in Ibibio} 
%\epigram{Change epigram in chapters/01.tex or remove it there }
\abstract{This paper reports on the `and'-word \textit{ny\'ʌŋ} in Ibibio verbal coordination. Like English \textit{and}, Ibibio possesses morphologically invariant coordinators linking NPs, PPs, and  CPs. However, these cannot coordinate verbs and predicates, unlike \textit{and} in English. Many African languages distinguish between nominal and verbal coordinators \citep[305]{welmers1973african}, but Ibibio showcases this distinction in a unique way. Subject agreement and inflection for tense and negation suggest that \textit{ny\'{ʌ}ŋ} is a verb, resembling ```and'-verbs'' in Walman \citep{brown2008verbs}. Closer inspection reveals that \textit{ny\'{ʌ}ŋ} patterns more like an adverb or functional head, expanding our understanding of what constitutes `and' cross-linguistically.}

\begin{document}
\maketitle

\section{Introduction}\label{sec:duncan-et-al:1} 
Across African and Niger-Congo languages, juxtaposition serves as a general strategy for coordinating clausal units \citep{zeller2015syntax,creissels2000typology,watters2000syntax}. African languages also commonly feature a distinction in coordinators triggered by categorial features of the conjuncts. Such distinction can be seen, for example, in \ili{Dagbani}, where \textit{mini} exclusively conjoins nominal expressions, and \textit{ka} is obligatory for coordinating verbal predicates and clauses. 

\protectedex{
\ea\label{ex:duncan-et-al:1}
\langinfo{Dagbani}{Gur}{Niger-Congo}\\
\ea[]{
\gll doo ŋ \textopeno~ \textbf{mini} m ba ch\textepsilon ni daa \\
     man this and my father go.{\IPFV} market \\
\glt `This man and my father go to the market.' \citep[44]{olawsky1999aspects}
}
\ex[]{
\gll o bi\textepsilon i \textbf{ka} k\textopeno \textbabygamma isi \textbf{ka} da\textbabygamma i \\
he be.bad and be.thin and be.dirty \\
\glt `He is bad and thin and dirty.' \citep[44]{olawsky1999aspects}
}
\ex[]{
\gll m ba wumdi da\textg banli \textbf{ka} tuz\textopeno hi wumdi silimiinsili \\
my father hear.{\IPFV} \ili{Dagbani} and brothers hear.{\IPFV} \ili{English} \\
\glt `My father knows \ili{Dagbani} and my brothers know \ili{English}.' \citep[51]{olawsky1999aspects}
}
\z
\z
}

\noindent \ili{Ibibio}, a Lower Cross Niger-Congo language spoken in Akwa Ibom State, Nigeria likewise showcases this division, but with an unexpected twist: the language recruits an unlikely candidate for verb and \isi{predicate coordination}, one that we show has verb- and adverb-like properties. 

\ili{Ibibio} uses an array of equivalent coordinators for NP/DP \isi{coordination}.\footnote{\citet[147]{essien1990grammar} treats these three coordinators as ``dialectal variants.''}$^,$\footnote{Unless otherwise noted, our Ibiibio data are from Mfon Udoinyang and reflect his judgments.}

\ea\label{ex:duncan-et-al:2}
\gll \'{E}kpê \textbf{y\`{e}/ǹd\`{o}/\`{m}m\`{e}} \`{A}kpán \`{e}-mà é-ŋ w\'ɔŋ~ úk\'ɔtńs\`{ʌ}ŋ. \\
Ekpe and Akpan 3{\PL}-{\PST} 3{\PL}-drink palmwine \\
\glt ‘Ekpe and Akpan drank palmwine.’
\z

\noindent These are, however, illicit when coordinating verbs and larger verbal constructions. Instead, \textit{ny\'{ʌ}ŋ} is used, which surfaces to the left of the main verb in the second conjunct.

\ea\label{ex:duncan-et-al:3}
\ea[]{
\gll \`{A}-mà à-díá àdésì \textbf{à-ny\`{ʌ}ŋ}/*y\`{e}/*ǹd\`{o}/*\`{m}m\`{e} à-ŋ w\'ɔŋ~ úk\'ɔtńs\`{ʌ}ŋ. \\
2{\SG}-{\PST} 2{\SG}-eat rice 2{\SG}-and 2{\SG}-drink palmwine \\
\glt `You ate rice and drank palmwine.'
}
\ex[]{
\gll \`{I}má á-kpón \textbf{á-ny\'{ʌ}ŋ}/*y\`{e}/*ǹd\`{o}/*\`{m}m\`{e} á-yáíyá. \\
Ima 3{\SG}-become.big 3{\SG}-and 3{\SG}-be.beautiful \\
\glt `Ima grew up and became beautiful.'
}
\z
\z

\noindent Cross-linguistically, `and'-words are typically not verbs, though they can be in some languages (e.g., \ili{Walman}; see \citealt{brown2008verbs}). One puzzling \isi{aspect} of \ili{Ibibio} verb and \isi{predicate coordination}, then, is the fact that the overt element that signals coordinate status bears person and number agreement, which is a property of verbs and other elements that comprise the clausal spine across the verbal and inflectional domains \citep{baker2010agreement}.

Our aim in this paper is to investigate distributional evidence for \textit{ny\'{ʌ}ŋ} in order to approach an understanding of its status in \ili{Ibibio}, and provide a foundation for further investigation of the structure(s) of \textit{ny\'{ʌ}ŋ} clauses. To clarify what \textit{ny\'{ʌ}ŋ} might be---and what it is not---we compare it with similar constructions involving verbs (e.g., serial verbs) and low adverbs. Traditionally in \ili{Ibibio} literature \citep{essien1985negation,essien1990grammar}, as well as in closely-related \ili{Efik} \citep{goldie1857principles,welmers1968efik,welmers1973african},\footnote{While \textit{ny\'{ʌ}ŋ} in \ili{Ibibio} and \ili{Efik} resemble each other morphosyntactically, there are important differences. For example, \ili{Efik} \textit{ny\'{ʌ}ŋ} cannot take the negative suffix, unlike Ibibio (see \sectref{sec:duncan-et-al:2.3}).} \textit{ny\'{ʌ}ŋ} has been analyzed as a coordinator itself (a conjunction) that is ``verbal grammatically and conjunctive in function'' \citep[148]{essien1990grammar}. Our work shows, though, that it is not entirely verbal. Moreover, it may not actually be the coordinator, but some third thing that surfaces in verbal \isi{coordination}. The data we present suggests that \textit{ny\'{ʌ}ŋ} inhabits a liminal space somewhere at or near the border of the inflectional and verbal layers. Current evidence seems to tip the balance toward an adverb-style analysis.

\section{Is \textit{ny\'{ʌ}ŋ} a serial verb?}\label{sec:duncan-et-al:2}

The verbal coordinator \textit{ny\'{ʌ}ŋ} bears person and number features. Other possible inflectional marking on \textit{ny\'{ʌ}ŋ} includes tense and \isi{negation} \citep{essien1985negation,essien1990grammar}. Moreover, \textit{ny\'{ʌ}ŋ} in many cases appears flanked by verbs, making it look (on the surface) like one verb in a series.

\ea\label{ex:duncan-et-al:4}
%\langinfo{Ibibio}{Niger-Congo}{personal knowledge}\\
\gll \'{I}nêm á-m\v{a}-k\`{o}p \textbf{á-ny\'{ʌ}ŋ} á-dí. \\
Inem 3{\SG-\PST}-hear 3{\SG}-and 3{\SG}-come \\
\glt ‘Inem heard it and came.’ \citep[86]{essien1985negation}
\z


\noindent Because of these properties, \citet[86]{essien1985negation} (and \citealt[142]{essien1990grammar}) treats \textit{ny\'{ʌ}ŋ} as a V in a V$_1$V$_n$ sequence, calling it a ``serial construction.''

However, \ili{Ibibio} \textit{ny\'{ʌ}ŋ} clauses do not exhibit features that have shown to be characteristically associated with seriality in the language \citep{major2015serial,duncan2016parallel}. In what follows, we consider \textit{ny\'{ʌ}ŋ} in light of the following properties of \isi{serial verbs} in \ili{Ibibio}, which we take as tests of seriality: (a) single tense marking, (b) obligatory \isi{subject} sharing, (c) availability of contrastive \isi{verb focus}, (d) single \isi{negation}, and (e) object sharing. 

\subsection{Single tense test}\label{sec:duncan-et-al:2.1}

\citet{collins1997argument} and \citet{hiraiwa2008object} argue that \isi{serial verb} constructions (SVCs) maximally contain a single \isi{tense marker}. This property obtains for true SVCs in \ili{Ibibio} \citep{major2015serial}.

\ea\label{ex:duncan-et-al:5}
\ea[]{
\gll \'{E}kpê \textbf{á-mà} á-dí (*\textbf{á-mà}) í-sé úf\^ɔk \`{m}mì.\\
Ekpe 3{\SG-\PST} 3{\SG}-come 3{\SG-\PST} \textsc{i}-see house 1{\SG-\POSS} \\
\glt `Ekpe came and saw my house.'
}
\ex[]{
\gll \'{I}nêm \textbf{á-mà} á-k\`{o}p \textbf{á-mà} \textbf{á-ny\'{ʌ}ŋ} á-dí.\\
Inem 3{\SG-\PST} 3{\SG}-hear 3{\SG-\PST} 3{\SG}-and 3{\SG}-come \\
\glt `Inem heard it and came.'
}
\z
\z

\noindent The SVC in (\ref{ex:duncan-et-al:5}a) is thus ungrammatical if the second \isi{tense marker} is added. \textit{Ny\'{ʌ}ŋ} clauses, though, may contain more than one \isi{tense marker}, depending on the number of conjuncts involved. In (\ref{ex:duncan-et-al:5}b), the past \isi{tense marker} \textit{mà} appears twice, once in the first conjunct and once in the second.

Related to this, verbs in \ili{Ibibio} SVCs obligatorily share a single \isi{subject}. Again, though, we find that this is not the case for \textit{ny\'{ʌ}ŋ} clauses.

\ea\label{ex:duncan-et-al:6}
\ea[*]{
\gll \`{O}k\^{o}n á-mà á-dùw\'ɔ \`{A}kpán á-d\'{ʌ}k àdùb\`{e}. \\ 
Okon 3{\SG-\PST} 3{\SG}-fall Akpan 3{\SG}-enter pit \\
\glt (Intended: `Okon fell (and) Akpan entered a pit.')
}
\ex[]{
\gll \`{E}n\`ɔ á-mà á-ká store \textbf{á-ny\'{ʌ}ŋ} \'{I}má á-mà á-dép ǹw\`{e}t.\\
Eno 3{\SG-\PST} 3{\SG}-go store 3{\SG}-and Ima 3{\SG-\PST} 3{\SG}-buy book \\
\glt `Eno went to the store and Ima bought a book.'
}
\z
\z

\noindent Subject restrictions in \ili{Ibibio} SVCs follow from the existence of a single TP layer in such constructions. The absence of this restriction in \textit{ny\'{ʌ}ŋ} clauses corresponds to the presence of a TP in each clausal conjunct.

\subsection{Contrastive focus test}\label{sec:duncan-et-al:2.2}

A second difference between SVCs and \textit{ny\'{ʌ}ŋ} clauses in \ili{Ibibio} pertains to the (un)availability of contrastive \isi{verb focus}. In \ili{Ibibio}, any (or all) verbs in an SVC can potentially undergo contrastive \isi{verb focus}.

\ea\label{ex:duncan-et-al:7}
\ea[]{
\gll \`{O}k\^{o}n á-mà á-t\`{e}m ńdídíyá á-nyàm.\\
Okon 3{\SG-\PST} 3{\SG}-cook food 3{\SG}-sell \\
\glt `Okon cooked food and sold it.'
}
\ex[]{
\gll \`{O}k\^{o}n á-mà á-\textbf{t\`{e}é-t\`{e}m} ńdídíyá á-nyàm\ldots\\
Okon 3{\SG-\PST} 3{\SG}-cook-cook food 3{\SG}-sell \\
\glt `Okon COOKED food and sold it\ldots'
}
\ex[]{
\gll \`{O}k\^{o}n á-mà á-t\`{e}m ńdídíyá á-\textbf{nyàá-ny\^{a}m}\ldots\\
Okon 3{\SG-\PST} 3{\SG}-cook food 3{\SG}-sell-sell \\
\glt `Okon cooked food and SOLD it\ldots'
}
\ex[]{
\gll \`{O}k\^{o}n á-mà á-\textbf{t\`{e}é-t\`{e}m} ńdídíyá á-\textbf{nyàá-ny\^{a}m}\ldots\\
Okon 3{\SG-\PST} 3{\SG}-cook-cook food 3{\SG}-sell-sell \\
\glt `Okon COOKED food and SOLD it\ldots'
}
\z
\z

\noindent Given the existence of a low \isi{focus} phrase near the verbal domain in \ili{Ibibio} \citep{duncan-toappear}, \citet{duncan2016parallel} proposes that the fact that any V in a V$_1$V$_n$ sequence can be contrastively focused follows from the \textit{v}P-internal nature of low FocP. Since SVCs contain at minimum two \textit{v}Ps, iterated FocPs are an outcome of iterated \textit{v}Ps \citep[98-100]{duncan2016parallel}. 

Interestingly, the verbal coordinator \textit{ny\'{ʌ}ŋ} cannot participate in contrastive \isi{verb focus}.\footnote{An audience member at ACAL 45 raised the question as to the intended meaning of contrastively focused \textit{ny\'{ʌ}ŋ} in the first place. We acknowledge that the meaning could be complicated, but presented the form as a diagnostic in the event that it were possible. (If, for example, \textit{ny\'{ʌ}ŋ} were a verb with a meaning like `do in addition to' then, potentially, a contrastive \isi{focus} reading might emphasize the nature of the event in relation to another.) Regardless, we are unaware of any semantic constraints on verbs that bar them from participation in contrastive \isi{verb focus}.}$^,$\footnote{For an overview of the formal features of \ili{Ibibio} contrastive \isi{verb focus} and its effects on \isi{vowel quality}, see \citet{akinlabi2003} and \citet{duncan-toappear}.}


\ea[*]{
%\langinfo{Ibibio}{Niger-Congo}{personal knowledge}\\
\gll \'{I}má á-kpón \textbf{á-ny\`ɔ\'ɔ-ny\^{ʌ}ŋ} á-yàìyá. \\
Ima 3{\SG}-become.big 3{\SG}-and-and 3{\SG}-be.beautiful \\
\glt (Intended: `Ima became big AND beautiful.')
} \label{ex:duncan-et-al:8}
\z

\noindent Again, this suggests that \textit{ny\'{ʌ}ŋ} clauses are not exactly SVCs. What makes contrastively focusing \textit{ny\'{ʌ}ŋ} impossible is not, however, due to the number of \textit{v}Ps present. Presumably, there are two \textit{v}Ps in (\ref{ex:duncan-et-al:8}), as there are two \textit{v}Ps in each on the sentences in (\ref{ex:duncan-et-al:7}). Instead, we posit that the site of attachment for \textit{ny\'{ʌ}ŋ} drives its inability to participate in contrastive \isi{verb focus}. That is, the attachment site of \textit{ny\'{ʌ}ŋ} is \textit{v}P-external.

\subsection{Single negation test}\label{sec:duncan-et-al:2.3}

Cross-linguistically, SVCs commonly allow for only one instance of \isi{negation} \citep{hiraiwa2008object}, and this holds for \ili{Ibibio}, as well. In \ili{Ibibio}, \isi{negation} scopes over V$_1$ and V$_2$, but only V$_1$ gets negated \citep{major2015serial}.\footnote{The negative suffix in \ili{Ibibio} has several allomorphs. See \citet[124-127]{akinlabi2003} and \citet[89]{duncan2016parallel} for discussion.}

\ea\label{ex:duncan-et-al:9}
\ea[]{
\gll \`{E}n\`ɔ í-ké í-dàká\textbf{-ké} í-dá. \\
Eno \textsc{i}-{\PST.\FOC} \textsc{i}-rise-\textsc{neg} \textsc{i}-stand \\
\glt `Eno didn't arise.'
}
\ex[*]{
\gll \`{E}n\`ɔ á-mà/í-ké á-/í-dàká í-dá\textbf{-há}.\\
Eno 3{\SG-\PST}/\textsc{i}-{\PST.\FOC} 3{\SG}/\textsc{i}-rise \textsc{i}-stand-\textsc{neg} \\
\glt (Intended: `Eno didn't arise.')
}
\ex[*]{
\gll \`{E}n\`ɔ í-ké í-dàká\textbf{-ké} í-dá\textbf{-há}. \\
Eno \textsc{i}-{\PST.\FOC} \textsc{i}-rise-\textsc{neg} \textsc{i}-stand-\textsc{neg} \\
\glt (Intended: `Eno didn't arise.')
}
\z
\z

\noindent The SVC meaning `arise' is comprised of the verbs `rise' and `stand'. As seen in (\ref{ex:duncan-et-al:9}a), when this construction is negated, only V$_1$ bears the negative suffix, meaning that only the highest verb in the sequence raises to Neg$^0$ \citep{duncan-toappear}, possibly as it travels en route to T$^0$.\footnote{\citet[120]{baker2010agreement} claim that ``the verb moves to T in \ili{Ibibio} and thus surfaces to the left of \isi{negation}.'' While we remain agnostic as to whether raising-to-T is a regular feature of \ili{Ibibio} grammar, for our purposes, either analysis successfully accounts for the distributional facts in \REF{ex:duncan-et-al:9}.} Thus, neither the lower verb can be negated, nor can both verbs be negated simultaneously. 

From this, one straightforward prediction is that, if \textit{ny\'{ʌ}ŋ} clauses are true SVCs, \textit{ny\'{ʌ}ŋ} should be non-negatable, given that on the surface it follows V$_1$ in the matrix clause. However, this is not the case.

\ea\label{ex:duncan-et-al:10}
%\langinfo{Ibibio}{Niger-Congo}{personal knowledge}\\
\gll \'{I}nêm í-kí-k\`{o}p-pó \textbf{í-ny\'{ʌ}ŋ-ŋ \'ɔ} í-dí. \\
Inem \textsc{i}-{\PST.\FOC}.\textsc{i}-hear-\textsc{neg} \textsc{i}-and-\textsc{neg} \textsc{i}-come \\
\glt ‘Inem did not hear it and did not come.’ \citep[86]{essien1985negation}
\z

\noindent Like the \isi{serial verbs} above, \textit{ny\'{ʌ}ŋ} follows a higher, negated verb. Unlike SVCs, though, \textit{ny\'{ʌ}ŋ} itself can be negated. This suggests that there is a NegP associated with the matrix verb, and there is a second NegP associated with the clause that houses \textit{ny\'{ʌ}ŋ}. In other words, \textit{ny\'{ʌ}ŋ} clauses have biclausal properties, whereas SVCs are monoclausal.

\subsection{Object sharing test}\label{sec:duncan-et-al:2.4}

The final property that we consider when comparing \textit{ny\'{ʌ}ŋ} with SVCs is object sharing \citep{baker1989object}, shown in the following examples.

\ea\label{ex:duncan-et-al:11}
\ea[]{
\gll \'{E}kpê á-mà á-tóp ítíyát á-ń-t\'ɔ.\\
Ekpe 3{\SG-\PST} 3{\SG}-throw stone 3{\SG-1\SG}-hit \\
\glt `Ekpe threw a stone and it hit me.'
}
\ex[]{
\gll \'{E}kpê á-mà á-tóp ítíyát \textbf{á-ny\'{ʌ}ŋ} á-ń-t\'ɔ.\\
Ekpe 3{\SG-\PST} 3{\SG}-throw stone 3{\SG}-and 3{\SG-1\SG}-hit \\
\glt `Ekpe threw a stone (somewhere) and (something else) hit me.'
}
\z
\z

\noindent In (\ref{ex:duncan-et-al:11}a), the overt object of V$_1$, \textit{ítíyát} `stone', is ``shared'' by V$_2$. This sentence thus has the interpretation that Ekpe threw a stone, and that same stone is what Ekpe hit me with. \textit{Ny\'{ʌ}ŋ} disrupts this pattern; as seen in (\ref{ex:duncan-et-al:11}b), object sharing is blocked when the verbal coordinator is present.

\subsection{Interim summary}\label{sec:duncan-et-al:2.5}

Although \textit{ny\'{ʌ}ŋ} clauses bear surface affinity to SVCs, the preceding discussion shows that these construction types fail to show key morphosyntactic attributes that are characteristic of SVCs. \tabref{tab:duncan-et-al:1} summarizes these properties and how they do (or do not) map onto each \isi{clause type}.

\begin{table}
\caption{Properties of Ibibio SVCs and \textit{ny\'{ʌ}ŋ} clauses.}
\label{tab:1:properties}
 \begin{tabularx}{\textwidth}{Xccccc} % add l for every additional column or remove as necessary
  \lsptoprule
            & Single & Obligatory & Contrastive & Single & O sharing\\ %table header
            & tense & S sharing & \isi{focus} & \isi{negation} & \\
  \midrule
  SVCs  &   Y &    Y  &    Y &    Y  & Y\\
  \textit{Ny\'{ʌ}ŋ} clauses  &   N &   N &    N    & N & N\\
  \lspbottomrule
 \end{tabularx} \label{tab:duncan-et-al:1}
\end{table}

\noindent While this does not amount to a positive account for what \textit{ny\'{ʌ}ŋ} is, we take the above data as evidence for what \textit{ny\'{ʌ}ŋ} is not: \ili{Ibibio} \textit{ny\'{ʌ}ŋ} clauses are not SVCs. Instead, \textit{ny\'{ʌ}ŋ} clauses exhibit parataxis. Moreover, \textit{ny\'{ʌ}ŋ} is verb-like in that it bears agreement and can be negated, but it also bears non-verb-like properties, such as the inability to undergo contrastive \isi{verb focus}. 

\section{Structural observations}\label{sec:duncan-et-al:3}

Structurally, it would appear that \textit{ny\'{ʌ}ŋ} attaches below NegP, which is dominated by TP, and above \textit{v}P. This yields the following hierarchy for the constituent containing \textit{ny\'{ʌ}ŋ}.

\ea\label{ex:duncan-et-al:12}
TP >> NegP >> \textit{ny\'{ʌ}ŋ} >> \textit{v}P \\
\z

\noindent The location of \textit{ny\'{ʌ}ŋ}---what we have been calling a coordinator---presents a bit of a puzzle. In a language like \ili{English}, `and' introduces (and precedes all overt material in) the second conjunct, allowing for a structure as follows with conjoined TPs.\footnote{We adopt the asymmetric strucures in \figref{fig:duncan-et-al:1} and \figref{fig:duncan-et-al:2} following, e.g., \citet{munn1987,munn1993,munn1999}, \citet{kayne1994}, and \citet{johanessen1998}, a.o. Our point here is not to commit to a particular analysis of \isi{coordination} for either \ili{English} or \ili{Ibibio}. Instead, we schematize \isi{coordination} in each language to illustrate the uniqueness of \textit{ny\'{ʌ}ŋ}'s place in the syntax, both in terms of word order and structurally in relation to the coordinator.}

%\ea\label{ex:13:simple-coord}
\begin{figure}[h]

\Tree [.\&P \qroof{\ldots}.TP$_1$ [.\&$'$ \textit{and} \qroof{\ldots}.TP$_2$ ] ]

\caption{TP coordination in English.}
\label{fig:duncan-et-al:1}
\end{figure}
%\z

%\ea\label{ex:simple-coord}
%\Tree [.and \qroof{\ldots}.TP$_1$ \qroof{\ldots}.TP$_2$ ]
%\z

\noindent This is quite common cross-linguistically: `and'-words typically intervene between conjuncts.

In \ili{Ibibio} verb and \isi{predicate coordination}, though, the `and'-word \textit{ny\'{ʌ}ŋ} is embedded deeply inside the second conjunct. Thus, it is not that the presence of a second T$^0$ is problematic, and the possibility of a different \isi{subject} for the lower clause containing \textit{ny\'{ʌ}ŋ} is similarly unproblematic. How, then, might we account for the location of \textit{ny\'{ʌ}ŋ}, and what might this indicate about its status?



%\ea\label{ex:14:simple-coord}
\begin{figure}[h]

\Tree [.\&P \qroof{\ldots}.TP$_1$ [.\&$'$ \& [.TP$_2$ \textsc{subj} [.T$'$ T$^0$ [.FP \textit{ny\'{ʌ}ŋ}  \qroof{\ldots}.\textit{v}P ] ] ] ] ]

\caption{TP coordination in Ibibio.} \label{fig:duncan-et-al:2}

\end{figure}
%\z

%\ea\label{ex:simple-coord}
%\Tree [.\& \qroof{\ldots}.TP$_1$ [.TP$_2$ \textsc{subj} [.T$'$ T$^0$ [.?P \textit{ny\'{ʌ}ŋ}  \qroof{\ldots}.\textit{v}P ] ] ] ]
%\z

We tentatively pose the structure in Figure 2 to account to account for the unique distribution of \textit{ny\'{ʌ}ŋ}. If this line of thought is on the right track then, given its place in the structure, \textit{ny\'{ʌ}ŋ} is not actually (or is very unlikely to be) a coordinator. Instead, it appears to be an associate of \isi{coordination} that is restricted to verbal \isi{coordination}. We leave the precise structure of verb and \isi{predicate coordination} to future investigation; for now, treating a structure like the one in \figref{fig:duncan-et-al:2} as a live option opens up other avenues to consider, such as whether \textit{ny\'{ʌ}ŋ} clauses really are coordinate structures.

\section{Are \textit{ny\'{ʌ}ŋ} clauses really coordinate structures?}\label{sec:duncan-et-al:4}

If \ili{Ibibio} \textit{ny\'{ʌ}ŋ} clauses involve parataxis, they should be sensitive to the Coordinate Structure Constraint (CSC) \citep{ross1967constraints}, wherein:

\begin{itemize}
\item Extraction from a single conjunct is impossible; and
\item Extraction from both conjuncts is grammatical (= across-the-board (ATB) extraction).
\end{itemize}

\noindent \ili{Ibibio} verbal \isi{coordination} is indeed island-inducing and sensitive to the CSC. When \textit{v}Ps are coordinated, object extraction becomes impossible. This supports the notion that \textit{ny\'{ʌ}ŋ} clauses do involve \isi{coordination} (whether or not \textit{ny\'{ʌ}ŋ} is the coordinator or an associate of such).

Evidence for this comes from \textit{wh}-\isi{movement}. Neither the object in the first conjunct nor the object in the second conjunct can be extracted in \textit{ny\'{ʌ}ŋ} clauses.

\ea\label{ex:duncan-et-al:13}
\ea[]{
\gll \'{A}-mà á-díá àdésì \textbf{á-ny\'{ʌ}ŋ} á-ŋ w\'ɔŋ~ úk\'ɔtńs\`{ʌ}ŋ.\\
3{\SG-\PST} 3{\SG}-eat rice 3{\SG}-and 3{\SG}-drink palmwine \\
\glt `She ate rice and drank palmwine.'
}
\ex[*]{
\gll \`{N}s\v{o} ké á-ké-díá \textbf{á-ny\'{ʌ}ŋ} á-ŋ w\'ɔŋ~ úk\'ɔtńs\`{ʌ}ŋ? \\
what {\FOC} 3{\SG-\PST.\FOC}-eat 3{\SG}-and 3{\SG}-drink palmwine \\
\glt (Intended: `What$_i$ did she eat \textit{t}$_i$ and drink palmwine?')
}
\ex[*]{
\gll \`{N}s\v{o} ké á-ké-díá àdésì \textbf{á-ny\'{ʌ}ŋ} á-ŋ w\'ɔŋ? \\
what {\FOC} 3{\SG-\PST.\FOC}-eat rice 3{\SG}-and 3{\SG}-drink \\
\glt (Intended: `What$_i$ did she eat rice and drink \textit{t}$_i$?')
}
\z
\z

\noindent ATB extraction is, however, permitted.

\ea\label{ex:duncan-et-al:14}
%\langinfo{Ibibio}{Niger-Congo}{personal knowledge}\\
\gll \`{N}s\v{o} ké á-ké-díá \textbf{á-ny\'{ʌ}ŋ} á-ŋ w\'ɔŋ? \\
what {\FOC} 3{\SG-\PST.\FOC}-eat 3{\SG}-and 3{\SG}-drink \\
\glt `What$_i$ did she eat \textit{t}$_i$ and drink \textit{t}$_i$?'
\z

\noindent This result is expected if, in fact, \textit{ny\'{ʌ}ŋ} clauses are coordinate structures.

\ili{Ibibio} has both overt \textit{wh}-\isi{movement} (\ref{ex:duncan-et-al:15}a) and \textit{wh-in-situ} questions (\ref{ex:duncan-et-al:15}b), the latter of which may involve covert \isi{movement}.

\ea\label{ex:duncan-et-al:15}
\ea[]{
\gll \`{N}s\v{o} ké á-ké/*mà á-nám?\\
what {\FOC} 3{\SG-\PST.\FOC}/*{\PST} 3{\SG}-do \\
\glt `What did she do?'
}
\ex[]{
\gll \'{A}-ké á-nám ǹs\v{o}? \\
3{\SG-\PST.\FOC} 3{\SG}-do what \\
\glt `What did she do?'
}
\ex[]{
\gll \'{A}-mà á-nám ǹs\v{o}? \\
3{\SG-\PST} 3{\SG}-do what \\
\glt `She did what?'
}
\z
\z

\noindent Whether overt or covert, \={A}-extraction is signaled by the use of special \isi{focus} tense morphology. In (\ref{ex:duncan-et-al:15}a-b), for example, the \isi{tense marker} \textit{ké}- is obligatory for \isi{past tense}; use of the unmarked past \isi{tense marker} \textit{mà} produces ungrammaticality when extraction is overt, or else it signals an echo question, as in (\ref{ex:duncan-et-al:15}c).

These facts help us further diagnose the presence of \isi{coordination} in \textit{ny\'{ʌ}ŋ} clauses. Interestingly, with verbal \isi{coordination} the object \textit{wh}-question can remain {in situ} in the second conjunct with no overt object in the first conjunct (\ref{ex:duncan-et-al:16}a), but the reverse does not hold (\ref{ex:duncan-et-al:16}b).\footnote{It is also possible to leave an ordinary NP object in the first conjunct and have an object \textit{wh}-element in the second.

\ea\label{ex:duncan-et-al:i}
%\langinfo{Ibibio}{Niger-Congo}{personal knowledge}\\
\gll \`{A}-ké à-díá ádésí à-ny\'{ʌ}ŋ~ à-ŋ w\'ɔŋ~ ǹs\v{o}? \\
2{\SG-\PST.\FOC} 2{\SG}-eat rice 2{\SG}-and 2{\SG}-drink what \\
\glt `You ate rice and drank what?'
\z
However, this blocks the wide \isi{scope} interpretation and forces an echo reading. It appears that the presence of the object `rice' in (i) blocks covert ATB \isi{movement}.}

\ea\label{ex:duncan-et-al:16}
\ea[]{
\gll \`{A}-ké à-díá à-ny\'{ʌ}ŋ~ à-ŋ w\'ɔŋ~ ǹs\v{o}? \\
2{\SG-\PST.\FOC} 2{\SG}-eat 2{\SG}-and 2{\SG}-drink what \\
\glt `What$_i$ did you eat \textit{t}$_i$ and drink \textit{t}$_i$?'
}
\ex[*]{
\gll \`{A}-ké à-díá ǹs\v{o} à-ny\'{ʌ}ŋ~ à-ŋ w\'ɔŋ? \\
2{\SG-\PST.\FOC} 2{\SG}-eat what 2{\SG}-and 2{\SG}-drink \\
\glt (Intended: `What did you eat and drink?')
}
\z
\z

\noindent Combining these two strategies yields a positive result: two {in situ} questions can be coordinated by \textit{ny\'{ʌ}ŋ}.\footnote{We do not attempt here a syntactic analysis of \textit{wh}-questions in \ili{Ibibio}, but the ungrammaticality of (\ref{ex:duncan-et-al:16}b) is interesting in light of the availability of partial \textit{wh}-\isi{movement} in the language. The impossibility of the object \textit{wh}-element stopping and being pronounced in object position of the first conjunct as it transits upwards is most likely an artifact of the type of conjuncts being coordinated (i.e., TPs or \textit{v}Ps, but not CPs).}

\ea\label{ex:19:covert-ATB-two-obj}
%\langinfo{Ibibio}{Niger-Congo}{personal knowledge}\\
\gll \`{A}-ké à-díá ǹs\v{o} à-ny\'{ʌ}ŋ~ à-ŋ w\'ɔŋ~ ǹs\v{o}? \\
2{\SG-\PST.\FOC} 2{\SG}-eat what 2{\SG}-and 2{\SG}-drink what \\
\glt `What did you eat and drink?'
\z

\noindent These facts suggest that both overt and covert ATB extraction are possible in \ili{Ibibio}. 

Thus, even though \textit{ny\'{ʌ}ŋ} itself may not be a coordinator, \isi{predicate coordination} behaves as if \isi{coordination} is present. Clauses coordinated with \textit{ny\'{ʌ}ŋ} behave like syntactic islands and obey CSC constraints. This makes a \isi{coordination} analysis of \textit{ny\'{ʌ}ŋ} clauses a viable option, even though the question of what \textit{ny\'{ʌ}ŋ} is remains unresolved.

\section{Is \textit{ny\'{ʌ}ŋ} a verb, or something else?}\label{sec:duncan-et-al:5}

In \sectref{sec:duncan-et-al:2} we argued against analyzing \textit{ny\'{ʌ}ŋ} as part of an SVC, but this by itself does not preclude \textit{ny\'{ʌ}ŋ} from being a verb of some kind. Even though \textit{ny\'{ʌ}ŋ} possesses verb-like qualities, in this section we show that it actually behaves more akin to a low preverbal adverb.

\ili{Ibibio} adverbs that attach low on the clausal spine commonly appear postverbally in reduplicant form (\ref{ex:duncan-et-al:18}a). Some of these adverbs, such as the one translated `quickly' below, alternate between postverbal and preverbal position.

\ea\label{ex:duncan-et-al:18}
\ea[]{
\gll \'{I}má á-mà á-f\`{e}hé ít\`ɔk \textbf{ù-s\'ɔp} \textbf{ù-s\'ɔp}. \\
Ima 3{\SG-\PST} 3{\SG}-run race {\NMLZ}-do.quickly {\NMLZ}-do.quickly \\
\glt `Ima ran the race quickly.'
}
\ex[]{
\gll \'{I}má á-mà \textbf{á-s\'ɔp} á-f\`{e}hé ít\`ɔk. \\
Ima 3{\SG-\PST} 3{\SG}-do.quickly 3{\SG}-run race \\
\glt `Ima ran the race quickly.'
}
\z
\z

\noindent Postverbal reduplicant adverbs are nominalized, but do not bear \isi{subject agreement}. When these adverbs appear preverbally, the reverse is true. This is significant for the purposes of the present paper because it potentially identifies intermediate space between T$^0$ and \textit{v}$^0$ where \isi{subject} agreeing elements can reside.

Also like \textit{ny\'{ʌ}ŋ}, main verbs, and V$_1$s in SVCs, low preverbal adverbs can bear \isi{negation}.

\ea\label{ex:duncan-et-al:19}
%\langinfo{Ibibio}{Niger-Congo}{personal knowledge}\\
\gll \'{I}má í-kí-\textbf{s\'ɔp-p\'ɔ} í-f\`{e}hé ít\`ɔk. \\
Ima \textsc{i}-{\PST.\FOC}.\textsc{i}-do.quickly-\textsc{neg} \textsc{i}-run race \\
\glt `Ima didn't run the race quickly.'
\z

\noindent Given the proposed site of low adverbs like `quickly', presumably they can be the goal of a higher probe that triggers raising-to-Neg, just as a main verb can, and just as \textit{ny\'{ʌ}ŋ} can. 

Unlike main verbs and V$_1$s in SVCs---but like \textit{ny\'{ʌ}ŋ}---low preverbal adverbs cannot be contrastively focused.

\label{ex:duncan-et-al:20}
\ea[*]{
\gll \'{I}má á-ké \textbf{á-s\`ɔ\'ɔ-s\'ɔp} á-f\`{e}hé ít\`ɔk. \\
Ima 3{\SG-\PST.\FOC} 3{\SG}-do.quickly-do.quickly 3{\SG}-run race \\
\glt (Intended: `Ima QUICKLY ran the race.')
}
\z

\noindent This restriction comports well with our understanding of where \textit{ny\'{ʌ}ŋ} is located. Distributionally, then, low adverbs may be significant for two reasons. On the one hand, they offer insight into the nature of \textit{ny\'{ʌ}ŋ} in terms of category. Second, they provide supporting evidence into the placement of \textit{ny\'{ʌ}ŋ} structurally. Elements that attach above \textit{v}P are not accessible to low Foc$^0$. However, \textit{ny\'{ʌ}ŋ} and low adverbs do display relevant differences. Specifically, \textit{ny\'{ʌ}ŋ} does not have an alternative postverbal reduplicative form.

\label{ex:duncan-et-al:21}
\ea[*]{
\gll \ldots \'{m}-f\'ɔp ùnàm \textbf{ǹ-ny\'{ʌ}ŋ} \textbf{ǹ-ny\'{ʌ}ŋ}. \\
1{\SG}-roast meat {\NMLZ}-and {\NMLZ}-and \\
\glt (Intended: `\ldots and I roasted meat.')
}
\z

\noindent \textit{Ny\'{ʌ}ŋ} therefore successfully negates and \textit{un}successfully undergoes contrastive \isi{verb focus}, just like a low adverb. But, simply identifying \textit{ny\'{ʌ}ŋ} as an adverb is potentially suspect, given that it cannot surface postverbally.\footnote{An anonymous reviewer rightfully notes that the attempt to put \textit{ny\'{ʌ}ŋ} postverbally may simply be disallowed for independent reasons, such as iconicity. If this is the case, then evidence for the adverb-like nature of \textit{ny\'{ʌ}ŋ} is even stronger.}

\textit{Ny\'{ʌ}ŋ} and `quickly' can also co-occur preverbally in the same clause, and stack like adverbs do elsewhere.

\ea\label{ex:duncan-et-al:22}
\ea[]{
\gll \'{M}-mà á-kót úy\`{o} \`{m}f\`{o} \textbf{ń-ny\'{ʌ}ŋ} \textbf{ń-s\'ɔp} ń-dí. \\
1{\SG-\PST} 3{\SG}-hear voice your 1{\SG}-and 1{\SG}-do.quickly 1{\SG}-come \\
\glt `I heard your voice and came quickly.'
}
\ex[*]{
\gll \'{M}-mà á-kót úy\`{o} \`{m}f\`{o} \textbf{ń-s\'ɔp} \textbf{ń-ny\'{ʌ}ŋ} ń-dí. \\
1{\SG-\PST} 3{\SG}-hear voice your 1{\SG}-do.quickly 1{\SG}-and 1{\SG}-come \\
\glt (Intended: `I heard your voice and came quickly.')
}
\z
\z

\noindent Importantly, a rigid ordering ensues when \textit{ny\'{ʌ}ŋ} and `quickly' appear together: the former must precede the latter, at least linearly.

As suggested previously, we take it that \textit{ny\'{ʌ}ŋ} attaches low in the clause (below NegP and above \textit{v}P), but the differential outcomes of (\ref{ex:duncan-et-al:22}a) and (\ref{ex:duncan-et-al:22}b) necessitate a bit more precision. One possible way to approach a more specific attachment site is to explore additionally available projections in the inflectional layer, which in \ili{Ibibio} is rather rich. \citet{baker2010agreement} motivate the following expanded architecture.

\ea\label{ex:duncan-et-al:23}
MoodP >> TP >> AspP >> \textit{v}P >> VP
\z

\noindent Additional layers might prove helpful for syntactic signposting, and, given the location of AspP, it stands out as a likely candidate for helping determine a more precise location for \textit{ny\'{ʌ}ŋ}.

Though the ordering of \textit{ny\'{ʌ}ŋ} is fairly predictable on account of its fixed order with respect to low adverbs, it appears to have a bit more flexibility with respect to Asp$^0$.

\ea\label{ex:duncan-et-al:24}
\ea[]{
\gll \ldots \'{m}-mà \textbf{ń-sé} \textbf{ń-ny\'{ʌ}ŋ} ń-tímmé ń-k\`{e}né \'{m}-f\'ɔp ùnàm.\\
1{\SG-\PST} 1{\SG}-\textsc{hab} 1{\SG}-and 1{\SG}-repeat 1{\SG}-emulate 1{\SG}-roast meat \\
\glt `\ldots and I also again with other folks had been roasting meat.'
}
\ex[]{
\gll \'{N}-kpá ń-ké ń-sé ń-kóót ǹw\`{e}t (ń-kpá ń-ké) \textbf{ń-ny\'{ʌ}ŋ} \textbf{ń-sé} \'{m}-br\v{e} \`{m}-br\v{e}\ldots \\
1{\SG-\COND} 1{\SG-\PST.\FOC} 1{\SG}-\textsc{hab} 1{\SG}-read.{\PL} book 1{\SG-\COND} 1{\SG-\PST.\FOC} 1{\SG}-and 1{\SG}-\textsc{hab} 1{\SG}-play {\NMLZ}-play \\
\glt `I would have read books and I would have played \ldots'
}
\z
\z

\noindent Thus, \textit{ny\'{ʌ}ŋ} can potentially attach above or below AspP, but it must always be below MoodP, TP, and NegP, and above \textit{v}P.

\ea\label{ex:duncan-et-al:25}
\gll \ldots ń-kpé ń-ké \textbf{í-ny\'{ʌ}ŋ-ŋ \'ɔ} \textbf{ń-sé} \'{m}-br\v{e} \`{m}-br\v{e}. \\
1{\SG-\COND} 1{\SG-\PST.\FOC} í-and-\textsc{neg} 1{\SG}-\textsc{hab} 1{\SG}-play {\NMLZ}-play \\
\glt `\ldots and I wouldn't have played.'
\z

Taken together, the data from this section shows that \textit{ny\'{ʌ}ŋ} is both verb-like and adverb-like. \tabref{tab:duncan-et-al:2} compares properties of verbs with that of low adverbs and \textit{ny\'{ʌ}ŋ}.

\begin{table}
\caption{Properties of verbs, low adverbs, and \textit{ny\'{ʌ}ŋ}.}
\label{tab:2:properties}
\fittable{
 \begin{tabular}{lllll}
  \lsptoprule
            & S-agreeing & Negatable & Focusable & Postverbal \\ %table header
            &  &  & contrastively & \\
  \midrule
  Main verbs \& V$_1$s in SVCs  &   Y &    Y  &    Y &   n/a  \\
  Low preverbal adverbs  &   Y &    Y  &    N &    Y  \\
  \textit{Ny\'{ʌ}ŋ}  &   Y &   Y &    N    & N \\
  \lspbottomrule 
 \end{tabular}
 \label{tab:duncan-et-al:2}
 }
\end{table}

\noindent Although the differences are not major, comparing \textit{ny\'{ʌ}ŋ} with similar elements reveals that it is both verb-like and adverb-like, but bears a stronger affinity to the latter, making it a special type of adverb.

%\section{Other (ad)verby things that co-occur with or replace \textit{ny\'{ʌ}ŋ}}

%In addition to \textit{ny\'{ʌ}ŋ}, \ili{Ibibio} has two other (ad)verb-like elements---\textit{tímmé} `do again' and \textit{k\`{e}né} `emulate'---which bear distributional and semantic affinity to \textit{ny\'{ʌ}ŋ}. In this section, we discuss some of the properties of these elements to offer a more rounded discussion of what \textit{ny\'{ʌ}ŋ} may be.

%First, both \textit{tímmé} and \textit{k\`{e}né} participate in \ili{Ibibio}'s robust system of \isi{subject agreement}.

%\ea
%\ea[]{
%\gll \'{M}-mà ú-kót \textbf{ń-tímmé} ú-w\`{e}t.\\
%1{\SG-\PST} 2{\SG}-call 1{\SG}-do.again 2{\SG}-write \\
%\glt `I called you and wrote you again.'
%}
%\ex[]{
%\gll \'{E}kpê á-mà á-bárá ìkáŋ~ \textbf{á-k\`{e}né} á-f\'ɔp ùnàm.\\
%Ekpe 3{\SG-\PST} 3{\SG}-ignite fire 3{\SG}-emulate 3{\SG}-roast meat \\
%\glt `Ekpe started a fire and (in addition to others) roasted meat.'
%}
%\z
%\z

%\noindent Structurally, these two elements appear in the same general vicinity as does \textit{ny\'{ʌ}ŋ}. That is, they all surface preverbally in what is or appears to be the second of two conjuncts. Together with \textit{ny\'{ʌ}ŋ}, then, these elements both seem to ``link'' clauses in some way. Semantically, they encode some rough approximation of ``doing again'' or doing ``in addition to.'' For \textit{tímmé}, the \isi{focus} seems to be on a subsequent iteration of the same type of action, whereas \textit{k\`{e}né} orients more toward modifying the \isi{subject} rather than the event itself.

%All three elements can co-occur, as seen in (29):

%\protectedex{
%\ea\label{ex:29:nyung-timme-kene}
%\gll \'{M}-mà \'{m}-bárá ìkáŋ~ \'{m}-mà \textbf{ń-ny\'{ʌ}ŋ} \textbf{ń-tímmé} \textbf{ń-k\`{e}né} ń-s\'ɔp \'{m}-f\'ɔp ùnàm. \\
%1{\SG-\PST} 1{\SG}-ignite fire 1{\SG-\PST} 1{\SG}-and 1{\SG}-do.again 1{\SG}-emulate 1{\SG}-do.quickly 1{\SG}-roast meat \\
%\glt `I started a fire and also again (along with other folks) roasted meat.'
%\z
%}

%\noindent Here, all three surface before the low preverbal adverb `quickly', which begins to hint at structural properties of each. Like \textit{ny\'{ʌ}ŋ} (/main verbs/V$_1$s in V$_1$ V$_n$ sequences/low adverbs), both \textit{tímmé} and \textit{k\`{e}né} can be negated:

%\ea\label{ex:30:timme-kene-negated}
%\gll \ldots ń-ké \textbf{í-tímmé/k\`{e}né-ké} \'{m}-f\'ɔp ùnàm. \\
%1{\SG-\PST.\FOC} \textsc{i}-do.again/emulate-\textsc{neg} 1{\SG}-roast meat \\
%\glt ‘\ldots I didn't roast meat again/with others.’
%\z

%\noindent Once again, \textit{ny\'{ʌ}ŋ} is observed to be like a chameleon, in that it resembles other things in its morphosyntactic environment. 

%Nevertheless, several key morphosyntactic features distinguish \textit{ny\'{ʌ}ŋ} from \textit{tímmé} and \textit{k\`{e}né}. The order in (29), for example, is fixed. Whenever more than one of the three elements is present, they appear in accordance with the following hiearchy:

%\ea
%\textit{ny\'{ʌ}ŋ} >> \textit{tímmé} >> \textit{k\`{e}né}
%\z

%\noindent The fact that \textit{tímmé} and \textit{k\`{e}né} follow---and thus presumably appear lower than---\textit{ny\'{ʌ}ŋ} is significant because it suggests that they do not occupy or compete for the same clausal position. 

%This point is further reinforced by the behavior of each (a) with respect to the low \isi{focus} projection and (b) in relation to T$^0$. 

%\ea
%\ea[]{
%\gll \'{A}-\textbf{t\`{e}é-tímmé} à-ŋ w\'ɔŋ~ úk\'ɔtńs\`{ʌ}ŋ?\\
%2{\SG}-do.again-do.again 2{\SG}-drink palmwine \\
%\glt `You are drinking palmwine AGAIN?'
%}
%\ex[]{
%\gll \'{A}-\textbf{k\`{e}é-k\`{e}né} à-ŋ w\'ɔŋ~ úk\'ɔtńs\`{ʌ}ŋ?\\
%2{\SG}-emulate-emulate 2{\SG}-drink palmwine \\
%\glt `You are FOLLOWING OTHERS in drinking palmwine?'
%}
%\ex[]{
%\gll \'{M}-mà ú-kót \textbf{\'{m}-mà} *(\textbf{ń-ny\'{ʌ}ŋ}) \textbf{ń-tímmé} ú-w\`{e}t.\\
%1{\SG-\PST} 2{\SG}-call 1{\SG-\PST} 1{\SG}-and 1{\SG}-do.again 2{\SG}-write \\
%\glt `I called you and I also wrote you again.'
%}
%\ex[]{
%\gll \'{M}-mà ú-kót (*\textbf{\'{m}-mà}) \textbf{ń-tímmé} ú-w\`{e}t.\\
%1{\SG-\PST} 2{\SG}-call 1{\SG-\PST} 1{\SG}-and 1{\SG}-do.again 2{\SG}-write \\
%\glt `I called you and wrote you again.'
%}
%\z
%\z

%\noindent In (32a-b), although the interrogative reading is unexpected, it is apparent that \textit{tímmé} and \textit{k\`{e}né} can be contrastively focused. As noted above, this is a property of verbs, but not of \textit{ny\'{ʌ}ŋ}. The pair of sentences in (32c-d) showcase another important difference: only the presence of \textit{ny\'{ʌ}ŋ} can correlate with the presence of a second T$^0$. If \textit{ny\'{ʌ}ŋ} is removed, the second T$^0$ in (32b) makes the sentence ungrammatical. This suggests, in accordance with observations above, that \textit{ny\'{ʌ}ŋ} clauses are biclausal. 

%Consequently, although \textit{tímmé} and \textit{k\`{e}né} share some properties with \textit{tímmé} and \textit{k\`{e}né}, they also showcase critical syntactic differences. Table 3 summarizes the properties of these three elements.

%\begin{table}
%\caption{Properties of \textit{tímmé} and \textit{k\`{e}né} compared to \textit{ny\'{ʌ}ŋ}.}
%\label{tab:3:properties2}
% \begin{tabular}{lllllll} % add l for every additional column or remove as necessary
%  \lsptoprule
%            & S-agreeing & Negatable & Precede & Focusable & Introduce & Coordinate \\ %table header
%            &  &  & low adverbs & contrastively & T$^0$ & clauses \\
%  \midrule
%  \textit{Tímmé}        &   Y &    Y  &    Y & Y & N & N \\
%  \textit{K\`{e}né}         &   Y &    Y  &    Y & Y & N & N \\
%  \textit{Ny\'{ʌ}ŋ}  &   Y &    Y  &    Y & N & Y & Y \\
%  \lspbottomrule
% \end{tabular}
%\end{table}

%\noindent The most important distinctions are apparent when comparing the values of the properties in the right half of the table. Together with the hiearchical ordering in (31), (in)ability to undergo contrastive \isi{verb focus} suggests that \textit{ny\'{ʌ}ŋ} is located outside of the c-command domain of low Foc$^0$, whereas the other elements are inside this domain. Given that only \textit{ny\'{ʌ}ŋ} can introduce T$^0$, we conclude that \textit{tímmé} and \textit{k\`{e}né} merely give the illusion of clausal \isi{coordination}, though they may, in fact, be a type of \isi{serial verb} construction that can occur within the second conjunct of a coordinate structure.

\section{Conclusion}\label{sec:duncan-et-al:6}

%\ili{Ibibio} verb and \isi{predicate coordination} present a typologically unique profile. The overt element that one might identify as an `and'-word, \textit{ny\'{ʌ}ŋ}, exhibits puzzling hybridity, making it difficult to define categorially. This paper sought to identify key morphosyntactic properties of \textit{Ny\'{ʌ}ŋ} clauses in an effort to approach a more nuanced understanding of (a) what \textit{ny\'{ʌ}ŋ} actually is and (b) what clause types it is deployed in beyond the descriptions currently available in \ili{Ibibio} literature.

Reminiscent of \ili{Walman} ```and'-verbs'' \citep{brown2008verbs}, \textit{ny\'{ʌ}ŋ} in \ili{Ibibio} displays several verb-like characteristics, such as \isi{subject agreement}, ability to bear \isi{negation}, and (potentially) being inflected for tense. Recognition of these properties has led to the standard assumption that \textit{ny\'{ʌ}ŋ} is part of a \isi{serial verb} construction. In light of recent developments regarding properties of \ili{Ibibio} serial verbs, though, we find that \textit{ny\'{ʌ}ŋ} effectively fails to meet all criteria for seriality. Distributional evidence similarly showed an affinity between \textit{ny\'{ʌ}ŋ} and low adverbs. Nevertheless, just as \textit{ny\'{ʌ}ŋ} is verb-like in degrees, we likewise find only partial correspondences with adverbs.

In our approach to \textit{ny\'{ʌ}ŋ} we largely focused on delineating what \textit{ny\'{ʌ}ŋ} is not, refraining from strong positive statements about what \textit{ny\'{ʌ}ŋ} actually is. Still, current evidence weighs in favor of \textit{ny\'{ʌ}ŋ} being an adverb of a special type. Moreover, the data reveal some promising directions that may shed light on the precise nature of \textit{ny\'{ʌ}ŋ} and \textit{ny\'{ʌ}ŋ} clauses. First, these clauses are island-inducing, which supports the claim that \textit{ny\'{ʌ}ŋ} truly participates in \isi{coordination}. Perhaps most surprisingly, though, our presentation casts doubt on the notion that \textit{ny\'{ʌ}ŋ} is itself a coordinator. Together, we take these observations as possible evidence for covert \isi{coordination} in the language. If this is on the right track then \textit{ny\'{ʌ}ŋ} operates as an associate of covert conjunction. 


%\isi{prolegomena}
 
\section*{Abbreviations}\label{sec:duncan-et-al:abbreviations}

Abbreviations follow the 2015 Leipzig Glossing Rules, with one addendum: \textsc{i} = \isi{default agreement} marker /í/, following \citet{baker2010agreement}.

\section*{Acknowledgements}\label{sec:duncan-et-al:acknowledgments}

Many people have offered great help and insightful comments throughout our work on this project. We would like to thank Harold Torrence, Jason Kandybowicz, Ibrahima Ba, Longcan Huang, Lydia Newkirk, Zhuo Chen, Masashi Harada, audience members of ACAL 47, and two anonymous reviewers. All remaining errors are our own.

\sloppy
\printbibliography[heading=subbibliography,notkeyword=this]

\end{document}