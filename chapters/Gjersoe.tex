\documentclass[output=paper]{LSP/langsci} 
\author{{Siri Gjers\o e}\affiliation{Leipzig University}\and Jude Nformi\affiliation{Leipzig University}\lastand Ludger Paschen\affiliation{Leipzig University}} 
\title{Hybrid falling tones in Limbum}

\abstract{This paper examines the interaction between lexical tone and phrase-level intonation in Limbum. On the basis of an acoustic study of novel data, we claim that Limbum has a phrase-final low boundary tone (L\%) that interacts with lexical tones to give rise to \textit{hybrid} falling tones: tones whose specifications are partially lexical and partially phrasal. We argue that hybrid tones and other tonal processes in Limbum are readily captured in an analysis that assumes tonal geometry and empty nodes. We propose to represent L\% as a floating low register feature (l) that links to lexical tonal root nodes, giving rise to various surface patterns depending on the tonal specifications of the root nodes. Our account supersedes previous analyses in terms of tone sandhi rules.}

\maketitle

%\begin{document}

\nonstopmode

% --------------------------------- %
\section{Introduction}

\ili{Limbum} is a \ili{Grassfields} \ili{Bantu} language spoken by about 130,000 speakers in the Donga Mantung division of the North West region of Cameroon.
\ili{Limbum} is an understudied language, especially with regards to its suprasegmental phonetics and phonology.
In previous work \citep{Fiore.1987,Fransen.1995}, \ili{Limbum} has been described as a \isi{tone} language with three level tones (H, L, M) and four contour tones (HL, LM, ML, LL).\footnote{The sources mentioned also discuss a somewhat dubious fifth contour \isi{tone}, HM. \citet{Fiore.1987} argues that HM is an allotone of HL and proposes segmental length as a factor conditioning allotony, a view that is shared in \citet{Fransen.1995}. However, \citet{Fiore.1987} presents only two examples of HM-toned words, and our informants accept this \isi{tone} on only a single lexical item, \textit{b\'a\=a} `two'. On the basis of its highly limited distribution, we decided not to include HM in our study.\label{footnote:gjersoe:testsentences}}
It has also been observed that low-falling tones appear as level tones when they occur in a non-sentence-final position, a process which \citet{Fransen.1995} argues is the result of a sandhi simplification rule.

In this paper, we present an acoustic analysis of novel data from recordings of three native speakers of \ili{Limbum}.
We show that the data are actually more complex and Fransen's analysis fails to account for the whole range of tonal alternations.
Instead, we claim that \ili{Limbum} has a final low boundary \isi{tone} (L\%) in certain syntactic contexts.
Adopting the decompositional approach by \citet{Snider.1999}, we argue that L\% is a floating low \isi{register feature} that can create phrase-final falling contour tones by associating to lexical tonal root nodes.
Crucially, we assume that falling contour tones are not falling underlyingly: they only differ from {level tones} by having an additional empty \isi{tonal root node} associated to their TBU.
L\% interacts with lexical tonal specifications to create \textit{hybrid} surface tones, i.e. tones that combine lexical and phrasal tonal features.

The paper is structured as follows.
In \sectref{sec:gjersoe:2}, we present our acoustic study and offer a qualitative analysis of F0 tracks for all tested items.
\sectref{sec:gjersoe:ouranalysis} comprises the formal part of this paper, in which we provide a unified analysis of the tonal processes described in the previous section.
In \sectref{sec:gjersoe:4}, we discuss why our analysis fares better than alternative accounts and probe typological implications.

% --------------------------------- %
\section{Acoustic Study}\label{sec:gjersoe:2}

\subsection{Data and Methods}\label{sec:gjersoe:2.1}
Data presented in this study were collected from two male (ages 23 and 29) and one female (age 26) speakers of \ili{Limbum} (Central/Warr dialect). % M\textsubscript{1}, M\textsubscript{2}, F\textsubscript{1}
%Both speakers grew up in Nkambe % (Divisional Headquarter of Donga-Mantung division) 
% which is considered by most Wimbum (\ili{Limbum} speakers) as the ``purest form'' of their language (\citealt[6]{Fiore.1987}). % which is spoken in the Northern part of the \ili{Mbum} land.
% The male speaker moved to Germany in 2015 for studies while
Recordings of one of the male speakers were conducted at the phonetics laboratory at Leipzig University in the winter of 2015 using a T-bone SC 440 supercardioid microphone (sampling rate 44.1 kHz, 16-bit).
The recordings of the two other speakers were conducted in Buea, Cameroon using an H5 Zoom recorder with a SM10A Shure microphone (same sampling and bit rates).
%At the time of recording, the female speaker had been living in Douala for 2 years.

The speakers were given a reading task with a set of constructed test sentences.
In the examples in \REF{ex:gjersoe:testsentences}, \textit{le} (in boldface) is the target word.
We tested five sentence types:
Declarative sentences in which the target word appears in a sentence-final position (\textit{Decl.Fin}), declaratives in which the target word appears in a non-sentence-final position (\textit{Decl.Med}), simple wh-questions with the target word as the last item (\textit{Wh.Fin}), wh-questions with the final question particle \textit{a} (\textit{Wh.Prt}), and polar questions which always end in the particle \textit{a} (\textit{Pol}).
The semantic difference between \textit{Wh.Fin} and \textit{Wh.Prt} is that the latter signals that the wh-element is a known referent.\footnote{See \citeauthor{Driemeletal2017} (forthc.) for further discussion of the  functional domains of particles in \ili{Limbum}.}
A complete list of target words (two words per \isi{tone})\footnote{We found two microprosodic effects of vowel height: (1) With low-vowel items, F0 values overlap for  HL and the ML; (2) with high-vowel items, LM undergoes flattening when it precedes a L \isi{tone}. Since these effects appear to be due to phonetic variation and distract away from the actual \isi{tone} patterns, we present the F0 traces of all items combined rather then separating them into high- and low-vowel items.\label{footnote:gjersoe:microprosody}} is given in \tabref{tab:gjersoe:targetwords}.\footnote{We adopt the convention of writing two vowels for syllables with contour tones in order to accommodate the tonal diacritics. However, this also reflects the extra length observed especially (but not exclusively) on sentence-final contour tones. Note that the use of two vowel symbols does \textit{not} represent a phonemic length contrast because such a contrast is absent in the dialect of \ili{Limbum} under discussion.}
% We picked one set of test items with the low vowel /a/ as nucleus (items which merge with the particle \textit{a}) and one set with a high or mid vowel nucleus (items which do not merge with the particle \textit{a})} in order to control for effects of microprosody.
In total, our study comprises 7 tones x 2 words x 5 sentence types x 3 speakers.  %Correction from 6-7 because LM is included
Each sentence was pronounced 1--2 times by each speaker.
Values for sentences with more than one repetition were aggregated in R studio (v. 3.2.2).\footnote{It was only possible to record \emph{t\=a\`a} `father' and \emph{s\`o\=o} `basket' for one speaker. We used two repetitions per item from that speaker, aggregated in R.} %Should we have some kind of explanation for why?
% dismiss the fact that MD misses one LM word

% Siri: I have now revised the graphs without 'bird' so only two HL items as in the list. I also added one more repetitions each of father and basket from Jude to compensate for the lack recordings from the two speakers of this item. \red{why 3 words for HL? let's decide on one word with /a/ (bird OR cutlass), otherwise this looks really fishy and people will wonder why we used three words for HL but only 2 words for all other tones}

\ea\label{ex:gjersoe:testsentences}
\gll T\'ank\'o \`am y{\=ɛ} \textbf{l\'e}\\ 
T. \textsc{pst} see bat\\
\glt `Tanko saw a bat.' \hfill \textit{(Decl.Fin)}
\z


\ea
\gll T\'{a}nk\'{o} \`am y{\=ɛ} \textbf{l\'e} f\={\i}\\
T. \textsc{pst} see bat new\\
\glt `Tanko saw a new bat.' \hfill \textit{(Decl.Med)}
\z


\ea
\gll \'a nd\={a} \`am y{\=ɛ} \textbf{l\'e}\\
\textsc{foc} who \textsc{pst} see bat\\
\glt `Who saw a bat?' \hfill \textit{(Wh.Fin)}
\z


\ea
\gll \'a nd\={a} \`am y{\=ɛ} \textbf{l\'e} a\\
\textsc{foc} who \textsc{pst} see  bat \textsc{prt}\\
\glt `Who saw a bat?' \hfill \textit{(Wh.Prt)}
\z


\ea\label{ex:gjersoe:testsentencesII}
\gll  T\'{a}nk\'{o} \`am y{\=ɛ} \textbf{l\'e} a\\
T. \textsc{pst} see bat \textsc{prt}\\
\glt `Did Tanko se a bat?' \hfill \textit{(Pol)}
\z


\vs \noindent The aim of our acoustic study was to test prior observations that contour tones alternate with level tones phrase-medially \citep{Fransen.1995}, and also to examine whether lexical tones interact with potential edge-bound prosodic phenomena such boundary tones.
In the following, we abbreviate alternating low-falling\slash level tones as \textit{L(L)}, \textit{M(L)}, and \textit{H(L)}, and we use \textit{T(L)} to refer to the whole group of alternating tones. Level tones are abbreviated  as \textit{L}, \textit{M}, \textit{H}, and \textit{T}, respectively.
Annotations were done in Praat \citep{Praat2016} and automatically extracted from TextGrid and PitchTier files.
Starting from the M-toned verb \textit{y{\=ɛ}} `see' (see \REF{ex:gjersoe:testsentences}), the onset (O) and nucleus (N) of the target words and any syllables following them (\textit{f\={\i}} in \textit{Decl.Med} and the particle \textit{a} in \textit{Pol} and \textit{Wh.Prt}) were annotated.
A Praat script by \citet{RemijsenTrimCheck} was run to generate Pitch objects that are automatically trimmed for spikes using the algorithm in \citet{Xu1999}.
The items were manually corrected for microprosodic effects on F0.   
Interpolation for words with voiceless consonantal onsets (for two out of our 14 test words) was done using the smoothing algorithm in Praat.
F0 values at equidistant time points within intervals were then extracted using another Praat script \citep{RemijsenScriptF0Norm}.
The F0 values were converted into semitones (st) in R, with the midpoint value of \textit{y{\=ɛ}} `see' serving as base line for the semitone scale for each individual item.

\begin{table}[h]
\caption{List of target words and attested tone types in Limbum}
\label{tab:gjersoe:targetwords}
\centering
\begin{tabular}{cccccc}
\lsptoprule
\textbf{Tone} & \textbf{Word 1} & \textbf{Gloss 1} & \textbf{Word 2} & \textbf{Gloss 2} \\\midrule
\multicolumn{5}{c}{\textsc{Level tones}} \\\hline
L &  \emph{b\`a} & `bag' & \emph{b\`i} & `people' \\
M & \emph{b\=a} & `fufu' & \emph{b\=o} & `children'\\
%M & \emph{nt\=a} & market & \emph{\textipa{N}g\=u} & wood\\
H & \emph{b\'a} & `hill' & \emph{l\'e} & `bat' \\
\midrule\multicolumn{5}{c}{\textsc{Low-falling contour tones}} \\\midrule
L(L) & \emph{r\`a\`a} & `bridge' & \emph{rd\`o\`o} & `going'\\
M(L) & \emph{t\=a\`a} & `father' & \emph{b\={\i}\`i} & `co-wife' \\
H(L) & \emph{d\'a\`a} & `cutlass' & \emph{k\'u\`u} & `funnel'\\
% \emph{cw\'a\`a} & `yellow bird'
\midrule\multicolumn{5}{c}{\textsc{Rising contour tones}} \\\midrule
LM & \emph{y\`a\=a} & `princess' & \emph{s\`o\=o} & `basket'\\ % speaker MD: "soo" missing
%M  & \emph{b\'a\=a} & `two' &  -- & -- \\
\lspbottomrule
\end{tabular}
\end{table}

% ----- %
\subsection{Results}\label{sec:gjersoe:2.2}
The graphs below show the descriptive statistics of the tones in each tested context with F0 traces normalized for all three speakers.

% ----- %
\subsubsection{Falling contour tones}\label{sec:gjersoe:2.2.1}
The nuclei of L(L), M(L), and H(L) toned words are all falling sentence-finally (\textit{Decl.Fin} and \textit{Wh.Fin}, left graph in \figref{fig:gjersoe:level-contour}).
Sentence-medially, no pronounced falling \isi{movement} can be observed in the nuclei, confirming the claim in \citet{Fransen.1995} that contour tones alternate with level tones sentence-medially (\textit{Decl.Med}, right graph in \figref{fig:gjersoe:level-contour}).
LM is rising in all sentence-types and the F0 traces show that LM is not lowered sentence-finally.
Low-falling L(L) is accompanied by breathy voice in \textit{Decl.Fin} and \textit{Wh.Fin} (see \citealt{Gjersoe.2016.SP} for discussion).
Pitch contours in \textit{Decl.Fin} largely overlap with those in \textit{Wh.Fin}.

\begin{figure}
\begin{center}
\caption{T(L) contour tones realized as falling tones sentence-finally (\textit{Decl.Fin} and \textit{Wh.Fin}, left graph); the same tones showing a flat pitch trace in non-final position (\textit{Decl.Med}, right graph).}
\label{fig:gjersoe:level-contour}
\includegraphics[width=0.45\linewidth]{figures/Contour-Decl-Wh2-AllSpeak-MinBirdplus_rep_father_basket_p.png}
\includegraphics[width=0.45\linewidth]{figures/Contour-con-AllSpeak-MinBirdplus_rep_father_basket_p.png}
\end{center}
\end{figure}

% ----- %
\subsubsection{Level tones}\label{sec:gjersoe:2.2.2}

\figref{fig:Level-decl-wh2-con} shows F0 traces for the level tones L, M, and H in \textit{Decl.Fin}, \textit{Wh.Fin}, and \textit{Decl.Med}.
In sentence-medial position (right graph), the three level tones are realized with a stable flat contour.
% In the sentence-medial position, the L F0 traces are closer to the M \isi{tone} (about 1.5 St lower).
Sentence-finally (left graph), H and M are also flat.
The L \isi{tone}, however, shows a conspicuous falling contour extending to almost six semitones below the mid level of \textit{y{\=ɛ}}.
That the L \isi{tone} is realized as low-falling sentence-finally is a new observation that has not been noted in \citet{Fiore.1987} or \citet{Fransen.1995}.
As with contour tones, F0 movements in \textit{Decl.Fin} were not different from those in \textit{Wh.Fin}.

\begin{figure}
\begin{center}
\caption{T level tones in final (\textit{Decl.Fin} and \textit{Wh.Fin}, left graph) and sentence-medial (\textit{Decl.Med}, right graph) position.}
\label{fig:Level-decl-wh2-con}
% \noautomath %This code is necessary for the gb4e package
\includegraphics[width=0.45\linewidth]{figures/Level-Decl-Wh2-AllSpeak_p.png}
\includegraphics[width=0.45\linewidth]{figures/Level-con-AllSpeak_p.png}
\end{center}
\end{figure}

% ----- %
\subsubsection{Questions with the final particle \textit{a}}\label{sec:gjersoe:resultsparticle}

There are a number of striking differences between the two sentence types with the final question particle \textit{a}, i.e. between \textit{Wh.Prt} and \textit{Pol}.
The main difference is that F0 trends on the particle are generally low-falling in \textit{Wh.Prt} while F0 remains on the same level as that of a previous T \isi{tone} in \textit{Pol}.
Following a T(L) \isi{tone}, particles have a mid \isi{tone} in \textit{Pol}.
In other words, \textit{Wh.Prt} is very similar to \textit{Wh.Fin} and \textit{Decl.Fin} whereas \textit{Pol} more closely resembles \textit{Decl.Med}.

T(L) tones in \textit{Wh.Prt} (left graph of \figref{fig:contour-pol-whp}) reach a low target on the particle.
Level tones in \textit{Wh.Prt} (gray F0 traces in \figref{fig:level-pol-whp}) also reach a low target on the particle.
Note that both the T(L) and level tones show a small anticipatory fall from the nucleus midpoint before the low target in the particle.
The rising \isi{tone} LM has only a small-scale rise from its nucleus to the particle.
The flattened LM trace appears to be an effect of the L target of the following particle, conditioned by a tonal coarticulation effect which lowers the mid peak in the sequence LM.L.
This effect was weaker for the low-vowel item (see footnote \ref{footnote:gjersoe:microprosody}).
% Some of the contour tones of \textit{Wh.Prt} attested a greater range of F0 fall (above 2 ST) within the nuclei compared to \textit{Pol} (not exceeding 1 ST).

In polar questions, the particle has a mid \isi{tone} when it follows a T(L) toned word (right graph in \figref{fig:contour-pol-whp}).
However, F0 on the particle remains stable after a level \isi{tone}, continuing its low, mid, or high pitch level (black F0 traces in \ref{fig:level-pol-whp}).
F0 on the particle shows a small but insignificant rise after L, and the mid target of LM seems a little higher than that of T(L) tones. We will briefly consider explanations for these rises in \sectref{sec:Gjersoe:InterimSum}.
The divergent tonal behavior of polar and wh-questions is another new observation missing in previous descriptions of \ili{Limbum}.
A final point to note is that F0 values for HL and ML appear to converge in pre-particle position.
However, this convergence only seems to occur on low-vowel items (see footnote \ref{footnote:gjersoe:microprosody}).\footnote{At present, we cannot offer a convincing explanation why the M and H targets converge for some items in this context. We suspect that it is due to an independent process that does not interfere with the tonal alternations that we consider in this paper. Further studies are needed to scrutinize the conditioning factors and the productivity of this process.}

\begin{figure}[ht!]
\begin{center}
\caption{Words with a T(L) contour tone preceding a final particle in \textit{Wh.Prt} (left graph) and \textit{Pol} (right graph).}
\label{fig:contour-pol-whp}\includegraphics[width=0.45\linewidth]{figures/Contour-Wh1-AllSpeak-MinBirdplus_rep_father_basket_p.png}
\includegraphics[width=0.45\linewidth]{figures/Contour-pol-AllSpeak-MinBirdplus_rep_father_basket_p.png}
\end{center}
\end{figure}

\begin{figure}
\begin{center}
\caption{Words with a level tones preceding a final particle in \textit{Wh.Prt} (gray F0 traces) and \textit{Pol} (black F0 traces).}
\label{fig:level-pol-whp}
\includegraphics[width=0.60\linewidth]{figures/Level-wh1-pol-AllSpeak_p}
\end{center}
\end{figure}

\clearpage


% ----- %
\subsubsection{Duration}\label{sec:gjersoe:duration}

Vowels on our target words are generally longer sentence-finally (\textit{Decl.Fin} and \textit{Wh.Fin}) than in other contexts.
Duration differences are most prominent for alternating falling/non-falling tones, which are realized as TL sentence-finally and as T sentence-medially.
For instance, 'bridge', 'father' and 'cutlass' have a long vowel sentence-finally (\textit{r\`a\`a}, \textit{t\=a\`a}, and \textit{d\'a\`a}) but a short vowel sentence-medially (\textit{r\`a}, \textit{t\=a}, and \textit{d\'a}).
Level tones, in particular H, are also longer sentence-finally.
For example, 'hill' and 'bat' are pronounced as long \textit{b\'a\'a} and \textit{l\'e\'e} in \textit{Decl.Fin} and \textit{Wh.Fin} but as short \textit{b\'a} and \textit{l\'e} in \textit{Decl.Med} and \textit{Wh.Prt}. 
The rising contour \isi{tone} LM shows no durational differences across the different sentence types.
% This exceptional behavior is clearly attributable to the fact that the LM rising \isi{tone} pattern surfaces as LM in both medial and final positions of all tested sentence types, unlike the falling contour forms which have a TL-T alternation medially and finally.
Even though differences in vowel duration are attested in the recordings of all of our three speakers, there is a great deal of inter- and intra-speaker variation as to how big these length differences are, and failure to lengthen a \isi{final vowel} in \textit{Decl.Fin} and \textit{Wh.Fin} is not considered ungrammatical.
We therefore attribute the observed durational differences to an optional pre-boundary lengthening effect.


% ---- % 
\subsection{Interim summary}\label{sec:Gjersoe:InterimSum}
\tabref{tab:gjersoe:summarytonalchanges} summarizes the tonal alternations described in this section.
Low-falling contour tones (LL, ML, HL) only occur in phrase-final position (\textit{Decl.Fin} and \textit{Wh.Fin}).
Elsewhere, the fall to L is missing, and the first part of the contour is realized as a level \isi{tone}.
Non-low level tones are invariant in all contexts, while L is lowered phrase-finally.
The question particle \textit{a} receives a L \isi{tone} in \textit{Wh.Prt}, while in \textit{Pol}, it copies the \isi{tone} of a preceding level \isi{tone} but receives a M \isi{tone} when it follows a contour \isi{tone}.
L can thus be distinguished from L(L) only in \textit{Pol}.
LM is always realized as LM in all tested environments.

Our data also reveal a small number of minor phonetic effects.
First, the mid target in the sequence LM.L is not reached in \textit{Wh.Prt}.
We assume that this is a coarticulatory effect conditioned by the two L targets, one from the lexical \isi{tone} and other from the particle.
As mentioned earlier, this effect is stronger for the high-vowel item than the low-vowel item.
We do not have a straightforward explanation for the small rise on the particle in \textit{Pol} following L and LM.
For now, we do not consider this a relevant phonological process because the extra rise on L does not reach a M target and the extra rise on LM does not reach a H target. %One plausible hypothesis for this rise is be that the speakers implementing the M \isi{tone} of the particle with the M \isi{tone} of the LM target. This M merged with M could result in a phonetic F0 rise.

\begin{table}[h]
\centering
\caption{Surface tones across all tested sentence types}
\label{tab:gjersoe:summarytonalchanges}
\begin{tabular}{lllllllllll}\lsptoprule
                 &L  &M  &H  &L(L)&M(L)&H(L)&LM&\\\midrule
\textit{Decl.Fin}&LL &M  &H  &LL  &ML&HL&LM& \\
\textit{Decl.Med}&L  &M  &H  &L   &M&H&LM&\\
\textit{Wh.Fin}  &LL &M  &H  &LL  &ML&HL&LM&\\
\textit{Wh.Prt}  &L.L&M.L&H.L&L.L &M.L&H.L&LM&\\
\textit{Pol}     &L.L&M.M&H.H&L.M &M.M&H.M&LM&\\
\lspbottomrule\end{tabular}
\end{table}

% --------------------------------- %
\section{A formal account of tone-intonation interaction}\label{sec:gjersoe:ouranalysis}

In this section, we present our formal analysis of tonal alternations in \ili{Limbum}.
We assume that each of our test sentences constitutes an Intonational Phrase (IP).
The core idea of our analysis is that \ili{Limbum} has a low boundary \isi{tone} L\% at the \isi{right edge} of an IP in \textit{Decl.Fin}, \textit{Decl.Med}, \textit{Wh.Prt}, and \textit{Wh.Fin}, but not in \textit{Pol}.
We represent L\% as a floating \isi{register feature} \textit{l}.
Lowering of L, the falling/non-falling alternations, and the divergent tonal patterns on the particle \textit{a} in \textit{Wh.Prt} and \textit{Pol} all result from the presence (or absence) of \textit{l} and constraints governing if and how \textit{l} associates to tonal root nodes.

% --------------- %
\subsection{Theoretical background}\label{sec:gjersoe:theorybackground}

% ----- %
\subsubsection{Tonal root nodes and floating tonal features}\label{sec:gjersoe:defective}

The central idea of our analysis is that boundary tones and lexical tones are crucially represented by the same tonal features.
Adopting the idea of tonal decomposition and geometry \citep{Clements.1983,Hyman.1986,Snider.1999,Yip.2002},
we assume that tones -- much like segments -- can be decomposed into distinctive features.
Following Snider's (1999) Register Tier Theory (RTT), we distinguish four different tiers:
a \isi{register} tier (with \isi{register} features h and l), a tonal tier (with tonal features H and L), a \isi{tonal root node} (or~o) tier, and a TBU tier.
A \isi{register feature} specifies whether it is higher or lower compared to an adjacent \isi{register feature}, while a tonal feature specifies whether a \isi{tone} is high or low within a given \isi{register}. 
As shown in \figref{fig:gjersoe:SniderReps}, RTT thus allows to distinguish four pitch levels: High (H/h), Mid1 (H/l), Mid2 (L/h), and Low (L/l) \citep[62]{Snider.1999}.
Since there is only one mid pitch level in \ili{Limbum}, we represent M as L/h and assume that the combination H/l (Mid1) is not part of the tonal lexicon.

We represent contour tones as two o's linked to a single TBU (following \citealt{Fiore.1987} and \citealt{Fransen.1995}, we assume that the syllable  is the TBU in \ili{Limbum}).
While LM, the only rising \isi{tone} in \ili{Limbum}, is fully specified for both o's, low-falling contour tones have one fully specified and one empty \isi{tonal root node} underlyingly (see \figref{fig:gjersoe:levelandcontourtones}).
Basing our analysis within the broader framework of featural affixation \citep{Akinlabi.1996}, we represent the boundary \isi{tone} L\% as a floating low \isi{register feature} l.
This floating feature interacts with lexically underspecified (and in some cases also with fully specified) o's, most notably by creating low-falling contour tones.
\tabref{tab:gjersoe:InventoryReps} gives a summary of the tonal features of underlying tones in \ili{Limbum}.

\begin{figure}
\caption{Tonal geometry in RTT}
\label{fig:gjersoe:SniderReps}
\centering
\begin{tabular}{cccc}
\textsc{High} & \textsc{Mid1} & \textsc{Mid2} & \textsc{Low} \\
{\RepLevelHh} & {\RepLevelHl} & {\RepLevelLh} & {\RepLevelLl}
\end{tabular}
\end{figure}

\begin{figure}
\caption{Level and partially specified contour tones}
\label{fig:gjersoe:levelandcontourtones}
\centering
\begin{tabular}{cc}
\textsc{Level tones} & \textsc{Contour tones}\\
{\RepLevel} & \RepContour
\end{tabular}
\end{figure}

\begin{table}[h]
\caption{Underlying tone inventory}
\label{tab:gjersoe:InventoryReps}
\centering
\begin{tabular}{ccccccccccc}\lsptoprule
   & L & M & H & L(L) & M(L) & H(L) & LM & L\% \\\midrule % HM
\textsc{Tone} (\texttau)   &L&L&H&L~o&L~o&H~o&L~L& \\ % H--H
\textsc{Register} (\textrho) &l&h&h&l~o&h~o&h~o&l~h&l\\ % h--l
\lspbottomrule\end{tabular}
\end{table}

\noindent While equating phrasal tones with \isi{register} features might seem ad-hoc and unwarranted at first sight, there is a crucial parallel between the two: both can be understood as abstract phonetic targets relative to a previous target.
Boundary tones following a pitch accent of the same type have the effect of intensifying an already initiated downward or upward \isi{movement} \citep{Pierrehumbert.1980}, while a sequence of two low \isi{register} features is phonetically realized as further lowering in RTT.
Lexical \isi{tone} features show a strikingly different behavior from both \isi{register} features and boundary tones in this respect, as a sequence of three H-toned TBU's is not expected to show a rising contour under standard assumptions.
Instead, it would be more likely for pitch to steadily decrease due to downdrift, or for some of the H tones to undergo downstep.
For that reason, we believe that there is a natural ontological link between \isi{register} features and boundary tones, and we capture this connection by the simplest formal means, viz. an identical representation of the low \isi{register feature} l and the low boundary \isi{tone} L\%.

% ---- %
\subsubsection{Constraining tonal processes}

Having established the representations of lexical and phrasal tones, we now detail how the tonal alternations described in the previous sections are derived, using the general framework of Optimality Theory \citep{Prince.1993}.
While our analysis is in principle compatible with most versions of OT, we couch our analysis in \textit{Coloured Containment} \citep{Trommer.2015.Dinka,Zimmermann.2017.OUP}, which provides the means to accurately constrain association lines within and across phonological (sub-)structures.
\textit{Containment Theory} \citep{Oostendorp.2004} restricts the generative power of \textsc{Gen} to manipulating association lines between phonological nodes and inserting epenthetic nodes.
This means while \textsc{Gen} can add new lines and mark existing lines as invisible, it cannot delete any phonological material that is present in the input.
This vastly reduces the number of possible candidates that need to be evaluated compared to analyses of \isi{tone} in Correspondence Theory \citep{Zoll.2003,Zhang.2007}.

In our analysis, we do not invoke the powerful machinery of multi-level markedness in Containment. 
We employ Containment solely for its precise way to evaluate association relations between phonological nodes, as illustrated in very general terms in \REF{ex:gjersoe:containmentarrowsI} and \REF{ex:gjersoe:containmentarrowsII}.
For our analysis, the relevant nodes are the \isi{tonal root node} (o), \isi{register} features (l, h; \textrho), and tonal features (L, M, H; \texttau).
The constraint \textrho$\to$o, for instance, should be read as ``Count one violation for each \isi{register feature} not associated to a \isi{tonal root node}''.

\ea \begin{tabular}{rl}\IllustrationDown&Count one {\viol} for each {\textalpha} not associated to a \textbeta.\\\end{tabular} \label{ex:gjersoe:containmentarrowsI}
\z


\ea \begin{tabular}{rl}\IllustrationUp&Count one {\viol} for each {\textbeta} not associated to a \textalpha.\\\end{tabular} \label{ex:gjersoe:containmentarrowsII}
\z


\vs \noindent Two constraints corresponding to classical OT faithfulness constraints \textsc{Max} and \textsc{Dep} are given in \REF{ex:gjersoe:containmentarrowsIII} and \REF{ex:gjersoe:containmentarrowsIV}.
Note that \textsc{Ident} does not apply in Containment because nodes present in the input cannot be altered in any way.

\ea \begin{tabular}{rl}\multirow{2}{*}{\MaxAB}&Count one {\viol} for each deleted association line\\&between a node {\textalpha} and a node \textbeta.\\\end{tabular} \label{ex:gjersoe:containmentarrowsIII}
\z


\ea \begin{tabular}{rl}\multirow{2}{*}{\DepAB}&Count one {\viol} for each inserted association line\\&between a node {\textalpha} and a node \textbeta.\\\end{tabular} \label{ex:gjersoe:containmentarrowsIV}
\z


\vs \noindent Another crucial set of constraints is given in \REF{ex:gjersoe:fullspec} and \REF{ex:gjersoe:fullspecsyl}.
The first constraint militates against not fully specified tonal root nodes while the second constraint demands a TBU (the syllable) to be minimally specified for a \isi{tone} and a \isi{register feature}.
Note that these constraints are different from a conjunction of \textrho$\leftarrow$o and \texttau$\leftarrow$o: while such a local constraint conjunction would penalize only those root nodes (syllables) that are linked to exactly zero tonal and zero \isi{register} features, the constraints here demand full specification.
The last constraint that needs to be introduced here is *loh \REF{ex:gjersoe:loh}, which penalizes tonal root nodes associated to two non-identical \isi{register} features.

\ea \begin{tabular}{rl}\multirow{2}{*}{\RegToneByRt}&Count one {\viol} for each tonal root $R$ such that $R$ is not\\&associated to both a \isi{register feature} and a tonal feature.\\\end{tabular} \label{ex:gjersoe:fullspec}
\z


\ea \begin{tabular}{rl}\multirow{3}{*}{\RegToneBySyl}&Count one {\viol} for each syllable node $S$ such that $S$ is not\\&linked to both a \isi{register feature} and a tonal feature\\&by a path of association lines.\\\end{tabular} \label{ex:gjersoe:fullspecsyl}
\z


\ea \begin{tabular}{rl}*loh&Count one {\viol} for each tonal root linked to both l and h.\\\end{tabular} \label{ex:gjersoe:loh}
\z


\vs \noindent 
We adopt the theory of morphological colors \citep{Oostendorp.2006} to restrain access to morphological information by the phonological component.
The theory of morphological colors forbids morphological look-up but enables the phonology to distinguish between elements of different morphological provenience.
This will become relevant in the analysis of particle tones below.

A final assumption underlying our analysis is a stratal organization of grammar as it is modeled in Stratal OT \citep{Kiparsky.2000,Bermudez.2012}.
All evaluations relevant for the tonal processes in \ili{Limbum} that we are concerned with at this point take place at a postlexical level corresponding to the IP domain.
The input to this stratum is a sentence, with all words bearing their lexical (and, if applicable, morphological) tones, plus either L\% or no boundary \isi{tone} depending on sentence type.
%\footnote{However, no morphological tones have been attested in the language by previous studies and so far, we have not found any yet.}
We do not engage in further discussion on tonal processes at lower levels and only concern ourselves with the level of the IP, i.e. the level where L\% is introduced.

% ----- %
\subsection{Tonal hybridity and tone-intonation interaction}\label{sec:gjersoe:analysis}

Recall from the previous section that there are three classes of tones in \ili{Limbum}: level tones which remain level tones in all positions (L, M, H), level tones that alternate with falling contour tones at the end of declarative sentences and wh-questions (L(L), M(L), H(L)), and a rising contour \isi{tone} (LM).

We begin with our analysis of T(L) (= falling/non-falling alternating) tones.
These tones are equipped with a fully specified \isi{tonal root node} and an additional empty \isi{tonal root node}.
In the presence of L\%, i.e. a floating low \isi{register feature}, a line is inserted between the empty root node and the low \isi{register feature} and an epenthetic L \isi{tone} is inserted to make the o fully specified.
These processes are driven by three constraints: 
\texttau$\leftarrow$o$\to$\textrho{} militating against empty o's, \textsc{Alt(ernation)} penalizing insertion of lines between material of the same color, and \textsc{Dep}(H) prohibiting insertion of a H \isi{tone}.
The whole picture is given in the tableau in \tabref{tab:gjersoe:otfalling}.\footnote{Our analysis makes the prediction that if other boundary tones such as H\% exist in \ili{Limbum}, they should also interact with empty tonal root nodes. At present, we have not found any evidence of such boundary tones. Our impressionistic judgment of list intonation in \ili{Limbum} is that non-final items are marked by a toneless prosodic boundary much like in polar questions, and T(L) tones are realized non-falling accordingly.}
The faithful candidate in a. (which is also the input) violates \textrho$\to$o and crucially also \texttau$\leftarrow$o$\to$\textrho.
Candidate b. incurs violations of \textsc{Dep}(L) and \textsc{Dep}(Line) but is optimal compared to candidates c. (violation of \textsc{Dep}(H)) and d. (violation of \textsc{Alt}).
The winning candidate b. is a tonal hybrid: it combines lexical tonal features on its first o and both phrasal and epenthetic tonal features on its second o.
Note that in the case of LL, the optimal candidate has two identical tonal root nodes associated to the same TBU.
The fact that LL is realized as falling follows directly from RTT: the second l must be realized low relative to the first l.
% not considering Dep(tonalRt) here

\setlength{\tabcolsep}{5pt} % for all tableaux in this section

\begin{table}[h]
\caption{Combining L\% and underspecified tonal root nodes creates hybrid tones}
\label{tab:gjersoe:otfalling}
{\fns
\begin{tabular}{|ccc|c|c:c|c|c|c|c|}\hline
&\multicolumn{2}{l|}{Input = a.} & \Alt & \DepH & \NoMultDiff & \RegToneByRt & \DepL & \RegDomRt & \DepRegRt \\\hline\hline
	  & a. &\OTHLInput &&&&\viol!&&\viol& \\\hline
\hand & b. &\OTHLWinner &&&&&\viol&&\viol \\\hline
      & c. &\OTHLInsertH &&\viol!&&&&&\viol \\\hline
      & d. &\OTHLSpreadingHOnly &\viol!&&&&&&\viol \\\hline
%     & z.&\OTHLOverwriting &&&&&&&& \\\hline
%     & z.&\OTHLSpreading &&&&&&&& \\\hline
\end{tabular}}
\end{table}

In phrase-medial position, empty tonal root nodes remain empty.
The reason for this is the absence of a boundary \isi{tone} locally adjacent to the o phrase-medially.
The tableau in \tabref{tab:gjersoe:contourmedial} shows how \textsc{Alt} and \textsc{Dep}(\textrho) conspire to render the fully faithful candidate a., which violates the \isi{markedness constraint} against empty o's, optimal.
For the same reason, low-falling do not occur in polar questions \textit{Pol} and wh-questions with the particle \textit{a} \textit{Wh.Prt}:
in \textit{Pol}, no L\% is present, and in \textit{Wh.Prt}, the low \isi{register feature} associates to the particle and is not available to fill the empty root node of the lexical word (see below).\footnote{It was mentioned in footnote \ref{footnote:gjersoe:testsentences} that there is (at least) one lexical item with a HM \isi{tone} in \ili{Limbum}. Our informants confirm that for this word, HM patterns like H(L) in that it alternates with a level H \isi{tone} when not adjacent to L\%. While this does not seem to a be productive alternation, it is compatible with our account if we choose to represent the second o of HM as being specified for a H \isi{tone} and underspecified for a \isi{register feature}.} % (H/h--H)

\begin{table}[h]
\caption{No falling contour tones in the absence of L\%}
\label{tab:gjersoe:contourmedial}
{\fns
\begin{tabular}{|ccc|c|c|c|c|c|c|}\hline
&\multicolumn{2}{l|}{Input = a.} & \Alt & \DepLReg & \DepHReg & \RegToneByRt & \DepL & \RegDomRt \\\hline\hline
\hand & a. &\OTHnoLInput &&&&\viol&& \\\hline
      & b. &\OTHnoLSpreading &\viol!&&&&& \\\hline
      & c. &\OTHnoLEpenth &&&\viol!&&\viol& \\\hline
\end{tabular}}
\end{table}

\noindent
The only rising contour \isi{tone} in \ili{Limbum}, LM, is unaffected by boundary tones.\footnote{See \sectref{sec:Gjersoe:InterimSum} for a discussion of incomplete plateauing of LM before L\% in \textit{Wh.Prt}.}
%  where LM is followed by a particle that acquires a low \isi{tone} from L\%: LML $\to$ LLL
The interplay of three constraints is responsible for the immunity of fully specified contour tones against {overwriting by} floating \isi{register} features: a high-ranked \textsc{Max} constraint against overwriting of \isi{register} features, a \textsc{Dep}(o) constraint penalizing insertion of tonal root nodes, and the \isi{markedness constraint} *loh.
The tableau in \tabref{tab:gjersoe:lm} shows how the fully faithful candidate a. is chosen as optimal.

\begin{table}[h]
\caption{Full specification as a protective shield: LM in the presence of L\%}
\label{tab:gjersoe:lm}
{\fns
\begin{tabular}{|ccc|c:c:c|c|}\hline
&\multicolumn{2}{l|}{Input = a.} & \DepTonalRt & \MaxRegRt & \NoMultDiff &  \RegDomRt \\\hline\hline
\hand& a. &\OTLMInput &&&&\viol \\\hline
     & b. &\OTLMReplace &&\viol!&& \\\hline
     & c. &\OTLMTwoReg &&&\viol!& \\\hline
\end{tabular}}
\end{table}

We now turn to the discussion of level tones.
One of the striking arguments in favor of an analysis with L\% as opposed to a phrase-medial contour simplification rule \citep{Fransen.1995} is the observation that L is realized as LL in \textit{Decl} and \textit{Wh}.
These are the exact same environments for which we independently assume a L\% based on the behavior of T(L) tones.
The fact that L tones are further lowered in these environments is strong evidence for the presence of L\%, and the fact that M and H tones are not affected by it follows directly from the constraint *loh.
The tableau in \tabref{tab:gjersoe:lowgetslower} illustrates this process.
Candidate b. is a hybrid that hosts two l features of different affiliation under the same o, satisfying *loh.
In RTT, this configuration is equivalent to that of a low-falling LL \isi{tone} spread over two tonal root nodes.
M and H level tones, however, have a h \isi{register feature} that blocks association of a floating l feature.
The immunity of M and H thus follows from the same set of constraints as the immunity of LM discussed above.
% strong argument for our approach !

\begin{table}[h]
\caption{L\% affects L but not M}
\label{tab:gjersoe:lowgetslower}
{\fns
\begin{tabular}{|ccc|c:c:c|c|c|}\hline
&\multicolumn{2}{l|}{Input = a./a'.} & \DepTonalRt & \MaxRegRt & \NoMultDiff & \RegDomRt &  \DepRegRt \\\hline\hline
      & a. & \OTLInput &&&&\viol!& \\\hline
\hand & b. & \OTLLowered &&&&&\viol \\\hline\hline
\hand & a'. & \OTMInput &&&&\viol&\\\hline
      & b'. & \OTMLowered &&&\viol!&&\viol\\\hline
\end{tabular}}
\end{table}

We now turn to polar and wh-questions with the final particle \textit{a}.
Polar questions are marked with a toneless particle \textit{a} but lack the L\% boundary \isi{tone}.
Recall from \sectref{sec:gjersoe:resultsparticle} that the flat contour of level tones extends to the particle \textit{a} while the particle receives a M \isi{tone} following T(L) toned words.
As shown in the tableaux in \tabref{tab:gjersoe:particlespreading} and \tabref{tab:gjersoe:particledefault}, the tonal features of an underlying level \isi{tone} with a single, fully specified o can spread to the tonally unspecified particle because this does not violate \textsc{NoSkip} or \textsc{Alt}.
When there is a o intervening between the fully specified root node and the \isi{tonal root node} on the particle, spreading with skipping is ruled out by \textsc{NoSkip} and across-the-board spreading is ruled out by \textsc{Alt}.
Therefore, the optimal repair for the toneless TBU is insertion of a default M \isi{tone}.
Empty o's remain empty in T(L) tones in \textit{Decl.Med} because the \textsc{Dep} constraints penalizing M \isi{tone} insertion outrank \texttau$\leftarrow$o$\to$\textrho.
Leaving the particle o empty, however, would fatally violate high-ranked \texttau$\leftarrow$\textsigma$\to$\textrho.

\begin{table}[h]
\caption{Spreading of a level tone in the absence of L\%}
\label{tab:gjersoe:particlespreading} 
{\fns
\begin{tabular}{|ccc|c:c|c:c|c|c|c|c|c|}\hline
&\multicolumn{2}{l|}{Input = a.} & \NoSkip & \Alt & \RegToneBySyl & \DepH & \DepLReg & \DepHReg & \RegToneByRt & \DepL & \DepRegRt \\\hline\hline
     &a.&\OTLPolIn&&&\viol!&&&&\viol&& \\\hline
\hand&b.&\OTLPolAlt&&&&&&&&&\viol \\\hline
     &c.&\OTLPolDef&&&&&&\viol!&&\viol& \\\hline
\end{tabular}}
\end{table}

\begin{table}[h]
\caption{Default M insertion in the absence of L\%}
\label{tab:gjersoe:particledefault} 
{\fns
\begin{tabular}{|ccc|c:c|c:c|c|c|c|c|c|}\hline
&\multicolumn{2}{l|}{Input = a.} & \NoSkip & \Alt & \RegToneBySyl & \DepH & \DepLReg & \DepHReg & \RegToneByRt & \DepL & \DepRegRt \\\hline\hline
     &a.&\OTLLPolIn&&&\viol!&&&&\viol\viol&&\\\hline
     &b.&\OTLLPolSkip&\viol!&&&&&&\viol&&\viol\\\hline
\hand&c.&\OTLLPolDef&&&&&&\viol&\viol&\viol&\\\hline
\end{tabular}}
\end{table}

\setlength{\tabcolsep}{6pt} % back to normal

In \textit{Wh.Prt}, the particle receives a low \isi{tone} and the preceding T(L) tones are realized as level tones .
This pattern follows assuming there is an additional \isi{markedness constraint} *\textrho\textsuperscript{2o}: ``Count one {\viol} for each \isi{register feature} associated to more than one o'', ranked below \textsc{Dep}(h) but outranking \texttau$\leftarrow$o$\to$\textrho.
The floating l links to the particle o (because of \textsigma$\to$\textrho) but not to the other empty \isi{tonal root node} due to *\textrho\textsuperscript{2o}.
In other words, it is better to use the floating l to fill one unspecified syllable but leave a o on a specified syllable empty than to violate *\textrho\textsuperscript{2o} and fill every o.
Spreading of level tones to the particle in polar questions (tableau in \tabref{tab:gjersoe:particlespreading}) is not affected by *\textrho\textsuperscript{2o} because it is ranked below \textsc{Dep}(h), leaving the potential repair of \texttau$\leftarrow$\textsigma$\to$\textrho{} by epenthesis suboptimal.

\clearpage

% --------------------------------- %
\section{Discussion}\label{sec:gjersoe:4}

In this section, we briefly consider three potential alternative analyses and discuss some typological implications of our own account.

% ----- %

\subsection{Alternative: Contour simplification}

Our analysis differs substantially from the rule-based account in \citet{Fransen.1995}.
Fransen proposes an analysis in which T(L) tones are fully specified as LL, ML, and HL underlyingly.
They are then \isi{subject} to a \isi{tone} sandhi rule, TL~$\to$~T, which applies in all environments except before a pause.
This means that contour tones always surface faithfully phrase-finally in all sentence types.
The \isi{tone} sandhi rule seem rather arbitrary, and it seems like a mere stipulation that the rising \isi{tone} LM is not \isi{subject} to simplification.
An even more severe drawback of Fransen's sandhi analysis is that it fails to predict the lowering of L.
On our account, the fact that L becomes LL in exactly the same environments in which T(L) are realized as TL follows from the presence of L\%.
Also, our account does not need to stipulate an exception to contour simplification for LM because its immunity follows directly from its full specification and high-ranked \textsc{Max}(Line) constraints. 
% Our analysis is therefore able to derive the lowering of L to LL in \textit{Decl} and \textit{Wh} and explains why this lowering does not affect H and M tones.
% her rule would apply so much that the frequent absence of a contour would raise the question of learnability


% ----- %
\subsection{Alternative: Moras}

Another possible approach would be an account on which the mora is the TBU.
Throughout the paper, we have followed \citet{Fiore.1987} and \citet{Fransen.1995} in assuming the syllable to be the TBU in \ili{Limbum}.
Since in \sectref{sec:gjersoe:duration} we reported that T(L) tones are longer phrase-finally than phrase-medially, it seems appropriate that we defend our decision to ignore the mora in our analysis.
First, our informants rejected all minimal pairs that were put forward to support a phonemic opposition of long vs. short vowels in \citet{Fiore.1987} and \citet{Fransen.1995}. % Fiore:48f,Fransen:39f
This could be due to dialectal differences between the Southern (Ndu) dialect described by Fiore and Fransen and the Central (Warr) dialect of our speakers.
%y\red{we say in section 2 that our speakers speak the ``Warr'' dialect}
We therefore conclude that there is no independent reason to assume a moraic level of representation in the variety of \ili{Limbum} discussed here.
Second, in order to account for the shortness of medial T(L) tones, a moraic analysis would have to argue that a prosodically fully integrated mora is only realized when it is also tonally specified.
This would require a rather unusual definition of structure integration and is at odds with standard assumptions about moras and \isi{prosodic structure} \citep{Hyman.1985,Hayes.1989,Davis.2011.HANDBOOK,Davis.2011.COMPANION,Zimmermann.2017.OUP}.
Third, phrase-final lengthening also applies to level tones, especially to H.
This shows that there is no 1:1~relationship between contour tones and \isi{vowel length}.  
Fourth, there seems to be a great deal of inter-speaker variation in how prominent the length differences are. %, as they were considerably lower for F\_1 than for M\_1.
It is therefore safe to assume that the emergence of \isi{vowel length} is best ascribed to boundary effects and accommodation to contour tones and needs not be reflected on an abstract phonological level.

% ----- %
\subsection{Alternative: Cophonologies}

Another possible approach to the data discussed here would be to adopt cophonologies \citep{Orgun.1996,InkelasandZoll.2007,Sande.2017}.
A cophonology approach to \ili{Limbum} \isi{tone} would assume that certain sentence types have their own grammar, each giving rise to a specific \isi{tone} pattern.
A cophonology analysis does not need resort to tonal decomposition, floating features, or assumptions about morphological colors.
Rather, it would have to stipulate specific (sub-)rankings for declarative sentences, wh-questions, and polar questions.
While such an approach might be technically feasible, we believe it would have a number of disadvantages over our unified item-based account as it would miss crucial generalizations about the data.
For instance, the asymmetry between alternating T(L) tones and non-alternating LM persists through all sentence types.
Also, under a cophonology account it would be entirely accidental that M and LM are both unaffected by L\%-induced lowering regardless of the sentential context.


% ----- %
\subsection{Typological considerations}
The interaction between lexical tones and intonation is a topic that has recently attracted growing attention by scholars \citep{Hyman.2011,Gussenhoven.2014.WorldEnglishes,IntonationAfric}.
% It is often held that the interaction between lexical and phrasal tones is governed by principles ensuring that phrasal tones are protected whenever a conflict between lexical and phrasal tones arises. % WHO SAYS THAT? Ladd p.31ff? Not clear!
In \ili{Limbum}, \textrho$\to$o is ranked relatively low which has the effect that the boundary \isi{tone} L\% fails to be realized in some cases (in particular following non-low level tones and LM).
\ili{Limbum} can therefore be characterized as an instance of \textit{incomplete avoidance} according to Hyman's (2011) typology:
\textit{avoidance} because lexical M and H block L\% from surfacing, \textit{incomplete} because L and toneless root nodes do allow it to surface.
% other categories being \textit{accommodation} (peaceful coexististence between lexical tones and intones) and submission (intones override lexical/grammatical tones)
It is also interesting to note that in \ili{Limbum}, boundary tones affect only final syllables, as opposed to other languages where sequences of more than one syllable are affected (see \citealt{KulaHaman.2017}).

From a functional point of view, it is not surprising that \ili{Limbum} makes use of intonational means to distinguish declarative sentences from polar questions,
and neither is it unusual that wh-questions pattern differently from polar questions (see e.g. the surveys in \citealt{Chisholm.1984} and \citealt{Jun.2005}). %  and \citealt{Jun.2014}
Curiously, the two wh-question constructions in \ili{Limbum} differ in the presence of the final particle \textit{a} but not in their tonal make-up.
A promising road for future research would be to investigate whether lexical optionality is more generally associated with prosodic uniformity, and if the opposite relation holds as well.


% --------------------------------- %
\section{Conclusion}\label{sec:gjersoe:5}

In this paper, we have argued that \ili{Limbum} has a  low boundary \isi{tone} L\% in declaratives and wh-questions but not in polar questions based on an acoustic study with three native speakers.
We have shown how tonal alternations, both across lexical items and across sentence types, follow from basic assumptions about tonal geometry and from the distinction between fully specified and empty tonal root nodes.
By representing L\% as a low \isi{register feature},  we have proposed a uniform way to model tonal alternations at phrasal edge positions.
On our account, tonal hybridity follows straightforwardly from autosegmental linking of phrasal tonal features to lexical tonal root nodes.
\ili{Limbum} thus illustrates the benefits of \isi{register} features and empty phonological representations, and provides justification for the use of geometry-oriented constraints for analyzing tone-intonation interactions.
% Underspecified tones surface in different guises depending on their phrasal tonal environment.


% ----- %
\section*{\textbf{Acknowledgments}}

This research was funded by the DFG-funded research training group \textit{Interaction of Grammatical Building Blocks} (IGRA) at Leipzig University. We thank Lukas Urmoneit and Soeren E. Worbs for assisting with recording and annotating the data. We express our gratitude to Jochen Trommer as well as to the audiences of ACAL~47 and SpeechProsody~8 for their helpful feedback and comments.
% We are indebted to Bert Remijsen for providing the Praat scripts for the overlaid time-normalized \isi{f0} tracks

\sloppy
\printbibliography[heading=subbibliography,notkeyword=this]

%\end{document}


% coalescence %
% When the particle follows an \textit{a}-final word, the two vowels coalesce, yielding a long [a:] with the tonal geometries from both the content word and the particle associated to the TBU.
% This always results in a falling contour in \textit{Wh} even if the lexical \isi{tone} of the content word is a level \isi{tone} because the coalesced vowel will then have an additional o to which l can associate.