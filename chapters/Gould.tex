\documentclass[output=paper,modfonts,nonflat,
colorlinks, citecolor=brown,
% draftmode,
% draft
%  hidelinks
]{langsci/langscibook} 

\title{On the derivation of Swahili \textit{amba} relative clauses: Evidence for movement}

\author{Isaac Gould\affiliation{Ewha Womans University}\lastand Tessa Scott \affiliation{University of California, Berkeley}}

\abstract{
This paper brings together two disparate strands of research in the literature on relative clauses (RCs) in Swahili. Our focus is to provide a unified analysis of various data involving a particular kind of head-external RC, namely \textit{amba}-RCs. Our interest is in whether these RCs involve movement of the head from inside the RC to its external position (i.e.\ head raising). To investigate this, we look at scope interactions between a quantified RC-head and some other quantifier. We propose a diagnostic test using constraints on long-distance QR (LDQR) from \citet{Fox2000} to provide evidence for the following claims: \textit{amba}-RCs involve head raising, and \textit{amba}-RCs are not islands for overt syntactic movement.
}

%\keywords{Swahili relative clauses, relative clause islands, scope economy, long-distance Quantifier Raising}

 
\IfFileExists{../localcommands.tex}{%hack to check whether this is being compiled as part of a collection or standalone
  \usepackage{pifont}
\usepackage{savesym}

\savesymbol{downingtriple}
\savesymbol{downingdouble}
\savesymbol{downingquad}
\savesymbol{downingquint}
\savesymbol{suph}
\savesymbol{supj}
\savesymbol{supw}
\savesymbol{sups}
\savesymbol{ts}
\savesymbol{tS}
\savesymbol{devi}
\savesymbol{devu}
\savesymbol{devy}
\savesymbol{deva}
\savesymbol{N}
\savesymbol{Z}
\savesymbol{circled}
\savesymbol{sem}
\savesymbol{row}
\savesymbol{tipa}
\savesymbol{tableauxcounter}
\savesymbol{tabhead}
\savesymbol{inp}
\savesymbol{inpno}
\savesymbol{g}
\savesymbol{hanl}
\savesymbol{hanr}
\savesymbol{kuku}
\savesymbol{ip}
\savesymbol{lipm}
\savesymbol{ripm}
\savesymbol{lipn}
\savesymbol{ripn} 
% \usepackage{amsmath} 
% \usepackage{multicol}
\usepackage{qtree} 
\usepackage{tikz-qtree,tikz-qtree-compat}
% \usepackage{tikz}
\usepackage{upgreek}


%%%%%%%%%%%%%%%%%%%%%%%%%%%%%%%%%%%%%%%%%%%%%%%%%%%%
%%%                                              %%%
%%%           Examples                           %%%
%%%                                              %%%
%%%%%%%%%%%%%%%%%%%%%%%%%%%%%%%%%%%%%%%%%%%%%%%%%%%%
% remove the percentage signs in the following lines
% if your book makes use of linguistic examples
\usepackage{tipa}  
\usepackage{pstricks,pst-xkey,pst-asr}

%for sande et al
\usepackage{pst-jtree}
\usepackage{pst-node}
%\usepackage{savesym}


% \usepackage{subcaption}
\usepackage{multirow}  
\usepackage{./langsci/styles/langsci-optional} 
\usepackage{./langsci/styles/langsci-lgr} 
\usepackage{./langsci/styles/langsci-glyphs} 
\usepackage[normalem]{ulem}
%% if you want the source line of examples to be in italics, uncomment the following line
% \def\exfont{\it}
\usetikzlibrary{arrows.meta,topaths,trees}
\usepackage[linguistics]{forest}
\forestset{
	fairly nice empty nodes/.style={
		delay={where content={}{shape=coordinate,for parent={
					for children={anchor=north}}}{}}
}}
\usepackage{soul}
\usepackage{arydshln}
% \usepackage{subfloat}
\usepackage{langsci/styles/langsci-gb4e} 
   
% \usepackage{linguex}
\usepackage{vowel}

\usepackage{pifont}% http://ctan.org/pkg/pifont
\newcommand{\cmark}{\ding{51}}%
\newcommand{\xmark}{\ding{55}}%
 
 
 %Lamont
 \makeatletter
\g@addto@macro\@floatboxreset\centering
\makeatother

\usepackage{newfloat} 
\DeclareFloatingEnvironment[fileext=tbx,name=Tableau]{tableau}
  %add all your local new commands to this file
\newcommand{\downingquad}[4]{\parbox{2.5cm}{#1}\parbox{3.5cm}{#2}\parbox{2.5cm}{#3}\parbox{3.5cm}{#4}}
\newcommand{\downingtriple}[3]{\parbox{4.5cm}{#1}\parbox{3cm}{#2}\parbox{3cm}{#3}}
\newcommand{\downingdouble}[2]{\parbox{4.5cm}{#1}\parbox{6cm}{#2}}
\newcommand{\downingquint}[5]{\parbox{1.75cm}{#1}\parbox{2.25cm}{#2}\parbox{2cm}{#3}\parbox{3cm}{#4}\parbox{2cm}{#5}}
\newcolumntype{Y}{>{\centering\arraybackslash}X}
\newcolumntype{T}{>{\centering\arraybackslash}m{2cm}}

%commands for Kusmer paper below
\newcommand{\ip}{$\upiota$}
\newcommand{\lipm}{(\_{\ip-Max}}
\newcommand{\ripm}{)\_{\ip-Max}}
\newcommand{\lipn}{(\_{\ip}}
\newcommand{\ripn}{)\_{\ip}}
\renewcommand{\_}[1]{\textsubscript{#1}}


%commands for Pillion paper below
\newcommand{\suph}{\textipa{\super h}}
\newcommand{\supj}{\textipa{\super j}}
\newcommand{\supw}{\textipa{\super w}}
\newcommand{\ts}{\textipa{\t{ts}}}
\newcommand{\tS}{\textipa{\t{tS}}}
\newcommand{\devi}{\textipa{\r*i}}
\newcommand{\devu}{\textipa{\r*u}}
\newcommand{\devy}{\textipa{\r*y}}
\newcommand{\deva}{\textipa{\r*a}}
\renewcommand{\N}{\textipa{N}}
\newcommand{\Z}{\textipa{Z}}
% 

%commands for Diercks paper below
\newcommand{\circled}[1]{\begin{tikzpicture}[baseline=(word.base)]
\node[draw, rounded corners, text height=8pt, text depth=2pt, inner sep=2pt, outer sep=0pt, use as bounding box] (word) {#1};
\end{tikzpicture}
}

%commands for Pesetsky paper below
% \newcommand{\sem}[2][]{\mbox{$[\![ $\textbf{#2}$ ]\!]^{#1}$}}
\newcommand{\sem}[2][]{\mbox{$[[ $\textbf{#2}$ ]]^{#1}$}}

% \newcommand{\ripn}{{\color{red}ripn}}%this is used but never defined. Please update the definition



%commands for Lamont paper below
\newcommand{\row}[4]{
	#1. & 
    /{#2}/ & 
    [{#3}] & 
    `#4' \\ 
}
%\newcounter{tableauxcounter}
\newcommand{\tabhead}[2]{
%     \captionsetup{labelformat=empty}
%     \stepcounter{tableauxcounter}
%     \addtocounter{table}{-1}
% 	\centering
% 	\caption{Tableau \thetableauxcounter: #1}
	\caption{#1}
	\label{#2}
}
\newcommand{\candref}[2]{{(\ref{#1}#2)}}
\newcommand{\tableauref}[1]{{Tableau~\ref{#1}}}
% tableaux
\newcommand{\inp}[1]{\multicolumn{2}{|l||}{{#1}}}
\newcommand{\inpno}[1]{\multicolumn{2}{|l||}{#1}}
\newcommand{\g}{\cellcolor{lightgray}}
\newcommand{\hanl}{\HandLeft}
\newcommand{\hanr}{\HandRight}
\newcommand{\kuku}{Kuk\'{u}}

% \newcommand{\nocaption}[1]{{\color{red} Please provide a caption}}

% \providecommand{\biberror}[1]{{\color{red}#1}}

\definecolor{RED}{cmyk}{0.05,1,0.8,0}


\newfontfamily\amharicfont[Script = Ethiopic, Scale = 1.0]{AbyssinicaSIL}
\newcommand{\amh}[1]{{\amharicfont #1}}

% 
% %Gjersoe
\usepackage{textgreek}
% 
\newcommand{\viol}{\fontfamily{MinionPro-OsF}\selectfont\rotatebox{60}{$\star$}}
\newcommand{\myscalex}{0.45}
\newcommand{\myscaley}{0.65}
%\newcommand{\red}[1]{\textcolor{red}{#1}}
%\newcommand{\blue}[1]{\textcolor{blue}{#1}}
\newcommand{\epen}[1]{\colorbox{jgray}{#1}}
\newcommand{\hand}{{\normalsize \ding{43}}}
\definecolor{jgray}{gray}{0.8} 
\usetikzlibrary{positioning}
\usetikzlibrary{matrix}
\newcommand{\mora}{\textmu\xspace}
\newcommand{\si}{\textsigma\xspace}
\newcommand{\ft}{\textPhi\xspace}
\newcommand{\tone}{\texttau\xspace}
\newcommand{\word}{\textomega\xspace}
% \newcommand{\ts}{\texttslig}
\newcommand{\fns}{\footnotesize}
\newcommand{\ns}{\normalsize}
\newcommand{\vs}{\vspace{1em}}
\newcommand{\bs}{\textbackslash}   % backslash
\newcommand{\cmd}[1]{{\bf \color{red}#1}}   % highlights command
\newcommand{\scell}[2][l]{\begin{tabular}[#1]{@{}c@{}}#2\end{tabular}}
% \interfootnotelinepenalty=10000

% --- Snider Representations --- %

\newcommand{\RepLevelHh}{
\begin{minipage}{0.10\textwidth}
\begin{tikzpicture}[xscale=\myscalex,yscale=\myscaley]
%\node (syl) at (0,0) {Hi};
\node (Rt) at (0,1) {o};
\node (H) at (-0.5,2) {H};
\node (R) at (0.5,3) {h};
%\draw [thick] (syl.north) -- (Rt.south) ;
\draw [thick] (Rt.north) -- (H.south) ;
\draw [thick] (Rt.north) -- (R.south) ;
\end{tikzpicture}
\end{minipage}
}

\newcommand{\RepLevelLh}{
\begin{minipage}{0.10\textwidth}
\begin{tikzpicture}[xscale=\myscalex,yscale=\myscaley]
%\node (syl) at (0,0) {Mid2};
\node (Rt) at (0,1) {o};
\node (H) at (-0.5,2) {L};
\node (R) at (0.5,3) {h};
%\draw [thick] (syl.north) -- (Rt.south) ;
\draw [thick] (Rt.north) -- (H.south) ;
\draw [thick] (Rt.north) -- (R.south) ;
\end{tikzpicture}
\end{minipage}
}

\newcommand{\RepLevelHl}{
\begin{minipage}{0.10\textwidth}
\begin{tikzpicture}[xscale=\myscalex,yscale=\myscaley]
%\node (syl) at (0,0) {Mid1};
\node (Rt) at (0,1) {o};
\node (H) at (-0.5,2) {H};
\node (R) at (0.5,3) {l};
%\draw [thick] (syl.north) -- (Rt.south) ;
\draw [thick] (Rt.north) -- (H.south) ;
\draw [thick] (Rt.north) -- (R.south) ;
\end{tikzpicture}
\end{minipage}
}

\newcommand{\RepLevelLl}{
\begin{minipage}{0.10\textwidth}
\begin{tikzpicture}[xscale=\myscalex,yscale=\myscaley]
%\node (syl) at (0,0) {Lo};
\node (Rt) at (0,1) {o};
\node (H) at (-0.5,2) {L};
\node (R) at (0.5,3) {l};
%\draw [thick] (syl.north) -- (Rt.south) ;
\draw [thick] (Rt.north) -- (H.south) ;
\draw [thick] (Rt.north) -- (R.south) ;
\end{tikzpicture}
\end{minipage}
}

% --- Representations --- %

\newcommand{\RepLevel}{
\begin{minipage}{0.10\textwidth}
\begin{tikzpicture}[xscale=\myscalex,yscale=\myscaley]
\node (syl) at (0,0) {\textsigma};
\node (Rt) at (0,1) {o};
\node (H) at (-0.5,2) {\texttau};
\node (R) at (0.5,3) {\textrho};
\draw [thick] (syl.north) -- (Rt.south) ;
\draw [thick] (Rt.north) -- (H.south) ;
\draw [thick] (Rt.north) -- (R.south) ;
\end{tikzpicture}
\end{minipage}
}

\newcommand{\RepContour}{
\begin{minipage}{0.10\textwidth}
\begin{tikzpicture}[xscale=\myscalex,yscale=\myscaley]
\node (syl) at (0,0) {\textsigma};
\node (Rt) at (0,1) {o};
\node (H) at (-0.5,2) {\texttau};
\node (R) at (0.5,3) {\textrho};
\node (Rt2) at (1.5,1.0) {o};
%\node (H2) at (1.0,2) {$\tau$};
%\node (R2) at (2.0,2.5) {R};
\draw [thick] (syl.north) -- (Rt.south) ;
\draw [thick] (Rt.north) -- (H.south) ;
\draw [thick] (Rt.north) -- (R.south) ;
\draw [thick] (syl.north) -- (Rt2.south) ;
%\draw [thick] (Rt2.north) -- (H2.south) ;
%\draw [thick] (Rt2.north) -- (R2.south) ;
\end{tikzpicture}
\end{minipage}
}


% --- OT constraints --- %

\newcommand{\IllustrationDown}{
\begin{minipage}{0.09\textwidth}
\begin{tikzpicture}[xscale=0.7,yscale=0.45]
\node (reg) at (0,0.75) {{\small \textalpha}};
\node (arrow) at (0,0) {{\fns $\downarrow$}};
\node (Rt) at (0,-0.75) {{\small \textbeta}};
\end{tikzpicture}
\end{minipage}
}

\newcommand{\IllustrationUp}{
\begin{minipage}{0.09\textwidth}
\begin{tikzpicture}[xscale=0.7,yscale=0.45]
\node (reg) at (0,0.75) {{\small \textalpha}};
\node (arrow) at (0,0) {{\fns $\uparrow$}};
\node (Rt) at (0,-0.75) {{\small \textbeta}};
\end{tikzpicture}
\end{minipage}
}

\newcommand{\MaxAB}{
\begin{minipage}{0.09\textwidth}
\begin{tikzpicture}[xscale=0.6,yscale=0.4]
\node (max) at (0,0) {{\small \textsc{Max}}};
\node (reg) at (0.75,0.5) {{\fns \textalpha}};
\node (arrow) at (0.75,0) {{\tiny $\downarrow$}};
\node (Rt) at (0.75,-0.5) {{\fns \textbeta}};
\end{tikzpicture}
\end{minipage}
}

\newcommand{\DepAB}{
\begin{minipage}{0.09\textwidth}
\begin{tikzpicture}[xscale=0.6,yscale=0.4]
\node (max) at (0,0) {{\small \textsc{Dep}}};
\node (reg) at (0.75,0.5) {{\fns \textalpha}};
\node (arrow) at (0.75,0) {{\tiny $\downarrow$}};
\node (Rt) at (0.75,-0.5) {{\fns \textbeta}};
\end{tikzpicture}
\end{minipage}
}

\newcommand{\DepHReg}{
\begin{minipage}{0.055\textwidth}
\begin{tikzpicture}[xscale=0.6,yscale=0.4]
\node (dep) at (0,0) {{\small \textsc{Dep}}};
\node (reg) at (0,-1.0) {{\small h}};
\end{tikzpicture}
\end{minipage}
}

\newcommand{\DepLReg}{
\begin{minipage}{0.055\textwidth}
\begin{tikzpicture}[xscale=0.6,yscale=0.4]
\node (dep) at (0,0) {{\small \textsc{Dep}}};
\node (reg) at (0,-1.0) {{\small l}};
\end{tikzpicture}
\end{minipage}
}

\newcommand{\DepReg}{
\begin{minipage}{0.055\textwidth}
\begin{tikzpicture}[xscale=0.6,yscale=0.4]
\node (dep) at (0,0) {{\small \textsc{Dep}}};
\node (reg) at (0,-1.0) {{\small \textrho}};
\end{tikzpicture}
\end{minipage}
}

\newcommand{\DepTRt}{
\begin{minipage}{0.1\textwidth}
\begin{tikzpicture}[xscale=0.6,yscale=0.4]
\node (dep) at (0,0) {{\small \textsc{Dep}}};
\node (t) at (0.75,0.5) {{\fns \texttau}};
\node (arrow) at (0.75,0) {{\tiny $\downarrow$}};
\node (Rt) at (0.75,-0.5) {{\fns o}};
\end{tikzpicture}
\end{minipage}
}

\newcommand{\MaxRegRt}{
\begin{minipage}{0.1\textwidth}
\begin{tikzpicture}[xscale=0.6,yscale=0.4]
\node (max) at (0,0) {{\small \textsc{Max}}};
\node (arrow) at (0.75,0) {{\tiny $\downarrow$}};
\node (Rt) at (0.75,-0.5) {{\fns o}};
\node (reg) at (0.75,0.5) {{\fns \textrho}};
\end{tikzpicture}
\end{minipage}
}

\newcommand{\RegToneByRt}{
\begin{minipage}{0.06\textwidth}
\begin{tikzpicture}[xscale=0.6,yscale=0.5]
\node[rotate=20] (arrow1) at (-0.15,0) {{\fns $\uparrow$}};
\node[rotate=340] (arrow2) at (0.15,0) {{\fns $\uparrow$}};
\node (Rt) at (0,-0.55) {{\small o}};
\node (reg) at (0.4,0.55) {{\small \textrho}};
\node (tone) at (-0.4,0.55) {{\small \texttau}};
\end{tikzpicture}
\end{minipage}
}

\newcommand{\RegToneBySyl}{
\begin{minipage}{0.06\textwidth}
\begin{tikzpicture}[xscale=0.6,yscale=0.5]
\node[rotate=20] (arrow1) at (-0.15,0) {{\fns $\uparrow$}};
\node[rotate=340] (arrow2) at (0.15,0) {{\fns $\uparrow$}};
\node (Rt) at (0,-0.55) {{\small \textsigma}};
\node (reg) at (0.4,0.55) {{\small \textrho}};
\node (tone) at (-0.4,0.55) {{\small \texttau}};
\end{tikzpicture}
\end{minipage}
}

\newcommand{\DepTone}{
\begin{minipage}{0.055\textwidth}
\begin{tikzpicture}[xscale=0.6,yscale=0.4]
\node (dep) at (0,0) {{\small \textsc{Dep}}};
\node (tone) at (0,-1.0) {{\small \texttau}};
\end{tikzpicture}
\end{minipage}
}

\newcommand{\DepTonalRt}{
\begin{minipage}{0.055\textwidth}
\begin{tikzpicture}[xscale=0.6,yscale=0.4]
\node (dep) at (0,0) {{\small \textsc{Dep}}};
\node (tone) at (0,-1.0) {{\small o}};
\end{tikzpicture}
\end{minipage}
}

\newcommand{\DepL}{
\begin{minipage}{0.055\textwidth}
\begin{tikzpicture}[xscale=0.6,yscale=0.4]
\node (dep) at (0,0) {{\small \textsc{Dep}}};
\node (tone) at (0,-1.0) {{\small L}};
\end{tikzpicture}
\end{minipage}
}

\newcommand{\DepH}{
\begin{minipage}{0.055\textwidth}
\begin{tikzpicture}[xscale=0.6,yscale=0.4]
\node (dep) at (0,0) {{\small \textsc{Dep}}};
\node (tone) at (0,-1.0) {{\small H}};
\end{tikzpicture}
\end{minipage}
}

\newcommand{\NoMultDiff}{{\small *loh}}
\newcommand{\Alt}{{\small \textsc{Alt}}}
\newcommand{\NoSkip}{{\small \scell{\textsc{No}\\\textsc{Skip}}}}


\newcommand{\RegDomRt}{
\begin{minipage}{0.030\textwidth}
\begin{tikzpicture}[xscale=0.6,yscale=0.5]
\node (arrow) at (0,0) {{\fns $\downarrow$}};
\node (Rt) at (0,-0.55) {{\small o}};
\node (reg) at (0,0.55) {{\small \textrho}};
\end{tikzpicture}
\end{minipage}
}

\newcommand{\DepRegRt}{
\begin{minipage}{0.1\textwidth}
\begin{tikzpicture}[xscale=0.6,yscale=0.4]
\node (dep) at (0,0) {{\small \textsc{Dep}}};
\node (arrow) at (0.75,0) {{\tiny $\downarrow$}};
\node (Rt) at (0.75,-0.5) {{\fns o}};
\node (reg) at (0.75,0.5) {{\fns \textrho}};
\end{tikzpicture}
\end{minipage}
}

% unused

\newcommand{\ToneByRt}{
\begin{minipage}{0.05\textwidth}
\begin{tikzpicture}[xscale=0.6,yscale=0.5]
\node (arrow) at (0,0) {{\fns $\uparrow$}};
\node (Rt) at (0,-0.55) {{\small o}};
\node (tone) at (0,0.55) {{\small \texttau}};
\end{tikzpicture}
\end{minipage}
}

\newcommand{\RegByRt}{
\begin{minipage}{0.05\textwidth}
\begin{tikzpicture}[xscale=0.6,yscale=0.5]
\node (arrow) at (0,0) {{\fns $\uparrow$}};
\node (Rt) at (0,-0.55) {{\small o}};
\node (reg) at (0,0.55) {{\small \textrho}};
\end{tikzpicture}
\end{minipage}
}

\newcommand{\ToneDomRt}{
\begin{minipage}{0.05\textwidth}
\begin{tikzpicture}[xscale=0.6,yscale=0.5]
\node (arrow) at (0,0) {{\fns $\downarrow$}};
\node (Rt) at (0,-0.55) {{\small o}};
\node (tone) at (0,0.55) {{\small \texttau}};
\end{tikzpicture}
\end{minipage}
}

% --- OT tableaus --- %

% Sec. 3.2, first tabl.

\newcommand{\OTHLInput}{
\begin{minipage}{0.17\textwidth}
\begin{tikzpicture}[xscale=\myscalex,yscale=\myscaley]
\node (tone) at (2,0) {(= H)};
\node (syl) at (0,0) {\textsigma};
\node (Rt) at (0,1) {o};
\node (H) at (-0.5,2) {H};
\node (R) at (0.5,3) {h};
\node (Rt2) at (1.5,1.0) {o};
%\node (H2) at (1.0,2) {\epen{L}};
\node (R2) at (2.0,3) {\blue{l}};
\draw [thick] (syl.north) -- (Rt.south) ;
\draw [thick] (Rt.north) -- (H.south) ;
\draw [thick] (Rt.north) -- (R.south) ;
\draw [thick] (syl.north) -- (Rt2.south) ;
%\draw [dashed] (Rt2.north) -- (H2.south) ;
%\draw [dashed] (Rt2.north) -- (R2.south) ;
\end{tikzpicture}
\end{minipage}
}

\newcommand{\OTHLWinner}{
\begin{minipage}{0.17\textwidth}
\begin{tikzpicture}[xscale=\myscalex,yscale=\myscaley]
\node (tone) at (2,0) {(= HL)};
\node (syl) at (0,0) {\textsigma};
\node (Rt) at (0,1) {o};
\node (H) at (-0.5,2) {H};
\node (R) at (0.5,3) {h};
\node (Rt2) at (1.5,1.0) {o};
\node (H2) at (1.0,2) {\epen{L}};
\node (R2) at (2.0,3) {\blue{l}};
\draw [thick] (syl.north) -- (Rt.south) ;
\draw [thick] (Rt.north) -- (H.south) ;
\draw [thick] (Rt.north) -- (R.south) ;
\draw [thick] (syl.north) -- (Rt2.south) ;
\draw [dashed] (Rt2.north) -- (H2.south) ;
\draw [dashed] (Rt2.north) -- (R2.south) ;
\end{tikzpicture}
\end{minipage}
}

\newcommand{\OTHLSpreadingHOnly}{
\begin{minipage}{0.17\textwidth}
\begin{tikzpicture}[xscale=\myscalex,yscale=\myscaley]
\node (tone) at (2,0) {(= HM)};
\node (syl) at (0,0) {\textsigma};
\node (Rt) at (0,1) {o};
\node (H) at (-0.5,2) {H};
\node (R) at (0.5,3) {h};
\node (Rt2) at (1.5,1.0) {o};
%\node (H2) at (1.0,2) {\epen{L}};
\node (R2) at (2.0,3) {\blue{l}};
\draw [thick] (syl.north) -- (Rt.south) ;
\draw [thick] (Rt.north) -- (H.south) ;
\draw [thick] (Rt.north) -- (R.south) ;
\draw [thick] (syl.north) -- (Rt2.south) ;
\draw [dashed] (Rt2.north) -- (R2.south) ;
\draw [dashed] (Rt2.north) -- (H.south) ;
\end{tikzpicture}
\end{minipage}
}

\newcommand{\OTHLInsertH}{
\begin{minipage}{0.17\textwidth}
\begin{tikzpicture}[xscale=\myscalex,yscale=\myscaley]
\node (tone) at (2,0) {(= HM)};
\node (syl) at (0,0) {\textsigma};
\node (Rt) at (0,1) {o};
\node (H) at (-0.5,2) {H};
\node (R) at (0.5,3) {h};
\node (Rt2) at (1.5,1.0) {o};
\node (H2) at (1.0,2) {\epen{H}};
\node (R2) at (2.0,3) {\blue{l}};
\draw [thick] (syl.north) -- (Rt.south) ;
\draw [thick] (Rt.north) -- (H.south) ;
\draw [thick] (Rt.north) -- (R.south) ;
\draw [thick] (syl.north) -- (Rt2.south) ;
\draw [dashed] (Rt2.north) -- (H2.south) ;
\draw [dashed] (Rt2.north) -- (R2.south) ;
\end{tikzpicture}
\end{minipage}
}

\newcommand{\OTHLOverwriting}{
\begin{minipage}{0.17\textwidth}
\begin{tikzpicture}[xscale=\myscalex,yscale=\myscaley]
\node (syl) at (0,0) {\textsigma};
\node (Rt) at (0,1) {o};
\node (H) at (-0.5,2) {H};
\node (R) at (0.5,3) {h};
\node (Rt2) at (1.5,1.0) {o};
%\node (H2) at (1.0,2) {\epen{L}};
\node (R2) at (2.0,3) {\blue{l}};
\draw [thick] (syl.north) -- (Rt.south) ;
\draw [thick] (Rt.north) -- (H.south) ;
\draw [thick] (Rt.north) -- (R.south) ;
\draw [thick] (syl.north) -- (Rt2.south) ;
%\draw [dashed] (Rt2.north) -- (H2.south) ;
\draw [dashed] (Rt.north) -- (R2.south) ;
\node (del) at (0.3,1.9) {\textbf{=}};
\end{tikzpicture}
\end{minipage}
}

\newcommand{\OTHLSpreading}{
\begin{minipage}{0.17\textwidth}
\begin{tikzpicture}[xscale=\myscalex,yscale=\myscaley]
\node (syl) at (0,0) {\textsigma};
\node (Rt) at (0,1) {o};
\node (H) at (-0.5,2) {H};
\node (R) at (0.5,3) {h};
\node (Rt2) at (1.5,1.0) {o};
%\node (H2) at (1.0,2) {\epen{L}};
\node (R2) at (2.0,3) {\blue{l}};
\draw [thick] (syl.north) -- (Rt.south) ;
\draw [thick] (Rt.north) -- (H.south) ;
\draw [thick] (Rt.north) -- (R.south) ;
\draw [thick] (syl.north) -- (Rt2.south) ;
%\draw [dashed] (Rt2.north) -- (H2.south) ;
\draw [dashed] (Rt2.north) -- (H.south) ;
\draw [dashed] (Rt2.north) -- (R.south) ;
\end{tikzpicture}
\end{minipage}
}

% Sec. 4.2, second tabl.: phrase-medial position

\newcommand{\OTHnoLInput}{
\begin{minipage}{0.17\textwidth}
\begin{tikzpicture}[xscale=\myscalex,yscale=\myscaley]
\node (tone) at (2,0) {(= H)};
\node (syl) at (0,0) {\textsigma};
\node (Rt) at (0,1) {o};
\node (H) at (-0.5,2) {H};
\node (R) at (0.5,3) {h};
\node (Rt2) at (1.5,1.0) {o};
%\node (H2) at (1.0,2) {\epen{L}};
%\node (R2) at (2.0,3) {\blue{l}};
\draw [thick] (syl.north) -- (Rt.south) ;
\draw [thick] (Rt.north) -- (H.south) ;
\draw [thick] (Rt.north) -- (R.south) ;
\draw [thick] (syl.north) -- (Rt2.south) ;
\end{tikzpicture}
\end{minipage}
}

\newcommand{\OTHnoLEpenth}{
\begin{minipage}{0.17\textwidth}
\begin{tikzpicture}[xscale=\myscalex,yscale=\myscaley]
\node (tone) at (2,0) {(= HM)};
\node (syl) at (0,0) {\textsigma};
\node (Rt) at (0,1) {o};
\node (H) at (-0.5,2) {H};
\node (R) at (0.5,3) {h};
\node (Rt2) at (1.5,1.0) {o};
\node (H2) at (1.0,2) {\epen{L}};
\node (R2) at (2.0,3) {\epen{h}};
\draw [thick] (syl.north) -- (Rt.south) ;
\draw [thick] (Rt.north) -- (H.south) ;
\draw [thick] (Rt.north) -- (R.south) ;
\draw [thick] (syl.north) -- (Rt2.south) ;
\draw [dashed] (Rt2.north) -- (H2.south) ;
\draw [dashed] (Rt2.north) -- (R2.south) ;
\end{tikzpicture}
\end{minipage}
}

\newcommand{\OTHnoLSpreading}{
\begin{minipage}{0.17\textwidth}
\begin{tikzpicture}[xscale=\myscalex,yscale=\myscaley]
\node (tone) at (2,0) {(= HH)};
\node (syl) at (0,0) {\textsigma};
\node (Rt) at (0,1) {o};
\node (H) at (-0.5,2) {H};
\node (R) at (0.5,3) {h};
\node (Rt2) at (1.5,1.0) {o};
%\node (H2) at (1.0,2) {\epen{L}};
%\node (R2) at (2.0,3) {\blue{l}};
\draw [thick] (syl.north) -- (Rt.south) ;
\draw [thick] (Rt.north) -- (H.south) ;
\draw [thick] (Rt.north) -- (R.south) ;
\draw [thick] (syl.north) -- (Rt2.south) ;
\draw [dashed] (Rt2.north) -- (H.south) ;
\draw [dashed] (Rt2.north) -- (R.south) ;
\end{tikzpicture}
\end{minipage}
}

% Sec. 4.2, third tabl., LM is unaffected by L\%

\newcommand{\OTLMInput}{
\begin{minipage}{0.2\textwidth}
\begin{tikzpicture}[xscale=\myscalex,yscale=\myscaley]
\node (tone) at (2,0) {(= LM)};
\node (syl) at (0,0) {\textsigma};
\node (Rt) at (0,1) {o};
\node (H) at (-0.5,2) {L};
\node (R) at (0.5,3) {l};
\node (Rt2) at (1.5,1.0) {o};
\node (H2) at (1.0,2) {L};
\node (R2) at (2.0,3) {h};
\node (R3) at (3.0,3) {\blue{l}};
\draw [thick] (syl.north) -- (Rt.south) ;
\draw [thick] (Rt.north) -- (H.south) ;
\draw [thick] (Rt.north) -- (R.south) ;
\draw [thick] (syl.north) -- (Rt2.south) ;
\draw [thick] (Rt2.north) -- (H2.south) ;
\draw [thick] (Rt2.north) -- (R2.south) ;
\end{tikzpicture}
\end{minipage}
}

\newcommand{\OTLMReplace}{
\begin{minipage}{0.2\textwidth}
\begin{tikzpicture}[xscale=\myscalex,yscale=\myscaley]
\node (tone) at (2,0) {(= LL)};
\node (syl) at (0,0) {\textsigma};
\node (Rt) at (0,1) {o};
\node (H) at (-0.5,2) {L};
\node (R) at (0.5,3) {l};
\node (Rt2) at (1.5,1.0) {o};
\node (H2) at (1.0,2) {L};
\node (R2) at (2.0,3) {h};
\node (R3) at (3.0,3) {\blue{l}};
\draw [thick] (syl.north) -- (Rt.south) ;
\draw [thick] (Rt.north) -- (H.south) ;
\draw [thick] (Rt.north) -- (R.south) ;
\draw [thick] (syl.north) -- (Rt2.south) ;
\draw [thick] (Rt2.north) -- (H2.south) ;
\draw [thick] (Rt2.north) -- (R2.south) ;
\draw [dashed] (Rt2.north) -- (R3.south) ;
\node (del) at (1.8,2.1) {\textbf{=}};
\end{tikzpicture}
\end{minipage}
}

\newcommand{\OTLMTwoReg}{
\begin{minipage}{0.2\textwidth}
\begin{tikzpicture}[xscale=\myscalex,yscale=\myscaley]
\node (tone) at (2,0) {(= LML)};
\node (syl) at (0,0) {\textsigma};
\node (Rt) at (0,1) {o};
\node (H) at (-0.5,2) {L};
\node (R) at (0.5,3) {l};
\node (Rt2) at (1.5,1.0) {o};
\node (H2) at (1.0,2) {L};
\node (R2) at (2.0,3) {h};
\node (R3) at (3.0,3) {\blue{l}};
\draw [thick] (syl.north) -- (Rt.south) ;
\draw [thick] (Rt.north) -- (H.south) ;
\draw [thick] (Rt.north) -- (R.south) ;
\draw [thick] (syl.north) -- (Rt2.south) ;
\draw [thick] (Rt2.north) -- (H2.south) ;
\draw [thick] (Rt2.north) -- (R2.south) ;
\draw [dashed] (Rt2.north) -- (R3.south) ;
\end{tikzpicture}
\end{minipage}
}

% Sec. 4.2, fourth tabl., L is affected by L\% but M is not

\newcommand{\OTLInput}{
\begin{minipage}{0.17\textwidth}
\begin{tikzpicture}[xscale=\myscalex,yscale=\myscaley]
\node (tone) at (2,0) {(= L)};
\node (syl) at (0,0) {\textsigma};
\node (Rt) at (0,1) {o};
\node (H) at (-0.5,2) {L};
\node (R) at (0.5,3) {l};
\node (R2) at (2,3) {\blue{l}};
\draw [thick] (syl.north) -- (Rt.south) ;
\draw [thick] (Rt.north) -- (H.south) ;
\draw [thick] (Rt.north) -- (R.south) ;
\end{tikzpicture}
\end{minipage}
}

\newcommand{\OTLLowered}{
\begin{minipage}{0.17\textwidth}
\begin{tikzpicture}[xscale=\myscalex,yscale=\myscaley]
\node (tone) at (2,0) {(= LL)};
\node (syl) at (0,0) {\textsigma};
\node (Rt) at (0,1) {o};
\node (H) at (-0.5,2) {L};
\node (R) at (0.5,3) {l};
\node (R2) at (2,3) {\blue{l}};
\draw [thick] (syl.north) -- (Rt.south) ;
\draw [thick] (Rt.north) -- (H.south) ;
\draw [thick] (Rt.north) -- (R.south) ;
\draw [dashed] (Rt.north) -- (R2.south) ;
\end{tikzpicture}
\end{minipage}
}

\newcommand{\OTMInput}{
\begin{minipage}{0.17\textwidth}
\begin{tikzpicture}[xscale=\myscalex,yscale=\myscaley]
\node (tone) at (2,0) {(= M)};
\node (syl) at (0,0) {\textsigma};
\node (Rt) at (0,1) {o};
\node (H) at (-0.5,2) {L};
\node (R) at (0.5,3) {h};
\node (R2) at (2,3) {\blue{l}};
\draw [thick] (syl.north) -- (Rt.south) ;
\draw [thick] (Rt.north) -- (H.south) ;
\draw [thick] (Rt.north) -- (R.south) ;
\end{tikzpicture}
\end{minipage}
}

\newcommand{\OTMLowered}{
\begin{minipage}{0.17\textwidth}
\begin{tikzpicture}[xscale=\myscalex,yscale=\myscaley]
\node (tone) at (2,0) {(= ML)};
\node (syl) at (0,0) {\textsigma};
\node (Rt) at (0,1) {o};
\node (H) at (-0.5,2) {L};
\node (R) at (0.5,3) {h};
\node (R2) at (2,3) {\blue{l}};
\draw [thick] (syl.north) -- (Rt.south) ;
\draw [thick] (Rt.north) -- (H.south) ;
\draw [thick] (Rt.north) -- (R.south) ;
\draw [dashed] (Rt.north) -- (R2.south) ;
\end{tikzpicture}
\end{minipage}
}

% Sec. 4.2, fifth tableau, polar questions with level tones

\newcommand{\OTLPolIn}{
\begin{minipage}{0.20\textwidth}
\begin{tikzpicture}[xscale=\myscalex-0.05,yscale=\myscaley-0.05]
\node (tone) at (3.5,0) {(= L)};
\node (syl) at (0,0) {\textsigma};
\node (syl2) at (2,0) {\red{\textsigma}};
\node (Rt) at (0,1) {o};
\node (H) at (-0.5,2) {L};
\node (R) at (0.5,3) {l};
\node (Rt2) at (2,1) {\red{o}};
\draw [thick] (syl.north) -- (Rt.south) ;
\draw [thick,red] (syl2.north) -- (Rt2.south) ;
\draw [thick] (Rt.north) -- (H.south) ;
\draw [thick] (Rt.north) -- (R.south) ;
\end{tikzpicture}
\end{minipage}
}

\newcommand{\OTLPolDef}{
\begin{minipage}{0.20\textwidth}
\begin{tikzpicture}[xscale=\myscalex-0.05,yscale=\myscaley-0.05]
\node (tone) at (3.5,0) {(= L.M)};
\node (syl) at (0,0) {\textsigma};
\node (syl2) at (2,0) {\red{\textsigma}};
\node (Rt) at (0,1) {o};
\node (H) at (-0.5,2) {L};
\node (R) at (0.5,3) {l};
\node (H2) at (1.5,2) {\epen{L}};
\node (R2) at (2.5,3) {\epen{h}};
\node (Rt2) at (2,1) {\red{o}};
\draw [thick] (syl.north) -- (Rt.south) ;
\draw [thick,red] (syl2.north) -- (Rt2.south) ;
\draw [thick] (Rt.north) -- (H.south) ;
\draw [thick] (Rt.north) -- (R.south) ;
\draw [semithick,dashed] (Rt2.north) -- (H2.south) ;
\draw [semithick,dashed] (Rt2.north) -- (R2.south) ;
\end{tikzpicture}
\end{minipage}
}

\newcommand{\OTLPolAlt}{
\begin{minipage}{0.20\textwidth}
\begin{tikzpicture}[xscale=\myscalex-0.05,yscale=\myscaley-0.05]
\node (tone) at (3.5,0) {(= L.L)};
\node (syl) at (0,0) {\textsigma};
\node (syl2) at (2,0) {\red{\textsigma}};
\node (Rt) at (0,1) {o};
\node (H) at (-0.5,2) {L};
\node (R) at (0.5,3) {l};
\node (Rt2) at (2,1) {\red{o}};
\draw [thick] (syl.north) -- (Rt.south) ;
\draw [thick,red] (syl2.north) -- (Rt2.south) ;
\draw [thick] (Rt.north) -- (H.south) ;
\draw [thick] (Rt.north) -- (R.south) ;
\draw [semithick,dashed] (Rt2.north) -- (H.south) ;
\draw [semithick,dashed] (Rt2.north) -- (R.south) ;
\end{tikzpicture}
\end{minipage}
}

% Sec. 4.2, sixth tableau, polar questions with contour tones

\newcommand{\OTLLPolIn}{
\begin{minipage}{0.23\textwidth}
\begin{tikzpicture}[xscale=\myscalex-0.05,yscale=\myscaley-0.05]
\node (tone) at (5.2,0) {(= L)};
\node (syl) at (0,0) {\textsigma};
\node (syl3) at (3.4,0) {\red{\textsigma}};
\node (Rt) at (0,1) {o};
\node (Rt2) at (1.7,1) {o};
\node (Rt3) at (3.4,1) {\red{o}};
\node (H) at (-0.5,2) {L};
\node (R) at (0.5,3) {l};
\draw [thick] (syl.north) -- (Rt.south) ;
\draw [thick] (syl.north) -- (Rt2.south) ;
\draw [thick,red] (syl3.north) -- (Rt3.south) ;
\draw [thick] (Rt.north) -- (H.south) ;
\draw [thick] (Rt.north) -- (R.south) ;
\end{tikzpicture}
\end{minipage}
}

\newcommand{\OTLLPolDef}{
\begin{minipage}{0.23\textwidth}
\begin{tikzpicture}[xscale=\myscalex-0.05,yscale=\myscaley-0.05]
\node (tone) at (5.2,0) {(= L.M)};
\node (syl) at (0,0) {\textsigma};
\node (syl3) at (3.4,0) {\red{\textsigma}};
\node (Rt) at (0,1) {o};
\node (Rt2) at (1.7,1) {o};
\node (Rt3) at (3.4,1) {\red{o}};
\node (H) at (-0.5,2) {L};
\node (R) at (0.5,3) {l};
\node (H3) at (2.9,2) {\epen{L}};
\node (R3) at (3.9,3) {\epen{h}};
\draw [thick] (syl.north) -- (Rt.south) ;
\draw [thick] (syl.north) -- (Rt2.south) ;
\draw [thick,red] (syl3.north) -- (Rt3.south) ;
\draw [thick] (Rt.north) -- (H.south) ;
\draw [thick] (Rt.north) -- (R.south) ;
\draw [dashed] (Rt3.north) -- (H3.south) ;
\draw [dashed] (Rt3.north) -- (R3.south) ;
\end{tikzpicture}
\end{minipage}
}

\newcommand{\OTLLPolSkip}{
\begin{minipage}{0.23\textwidth}
\begin{tikzpicture}[xscale=\myscalex-0.05,yscale=\myscaley-0.05]
\node (tone) at (5.2,0) {(= L.L)};
\node (syl) at (0,0) {\textsigma};
\node (syl3) at (3.4,0) {\red{\textsigma}};
\node (Rt) at (0,1) {o};
\node (Rt2) at (1.7,1) {o};
\node (Rt3) at (3.4,1) {\red{o}};
\node (H) at (-0.5,2) {L};
\node (R) at (0.5,3) {l};
\draw [thick] (syl.north) -- (Rt.south) ;
\draw [thick] (syl.north) -- (Rt2.south) ;
\draw [thick,red] (syl3.north) -- (Rt3.south) ;
\draw [thick] (Rt.north) -- (H.south) ;
\draw [thick] (Rt.north) -- (R.south) ;
\draw [dashed] (Rt3.north) -- (H.south) ;
\draw [dashed] (Rt3.north) -- (R.south) ;
\end{tikzpicture}
\end{minipage}
}  
  
\newcommand{\ilit}[1]{#1\il{#1}}    
\newcommand{\isit}[1]{#1\is{#1}}  

\makeatletter
\let\thetitle\@title
\let\theauthor\@author 
\makeatother

\newcommand{\togglepaper}[1][0]{ 
  \bibliography{../localbibliography}
  %% hyphenation points for line breaks
%% Normally, automatic hyphenation in LaTeX is very good
%% If a word is mis-hyphenated, add it to this file
%%
%% add information to TeX file before \begin{document} with:
%% %% hyphenation points for line breaks
%% Normally, automatic hyphenation in LaTeX is very good
%% If a word is mis-hyphenated, add it to this file
%%
%% add information to TeX file before \begin{document} with:
%% \include{localhyphenation}
\hyphenation{
affri-ca-te
affri-ca-tes
com-ple-ments
par-a-digm
Sha-ron
Kings-ton
phe-nom-e-non
Daul-ton
Abu-ba-ka-ri
Ngo-nya-ni
Clem-ents 
King-ston
Tru-cken-brodt
Tab-leau
cophono-logies
mark-edness
Ti-gri-nya
a-mong
Car-stens
Lu-bu-ku-su
}
\hyphenation{
affri-ca-te
affri-ca-tes
com-ple-ments
par-a-digm
Sha-ron
Kings-ton
phe-nom-e-non
Daul-ton
Abu-ba-ka-ri
Ngo-nya-ni
Clem-ents 
King-ston
Tru-cken-brodt
Tab-leau
cophono-logies
mark-edness
Ti-gri-nya
a-mong
Car-stens
Lu-bu-ku-su
}
  \papernote{\scriptsize\normalfont
    \theauthor.
    \thetitle. 
    To appear in: 
    Emily Clem,   Peter Jenks \& Hannah Sande.
    Theory and description in African Linguistics: Selected papers from the 47th Annual Conference on African Linguistics.
    Berlin: Language Science Press. [preliminary page numbering]
  }
  \pagenumbering{roman}
  \setcounter{chapter}{#1}
  \addtocounter{chapter}{-1}
}

\newcommand{\upstep}{\textupstep}


% \newcounter{tableauxcounter}

\renewcommand{\textltailn}{ɲ}
\renewcommand{\textbardotlessj}{ɟ}

\newcommand{\emphkh}[1]{\textit{#1}} %originally \textbf, banned by the guidelines



\definecolor{lsDOIGray}{cmyk}{0,0,0,0.45}


\newcommand{\xuparrow}[1]{%
  {\left\uparrow\vbox to #1{}\right.\kern-\nulldelimiterspace}
}
\renewcommand \textupstep[1]{\char"A71B#1}
\renewcommand \textdownstep[1]{\char"A71C#1}
 
 \newcommand{\ꜛ}{\textsf{ꜛ}}
 
\def\biberror{\undefined}


\newcommand{\OTbox}[1]{\resizebox{.88\textwidth}{!}{#1}}
 
  %% hyphenation points for line breaks
%% Normally, automatic hyphenation in LaTeX is very good
%% If a word is mis-hyphenated, add it to this file
%%
%% add information to TeX file before \begin{document} with:
%% %% hyphenation points for line breaks
%% Normally, automatic hyphenation in LaTeX is very good
%% If a word is mis-hyphenated, add it to this file
%%
%% add information to TeX file before \begin{document} with:
%% %% hyphenation points for line breaks
%% Normally, automatic hyphenation in LaTeX is very good
%% If a word is mis-hyphenated, add it to this file
%%
%% add information to TeX file before \begin{document} with:
%% \include{localhyphenation}
\hyphenation{
affri-ca-te
affri-ca-tes
com-ple-ments
par-a-digm
Sha-ron
Kings-ton
phe-nom-e-non
Daul-ton
Abu-ba-ka-ri
Ngo-nya-ni
Clem-ents 
King-ston
Tru-cken-brodt
Tab-leau
cophono-logies
mark-edness
Ti-gri-nya
a-mong
Car-stens
Lu-bu-ku-su
}
\hyphenation{
affri-ca-te
affri-ca-tes
com-ple-ments
par-a-digm
Sha-ron
Kings-ton
phe-nom-e-non
Daul-ton
Abu-ba-ka-ri
Ngo-nya-ni
Clem-ents 
King-ston
Tru-cken-brodt
Tab-leau
cophono-logies
mark-edness
Ti-gri-nya
a-mong
Car-stens
Lu-bu-ku-su
}
\hyphenation{
affri-ca-te
affri-ca-tes
com-ple-ments
par-a-digm
Sha-ron
Kings-ton
phe-nom-e-non
Daul-ton
Abu-ba-ka-ri
Ngo-nya-ni
Clem-ents 
King-ston
Tru-cken-brodt
Tab-leau
cophono-logies
mark-edness
Ti-gri-nya
a-mong
Car-stens
Lu-bu-ku-su
} 
  \togglepaper[22]
}{}

 
 
\begin{document}
\maketitle

\section{Introduction}\label{sec:gould:1}

In this paper we discuss a previously undiscussed puzzle that emerges from the literature on the derivation of relative clauses (RCs) in \ili{Swahili}. Two conflicting analyses have been proposed: one that involves \isi{syntactic movement} of the RC-head (i.e.\ \isi{head raising}) and one that does not. On the one hand, there is \citeauthor{Ngonyani2001}'s (\citeyear{Ngonyani2001}; \citeyear{Ngonyani2006}) \isi{movement} analysis, which is largely based on inverse \isi{scope} data involving \isi{pronoun} binding or multiple quantifiers. On the other hand, there is \posscitet{Barrett-Keach1985} and \posscitet{Keach2004} non-\isi{movement} analysis, which is based primarily on data related to \isi{relative clause} islands. The apparent incompatibility of these two arguments necessitates a more detailed view of the data with the aim of developing a uniform analysis of all the data. Our \isi{focus} in this paper is to do so with a particular kind of \isi{relative clause} in \ili{Swahili}, namely \textit{amba} relative clauses, which are RCs that contain the relativizing morpheme \textit{amba}. 

In this paper, we propose such a uniform analysis of \textit{amba}{}-RCs that relies on \isi{syntactic movement}. In addition to confirming grammaticality judgments for some of the data from the literature, we conducted a more careful investigation of the inverse \isi{scope} interpretation of quantifiers in multiply-embedded \textit{amba}{}-RCs, which are putatively islands for \isi{movement}. As we discuss, the kind of inverse \isi{scope} we consider in \textit{amba}{}-RCs could be tied to either \isi{movement} of a quantificational RC-head or to long-distance Quantifier Raising (QR) of some other quantifier out of an RC, in which case the RC-head need not move. We then consider when general constraints on QR would and would not allow for long-distance QR out of an RC to be licensed. In part by controlling for when QR should not be possible, we are led to expect that the relevant inverse \isi{scope} interpretation will be possible if there is \isi{movement} of the relative’s head, but impossible if relativization does not involve \isi{movement} of the head. As the data we present indicate that inverse \isi{scope} is indeed possible, we conclude (a) that \textit{amba}{}-RCs involve \isi{movement} of the relative’s head, and (b) that \textit{amba}{}-RCs are not islands for \isi{overt movement}. Possible supporting evidence, which we discuss, comes from looking at another \isi{long-distance dependency} that is also possible across \textit{amba} \isi{relative clause} boundaries.


In using constraints on QR to establish an argument as to whether a \isi{movement} dependency exists elsewhere in the syntactic structure, we are inspired by \citet{Fox2000} with regard to both the constraints themselves, and how they are used to establish an argument for or against \isi{movement}. Our \isi{focus} is somewhat different from Fox’s, though, in that we use constraints on long-distance QR out of an RC to test for whether that RC’s head has undergone \isi{movement} for relativization purposes. As far as we know, this is a novel attempt at (a) considering when long-distance QR out of an RC would be licensed, as well as (b) using such QR as part of a test for whether the RC-head undergoes \isi{movement}.

On a more general level, this paper can be seen as an experiment in rigorously investigating one particular kind of evidence with an eye toward reconciling other, potentially disparate strands of evidence. To the extent that this experiment succeeds, our hope is that it can serve as a kind of blueprint for investigating additional phenomena involving displacement that at first glance suggest multiple contrasting analyses.

 
The judgments we report in this paper (some of which confirm earlier judgments from the literature) represent a unified, speaker-internal set of data from a Kenyan speaker of the standard Kenya-Tanzania variety of \ili{Swahili}. Our hope is that these judgments can be replicated in future work with further speakers. The data were gathered via elicitations sessions using constructed examples. The set of examples provided in \sectref{sec:gould:2} and \sectref{sec:gould:3} is based on existing examples from the literature, whereas the set of examples in \sectref{sec:gould:4} and \sectref{sec:gould:5} was constructed for the purpose of this paper. As indicated above, data involving quantifier \isi{scope} are of particular importance for the argument being developed in this paper. For each quantifier \isi{scope} relation between two quantifiers that we tested, the following procedure was used. The speaker was presented with some illustration and was instructed on what was being depicted in that illustration. The illustration depicted a scenario that would be true under one \isi{scope} relation between two quantifiers for some \ili{Swahili} sentence (which had not been presented to the speaker), but false under the other \isi{scope} relation. The speaker was then presented with the relevant \ili{Swahili} example and asked to evaluate the well-formedness of such an example given the scenario depicted in the illustration. These evaluations are what we report with \isi{scope} judgments in the relevant examples.

The structure of this paper is as follows. In \sectref{sec:gould:2} we introduce the form of \textit{amba}{}-RCs. \sectref{sec:gould:3} briefly reviews two existing analyses of \textit{amba}{}-RCs and some of the core data that have been discussed in support of these analyses. We then propose a test in \sectref{sec:gould:4.1} that can help us adjudicate between these analyses, and \sectref{sec:gould:4.2} provides an illustration of this test and some discussion of its implications. \sectref{sec:gould:5} contains an additional data point from a further \isi{long-distance dependency} that is consistent with our proposal, and \sectref{sec:gould:6} concludes the paper. 

\section{The form of \textit{amba} relatives}\label{sec:gould:2}
 
\ili{Swahili} has a number of different types of \isi{relative clause} constructions \citep[cf.][]{Ngonyani2001}, but in this paper we restrict our attention to what we call \textit{amba} relatives. These are relative clauses that contain the relativizing morpheme \textit{amba}, as illustrated in \REF{ex:gould:1}. 
 
\ea
\gll Ni-li-nunu-a \textbf{vi-tabu} amba-vyo Juma a-li-vi-som-a.\\
\textsc{1\textsuperscript{st}}\textsc{.sg-pst}{}-buy-\textsc{fv} 8-book amba{}-8\textsc{agr} Juma \textsc{1s-pst-8o}{}-read-\textsc{fv}\\
\glt ‘I bought the books that Juma read.’  \label{ex:gould:1}
\todo{moved example up}
\todo{1\textsc{st} should drop the \textsc{st}. Same for 2 and 3.}
\z

\REF{ex:gould:1} shows that these are head-external RCs, with the head (here \textit{vi-tabu} ‘books’) preceding first \textit{amba}, then an \isi{agreement marker} ending in -\textit{o} or -\textit{e} (which we gloss as \textsc{agr}), and then the \isi{relative clause} proper.   We use the term \isi{agreement marker} here descriptively, simply to indicate that its morphology corresponds  with the \isi{noun class} of the head of the relative.\footnote{There are various analytical possibilities for what this \isi{agreement marker} might be. For instance, it could be the reflex of agreement between some syntactic head (perhaps C) and the head of the relative, or it could be, as \citet{Henderson2006} suggests, a \isi{resumptive pronoun}. As far as we can tell, either of these analyses is in principle a viable one, and both are compatible with the overall discussion in this paper (see also note 4). Yet another possibility, which we do not consider any further is \posscitet{Keach2004} own claim, which we discuss in the following section, that it is a \isi{relative pronoun}.} In examples throughout, we will indicate the head of an \textit{amba}{}-RC in boldface.

% \ea
% \gll Ni-li-nunu-a \textbf{vi-tabu} amba-vyo Juma a-li-vi-som-a.\\
% \textsc{1\textsuperscript{st}}\textsc{.sg-pst}{}-buy-\textsc{fv} 8-book amba{}-8\textsc{agr} Juma \textsc{1s-pst-8o}{}-read-\textsc{fv}\\
% \glt ‘I bought the books that Juma read.’  \label{ex:gould:1}
% \z

Other \isi{relative clause} constructions in \ili{Swahili} have different forms and do not contain the morpheme \textit{amba}. In some research, such as \citeauthor{Ngonyani2001} (\citeyear{Ngonyani2001}; \citeyear{Ngonyani2006}), both \textit{amba}{}-RCs and non-\textit{amba}{}-RCs are used interchangeably in constructing a theoretical analysis and are given the same analytical treatment. However, we believe this approach introduces a potential confound, as it has been proposed (e.g. \citealt{Barrett-Keach1985}) that the different types of \ili{Swahili} relatives involve different syntactic structures. To avoid this potential confound, each type of RC can be investigated systematically and independently of the other RC types. This is the approach we take here by focusing on \textit{amba}{}-RCs; future research can look at extending this approach to the other RC types in \ili{Swahili}. 

\section{The puzzle of previous approaches}\label{sec:gould:3}
 
In this section we review two competing analyses of \textit{amba}{}-RCs, one with and one without \isi{syntactic movement} of the RC-head. The disparity of these analyses leaves us with a puzzle as to how to analytically approach these relatives. It should be noted, though, that the different analyses are not based on the same core set of data. In this paper, we address this shortcoming by investigating a more comprehensive data set, which we then use as the foundation for our analysis of \textit{amba}{}-RCs.

We begin with \posscitet{Barrett-Keach1985} and \posscitet{Keach2004} non-\isi{movement} analysis. As regards an implementation of such a non-\isi{movement} approach, we will follow \citet{Keach2004} in our discussion here, but the approach in \citet{Barrett-Keach1985} is highly parallel. A schematic structure for the head-external relative in \REF{ex:gould:1} is given in \REF{ex:gould:2}, which is based on \citet[126]{Keach2004}. Keach treats the \isi{agreement marker} suffixed to \textit{amba} as a relative \isi{pronoun}, which we represent here as \textsc{agr}. For Keach, this relative \isi{pronoun} is co-indexed with a null \isi{pronoun} (\textit{pro}) in the \isi{gap position} and also, presumably, the external head (e.g. ‘books’ in \REF{ex:gould:1}):

  
\ea\label{ex:gould:2}
Non-\isi{movement} analysis of \textit{amba}{}-RCs (cf. \citealt{Keach2004})\\
Head\textsubscript{i} [ \textit{amba}{}-\textsc{agr}\textsubscript{i} [ … \textit{pro}\textsubscript{i} … ] ]
\z

  
Note that the \isi{long-distance dependency} in \REF{ex:gould:2} is established via co-indexa\-tion and by binding of \textit{pro} by the relative \isi{pronoun}. Whether the \isi{agreement marker} should be treated as a relative \isi{pronoun} or as simply being the realization of phi-feature agreement is not crucial to our concerns here (which have to do with the presence or absence of \isi{movement}), and we will thus abstract away from this point. However, we believe that a more semantically transparent representation of \REF{ex:gould:2} involves something along the lines of inserting an appropriately co-indexed null operator at the edge of the \isi{embedded clause}. In line with this, we will not treat the \isi{agreement marker} as denoting an individual, and in fact will treat the entire \textit{amba}+\textsc{agr} complex as a formative of RCs that is semantically vacuous (cf. the treatment found in \citet{Heim1998} for the complementizer \textit{that} of \ili{English} relative clauses).\footnote{And should \textsc{agr} turn out to be a resumptive \isi{pronoun} (which is interpreted as a variable ranging over individuals) under the \isi{movement} analysis that we consider later in this section, this will not affect the discussion in \sectref{sec:gould:4} of Quantifier Raising with regard to violations of \isi{scope economy}.}

Crucially, according to the analysis in \REF{ex:gould:2} or any such similar analysis (including \citealt{Barrett-Keach1985}), the external head ‘books’ is not extracted via \isi{movement} from the \isi{gap position} within the relative; instead, this analysis proposes that the head is base-generated in its external position outside of the relative. Indeed, relativization according to this kind of analysis does not involve any \isi{movement} at all (such as, for example, null operator \isi{movement}).

\largerpage[-1]
Barrett-Keach/Keach’s primary evidence to support \REF{ex:gould:2} comes from the absence of \isi{relative clause} island effects.\footnote{\citet{Keach2004} offers, in passing, another type of data as evidence against \isi{movement}, but given Keach’s limited discussion it is not currently clear to us that the data indeed provide a strong argument against \isi{movement}. Keach observes that parasitic gaps do not appear to be licensed by \textit{amba}{}-RCs. We have not been able to thoroughly investigate this construction, but we note that the conditions on parasitic gaps (e.g. the structural position of the parasitic gap and the non-parasitic gap with respect to each other) might independently not be met in \textit{amba}{}-RCs even if they do involve \isi{movement}.}  We can see this, for instance, with grammatical examples that involve relativizing two elements from an \textit{amba}{}-RC. We will call these constructions doubly-embedded RCs, as they involve embedding one \textit{amba}{}-RC inside of another. Further, in examples of what we call doubly-embedded RCs, the sites of the gaps for the two relativized elements occur within the most deeply embedded \textit{amba}{}-RC. We will use the notation \textit{e} and co-indexa\-tion as a neutral way of representing the site of the gaps and the relationship between these gaps and the relativized elements. We were able to confirm Barrett-Keach’s basic observations by constructing the doubly-embedded \textit{amba}{}-RCs in \REF{ex:gould:3} and \REF{ex:gould:4}; these examples were judged grammatical. \REF{ex:gould:3} illustrates this with nested filler-gap dependencies, whereas \REF{ex:gould:4} does so with crossing ones. In \REF{ex:gould:3} and \REF{ex:gould:4} we can call \textit{ki}{}-\textit{tabu} ‘book’ and \textit{m}{}-\textit{tu} ‘person’ the highest RC head, as they are the heads of the superordinate \textit{amba}{}-RCs.

\ea\label{ex:gould:3} Doubly-embedded \textit{amba}{}-RC: nested dependency\\
\gll Nick a-li-ki-nunu-a \textbf{ki-tabu}\textsubscript{j} amba-cho ni-li-wa-on-a \textbf{wa-toto}\textsubscript{i} amba-o [ \textit{e}\textsubscript{i} wa-li-ki-som-a \textit{e}\textsubscript{j} ]. \\
Nick \textsc{1s-pst-7o}{}-buy-\textsc{fv} 7-book\textsubscript{j} amba{}-\textsc{7agr} \textsc{1\textsuperscript{st}}\textsc{.sg-pst-2o}{}-see-\textsc{fv} 2-child\textsubscript{i} amba{}-\textsc{2agr} [ \textit{e}\textsubscript{i} 2s-\textsc{pst-7o}{}-read-\textsc{fv} \textit{e}\textsubscript{j} ]\\
\glt ‘Nick bought the book that I saw the children who read (it).’
\z


\ea\label{ex:gould:4}Doubly-embedded \textit{amba}{}-RC: crossing dependency\\
\gll Ni-li-mw-it-a \textbf{m-tu}\textsubscript{i} amba-ye u-li-wa-on-a \textbf{wa-toto}\textsubscript{j} amba-o [ \textit{e}\textsubscript{i} a-na-wa-pend-a \textit{e}\textsubscript{j} ].\\
\textsc{1\textsuperscript{st}}\textsc{.sg-pst-1o}{}-call-\textsc{fv} 1-person\textsubscript{i} amba{}-1\textsc{agr} 2\textsc{\textsuperscript{nd}}\textsc{.sg-pst-2o}{}-see-\textsc{fv} 2-child\textsubscript{j} amba{}-2\textsc{agr} [ \textit{e}\textsubscript{i} 1\textsc{s-prs}{}-2o-like-\textsc{fv} \textit{e}\textsubscript{j} ]\\
\glt ‘I called the person who you saw the children who (he) likes (them).’
\z


Barrett-Keach/Keach’s argument is that if examples like \REF{ex:gould:3} and \REF{ex:gould:4} involved \isi{syntactic movement}, then they should be ungrammatical, as they would incur a subjacency violation. As no island effect occurs, Barrett-Keach/Keach’s conclusion is that these RCs must be derived without \isi{movement}. As already mentioned, the non-\isi{movement} derivation Keach proposes is given in \REF{ex:gould:2} above. Note that Barrett-Keach/Keach’s argument crucially hinges on the assumption that doubly-embedded RCs should be islands to \isi{movement}. RCs can indeed be islands in languages such as \ili{English}, however in \sectref{sec:gould:4} we dispute the claim that \textit{amba}{}-RCs are necessarily islands in \ili{Swahili}.

\largerpage[-1]
Next, we consider the \isi{movement} analysis of \citeauthor{Ngonyani2001} (\citeyear{Ngonyani2001}; \citeyear{Ngonyani2006}). In contrast to Barrett-Keach/Keach, Ngonyani proposes a \isi{head raising} analysis (cf. \citealt{kayne1994}). According to this analysis, the head of the relative (again, ‘books’ in \REF{ex:gould:1}) moves from the \isi{gap position} to its relative clause-external position. This is shown schematically in \REF{ex:gould:5}, where we assume \isi{movement} dependencies are instantiated by copies in a copy-chain \citep{Chomsky1995}; we use a strikethrough to indicate the positions of unpronounced copies. \REF{ex:gould:5} illustrates this dependency by representing simply two (of a potentially larger number of) copies of the dependency: the pronounced external head and the lowest copy of the head in the \isi{gap position}.\footnote{We note that the possibility of analyzing \textit{amba}{}-RCs as involving resumptive pronouns, which was mentioned in note 1, does not preclude the possibility of their being derived via raising of the RC-head. Support for this view comes from work such as \citet{Aoun2003}, which illustrates that \isi{movement} of a particular constituent is still possible with a resumptive \isi{pronoun} corresponding to that constituent.} 

\ea\label{ex:gould:5}Movement analysis of \textit{amba}{}-RCs (cf. \citealt{Ngonyani2006})\\
\begin{quote}
Head\textsubscript{i} [ \textit{amba}{}-\textsc{agr}\textsubscript{i} [ … \sout{Head\textsubscript{i}} … ] ]
\end{quote}
\z

Ngonyani’s core evidence for \isi{movement} comes from the possibility of inverse \isi{scope} involving (a) \isi{scope} relations between multiple quantifiers and (b) binding data. First, the example in \REF{ex:gould:6} is based on \citet[66]{Ngonyani2001} – but note that Ngonyani’s actual example involves a type of RC that is not an \textit{amba}{}-RC – and supports Ngonyani’s basic finding regarding inverse \isi{scope} of quantifiers. In \REF{ex:gould:6}, the external head contains the numeral -\textit{wili} ‘two’, and the relative contains the \isi{universal quantifier} \textit{kila} ‘each’. Nevertheless, inverse \isi{scope} is possible: the embedded universal can take \isi{scope} over the numeral, resulting in a distributed reading.

\ea\label{ex:gould:6}Inverse \isi{scope} with two quantifiers possible: \ding{51} $\forall$ > 2\\
\gll Ni-li-wa-it-a \textbf{[} \textbf{wa-le} \textbf{wa-gonjwa} \textbf{wa-wili} \textbf{]\textsubscript{i}} amba-o [ kila daktari a-ta-wa-pim-a e\textsubscript{i} ].\\
\textsc{1\textsuperscript{st}.sg}\textsc{{}-pst-2o}{}-call-\textsc{fv} [ \textsc{2-dem} 2-patient 2-two ]\textsubscript{i} amba{}-2\textsc{agr} [ each doctor \textsc{1s-fut-2o}{}-examine-\textsc{fv} \textit{e}\textsubscript{i} ]\\
\glt ‘I called those two patients that each doctor will treat.’
\z

Second, \REF{ex:gould:7} repeats \citeauthor{Ngonyani2001}'s ({\citeyear[65]{Ngonyani2001}}) example and replicates Ngonyani’s judgment that the possessive \isi{pronoun} –\textit{ake} in the external head can be bound by the universal \textit{kila} ‘each’ in the relative, again resulting in a distributed reading. 

\ea\label{ex:gould:7}Inverse binding of pronouns possible\\
\gll \textbf{[} \textbf{Ki-tabu} \textbf{ch-ake\textsubscript{i}} \textbf{ch-a} \textbf{kwanza} \textbf{]\textsubscript{j}} amba-cho [ [ kila mw-andishi~]\textsubscript{i}\hspace*{-3mm} hu-ji-vun-i-a e\textsubscript{j} ] hu-w-a ki-zuri sana.\\
[ 7-book 7-\textsc{3\textsuperscript{rd}.sg.poss}\textsubscript{i} 7-of first ]\textsubscript{j} amba-\textsc{7agr} [ [ every 1-writer~]\textsubscript{i} \textsc{hab}{}-\textsc{refl}{}-be.proud-\textsc{appl}{}-\textsc{fv} e\textsubscript{j} ] \textsc{hab}{}-be-\textsc{fv} 7-good very\\
\glt ‘His first book that every writer is proud of is very good.’
\z

The thrust of Ngonyani’s argument is as follows. In order for the readings in \REF{ex:gould:6} and \REF{ex:gould:7} to be possible, we assume that the position where the \isi{universal quantifier} is interpreted must be in a structurally higher position than the position where the RC-head is interpreted \citep[cf.][]{Heim1998}. Assuming that the universal is interpreted in the RC (but see \sectref{sec:gould:4} for an alternative view), then it follows that the head is also interpreted in a lower position internal to the RC. A \isi{movement} dependency with multiple copies of the head can capture this: in \REF{ex:gould:6} and \REF{ex:gould:7} the higher copy of the head is \textit{pronounced} external to the relative, whereas the quantificational/pronominal material of the head is \textit{interpreted} in a lower copy internal to the relative (and structurally lower than the embedded universal). Note that under a non-\isi{movement} approach, these interpretive facts are not accounted for with the analysis in \REF{ex:gould:2} by itself (cf. \sectref{sec:gould:4} for further discussion of this point). Given \REF{ex:gould:2} alone, the quantificational force of relative’s head in \REF{ex:gould:6} would be interpreted outside the RC in a position that is structurally higher than the embedded universal. Further, the pronominal variable of the relative’s head in \REF{ex:gould:7} would also be interpreted outside the RC in the same high structural position.{}\footnote{\citet{Ngonyani2001} considers two other phenomena as evidence for a \isi{head raising} analysis. The first involves connectivity effects with idioms: certain phrasal idioms in \ili{Swahili} allow for an idiomatic interpretation when part of the idiom is relativized as the RC-head, with the remainder of the idiom occurring inside the relative. However, we are not aware of any theory of semantics that would preclude an idiomatic interpretation given the non-\isi{movement} analysis in \REF{ex:gould:2}. Second, Ngonyani observes that the \isi{agreement marker} following \textit{amba} must agree with the head of the relative. Again, it is not clear to us that a theory of agreement a priori prevents such agreement from occurring given the analysis in \REF{ex:gould:2}. Consequently, we do not think these phenomena present compelling arguments for or against \isi{movement}, and we will not consider them further.}

Given the contrasting analyses by Barrett-Keach/Keach and Ngonyani, we are now faced with the following puzzle. How can we make sense of the interpretative facts in \REF{ex:gould:6} and \REF{ex:gould:7}, which suggest that the head originates within the RC, while at the same time allowing for relativization of heads from doubly-embedded RCs? The interpretations put forward in the literature of the kinds of data presented above have so far pulled us in two different directions. On the one hand, it has been assumed that \textit{amba}{}-RCs are syntactic islands, which pushes us away from a \isi{movement} analysis. On the other hand, the interpretative facts have pushed us toward a \isi{movement} dependency between the external head and the \isi{gap position}.

In this paper, we attempt to resolve this tension by investigating a more comprehensive set of data, as well as a more refined set of analytical hypotheses. Crucially, neither Barrett-Keach/Keach nor Ngonyani considers the same set of core data. Thus Barrett-Keach/Keach does not consider the interpretative facts in \REF{ex:gould:6} and \REF{ex:gould:7}, and Ngonyani does not look at doubly-embedded RCs. So far, we have gone beyond the existing literature by presenting a speaker-internal set of judgments involving both types of data. But we will go further. A natural next step would be to consider the interpretive possibilities of doubly-embedded RCs, i.e.\ a synthesis of the phenomena in (\ref{ex:gould:3}--\ref{ex:gould:4} and (\ref{ex:gould:6}--\ref{ex:gould:7}). This is essentially how in \sectref{sec:gould:4} we approach the tension mentioned above, although the discussion will be limited to considering inverse \isi{scope} involving multiple quantifiers (and not pronominal binding), and as mentioned in the following section, our general approach is not specific to doubly-embedded RCs. By presenting novel data, we will show that the balance of evidence is in favor of a \isi{movement} approach to \textit{amba}{}-RCs. We will give a uniform analysis that accounts for all the data we have seen so far. A consequence of this analysis is that it will force us to reject the assumption that \textit{amba}{}-RCs are islands for \isi{overt movement} in \ili{Swahili}. This is perhaps a desirable outcome, as it dovetails with a further \isi{long-distance dependency} fact in the language, as we show in \sectref{sec:gould:5}.

\section{A closer look at inverse scope}\label{sec:gould:4}

\subsection{Introducing the hypotheses}\label{sec:gould:4.1}

As a way of better understanding \textit{amba}{}-RCs, in \sectref{sec:gould:4} we take a closer look at inverse \isi{scope} data such as \REF{ex:gould:6} and their relation to doubly-embedded constructions such as in \REF{ex:gould:3} and \REF{ex:gould:4}. In particular, we investigate whether inverse \isi{scope} is possible with doubly-embedded RCs. That is, we test to see whether, when there are doubly-embedded RCs, a quantifier pronounced inside one of the RCs can take inverse \isi{scope} over the highest RC head. The initial motivation behind looking more carefully at doubly-embedded RCs is to see whether this inverse \isi{scope}, which we associated with a \isi{movement} analysis of the RC-head in previous section, is also found with doubly-embedded RCs, which are putatively islands for \isi{movement}. If such inverse \isi{scope} is possible, then we might conclude that there is always \isi{movement} in \textit{amba}{}-RCs, and that these RCs are not in fact islands for \isi{movement}. However, as we discuss, we will need to be careful in constructing examples of this sort, in order to control for another potential way in which inverse \isi{scope} could be derived (i.e. one with Quantifier Raising, but without \isi{movement} of the RC-head). Ultimately, the test that we end up with is not specific to doubly-embedded RCs, although we find that the relevant examples with these RCs provide an especially clear way of seeing both (a) that inverse \isi{scope} is possible, and (b) an argument in favor of \textit{amba}{}-RCs involving \isi{movement} of the head.\footnote{Strictly speaking, then, the test we consider and the argument we propose could be reconstructed using examples of non-doubly-embedded \textit{amba}{}-RCs, similar to the one in \REF{ex:gould:6}. However, we find that the doubly-embedded RCs provide a straightforward and clear way of illustrating both the test and argument.} Consequently, we will \isi{focus} on these examples of doubly-embedded RCs, and will frame the discussion below around them. To understand this argument and thus the significance of these novel data, we first present a set of hypotheses regarding \textit{amba}{}-RCs in \sectref{sec:gould:4.1}, before presenting our core data and testing these hypotheses in \sectref{sec:gould:4.2}.

Our discussion and the hypotheses we introduce here hinge on the question of whether \textit{amba}{}-RCs are in fact islands for \isi{overt movement}. As we discuss below, this question bears directly on the issue of analyzing \textit{amba}{}-RCs as involving raising of the head. Our goal is not to strictly falsify one of these hypotheses, but to use these hypotheses as a jumping off point for (a) the question of a \isi{movement}/non-\isi{movement} analysis of the derivation of \textit{amba}{}-RCs, and (b) an account of all the data we have seen so far. The hypotheses we introduce refer to \isi{overt movement}, by which we mean \isi{movement} that must occur before Spell-Out (and thus not at LF) and that feeds PF in that a higher copy of the \isi{movement} dependency is pronounced (cf. \citealt{Chomsky1995}). Ngonyani’s analysis of RCs would thus be an example of \isi{overt movement}, as the highest copy of the head is pronounced at PF. We contrast this with Quantifier Raising, or QR, (cf. \citealt{May1977}; \citeyear{May1985}), which may be covert in that it occurs only at LF and has no detectable effects on PF. 

The two core hypotheses we investigate are given in \REF{ex:gould:8}.\footnote{To be precise, within the context of the hypotheses in \REF{ex:gould:8}, by “\textit{amba}{}-RC” we mean the constituent formed by the RC-head and the RC (including \textit{amba}+\textsc{agr}).}  Note that in the discussion below we will follow the null hypothesis in assuming that all \textit{amba}{}-RCs are derived uniformly, i.e. either uniformly via non-\isi{movement}, as in \REF{ex:gould:2}, or uniformly via \isi{movement}, as in \REF{ex:gould:5}.

\ea\label{ex:gould:8}
\ea\label{ex:gould:8a} Hypothesis 1 (H1): \textit{amba}{}-RCs are not islands for \isi{overt movement}.

\ex\label{ex:gould:8b} Hypothesis 2 (H2): \textit{amba}{}-RCs are islands for \isi{overt movement}.
\z
\z

 
 
We now consider the implications of these two hypotheses, starting with H2. To begin with, we can observe that doubly-embedded RCs play an important role in helping to see the relation between the hypotheses in \REF{ex:gould:8} and whether there is \isi{head raising} in the relatives. According to H2, doubly-embedded RCs as in (\REF{ex:gould:3}-\REF{ex:gould:4}) can only be possible by base-generating the highest RC-head outside its RC as in the non-\isi{movement} analysis in \REF{ex:gould:2}. As the embedded \textit{amba}{}-RC, once built, would constitute an island, it would not be possible to overtly extract another RC-head from within it via \isi{head raising}. Thus, if we follow H2, then the \isi{movement} analysis cannot be adopted for \textit{amba}{}-RCs. 

A further consequence of following H2 regards interpretation. As the highest RC-head must be generated outside its RC, this highest RC head cannot be interpreted inside an \textit{amba}{}-RC for purposes of inverse \isi{scope}. This does not mean, however, that H2 predicts that there cannot be inverse \isi{scope}. Inverse \isi{scope} could be possible on the assumption that covert \isi{movement} is possible out of the relative. In doubly-embedded RCs, for example, it could be the case that QR of a \isi{universal quantifier} from an \textit{amba}{}-RC is possible, thereby allowing the universal to take \isi{scope} over the highest RC head. This kind of QR is also in principle possible to derive inverse \isi{scope} in (\ref{ex:gould:6}--\ref{ex:gould:7}).\footnote{This QR analysis assumes that Weak Crossover Effects would not obtain for \isi{pronoun} binding in \REF{ex:gould:7}. Further research can investigate whether such effects exist more broadly in \ili{Swahili}.} This analysis is along the lines of what \citet{Hulsey2006} propose for QR out of RCs in \ili{English}. We go beyond Hulsey and Sauerland, though, by embedding this proposal in some more general theory, namely \posscitet{Fox2000} theory of QR. In \sectref{sec:gould:4.2} we adopt Fox’s approach in investigating what conditions might allow this kind of non-clause-bounded, or long-distance, QR to be possible. Fox suggests that QR can sometimes be possible out of embedded clauses (although he does not consider relative clauses) if these conditions are met. 

% EDWIN: RETURN 

Crucially, when we conduct the test for inverse \isi{scope} mentioned at the beginning of this section, we will do so in doubly-embedded RCs in which these conditions on long-distance QR are \textit{not} met. Under the assumption that these conditions are operational in \ili{Swahili}, if inverse \isi{scope} is still possible when these conditions are not met, then everything else being equal, we have evidence against H2. In other words, if we find that inverse \isi{scope} is possible in an environment where under a non-\isi{movement} analysis we would not expect it to be possible (because by hypothesis the relevant QR is not possible), then we have evidence against H2 and a non-\isi{movement} analysis. (This raises the question of how such inverse \isi{scope} might be possible, a question that we take up below in considering a \isi{movement} analysis of the relatives under H1.) But if it turns out that such inverse \isi{scope} is impossible, then we have evidence in support of H2 and a non-\isi{movement} analysis. This is because such inverse \isi{scope} is expected to be impossible under a non-\isi{movement} approach, as the relevant conditions on QR are not met.

\newpage 
We now consider H1. Under H1, with everything else being equal, a \isi{movement} analysis as in \REF{ex:gould:5} should be in principle possible for all \textit{amba}{}-RCs, and such an analysis can account for all our data. First, multiple cases of relativization as in the doubly-embedded RCs in (\ref{ex:gould:3}--\ref{ex:gould:4}) are expected to be grammatical because \textit{amba}{}-RCs, not being islands for \isi{overt movement}, will not block this kind of overt extraction. Second, the inverse \isi{scope} facts of (\ref{ex:gould:6}--\ref{ex:gould:7}) can also be accounted for with a full lower copy of the RC-head being interpreted inside the RC. Third, we also expect inverse \isi{scope} to be possible in cases of doubly-embedded RCs because a full lower copy of an extracted head can in principle be interpreted in the most deeply embedded RC. (In \footref{fnlabel}, we mention a slight qualification of the expectation that a full lower copy in a copy-chain can be interpreted, but for the discussion at hand, the general expectation that a full lower copy can be interpreted is sufficient.) As mentioned above, in the following section we will test for inverse \isi{scope} with doubly-embedded RCs. Recall we proposed that under H2, QR is necessary to account for inverse \isi{scope}, and that there must be a non-\isi{movement} analysis of \textit{amba}{}-RCs under H2. Similarly, for a non-\isi{movement} analysis under H1, QR is necessary to account for inverse \isi{scope}. As the RCs we test will be doubly-embedded RCs that are not expected to allow inverse \isi{scope} via QR, we do not expect inverse \isi{scope} with a non-\isi{movement} analysis under H1. In contrast we expect such inverse \isi{scope} to be generally possible with doubly-embedded RCs given H1 and the possibility of interpreting full lower copies under a \isi{movement} analysis. Thus if we find that such inverse \isi{scope} is indeed possible, then we have support for H1 and a \isi{movement} analysis of \textit{amba}{}-RCs. But if such inverse \isi{scope} turns out to not be possible with doubly-embedded RCs, then everything else being equal, we (a) have a reason to reject a \isi{movement} analysis under H1, and (b) have evidence in support of a non-\isi{movement} analysis under H1. 

\largerpage[-2]
In sum, we want to construct examples of doubly-embedded \textit{amba}{}-RCs in which we expect long-distance QR to be impossible given the conditions in \citet{Fox2000}. If inverse \isi{scope} is possible, then we have support \textit{for} H1 and a \isi{movement} analysis (because QR is not relevant, with inverse \isi{scope} being possible via interpreting a full lower copy of the moved RC-head), and \textit{against} H2 and a non-\isi{movement} analysis, which relies on QR being possible. In contrast, if inverse \isi{scope} is impossible, then we have support \textit{for} a non-\isi{movement} analysis under either H1 or H2, and \textit{against} H1 and a \isi{movement} analysis. Thus, testing for inverse \isi{scope} becomes a way of testing for raising or base-generating the head in \textit{amba}{}-RCs. Again, our goal is ultimately not to decide between H1 and H2, but to use these hypotheses as a tool for identifying \isi{head raising} and accounting for our data set. Anticipating the discussion below, though, we will see evidence for \isi{head raising}, and thus evidence for H1 and against H2.

Methodologically, our approach here builds on that in \citet{Fox2000}, which also uses the absence/presence of some QR dependencies as part of a test for diagnosing other QR dependencies. We broaden the empirical \isi{focus} of this approach with the aim of implicating the potential of QR out of an RC in a test for whether the RC-head has itself moved out of the RC. Again, we are not aware of any previous literature that has applied this treatment to long-distance QR out of RCs. 

In \sectref{sec:gould:4.2}, we review the conditions on QR given the discussion in \citet{Fox2000}, and then test our hypotheses with the relevant examples of doubly-embedded RCs.

\subsection{Testing the hypotheses}\label{sec:gould:4.2}
\largerpage[-3]
In this section we consider novel data from \ili{Swahili} in order to implement the test mentioned in the previous section, which involves inverse \isi{scope} in doubly-embedded RCs. Recall that the test involves seeing whether inverse \isi{scope} is possible in a structure where we do not expect long-distance QR (LDQR) to be possible as per \citet{Fox2000}. Such a test can be used to argue for or against a \isi{movement} analysis of \textit{amba}{}-RCs, and we will see in this section that our test pushes us toward adopting the analysis in \REF{ex:gould:5}, namely that there is \isi{head raising} in all \textit{amba}{}-RCs. Before presenting the test results, we begin with a review of the conditions in \citet{Fox2000} under which LDQR is possible. Again, we are interested in testing examples in which LDQR should not be possible, as the non-\isi{movement} analysis of inverse \isi{scope} would crucially rely on this kind of QR. Reviewing these conditions is thus crucial for laying the groundwork for and understanding the test itself. Then after discussing the test results, we consider and reject an alternative analysis of the results, according to which \ili{Swahili} simply does not follow all the constraints on LDQR.


A second constraint on QR is a \isi{locality} constraint. \citet[23, 63]{Fox2000} suggests that each iteration of QR of a quantified expression Q must adjoin Q to the closest clause-denoting constituent that dominates Q before QR. We understand a clause-denoting constituent to be a closed proposition that is a maximal projection (i.e. a projection that is maximal in all regards other than the adjunction involved in QR). An example of such a clause-denoting constituent that could be adjoined to would be the maximal projection of TP, which is a saturated predicate before adjunction.\footnote{A related point is whether \isi{scope economy} and the \isi{locality} constraint apply to interpreting a full lower copy of \isi{movement}. Based on \citet[p. 23; n. 6, p. 23]{Fox2000}, we can say that \isi{scope economy} does apply, but that the \isi{locality} constraint does not. Thus semantic equivalence must not hold between the relevant \isi{scope} relations with regard to interpreting a higher copy in a copy-chain or a full lower copy in that copy-chain. This lack of semantic equivalence is found in all the examples in this paper where we propose a full lower copy is interpreted. So long as \isi{scope economy} holds, though, a full lower copy can be interpreted without regard to what kinds of projections intervene between the higher and lower copies of a copy-chain.\label{fnlabel}}

Let us now consider schematically in \REF{ex:gould:9} what QR from a \isi{relative clause} would look like and how it could satisfy these constraints. In \REF{ex:gould:9}, the RC-head is Q\textsubscript{1}, which corresponds to the gap in the object position of the relative. Next, Q\textsubscript{2} in \REF{ex:gould:9} is the \isi{subject} inside the RC. Q\textsubscript{2} then undergoes QR (indicated by a strikethrough) to adjoin to a clause-denoting constituent outside the relative that is structurally higher than the RC-head Q\textsubscript{1}. 

\ea\label{ex:gould:9}Proposal for Quantifier Raising from an \textit{amba}{}-RC: \ding{51} Q\textsubscript{2} > Q\textsubscript{1}
\begin{quote}
Q\textsubscript{2-Subj} [ \textbf{Q\textsubscript{1-Obj}} \textit{amba}{}-\textsc{agr} [ \sout{Q\textsubscript{2-Subj}} … \textit{e}\textsubscript{Obj} ] ]
\end{quote}
\z

This QR will be licensed as follows. First, the new \isi{scope} relation Q\textsubscript{2} > Q\textsubscript{1} must establish a new meaning (\isi{scope economy}). Second, there must be no clause-denoting maximal projection between the position Q\textsubscript{2} undergoes QR from and the position of the RC-head (\isi{locality}). This \isi{locality} constraint can be satisfied if we assume subjects in \ili{Swahili} occupy a high structural position within the clause, say at the TP level (cf. \citealt{Ngonyani2006}), such that no clause-denoting intervening maximal projection of this sort occurs between the embedded \isi{subject} and the RC-head. We will indeed assume that a configuration such as \REF{ex:gould:9} licenses LDQR in \ili{Swahili} as per the discussion in \citet{Fox2000}.\footnote{Recall that we assume \textit{amba}+\textsc{agr} is semantically vacuous, in which case the CP of the RC is simply an open proposition \citep[cf.][]{Heim1998}. Consequently, the CP level of the RC is not an intervening clause-denoting constituent.} 

Note that the configuration in \REF{ex:gould:9} is precisely the sort of analysis that would allow for inverse \isi{scope} in \REF{ex:gould:6}, which involves a single \textit{amba}{}-RC. 

In our test related to our hypotheses in \REF{ex:gould:8}, though, we will consider the possibility of inverse \isi{scope} in examples that involve two manipulations to the schema in \REF{ex:gould:9}. First, we will have Q\textsubscript{1} in \REF{ex:gould:9} be the highest RC-head of a doubly-embedded RC construction. As discussed in \sectref{sec:gould:4.1}, under a non-\isi{movement} analysis, turning \REF{ex:gould:9} into a doubly-embedded RC construction would force Q\textsubscript{2} in \REF{ex:gould:9}, if it is merged in the most deeply embedded RC, to undergo QR in order to take \isi{scope} over Q\textsubscript{1}.  Second, we will manipulate \REF{ex:gould:9} such that QR of Q\textsubscript{2} would be possible only by violating the \isi{locality} constraint. This manipulation is an attempt to eliminate base generation of the RC-head as a possible analysis. If inverse \isi{scope} is still possible, but if \isi{locality} is violated, then we have reason to think that LDQR is not taking place. Our conclusion, then, would be in favor of H1 and a \isi{movement} analysis, according to which inverse \isi{scope} is possible by raising the RC-head and interpreting a full lower copy of that head.

The crucial data are given in \REF{ex:gould:10}, which contain doubly-embedded \textit{amba}{}-RCs (but see note 12 below for a potential complication with \REF{ex:gould:10b}). We see that the embedded universal can take \isi{scope} over the numeral in the RC-head, resulting in a distributed reading. 

\ea\label{ex:gould:10}
Inverse \isi{scope} possible in doubly-embedded \textit{amba}{}-RC: \ding{51} $\forall$ > 2\\
\ea\label{ex:gould:10a}
\gll Ni-li-wa-it-a \textbf{[} \textbf{wa-gonjwa} \textbf{wa-wili} \textbf{]\textsubscript{j}} amba-o duka la dawa hi-li li-li-m-p-a vi-donge \textbf{[} \textbf{kila} \textbf{daktari} \textbf{]\textsubscript{i}} amba-ye [ e\textsubscript{i} a-li-wa-pim-a e\textsubscript{j} ].\\
\textsc{1\textsuperscript{st}}\textsc{.sg-pst-2o}{}-call-\textsc{fv} [ 2-patient 2-two ]\textsubscript{j} amba{}-\textsc{2agr} store of medicine  \textsc{dem}{}-5 5-\textsc{pst-1io-}give-\textsc{fv} 8-pill [ every doctor ]\textsubscript{i} amba{}-\textsc{1agr} [ e\textsubscript{i} \textsc{1s-pst-2o}{}-examine-\textsc{fv} e\textsubscript{j} ]\\
\glt ‘I called the two patients that this pharmacy gave pills to every doctor that treated (them).’

\ex\label{ex:gould:10b}
\gll Ni-li-wa-it-a \textbf{[} \textbf{wa-gonjwa} \textbf{wa-wili} \textbf{]\textsubscript{j}} amba-o ni-na-m-fahamu \textbf{[} \textbf{kila} \textbf{daktari} \textbf{]\textsubscript{i}} amba-ye [ e\textsubscript{i} a-li-wa-pim-a e\textsubscript{j} ].\\
\textsc{1\textsuperscript{st}}\textsc{.sg-pst}{}-2\textsc{o}{}-call-\textsc{fv} [ 2-patient 2-two ]\textsubscript{j} amba{}-2\textsc{agr} \textsc{1\textsuperscript{st}}\textsc{.sg-prs-1o}{}-know [ every doctor ]\textsubscript{i} amba{}-1\textsc{agr} [ e\textsubscript{i} \textsc{1s-pst-2o}{}-examine-\textsc{fv} e\textsubscript{j} ]\\
\glt ‘I called the two patients that I know every doctor who treated (them).’
\z
\z

Importantly, we claim that for QR to result in inverse \isi{scope} in \REF{ex:gould:10}, the QR would necessarily involve violating a constraint on QR. To see this, first note that the \isi{universal quantifier} is now the \isi{indirect object} of the verb ‘give’ in the higher \textit{amba}{}-RC in \REF{ex:gould:10a}, and the \isi{direct object} of the verb ‘know’ in the higher \textit{amba}{}-RC in \REF{ex:gould:10b}. This contrasts with (\ref{ex:gould:6}--\ref{ex:gould:9}), where the universal is in an embedded \isi{subject} position. Crucially, the \isi{subject} of ‘give’ is a definite description, and the \isi{subject} of ‘know’ is an indexical. We assume that the \isi{subject} of ‘know’ in \REF{ex:gould:10b} is represented in the syntax with a \textit{pro} that occupies the same structural position as the overt DP \isi{subject} of ‘give’ in \REF{ex:gould:10a}, making these two examples highly parallel. We further assume that in the \ili{Swahili} data here, QR over a definite description or a \isi{pronoun} does not establish a new meaning and that QR over such an element would not by itself satisfy \isi{scope economy}. Now, in order for the various iterations of QR to proceed locally in \REF{ex:gould:10}, QR of the universal would have to first move from an interpretable position (by hypothesis, the \textit{v}P edge) and adjoin above the \isi{subject} ‘this pharmacy’ or \textit{pro} at the TP layer (a clause-denoting constituent) of the ‘give’-clause or ‘know’-clause, before subsequently adjoining to a position higher than the RC-head with the numeral. This is shown schematically in \REF{ex:gould:11}, where adjunction positions for QR are underlined and indexed. However, adjoining in this lower position (i.e. adjunction in position $\alpha $), as required by \isi{locality}, would not establish a new meaning, and thus would violate \isi{scope economy}.\footnote{There is a potential complication in \REF{ex:gould:10b} involving the position of the verb ‘know’. As \citet[65]{Fox2000} discusses, QR over a verb such as ‘know’ can satisfy \isi{scope economy} by establishing a new scopal relation with the intensional verb. For our purposes, this is only relevant if the verb in \ili{Swahili} raises to a relatively high position. For example, if the verb ‘know’ in (\ref{ex:gould:10b}--\ref{ex:gould:11b}) raises to T, then \isi{scope economy} via QR and adjunction to TP would be satisfied because even though the \isi{universal quantifier} would not establish a new meaning with respect to \textit{pro}, it could do so with respect to ‘know’. However, if the verb in \ili{Swahili} raises only to some lower position, such as \textit{v} or some aspectual head (cf. \citealt{Ngonyani2006}), then the discussion in the main text remains unaffected. Regardless of the height of \isi{verb movement} in \ili{Swahili}, though, the argument presented here based on \REF{ex:gould:10a}, which involves the non-intensional verb ‘give’, still stands.} Conversely, if QR skipped over position $\alpha $ (thereby satisfying \isi{scope economy} with the new \isi{scope} relation established at position $\beta $), \isi{locality} would be violated.

\ea\label{ex:gould:11}
Local QR violating \isi{scope economy}:
\ea\label{ex:gould:11a}
\underline{\ \ \ }\textsubscript{$\beta $} … [ two patients [ \textit{amba}{}-\textsc{agr} [\textsubscript{TP} \underline{\ \ \ }\textsubscript{$\alpha $} this pharmacy [ … every doctor …] ] ] ] (cf. \REF{ex:gould:10a})

\ex\label{ex:gould:11b}
\underline{\ \ \ }\textsubscript{$\beta $} … [ two patients [ \textit{amba}{}-\textsc{agr} [\textsubscript{TP} \underline{\ \ \ }\textsubscript{$\alpha $} \textit{pro} [ … every doctor …] ] ] ]  (cf. \REF{ex:gould:10b})
\z
\z

Consequently, the data in \REF{ex:gould:10} constitute evidence against LDQR from \textit{amba}{}-RCs: inverse \isi{scope} appears to be possible even when the constraints on QR are violated. Accordingly, \REF{ex:gould:10} is evidence against H2 in \REF{ex:gould:8b} and the non-\isi{movement} analysis (under either H1 or H2) in \REF{ex:gould:2}. According to \REF{ex:gould:8b} and \REF{ex:gould:2}, we predict inverse \isi{scope} to be impossible, contrary to \REF{ex:gould:10}.

\newpage 
In contrast, \REF{ex:gould:10} is possible under H1 in \REF{ex:gould:8a} and the \isi{movement} analysis in \REF{ex:gould:5}. Recall that for inverse \isi{scope} to be possible under a \isi{head raising} account we simply need to interpret a full lower copy of the RC-head with the numeral in the lower \textit{amba}{}-RC, which is in a position below the universal in the higher \textit{amba}{}-RC. This analysis follows if \isi{movement} is possible out of doubly-embedded \textit{amba}{}-RCs. If this line of reasoning is on the track, and it is indeed supported by the empirical finding in \REF{ex:gould:10}, then we are forced to conclude in favor of H1, namely that \textit{amba}{}-RCs are not islands for \isi{overt movement}.

Before concluding this section, we discuss one final alternative to the \isi{movement} analysis under H1. For this final alternative we consider relaxing the constraints on QR. Suppose that there is cross-linguistic variation such that in some languages (e.g. \ili{English}, as per \citealt{Fox2000}) the \isi{locality} constraint is operative for QR, whereas in other languages, such as possibly \ili{Swahili}, the \isi{locality} constraint on QR is \textit{not} operative. What this would mean is that \isi{scope economy} would not be violated as in \REF{ex:gould:11}, because no intermediate step of QR is forced by \isi{locality}: this putative grammar for \ili{Swahili} would allow for QR to adjoin directly to the higher QR position $\beta $ in \REF{ex:gould:11}, without first adjoining to QR position $\alpha $. 

However, we reject this parametric view of \isi{locality} for the following reason. Under the null hypothesis, we would expect \isi{locality} to not be operative in other embedding constructions in \ili{Swahili}.\footnote{A reviewer asks whether it might be the case that there is also variation across syntactic constructions with regard to the \isi{locality} constraint. This view would hold that the null hypothesis mentioned in the main text is false because according to this view, the \isi{locality} constraint could be inoperative for QR out of, say RCs, but might be operative for QR out of other kinds of embedded clauses. As a way of countering this view, the reviewer suggests providing some independent evidence that QR out of RCs is sensitive to the \isi{locality} constraint. We believe that such independent evidence can be found, in part, by looking at RCs in \ili{English}. Consider the example in (i) below.
\begin{enumerate} \item[(i)] I called the two journalists that described the award that Obama gave every soldier. (*$\forall$>2)
\end{enumerate}
Inverse \isi{scope} of the universal over the numeral appears to be impossible. This is unexpected if the \isi{locality} constraint did not apply to RCs. Indeed, the only kind of \isi{embedded clause} that the universal would have to QR out of is an RC. If the \isi{locality} constraint did not apply, the \isi{universal quantifier} could QR to a position higher than the numeral, where it could take \isi{scope} over the numeral and satisfy \isi{scope economy}. However, the lack of inverse \isi{scope} is expected given a \isi{locality} constraint on QR. With such a constraint, the \isi{universal quantifier} would have to adjoin to the TP that immediately dominates \textit{Obama}. Such a step of QR, though, would violate \isi{scope economy}, as no new meaning results from the universal taking \isi{scope} over a name. Consequently any further QR in (i) is ruled out, and inverse \isi{scope} becomes impossible.} We can test for this with regular sentential complements to see whether a \isi{universal quantifier} can QR from the embedded to the matrix clause.\footnote{We test this by looking at whether QR is possible with a universally quantified \isi{direct object} in \REF{ex:gould:12}, and with a universally quantified applied object in \REF{ex:gould:13}. In applying this test, and in drawing conclusions from the test data, we assume that there is no relevant difference involving the possibility of QR with respect to these two types of objects. Further research can look more carefully at whether there are any relevant differences between the object types with regard to QR that might act as a potential confound in our application of the test here.} As a baseline, we first give a simple matrix transitive example in \REF{ex:gould:12} to show that an object can indeed take \isi{scope} over a \isi{subject} in \ili{Swahili} independent of RC constructions. We assume that inverse \isi{scope} is possible in \REF{ex:gould:12} via QR. This QR respects \isi{scope economy} and is local (from, say, the \textit{v}P to the TP level).

\ea\label{ex:gould:12}
Inverse \isi{scope} of quantifiers possible in simple transitives: \ding{51} $\forall$ > 2\\
\gll Wa-vulana wa-wili wa-na-m-pend-a kila m-sichana.\\
2-boy 2-two \textsc{2s-prs}{}-\textsc{1o}{}-like-\textsc{fv} every 1-girl\\
\glt ‘Two boys like every girl.’
\z

The crucial data point is \REF{ex:gould:13}. There the universal is embedded as an applied object inside a sentential complement.\footnote{In a manuscript version of this paper, we had transcribed the embedded verb without an applicative suffix, but we suspect that this was a typo. We thank a reviewer for pointing this out.} Importantly, the universal cannot take \isi{scope} over the numeral \isi{subject} in the matrix clause (which is base generated there, as there is no embedded gap), and a distributed reading is not possible. 

\ea\label{ex:gould:13}
Inverse \isi{scope} of quantifiers not possible from \isi{complement clause}: *$\forall$>2\\
\gll Wana-funzi wa-wili wa-li-dai kwamba Juma a-li-m-fok-e-a kila mw-alimu.\\
2-student 2-two 2s-\textsc{pst}{}-claim that Juma \textsc{1s-pst-1o}{}-scold-\textsc{appl}{}-\textsc{fv} every 1-teacher\\
\glt ‘Two students claimed that Juma scolded every teacher.’
\z

Note that such a distributed reading would be possible if there were no \isi{locality} constraint on QR: this unconstrained QR would obey \isi{scope economy} (giving rise to the distributed reading) but would not have to proceed cyclically in a local manner by adjoining at the TP level (above the \isi{subject}) in the ‘scold’-clause. 

The fact that inverse \isi{scope} is not possible in \REF{ex:gould:13} supports the conclusion that \isi{locality} is an operative constraint on QR in \ili{Swahili}. Locality forces local QR above \textit{Juma} within the \isi{embedded clause}, but as \textit{Juma} is simply a name, no new meaning is established and \isi{scope economy} is violated. Consequently, it is not possible to have further QR of the universal above the matrix \isi{subject}.

In sum, what \REF{ex:gould:13} suggests with regard to this final alternative analysis is that what this alternative would call LDQR in \REF{ex:gould:10} is not really QR at all. Accordingly, QR in \ili{Swahili} (as in \ili{English}) would be constrained by \isi{scope economy} and \isi{locality}. Further, the inverse \isi{scope} relation in \REF{ex:gould:10} is established by interpreting a full lower copy of the RC-head within the relative. This interpretive option is possible because \isi{movement} of the RC-head involves leaving a copy of the head that can be interpreted within the relative. Thus on the basis of the detailed \isi{scope} data considered in this section we conclude that \textit{amba}{}-RCs involve \isi{head raising} and are consequently not islands for \isi{overt movement} (cf. \citealt{Sichel2014} for a similar claim regarding certain \isi{relative clause} constructions beyond \ili{Swahili}). In the following section we provide an additional data point involving another \isi{long-distance dependency} that provides potential support for the claim that \textit{amba}{}-RCs are not islands for \isi{overt movement}.

\section{Another long-distance dependency}\label{sec:gould:5}

In the previous section we provided an argument in favor of a \isi{movement} analysis of \textit{amba}{}-RCs along the lines of \REF{ex:gould:5}. We also claimed that \textit{amba}{}-RCs are not islands for \isi{overt movement}. Consequently, it is possible to relativize another RC-head by moving it past an RC-head+\textit{amba} boundary, as in the case of doubly-embedded \textit{amba}{}-RCs. The null hypothesis is that all instances of \isi{overt movement} can move past this boundary (i.e. not just in cases of relativization). To the extent that we find such evidence, it supports our analysis of \textit{amba}{}-RCs and their status as non-islands. In this brief section, we present preliminary data suggesting this null hypothesis is on the right track. What we see here is that another type of \isi{long-distance dependency} that involves displacement is also possible across an \textit{amba}{}-RC boundary. Our example of this involves the case of long-distance topicalization in (14), where we see topicalization of an argument past the RC-head+\textit{amba} boundary.

\ea\label{ex:gould:14}
Topicalization out of an \textit{amba}{}-RC is possible\\
\gll [ Ki-tabu hi-ki ]\textsubscript{j}, ni-na-m-fahamu m-tu\textsubscript{i} amba-ye [ e\textsubscript{i} a-li-ki-andik-a e\textsubscript{j} ]. \\
[ 7-book \textsc{dem}{}-7 ]\textsubscript{j} \textsc{1\textsuperscript{st}}\textsc{.sg-prs-1o}{}-know 1-person\textsubscript{i} amba{}-\textsc{1agr} [ e\textsubscript{i} \textsc{1s-pst-7o}{}-write-\textsc{fv} e\textsubscript{j} ].\\
\glt ‘This book, I know the person who wrote (it).’
\z

It remains to be shown that topicalization in \ili{Swahili} does in fact involve \isi{movement}. This could involve, for example, repeating the argument of inverse \isi{scope} from the previous section with appropriately modified versions of \REF{ex:gould:14}. At this point we have no further data that would shed light on this issue, but given that the dependency in \REF{ex:gould:14} involves displacement, it is a likely candidate for \isi{overt movement}. Thus \REF{ex:gould:14} is consistent with our claim that \textit{amba}{}-RCs are not islands, and this would be a welcome finding should topicalization indeed involve \isi{movement} in \ili{Swahili}. Given our current data limitations, though, we will leave this as a topic for future research.

\section{Final remarks}\label{sec:gould:6}

The literature on \ili{Swahili} has offered contrasting accounts of \rephrase{\textit{amba}{}-relative clauses}{relative clauses with \textit{amba} } that are based on separate types of evidence. In this paper, we took seriously the challenge of attempting to integrate these different sources of evidence into a unified analysis of these RCs. Our investigation hinged on a detailed look at novel data involving inverse \isi{scope} relationships between quantifiers. Based on these data, we concluded that \textit{amba}{}-RCs involve moving the RC-head from a position inside the relative to a position outside it, and that \textit{amba}{}-RCs are not islands for \isi{overt movement}. 

To be sure, the discussion here should be seen as just an initial step of much broader and more far-reaching potential investigations of RCs in \ili{Swahili}. For instance, as regards \textit{amba}{}-RCs, the binding fact we illustrated in \REF{ex:gould:7} can be explored in the same rigorous way as was done in \sectref{sec:gould:4.2}. A more analytical question that we have not considered concerns the internal structure of \textit{amba}{}-RCs. In particular, can the absence of island effects in \ili{Swahili} versus the presence of island effects in \ili{English} RCs be tied to some structural difference of the RC itself? More generally, we have not looked at the other types of RC constructions in \ili{Swahili} (i.e. non-\textit{amba}{}-RCs; cf. \citealt{Ngonyani2001}), and it remains to be seen to what extent they can be assimilated to our overall analysis presented here. Our hope is that the systematic, empirical and analytical tack we have followed here can be used fruitfully for the future study of \ili{Swahili} RCs, as well as those found in other languages.

\section*{Acknowledgements}
We would like to thank audiences at The University of Kansas and ACAL 47 for their feedback, Michael Yoshitaka Erlewine for his help, two anonymous reviewers, and the editors of this volume. Above all we thank David Mburu for being a patient teacher and sharing his knowledge of his language with us. This paper is dedicated to his memory.

\section*{Abbreviations}
\begin{tabularx}{.45\textwidth}{ll}
1-8 & noun classes\\
{1\textsuperscript{st}} & \isi{first person}\\
{2\textsuperscript{nd}} & \isi{second person}\\
{3\textsuperscript{rd}} & \isi{third person}\\
\textsc{agr} & agreement\\
\textsc{appl} & applicative\\
\textsc{dem}  & demonstrative\\
\textsc{fut} & future\\
\textsc{fv} & \isi{final vowel}\\
\end{tabularx}
\begin{tabularx}{.45\textwidth}{ll}
\textsc{hab} & habitual \\
\textsc{io} & \isi{indirect object}\\
\textsc{o} & object\\
\textsc{prs} & present\\
\textsc{pst} & past\\
\textsc{refl} & reflexive\\
\textsc{s} & \isi{subject}\\
\textsc{sg} & singular\\
\\
\end{tabularx}

\sloppy
\printbibliography[heading=subbibliography,notkeyword=this]

\end{document}
