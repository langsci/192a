\documentclass[output=paper,modfonts,nonflat,
colorlinks, citecolor=brown,
draftmode
]{langsci/langscibook} 

\IfFileExists{../localpackages.tex}{%
  \setcounter{chapter}{22}
  \usepackage{pifont}
\usepackage{savesym}

\savesymbol{downingtriple}
\savesymbol{downingdouble}
\savesymbol{downingquad}
\savesymbol{downingquint}
\savesymbol{suph}
\savesymbol{supj}
\savesymbol{supw}
\savesymbol{sups}
\savesymbol{ts}
\savesymbol{tS}
\savesymbol{devi}
\savesymbol{devu}
\savesymbol{devy}
\savesymbol{deva}
\savesymbol{N}
\savesymbol{Z}
\savesymbol{circled}
\savesymbol{sem}
\savesymbol{row}
\savesymbol{tipa}
\savesymbol{tableauxcounter}
\savesymbol{tabhead}
\savesymbol{inp}
\savesymbol{inpno}
\savesymbol{g}
\savesymbol{hanl}
\savesymbol{hanr}
\savesymbol{kuku}
\savesymbol{ip}
\savesymbol{lipm}
\savesymbol{ripm}
\savesymbol{lipn}
\savesymbol{ripn} 
% \usepackage{amsmath} 
% \usepackage{multicol}
\usepackage{qtree} 
\usepackage{tikz-qtree,tikz-qtree-compat}
% \usepackage{tikz}
\usepackage{upgreek}


%%%%%%%%%%%%%%%%%%%%%%%%%%%%%%%%%%%%%%%%%%%%%%%%%%%%
%%%                                              %%%
%%%           Examples                           %%%
%%%                                              %%%
%%%%%%%%%%%%%%%%%%%%%%%%%%%%%%%%%%%%%%%%%%%%%%%%%%%%
% remove the percentage signs in the following lines
% if your book makes use of linguistic examples
\usepackage{tipa}  
\usepackage{pstricks,pst-xkey,pst-asr}

%for sande et al
\usepackage{pst-jtree}
\usepackage{pst-node}
%\usepackage{savesym}


% \usepackage{subcaption}
\usepackage{multirow}  
\usepackage{./langsci/styles/langsci-optional} 
\usepackage{./langsci/styles/langsci-lgr} 
\usepackage{./langsci/styles/langsci-glyphs} 
\usepackage[normalem]{ulem}
%% if you want the source line of examples to be in italics, uncomment the following line
% \def\exfont{\it}
\usetikzlibrary{arrows.meta,topaths,trees}
\usepackage[linguistics]{forest}
\forestset{
	fairly nice empty nodes/.style={
		delay={where content={}{shape=coordinate,for parent={
					for children={anchor=north}}}{}}
}}
\usepackage{soul}
\usepackage{arydshln}
% \usepackage{subfloat}
\usepackage{langsci/styles/langsci-gb4e} 
   
% \usepackage{linguex}
\usepackage{vowel}

\usepackage{pifont}% http://ctan.org/pkg/pifont
\newcommand{\cmark}{\ding{51}}%
\newcommand{\xmark}{\ding{55}}%
 
 
 %Lamont
 \makeatletter
\g@addto@macro\@floatboxreset\centering
\makeatother

\usepackage{newfloat} 
\DeclareFloatingEnvironment[fileext=tbx,name=Tableau]{tableau}
  %add all your local new commands to this file
\newcommand{\downingquad}[4]{\parbox{2.5cm}{#1}\parbox{3.5cm}{#2}\parbox{2.5cm}{#3}\parbox{3.5cm}{#4}}
\newcommand{\downingtriple}[3]{\parbox{4.5cm}{#1}\parbox{3cm}{#2}\parbox{3cm}{#3}}
\newcommand{\downingdouble}[2]{\parbox{4.5cm}{#1}\parbox{6cm}{#2}}
\newcommand{\downingquint}[5]{\parbox{1.75cm}{#1}\parbox{2.25cm}{#2}\parbox{2cm}{#3}\parbox{3cm}{#4}\parbox{2cm}{#5}}
\newcolumntype{Y}{>{\centering\arraybackslash}X}
\newcolumntype{T}{>{\centering\arraybackslash}m{2cm}}

%commands for Kusmer paper below
\newcommand{\ip}{$\upiota$}
\newcommand{\lipm}{(\_{\ip-Max}}
\newcommand{\ripm}{)\_{\ip-Max}}
\newcommand{\lipn}{(\_{\ip}}
\newcommand{\ripn}{)\_{\ip}}
\renewcommand{\_}[1]{\textsubscript{#1}}


%commands for Pillion paper below
\newcommand{\suph}{\textipa{\super h}}
\newcommand{\supj}{\textipa{\super j}}
\newcommand{\supw}{\textipa{\super w}}
\newcommand{\ts}{\textipa{\t{ts}}}
\newcommand{\tS}{\textipa{\t{tS}}}
\newcommand{\devi}{\textipa{\r*i}}
\newcommand{\devu}{\textipa{\r*u}}
\newcommand{\devy}{\textipa{\r*y}}
\newcommand{\deva}{\textipa{\r*a}}
\renewcommand{\N}{\textipa{N}}
\newcommand{\Z}{\textipa{Z}}
% 

%commands for Diercks paper below
\newcommand{\circled}[1]{\begin{tikzpicture}[baseline=(word.base)]
\node[draw, rounded corners, text height=8pt, text depth=2pt, inner sep=2pt, outer sep=0pt, use as bounding box] (word) {#1};
\end{tikzpicture}
}

%commands for Pesetsky paper below
% \newcommand{\sem}[2][]{\mbox{$[\![ $\textbf{#2}$ ]\!]^{#1}$}}
\newcommand{\sem}[2][]{\mbox{$[[ $\textbf{#2}$ ]]^{#1}$}}

% \newcommand{\ripn}{{\color{red}ripn}}%this is used but never defined. Please update the definition



%commands for Lamont paper below
\newcommand{\row}[4]{
	#1. & 
    /{#2}/ & 
    [{#3}] & 
    `#4' \\ 
}
%\newcounter{tableauxcounter}
\newcommand{\tabhead}[2]{
%     \captionsetup{labelformat=empty}
%     \stepcounter{tableauxcounter}
%     \addtocounter{table}{-1}
% 	\centering
% 	\caption{Tableau \thetableauxcounter: #1}
	\caption{#1}
	\label{#2}
}
\newcommand{\candref}[2]{{(\ref{#1}#2)}}
\newcommand{\tableauref}[1]{{Tableau~\ref{#1}}}
% tableaux
\newcommand{\inp}[1]{\multicolumn{2}{|l||}{{#1}}}
\newcommand{\inpno}[1]{\multicolumn{2}{|l||}{#1}}
\newcommand{\g}{\cellcolor{lightgray}}
\newcommand{\hanl}{\HandLeft}
\newcommand{\hanr}{\HandRight}
\newcommand{\kuku}{Kuk\'{u}}

% \newcommand{\nocaption}[1]{{\color{red} Please provide a caption}}

% \providecommand{\biberror}[1]{{\color{red}#1}}

\definecolor{RED}{cmyk}{0.05,1,0.8,0}


\newfontfamily\amharicfont[Script = Ethiopic, Scale = 1.0]{AbyssinicaSIL}
\newcommand{\amh}[1]{{\amharicfont #1}}

% 
% %Gjersoe
\usepackage{textgreek}
% 
\newcommand{\viol}{\fontfamily{MinionPro-OsF}\selectfont\rotatebox{60}{$\star$}}
\newcommand{\myscalex}{0.45}
\newcommand{\myscaley}{0.65}
%\newcommand{\red}[1]{\textcolor{red}{#1}}
%\newcommand{\blue}[1]{\textcolor{blue}{#1}}
\newcommand{\epen}[1]{\colorbox{jgray}{#1}}
\newcommand{\hand}{{\normalsize \ding{43}}}
\definecolor{jgray}{gray}{0.8} 
\usetikzlibrary{positioning}
\usetikzlibrary{matrix}
\newcommand{\mora}{\textmu\xspace}
\newcommand{\si}{\textsigma\xspace}
\newcommand{\ft}{\textPhi\xspace}
\newcommand{\tone}{\texttau\xspace}
\newcommand{\word}{\textomega\xspace}
% \newcommand{\ts}{\texttslig}
\newcommand{\fns}{\footnotesize}
\newcommand{\ns}{\normalsize}
\newcommand{\vs}{\vspace{1em}}
\newcommand{\bs}{\textbackslash}   % backslash
\newcommand{\cmd}[1]{{\bf \color{red}#1}}   % highlights command
\newcommand{\scell}[2][l]{\begin{tabular}[#1]{@{}c@{}}#2\end{tabular}}
% \interfootnotelinepenalty=10000

% --- Snider Representations --- %

\newcommand{\RepLevelHh}{
\begin{minipage}{0.10\textwidth}
\begin{tikzpicture}[xscale=\myscalex,yscale=\myscaley]
%\node (syl) at (0,0) {Hi};
\node (Rt) at (0,1) {o};
\node (H) at (-0.5,2) {H};
\node (R) at (0.5,3) {h};
%\draw [thick] (syl.north) -- (Rt.south) ;
\draw [thick] (Rt.north) -- (H.south) ;
\draw [thick] (Rt.north) -- (R.south) ;
\end{tikzpicture}
\end{minipage}
}

\newcommand{\RepLevelLh}{
\begin{minipage}{0.10\textwidth}
\begin{tikzpicture}[xscale=\myscalex,yscale=\myscaley]
%\node (syl) at (0,0) {Mid2};
\node (Rt) at (0,1) {o};
\node (H) at (-0.5,2) {L};
\node (R) at (0.5,3) {h};
%\draw [thick] (syl.north) -- (Rt.south) ;
\draw [thick] (Rt.north) -- (H.south) ;
\draw [thick] (Rt.north) -- (R.south) ;
\end{tikzpicture}
\end{minipage}
}

\newcommand{\RepLevelHl}{
\begin{minipage}{0.10\textwidth}
\begin{tikzpicture}[xscale=\myscalex,yscale=\myscaley]
%\node (syl) at (0,0) {Mid1};
\node (Rt) at (0,1) {o};
\node (H) at (-0.5,2) {H};
\node (R) at (0.5,3) {l};
%\draw [thick] (syl.north) -- (Rt.south) ;
\draw [thick] (Rt.north) -- (H.south) ;
\draw [thick] (Rt.north) -- (R.south) ;
\end{tikzpicture}
\end{minipage}
}

\newcommand{\RepLevelLl}{
\begin{minipage}{0.10\textwidth}
\begin{tikzpicture}[xscale=\myscalex,yscale=\myscaley]
%\node (syl) at (0,0) {Lo};
\node (Rt) at (0,1) {o};
\node (H) at (-0.5,2) {L};
\node (R) at (0.5,3) {l};
%\draw [thick] (syl.north) -- (Rt.south) ;
\draw [thick] (Rt.north) -- (H.south) ;
\draw [thick] (Rt.north) -- (R.south) ;
\end{tikzpicture}
\end{minipage}
}

% --- Representations --- %

\newcommand{\RepLevel}{
\begin{minipage}{0.10\textwidth}
\begin{tikzpicture}[xscale=\myscalex,yscale=\myscaley]
\node (syl) at (0,0) {\textsigma};
\node (Rt) at (0,1) {o};
\node (H) at (-0.5,2) {\texttau};
\node (R) at (0.5,3) {\textrho};
\draw [thick] (syl.north) -- (Rt.south) ;
\draw [thick] (Rt.north) -- (H.south) ;
\draw [thick] (Rt.north) -- (R.south) ;
\end{tikzpicture}
\end{minipage}
}

\newcommand{\RepContour}{
\begin{minipage}{0.10\textwidth}
\begin{tikzpicture}[xscale=\myscalex,yscale=\myscaley]
\node (syl) at (0,0) {\textsigma};
\node (Rt) at (0,1) {o};
\node (H) at (-0.5,2) {\texttau};
\node (R) at (0.5,3) {\textrho};
\node (Rt2) at (1.5,1.0) {o};
%\node (H2) at (1.0,2) {$\tau$};
%\node (R2) at (2.0,2.5) {R};
\draw [thick] (syl.north) -- (Rt.south) ;
\draw [thick] (Rt.north) -- (H.south) ;
\draw [thick] (Rt.north) -- (R.south) ;
\draw [thick] (syl.north) -- (Rt2.south) ;
%\draw [thick] (Rt2.north) -- (H2.south) ;
%\draw [thick] (Rt2.north) -- (R2.south) ;
\end{tikzpicture}
\end{minipage}
}


% --- OT constraints --- %

\newcommand{\IllustrationDown}{
\begin{minipage}{0.09\textwidth}
\begin{tikzpicture}[xscale=0.7,yscale=0.45]
\node (reg) at (0,0.75) {{\small \textalpha}};
\node (arrow) at (0,0) {{\fns $\downarrow$}};
\node (Rt) at (0,-0.75) {{\small \textbeta}};
\end{tikzpicture}
\end{minipage}
}

\newcommand{\IllustrationUp}{
\begin{minipage}{0.09\textwidth}
\begin{tikzpicture}[xscale=0.7,yscale=0.45]
\node (reg) at (0,0.75) {{\small \textalpha}};
\node (arrow) at (0,0) {{\fns $\uparrow$}};
\node (Rt) at (0,-0.75) {{\small \textbeta}};
\end{tikzpicture}
\end{minipage}
}

\newcommand{\MaxAB}{
\begin{minipage}{0.09\textwidth}
\begin{tikzpicture}[xscale=0.6,yscale=0.4]
\node (max) at (0,0) {{\small \textsc{Max}}};
\node (reg) at (0.75,0.5) {{\fns \textalpha}};
\node (arrow) at (0.75,0) {{\tiny $\downarrow$}};
\node (Rt) at (0.75,-0.5) {{\fns \textbeta}};
\end{tikzpicture}
\end{minipage}
}

\newcommand{\DepAB}{
\begin{minipage}{0.09\textwidth}
\begin{tikzpicture}[xscale=0.6,yscale=0.4]
\node (max) at (0,0) {{\small \textsc{Dep}}};
\node (reg) at (0.75,0.5) {{\fns \textalpha}};
\node (arrow) at (0.75,0) {{\tiny $\downarrow$}};
\node (Rt) at (0.75,-0.5) {{\fns \textbeta}};
\end{tikzpicture}
\end{minipage}
}

\newcommand{\DepHReg}{
\begin{minipage}{0.055\textwidth}
\begin{tikzpicture}[xscale=0.6,yscale=0.4]
\node (dep) at (0,0) {{\small \textsc{Dep}}};
\node (reg) at (0,-1.0) {{\small h}};
\end{tikzpicture}
\end{minipage}
}

\newcommand{\DepLReg}{
\begin{minipage}{0.055\textwidth}
\begin{tikzpicture}[xscale=0.6,yscale=0.4]
\node (dep) at (0,0) {{\small \textsc{Dep}}};
\node (reg) at (0,-1.0) {{\small l}};
\end{tikzpicture}
\end{minipage}
}

\newcommand{\DepReg}{
\begin{minipage}{0.055\textwidth}
\begin{tikzpicture}[xscale=0.6,yscale=0.4]
\node (dep) at (0,0) {{\small \textsc{Dep}}};
\node (reg) at (0,-1.0) {{\small \textrho}};
\end{tikzpicture}
\end{minipage}
}

\newcommand{\DepTRt}{
\begin{minipage}{0.1\textwidth}
\begin{tikzpicture}[xscale=0.6,yscale=0.4]
\node (dep) at (0,0) {{\small \textsc{Dep}}};
\node (t) at (0.75,0.5) {{\fns \texttau}};
\node (arrow) at (0.75,0) {{\tiny $\downarrow$}};
\node (Rt) at (0.75,-0.5) {{\fns o}};
\end{tikzpicture}
\end{minipage}
}

\newcommand{\MaxRegRt}{
\begin{minipage}{0.1\textwidth}
\begin{tikzpicture}[xscale=0.6,yscale=0.4]
\node (max) at (0,0) {{\small \textsc{Max}}};
\node (arrow) at (0.75,0) {{\tiny $\downarrow$}};
\node (Rt) at (0.75,-0.5) {{\fns o}};
\node (reg) at (0.75,0.5) {{\fns \textrho}};
\end{tikzpicture}
\end{minipage}
}

\newcommand{\RegToneByRt}{
\begin{minipage}{0.06\textwidth}
\begin{tikzpicture}[xscale=0.6,yscale=0.5]
\node[rotate=20] (arrow1) at (-0.15,0) {{\fns $\uparrow$}};
\node[rotate=340] (arrow2) at (0.15,0) {{\fns $\uparrow$}};
\node (Rt) at (0,-0.55) {{\small o}};
\node (reg) at (0.4,0.55) {{\small \textrho}};
\node (tone) at (-0.4,0.55) {{\small \texttau}};
\end{tikzpicture}
\end{minipage}
}

\newcommand{\RegToneBySyl}{
\begin{minipage}{0.06\textwidth}
\begin{tikzpicture}[xscale=0.6,yscale=0.5]
\node[rotate=20] (arrow1) at (-0.15,0) {{\fns $\uparrow$}};
\node[rotate=340] (arrow2) at (0.15,0) {{\fns $\uparrow$}};
\node (Rt) at (0,-0.55) {{\small \textsigma}};
\node (reg) at (0.4,0.55) {{\small \textrho}};
\node (tone) at (-0.4,0.55) {{\small \texttau}};
\end{tikzpicture}
\end{minipage}
}

\newcommand{\DepTone}{
\begin{minipage}{0.055\textwidth}
\begin{tikzpicture}[xscale=0.6,yscale=0.4]
\node (dep) at (0,0) {{\small \textsc{Dep}}};
\node (tone) at (0,-1.0) {{\small \texttau}};
\end{tikzpicture}
\end{minipage}
}

\newcommand{\DepTonalRt}{
\begin{minipage}{0.055\textwidth}
\begin{tikzpicture}[xscale=0.6,yscale=0.4]
\node (dep) at (0,0) {{\small \textsc{Dep}}};
\node (tone) at (0,-1.0) {{\small o}};
\end{tikzpicture}
\end{minipage}
}

\newcommand{\DepL}{
\begin{minipage}{0.055\textwidth}
\begin{tikzpicture}[xscale=0.6,yscale=0.4]
\node (dep) at (0,0) {{\small \textsc{Dep}}};
\node (tone) at (0,-1.0) {{\small L}};
\end{tikzpicture}
\end{minipage}
}

\newcommand{\DepH}{
\begin{minipage}{0.055\textwidth}
\begin{tikzpicture}[xscale=0.6,yscale=0.4]
\node (dep) at (0,0) {{\small \textsc{Dep}}};
\node (tone) at (0,-1.0) {{\small H}};
\end{tikzpicture}
\end{minipage}
}

\newcommand{\NoMultDiff}{{\small *loh}}
\newcommand{\Alt}{{\small \textsc{Alt}}}
\newcommand{\NoSkip}{{\small \scell{\textsc{No}\\\textsc{Skip}}}}


\newcommand{\RegDomRt}{
\begin{minipage}{0.030\textwidth}
\begin{tikzpicture}[xscale=0.6,yscale=0.5]
\node (arrow) at (0,0) {{\fns $\downarrow$}};
\node (Rt) at (0,-0.55) {{\small o}};
\node (reg) at (0,0.55) {{\small \textrho}};
\end{tikzpicture}
\end{minipage}
}

\newcommand{\DepRegRt}{
\begin{minipage}{0.1\textwidth}
\begin{tikzpicture}[xscale=0.6,yscale=0.4]
\node (dep) at (0,0) {{\small \textsc{Dep}}};
\node (arrow) at (0.75,0) {{\tiny $\downarrow$}};
\node (Rt) at (0.75,-0.5) {{\fns o}};
\node (reg) at (0.75,0.5) {{\fns \textrho}};
\end{tikzpicture}
\end{minipage}
}

% unused

\newcommand{\ToneByRt}{
\begin{minipage}{0.05\textwidth}
\begin{tikzpicture}[xscale=0.6,yscale=0.5]
\node (arrow) at (0,0) {{\fns $\uparrow$}};
\node (Rt) at (0,-0.55) {{\small o}};
\node (tone) at (0,0.55) {{\small \texttau}};
\end{tikzpicture}
\end{minipage}
}

\newcommand{\RegByRt}{
\begin{minipage}{0.05\textwidth}
\begin{tikzpicture}[xscale=0.6,yscale=0.5]
\node (arrow) at (0,0) {{\fns $\uparrow$}};
\node (Rt) at (0,-0.55) {{\small o}};
\node (reg) at (0,0.55) {{\small \textrho}};
\end{tikzpicture}
\end{minipage}
}

\newcommand{\ToneDomRt}{
\begin{minipage}{0.05\textwidth}
\begin{tikzpicture}[xscale=0.6,yscale=0.5]
\node (arrow) at (0,0) {{\fns $\downarrow$}};
\node (Rt) at (0,-0.55) {{\small o}};
\node (tone) at (0,0.55) {{\small \texttau}};
\end{tikzpicture}
\end{minipage}
}

% --- OT tableaus --- %

% Sec. 3.2, first tabl.

\newcommand{\OTHLInput}{
\begin{minipage}{0.17\textwidth}
\begin{tikzpicture}[xscale=\myscalex,yscale=\myscaley]
\node (tone) at (2,0) {(= H)};
\node (syl) at (0,0) {\textsigma};
\node (Rt) at (0,1) {o};
\node (H) at (-0.5,2) {H};
\node (R) at (0.5,3) {h};
\node (Rt2) at (1.5,1.0) {o};
%\node (H2) at (1.0,2) {\epen{L}};
\node (R2) at (2.0,3) {\blue{l}};
\draw [thick] (syl.north) -- (Rt.south) ;
\draw [thick] (Rt.north) -- (H.south) ;
\draw [thick] (Rt.north) -- (R.south) ;
\draw [thick] (syl.north) -- (Rt2.south) ;
%\draw [dashed] (Rt2.north) -- (H2.south) ;
%\draw [dashed] (Rt2.north) -- (R2.south) ;
\end{tikzpicture}
\end{minipage}
}

\newcommand{\OTHLWinner}{
\begin{minipage}{0.17\textwidth}
\begin{tikzpicture}[xscale=\myscalex,yscale=\myscaley]
\node (tone) at (2,0) {(= HL)};
\node (syl) at (0,0) {\textsigma};
\node (Rt) at (0,1) {o};
\node (H) at (-0.5,2) {H};
\node (R) at (0.5,3) {h};
\node (Rt2) at (1.5,1.0) {o};
\node (H2) at (1.0,2) {\epen{L}};
\node (R2) at (2.0,3) {\blue{l}};
\draw [thick] (syl.north) -- (Rt.south) ;
\draw [thick] (Rt.north) -- (H.south) ;
\draw [thick] (Rt.north) -- (R.south) ;
\draw [thick] (syl.north) -- (Rt2.south) ;
\draw [dashed] (Rt2.north) -- (H2.south) ;
\draw [dashed] (Rt2.north) -- (R2.south) ;
\end{tikzpicture}
\end{minipage}
}

\newcommand{\OTHLSpreadingHOnly}{
\begin{minipage}{0.17\textwidth}
\begin{tikzpicture}[xscale=\myscalex,yscale=\myscaley]
\node (tone) at (2,0) {(= HM)};
\node (syl) at (0,0) {\textsigma};
\node (Rt) at (0,1) {o};
\node (H) at (-0.5,2) {H};
\node (R) at (0.5,3) {h};
\node (Rt2) at (1.5,1.0) {o};
%\node (H2) at (1.0,2) {\epen{L}};
\node (R2) at (2.0,3) {\blue{l}};
\draw [thick] (syl.north) -- (Rt.south) ;
\draw [thick] (Rt.north) -- (H.south) ;
\draw [thick] (Rt.north) -- (R.south) ;
\draw [thick] (syl.north) -- (Rt2.south) ;
\draw [dashed] (Rt2.north) -- (R2.south) ;
\draw [dashed] (Rt2.north) -- (H.south) ;
\end{tikzpicture}
\end{minipage}
}

\newcommand{\OTHLInsertH}{
\begin{minipage}{0.17\textwidth}
\begin{tikzpicture}[xscale=\myscalex,yscale=\myscaley]
\node (tone) at (2,0) {(= HM)};
\node (syl) at (0,0) {\textsigma};
\node (Rt) at (0,1) {o};
\node (H) at (-0.5,2) {H};
\node (R) at (0.5,3) {h};
\node (Rt2) at (1.5,1.0) {o};
\node (H2) at (1.0,2) {\epen{H}};
\node (R2) at (2.0,3) {\blue{l}};
\draw [thick] (syl.north) -- (Rt.south) ;
\draw [thick] (Rt.north) -- (H.south) ;
\draw [thick] (Rt.north) -- (R.south) ;
\draw [thick] (syl.north) -- (Rt2.south) ;
\draw [dashed] (Rt2.north) -- (H2.south) ;
\draw [dashed] (Rt2.north) -- (R2.south) ;
\end{tikzpicture}
\end{minipage}
}

\newcommand{\OTHLOverwriting}{
\begin{minipage}{0.17\textwidth}
\begin{tikzpicture}[xscale=\myscalex,yscale=\myscaley]
\node (syl) at (0,0) {\textsigma};
\node (Rt) at (0,1) {o};
\node (H) at (-0.5,2) {H};
\node (R) at (0.5,3) {h};
\node (Rt2) at (1.5,1.0) {o};
%\node (H2) at (1.0,2) {\epen{L}};
\node (R2) at (2.0,3) {\blue{l}};
\draw [thick] (syl.north) -- (Rt.south) ;
\draw [thick] (Rt.north) -- (H.south) ;
\draw [thick] (Rt.north) -- (R.south) ;
\draw [thick] (syl.north) -- (Rt2.south) ;
%\draw [dashed] (Rt2.north) -- (H2.south) ;
\draw [dashed] (Rt.north) -- (R2.south) ;
\node (del) at (0.3,1.9) {\textbf{=}};
\end{tikzpicture}
\end{minipage}
}

\newcommand{\OTHLSpreading}{
\begin{minipage}{0.17\textwidth}
\begin{tikzpicture}[xscale=\myscalex,yscale=\myscaley]
\node (syl) at (0,0) {\textsigma};
\node (Rt) at (0,1) {o};
\node (H) at (-0.5,2) {H};
\node (R) at (0.5,3) {h};
\node (Rt2) at (1.5,1.0) {o};
%\node (H2) at (1.0,2) {\epen{L}};
\node (R2) at (2.0,3) {\blue{l}};
\draw [thick] (syl.north) -- (Rt.south) ;
\draw [thick] (Rt.north) -- (H.south) ;
\draw [thick] (Rt.north) -- (R.south) ;
\draw [thick] (syl.north) -- (Rt2.south) ;
%\draw [dashed] (Rt2.north) -- (H2.south) ;
\draw [dashed] (Rt2.north) -- (H.south) ;
\draw [dashed] (Rt2.north) -- (R.south) ;
\end{tikzpicture}
\end{minipage}
}

% Sec. 4.2, second tabl.: phrase-medial position

\newcommand{\OTHnoLInput}{
\begin{minipage}{0.17\textwidth}
\begin{tikzpicture}[xscale=\myscalex,yscale=\myscaley]
\node (tone) at (2,0) {(= H)};
\node (syl) at (0,0) {\textsigma};
\node (Rt) at (0,1) {o};
\node (H) at (-0.5,2) {H};
\node (R) at (0.5,3) {h};
\node (Rt2) at (1.5,1.0) {o};
%\node (H2) at (1.0,2) {\epen{L}};
%\node (R2) at (2.0,3) {\blue{l}};
\draw [thick] (syl.north) -- (Rt.south) ;
\draw [thick] (Rt.north) -- (H.south) ;
\draw [thick] (Rt.north) -- (R.south) ;
\draw [thick] (syl.north) -- (Rt2.south) ;
\end{tikzpicture}
\end{minipage}
}

\newcommand{\OTHnoLEpenth}{
\begin{minipage}{0.17\textwidth}
\begin{tikzpicture}[xscale=\myscalex,yscale=\myscaley]
\node (tone) at (2,0) {(= HM)};
\node (syl) at (0,0) {\textsigma};
\node (Rt) at (0,1) {o};
\node (H) at (-0.5,2) {H};
\node (R) at (0.5,3) {h};
\node (Rt2) at (1.5,1.0) {o};
\node (H2) at (1.0,2) {\epen{L}};
\node (R2) at (2.0,3) {\epen{h}};
\draw [thick] (syl.north) -- (Rt.south) ;
\draw [thick] (Rt.north) -- (H.south) ;
\draw [thick] (Rt.north) -- (R.south) ;
\draw [thick] (syl.north) -- (Rt2.south) ;
\draw [dashed] (Rt2.north) -- (H2.south) ;
\draw [dashed] (Rt2.north) -- (R2.south) ;
\end{tikzpicture}
\end{minipage}
}

\newcommand{\OTHnoLSpreading}{
\begin{minipage}{0.17\textwidth}
\begin{tikzpicture}[xscale=\myscalex,yscale=\myscaley]
\node (tone) at (2,0) {(= HH)};
\node (syl) at (0,0) {\textsigma};
\node (Rt) at (0,1) {o};
\node (H) at (-0.5,2) {H};
\node (R) at (0.5,3) {h};
\node (Rt2) at (1.5,1.0) {o};
%\node (H2) at (1.0,2) {\epen{L}};
%\node (R2) at (2.0,3) {\blue{l}};
\draw [thick] (syl.north) -- (Rt.south) ;
\draw [thick] (Rt.north) -- (H.south) ;
\draw [thick] (Rt.north) -- (R.south) ;
\draw [thick] (syl.north) -- (Rt2.south) ;
\draw [dashed] (Rt2.north) -- (H.south) ;
\draw [dashed] (Rt2.north) -- (R.south) ;
\end{tikzpicture}
\end{minipage}
}

% Sec. 4.2, third tabl., LM is unaffected by L\%

\newcommand{\OTLMInput}{
\begin{minipage}{0.2\textwidth}
\begin{tikzpicture}[xscale=\myscalex,yscale=\myscaley]
\node (tone) at (2,0) {(= LM)};
\node (syl) at (0,0) {\textsigma};
\node (Rt) at (0,1) {o};
\node (H) at (-0.5,2) {L};
\node (R) at (0.5,3) {l};
\node (Rt2) at (1.5,1.0) {o};
\node (H2) at (1.0,2) {L};
\node (R2) at (2.0,3) {h};
\node (R3) at (3.0,3) {\blue{l}};
\draw [thick] (syl.north) -- (Rt.south) ;
\draw [thick] (Rt.north) -- (H.south) ;
\draw [thick] (Rt.north) -- (R.south) ;
\draw [thick] (syl.north) -- (Rt2.south) ;
\draw [thick] (Rt2.north) -- (H2.south) ;
\draw [thick] (Rt2.north) -- (R2.south) ;
\end{tikzpicture}
\end{minipage}
}

\newcommand{\OTLMReplace}{
\begin{minipage}{0.2\textwidth}
\begin{tikzpicture}[xscale=\myscalex,yscale=\myscaley]
\node (tone) at (2,0) {(= LL)};
\node (syl) at (0,0) {\textsigma};
\node (Rt) at (0,1) {o};
\node (H) at (-0.5,2) {L};
\node (R) at (0.5,3) {l};
\node (Rt2) at (1.5,1.0) {o};
\node (H2) at (1.0,2) {L};
\node (R2) at (2.0,3) {h};
\node (R3) at (3.0,3) {\blue{l}};
\draw [thick] (syl.north) -- (Rt.south) ;
\draw [thick] (Rt.north) -- (H.south) ;
\draw [thick] (Rt.north) -- (R.south) ;
\draw [thick] (syl.north) -- (Rt2.south) ;
\draw [thick] (Rt2.north) -- (H2.south) ;
\draw [thick] (Rt2.north) -- (R2.south) ;
\draw [dashed] (Rt2.north) -- (R3.south) ;
\node (del) at (1.8,2.1) {\textbf{=}};
\end{tikzpicture}
\end{minipage}
}

\newcommand{\OTLMTwoReg}{
\begin{minipage}{0.2\textwidth}
\begin{tikzpicture}[xscale=\myscalex,yscale=\myscaley]
\node (tone) at (2,0) {(= LML)};
\node (syl) at (0,0) {\textsigma};
\node (Rt) at (0,1) {o};
\node (H) at (-0.5,2) {L};
\node (R) at (0.5,3) {l};
\node (Rt2) at (1.5,1.0) {o};
\node (H2) at (1.0,2) {L};
\node (R2) at (2.0,3) {h};
\node (R3) at (3.0,3) {\blue{l}};
\draw [thick] (syl.north) -- (Rt.south) ;
\draw [thick] (Rt.north) -- (H.south) ;
\draw [thick] (Rt.north) -- (R.south) ;
\draw [thick] (syl.north) -- (Rt2.south) ;
\draw [thick] (Rt2.north) -- (H2.south) ;
\draw [thick] (Rt2.north) -- (R2.south) ;
\draw [dashed] (Rt2.north) -- (R3.south) ;
\end{tikzpicture}
\end{minipage}
}

% Sec. 4.2, fourth tabl., L is affected by L\% but M is not

\newcommand{\OTLInput}{
\begin{minipage}{0.17\textwidth}
\begin{tikzpicture}[xscale=\myscalex,yscale=\myscaley]
\node (tone) at (2,0) {(= L)};
\node (syl) at (0,0) {\textsigma};
\node (Rt) at (0,1) {o};
\node (H) at (-0.5,2) {L};
\node (R) at (0.5,3) {l};
\node (R2) at (2,3) {\blue{l}};
\draw [thick] (syl.north) -- (Rt.south) ;
\draw [thick] (Rt.north) -- (H.south) ;
\draw [thick] (Rt.north) -- (R.south) ;
\end{tikzpicture}
\end{minipage}
}

\newcommand{\OTLLowered}{
\begin{minipage}{0.17\textwidth}
\begin{tikzpicture}[xscale=\myscalex,yscale=\myscaley]
\node (tone) at (2,0) {(= LL)};
\node (syl) at (0,0) {\textsigma};
\node (Rt) at (0,1) {o};
\node (H) at (-0.5,2) {L};
\node (R) at (0.5,3) {l};
\node (R2) at (2,3) {\blue{l}};
\draw [thick] (syl.north) -- (Rt.south) ;
\draw [thick] (Rt.north) -- (H.south) ;
\draw [thick] (Rt.north) -- (R.south) ;
\draw [dashed] (Rt.north) -- (R2.south) ;
\end{tikzpicture}
\end{minipage}
}

\newcommand{\OTMInput}{
\begin{minipage}{0.17\textwidth}
\begin{tikzpicture}[xscale=\myscalex,yscale=\myscaley]
\node (tone) at (2,0) {(= M)};
\node (syl) at (0,0) {\textsigma};
\node (Rt) at (0,1) {o};
\node (H) at (-0.5,2) {L};
\node (R) at (0.5,3) {h};
\node (R2) at (2,3) {\blue{l}};
\draw [thick] (syl.north) -- (Rt.south) ;
\draw [thick] (Rt.north) -- (H.south) ;
\draw [thick] (Rt.north) -- (R.south) ;
\end{tikzpicture}
\end{minipage}
}

\newcommand{\OTMLowered}{
\begin{minipage}{0.17\textwidth}
\begin{tikzpicture}[xscale=\myscalex,yscale=\myscaley]
\node (tone) at (2,0) {(= ML)};
\node (syl) at (0,0) {\textsigma};
\node (Rt) at (0,1) {o};
\node (H) at (-0.5,2) {L};
\node (R) at (0.5,3) {h};
\node (R2) at (2,3) {\blue{l}};
\draw [thick] (syl.north) -- (Rt.south) ;
\draw [thick] (Rt.north) -- (H.south) ;
\draw [thick] (Rt.north) -- (R.south) ;
\draw [dashed] (Rt.north) -- (R2.south) ;
\end{tikzpicture}
\end{minipage}
}

% Sec. 4.2, fifth tableau, polar questions with level tones

\newcommand{\OTLPolIn}{
\begin{minipage}{0.20\textwidth}
\begin{tikzpicture}[xscale=\myscalex-0.05,yscale=\myscaley-0.05]
\node (tone) at (3.5,0) {(= L)};
\node (syl) at (0,0) {\textsigma};
\node (syl2) at (2,0) {\red{\textsigma}};
\node (Rt) at (0,1) {o};
\node (H) at (-0.5,2) {L};
\node (R) at (0.5,3) {l};
\node (Rt2) at (2,1) {\red{o}};
\draw [thick] (syl.north) -- (Rt.south) ;
\draw [thick,red] (syl2.north) -- (Rt2.south) ;
\draw [thick] (Rt.north) -- (H.south) ;
\draw [thick] (Rt.north) -- (R.south) ;
\end{tikzpicture}
\end{minipage}
}

\newcommand{\OTLPolDef}{
\begin{minipage}{0.20\textwidth}
\begin{tikzpicture}[xscale=\myscalex-0.05,yscale=\myscaley-0.05]
\node (tone) at (3.5,0) {(= L.M)};
\node (syl) at (0,0) {\textsigma};
\node (syl2) at (2,0) {\red{\textsigma}};
\node (Rt) at (0,1) {o};
\node (H) at (-0.5,2) {L};
\node (R) at (0.5,3) {l};
\node (H2) at (1.5,2) {\epen{L}};
\node (R2) at (2.5,3) {\epen{h}};
\node (Rt2) at (2,1) {\red{o}};
\draw [thick] (syl.north) -- (Rt.south) ;
\draw [thick,red] (syl2.north) -- (Rt2.south) ;
\draw [thick] (Rt.north) -- (H.south) ;
\draw [thick] (Rt.north) -- (R.south) ;
\draw [semithick,dashed] (Rt2.north) -- (H2.south) ;
\draw [semithick,dashed] (Rt2.north) -- (R2.south) ;
\end{tikzpicture}
\end{minipage}
}

\newcommand{\OTLPolAlt}{
\begin{minipage}{0.20\textwidth}
\begin{tikzpicture}[xscale=\myscalex-0.05,yscale=\myscaley-0.05]
\node (tone) at (3.5,0) {(= L.L)};
\node (syl) at (0,0) {\textsigma};
\node (syl2) at (2,0) {\red{\textsigma}};
\node (Rt) at (0,1) {o};
\node (H) at (-0.5,2) {L};
\node (R) at (0.5,3) {l};
\node (Rt2) at (2,1) {\red{o}};
\draw [thick] (syl.north) -- (Rt.south) ;
\draw [thick,red] (syl2.north) -- (Rt2.south) ;
\draw [thick] (Rt.north) -- (H.south) ;
\draw [thick] (Rt.north) -- (R.south) ;
\draw [semithick,dashed] (Rt2.north) -- (H.south) ;
\draw [semithick,dashed] (Rt2.north) -- (R.south) ;
\end{tikzpicture}
\end{minipage}
}

% Sec. 4.2, sixth tableau, polar questions with contour tones

\newcommand{\OTLLPolIn}{
\begin{minipage}{0.23\textwidth}
\begin{tikzpicture}[xscale=\myscalex-0.05,yscale=\myscaley-0.05]
\node (tone) at (5.2,0) {(= L)};
\node (syl) at (0,0) {\textsigma};
\node (syl3) at (3.4,0) {\red{\textsigma}};
\node (Rt) at (0,1) {o};
\node (Rt2) at (1.7,1) {o};
\node (Rt3) at (3.4,1) {\red{o}};
\node (H) at (-0.5,2) {L};
\node (R) at (0.5,3) {l};
\draw [thick] (syl.north) -- (Rt.south) ;
\draw [thick] (syl.north) -- (Rt2.south) ;
\draw [thick,red] (syl3.north) -- (Rt3.south) ;
\draw [thick] (Rt.north) -- (H.south) ;
\draw [thick] (Rt.north) -- (R.south) ;
\end{tikzpicture}
\end{minipage}
}

\newcommand{\OTLLPolDef}{
\begin{minipage}{0.23\textwidth}
\begin{tikzpicture}[xscale=\myscalex-0.05,yscale=\myscaley-0.05]
\node (tone) at (5.2,0) {(= L.M)};
\node (syl) at (0,0) {\textsigma};
\node (syl3) at (3.4,0) {\red{\textsigma}};
\node (Rt) at (0,1) {o};
\node (Rt2) at (1.7,1) {o};
\node (Rt3) at (3.4,1) {\red{o}};
\node (H) at (-0.5,2) {L};
\node (R) at (0.5,3) {l};
\node (H3) at (2.9,2) {\epen{L}};
\node (R3) at (3.9,3) {\epen{h}};
\draw [thick] (syl.north) -- (Rt.south) ;
\draw [thick] (syl.north) -- (Rt2.south) ;
\draw [thick,red] (syl3.north) -- (Rt3.south) ;
\draw [thick] (Rt.north) -- (H.south) ;
\draw [thick] (Rt.north) -- (R.south) ;
\draw [dashed] (Rt3.north) -- (H3.south) ;
\draw [dashed] (Rt3.north) -- (R3.south) ;
\end{tikzpicture}
\end{minipage}
}

\newcommand{\OTLLPolSkip}{
\begin{minipage}{0.23\textwidth}
\begin{tikzpicture}[xscale=\myscalex-0.05,yscale=\myscaley-0.05]
\node (tone) at (5.2,0) {(= L.L)};
\node (syl) at (0,0) {\textsigma};
\node (syl3) at (3.4,0) {\red{\textsigma}};
\node (Rt) at (0,1) {o};
\node (Rt2) at (1.7,1) {o};
\node (Rt3) at (3.4,1) {\red{o}};
\node (H) at (-0.5,2) {L};
\node (R) at (0.5,3) {l};
\draw [thick] (syl.north) -- (Rt.south) ;
\draw [thick] (syl.north) -- (Rt2.south) ;
\draw [thick,red] (syl3.north) -- (Rt3.south) ;
\draw [thick] (Rt.north) -- (H.south) ;
\draw [thick] (Rt.north) -- (R.south) ;
\draw [dashed] (Rt3.north) -- (H.south) ;
\draw [dashed] (Rt3.north) -- (R.south) ;
\end{tikzpicture}
\end{minipage}
}  
  
\newcommand{\ilit}[1]{#1\il{#1}}    
\newcommand{\isit}[1]{#1\is{#1}}  

\makeatletter
\let\thetitle\@title
\let\theauthor\@author 
\makeatother

\newcommand{\togglepaper}[1][0]{ 
  \bibliography{../localbibliography}
  %% hyphenation points for line breaks
%% Normally, automatic hyphenation in LaTeX is very good
%% If a word is mis-hyphenated, add it to this file
%%
%% add information to TeX file before \begin{document} with:
%% %% hyphenation points for line breaks
%% Normally, automatic hyphenation in LaTeX is very good
%% If a word is mis-hyphenated, add it to this file
%%
%% add information to TeX file before \begin{document} with:
%% \include{localhyphenation}
\hyphenation{
affri-ca-te
affri-ca-tes
com-ple-ments
par-a-digm
Sha-ron
Kings-ton
phe-nom-e-non
Daul-ton
Abu-ba-ka-ri
Ngo-nya-ni
Clem-ents 
King-ston
Tru-cken-brodt
Tab-leau
cophono-logies
mark-edness
Ti-gri-nya
a-mong
Car-stens
Lu-bu-ku-su
}
\hyphenation{
affri-ca-te
affri-ca-tes
com-ple-ments
par-a-digm
Sha-ron
Kings-ton
phe-nom-e-non
Daul-ton
Abu-ba-ka-ri
Ngo-nya-ni
Clem-ents 
King-ston
Tru-cken-brodt
Tab-leau
cophono-logies
mark-edness
Ti-gri-nya
a-mong
Car-stens
Lu-bu-ku-su
}
  \papernote{\scriptsize\normalfont
    \theauthor.
    \thetitle. 
    To appear in: 
    Emily Clem,   Peter Jenks \& Hannah Sande.
    Theory and description in African Linguistics: Selected papers from the 47th Annual Conference on African Linguistics.
    Berlin: Language Science Press. [preliminary page numbering]
  }
  \pagenumbering{roman}
  \setcounter{chapter}{#1}
  \addtocounter{chapter}{-1}
}

\newcommand{\upstep}{\textupstep}


% \newcounter{tableauxcounter}

\renewcommand{\textltailn}{ɲ}
\renewcommand{\textbardotlessj}{ɟ}

\newcommand{\emphkh}[1]{\textit{#1}} %originally \textbf, banned by the guidelines



\definecolor{lsDOIGray}{cmyk}{0,0,0,0.45}


\newcommand{\xuparrow}[1]{%
  {\left\uparrow\vbox to #1{}\right.\kern-\nulldelimiterspace}
}
\renewcommand \textupstep[1]{\char"A71B#1}
\renewcommand \textdownstep[1]{\char"A71C#1}
 
 \newcommand{\ꜛ}{\textsf{ꜛ}}
 
\def\biberror{\undefined}


\newcommand{\OTbox}[1]{\resizebox{.88\textwidth}{!}{#1}}

  \bibliography{localbibliography} 
  \papernote{\scriptsize\normalfont
  Christopher R. Green \&  Evan Jones.
  Notes on the morphology of Marka (Af-Ashraaf). 
  To appear in: 
  Emily Clem,   Peter Jenks \& Hannah Sande.
  Theory and description in African Linguistics: Selected papers from the 47th Annual Conference on African Linguistics.
  Berlin: Language Science Press. [preliminary page numbering]
  }
  \pagenumbering{roman}
}{}


\author{Christopher R. Green\affiliation{Syracuse University} 
\lastand Evan Jones\affiliation{University of Maryland}  
}
\title{Notes on the morphology of Marka (Af-Ashraaf)}  
\abstract{This paper provides an overview of selected aspects of the nominal, pronominal, and verbal morphology of the Marka (Merca) dialect of Af-Ashraaf, a Cushitic language variety spoken primarily in the city of Merca in southern Somalia, as well as by several diaspora communities around the world, and in particular, in the United States. Marka is interesting to us for a variety of reasons, not the least of which is the general dearth of descriptive work on the language in comparison to two of its closest relatives, Somali and Maay. While many details of the structure of Somali are fairly well established (e.g., \citealt{Bell1953,Saeed1999}), and those of Maay are the subject of several recent works (e.g., \citealt{Paster2010,PasterMaayGender}), the various ways in which Marka relates to and/or differs from these languages, are yet poorly understood. Our goal in this paper is to begin to remedy this situation, beginning with a comparison of selected morphological characteristics across the three languages.}


\begin{document}
\maketitle
\newcommand{\emphgj}[1]{\textit{#1}} %the original had \textbf for emphasis
\section{Introduction} 
 

 This paper describes aspects of the morphology of \ili{Marka}, a variety of \ili{Af-Ashraaf} spoken in and around the city of \ili{Merca} in Southern Somalia, as well as by diaspora communities in the United States and elsewhere. The data that we present are from our own fieldwork with our main consultant, a mother tongue speaker of \ili{Marka}, conducted in three locations across the United States over a span of several years. The data were collected by the first author in Minneapolis, Minnesota, in October 2014 and in Phoenix, Arizona, in October 2015. Data were also collected by the second author in Minneapolis in 2009 and 2010. These cities, among a few others in the United States, are home to sizable diaspora populations of \ili{Marka} speakers. 
 
 \ili{Marka} is one of two varieties of \ili{Af-Ashraaf}, the other being \ili{Shingani}, which is spoken primarily in and around the \ili{Somali} capital, Mogadishu; \ili{Shingani} is also sometimes called \ili{Xamar}, which is the name locals attribute to Mogadishu itself. To our knowledge, there is one published theoretical article on \ili{Shingani} which pertains to so-called ``{theme constructions}'' (\citealt{Ajello1984}). There is also a self-published book of pedagogical materials for the dialect (\citealt{Abo2007}) and a short grammatical sketch (\citealt{Moreno1953}). There is less available for \ili{Marka}; this includes an unpublished grammatical sketch [in \ili{German}] (\citealt{Lamberti1980}), and one article on aspects of its \isi{verbal inflection} (\citealt{Ajello1988}). In addition, both Ashraaf varieties are briefly mentioned in several classificatory works (as cited below) and in \citet{Banti2011}. Compared even to other African languages, the varieties of \ili{Af-Ashraaf} are under-described and certainly under-documented.
 
 In this paper, we present data highlighting certain morphological characteristics of \ili{Marka}. Our immediate goal in this paper is to begin to establish (and in some instances reaffirm) characteristics of contemporary \ili{Marka}. In order to better situate this language variety alongside two of its closest and better-described cousins, namely \ili{Somali} and \ili{Maay}, we provide comparable examples from these languages wherever possible. We believe that this is an important component of our ongoing work on \ili{Marka}. While we have not yet explored it empirically, and despite all classifications of Ashraaf treating it as a dialect of \ili{Somali}, our \ili{Marka} speakers have intimated to us that both \ili{Marka}/\ili{Somali} and \ili{Marka}/\ili{Maay} intelligibility presents a challenge, though they deem \ili{Somali} to be somewhat more intelligible to them than \ili{Maay}. Our hope that by directly comparing these three languages throughout our ongoing research wherever possible, it will permit further discussion concerning the classificatory and structural relationships between them. 
 
 As we mention above, the \ili{Marka} data that we present are our own. Comparative lexical and morphological data for \ili{Somali} are drawn primarily from \citet{Greenetal2015}, and the data therein are in line with other published sources on the language (e.g., \citealt{Bell1953,Saeed1999}). These data are from Northern \ili{Somali}; hereafter, any reference to \ili{Somali} refers to Northern \ili{Somali} unless otherwise indicated. Corresponding \ili{Maay} data are drawn from a recent grammatical sketch of the Lower Jubba variety of the language \citet{PasterRanero2015}, which itself is in line with other published materials on the language (e.g., \citealt{Paster2007,Paster2010,PasterMaayGender}). The comparative data that we present allow us to begin to draw some generalizations, though preliminary, about morphological similarities and differences between \ili{Marka}, \ili{Somali}, and \ili{Maay}. We highlight two unique characteristics of \ili{Marka} that stand out in comparison to \ili{Somali} and \ili{Maay}; these include the morphological encoding of pluralization and grammatical \isi{gender}. 
 
 The \ili{Marka} data presented below are transcribed using the International Phonetic Alphabet (IPA). \ili{Somali} data are given in the standard \ili{Somali} orthography (\citealt{Andrzejewski1978}); in this orthography, certain written symbols differ markedly from their IPA counterparts. These and their phonetic equivalents are as follows: c [ʕ], dh [ɖ], kh [χ], x [ħ], j [tʃ], and sh [ʃ]. Although \ili{Maay} does not have an official or standard orthography, we follow the conventions used in \citet{PasterRanero2015} in presenting
\ili{Maay} data below. Like in the case of \ili{Somali}, some \ili{Maay} written symbols differ from their IPA counterparts. For \ili{Maay}, these letters and their phonetic equivalents are as follows: j [tʃ], sh [ʃ], ny [ɲ], d' [ɗ], y' [ʄ], and g' [ɠ]. Data for all three languages include morpheme breaks which are indicated by a hyphen; finer-grained distinctions such as clitic boundaries are not indicated. 
 
 Arriving at a better understanding of \ili{Marka}'s place alongside \ili{Somali} and \ili{Maay} has broader implications, as its place (and of \ili{Af-Ashraaf}, more broadly) in classifications of Lowland East \ili{Cushitic} languages is not entirely clear. As we mention above, despite the fact that some classifications treat Ashraaf as a dialect of \ili{Somali}, \ili{Marka} and \ili{Somali} appear not to have a high degree of mutual intelligibility, begging the question as to whether the former is properly classified as a dialect of the latter. Although it is not our intent to engage in a lengthy discussion of classification, we believe that it is nonetheless important to ground our paper in a short description of the state of the science concerning the internal classification of languages believed to be most closely related to \ili{Marka}. 
 
 Generally speaking, there are several competing classifications concerning the composition of the so-called `\ili{Somali}' branch of the Lowland East \ili{Cushitic} languages in the larger \ili{Afro-Asiatic} language family (e.g., \citealt{Abdhullahi2000,EhretAli1984,Heine1978,Lamberti1984,Moreno1955}). \citet{Lamberti1984} and \citet{EhretAli1984} are of importance to our interests, as they specifically refer to Ashraaf varieties in their classifications. Note that `\ili{Somali}' is the name of both the sub-group as a whole and of a language within the sub-group designated ISO:som in \citet{Ethnologue19}. \citet{Lamberti1984} defines five dialect groups of `\ili{Somali}' wherein Ashraaf is considered a separate dialect group from both the better-described Northern and \ili{Benaadir} \ili{Somali} dialects. He further divides Ashraaf into \ili{Shingani} and Lower Shabelle varieties, of which the latter is the \ili{Marka} variety discussed elsewhere. Examples provided compare only the ``peculiarities'' (to use Lamberti's term) of the \ili{Shingani} variety to Af-Maxaad Tidhi (i.e., a group composed of Northern and \ili{Benaadir} \ili{Somali}), but no diffentiation is provided pertaining to the \ili{Marka} variety of Ashraaf, which is the \isi{focus} of the current paper. \citet{EhretAli1984}, on the other hand, group \ili{Xamar} and \ili{Marka} (i.e., Ashraaf) varieties with \ili{Benaadir} \ili{Somali} and little detail about their properties relative to one another or to other varieties/dialects is given. We certainly do not mean to imply that we are the first to look at \ili{Af-Ashraaf}, nor is it our intent to engage in a classification debate in this paper, but we believe that it there is much more to learn about the properties of this language group (i.e., \ili{Af-Ashraaf}'s two constituent varieties, \ili{Shingani} and \ili{Marka}) and its relationship to its closest relatives. In order to begin to do so, we turn our attention first in this paper to properties of \ili{Marka} morphology.
 
 %\section{Phonology and morphophonemics}
 
 %The \ili{Marka} (M) sound inventory is nearly identical to that of \ili{Benaadir} (also called Coastal) \ili{Somali} (BS); BS in turn differs only slightly from Northern \ili{Somali} (NS). The differences between these varieties are largely predictable and exist in both written and spoken forms of the language. For example, the NS uvular stop [q], also written <q>, corresponds to the BS/M uvular fricative [χ], written <kh>. Compare, for example, the word for `rifle' in NS \textbf{bandu[q]} to BS/M \textbf{bandu[χ]}. Both languages, however, have written <x>, pronounced variably as [ħ] or [ʜ], depending on the speake. Similarly, the retroflex, implosive alveolar stop in NS, written <dh> is retroflex [ɖ] in M and the flap [ɾ] in BS in careful speech; in M and BS, these are both written <r>. The [ɖ] in M is often flapped intervocalically, being pronounced instead as [ɽ]. Like NS/BS, M has an inventory of five vowels that exhibit a length contrast. We have not yet had the opportunity to explore systematically whether M exhibits any semblance of \isi{vowel harmony} similar to that reported for NS/BS (e.g., \citealt{Andrzejewski1955,Armstrong1934}).
 
 %Morphophonological alternations found in M are similar, but not identical to those seen in NS/BS. For example, M has word- and syllable-final devoicing, which is analogous to the deaspiration found in the same environments in NS/BS \citet{Greenetal2015}. This generalization holds except in those instances where a syllable-final voiced stop is followed by another stop at the same place of articulation. In these instances, voicing prevails. Also like NS/BS, M stops spirantize intervocalically. Examples of these alternations are in \tabref{tab:1:FinalAlternations}. Note that there is slight difference between M and NS/BS concerning intervocalic spirantization in that the phenomenon does not appear to be conditioned by stress location in M. In NS/BS, however, a stop is spirantized intervocalically only after a stressed syllables, while after an unstressed syllable, a stop will simply be (or become) voiced. Furthermore, intervocalic flapping and spirantization of /r/ are unique to M; for some speakers /r/ will flap to [ɾ], as in \textbf{saana[ɾ]e} `the year.' For others, it will spirantize to [ð]; that is, \textbf{saana[ð]e}.
  
%\ea\label{ex:1:descartes}
%\langinfo{Latin}{}{personal knowledge}\\
%\gll cogit-o ergo sum \\
%     think-1{\sg}.{\prs}.{\ind} hence exist.1{\sg}.{\prs}.{\ind}\\
%\glt `I think therefore I am'
%\z

%\begin{table}
%\caption{Stop alternations in final positions}
%\label{tab:1:FinalAlternations}
 %\begin{tabularx}{\textwidth}{Xlllll} 
  %\lsptoprule
   %       /naag/  &  & /irbad/ &  & /kab/ & \\ 
%  \midrule
 % naa[k]  &   woman &    irba[t]  &    needle     & ka[p] & shoe\\
  %naa[k]te  &   the woman &   irba[d]de &    the needle    & ka[p]te & the shoe\\
  %naa[ɣ]aynyo & women & irba[ð]aynyo & needles	& ka[β]aynyo & shoes \\
  %\lspbottomrule
% \end{tabular}
%\end{table}

 %There are other final position alternations that affect nasals, and /m/ in particular. Like in NS/BS, when a stem in M ends in /m/, it is realized as [m] only intervocalically and before another labial consonant. Word-finally, it is realized [ŋ] and will regressively assimilate to the place of articulation of a following consonant in other instances. These possibilities are shown in \tabref{tab:1:NasalAlternations}.
 
 %\begin{table}
 %	\caption{Nasal alternations in final positions}
 %	\label{tab:1:NasalAlternations}
 %	\begin{tabularx}{\textwidth}{Xl} 
 %		\lsptoprule
 %		/irim/  &   \\ 
 %		\midrule
 %		iri[ŋ]  &   a sheep \\
 %		iri[n]te  &   the (female) sheep \\
 %		iri[ŋ]ke & the (male) sheep  \\
 %		iri[m]aynyo & sheep (plural) \\
 %		\lspbottomrule
 %	\end{tabular}
 %\end{table}

 %Stem-final /e/ and /o/ in M also alternate when followed by some suffix. Both vowels become [a], as in \textbf{sanno} `year' vs. \textbf{sannare} `the year' and thereafter \textbf{sannarayte} `the years.' Analogous alternations are found in NS/BS, but their exact outcomes vary somewhat by variety.

% Turning to alternations presumably related to \isi{prosodic structure}, we find that syllable reduction in what are arguably weak metrical positions occurs in M, as it does in NS/BS. When three syllables come in sequence, the first of which is prominent, and the second of which is short, the vowel of the second syllable is deleted except when an unfavorable phonotactic sequence would result. We can see this occurring in the singular vs. plural forms of `day,' which are \textbf{maaliŋ} and \textbf{maalmaynyo}, respectively. Finally, and although there is much more to be said on the topic, \ili{Marka} appears to maintain a weak high (H) tonal accent that can be found on nouns, verbs, and some affixes. Like \citet{GreenMorrison2016} have discussed for NS/BS, the location of this tonal accent in M appears to shift as the result of some morphological operations. For example, the noun \textbf{s\'{a}nno} `year' has a H tonal accent (indicated by an acute accent) on its first syllable. The location of the word's tonal accent is not affected upon pluralization, \textbf{s\'{a}nnare} `the year,' yet it shifts as the resulting of pluralization, as in \textbf{sannar\'{a}ynyo} `years.' This is in line with Green \& Morrison's analysis of \ili{Somali} wherein definite determiners are prosodically-inert clitics, while pluralization is accomplished by prosodic affixes that can host a tonal accent.
 
 \section{Nominal morphology}
 
 Singular nouns in \ili{Marka} are unmarked, and their plural counterparts are all formed by the addition of the suffix \emphgj{-(r)ajɲo} wherein an epenthetic rhotic appears after vowel-final stems. We illustrate in \tabref{tab:1:Pluralization} that \ili{Marka} adopts a single strategy to pluralize nearly every noun. The exception to this is a few high frequency nouns that are used in proverbs whose plurals are identical to those found in \ili{Somali} (e.g., \emphgj{ilig} `tooth' vs. \emphgj{ilko} `teeth'). Corresponding \ili{Somali} plurals are provided for comparison, wherever possible. The fact that outside of these few outliers, \ili{Marka} adopts a single pluralization strategy distinguishes it from both \ili{Somali} and \ili{Maay}. This is because \ili{Somali} adopts at least five different pluralization strategies (e.g., suffixation of \emphgj{-o} or \emphgj{-yaal}, partial suffixing reduplication, tonal accent shift, and both broken and sound pluralization in some \ili{Arabic} borrowings), while \ili{Maay} adopts two or three, depending on the particular noun (\citealt{Paster2010}), all of which involve suffixation.
% Should this be \label{tab:Green:1}?
 \begin{table}
 	\caption{{Pluralization}}
 	\label{tab:1:Pluralization} 
 	\begin{tabularx}{\textwidth}{Xllll} 
 		\lsptoprule
 		\ilit{Marka} Singular  &  & \ilit{Marka} Plural & & \ilit{Somali} plural\\ 
 		\midrule
 		dabaal  & fool & dabaal-ajɲo & fools & dabbaal-o \\
 		af  &   language & af-ajɲo & languages & af-af \\
 		karfin & tomb & karfim-ajɲo & tombs & \\
 		khoor & necks & khoor-ajɲo & necks & qoor-ar \\
 		\tablevspace
 		mindi & knife & mindi-rajɲo & knives & mindi-yo \\
 		maro & head & mara-rajɲo & heads & mad\'{a}x \\
 		guddoomije & chairman & guddoomija-rajɲo & chairmen & guddoomiya-yaal \\
 		\lspbottomrule
 	\end{tabularx} 
 \end{table}
 
 Like \ili{Somali} and \ili{Maay}, \ili{Marka} encodes two grammatical genders in its nominal system: \isi{masculine} and \isi{feminine}. Nouns have inherent \isi{gender}, however, there is no overt segmental indication of \isi{gender} on nouns themselves. Rather, a given noun's grammatical \isi{gender} is recoverable from the patterns of agreement that it requires on its modifiers. This can be seen, for example, in definite determiners, wherein the initial consonant of the determiner (except in one context discussed below) reveals the noun's \isi{gender}. These consonants, however, often alternate following particular stem-final segments. The \isi{masculine} \isi{definite determiner} is \emphgj{-e} after liquids and pharyngeals and \emphgj{-ke} in most other contexts. The \isi{feminine} \isi{definite determiner} is \emphgj{-de} after [d] and pharyngeals and \emphgj{-te} in most other instances. Following vowel-final stems, the \isi{definite determiner} is always \emphgj{-re}, even in association with those nouns that are biologically \isi{masculine} or \isi{feminine}. This points towards a neutralization of the morphological encoding of \isi{gender} in such contexts. Thus, both \isi{masculine} and \isi{feminine} nouns whose stem ends in a vowel take the \isi{definite determiner} \emphgj{-re}. In addition, and as one might expect, certain nouns are free to change their \isi{gender} in accord with the biological \isi{gender} of their referent, as in \emphgj{saaxibke} `the (male) friend' vs. \emphgj{saaxibte} `the (female) friend.' Examples of \ili{Marka} \isi{masculine} and \isi{feminine} singular nouns in their indefinite and definite forms are in \tabref{tab:1:Grammatical Gender}.
 


\begin{table}
 	\caption{{Grammatical gender and definite determiners (Marka)}}
 	\label{tab:1:Grammatical Gender}
 	\begin{tabularx}{\textwidth}{Xllll} 
 		\lsptoprule
 		 & Indefinite  &  & Definite &  \\ 
 		\midrule
 		Masculine: & nin & `man' & niŋ-ke & `the man' \\
 		 & saŋ  &   `nose' & saŋ-ke & `the nose' \\
 		 & abti & `maternal uncle' & abti-re & `the maternal uncle'  \\
 		 & dabaal & `fool' & dabaal-e & `the fool' \\
 		  & gasaʕ & `can' & geseʕ-e & `the can'\\
 		\tablevspace
 		Feminine: & maaliŋ' & `day & maalin-te & `the day' \\
 		 & kab & `shoe' & kab-te & `the shoe' \\
 		 & irbad & `needle' & irbad-de & `the needle' \\
 		  & saddeχ & `three' & saddeχ-de & `the three' \\
	 	& iŋgo & `mother' & iŋga-re & `the mother' \\
 		\lspbottomrule
 	\end{tabularx}
 \end{table}
 
Although there is no overt \isi{gender} marking on \ili{Marka} nouns, it appears at least preliminarily that the accentual \isi{gender} distinction found in \ili{Somali} is maintained in \ili{Marka}. As discussed in detail in \citet{Hyman1981somali} and \citet{GreenMorrison2016}, \ili{Somali} nouns exhibit a tonal accent on either their final or penultimate mora; the mora is the \isi{tone} and accent bearing unit in the language. It is typically the case that non-derived \isi{masculine} singular nouns have a tonal accent on their penultimate mora while non-derived \isi{feminine} singular nouns have a tonal accent on their final mora. Like \ili{Somali}, \ili{Marka} appears to exhibit this same phenomena, as seen for example in a comparison of \isi{masculine} \emphgj{k\'{a}rfin-ke} `the tomb' and \isi{feminine} \emphgj{mind\'{\i}-re} `the knife.' This accentual distinction is helpful in determining the grammatical \isi{gender} of nouns with vowel-final stems. Compare, for example, the \isi{masculine} noun \emphgj{s\'{a}nno} `year' to the \isi{feminine} noun \emphgj{mind\'{\i}} `knife,' both of which take the same \isi{definite determiner} \emphgj{-re}. Their corresponding definite forms are \emphgj{s\'{a}nna-re} `the year' and \emphgj{mind\'{\i}-re} `the knife.'
 
 \newpage 
While \ili{Marka} maintains a fairly clear distinction between \isi{masculine} and \isi{feminine} grammatical \isi{gender} in singular nouns, whether segmental, accentual, or both, this distinction is lost upon pluralization. That is, all plural nouns require \isi{feminine} \isi{gender} agreement. This characteristic distinguishes \ili{Marka} from both \ili{Somali} and \ili{Maay}. \ili{Somali} has a complex grammatical \isi{gender} system; following the noun classification adopted in \citet{Greenetal2015}, nouns in Classes 1c and 2 maintain the same \isi{gender} in both the singular and plural, while nouns in Classes 1a, 1b, 3, 4, and 5 exhibit so-called \textit{\isi{gender} polarity} (\citealt{Meinhof1912}) where a noun's \isi{gender} changes from \isi{masculine} to \isi{feminine} (or vice versa) upon pluralization. \ili{Maay}, on the other hand, also collapses its grammatical \isi{gender} distinction in nouns upon pluralization, but unlike \ili{Marka} which levels \isi{gender} to \isi{feminine}, all \ili{Maay} plural nouns are \isi{masculine}. A summarized comparison of these three systems is in \tabref{tab:1:Grammatical gender comparison}. 
 
  \begin{table}
  	\caption{{Grammatical gender -- singular vs. plural}}
  	\label{tab:1:Grammatical gender comparison} 
  	\begin{tabularx}{\textwidth}{Xlll} 
  		\lsptoprule
  		\ilit{Marka} & \ilit{Somali} & \ilit{Maay}  &   Gloss  \\ 
  		\midrule
  		igaar & inan & dinaŋ &  `boy' \\
  		igaar\textbf{e} (m) & inan\textbf{ka} (m) & dinaŋ\textbf{ki} (m) &  `the boy'  \\
  		igaarajɲo & inammo & dinamo/dinanyyal/dinamoyal &  `boys' \\
  		igaaraj\textbf{te} (f) & inamma\textbf{da} (f) & dinamo\textbf{ɣi}/dinanyyal\textbf{ki}/ &  `the boys'\\
  		&& dinanmoyal\textbf{ki} (m) & \\
  		\tablevspace
  		naag &  naag & bilaŋ & `woman' \\
  		naag\textbf{te} (f) & naag\textbf{ta} (f) & bilan\textbf{ti} (f) &  `the woman'  \\
  		naagajɲo & naago & bilamo/bilanyyal/bilamoyal &  `women' \\
  		naagaj\textbf{te} (f) & naaga\textbf{ha} (m) & bilamo\textbf{ɣi}/bilanyyal\textbf{ki}/ &  `the women' \\
  		&& bilamoyal\textbf{ki} (m) & \\
  		\lspbottomrule
  	\end{tabularx} 
  \end{table}
 
 In addition to the definite determiners described above, \ili{Marka} has four additional determiner which can modify nouns. The initial consonant of each determiner alternates under the same conditions described above for definite determiners. There are two demonstrative determiners: \emphgj{koŋ/toŋ} `this' and \emphgj{kaas/taas} `that.' These have direct correspondents in both \ili{Somali} and \ili{Maay}, although \ili{Somali} has an additional distal demonstrative to point out `that yonder.' The \ili{Marka} interrogative determiner is \emphgj{kee/tee} `which?,' which, once again, has direct correspondents in both \ili{Somali} and \ili{Maay}. Like \ili{Somali}, \ili{Marka} exhibits so-called \textit{remote} or \textit{anaphoric} definite determiners, namely \emphgj{kii/tii}. In \ili{Somali}, these are described as being associated with \isi{past tense} referents (\citealt{Lecarme2008,Tosco1994}). They appear to instead have a disambiguating function in \ili{Marka}, which we gloss as `the/that (one) X.' In addition, \ili{Marka} has a determiner, \emphgj{koo/too}, that speakers use to point out an item that the speaker knows about but the hearer does not. There is a great deal of similarity in the determiners discussed thus far when comparing \ili{Marka} to both \ili{Somali} and \ili{Maay}; however, the possessive determiners in each are more divergent. Possessive determiners in the three varieties are shown in \tabref{tab:1:Possessive determiners}; they are presented in \isi{masculine}/\isi{feminine} pairs in their default forms. Note that \ili{Marka} and \ili{Maay} lack the exclusive vs. inclusive distinction encoded in \ili{Somali} for \isi{first person} plural. Also, \isi{third person} \isi{masculine} possessive determiners in both the singular and plural in \ili{Maay} differ greatly from those found in both \ili{Somali} and \ili{Marka}.
 
 \begin{table}
 	\caption{{Possessive determiners}}
 	\label{tab:1:Possessive determiners}
 	\begin{tabularx}{\textwidth}{XXXX} 
 		\lsptoprule
 		& \ilit{Marka}  & \ilit{Somali} & \ilit{Maay}   \\ 
 		\midrule
 		1\textsc{sg} & kee/tee & kay/tay & key/tey \\
 		2\textsc{sg} & kaa/taa  &   kaa/taa & ka/ta  \\
 		3\textsc{sg.m} & kiis/tiis & kiis/tiis & y'e/tis \\
 		3\textsc{sg.f} & kiiʃe/tiiʃe & keed/teed & y'e/tie \\
 		1\textsc{pl} & kaŋ/taŋ & kayo/tayo (exc.) & kaynu/taynu \\
 		 & & keen/teen (inc.) & \\
 		2\textsc{pl} & kiiŋ/tiiŋ & kiin/tiin & kiŋ/tiŋ  \\
 		3\textsc{pl} & kiiʃoŋ/tiiʃoŋ & kood/tood & y'o/tio \\
 		\lspbottomrule
 	\end{tabularx}
 \end{table}
 
 Concerning the derivational morphology that can be added to nouns, there are several parallels between \ili{Marka} and \ili{Somali}; the following list should not be taken as exhaustive. Thus far, we find that there are two \ili{Marka} suffixes, \mbox{\emphgj{-nimo}} and \emphgj{-ija}, that derive abstract nouns. Examples include: \emphgj{ħurnimo} `freedom' (cf. \emphgj{ħur} `free') and \emphgj{insaanija} `humanity' (cf. \emphgj{insaan} `human'). These correspond to \mbox{\emphgj{-nimo}} and \emphgj{-iyad} in \ili{Somali}. The \ili{Somali} suffix \emphgj{-tooyo}, which derives stative abstract nouns is absent in \ili{Marka}, and we have not yet been able to find another morpheme that accomplishes this function. The \ili{Marka} suffix \emphgj{-dari} derives antonyms, as in \emphgj{naħariisdari} `merciless' (cf. \emphgj{naħaris} `mercy'); this corresponds to \emphgj{-darro} in \ili{Somali}, which accomplishes the same function. The \ili{Marka} suffix \emphgj{-lo} corresponds to \ili{Somali} \emphgj{-le} and is used to derive agentive nouns, as in \emphgj{dukaanlo} `store owner' (cf. \emphgj{dukaan} `store'). Finally, we have found that inchoative and experiencer verbs can be derived from nouns in \ili{Marka} via the suffixes \emphgj{-wow} and \emphgj{-ʃow}, respectively, as in \emphgj{duqowow} `to become old' (cf. \emphgj{duq} `elder') and \emphgj{rijoʃow} `to have a dream' (cf. \emphgj{rijo} `dream'). 
 
 \section{Pronouns}
 
 \ili{Marka} has a single series of \isi{subject} pronouns which are inflected for person, number, and for biological \isi{gender} with human referents; \ili{Marka} does not encode an exclusive vs. inclusive distinction in its \isi{first person} plural \isi{subject} pronouns. \ili{Marka} \isi{subject} pronouns may be used independently whereupon they take on characteristics similar to other nouns. In addition, they may also cliticize to complementizers and negative markers under some conditions. A comparison between \isi{subject} pronouns in \ili{Marka}, \ili{Somali}, and \ili{Maay} is in \tabref{tab:1:Subject pronouns}. In addition to these \isi{subject} pronouns, \ili{Marka} (like \ili{Somali}) has a non-specific \isi{subject pronoun}, \emphgj{la}.
 
 \begin{table}
 	\caption{{Subject pronouns}}
 	\label{tab:1:Subject pronouns}
 	\begin{tabularx}{\textwidth}{XXXX} 
 		\lsptoprule
 		& \ilit{Marka}  & \ilit{Somali} & \ilit{Maay}   \\ 
 		\midrule
 		1\textsc{sg} & aan & aan & ani \\
 		2\textsc{sg} & at  &   aad & aði  \\
 		3\textsc{sg.m} & uus & uu & usu \\
 		3\textsc{sg.f} & ishe & ay & ii \\
 		1\textsc{pl} & annuŋ & aannu (exc.) & unu \\
 		& & aynu (inc.) & \\
 		2\textsc{pl} & asiin & aydin & isiŋ  \\
 		3\textsc{pl} & ishoon & ay & iyo \\
 		\lspbottomrule
 	\end{tabularx}
 \end{table}
 
 \tabref{tab:1:Subject pronouns} reveals that there are many similarities across the three language varieties under consideration regarding their \isi{subject} pronouns. A comparison of their object pronouns in \tabref{tab:1:Object pronouns}, however, shows far fewer similarities in this particular category. To begin, \ili{Somali} has so-called \textit{first series} (OP1) and \textit{second series} (OP2) object pronouns, the latter of which appear only in those instances where two non-{third person} pronominal objects are required. \ili{Somali} maintains an exclusive vs. inclusive distinction in both series of its object pronouns; neither \ili{Marka} nor \ili{Maay} encode such a distinction, and both have only a single series of object pronouns. Both series of \ili{Somali} object pronouns have \isi{third person} gaps in both the singular and plural. \ili{Marka} and \ili{Maay} differ in that each has \isi{third person} object pronouns. While \ili{Marka}'s \isi{third person} object pronouns appear innovative in all instances, the situation with \ili{Maay} is somewhat different. A comparison of \ili{Maay} \isi{subject} vs. object pronouns in Tables \ref{tab:1:Subject pronouns} and \ref{tab:1:Object pronouns} shows that they are in many instances identical. The exception of the first and \isi{second person} singular, and the \isi{second person} plural to some degree. In addition to its other object pronouns, \ili{Marka} has the reflexive/reciprocal \isi{pronoun} \emphgj{is}, similar to that found in \ili{Somali}.
 
 \begin{table}
 	\caption{{Object pronouns}}
 	\label{tab:1:Object pronouns}
 	\begin{tabularx}{\textwidth}{Xllll} 
 		\lsptoprule
 		& \ilit{Marka}  & \ilit{Somali} (OP1) & \ilit{Somali} (OP2) & \ilit{Maay}   \\ 
 		\midrule
 		1\textsc{sg} & iŋ & i & kay & i \\
 		2\textsc{sg} & ku  & ku & kaa  & ki  \\
 		3\textsc{sg.m} & su & - & - & usu \\
 		3\textsc{sg.f} & sa & - & - & ii \\
 		1\textsc{pl} & nuŋ & na (exc.) & kayo (exc.) & unu \\
 		& & ina (inc.) & keen (inc.) &  \\
 		2\textsc{pl} & siin & idin & kiin & isiŋ-siŋ  \\
 		3\textsc{pl} & soo & - & - & iyo \\
 		\lspbottomrule
 	\end{tabularx}
 \end{table}
 
 \ili{Marka} object pronouns cliticize onto adpositional particles, of which there are three. Object pronouns also co-occur with a non-specific \isi{subject pronoun} (NSP) meaning `one.' We notice no prosodic difference between them, but according to our speaker's intuition, sequences of NSP$+$object \isi{pronoun} are divisible, while object \isi{pronoun}$+$adposition are a single unit. Examples are in \tabref{tab:1:Pronouns Adpositions}.\\
  
\begin{table}
 	\caption{{Pronouns with adpositional particles (Marka)}}
 	\label{tab:1:Pronouns Adpositions}
 	\begin{tabularx}{\textwidth}{Xlllll} 
 		\lsptoprule
 		& Object \isi{pronoun} & NSP  & ka `in/from' & u `to/for' & la `with' \\ 
 		\midrule
 		1\textsc{sg} & iŋ & la iŋ & iŋka & iiŋ & inla \\
 		2\textsc{sg} & ku  & la ku & kuka (koo)  & kuuŋ & kula  \\
 		3\textsc{sg.m} & su & la su & suka & suuŋ & sula \\
 		3\textsc{sg.f} & sa & la sa & saka & saaŋ & sala \\
 		1\textsc{pl} & nuŋ & la nuŋ & nuŋka & nuuŋ & nunla \\
 		2\textsc{pl} & siin & la siin & siiŋka & siiŋ & siinla  \\
 		3\textsc{pl} & soo & la soo & sooka & sooŋ & soola \\
 		\lspbottomrule
 	\end{tabularx}
 \end{table}
 
 \section{Verbal morphology}
 
The simplest \ili{Marka} verbs are formed by a single verbal base. These simple bases may contain just the verb root itself, but more complex bases can contain one or more derivational affixes, such as a Weak Causative, Middle, or even a combination of the two. Suffixes inflecting for person, number, and \isi{gender} follow the stem. \ili{Marka} has two verb contexts with a single verbal base, namely the Present Habitual and Past Simple. These contexts correspond go the Present Habitual and Simple Past in \ili{Somali} (\citealt{Greenetal2015}), and to the Simple Present A and Simple Past in \ili{Maay} (\citealt{PasterRanero2015}). Like both \ili{Somali} and \ili{Maay}, inflection in \ili{Marka} for \isi{first person} singular and \isi{third person} \isi{masculine} singular are identical. Likewise, inflection for \isi{second person} singular and \isi{third person} \isi{feminine} singular are identical. The basic inflectional properties of \ili{Marka} verbs for four stem types (Bare, Weak Causative, Weak Causative + Middle, and Middle) are given in \tabref{tab:1:Present Habitual}, which shows inflection for the Present Habitual and \tabref{tab:1:Past Simple}, which shows inflection for the Past Simple. 
 
 \begin{table}
 	\caption{{Present Habitual (Marka)}}
 	\label{tab:1:Present Habitual}
 	\begin{tabularx}{\textwidth}{Xllll} 
 		\lsptoprule
 		& Bare   & WeakCaus  & WeakCaus+Middle & Middle   \\
 		& `see' & `cook' & `sell' & `sink' \\ 
 		\midrule
 		1\textsc{sg}/3\textsc{sg.m} & deje & kariʃe & iibsade & ɖubme \\
 		2\textsc{sg}/3\textsc{sg.f} & dejte  & karise & iibsate  & ɖubmate  \\
 		1\textsc{pl} & dejne & karine & iibsane & ɖubmane \\
 		2\textsc{pl} & dejtiin & karisiin & iibsatiin & ɖubmatiin \\
 		3\textsc{pl} & dejaan & kariʃaan & iibsadaan & ɖubmadaan \\
 	
 		\lspbottomrule
 	\end{tabularx}
 \end{table}
 
 \begin{table}
 	\caption{{Past Simple (Marka)}}
 	\label{tab:1:Past Simple}
 	\begin{tabularx}{\textwidth}{Xllll} 
 		\lsptoprule
 		& Bare   & WeakCaus  & WeakCaus+Middle & Middle   \\
 		& `see' & `cook' & `sell' & `sink' \\ 
 		\midrule
 		1\textsc{sg}/3\textsc{sg.m} & deji & kariʃi & iibsadi & ɖubmi \\
 		2\textsc{sg}/3\textsc{sg.f} & dejti  & karisi & iibsati  & ɖubmati  \\
 		1\textsc{pl} & dejni & karini & iibsani & ɖubmani \\
 		2\textsc{pl} & dejteen & kariseen & iibsateen & ɖubmateen \\
 		3\textsc{pl} & dejeen & kariʃeen & iibsadeen & ɖubmadeen \\
 		\lspbottomrule
 	\end{tabularx}
 \end{table}
 
 Other contexts (e.g., Present Progressive, Past Progressive, Past Habitual, and Assumptive) are formed via \isi{auxiliary} constructions containing two verbal bases; the first base is the infinitival form of the main verb which is, in turn, followed by an inflected form of an \isi{auxiliary} verb. These are comparable to those found in \ili{Somali} (\citealt{Greenetal2015}), and also to the Present Progressive, Past Progressive, and Generic Future in \ili{Maay} (\citealt{PasterRanero2015}); exceptions, however, include the Near Future and Conditional in \ili{Maay}, in which both the main verb and \isi{auxiliary} are inflected.
 
 In the \ili{Marka} Present Progressive, the infinitival main verb is followed by an inflected Present Habitual form of \emphgj{rebo} `to do.' For the Past Habitual, the main verb infinitive is followed by an inflected Past Simple form of \emphgj{jiro} `to be, exist.' The Past Progressive and Assumptive are similar in that they involve Present Habitual and Past Simple forms of \emphgj{rejo}, respectively; the precise meaning of this verb is unclear. In the interest of space, we illustrate the formation of only one \isi{auxiliary} construction, the Present Progressive of \emphgj{sugo} `to wait,' in \tabref{tab:1:Auxiliary}. 
 
 \begin{table}
 	\caption{{Auxiliary constructions -- Present Progressive (Marka)}}
 	\label{tab:1:Auxiliary}
 	\begin{tabularx}{\textwidth}{Xll} 
 		\lsptoprule
 		& \ilit{Marka}   & Gloss    \\
 		\midrule
 		1\textsc{sg}/3\textsc{sg.m} & sugo rebe & `I am/he is waiting' \\
 		2\textsc{sg}/3\textsc{sg.f} & sugo rebte  & `you are/she is waiting'  \\
 		1\textsc{pl} & sugo rebne & `we are waiting'  \\
 		2\textsc{pl} & sugo rebtiin & `you (\textsc{pl}) are waiting'  \\
 		3\textsc{pl} & sugo rebaan & `they are waiting'  \\
 		\lspbottomrule
 	\end{tabularx}
 \end{table}
 
 \ili{Marka} creates stative verbs via an \isi{auxiliary} construction composed of an adjective or adjectival participle followed by an inflected form of the irregular verb \emphgj{ahaan} `to be.' Such stative verbs are used in instances where one might find an attributive or predicate adjective in other languages. In our description of \ili{Marka}, we follow others (e.g., \citealt{Andrzejewski1969,AjelloPuglielli1988}) who have called such verbs in \ili{Somali} \textit{hybrid verbs}, although other names have also been used elsewhere in the literature. \citet{PasterRanero2015} refer to such verbs as the Simple Present B in \ili{Maay}. For the sake of comparison, one might encounter \emphgj{Way \textbf{adag}tahay} `It is difficult' in \ili{Somali}, which is similar in form to \emphgj{Ani \textbf{farahsiny}-ya} `I am happy' in \ili{Maay}. In \ili{Marka}, the situation is similar, as in \emphgj{Uus \textbf{weyn}ye} `It is big.' In each of these examples, the adjectival portion of the \isi{auxiliary} construction is italicized.
 
 Like in \ili{Maay} (and some southern dialects of \ili{Somali}), all \isi{verbal inflection} in \ili{Marka} is accomplished via suffixation. Northern \ili{Somali}, however, maintains a small class of four irregular verbs whose inflection is accomplished through prefixation in non-\isi{auxiliary} contexts. These include \emphgj{ool} `to be located,' \emphgj{odhan} `to say,' \emphgj{oqoon} `to know,' and \emphgj{imow} `to come.' These four verbs correspond to \emphgj{jaalo} `to be located,' \emphgj{ɖoho} `to say,' \emphgj{aqaano} `to know,' and \emphgj{imaʃo} `to come,' in \ili{Marka}. \tabref{tab:1:Irregular Verbs} compares inflection in Northern \ili{Somali} vs. \ili{Marka} in the Past Simple and the Past Progressive for the verb `to say.' In the Past Simple, this irregular verb is inflected via prefixation in \ili{Somali}, while in \ili{Marka}, inflection is via suffixation. Both languages employ an \isi{auxiliary} construction in the Present Progressive.
 
 \begin{table}
 	\caption{{Northern Somali vs. Marka -- `to say'}}
 	\label{tab:1:Irregular Verbs}
 	\begin{tabularx}{\textwidth}{Xllll} 
 		\lsptoprule
 		& Past Simple   &   & Past Progressive &    \\
 		& \ilit{Somali} & \ilit{Marka} & \ilit{Somali} & \ilit{Marka} \\ 
 		\midrule
 		1\textsc{sg} & idhi & ɖihi & odhanayay & ɖoho reji \\
 		2\textsc{sg}/3\textsc{sg.f} & tidhi  & ɖahti & odhanaysay  & ɖoho reti  \\
 		3\textsc{sg.m} & yidhi & ɖahji & odhanayay & ɖoho reji \\
 		1\textsc{pl} & nidhi & ɖahni & odhanaynay & ɖoho reni \\
 		2\textsc{pl} & tidhaahdeen & ɖahteen & odhanayseen & ɖoho reteen \\
 		3\textsc{pl} & yidhaahdeen & ɖahjeen & odhanayeen & ɖoho rejeen \\
 		\lspbottomrule
 	\end{tabularx}
 \end{table}
 
 Inflection in \ili{Marka} of the verb \emphgj{ahaaʃo} `to be' is irregular. \tabref{tab:1:ToBe} shows that `to be' is conjugated as expected in \isi{auxiliary} contexts like the Past Progressive, nstances and differs somewhat in the Present Habitual compared to other verbs in maintaining a unique \isi{third person} singular \isi{masculine} form (see \tabref{tab:1:Present Habitual}). For the Past Simple, \ili{Marka} has a single invariable form of 'to be' for all person/number/\isi{gender} combinations.
 
  \begin{table}
 	\caption{{Inflection of `to be' (Marka)}}
 	\label{tab:1:ToBe}
 	\begin{tabularx}{\textwidth}{Xlll} 
 		\lsptoprule
 		& Past Simple   &  Present Habitual & Past Progressive    \\  
 		\midrule
 		1\textsc{sg} & ahaaj & iʃe & ahaadeje \\
 		2\textsc{sg}/3\textsc{sg.f} & ahaaj  & ite & ahaadete   \\
 		3\textsc{sg.m} & ahaaj & ije & ahaadeje  \\
 		1\textsc{pl} & ahaaj & ine & ahaadene  \\
 		2\textsc{pl} & ahaaj & itiin & ahaadetiin  \\
 		3\textsc{pl} & ahaaj & ijaan & ahaadejaan \\
 		\lspbottomrule
 	\end{tabularx}
 \end{table}
 
 A last point pertaining to \isi{verbal morphology} in \ili{Marka} verbs concerns reduplication. Partial prefixing reduplication is used to indicate intensity or iteration of action in some verbs.  When this occurs, the maximum size of the reduplicant appears to be CVV; for example, \emphgj{dhadhaqaaqo} `to move about restlessly, fidget.' In such instances of reduplication, \ili{Marka} remains faithful to the underlying quality of the vowel in its reduplicants. We have found that \ili{Marka} also employs total prefixing reduplication to derive an adjective from a noun, as in \emphgj{buurbuur} `mountainous' (cf. \emphgj{buur} `mountain'). 
 
 \section{Concluding thoughts}
 
 This paper offers a renewed look at the nominal, pronominal, and \isi{verbal morphology} of the \ili{Marka} variety of \ili{Af-Ashraaf}. While we have not yet had the opportunity to conduct a systematic comparison of \ili{Marka} and its closest relative, \ili{Shingani}, we have taken the first steps to compare \ili{Marka} directly to two of its better-known and better-documented relatives, \ili{Maay} and \ili{Somali}. \ili{Marka} shares characteristics with both \ili{Somali} and \ili{Maay}, but conclusions concerning the extent to which \ili{Marka} aligns more closely with one or the other must await further research. At present, we endeavor to highlight those properties of \ili{Marka} that distinguish it from both \ili{Somali} and \ili{Maay}, such as its methods of encoding pluralization and \isi{gender}. While there is most certainly a great deal more work to be done, we hope that this short description lays the foundation for further inquiries into \ili{Marka} grammar and provides those with interest in the ongoing debate concerning the internal classification of East \ili{Cushitic} languages new information upon which to justify their analyses.
 
\section*{Abbreviations}

\begin{tabularx}{.45\textwidth}{ll}
\textsc{caus} & causative \\
\textsc{exc} & exclusive \\
\textsc{f} & \isi{feminine} \\
\textsc{inc} & inclusive \\
\textsc{m} & \isi{masculine} \\
\end{tabularx}
\begin{tabularx}{.55\textwidth}{ll}
\textsc{nsp} & non-specific \isi{subject pronoun} \\
\textsc{op} & object \isi{pronoun} \\
\textsc{pl} & plural \\
\textsc{sg} & singular \\
\\
\end{tabularx} 

\sloppy
\printbibliography[heading=subbibliography,notkeyword=this]

\end{document}