\documentclass[output=paper,newtxmath,modfonts,nonflat,draftmode]{langsci/langscibook} 

\author{Samson Lotven\affiliation{Indiana University Bloomington}\lastand Kelly Berkson\affiliation{Indiana University Bloomington} }

\title{The phonetics and phonology of depressor consonants in Gengbe}

\IfFileExists{../localcommands.tex}{%hack to check whether this is being compiled as part of a collection or standalone
  \usepackage{pifont}
\usepackage{savesym}

\savesymbol{downingtriple}
\savesymbol{downingdouble}
\savesymbol{downingquad}
\savesymbol{downingquint}
\savesymbol{suph}
\savesymbol{supj}
\savesymbol{supw}
\savesymbol{sups}
\savesymbol{ts}
\savesymbol{tS}
\savesymbol{devi}
\savesymbol{devu}
\savesymbol{devy}
\savesymbol{deva}
\savesymbol{N}
\savesymbol{Z}
\savesymbol{circled}
\savesymbol{sem}
\savesymbol{row}
\savesymbol{tipa}
\savesymbol{tableauxcounter}
\savesymbol{tabhead}
\savesymbol{inp}
\savesymbol{inpno}
\savesymbol{g}
\savesymbol{hanl}
\savesymbol{hanr}
\savesymbol{kuku}
\savesymbol{ip}
\savesymbol{lipm}
\savesymbol{ripm}
\savesymbol{lipn}
\savesymbol{ripn} 
% \usepackage{amsmath} 
% \usepackage{multicol}
\usepackage{qtree} 
\usepackage{tikz-qtree,tikz-qtree-compat}
% \usepackage{tikz}
\usepackage{upgreek}


%%%%%%%%%%%%%%%%%%%%%%%%%%%%%%%%%%%%%%%%%%%%%%%%%%%%
%%%                                              %%%
%%%           Examples                           %%%
%%%                                              %%%
%%%%%%%%%%%%%%%%%%%%%%%%%%%%%%%%%%%%%%%%%%%%%%%%%%%%
% remove the percentage signs in the following lines
% if your book makes use of linguistic examples
\usepackage{tipa}  
\usepackage{pstricks,pst-xkey,pst-asr}

%for sande et al
\usepackage{pst-jtree}
\usepackage{pst-node}
%\usepackage{savesym}


% \usepackage{subcaption}
\usepackage{multirow}  
\usepackage{./langsci/styles/langsci-optional} 
\usepackage{./langsci/styles/langsci-lgr} 
\usepackage{./langsci/styles/langsci-glyphs} 
\usepackage[normalem]{ulem}
%% if you want the source line of examples to be in italics, uncomment the following line
% \def\exfont{\it}
\usetikzlibrary{arrows.meta,topaths,trees}
\usepackage[linguistics]{forest}
\forestset{
	fairly nice empty nodes/.style={
		delay={where content={}{shape=coordinate,for parent={
					for children={anchor=north}}}{}}
}}
\usepackage{soul}
\usepackage{arydshln}
% \usepackage{subfloat}
\usepackage{langsci/styles/langsci-gb4e} 
   
% \usepackage{linguex}
\usepackage{vowel}

\usepackage{pifont}% http://ctan.org/pkg/pifont
\newcommand{\cmark}{\ding{51}}%
\newcommand{\xmark}{\ding{55}}%
 
 
 %Lamont
 \makeatletter
\g@addto@macro\@floatboxreset\centering
\makeatother

\usepackage{newfloat} 
\DeclareFloatingEnvironment[fileext=tbx,name=Tableau]{tableau}
  %add all your local new commands to this file
\newcommand{\downingquad}[4]{\parbox{2.5cm}{#1}\parbox{3.5cm}{#2}\parbox{2.5cm}{#3}\parbox{3.5cm}{#4}}
\newcommand{\downingtriple}[3]{\parbox{4.5cm}{#1}\parbox{3cm}{#2}\parbox{3cm}{#3}}
\newcommand{\downingdouble}[2]{\parbox{4.5cm}{#1}\parbox{6cm}{#2}}
\newcommand{\downingquint}[5]{\parbox{1.75cm}{#1}\parbox{2.25cm}{#2}\parbox{2cm}{#3}\parbox{3cm}{#4}\parbox{2cm}{#5}}
\newcolumntype{Y}{>{\centering\arraybackslash}X}
\newcolumntype{T}{>{\centering\arraybackslash}m{2cm}}

%commands for Kusmer paper below
\newcommand{\ip}{$\upiota$}
\newcommand{\lipm}{(\_{\ip-Max}}
\newcommand{\ripm}{)\_{\ip-Max}}
\newcommand{\lipn}{(\_{\ip}}
\newcommand{\ripn}{)\_{\ip}}
\renewcommand{\_}[1]{\textsubscript{#1}}


%commands for Pillion paper below
\newcommand{\suph}{\textipa{\super h}}
\newcommand{\supj}{\textipa{\super j}}
\newcommand{\supw}{\textipa{\super w}}
\newcommand{\ts}{\textipa{\t{ts}}}
\newcommand{\tS}{\textipa{\t{tS}}}
\newcommand{\devi}{\textipa{\r*i}}
\newcommand{\devu}{\textipa{\r*u}}
\newcommand{\devy}{\textipa{\r*y}}
\newcommand{\deva}{\textipa{\r*a}}
\renewcommand{\N}{\textipa{N}}
\newcommand{\Z}{\textipa{Z}}
% 

%commands for Diercks paper below
\newcommand{\circled}[1]{\begin{tikzpicture}[baseline=(word.base)]
\node[draw, rounded corners, text height=8pt, text depth=2pt, inner sep=2pt, outer sep=0pt, use as bounding box] (word) {#1};
\end{tikzpicture}
}

%commands for Pesetsky paper below
% \newcommand{\sem}[2][]{\mbox{$[\![ $\textbf{#2}$ ]\!]^{#1}$}}
\newcommand{\sem}[2][]{\mbox{$[[ $\textbf{#2}$ ]]^{#1}$}}

% \newcommand{\ripn}{{\color{red}ripn}}%this is used but never defined. Please update the definition



%commands for Lamont paper below
\newcommand{\row}[4]{
	#1. & 
    /{#2}/ & 
    [{#3}] & 
    `#4' \\ 
}
%\newcounter{tableauxcounter}
\newcommand{\tabhead}[2]{
%     \captionsetup{labelformat=empty}
%     \stepcounter{tableauxcounter}
%     \addtocounter{table}{-1}
% 	\centering
% 	\caption{Tableau \thetableauxcounter: #1}
	\caption{#1}
	\label{#2}
}
\newcommand{\candref}[2]{{(\ref{#1}#2)}}
\newcommand{\tableauref}[1]{{Tableau~\ref{#1}}}
% tableaux
\newcommand{\inp}[1]{\multicolumn{2}{|l||}{{#1}}}
\newcommand{\inpno}[1]{\multicolumn{2}{|l||}{#1}}
\newcommand{\g}{\cellcolor{lightgray}}
\newcommand{\hanl}{\HandLeft}
\newcommand{\hanr}{\HandRight}
\newcommand{\kuku}{Kuk\'{u}}

% \newcommand{\nocaption}[1]{{\color{red} Please provide a caption}}

% \providecommand{\biberror}[1]{{\color{red}#1}}

\definecolor{RED}{cmyk}{0.05,1,0.8,0}


\newfontfamily\amharicfont[Script = Ethiopic, Scale = 1.0]{AbyssinicaSIL}
\newcommand{\amh}[1]{{\amharicfont #1}}

% 
% %Gjersoe
\usepackage{textgreek}
% 
\newcommand{\viol}{\fontfamily{MinionPro-OsF}\selectfont\rotatebox{60}{$\star$}}
\newcommand{\myscalex}{0.45}
\newcommand{\myscaley}{0.65}
%\newcommand{\red}[1]{\textcolor{red}{#1}}
%\newcommand{\blue}[1]{\textcolor{blue}{#1}}
\newcommand{\epen}[1]{\colorbox{jgray}{#1}}
\newcommand{\hand}{{\normalsize \ding{43}}}
\definecolor{jgray}{gray}{0.8} 
\usetikzlibrary{positioning}
\usetikzlibrary{matrix}
\newcommand{\mora}{\textmu\xspace}
\newcommand{\si}{\textsigma\xspace}
\newcommand{\ft}{\textPhi\xspace}
\newcommand{\tone}{\texttau\xspace}
\newcommand{\word}{\textomega\xspace}
% \newcommand{\ts}{\texttslig}
\newcommand{\fns}{\footnotesize}
\newcommand{\ns}{\normalsize}
\newcommand{\vs}{\vspace{1em}}
\newcommand{\bs}{\textbackslash}   % backslash
\newcommand{\cmd}[1]{{\bf \color{red}#1}}   % highlights command
\newcommand{\scell}[2][l]{\begin{tabular}[#1]{@{}c@{}}#2\end{tabular}}
% \interfootnotelinepenalty=10000

% --- Snider Representations --- %

\newcommand{\RepLevelHh}{
\begin{minipage}{0.10\textwidth}
\begin{tikzpicture}[xscale=\myscalex,yscale=\myscaley]
%\node (syl) at (0,0) {Hi};
\node (Rt) at (0,1) {o};
\node (H) at (-0.5,2) {H};
\node (R) at (0.5,3) {h};
%\draw [thick] (syl.north) -- (Rt.south) ;
\draw [thick] (Rt.north) -- (H.south) ;
\draw [thick] (Rt.north) -- (R.south) ;
\end{tikzpicture}
\end{minipage}
}

\newcommand{\RepLevelLh}{
\begin{minipage}{0.10\textwidth}
\begin{tikzpicture}[xscale=\myscalex,yscale=\myscaley]
%\node (syl) at (0,0) {Mid2};
\node (Rt) at (0,1) {o};
\node (H) at (-0.5,2) {L};
\node (R) at (0.5,3) {h};
%\draw [thick] (syl.north) -- (Rt.south) ;
\draw [thick] (Rt.north) -- (H.south) ;
\draw [thick] (Rt.north) -- (R.south) ;
\end{tikzpicture}
\end{minipage}
}

\newcommand{\RepLevelHl}{
\begin{minipage}{0.10\textwidth}
\begin{tikzpicture}[xscale=\myscalex,yscale=\myscaley]
%\node (syl) at (0,0) {Mid1};
\node (Rt) at (0,1) {o};
\node (H) at (-0.5,2) {H};
\node (R) at (0.5,3) {l};
%\draw [thick] (syl.north) -- (Rt.south) ;
\draw [thick] (Rt.north) -- (H.south) ;
\draw [thick] (Rt.north) -- (R.south) ;
\end{tikzpicture}
\end{minipage}
}

\newcommand{\RepLevelLl}{
\begin{minipage}{0.10\textwidth}
\begin{tikzpicture}[xscale=\myscalex,yscale=\myscaley]
%\node (syl) at (0,0) {Lo};
\node (Rt) at (0,1) {o};
\node (H) at (-0.5,2) {L};
\node (R) at (0.5,3) {l};
%\draw [thick] (syl.north) -- (Rt.south) ;
\draw [thick] (Rt.north) -- (H.south) ;
\draw [thick] (Rt.north) -- (R.south) ;
\end{tikzpicture}
\end{minipage}
}

% --- Representations --- %

\newcommand{\RepLevel}{
\begin{minipage}{0.10\textwidth}
\begin{tikzpicture}[xscale=\myscalex,yscale=\myscaley]
\node (syl) at (0,0) {\textsigma};
\node (Rt) at (0,1) {o};
\node (H) at (-0.5,2) {\texttau};
\node (R) at (0.5,3) {\textrho};
\draw [thick] (syl.north) -- (Rt.south) ;
\draw [thick] (Rt.north) -- (H.south) ;
\draw [thick] (Rt.north) -- (R.south) ;
\end{tikzpicture}
\end{minipage}
}

\newcommand{\RepContour}{
\begin{minipage}{0.10\textwidth}
\begin{tikzpicture}[xscale=\myscalex,yscale=\myscaley]
\node (syl) at (0,0) {\textsigma};
\node (Rt) at (0,1) {o};
\node (H) at (-0.5,2) {\texttau};
\node (R) at (0.5,3) {\textrho};
\node (Rt2) at (1.5,1.0) {o};
%\node (H2) at (1.0,2) {$\tau$};
%\node (R2) at (2.0,2.5) {R};
\draw [thick] (syl.north) -- (Rt.south) ;
\draw [thick] (Rt.north) -- (H.south) ;
\draw [thick] (Rt.north) -- (R.south) ;
\draw [thick] (syl.north) -- (Rt2.south) ;
%\draw [thick] (Rt2.north) -- (H2.south) ;
%\draw [thick] (Rt2.north) -- (R2.south) ;
\end{tikzpicture}
\end{minipage}
}


% --- OT constraints --- %

\newcommand{\IllustrationDown}{
\begin{minipage}{0.09\textwidth}
\begin{tikzpicture}[xscale=0.7,yscale=0.45]
\node (reg) at (0,0.75) {{\small \textalpha}};
\node (arrow) at (0,0) {{\fns $\downarrow$}};
\node (Rt) at (0,-0.75) {{\small \textbeta}};
\end{tikzpicture}
\end{minipage}
}

\newcommand{\IllustrationUp}{
\begin{minipage}{0.09\textwidth}
\begin{tikzpicture}[xscale=0.7,yscale=0.45]
\node (reg) at (0,0.75) {{\small \textalpha}};
\node (arrow) at (0,0) {{\fns $\uparrow$}};
\node (Rt) at (0,-0.75) {{\small \textbeta}};
\end{tikzpicture}
\end{minipage}
}

\newcommand{\MaxAB}{
\begin{minipage}{0.09\textwidth}
\begin{tikzpicture}[xscale=0.6,yscale=0.4]
\node (max) at (0,0) {{\small \textsc{Max}}};
\node (reg) at (0.75,0.5) {{\fns \textalpha}};
\node (arrow) at (0.75,0) {{\tiny $\downarrow$}};
\node (Rt) at (0.75,-0.5) {{\fns \textbeta}};
\end{tikzpicture}
\end{minipage}
}

\newcommand{\DepAB}{
\begin{minipage}{0.09\textwidth}
\begin{tikzpicture}[xscale=0.6,yscale=0.4]
\node (max) at (0,0) {{\small \textsc{Dep}}};
\node (reg) at (0.75,0.5) {{\fns \textalpha}};
\node (arrow) at (0.75,0) {{\tiny $\downarrow$}};
\node (Rt) at (0.75,-0.5) {{\fns \textbeta}};
\end{tikzpicture}
\end{minipage}
}

\newcommand{\DepHReg}{
\begin{minipage}{0.055\textwidth}
\begin{tikzpicture}[xscale=0.6,yscale=0.4]
\node (dep) at (0,0) {{\small \textsc{Dep}}};
\node (reg) at (0,-1.0) {{\small h}};
\end{tikzpicture}
\end{minipage}
}

\newcommand{\DepLReg}{
\begin{minipage}{0.055\textwidth}
\begin{tikzpicture}[xscale=0.6,yscale=0.4]
\node (dep) at (0,0) {{\small \textsc{Dep}}};
\node (reg) at (0,-1.0) {{\small l}};
\end{tikzpicture}
\end{minipage}
}

\newcommand{\DepReg}{
\begin{minipage}{0.055\textwidth}
\begin{tikzpicture}[xscale=0.6,yscale=0.4]
\node (dep) at (0,0) {{\small \textsc{Dep}}};
\node (reg) at (0,-1.0) {{\small \textrho}};
\end{tikzpicture}
\end{minipage}
}

\newcommand{\DepTRt}{
\begin{minipage}{0.1\textwidth}
\begin{tikzpicture}[xscale=0.6,yscale=0.4]
\node (dep) at (0,0) {{\small \textsc{Dep}}};
\node (t) at (0.75,0.5) {{\fns \texttau}};
\node (arrow) at (0.75,0) {{\tiny $\downarrow$}};
\node (Rt) at (0.75,-0.5) {{\fns o}};
\end{tikzpicture}
\end{minipage}
}

\newcommand{\MaxRegRt}{
\begin{minipage}{0.1\textwidth}
\begin{tikzpicture}[xscale=0.6,yscale=0.4]
\node (max) at (0,0) {{\small \textsc{Max}}};
\node (arrow) at (0.75,0) {{\tiny $\downarrow$}};
\node (Rt) at (0.75,-0.5) {{\fns o}};
\node (reg) at (0.75,0.5) {{\fns \textrho}};
\end{tikzpicture}
\end{minipage}
}

\newcommand{\RegToneByRt}{
\begin{minipage}{0.06\textwidth}
\begin{tikzpicture}[xscale=0.6,yscale=0.5]
\node[rotate=20] (arrow1) at (-0.15,0) {{\fns $\uparrow$}};
\node[rotate=340] (arrow2) at (0.15,0) {{\fns $\uparrow$}};
\node (Rt) at (0,-0.55) {{\small o}};
\node (reg) at (0.4,0.55) {{\small \textrho}};
\node (tone) at (-0.4,0.55) {{\small \texttau}};
\end{tikzpicture}
\end{minipage}
}

\newcommand{\RegToneBySyl}{
\begin{minipage}{0.06\textwidth}
\begin{tikzpicture}[xscale=0.6,yscale=0.5]
\node[rotate=20] (arrow1) at (-0.15,0) {{\fns $\uparrow$}};
\node[rotate=340] (arrow2) at (0.15,0) {{\fns $\uparrow$}};
\node (Rt) at (0,-0.55) {{\small \textsigma}};
\node (reg) at (0.4,0.55) {{\small \textrho}};
\node (tone) at (-0.4,0.55) {{\small \texttau}};
\end{tikzpicture}
\end{minipage}
}

\newcommand{\DepTone}{
\begin{minipage}{0.055\textwidth}
\begin{tikzpicture}[xscale=0.6,yscale=0.4]
\node (dep) at (0,0) {{\small \textsc{Dep}}};
\node (tone) at (0,-1.0) {{\small \texttau}};
\end{tikzpicture}
\end{minipage}
}

\newcommand{\DepTonalRt}{
\begin{minipage}{0.055\textwidth}
\begin{tikzpicture}[xscale=0.6,yscale=0.4]
\node (dep) at (0,0) {{\small \textsc{Dep}}};
\node (tone) at (0,-1.0) {{\small o}};
\end{tikzpicture}
\end{minipage}
}

\newcommand{\DepL}{
\begin{minipage}{0.055\textwidth}
\begin{tikzpicture}[xscale=0.6,yscale=0.4]
\node (dep) at (0,0) {{\small \textsc{Dep}}};
\node (tone) at (0,-1.0) {{\small L}};
\end{tikzpicture}
\end{minipage}
}

\newcommand{\DepH}{
\begin{minipage}{0.055\textwidth}
\begin{tikzpicture}[xscale=0.6,yscale=0.4]
\node (dep) at (0,0) {{\small \textsc{Dep}}};
\node (tone) at (0,-1.0) {{\small H}};
\end{tikzpicture}
\end{minipage}
}

\newcommand{\NoMultDiff}{{\small *loh}}
\newcommand{\Alt}{{\small \textsc{Alt}}}
\newcommand{\NoSkip}{{\small \scell{\textsc{No}\\\textsc{Skip}}}}


\newcommand{\RegDomRt}{
\begin{minipage}{0.030\textwidth}
\begin{tikzpicture}[xscale=0.6,yscale=0.5]
\node (arrow) at (0,0) {{\fns $\downarrow$}};
\node (Rt) at (0,-0.55) {{\small o}};
\node (reg) at (0,0.55) {{\small \textrho}};
\end{tikzpicture}
\end{minipage}
}

\newcommand{\DepRegRt}{
\begin{minipage}{0.1\textwidth}
\begin{tikzpicture}[xscale=0.6,yscale=0.4]
\node (dep) at (0,0) {{\small \textsc{Dep}}};
\node (arrow) at (0.75,0) {{\tiny $\downarrow$}};
\node (Rt) at (0.75,-0.5) {{\fns o}};
\node (reg) at (0.75,0.5) {{\fns \textrho}};
\end{tikzpicture}
\end{minipage}
}

% unused

\newcommand{\ToneByRt}{
\begin{minipage}{0.05\textwidth}
\begin{tikzpicture}[xscale=0.6,yscale=0.5]
\node (arrow) at (0,0) {{\fns $\uparrow$}};
\node (Rt) at (0,-0.55) {{\small o}};
\node (tone) at (0,0.55) {{\small \texttau}};
\end{tikzpicture}
\end{minipage}
}

\newcommand{\RegByRt}{
\begin{minipage}{0.05\textwidth}
\begin{tikzpicture}[xscale=0.6,yscale=0.5]
\node (arrow) at (0,0) {{\fns $\uparrow$}};
\node (Rt) at (0,-0.55) {{\small o}};
\node (reg) at (0,0.55) {{\small \textrho}};
\end{tikzpicture}
\end{minipage}
}

\newcommand{\ToneDomRt}{
\begin{minipage}{0.05\textwidth}
\begin{tikzpicture}[xscale=0.6,yscale=0.5]
\node (arrow) at (0,0) {{\fns $\downarrow$}};
\node (Rt) at (0,-0.55) {{\small o}};
\node (tone) at (0,0.55) {{\small \texttau}};
\end{tikzpicture}
\end{minipage}
}

% --- OT tableaus --- %

% Sec. 3.2, first tabl.

\newcommand{\OTHLInput}{
\begin{minipage}{0.17\textwidth}
\begin{tikzpicture}[xscale=\myscalex,yscale=\myscaley]
\node (tone) at (2,0) {(= H)};
\node (syl) at (0,0) {\textsigma};
\node (Rt) at (0,1) {o};
\node (H) at (-0.5,2) {H};
\node (R) at (0.5,3) {h};
\node (Rt2) at (1.5,1.0) {o};
%\node (H2) at (1.0,2) {\epen{L}};
\node (R2) at (2.0,3) {\blue{l}};
\draw [thick] (syl.north) -- (Rt.south) ;
\draw [thick] (Rt.north) -- (H.south) ;
\draw [thick] (Rt.north) -- (R.south) ;
\draw [thick] (syl.north) -- (Rt2.south) ;
%\draw [dashed] (Rt2.north) -- (H2.south) ;
%\draw [dashed] (Rt2.north) -- (R2.south) ;
\end{tikzpicture}
\end{minipage}
}

\newcommand{\OTHLWinner}{
\begin{minipage}{0.17\textwidth}
\begin{tikzpicture}[xscale=\myscalex,yscale=\myscaley]
\node (tone) at (2,0) {(= HL)};
\node (syl) at (0,0) {\textsigma};
\node (Rt) at (0,1) {o};
\node (H) at (-0.5,2) {H};
\node (R) at (0.5,3) {h};
\node (Rt2) at (1.5,1.0) {o};
\node (H2) at (1.0,2) {\epen{L}};
\node (R2) at (2.0,3) {\blue{l}};
\draw [thick] (syl.north) -- (Rt.south) ;
\draw [thick] (Rt.north) -- (H.south) ;
\draw [thick] (Rt.north) -- (R.south) ;
\draw [thick] (syl.north) -- (Rt2.south) ;
\draw [dashed] (Rt2.north) -- (H2.south) ;
\draw [dashed] (Rt2.north) -- (R2.south) ;
\end{tikzpicture}
\end{minipage}
}

\newcommand{\OTHLSpreadingHOnly}{
\begin{minipage}{0.17\textwidth}
\begin{tikzpicture}[xscale=\myscalex,yscale=\myscaley]
\node (tone) at (2,0) {(= HM)};
\node (syl) at (0,0) {\textsigma};
\node (Rt) at (0,1) {o};
\node (H) at (-0.5,2) {H};
\node (R) at (0.5,3) {h};
\node (Rt2) at (1.5,1.0) {o};
%\node (H2) at (1.0,2) {\epen{L}};
\node (R2) at (2.0,3) {\blue{l}};
\draw [thick] (syl.north) -- (Rt.south) ;
\draw [thick] (Rt.north) -- (H.south) ;
\draw [thick] (Rt.north) -- (R.south) ;
\draw [thick] (syl.north) -- (Rt2.south) ;
\draw [dashed] (Rt2.north) -- (R2.south) ;
\draw [dashed] (Rt2.north) -- (H.south) ;
\end{tikzpicture}
\end{minipage}
}

\newcommand{\OTHLInsertH}{
\begin{minipage}{0.17\textwidth}
\begin{tikzpicture}[xscale=\myscalex,yscale=\myscaley]
\node (tone) at (2,0) {(= HM)};
\node (syl) at (0,0) {\textsigma};
\node (Rt) at (0,1) {o};
\node (H) at (-0.5,2) {H};
\node (R) at (0.5,3) {h};
\node (Rt2) at (1.5,1.0) {o};
\node (H2) at (1.0,2) {\epen{H}};
\node (R2) at (2.0,3) {\blue{l}};
\draw [thick] (syl.north) -- (Rt.south) ;
\draw [thick] (Rt.north) -- (H.south) ;
\draw [thick] (Rt.north) -- (R.south) ;
\draw [thick] (syl.north) -- (Rt2.south) ;
\draw [dashed] (Rt2.north) -- (H2.south) ;
\draw [dashed] (Rt2.north) -- (R2.south) ;
\end{tikzpicture}
\end{minipage}
}

\newcommand{\OTHLOverwriting}{
\begin{minipage}{0.17\textwidth}
\begin{tikzpicture}[xscale=\myscalex,yscale=\myscaley]
\node (syl) at (0,0) {\textsigma};
\node (Rt) at (0,1) {o};
\node (H) at (-0.5,2) {H};
\node (R) at (0.5,3) {h};
\node (Rt2) at (1.5,1.0) {o};
%\node (H2) at (1.0,2) {\epen{L}};
\node (R2) at (2.0,3) {\blue{l}};
\draw [thick] (syl.north) -- (Rt.south) ;
\draw [thick] (Rt.north) -- (H.south) ;
\draw [thick] (Rt.north) -- (R.south) ;
\draw [thick] (syl.north) -- (Rt2.south) ;
%\draw [dashed] (Rt2.north) -- (H2.south) ;
\draw [dashed] (Rt.north) -- (R2.south) ;
\node (del) at (0.3,1.9) {\textbf{=}};
\end{tikzpicture}
\end{minipage}
}

\newcommand{\OTHLSpreading}{
\begin{minipage}{0.17\textwidth}
\begin{tikzpicture}[xscale=\myscalex,yscale=\myscaley]
\node (syl) at (0,0) {\textsigma};
\node (Rt) at (0,1) {o};
\node (H) at (-0.5,2) {H};
\node (R) at (0.5,3) {h};
\node (Rt2) at (1.5,1.0) {o};
%\node (H2) at (1.0,2) {\epen{L}};
\node (R2) at (2.0,3) {\blue{l}};
\draw [thick] (syl.north) -- (Rt.south) ;
\draw [thick] (Rt.north) -- (H.south) ;
\draw [thick] (Rt.north) -- (R.south) ;
\draw [thick] (syl.north) -- (Rt2.south) ;
%\draw [dashed] (Rt2.north) -- (H2.south) ;
\draw [dashed] (Rt2.north) -- (H.south) ;
\draw [dashed] (Rt2.north) -- (R.south) ;
\end{tikzpicture}
\end{minipage}
}

% Sec. 4.2, second tabl.: phrase-medial position

\newcommand{\OTHnoLInput}{
\begin{minipage}{0.17\textwidth}
\begin{tikzpicture}[xscale=\myscalex,yscale=\myscaley]
\node (tone) at (2,0) {(= H)};
\node (syl) at (0,0) {\textsigma};
\node (Rt) at (0,1) {o};
\node (H) at (-0.5,2) {H};
\node (R) at (0.5,3) {h};
\node (Rt2) at (1.5,1.0) {o};
%\node (H2) at (1.0,2) {\epen{L}};
%\node (R2) at (2.0,3) {\blue{l}};
\draw [thick] (syl.north) -- (Rt.south) ;
\draw [thick] (Rt.north) -- (H.south) ;
\draw [thick] (Rt.north) -- (R.south) ;
\draw [thick] (syl.north) -- (Rt2.south) ;
\end{tikzpicture}
\end{minipage}
}

\newcommand{\OTHnoLEpenth}{
\begin{minipage}{0.17\textwidth}
\begin{tikzpicture}[xscale=\myscalex,yscale=\myscaley]
\node (tone) at (2,0) {(= HM)};
\node (syl) at (0,0) {\textsigma};
\node (Rt) at (0,1) {o};
\node (H) at (-0.5,2) {H};
\node (R) at (0.5,3) {h};
\node (Rt2) at (1.5,1.0) {o};
\node (H2) at (1.0,2) {\epen{L}};
\node (R2) at (2.0,3) {\epen{h}};
\draw [thick] (syl.north) -- (Rt.south) ;
\draw [thick] (Rt.north) -- (H.south) ;
\draw [thick] (Rt.north) -- (R.south) ;
\draw [thick] (syl.north) -- (Rt2.south) ;
\draw [dashed] (Rt2.north) -- (H2.south) ;
\draw [dashed] (Rt2.north) -- (R2.south) ;
\end{tikzpicture}
\end{minipage}
}

\newcommand{\OTHnoLSpreading}{
\begin{minipage}{0.17\textwidth}
\begin{tikzpicture}[xscale=\myscalex,yscale=\myscaley]
\node (tone) at (2,0) {(= HH)};
\node (syl) at (0,0) {\textsigma};
\node (Rt) at (0,1) {o};
\node (H) at (-0.5,2) {H};
\node (R) at (0.5,3) {h};
\node (Rt2) at (1.5,1.0) {o};
%\node (H2) at (1.0,2) {\epen{L}};
%\node (R2) at (2.0,3) {\blue{l}};
\draw [thick] (syl.north) -- (Rt.south) ;
\draw [thick] (Rt.north) -- (H.south) ;
\draw [thick] (Rt.north) -- (R.south) ;
\draw [thick] (syl.north) -- (Rt2.south) ;
\draw [dashed] (Rt2.north) -- (H.south) ;
\draw [dashed] (Rt2.north) -- (R.south) ;
\end{tikzpicture}
\end{minipage}
}

% Sec. 4.2, third tabl., LM is unaffected by L\%

\newcommand{\OTLMInput}{
\begin{minipage}{0.2\textwidth}
\begin{tikzpicture}[xscale=\myscalex,yscale=\myscaley]
\node (tone) at (2,0) {(= LM)};
\node (syl) at (0,0) {\textsigma};
\node (Rt) at (0,1) {o};
\node (H) at (-0.5,2) {L};
\node (R) at (0.5,3) {l};
\node (Rt2) at (1.5,1.0) {o};
\node (H2) at (1.0,2) {L};
\node (R2) at (2.0,3) {h};
\node (R3) at (3.0,3) {\blue{l}};
\draw [thick] (syl.north) -- (Rt.south) ;
\draw [thick] (Rt.north) -- (H.south) ;
\draw [thick] (Rt.north) -- (R.south) ;
\draw [thick] (syl.north) -- (Rt2.south) ;
\draw [thick] (Rt2.north) -- (H2.south) ;
\draw [thick] (Rt2.north) -- (R2.south) ;
\end{tikzpicture}
\end{minipage}
}

\newcommand{\OTLMReplace}{
\begin{minipage}{0.2\textwidth}
\begin{tikzpicture}[xscale=\myscalex,yscale=\myscaley]
\node (tone) at (2,0) {(= LL)};
\node (syl) at (0,0) {\textsigma};
\node (Rt) at (0,1) {o};
\node (H) at (-0.5,2) {L};
\node (R) at (0.5,3) {l};
\node (Rt2) at (1.5,1.0) {o};
\node (H2) at (1.0,2) {L};
\node (R2) at (2.0,3) {h};
\node (R3) at (3.0,3) {\blue{l}};
\draw [thick] (syl.north) -- (Rt.south) ;
\draw [thick] (Rt.north) -- (H.south) ;
\draw [thick] (Rt.north) -- (R.south) ;
\draw [thick] (syl.north) -- (Rt2.south) ;
\draw [thick] (Rt2.north) -- (H2.south) ;
\draw [thick] (Rt2.north) -- (R2.south) ;
\draw [dashed] (Rt2.north) -- (R3.south) ;
\node (del) at (1.8,2.1) {\textbf{=}};
\end{tikzpicture}
\end{minipage}
}

\newcommand{\OTLMTwoReg}{
\begin{minipage}{0.2\textwidth}
\begin{tikzpicture}[xscale=\myscalex,yscale=\myscaley]
\node (tone) at (2,0) {(= LML)};
\node (syl) at (0,0) {\textsigma};
\node (Rt) at (0,1) {o};
\node (H) at (-0.5,2) {L};
\node (R) at (0.5,3) {l};
\node (Rt2) at (1.5,1.0) {o};
\node (H2) at (1.0,2) {L};
\node (R2) at (2.0,3) {h};
\node (R3) at (3.0,3) {\blue{l}};
\draw [thick] (syl.north) -- (Rt.south) ;
\draw [thick] (Rt.north) -- (H.south) ;
\draw [thick] (Rt.north) -- (R.south) ;
\draw [thick] (syl.north) -- (Rt2.south) ;
\draw [thick] (Rt2.north) -- (H2.south) ;
\draw [thick] (Rt2.north) -- (R2.south) ;
\draw [dashed] (Rt2.north) -- (R3.south) ;
\end{tikzpicture}
\end{minipage}
}

% Sec. 4.2, fourth tabl., L is affected by L\% but M is not

\newcommand{\OTLInput}{
\begin{minipage}{0.17\textwidth}
\begin{tikzpicture}[xscale=\myscalex,yscale=\myscaley]
\node (tone) at (2,0) {(= L)};
\node (syl) at (0,0) {\textsigma};
\node (Rt) at (0,1) {o};
\node (H) at (-0.5,2) {L};
\node (R) at (0.5,3) {l};
\node (R2) at (2,3) {\blue{l}};
\draw [thick] (syl.north) -- (Rt.south) ;
\draw [thick] (Rt.north) -- (H.south) ;
\draw [thick] (Rt.north) -- (R.south) ;
\end{tikzpicture}
\end{minipage}
}

\newcommand{\OTLLowered}{
\begin{minipage}{0.17\textwidth}
\begin{tikzpicture}[xscale=\myscalex,yscale=\myscaley]
\node (tone) at (2,0) {(= LL)};
\node (syl) at (0,0) {\textsigma};
\node (Rt) at (0,1) {o};
\node (H) at (-0.5,2) {L};
\node (R) at (0.5,3) {l};
\node (R2) at (2,3) {\blue{l}};
\draw [thick] (syl.north) -- (Rt.south) ;
\draw [thick] (Rt.north) -- (H.south) ;
\draw [thick] (Rt.north) -- (R.south) ;
\draw [dashed] (Rt.north) -- (R2.south) ;
\end{tikzpicture}
\end{minipage}
}

\newcommand{\OTMInput}{
\begin{minipage}{0.17\textwidth}
\begin{tikzpicture}[xscale=\myscalex,yscale=\myscaley]
\node (tone) at (2,0) {(= M)};
\node (syl) at (0,0) {\textsigma};
\node (Rt) at (0,1) {o};
\node (H) at (-0.5,2) {L};
\node (R) at (0.5,3) {h};
\node (R2) at (2,3) {\blue{l}};
\draw [thick] (syl.north) -- (Rt.south) ;
\draw [thick] (Rt.north) -- (H.south) ;
\draw [thick] (Rt.north) -- (R.south) ;
\end{tikzpicture}
\end{minipage}
}

\newcommand{\OTMLowered}{
\begin{minipage}{0.17\textwidth}
\begin{tikzpicture}[xscale=\myscalex,yscale=\myscaley]
\node (tone) at (2,0) {(= ML)};
\node (syl) at (0,0) {\textsigma};
\node (Rt) at (0,1) {o};
\node (H) at (-0.5,2) {L};
\node (R) at (0.5,3) {h};
\node (R2) at (2,3) {\blue{l}};
\draw [thick] (syl.north) -- (Rt.south) ;
\draw [thick] (Rt.north) -- (H.south) ;
\draw [thick] (Rt.north) -- (R.south) ;
\draw [dashed] (Rt.north) -- (R2.south) ;
\end{tikzpicture}
\end{minipage}
}

% Sec. 4.2, fifth tableau, polar questions with level tones

\newcommand{\OTLPolIn}{
\begin{minipage}{0.20\textwidth}
\begin{tikzpicture}[xscale=\myscalex-0.05,yscale=\myscaley-0.05]
\node (tone) at (3.5,0) {(= L)};
\node (syl) at (0,0) {\textsigma};
\node (syl2) at (2,0) {\red{\textsigma}};
\node (Rt) at (0,1) {o};
\node (H) at (-0.5,2) {L};
\node (R) at (0.5,3) {l};
\node (Rt2) at (2,1) {\red{o}};
\draw [thick] (syl.north) -- (Rt.south) ;
\draw [thick,red] (syl2.north) -- (Rt2.south) ;
\draw [thick] (Rt.north) -- (H.south) ;
\draw [thick] (Rt.north) -- (R.south) ;
\end{tikzpicture}
\end{minipage}
}

\newcommand{\OTLPolDef}{
\begin{minipage}{0.20\textwidth}
\begin{tikzpicture}[xscale=\myscalex-0.05,yscale=\myscaley-0.05]
\node (tone) at (3.5,0) {(= L.M)};
\node (syl) at (0,0) {\textsigma};
\node (syl2) at (2,0) {\red{\textsigma}};
\node (Rt) at (0,1) {o};
\node (H) at (-0.5,2) {L};
\node (R) at (0.5,3) {l};
\node (H2) at (1.5,2) {\epen{L}};
\node (R2) at (2.5,3) {\epen{h}};
\node (Rt2) at (2,1) {\red{o}};
\draw [thick] (syl.north) -- (Rt.south) ;
\draw [thick,red] (syl2.north) -- (Rt2.south) ;
\draw [thick] (Rt.north) -- (H.south) ;
\draw [thick] (Rt.north) -- (R.south) ;
\draw [semithick,dashed] (Rt2.north) -- (H2.south) ;
\draw [semithick,dashed] (Rt2.north) -- (R2.south) ;
\end{tikzpicture}
\end{minipage}
}

\newcommand{\OTLPolAlt}{
\begin{minipage}{0.20\textwidth}
\begin{tikzpicture}[xscale=\myscalex-0.05,yscale=\myscaley-0.05]
\node (tone) at (3.5,0) {(= L.L)};
\node (syl) at (0,0) {\textsigma};
\node (syl2) at (2,0) {\red{\textsigma}};
\node (Rt) at (0,1) {o};
\node (H) at (-0.5,2) {L};
\node (R) at (0.5,3) {l};
\node (Rt2) at (2,1) {\red{o}};
\draw [thick] (syl.north) -- (Rt.south) ;
\draw [thick,red] (syl2.north) -- (Rt2.south) ;
\draw [thick] (Rt.north) -- (H.south) ;
\draw [thick] (Rt.north) -- (R.south) ;
\draw [semithick,dashed] (Rt2.north) -- (H.south) ;
\draw [semithick,dashed] (Rt2.north) -- (R.south) ;
\end{tikzpicture}
\end{minipage}
}

% Sec. 4.2, sixth tableau, polar questions with contour tones

\newcommand{\OTLLPolIn}{
\begin{minipage}{0.23\textwidth}
\begin{tikzpicture}[xscale=\myscalex-0.05,yscale=\myscaley-0.05]
\node (tone) at (5.2,0) {(= L)};
\node (syl) at (0,0) {\textsigma};
\node (syl3) at (3.4,0) {\red{\textsigma}};
\node (Rt) at (0,1) {o};
\node (Rt2) at (1.7,1) {o};
\node (Rt3) at (3.4,1) {\red{o}};
\node (H) at (-0.5,2) {L};
\node (R) at (0.5,3) {l};
\draw [thick] (syl.north) -- (Rt.south) ;
\draw [thick] (syl.north) -- (Rt2.south) ;
\draw [thick,red] (syl3.north) -- (Rt3.south) ;
\draw [thick] (Rt.north) -- (H.south) ;
\draw [thick] (Rt.north) -- (R.south) ;
\end{tikzpicture}
\end{minipage}
}

\newcommand{\OTLLPolDef}{
\begin{minipage}{0.23\textwidth}
\begin{tikzpicture}[xscale=\myscalex-0.05,yscale=\myscaley-0.05]
\node (tone) at (5.2,0) {(= L.M)};
\node (syl) at (0,0) {\textsigma};
\node (syl3) at (3.4,0) {\red{\textsigma}};
\node (Rt) at (0,1) {o};
\node (Rt2) at (1.7,1) {o};
\node (Rt3) at (3.4,1) {\red{o}};
\node (H) at (-0.5,2) {L};
\node (R) at (0.5,3) {l};
\node (H3) at (2.9,2) {\epen{L}};
\node (R3) at (3.9,3) {\epen{h}};
\draw [thick] (syl.north) -- (Rt.south) ;
\draw [thick] (syl.north) -- (Rt2.south) ;
\draw [thick,red] (syl3.north) -- (Rt3.south) ;
\draw [thick] (Rt.north) -- (H.south) ;
\draw [thick] (Rt.north) -- (R.south) ;
\draw [dashed] (Rt3.north) -- (H3.south) ;
\draw [dashed] (Rt3.north) -- (R3.south) ;
\end{tikzpicture}
\end{minipage}
}

\newcommand{\OTLLPolSkip}{
\begin{minipage}{0.23\textwidth}
\begin{tikzpicture}[xscale=\myscalex-0.05,yscale=\myscaley-0.05]
\node (tone) at (5.2,0) {(= L.L)};
\node (syl) at (0,0) {\textsigma};
\node (syl3) at (3.4,0) {\red{\textsigma}};
\node (Rt) at (0,1) {o};
\node (Rt2) at (1.7,1) {o};
\node (Rt3) at (3.4,1) {\red{o}};
\node (H) at (-0.5,2) {L};
\node (R) at (0.5,3) {l};
\draw [thick] (syl.north) -- (Rt.south) ;
\draw [thick] (syl.north) -- (Rt2.south) ;
\draw [thick,red] (syl3.north) -- (Rt3.south) ;
\draw [thick] (Rt.north) -- (H.south) ;
\draw [thick] (Rt.north) -- (R.south) ;
\draw [dashed] (Rt3.north) -- (H.south) ;
\draw [dashed] (Rt3.north) -- (R.south) ;
\end{tikzpicture}
\end{minipage}
}  
  
\newcommand{\ilit}[1]{#1\il{#1}}    
\newcommand{\isit}[1]{#1\is{#1}}  

\makeatletter
\let\thetitle\@title
\let\theauthor\@author 
\makeatother

\newcommand{\togglepaper}[1][0]{ 
  \bibliography{../localbibliography}
  %% hyphenation points for line breaks
%% Normally, automatic hyphenation in LaTeX is very good
%% If a word is mis-hyphenated, add it to this file
%%
%% add information to TeX file before \begin{document} with:
%% %% hyphenation points for line breaks
%% Normally, automatic hyphenation in LaTeX is very good
%% If a word is mis-hyphenated, add it to this file
%%
%% add information to TeX file before \begin{document} with:
%% \include{localhyphenation}
\hyphenation{
affri-ca-te
affri-ca-tes
com-ple-ments
par-a-digm
Sha-ron
Kings-ton
phe-nom-e-non
Daul-ton
Abu-ba-ka-ri
Ngo-nya-ni
Clem-ents 
King-ston
Tru-cken-brodt
Tab-leau
cophono-logies
mark-edness
Ti-gri-nya
a-mong
Car-stens
Lu-bu-ku-su
}
\hyphenation{
affri-ca-te
affri-ca-tes
com-ple-ments
par-a-digm
Sha-ron
Kings-ton
phe-nom-e-non
Daul-ton
Abu-ba-ka-ri
Ngo-nya-ni
Clem-ents 
King-ston
Tru-cken-brodt
Tab-leau
cophono-logies
mark-edness
Ti-gri-nya
a-mong
Car-stens
Lu-bu-ku-su
}
  \papernote{\scriptsize\normalfont
    \theauthor.
    \thetitle. 
    To appear in: 
    Emily Clem,   Peter Jenks \& Hannah Sande.
    Theory and description in African Linguistics: Selected papers from the 47th Annual Conference on African Linguistics.
    Berlin: Language Science Press. [preliminary page numbering]
  }
  \pagenumbering{roman}
  \setcounter{chapter}{#1}
  \addtocounter{chapter}{-1}
}

\newcommand{\upstep}{\textupstep}


% \newcounter{tableauxcounter}

\renewcommand{\textltailn}{ɲ}
\renewcommand{\textbardotlessj}{ɟ}

\newcommand{\emphkh}[1]{\textit{#1}} %originally \textbf, banned by the guidelines



\definecolor{lsDOIGray}{cmyk}{0,0,0,0.45}


\newcommand{\xuparrow}[1]{%
  {\left\uparrow\vbox to #1{}\right.\kern-\nulldelimiterspace}
}
\renewcommand \textupstep[1]{\char"A71B#1}
\renewcommand \textdownstep[1]{\char"A71C#1}
 
 \newcommand{\ꜛ}{\textsf{ꜛ}}
 
\def\biberror{\undefined}


\newcommand{\OTbox}[1]{\resizebox{.88\textwidth}{!}{#1}}
 
  \togglepaper[14]
}{}


\abstract{The interaction between initial voiced obstruents and lower f0 has been noted for a variety of languages (\citealt{Chistovich1969,Stevens1973,Bradshaw1999,Tang2008}, to name a few). In some languages, phonetic consonant-f0 interactions that alter f0 register and/or contour can be phonologized as consonant-tone interactions \citep{Maran1973,Matisoff1973}. In Gengbe, a Gbe language spoken in Southern Togo and Benin, obstruent voicing displays several interactions with f0 register and contour. The goal of this study is to present the synchronic system of Gengbe consonant-f0 interactions with an eye toward the larger question of phonologization. This paper presents both phonetic and phonological data for discussion. Preliminary acoustic data suggest initial voiced obstruents lower the register f0 of following Low and High tone vowels. Phonological data suggest that tonal contour effects, which change underlying High tone to Rising tone in some environments, vary based on syntactic category – voiced obstruents trigger Rising tone in nouns, while voiced obstruents and sonorants trigger Rising tone in verbs. This paper offers a snapshot of a system where at least some consonant-f0 interactions have been phonologized, adding to the broader understanding of tonogenetic processes.}

\begin{document}
\maketitle
 \newcommand{\nocaption}[1]{{\color{red} Please provide a caption}}
 
%%please move the includegraphics inside the {figure} environment
%%\includegraphics[width=\textwidth]{figures/LotvenBerksonGengbeREVMay1-img1}

 
%%please move the includegraphics inside the {figure} environment
%%\includegraphics[width=\textwidth]{figures/LotvenBerksonGengbeREVMay1-img2}

 
%%please move the includegraphics inside the {figure} environment
%%\includegraphics[width=\textwidth]{figures/LotvenBerksonGengbeREVMay1-img3.png}

 
%%please move the includegraphics inside the {figure} environment
%%\includegraphics[width=\textwidth]{figures/LotvenBerksonGengbeREVMay1-img4.jpg}

 
%%please move the includegraphics inside the {figure} environment
%%\includegraphics[width=\textwidth]{figures/LotvenBerksonGengbeREVMay1-img5.png}

 
%%please move the includegraphics inside the {figure} environment
%%\includegraphics[width=\textwidth]{figures/LotvenBerksonGengbeREVMay1-img6.png}

\section{Introduction}\label{sec:lotven:1}
It is often the case that a binary phonological contrast, for example [+/−voice] in onset consonants, is realized via differences in multiple phonetic correlates \citep{wright2004}. Cues related to the voicing of onset consonants, for instance, may include (but are not limited to) Voice Onset Time \citep{Lisker1964}, formant transitions \citep{Stevens1973}, \isi{f0 contour} of the following vowel \citep{Chistovich1969}, and fundamental frequency (\isi{f0}) \isi{register} of the following vowel \citep{Shimizu1989}. In the present work, we probe the connection between the feature [+voice] in onset consonants and lowered \isi{f0} in subsequent vowels, a relationship observed in many unrelated languages (\citealt{Bradshaw1999}; \citealt{Tang2008}). Though voicing-\isi{f0} interactions can occur with coda consonants—as in \ili{Vietnamese} and some \ili{Tibeto-Burman} languages (\citealt{Maran1973}; \citealt{Matisoff1973}), for instance—we \isi{focus} on the interaction between onset consonants and \isi{f0} in \ili{Gengbe}, a \ili{Gbe} language spoken in southern Togo and Benin.

Consonants that trigger \isi{f0} lowering are generally called “depressor consonants”, and have been studied in a variety of other languages but not systematically in \ili{Gengbe}. Two effects are discussed here. First, depressor consonants in \ili{Gengbe} trigger \isi{f0} \isi{register} lowering of Low (L) \isi{tone}, meaning that L is lower across the entire vowel after a \isi{depressor consonant} than after other consonants. This \isi{register} effect is seen in some High (H) \isi{tone} contexts as well. Second, depressor consonants in \ili{Gengbe} can allow initial \isi{f0} lowering of H \isi{tone} in some phonological, morphophonological, and syntactic contexts. This initial f0 lowering results in a contour effect, where \isi{f0} is low at the onset of the vowel and rises across the time-course of the vowel. This contour effect differs across morphological domains in that different sets of consonants act as depressors in nouns and in verbs (where different environments produce the contour effect). While only voiced obstruents act as depressors in nouns, voiced obstruents and sonorants act as depressors in verbs. Similar observations have been made about other languages: differential treatment of onset types as depressors is attested in \ili{Ewe} \citep{Bradshaw1999} and in \ili{Zina Kotoko} \citep{Odden2007}, for instance. 

As no prior discussion of these effects in \ili{Gengbe} exists, this paper presents a thorough description of the depressor effects that have been observed in \ili{Gengbe} thus far. The data were collected during 18 months of fieldwork with a native speaker consultant. Acoustic analysis is included where possible, in order to illustrate the phonetic effect of depressor consonants on \isi{f0} in following vowels. 

Our ultimate goal is to produce a thorough description and analysis of depressor effects in \ili{Gengbe}, and doing so will necessitate taking into account phonetic, phonological, morphophonological, and syntactic factors, at a minimum. Here, however, our \isi{focus} is on the phonetics and phonology of \ili{Gengbe} depressor consonants, and so it is important to be clear about what we mean when we refer to phonetics and phonology. The distinction we aim to highlight is this: a phonetic depressor effect is one wherein \isi{f0} is lowered but a new tonal category is not created, such that L after a \isi{depressor consonant} has a lower fundamental frequency but is still L; a phonologization of that effect yields a new tonal category. This hinges on the distinction between \isi{f0} and \isi{tone}, which is described by \citet[5]{Yip2002} as follows: “\isi{f0} is an acoustic term referring to the signal itself… Tone…is a linguistic term. It refers to a phonological category that distinguishes two words or utterances.” The distinction between those effects that are phonetic and those that have been phonologized is not always straightforward, and our understanding of depressor effects in \ili{Gengbe} is under development. To understand whether phonologization of depressor effects in \ili{Gengbe} is in progress, however, we must first develop an understanding of the depressor effects that exist in the language at present. That is the goal here, and doing so helps suggest important types of data to elicit in future. 

The remainder of the paper reads as follows. \sectref{sec:lotven:2} provides relevant background on depressor consonants and \isi{tone} in \ili{Gbe} languages. \sectref{sec:lotven:3} discusses the research aims and methodology employed in the current study. \sectref{sec:lotven:4} discusses \isi{f0} effects in nouns (\sectref{sec:lotven:4.1}) and verbs (\sectref{sec:lotven:4.2}). \sectref{sec:lotven:5} concludes the paper. 


\section{Background on depressor consonants}\label{sec:lotven:2}

Previous research indicates that voiced obstruents—and sonorants, albeit less commonly—can act as depressor consonants (\citealt{Ohala1973}; \citealt{Bradshaw1999}; \citealt{Tang2008}), meaning that they can trigger lowering (or depression) of \isi{f0} in adjacent vowels. Relevant for this discussion is the vowel immediately following a \isi{depressor consonant} onset. Depressor effects fall into two broad categories. F0 \isi{register} effects—schematized below in \figref{sec:lotven:1}a—are those that perseverate across the entire vowel, as in \ili{Japanese} \citep{Oglesbee2008}. F0 contour effects, schematized in \figref{sec:lotven:1}b, are localized to the left edge of the vowel, following the consonant constriction release. This results in a rising \isi{pitch pattern}, as in \ili{English} (\citealt{Lea1973}; \citealt{Oglesbee2008}). 


\begin{figure} 
\todo[inline]{Crop file and use subfigure}

\includegraphics[width=\textwidth]{figures/fig-lotven-1.png}  
\caption{\label{fig:lotven:samson:1} Schematized representation of f0 register depressor effect (1a, in the left panel) and f0 contour depressor effect (1b, in the right panel). T represents a voiceless obstruent and D represents a voiced obstruent}
\end{figure}


Evidence for a link between [+voice] in consonants and \isi{f0} lowering in subsequent vowels has been found in many languages (\citealt{Bradshaw1999}; \citealt{Tang2008}), and a relation between voiced obstruents and \isi{tone} may also exist \citep[5]{Yip2002}. In a \isi{tone} language where lowered \isi{f0} on a vowel already serves as a crucial cue for L \isi{tone}, co-opting \isi{f0} lowering as a redundant cue for the feature [+voice] leads to complications in the phonological system. This conflict may lead to distributional restrictions: \ili{Thai}, for instance, disallows voiced stops in the onsets of high \isi{tone} syllables \citep{perkins2011}; in \ili{Kera}, limitations on obstruent voicing and \isi{tone} produce a situation where the full Low-Mid-High tonal contrast is only available in syllables with sonorant onsets \citep{Pearce2005}; and in \ili{Ewe}, a language closely related to \ili{Gengbe}, \citet{Ansre1961} analyzes the tonal system as having a “Non-High” toneme that is realized as L after a \isi{voiced obstruent} and Mid (M) after a voiceless obstruent, a claim that we will revisit in \sectref{sec:lotven:4}. 

These distributional restrictions hold true synchronically, but it is also worth considering their diachronic development. What might phonologization of a phonetic depressor effect look like? One process—dubbed “Tonal Bifurcation” in \citet{Hyman2013enlarging}—is outlined in \figref{sec:lotven:2}. In the first stage of Tonal Bifurcation (\figref{sec:lotven:2}a), an \isi{f0 contour} effect (as discussed earlier for \ili{English}) is present in a language with two \isi{register} tones (H and L). The next stage of the process (\figref{sec:lotven:2}b) involves innovation of a contrasting Rising (LH) \isi{tone} due to realization of H \isi{tone} syllables as LH following voiced onsets. This occurs in languages like \ili{Ewe} and \ili{Gengbe} \citep{Ansre1961,Bole-Richard1983}. In the final stages of this process—seen in languages like \ili{Nguni} and \ili{Shona} \citep{Downing2009}, and illustrated in \figref{sec:lotven:2}c—the voicing distinction has been lost in favor of a tonal distinction.

\begin{figure}
%\begin{tabularx}{\textwidth}{X}
%\lsptoprule

\begin{enumerate} 
\item[a.] \parbox[t]{2cm}{tá vs. dá}\parbox[t]{9cm}{[+voice] manifests phonetically as a redundant \isi{f0} cue on the left edge of the vowel}\\
\item[b.] \parbox[t]{2cm}{tá vs. dǎ}\parbox[t]{9cm}{Voiced obstruents phonologically trigger Rising rather than level High tone}\\
\item[c.] \parbox[t]{2cm}{tá vs. tǎ}\parbox[t]{9cm}{Voicing contrast is lost, contrasting lexical \isi{tone} remains}\\
\end{enumerate}
%\lspbottomrule
%\end{tabularx}

\caption{\label{fig:lotven:2} Illustration of tonal bifurcation (adapted from \citealt{Hyman2013enlarging}).}
\end{figure}


\ili{Gengbe} exhibits the pattern illustrated in \figref{sec:lotven:2}b: it retains a voicing contrast that results in the realization of underlying H \isi{tone} with a rising \isi{pitch pattern} in some contexts. We are not the first to note such an interaction in the \ili{Gbe} languages. \citet{Westermann1928} illustrated the presence of a non-lexical distinction between Low and Mid \isi{tone} in \ili{Ewe}, and—as noted previously—\posscitet{Ansre1961} study of \ili{Ewe} \isi{tone} concludes that the language has two tonemes, High and Non-High, with the latter realized as Low after a \isi{voiced obstruent} and Mid after a voiceless obstruent. \citet{Smith1968} and \citet{Stahlke1971} both take on the task of formalizing this interaction, focusing on the various morphophonological processes that interact with the realization of Low and Mid \isi{tone} as well as some differences found across \ili{Ewe} dialects. But Stahlke rejects the analysis of Mid \isi{tone} as non-underlying, arguing instead for instances of predictable, lexically specified, and floating Mid tones in \ili{Ewe}. \citet{Bole-Richard1983} notes that the link between rising pitch patterns and voiced obstruents also holds for \ili{Gengbe}.

\ili{Ewe} is known for its typologically irregular treatment of sonorants as depressor consonants in some phonological and morphophonological contexts, earning it a slot in \posscitet{Bradshaw1999} study of depressor effects under the section detailing “problem cases”. Bradshaw analyzes the depressor effect as an interaction between L \isi{tone} and the privative feature [L/voice]. For Bradshaw, this feature is generally a property of voiced obstruents and is underspecified for sonorants, but she suggests that languages like Ewe reveal the [L/voice] feature specification in sonorants may vary from language to language \citep[169-170]{Bradshaw1999}. As discussed in \sectref{sec:lotven:4}, \ili{Gengbe} does exhibit interactions between sonorants and a rising \isi{pitch pattern}, like \ili{Ewe}. Although the details of the pattern differ from \ili{Ewe}, under Bradshaw's analysis this would suggest that sonorants in \ili{Gengbe}, as in \ili{Ewe}, are specified for [L/voice]. Note that another element of Bradshaw’s analysis is that the feature [L/voice] can render an onset transparent to L \isi{tone} spreading and may also serve as the source for the L \isi{tone} that spreads onto neighboring vowels. This is not investigated here but, given the other similarities between \ili{Ewe} and \ili{Gengbe}, it should prove valuable to investigate in the future.

In this paper we use the term “\isi{depressor consonant}” as a cover term for the onsets that participate in the various pitch-lowering processes found in \ili{Gengbe}—that is, for both voiced obstruents and for sonorants, in instances where they cause either \isi{register} or contour effects. Some of the effects outlined may be indicative of a phonologization process—in particular, those contexts where an underlying /H/ \isi{tone} mandatorily surfaces as [LH]. These should prove valuable in future discussion of how phonetic effects may become phonologized, but the goal here is to present a clear overview of the depressor effects found in nominal and verbal domains. We turn now to a review of methods and aims.

\section{Methods and aims}\label{sec:lotven:3}

The present study surveys phonetic and phonological aspects of \ili{Gengbe} depressor consonants with a \isi{focus} on probing the differences between nouns and verbs in realizing such effects. The data here are from a single native speaker of \ili{Gengbe} who is in his fifties and is from Batonou, Togo. They were gathered in elicitation sessions conducted weekly from August 2014 to June 2016. 

Those items which were subjected to acoustic analysis were recorded in randomized order. All items were embedded in carrier sentences: nouns appeared in the frame \textit{Kòfí bé \underline{ }\underline{ }\underline{ }\underline{ } kèà} ‘Kofi said \underline{ }\underline{ }\underline{ }\underline{ } again,’ and verbs appeared in the frame  \textit{ṹsùà \underline{ }\underline{ }\underline{ }\underline{ } vɔ̀} ‘The man \underline{ }\underline{ }\underline{ }\underline{ }\underline{ }\underline{ }ed.’ Recordings were made in a sound-attenuated booth (WhisperRoom Model \#6084), annotated using Praat \citep{Boersma2016}, and measured using Prosody Pro \citep{xu2013prosodypro}, a script which automates the taking of acoustic measurements. A random sub-sample of the data was hand-checked to ensure validity. 

Measures reported here are for time-normalized \isi{f0}: each vowel was divided into ten equal portions and a mean \isi{f0} measure was calculated for each portion. This method facilitates cross-token comparison. Note that \isi{vowel length} in \ili{Gengbe} is not lexically contrastive, although we will see that allophonic lengthening does occur in some environments. That said, in the data that follow we have indicated Rising \isi{tone} as a series of L and H on identical adjacent vowels rather than on a single vowel (i.e. \textit{àá} rather than \textit{ǎ}). This is a stylistic choice. It does not indicate a phonemically long vowel.

\section{F0 contour effects}\label{sec:lotven:4}

As noted, an underlying H \isi{tone} in \ili{Gengbe} is mandatorily realized as LH in some contexts. In \sectref{sec:lotven:4.1} and \sectref{sec:lotven:4.2} below, we present an overview of the onset types and environments that produce LH \isi{tone} in nouns and verbs respectively. The most notable difference between the two is that sonorants do not act as depressor consonants in nouns, but do so in verbs. 

\subsection{LH tone in nouns} \label{sec:lotven:4.1}

Most nouns in \ili{Gengbe} are monosyllabic, with a lexically determined L \isi{tone} nominal prefix \textit{è-} or \textit{à-}. In this environment, when H \isi{tone} is preceded by a syllable with L \isi{tone}, the surface realization of H is determined by the consonant that precedes it. Depressor consonants in this environment are followed by LH, while other consonants are followed by H. That H and LH are both realizations of the H toneme is evinced by the numerous tonal minimal pairs included in \tabref{tab:lotven:1}. In these minimal pairs, L contrasts with H following voiceless obstruents (\tabref{tab:lotven:1} a–c) and sonorants (\tabref{tab:lotven:1} d–f). Following voiced obstruents, however—as in (\tabref{tab:lotven:1} g–i)—L contrasts with LH. Recall that there is no phonemic \isi{vowel length} contrast in these minimal pairs: what distinguishes them is the \isi{tone} of the final syllable.

%\textbf{\tabref{tab:lotven:1}: Tonal minimal pairs}
\begin{table}
\begin{tabularx}{\textwidth}{XlXlXlX}
\lsptoprule
& {L}  & {Gloss} & {H} & {Gloss} & {LH}  & {Gloss}\\
\midrule
a) & èk͡pɛ̃̀ & ‘whistle’   & èk͡pɛ̃́ & ‘cough’ &  & \\
b) & èkɔ̀  & ‘neck’      & èkɔ́  & ‘sand’  &  & \\
c) & ɑ̀tɔ̃̀  & ‘nest’      & ɑ̀tɔ̃́  & ‘apple’ &  & \\
d) & èɲĩ̀  & ‘cow’       & èɲĩ́  & ‘bee’   &  & \\
e) & èmɔ̃̀  & ‘corn mill’ & èmɔ̃́  & ‘way’   &  & \\
f) & ɑ̀l̃ɛ̃̀  & ‘stupidity’ & ɑl̃ɛ̃́  & ‘sheep’ &  & \\
g) & ègɑ̀  & ‘metal’     &  &  & ègɑ̀ɑ́ & ‘chief’\\
h) & èdɔ̀  & ‘sickness’  &  &  & èdɔ̀ɔ́ & ‘work’\\
i) & ɑ̀dɔ̃̀  & ‘squirrel’  &  &  & ɑ̀dɔ̃̀ɔ̃́ & ‘beak’\\
\lspbottomrule
\end{tabularx}
\caption{Tonal minimal pairs}
\label{tab:lotven:1}
\end{table}

Since H and LH \isi{tone} are in \isi{complementary distribution} here, we consider H and LH \isi{tone} allotones of the same H toneme. The data in \tabref{tab:lotven:1} conform to the process \citet{Bradshaw1999} describes as L \isi{tone} spreading from the initial L \isi{tone} realized on the nominal prefix over the \isi{voiced obstruent} (with [L/voice]) and onto the following vowel, a process that does not occur in nouns with voiceless obstruent and sonorant onsets lacking this feature. 

There are several things to note about the phonetic realization of H and LH as illustrated by the above data. First, LH \isi{tone} is associated with vowel lengthening. Average duration of vowels after voiced and voiceless obstruents—calculated over 66 items in each category, for a total of 264 token—is shown in \figref{sec:lotven:3}. Duration is of course affected by factors such as speaker and speaking rate, but in these data vowels with LH \isi{tone} (shown in the bottom bar on the chart) are longer than other vowels by approximately 60ms.


\begin{figure} 
\todo[inline]{use pgf barplot}
\includegraphics[width=\textwidth]{figures/Lotven-img3.png}
\caption{\label{fig:lotven:3} Average duration of vowels after voiced (black) and voiceless (gray) obstruents in nouns. Underlying H is longer after voiced than after voiceless obstruents. “T” represents a voiceless obstruent and “D” represents a voiced obstruent.}
\end{figure}


We can also look at the time-normalized \isi{f0} tracks of these vowels, shown in \figref{sec:lotven:4}. Here, gray lines represent vowels after voiceless obstruents (referred to as “T” in the key) and black lines represent vowels after voiced obstruents (referred to as “D” in the key). Solid lines represent L \isi{tone}, and dotted lines represent H (after voiceless obstruents) and LH (after voiced obstruents). Note the presence of what looks like an \isi{f0} \isi{register} effect in Low \isi{tone}: L after voiced obstruents is approximately 20Hz lower than after voiceless obstruents, and this difference perseverates across the entirety of the subsequent vowel. The dashed lines, meanwhile, illustrate the robust \isi{f0} differences between H and LH, which is realized as a contour effect. 

Means and standard deviations are included in \tabref{tab:lotven:2}; commentary follows. The L \isi{tone} difference illustrated in \figref{sec:lotven:4} conforms to \posscitet{Ansre1961} analysis of \ili{Ewe}, which claims that there are two realizations of the Non-High toneme: Low after a \isi{voiced obstruent} and Mid after a voiceless obstruent. But is this a phonetic effect as discussed above for \ili{Japanese} \citep{Oglesbee2008} or is this a phonological effect as Ansre proposes for \ili{Ewe}? Unlike \ili{Ewe}, neither previous \ili{Gengbe} literature nor our elicitation has provided evidence for Mid tones that are not phonologically conditioned, nor have we discovered lexical Mid tones or floating morphological Mid tones in \ili{Gengbe}. The appearance of the phonetically lower Low after a \isi{voiced obstruent}, in other words, is regular and predictable in \ili{Gengbe}, whereas that is not always the case in \ili{Ewe}. For the time being, then, we analyze this \isi{register} lowering as a purely phonetic effect.


\begin{figure}
\todo[inline]{crop files to remove redundant information}
	\includegraphics[width=\textwidth]{figures/fig-lotven-4.png}
	\caption{\label{fig:lotven:4} Time-normalized f0 tracks of High (dotted lines) and Low (solid lines) tones after voiced (black lines) and voiceless (gray lines) obstruents in nouns. H follows voiceless obstruents and LH follows voiced obstruents.}
\end{figure}


\begin{table} 
\todo[inline]{redo as table}
	\includegraphics[width=\textwidth]{figures/tab-lotven-2.png}
	\caption{Means and standard deviations (in Hz) for the ten timepoints included in time-normalized f0 tracks shown in \figref{fig:lotven:4}.}
	\label{tab:lotven:2}
\end{table}


As a side note, \isi{f0} \isi{register} lowering is not limited to L \isi{tone} contexts. As discussed more thoroughly in \figref{sec:lotven:4.2}, there are contexts in \ili{Gengbe} where High \isi{tone} is realized as H rather than as LH after voiced obstruents. In these instances, shown in \figref{sec:lotven:5}, we again see what looks like \isi{register} lowering of \isi{f0}. High \isi{tone} is realized with higher \isi{f0} after voiceless obstruents than after voiced obstruents. Note here that the pitch range for H \isi{tone} in \figref{sec:lotven:5} is comparable to the pitch range for L \isi{tone} in \figref{sec:lotven:4} above, but this is most likely a result of final lowering—a topic to be investigated in future work. 

  
\begin{figure}
\todo[inline]{crop files to remove redundant information}
\includegraphics[width=\textwidth]{figures/Lotven-img6.png}
\caption{\label{fig:lotven:5} Time-normalized f0 tracks of level High tones after voiced (black) and voiceless (gray) obstruents in verbs. H is lower following voiced obstruents.}
\end{figure}


At this time we do not have a firm answer on whether \isi{register} \isi{f0} lowering is a phonetic or phonological effect in \ili{Gengbe}. The data may support an analysis in \ili{Gengbe} that parallels that adopted for the L toneme in \ili{Ewe}—that is, an analysis that posits two allotones for the L toneme (in \figref{sec:lotven:4}) and two allotones for the H toneme (in \figref{sec:lotven:5})—but this is a question we can not answer yet. 

For now, we leave the topic of \isi{f0} \isi{register} effects and turn back to the LH \isi{tone} in \ili{Gengbe}. The \isi{f0 contour} effect in nouns—which manifests as a Rising \isi{pitch pattern} in \tabref{tab:lotven:1} (g–i), is shown to have longer duration than H in \figref{sec:lotven:3}, and displays a >50 Hz \isi{f0} difference localized to the left edge of the vowel in \figref{sec:lotven:4}—is tied to the \isi{tone} of the preceding syllable, not just preceding L \isi{tone} nominal prefixes. In nominal compounds, for instance, word-medial nominal prefixes are deleted. If this means that the target H \isi{tone} syllable is preceded by a surface H, as in \REF{ex:lotven:1}, or a surface LH, as in \sectref{ex:lotven:2}, no \isi{f0 contour} effect occurs. Rather, underlying H surfaces as H even after voiced obstruents. 

\ea\label{ex:lotven:1}
    \gll ɑ̀ɲĩ́g͡bɑ̃́ + è\textbf{dɔ̀ɔ́} → ɑ̀ɲĩ́g͡bɑ̃́\textbf{dɔ́}\\
    ‘earth’ {} ‘work’ {} {‘earth work’}     \\
    \glt 
    \z

\ea\label{ex:lotven:2}
    \gll èg͡bèé + ɑ̀v\textbf{ũ̀ṹ} → èg͡bèé\textbf{vṹ}\\
    ‘bush’ {} ‘dog’ {} {‘bush dog’}\\
    \glt
    \z
          

This interaction is relevant, for it helps to define the phenomenon as morphophonological in the sense that depressor consonants are not the source of the L \isi{tone} (as is argued for in \citealt{Bradshaw1999} for some depressor effects). Rather, depressor consonants allow L \isi{tone} to spread over them from the preceding vowel. Using preceding L \isi{tone} nominal prefixes as an illustrative environment, we pre-sent the list of \ili{Gengbe} onsets that nouns treat as depressors in \tabref{tab:lotven:3}. This includes all voiced obstruents—stops in (a–d), affricates in (e), fricatives in (f–i), and the retroflex [ɖ] in (j).

\begin{table}
\begin{tabularx}{\textwidth}{lllXlllX} 
\lsptoprule
&  Onset &  Noun &  Gloss &  &  Onset &  Noun &  Gloss\\
\midrule 
a) & [b]  & ɑ̀bɔ̀ɔ́  & ‘arm’      & {g)} & [z]  & èzɑ̃̀ɑ̃́  & ‘night’\\
b) & [d]  & èdɔ̀ɔ́  & ‘work’     & {h)} & [β]  & èβɑ̃̀ɑ̃́  & ‘spear’\\
c) & [g]  & ègɑ̃̀ɑ̃́  & ‘bigness’  & {i)} & [ɦ]  & èɦɑ̀ɑ́  & ‘group’\\
d) & [g͡b] & èg͡bĩ̀ĩ́ & ‘buttocks’ & {j)} & [ɖ]  & èɖìí          & ‘dirt’\\
e) & [d͡ʒ] & èd͡ʒɑ̃̀ɑ̃́ & ‘bow’      & {k)} & [gl] & ɑ̀glòó         & ‘joy’\\
f) & [v]  & ɑ̀vɔ̀ɔ́  & ‘cloth’    & {l)} & [ɦj] & èɦjɛ̃̀ɛ̃́ & ‘poverty’\\
\lspbottomrule
\end{tabularx}
\caption{List of Gengbe depressor consonants in nouns}
\label{tab:lotven:3}
\end{table}

\tabref{tab:lotven:4}, meanwhile, presents the consonants that do not act as depressors in \ili{Gengbe} nouns. These include voiceless obstruents, as in (a–f), and sonorants, as in (g–l). By contrasting \tabref{tab:lotven:3} items (k–l) with \tabref{tab:lotven:4} items (m–o), we can also see that the second member of an onset cluster is disregarded when calculating depressor effects in nouns. In other words, it is C\textsubscript{1} in a C\textsubscript{1}C\textsubscript{2} onset cluster that determines how an underlying H is realized in \ili{Gengbe} nouns—clusters that begin with a depressor, as in \tabref{tab:lotven:3} items (k–l), pattern with other onset depressors. Clusters that do not, as in \tabref{tab:lotven:4} items (m–o), pattern with the other non-depressor onsets. Note that only liquids and glides may appear as C\textsubscript{2} in consonant clusters in \ili{Gengbe}. \posscitet{Bradshaw1999} analysis of sonorants as unspecified for [L/voice] may prove useful here. While it is beyond the \isi{scope} of the present work, investigation of this possibility will prove valuable in future work.

\begin{table}
\begin{tabularx}{.66\textwidth}{lllX}
\lsptoprule
&  Onset &  Noun &  Gloss\\
\midrule
a) & [t]     & ɑ̀tí  &  {‘tree’}\\
b) & [k]     & èkú  &  {‘death’}\\
c) & [k͡p]    & èk͡pɑ́ &  {‘fence’} \\
d) & [ɸ]/[p] & ɑ̀ɸɑ́/ɑ̀pɑ́ &  {‘shout’}\\
e) & [f]     & ɑ̀fí  &  {‘here’}\\
f) & [s]     & èsɔ̃́  &  {‘horse’}\\
g) & [m]     & èmṹ  &  {‘mosquito’}\\
h) & [n]     & ɑ̀nɑ̃́  &  {‘bridge’}\\
i) & [ɲ]     & èɲĩ́  &  {‘bee’}\\
j) & [l]     & èló  &  {‘crocodile’}\\
k) & [w]     & èwɔ́  &  {‘corn flour’}\\
l) & [j]     & ɑ̀jɑ́  &  {‘air’}\\
m) & [kl]    & ɑ̀kló &  {‘flat boat’}\\
n) & [fj]    & èfjɔ́ &  {‘monkey}\\
o) & [wl]    & èwlí &  {‘shout’}\\
\lspbottomrule
\end{tabularx}
\caption{Non-depressor consonants in Gengbe nouns}
\label{tab:lotven:4}
\end{table}

While the pattern seen in verbs—shown next, in \sectref{sec:lotven:4.2}—differs, depressor consonants in nouns are limited to voiced obstruents. Sonorants do not act as depressors in nouns, and they are disregarded in C\textsubscript{1}C\textsubscript{2} clusters, indicating that it is the featural specification of C\textsubscript{1} that is relevant for this process. Furthermore, LH \isi{tone} in nouns is triggered by L \isi{tone} in a preceding syllable. When preceded by H or LH \isi{tone}, as in nominal compounds, underlying H surfaces as H even after voiced obstruents, so this phenomena requires an external L \isi{tone} to trigger spreading. This contrasts with the verbal pattern, which is outlined in the next section, where we will see that the occurrence of depressor effects is based on syntactic position.

\subsection{LH tone in verbs}\label{sec:lotven:4.2}

Verbs differ phonologically from nouns in several ways. First, more onset types (sonorants and voiceless obstruent-\isi{liquid} sequences) act as depressors in verbs. In addition, LH \isi{tone} surfaces not after a preceding L \isi{tone} vowel, but after a preceding phrase boundary. We begin by presenting data motivating the claim that verbs are sensitive to initial phrase-boundaries, then use the structural positions in which LH \isi{tone} manifests to illustrate the onset types that act as depressors in the verbal domain. Data are drawn from three contrasting syntactic situations: predication vs. citation, plural imperative vs. singular imperative, and reduplication with vs. without a pre-posed logical object.

In predication, as shown in \REF{ex:lotven:3}, even when the preceding vowel has L \isi{tone} and even following voiced obstruents, as in (\ref{ex:lotven:3}b), the H \isi{tone} verb is not realized as LH. In citation forms, however—shown in the examples in \REF{ex:lotven:4} — there is no overt \isi{subject} present. Here we see that what surfaced as H in the examples in \REF{ex:lotven:3} is still realized as H after voiceless obstruents (\ref{ex:lotven:4}a) but as LH after voiced obstruents (\ref{ex:lotven:4}b) and sonorants (\ref{ex:lotven:4}c).

\ea\label{ex:lotven:3} Predication (overt \isi{subject})
\ea\label{ex:lotven:3a} \gll mũ̀ k͡pɔ́ ǹtísì\\
1\textsc{sg} see lime \\
\glt ‘I saw a lime.’ 
\ex\label{ex:lotven:3b}	\gll mũ̀ bú ǹtísì\\
1\textsc{sg} lose lime\\
\glt ‘I lost a lime.’
\ex\label{ex:lotven:3c} \gll mũ̀ ɲɑ̃́ gɔ̃̀mɛ̃̀d͡ʒèd͡ʒèé-ɑ́\\
1\textsc{sg} know beginning-\textsc{def} \\
\glt ‘I know the beginning.’
\z
\z


\ea\label{ex:lotven:4} Citation (no overt \isi{subject})
\ea\label{ex:lotven:4a}
    \gll k͡pɔ́\\
    see    \\
    \glt ‘to see’
\ex\label{ex:lotven:4b}
	\gll bùú\\
    lose\\
    \glt ‘to lose’
\ex\label{ex:lotven:4c}
	\gll ɲɑ̃̀ɑ̃́\\
    know\\
    \glt ‘to know’
\z
\z
                           
The examples in \REF{ex:lotven:4} illustrate that the verbal domain differs from the nominal domain in both the context and onset types that are required for the realization of LH. As we saw in \sectref{sec:lotven:4.1}, the context that produces LH \isi{tone} in nouns is morphophonological, in the sense that it results when an underlying H surfaces after a \isi{depressor consonant} preceded by a Low-toned syllable. The context that gives us LH in verbs is syntactic, however. In addition, both the voicing and obstruency of an onset is relevant in the nominal domain where only voiced obstruents act as depressors. In the verbal domain, however, it appears that only voicing matters: here, as shown in (\ref{ex:lotven:4}b) and (\ref{ex:lotven:4}c), both voiced obstruents and sonorants act as depressors.

The same observations made in (\ref{ex:lotven:3}–\ref{ex:lotven:4}) hold for overt and non-overt subjects in imperatives. Plural imperatives, which require the overt L \isi{tone} \isi{subject} \textit{mĩ̀}, as in \REF{ex:lotven:5}, exhibit no depressor effect. Singular imperatives, on the other hand, lack overt subjects, as in \REF{ex:lotven:6}, and we see the same depressor effect shown in \REF{ex:lotven:4} in citation form. 
%  \langinfo{lg}{fam}{src}\\
\ea\label{ex:lotven:5} Plural imperative (overt \isi{subject})
\ea\label{ex:lotven:5a}
    \gll mĩ̀ \textbf{tú} èɦɔ̃̀tɾú\\
    2\textsc{pl}  close door    \\
    \glt ‘Close the door, you all!’
\ex\label{ex:lotven:5b}
	\gll mĩ̀ \textbf{vɑ́} \\
    2\textsc{pl} come\\
    \glt ‘Come, you all!’
\ex\label{ex:lotven:5c}
	\gll mĩ̀ \textbf{lé} ṹsù-ɑ̀ \\
    2\textsc{pl}  arrest man-\textsc{def}\\
    \glt ‘You all arrest the man!
\z
\z
 
\ea\label{ex:lotven:6}Singular imperative (no overt \isi{subject})
\ea\label{ex:lotven:6a}
    \gll {tú} èɦɔ̃̀tɾú\\
    close door    \\
    \glt ‘Close the door!’
\ex\label{ex:lotven:6b}
	\gll {vɑ̀ɑ́}\\
	come\\
    \glt ‘Come!’
\ex\label{ex:lotven:6c}
	\gll {lèé} ṹsù-ɑ̀\\
    arrest man-\textsc{def}\\
    \glt ‘Arrest the man!’
\z
\z

\citet{Bradshaw1999} analyzes the singular imperative in \ili{Ewe} as formed by a prefixed L \isi{tone} morpheme that docks with the vowel only when the onset is voiced. Since our data suggests the trigger of LH is present in citation form as well as the singular imperative, we describe the phenomenon in terms of an initial syntactic boundary (possibly an initial L boundary \isi{tone}) rather than a morphological affix. It is possible still that the L \isi{tone} is a morphological affix, although with the two situations described (as well as reduplication data below), we posit a single positional explanation rather than three independent L \isi{tone} morphemes. 

When a verb is reduplicated, the logical object, normally following the bare verb, is moved to precede the reduplicated verb. Where there is such pre-verbal information, there is no depressor effect, as in \REF{ex:lotven:7}, and where there is no preverbal material, we again see the depressor effect, as in \REF{ex:lotven:8}. Note that sonorants are still considered depressors here despite the fact that in (\ref{ex:lotven:7}-\ref{ex:lotven:8}) we are deriving nouns from verbal roots. If we are to assume that category-changing derivation processes are done in the lexicon, this introduces an as-yet unsolved mystery as to the nature of the relevant property that determines which set of onsets counts as depressors. For now we can tentatively define the distinction as derivation from underlying nominal or verbal roots. 
\\
\ea\label{ex:lotven:7}Reduplication (pre-posed object)
\ea\label{ex:lotven:7a}
    \gll èlɑ̃̀ {fɑ́}{\textasciitilde}fɑ́\\
    meat  cool{\textasciitilde}\textsc{nom} \\
    \glt ‘cooling meat’
\ex\label{ex:lotven:7b}
	\gll èlɑ̃̀ {vó}{\textasciitilde}vó\\
    meat decay{\textasciitilde}\textsc{nom}\\
    \glt ‘decaying meat’
\ex\label{ex:lotven:7c}
	\gll ɲɔ̃́nũ̀ {jɔ́}{\textasciitilde}jɔ́\\
	woman call{\textasciitilde}\textsc{nom}\\
    \glt ‘calling a woman’
\z
\z

\ea\label{ex:lotven:8}Reduplication (no pre-posed object)
\ea\label{ex:lotven:8a}    
    \gll {fɑ́}{\textasciitilde}fɑ́\\
    cool{\textasciitilde}\textsc{nom}    \\
    \glt ‘cooling’
\ex\label{ex:lotven:8b}
	\gll {vòó}{\textasciitilde}vó\\
    decay{\textasciitilde}\textsc{nom}\\
    \glt ‘decaying’
\ex\label{ex:lotven:8c}
	\gll {jɔ̀ɔ́}{\textasciitilde}jɔ́\\
    call{\textasciitilde}\textsc{nom} \\
    \glt ‘calling’
\z
\z


We analyze this process in terms of syntax rather than morphology or phonology for the following reasons. As noted before, we do not posit tonal morphology that affects these three processes independently, although we leave open the possibility. We also do not see a clear path to a phonological explanation in terms of (prosodic) word-initial position. If we were to explore this possibility, we would need to describe the verbs in \REF{ex:lotven:3}, \REF{ex:lotven:5}, and \REF{ex:lotven:7} as non-initial. Pronouns in \REF{ex:lotven:3} and \REF{ex:lotven:5} can — and have in the case of \ili{Ewe} \citep{Duthie1996} — been analyzed as clitics, however, full NP subjects also fail to trigger LH \isi{tone} in following predicates, for example \textit{ènɔ\`{} ã\`{}  bé} ‘mother said,’ suggesting that the right environment for LH \isi{tone} in verbs has to do with phrase position (possibly utterance-initial position) rather than word position. As of yet, we leave the term ‘phrase-initial position’ purposefully vague. The importance of syntactic position in \isi{tone} rules is well established \citep{Snider2014}, but we leave the definition of such positioning to future syntactic work.

The data in (\ref{ex:lotven:3}–\ref{ex:lotven:8}) indicate that depressor consonants in the verbal domain include voiced obstruents and sonorants and that the phrase-initial position, rather than a preceding L \isi{tone} vowel, is the trigger for LH \isi{tone}. The data in (\ref{ex:lotven:9}-\ref{ex:lotven:10}) present verbs with initial consonant clusters, using the citation form as illustration, though reduplication and imperative data were also investigated. \REF{ex:lotven:9} reveals that consonant-\isi{liquid} clusters act as depressors, regardless of the identity of C\textsubscript{1}; \REF{ex:lotven:10} reveals that consonant-glide clusters do not. It is again valuable to note that liquids and glides are the only consonants that can serve as C\textsubscript{2} in a consonant cluster in \ili{Gengbe}. The crucial data points here are (\ref{ex:lotven:9}a) where a voiceless onset-\isi{liquid} cluster shows a depressor effect and (\ref{ex:lotven:10}a) where a voiceless onset-glide cluster does not. We resist the urge here to speculate about \isi{syllable structure} based on these verbal data since the difference between C\textsubscript{2} liquids and glides in the verbal domain does not hold in the nominal domain, as illustrated previously in \tabref{tab:lotven:4} (m–n).

\ea\label{ex:lotven:9}Consonant-\isi{liquid} clusters in \ili{Gengbe} Verbs
\ea\label{ex:lotven:9a}
    klòó\\
    \glt ‘to fade’
\ex\label{ex:lotven:9b}
	ŋlɔ̃̀ɔ̃́\\
    \glt ‘to fold’
\ex\label{ex:lotven:9c}
	glòó\\
    \glt ‘to boast’
\z
\z

\ea\label{ex:lotven:10}Consonant-glide clusters in \ili{Gengbe} Verbs
\ea\label{ex:lotven:10a}
    \textbf{fjɔ́}\\
    \glt ‘to teach’
\ex\label{ex:lotven:10b}
	ljàá\\
    \glt ‘to climb\\
\ex\label{ex:lotven:10c}
	ɦjɛ̃̀ɛ̃́\\
    \glt ‘to need’
\z
\z
        

Taking all of these data together, then, verbs differ from nouns (or more specifically verbal roots differ from nominal roots) in that single onset sonorants act as depressors for the former, but not the latter. Furthermore, C\textsubscript{1} determines whether or not a depressor effect emerges in nouns and consonant-\isi{liquid} sequences fail to act as depressors if C\textsubscript{1} is not a \isi{voiced obstruent}. Yet in verbs, regardless of the identity of C\textsubscript{1}, consonant-\isi{liquid} but not consonant-glide clusters act as depressors. Finally, LH \isi{tone} in verbs is triggered in phrase-initial position rather than by preceding L \isi{tone} vowels. A breakdown of the onset types that pattern as depressors in the nominal and verbal domains is given in 
\figref{fig:lotven:6}, where a shaded box indicates that a depressor effect obtains in that environment: in other words, depressor effects occur after voiced obstruents—still represented with a capital D—in both nominal and verbal domains. Note that a noun consisting of a H \isi{tone} syllable with Nasal-Glide (NG) onset has yet to be elicited and is marked “n/a.”

\begin{table}
\begin{tabularx}{\textwidth}{Xlllllllll}
\lsptoprule
& {T} & {TL} & {TG} & {D} & {DL} & {DG} & {N} & {NL} & {NG}\\
\midrule
Nouns &  &  &  & $\bullet$ & $\bullet$ & $\bullet$ &  &  & n/a\\
Verbs &  & $\bullet$ &  & $\bullet$ & $\bullet$ & $\bullet$ & $\bullet$ & $\bullet$ & $\bullet$ \\
\lspbottomrule
\end{tabularx}
\caption{Summary of onsets considered depressors by nouns and verbs (T=Voiceless Obstruent, D=Voiced Obstruent, N=Sonorant, L=Liquid, G=Glide)}
\label{fig:lotven:6}
\end{table}

In this survey, we have presented a brief overview of the phonetic, phonological, morphological, and syntactic contexts in which we can, on the surface, observe \isi{f0} \isi{register} and \isi{f0 contour} lowering. This overview is preliminary, and is intended to inform future investigation.

\section{Summary}\label{sec:lotven:5}

Our preliminary research on \ili{Gengbe} has highlighted relevant observable phenomena as well as mysteries in need of further investigation. We have shown two types of observable effect. An \isi{f0 contour} effect occurs when an underlying H follows specific depressor onsets (a category which differs based on whether a root is nominal or verbal) and is realized with a Rising \isi{pitch pattern}. It is realized phonetically through both lengthening (by about 60ms) and \isi{f0} lowering (by about 50Hz at the left edge of the vowel). There is also an \isi{f0} \isi{register} effect, wherein an underlying L is realized as lower following a \isi{voiced obstruent} (by about 20 Hz across the duration of the vowel). We have shown that \isi{register} \isi{f0} lowering is also present in some verbal contexts when an underlying H surfaces as H (rather than LH) after a \isi{voiced obstruent}. 

In the contexts investigated in this study, we find that nominal and verbal roots differ both in the onset types that are followed by LH and in the contexts that trigger this Rising \isi{pitch pattern}. Nominal roots in \ili{Gengbe} treat voiced obstruents in C\textsubscript{1} position as depressors, revealed as such when preceded morphologically by a L \isi{tone} syllable. C\textsubscript{2} consonants do not alter this effect in nouns. Verbal roots, on the other hand, treat both voiced obstruents and sonorants as depressors, revealed as such when placed in phrase-initial position. Unlike nouns, C\textsubscript{2} liquids—but not glides—are also followed by LH in these verbal contexts. 

Although this study is preliminary and there is much work to be done on \ili{Gengbe}, it is our expectation that further investigation of the behavior and identity of depressor consonants in the many \ili{Gbe} languages will provide a rich ground for the study of tonal bifurcation and the phonologization of \isi{tone}.

%%move bib entries to localbibliography.bib

%% Edwin: bib entries have been moved to localbibliography.bib

%Chistovich, L. A. 1969. Variations of the fundamental voice pitch as a discriminatory cue for consonants. Soviet Physics-Acoustics l4. 372-378.

%Perkins, Jeremy. 2011. Consonant-\isi{tone} interaction and laryngealization in \ili{Thai}. In presentation, 21st Annual Conference of the Southeast Asian Linguistics Society (SEALS 21). 11-13.

%Wright, Richard. 2004. A review of perceptual cues and cue robustness. Phonetically based phonology, 34-57.

%Xu, Yi. 2013. ProsodyPro — A Tool for Large-scale Systematic Prosody Analysis. In Proceedings of Tools and Resources for the Analysis of Speech Prosody (TRASP 2013). Aix-en-Provence, France. 7-10.

\section*{Abbreviations}
\begin{tabularx}{.55\textwidth}{ll}
1\textsc{sg} & 1st Person Singular Subject \\
2\textsc{pl} & 2nd Person Plural Subject\\
\textsc{def} & Definite Determiner\\
\textsc{nom} & Nominalizer \\
\end{tabularx}

\sloppy
\printbibliography[heading=subbibliography,notkeyword=this]

\end{document}