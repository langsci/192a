\documentclass[output=paper]{langsci/langscibook}  
\author{Maria Khachaturyan\affiliation{University of Helsinki}}
\title{The Aorist and the Perfect in Mano} 
\abstract{The foci of this paper are the semantic differences between two perfective constructions in the Mano language, the Aorist and the Perfect. The paper is based on Östen Dahl's classic questionnaire, as well as various sources of natural speech data, including narratives, routine conversations, and ritual speech, Christian and traditional. The core semantic property of the Mano Perfect is event relevance, which is confirmed by the annulled result test. The core function of the Aorist is being the narrative tense. The paper also includes discussion of two secondary functions of the Perfect and the Aorist, namely, anticipation of future events and transposition to the past. The secondary functions confirm the basic distinction between the Aorist and the Perfect, the latter maintaining a closer connection with the reference point.}

\IfFileExists{../localcommands.tex}{%hack to check whether this is being compiled as part of a collection or standalone
  \usepackage{pifont}
\usepackage{savesym}

\savesymbol{downingtriple}
\savesymbol{downingdouble}
\savesymbol{downingquad}
\savesymbol{downingquint}
\savesymbol{suph}
\savesymbol{supj}
\savesymbol{supw}
\savesymbol{sups}
\savesymbol{ts}
\savesymbol{tS}
\savesymbol{devi}
\savesymbol{devu}
\savesymbol{devy}
\savesymbol{deva}
\savesymbol{N}
\savesymbol{Z}
\savesymbol{circled}
\savesymbol{sem}
\savesymbol{row}
\savesymbol{tipa}
\savesymbol{tableauxcounter}
\savesymbol{tabhead}
\savesymbol{inp}
\savesymbol{inpno}
\savesymbol{g}
\savesymbol{hanl}
\savesymbol{hanr}
\savesymbol{kuku}
\savesymbol{ip}
\savesymbol{lipm}
\savesymbol{ripm}
\savesymbol{lipn}
\savesymbol{ripn} 
% \usepackage{amsmath} 
% \usepackage{multicol}
\usepackage{qtree} 
\usepackage{tikz-qtree,tikz-qtree-compat}
% \usepackage{tikz}
\usepackage{upgreek}


%%%%%%%%%%%%%%%%%%%%%%%%%%%%%%%%%%%%%%%%%%%%%%%%%%%%
%%%                                              %%%
%%%           Examples                           %%%
%%%                                              %%%
%%%%%%%%%%%%%%%%%%%%%%%%%%%%%%%%%%%%%%%%%%%%%%%%%%%%
% remove the percentage signs in the following lines
% if your book makes use of linguistic examples
\usepackage{tipa}  
\usepackage{pstricks,pst-xkey,pst-asr}

%for sande et al
\usepackage{pst-jtree}
\usepackage{pst-node}
%\usepackage{savesym}


% \usepackage{subcaption}
\usepackage{multirow}  
\usepackage{./langsci/styles/langsci-optional} 
\usepackage{./langsci/styles/langsci-lgr} 
\usepackage{./langsci/styles/langsci-glyphs} 
\usepackage[normalem]{ulem}
%% if you want the source line of examples to be in italics, uncomment the following line
% \def\exfont{\it}
\usetikzlibrary{arrows.meta,topaths,trees}
\usepackage[linguistics]{forest}
\forestset{
	fairly nice empty nodes/.style={
		delay={where content={}{shape=coordinate,for parent={
					for children={anchor=north}}}{}}
}}
\usepackage{soul}
\usepackage{arydshln}
% \usepackage{subfloat}
\usepackage{langsci/styles/langsci-gb4e} 
   
% \usepackage{linguex}
\usepackage{vowel}

\usepackage{pifont}% http://ctan.org/pkg/pifont
\newcommand{\cmark}{\ding{51}}%
\newcommand{\xmark}{\ding{55}}%
 
 
 %Lamont
 \makeatletter
\g@addto@macro\@floatboxreset\centering
\makeatother

\usepackage{newfloat} 
\DeclareFloatingEnvironment[fileext=tbx,name=Tableau]{tableau}
  %add all your local new commands to this file
\newcommand{\downingquad}[4]{\parbox{2.5cm}{#1}\parbox{3.5cm}{#2}\parbox{2.5cm}{#3}\parbox{3.5cm}{#4}}
\newcommand{\downingtriple}[3]{\parbox{4.5cm}{#1}\parbox{3cm}{#2}\parbox{3cm}{#3}}
\newcommand{\downingdouble}[2]{\parbox{4.5cm}{#1}\parbox{6cm}{#2}}
\newcommand{\downingquint}[5]{\parbox{1.75cm}{#1}\parbox{2.25cm}{#2}\parbox{2cm}{#3}\parbox{3cm}{#4}\parbox{2cm}{#5}}
\newcolumntype{Y}{>{\centering\arraybackslash}X}
\newcolumntype{T}{>{\centering\arraybackslash}m{2cm}}

%commands for Kusmer paper below
\newcommand{\ip}{$\upiota$}
\newcommand{\lipm}{(\_{\ip-Max}}
\newcommand{\ripm}{)\_{\ip-Max}}
\newcommand{\lipn}{(\_{\ip}}
\newcommand{\ripn}{)\_{\ip}}
\renewcommand{\_}[1]{\textsubscript{#1}}


%commands for Pillion paper below
\newcommand{\suph}{\textipa{\super h}}
\newcommand{\supj}{\textipa{\super j}}
\newcommand{\supw}{\textipa{\super w}}
\newcommand{\ts}{\textipa{\t{ts}}}
\newcommand{\tS}{\textipa{\t{tS}}}
\newcommand{\devi}{\textipa{\r*i}}
\newcommand{\devu}{\textipa{\r*u}}
\newcommand{\devy}{\textipa{\r*y}}
\newcommand{\deva}{\textipa{\r*a}}
\renewcommand{\N}{\textipa{N}}
\newcommand{\Z}{\textipa{Z}}
% 

%commands for Diercks paper below
\newcommand{\circled}[1]{\begin{tikzpicture}[baseline=(word.base)]
\node[draw, rounded corners, text height=8pt, text depth=2pt, inner sep=2pt, outer sep=0pt, use as bounding box] (word) {#1};
\end{tikzpicture}
}

%commands for Pesetsky paper below
% \newcommand{\sem}[2][]{\mbox{$[\![ $\textbf{#2}$ ]\!]^{#1}$}}
\newcommand{\sem}[2][]{\mbox{$[[ $\textbf{#2}$ ]]^{#1}$}}

% \newcommand{\ripn}{{\color{red}ripn}}%this is used but never defined. Please update the definition



%commands for Lamont paper below
\newcommand{\row}[4]{
	#1. & 
    /{#2}/ & 
    [{#3}] & 
    `#4' \\ 
}
%\newcounter{tableauxcounter}
\newcommand{\tabhead}[2]{
%     \captionsetup{labelformat=empty}
%     \stepcounter{tableauxcounter}
%     \addtocounter{table}{-1}
% 	\centering
% 	\caption{Tableau \thetableauxcounter: #1}
	\caption{#1}
	\label{#2}
}
\newcommand{\candref}[2]{{(\ref{#1}#2)}}
\newcommand{\tableauref}[1]{{Tableau~\ref{#1}}}
% tableaux
\newcommand{\inp}[1]{\multicolumn{2}{|l||}{{#1}}}
\newcommand{\inpno}[1]{\multicolumn{2}{|l||}{#1}}
\newcommand{\g}{\cellcolor{lightgray}}
\newcommand{\hanl}{\HandLeft}
\newcommand{\hanr}{\HandRight}
\newcommand{\kuku}{Kuk\'{u}}

% \newcommand{\nocaption}[1]{{\color{red} Please provide a caption}}

% \providecommand{\biberror}[1]{{\color{red}#1}}

\definecolor{RED}{cmyk}{0.05,1,0.8,0}


\newfontfamily\amharicfont[Script = Ethiopic, Scale = 1.0]{AbyssinicaSIL}
\newcommand{\amh}[1]{{\amharicfont #1}}

% 
% %Gjersoe
\usepackage{textgreek}
% 
\newcommand{\viol}{\fontfamily{MinionPro-OsF}\selectfont\rotatebox{60}{$\star$}}
\newcommand{\myscalex}{0.45}
\newcommand{\myscaley}{0.65}
%\newcommand{\red}[1]{\textcolor{red}{#1}}
%\newcommand{\blue}[1]{\textcolor{blue}{#1}}
\newcommand{\epen}[1]{\colorbox{jgray}{#1}}
\newcommand{\hand}{{\normalsize \ding{43}}}
\definecolor{jgray}{gray}{0.8} 
\usetikzlibrary{positioning}
\usetikzlibrary{matrix}
\newcommand{\mora}{\textmu\xspace}
\newcommand{\si}{\textsigma\xspace}
\newcommand{\ft}{\textPhi\xspace}
\newcommand{\tone}{\texttau\xspace}
\newcommand{\word}{\textomega\xspace}
% \newcommand{\ts}{\texttslig}
\newcommand{\fns}{\footnotesize}
\newcommand{\ns}{\normalsize}
\newcommand{\vs}{\vspace{1em}}
\newcommand{\bs}{\textbackslash}   % backslash
\newcommand{\cmd}[1]{{\bf \color{red}#1}}   % highlights command
\newcommand{\scell}[2][l]{\begin{tabular}[#1]{@{}c@{}}#2\end{tabular}}
% \interfootnotelinepenalty=10000

% --- Snider Representations --- %

\newcommand{\RepLevelHh}{
\begin{minipage}{0.10\textwidth}
\begin{tikzpicture}[xscale=\myscalex,yscale=\myscaley]
%\node (syl) at (0,0) {Hi};
\node (Rt) at (0,1) {o};
\node (H) at (-0.5,2) {H};
\node (R) at (0.5,3) {h};
%\draw [thick] (syl.north) -- (Rt.south) ;
\draw [thick] (Rt.north) -- (H.south) ;
\draw [thick] (Rt.north) -- (R.south) ;
\end{tikzpicture}
\end{minipage}
}

\newcommand{\RepLevelLh}{
\begin{minipage}{0.10\textwidth}
\begin{tikzpicture}[xscale=\myscalex,yscale=\myscaley]
%\node (syl) at (0,0) {Mid2};
\node (Rt) at (0,1) {o};
\node (H) at (-0.5,2) {L};
\node (R) at (0.5,3) {h};
%\draw [thick] (syl.north) -- (Rt.south) ;
\draw [thick] (Rt.north) -- (H.south) ;
\draw [thick] (Rt.north) -- (R.south) ;
\end{tikzpicture}
\end{minipage}
}

\newcommand{\RepLevelHl}{
\begin{minipage}{0.10\textwidth}
\begin{tikzpicture}[xscale=\myscalex,yscale=\myscaley]
%\node (syl) at (0,0) {Mid1};
\node (Rt) at (0,1) {o};
\node (H) at (-0.5,2) {H};
\node (R) at (0.5,3) {l};
%\draw [thick] (syl.north) -- (Rt.south) ;
\draw [thick] (Rt.north) -- (H.south) ;
\draw [thick] (Rt.north) -- (R.south) ;
\end{tikzpicture}
\end{minipage}
}

\newcommand{\RepLevelLl}{
\begin{minipage}{0.10\textwidth}
\begin{tikzpicture}[xscale=\myscalex,yscale=\myscaley]
%\node (syl) at (0,0) {Lo};
\node (Rt) at (0,1) {o};
\node (H) at (-0.5,2) {L};
\node (R) at (0.5,3) {l};
%\draw [thick] (syl.north) -- (Rt.south) ;
\draw [thick] (Rt.north) -- (H.south) ;
\draw [thick] (Rt.north) -- (R.south) ;
\end{tikzpicture}
\end{minipage}
}

% --- Representations --- %

\newcommand{\RepLevel}{
\begin{minipage}{0.10\textwidth}
\begin{tikzpicture}[xscale=\myscalex,yscale=\myscaley]
\node (syl) at (0,0) {\textsigma};
\node (Rt) at (0,1) {o};
\node (H) at (-0.5,2) {\texttau};
\node (R) at (0.5,3) {\textrho};
\draw [thick] (syl.north) -- (Rt.south) ;
\draw [thick] (Rt.north) -- (H.south) ;
\draw [thick] (Rt.north) -- (R.south) ;
\end{tikzpicture}
\end{minipage}
}

\newcommand{\RepContour}{
\begin{minipage}{0.10\textwidth}
\begin{tikzpicture}[xscale=\myscalex,yscale=\myscaley]
\node (syl) at (0,0) {\textsigma};
\node (Rt) at (0,1) {o};
\node (H) at (-0.5,2) {\texttau};
\node (R) at (0.5,3) {\textrho};
\node (Rt2) at (1.5,1.0) {o};
%\node (H2) at (1.0,2) {$\tau$};
%\node (R2) at (2.0,2.5) {R};
\draw [thick] (syl.north) -- (Rt.south) ;
\draw [thick] (Rt.north) -- (H.south) ;
\draw [thick] (Rt.north) -- (R.south) ;
\draw [thick] (syl.north) -- (Rt2.south) ;
%\draw [thick] (Rt2.north) -- (H2.south) ;
%\draw [thick] (Rt2.north) -- (R2.south) ;
\end{tikzpicture}
\end{minipage}
}


% --- OT constraints --- %

\newcommand{\IllustrationDown}{
\begin{minipage}{0.09\textwidth}
\begin{tikzpicture}[xscale=0.7,yscale=0.45]
\node (reg) at (0,0.75) {{\small \textalpha}};
\node (arrow) at (0,0) {{\fns $\downarrow$}};
\node (Rt) at (0,-0.75) {{\small \textbeta}};
\end{tikzpicture}
\end{minipage}
}

\newcommand{\IllustrationUp}{
\begin{minipage}{0.09\textwidth}
\begin{tikzpicture}[xscale=0.7,yscale=0.45]
\node (reg) at (0,0.75) {{\small \textalpha}};
\node (arrow) at (0,0) {{\fns $\uparrow$}};
\node (Rt) at (0,-0.75) {{\small \textbeta}};
\end{tikzpicture}
\end{minipage}
}

\newcommand{\MaxAB}{
\begin{minipage}{0.09\textwidth}
\begin{tikzpicture}[xscale=0.6,yscale=0.4]
\node (max) at (0,0) {{\small \textsc{Max}}};
\node (reg) at (0.75,0.5) {{\fns \textalpha}};
\node (arrow) at (0.75,0) {{\tiny $\downarrow$}};
\node (Rt) at (0.75,-0.5) {{\fns \textbeta}};
\end{tikzpicture}
\end{minipage}
}

\newcommand{\DepAB}{
\begin{minipage}{0.09\textwidth}
\begin{tikzpicture}[xscale=0.6,yscale=0.4]
\node (max) at (0,0) {{\small \textsc{Dep}}};
\node (reg) at (0.75,0.5) {{\fns \textalpha}};
\node (arrow) at (0.75,0) {{\tiny $\downarrow$}};
\node (Rt) at (0.75,-0.5) {{\fns \textbeta}};
\end{tikzpicture}
\end{minipage}
}

\newcommand{\DepHReg}{
\begin{minipage}{0.055\textwidth}
\begin{tikzpicture}[xscale=0.6,yscale=0.4]
\node (dep) at (0,0) {{\small \textsc{Dep}}};
\node (reg) at (0,-1.0) {{\small h}};
\end{tikzpicture}
\end{minipage}
}

\newcommand{\DepLReg}{
\begin{minipage}{0.055\textwidth}
\begin{tikzpicture}[xscale=0.6,yscale=0.4]
\node (dep) at (0,0) {{\small \textsc{Dep}}};
\node (reg) at (0,-1.0) {{\small l}};
\end{tikzpicture}
\end{minipage}
}

\newcommand{\DepReg}{
\begin{minipage}{0.055\textwidth}
\begin{tikzpicture}[xscale=0.6,yscale=0.4]
\node (dep) at (0,0) {{\small \textsc{Dep}}};
\node (reg) at (0,-1.0) {{\small \textrho}};
\end{tikzpicture}
\end{minipage}
}

\newcommand{\DepTRt}{
\begin{minipage}{0.1\textwidth}
\begin{tikzpicture}[xscale=0.6,yscale=0.4]
\node (dep) at (0,0) {{\small \textsc{Dep}}};
\node (t) at (0.75,0.5) {{\fns \texttau}};
\node (arrow) at (0.75,0) {{\tiny $\downarrow$}};
\node (Rt) at (0.75,-0.5) {{\fns o}};
\end{tikzpicture}
\end{minipage}
}

\newcommand{\MaxRegRt}{
\begin{minipage}{0.1\textwidth}
\begin{tikzpicture}[xscale=0.6,yscale=0.4]
\node (max) at (0,0) {{\small \textsc{Max}}};
\node (arrow) at (0.75,0) {{\tiny $\downarrow$}};
\node (Rt) at (0.75,-0.5) {{\fns o}};
\node (reg) at (0.75,0.5) {{\fns \textrho}};
\end{tikzpicture}
\end{minipage}
}

\newcommand{\RegToneByRt}{
\begin{minipage}{0.06\textwidth}
\begin{tikzpicture}[xscale=0.6,yscale=0.5]
\node[rotate=20] (arrow1) at (-0.15,0) {{\fns $\uparrow$}};
\node[rotate=340] (arrow2) at (0.15,0) {{\fns $\uparrow$}};
\node (Rt) at (0,-0.55) {{\small o}};
\node (reg) at (0.4,0.55) {{\small \textrho}};
\node (tone) at (-0.4,0.55) {{\small \texttau}};
\end{tikzpicture}
\end{minipage}
}

\newcommand{\RegToneBySyl}{
\begin{minipage}{0.06\textwidth}
\begin{tikzpicture}[xscale=0.6,yscale=0.5]
\node[rotate=20] (arrow1) at (-0.15,0) {{\fns $\uparrow$}};
\node[rotate=340] (arrow2) at (0.15,0) {{\fns $\uparrow$}};
\node (Rt) at (0,-0.55) {{\small \textsigma}};
\node (reg) at (0.4,0.55) {{\small \textrho}};
\node (tone) at (-0.4,0.55) {{\small \texttau}};
\end{tikzpicture}
\end{minipage}
}

\newcommand{\DepTone}{
\begin{minipage}{0.055\textwidth}
\begin{tikzpicture}[xscale=0.6,yscale=0.4]
\node (dep) at (0,0) {{\small \textsc{Dep}}};
\node (tone) at (0,-1.0) {{\small \texttau}};
\end{tikzpicture}
\end{minipage}
}

\newcommand{\DepTonalRt}{
\begin{minipage}{0.055\textwidth}
\begin{tikzpicture}[xscale=0.6,yscale=0.4]
\node (dep) at (0,0) {{\small \textsc{Dep}}};
\node (tone) at (0,-1.0) {{\small o}};
\end{tikzpicture}
\end{minipage}
}

\newcommand{\DepL}{
\begin{minipage}{0.055\textwidth}
\begin{tikzpicture}[xscale=0.6,yscale=0.4]
\node (dep) at (0,0) {{\small \textsc{Dep}}};
\node (tone) at (0,-1.0) {{\small L}};
\end{tikzpicture}
\end{minipage}
}

\newcommand{\DepH}{
\begin{minipage}{0.055\textwidth}
\begin{tikzpicture}[xscale=0.6,yscale=0.4]
\node (dep) at (0,0) {{\small \textsc{Dep}}};
\node (tone) at (0,-1.0) {{\small H}};
\end{tikzpicture}
\end{minipage}
}

\newcommand{\NoMultDiff}{{\small *loh}}
\newcommand{\Alt}{{\small \textsc{Alt}}}
\newcommand{\NoSkip}{{\small \scell{\textsc{No}\\\textsc{Skip}}}}


\newcommand{\RegDomRt}{
\begin{minipage}{0.030\textwidth}
\begin{tikzpicture}[xscale=0.6,yscale=0.5]
\node (arrow) at (0,0) {{\fns $\downarrow$}};
\node (Rt) at (0,-0.55) {{\small o}};
\node (reg) at (0,0.55) {{\small \textrho}};
\end{tikzpicture}
\end{minipage}
}

\newcommand{\DepRegRt}{
\begin{minipage}{0.1\textwidth}
\begin{tikzpicture}[xscale=0.6,yscale=0.4]
\node (dep) at (0,0) {{\small \textsc{Dep}}};
\node (arrow) at (0.75,0) {{\tiny $\downarrow$}};
\node (Rt) at (0.75,-0.5) {{\fns o}};
\node (reg) at (0.75,0.5) {{\fns \textrho}};
\end{tikzpicture}
\end{minipage}
}

% unused

\newcommand{\ToneByRt}{
\begin{minipage}{0.05\textwidth}
\begin{tikzpicture}[xscale=0.6,yscale=0.5]
\node (arrow) at (0,0) {{\fns $\uparrow$}};
\node (Rt) at (0,-0.55) {{\small o}};
\node (tone) at (0,0.55) {{\small \texttau}};
\end{tikzpicture}
\end{minipage}
}

\newcommand{\RegByRt}{
\begin{minipage}{0.05\textwidth}
\begin{tikzpicture}[xscale=0.6,yscale=0.5]
\node (arrow) at (0,0) {{\fns $\uparrow$}};
\node (Rt) at (0,-0.55) {{\small o}};
\node (reg) at (0,0.55) {{\small \textrho}};
\end{tikzpicture}
\end{minipage}
}

\newcommand{\ToneDomRt}{
\begin{minipage}{0.05\textwidth}
\begin{tikzpicture}[xscale=0.6,yscale=0.5]
\node (arrow) at (0,0) {{\fns $\downarrow$}};
\node (Rt) at (0,-0.55) {{\small o}};
\node (tone) at (0,0.55) {{\small \texttau}};
\end{tikzpicture}
\end{minipage}
}

% --- OT tableaus --- %

% Sec. 3.2, first tabl.

\newcommand{\OTHLInput}{
\begin{minipage}{0.17\textwidth}
\begin{tikzpicture}[xscale=\myscalex,yscale=\myscaley]
\node (tone) at (2,0) {(= H)};
\node (syl) at (0,0) {\textsigma};
\node (Rt) at (0,1) {o};
\node (H) at (-0.5,2) {H};
\node (R) at (0.5,3) {h};
\node (Rt2) at (1.5,1.0) {o};
%\node (H2) at (1.0,2) {\epen{L}};
\node (R2) at (2.0,3) {\blue{l}};
\draw [thick] (syl.north) -- (Rt.south) ;
\draw [thick] (Rt.north) -- (H.south) ;
\draw [thick] (Rt.north) -- (R.south) ;
\draw [thick] (syl.north) -- (Rt2.south) ;
%\draw [dashed] (Rt2.north) -- (H2.south) ;
%\draw [dashed] (Rt2.north) -- (R2.south) ;
\end{tikzpicture}
\end{minipage}
}

\newcommand{\OTHLWinner}{
\begin{minipage}{0.17\textwidth}
\begin{tikzpicture}[xscale=\myscalex,yscale=\myscaley]
\node (tone) at (2,0) {(= HL)};
\node (syl) at (0,0) {\textsigma};
\node (Rt) at (0,1) {o};
\node (H) at (-0.5,2) {H};
\node (R) at (0.5,3) {h};
\node (Rt2) at (1.5,1.0) {o};
\node (H2) at (1.0,2) {\epen{L}};
\node (R2) at (2.0,3) {\blue{l}};
\draw [thick] (syl.north) -- (Rt.south) ;
\draw [thick] (Rt.north) -- (H.south) ;
\draw [thick] (Rt.north) -- (R.south) ;
\draw [thick] (syl.north) -- (Rt2.south) ;
\draw [dashed] (Rt2.north) -- (H2.south) ;
\draw [dashed] (Rt2.north) -- (R2.south) ;
\end{tikzpicture}
\end{minipage}
}

\newcommand{\OTHLSpreadingHOnly}{
\begin{minipage}{0.17\textwidth}
\begin{tikzpicture}[xscale=\myscalex,yscale=\myscaley]
\node (tone) at (2,0) {(= HM)};
\node (syl) at (0,0) {\textsigma};
\node (Rt) at (0,1) {o};
\node (H) at (-0.5,2) {H};
\node (R) at (0.5,3) {h};
\node (Rt2) at (1.5,1.0) {o};
%\node (H2) at (1.0,2) {\epen{L}};
\node (R2) at (2.0,3) {\blue{l}};
\draw [thick] (syl.north) -- (Rt.south) ;
\draw [thick] (Rt.north) -- (H.south) ;
\draw [thick] (Rt.north) -- (R.south) ;
\draw [thick] (syl.north) -- (Rt2.south) ;
\draw [dashed] (Rt2.north) -- (R2.south) ;
\draw [dashed] (Rt2.north) -- (H.south) ;
\end{tikzpicture}
\end{minipage}
}

\newcommand{\OTHLInsertH}{
\begin{minipage}{0.17\textwidth}
\begin{tikzpicture}[xscale=\myscalex,yscale=\myscaley]
\node (tone) at (2,0) {(= HM)};
\node (syl) at (0,0) {\textsigma};
\node (Rt) at (0,1) {o};
\node (H) at (-0.5,2) {H};
\node (R) at (0.5,3) {h};
\node (Rt2) at (1.5,1.0) {o};
\node (H2) at (1.0,2) {\epen{H}};
\node (R2) at (2.0,3) {\blue{l}};
\draw [thick] (syl.north) -- (Rt.south) ;
\draw [thick] (Rt.north) -- (H.south) ;
\draw [thick] (Rt.north) -- (R.south) ;
\draw [thick] (syl.north) -- (Rt2.south) ;
\draw [dashed] (Rt2.north) -- (H2.south) ;
\draw [dashed] (Rt2.north) -- (R2.south) ;
\end{tikzpicture}
\end{minipage}
}

\newcommand{\OTHLOverwriting}{
\begin{minipage}{0.17\textwidth}
\begin{tikzpicture}[xscale=\myscalex,yscale=\myscaley]
\node (syl) at (0,0) {\textsigma};
\node (Rt) at (0,1) {o};
\node (H) at (-0.5,2) {H};
\node (R) at (0.5,3) {h};
\node (Rt2) at (1.5,1.0) {o};
%\node (H2) at (1.0,2) {\epen{L}};
\node (R2) at (2.0,3) {\blue{l}};
\draw [thick] (syl.north) -- (Rt.south) ;
\draw [thick] (Rt.north) -- (H.south) ;
\draw [thick] (Rt.north) -- (R.south) ;
\draw [thick] (syl.north) -- (Rt2.south) ;
%\draw [dashed] (Rt2.north) -- (H2.south) ;
\draw [dashed] (Rt.north) -- (R2.south) ;
\node (del) at (0.3,1.9) {\textbf{=}};
\end{tikzpicture}
\end{minipage}
}

\newcommand{\OTHLSpreading}{
\begin{minipage}{0.17\textwidth}
\begin{tikzpicture}[xscale=\myscalex,yscale=\myscaley]
\node (syl) at (0,0) {\textsigma};
\node (Rt) at (0,1) {o};
\node (H) at (-0.5,2) {H};
\node (R) at (0.5,3) {h};
\node (Rt2) at (1.5,1.0) {o};
%\node (H2) at (1.0,2) {\epen{L}};
\node (R2) at (2.0,3) {\blue{l}};
\draw [thick] (syl.north) -- (Rt.south) ;
\draw [thick] (Rt.north) -- (H.south) ;
\draw [thick] (Rt.north) -- (R.south) ;
\draw [thick] (syl.north) -- (Rt2.south) ;
%\draw [dashed] (Rt2.north) -- (H2.south) ;
\draw [dashed] (Rt2.north) -- (H.south) ;
\draw [dashed] (Rt2.north) -- (R.south) ;
\end{tikzpicture}
\end{minipage}
}

% Sec. 4.2, second tabl.: phrase-medial position

\newcommand{\OTHnoLInput}{
\begin{minipage}{0.17\textwidth}
\begin{tikzpicture}[xscale=\myscalex,yscale=\myscaley]
\node (tone) at (2,0) {(= H)};
\node (syl) at (0,0) {\textsigma};
\node (Rt) at (0,1) {o};
\node (H) at (-0.5,2) {H};
\node (R) at (0.5,3) {h};
\node (Rt2) at (1.5,1.0) {o};
%\node (H2) at (1.0,2) {\epen{L}};
%\node (R2) at (2.0,3) {\blue{l}};
\draw [thick] (syl.north) -- (Rt.south) ;
\draw [thick] (Rt.north) -- (H.south) ;
\draw [thick] (Rt.north) -- (R.south) ;
\draw [thick] (syl.north) -- (Rt2.south) ;
\end{tikzpicture}
\end{minipage}
}

\newcommand{\OTHnoLEpenth}{
\begin{minipage}{0.17\textwidth}
\begin{tikzpicture}[xscale=\myscalex,yscale=\myscaley]
\node (tone) at (2,0) {(= HM)};
\node (syl) at (0,0) {\textsigma};
\node (Rt) at (0,1) {o};
\node (H) at (-0.5,2) {H};
\node (R) at (0.5,3) {h};
\node (Rt2) at (1.5,1.0) {o};
\node (H2) at (1.0,2) {\epen{L}};
\node (R2) at (2.0,3) {\epen{h}};
\draw [thick] (syl.north) -- (Rt.south) ;
\draw [thick] (Rt.north) -- (H.south) ;
\draw [thick] (Rt.north) -- (R.south) ;
\draw [thick] (syl.north) -- (Rt2.south) ;
\draw [dashed] (Rt2.north) -- (H2.south) ;
\draw [dashed] (Rt2.north) -- (R2.south) ;
\end{tikzpicture}
\end{minipage}
}

\newcommand{\OTHnoLSpreading}{
\begin{minipage}{0.17\textwidth}
\begin{tikzpicture}[xscale=\myscalex,yscale=\myscaley]
\node (tone) at (2,0) {(= HH)};
\node (syl) at (0,0) {\textsigma};
\node (Rt) at (0,1) {o};
\node (H) at (-0.5,2) {H};
\node (R) at (0.5,3) {h};
\node (Rt2) at (1.5,1.0) {o};
%\node (H2) at (1.0,2) {\epen{L}};
%\node (R2) at (2.0,3) {\blue{l}};
\draw [thick] (syl.north) -- (Rt.south) ;
\draw [thick] (Rt.north) -- (H.south) ;
\draw [thick] (Rt.north) -- (R.south) ;
\draw [thick] (syl.north) -- (Rt2.south) ;
\draw [dashed] (Rt2.north) -- (H.south) ;
\draw [dashed] (Rt2.north) -- (R.south) ;
\end{tikzpicture}
\end{minipage}
}

% Sec. 4.2, third tabl., LM is unaffected by L\%

\newcommand{\OTLMInput}{
\begin{minipage}{0.2\textwidth}
\begin{tikzpicture}[xscale=\myscalex,yscale=\myscaley]
\node (tone) at (2,0) {(= LM)};
\node (syl) at (0,0) {\textsigma};
\node (Rt) at (0,1) {o};
\node (H) at (-0.5,2) {L};
\node (R) at (0.5,3) {l};
\node (Rt2) at (1.5,1.0) {o};
\node (H2) at (1.0,2) {L};
\node (R2) at (2.0,3) {h};
\node (R3) at (3.0,3) {\blue{l}};
\draw [thick] (syl.north) -- (Rt.south) ;
\draw [thick] (Rt.north) -- (H.south) ;
\draw [thick] (Rt.north) -- (R.south) ;
\draw [thick] (syl.north) -- (Rt2.south) ;
\draw [thick] (Rt2.north) -- (H2.south) ;
\draw [thick] (Rt2.north) -- (R2.south) ;
\end{tikzpicture}
\end{minipage}
}

\newcommand{\OTLMReplace}{
\begin{minipage}{0.2\textwidth}
\begin{tikzpicture}[xscale=\myscalex,yscale=\myscaley]
\node (tone) at (2,0) {(= LL)};
\node (syl) at (0,0) {\textsigma};
\node (Rt) at (0,1) {o};
\node (H) at (-0.5,2) {L};
\node (R) at (0.5,3) {l};
\node (Rt2) at (1.5,1.0) {o};
\node (H2) at (1.0,2) {L};
\node (R2) at (2.0,3) {h};
\node (R3) at (3.0,3) {\blue{l}};
\draw [thick] (syl.north) -- (Rt.south) ;
\draw [thick] (Rt.north) -- (H.south) ;
\draw [thick] (Rt.north) -- (R.south) ;
\draw [thick] (syl.north) -- (Rt2.south) ;
\draw [thick] (Rt2.north) -- (H2.south) ;
\draw [thick] (Rt2.north) -- (R2.south) ;
\draw [dashed] (Rt2.north) -- (R3.south) ;
\node (del) at (1.8,2.1) {\textbf{=}};
\end{tikzpicture}
\end{minipage}
}

\newcommand{\OTLMTwoReg}{
\begin{minipage}{0.2\textwidth}
\begin{tikzpicture}[xscale=\myscalex,yscale=\myscaley]
\node (tone) at (2,0) {(= LML)};
\node (syl) at (0,0) {\textsigma};
\node (Rt) at (0,1) {o};
\node (H) at (-0.5,2) {L};
\node (R) at (0.5,3) {l};
\node (Rt2) at (1.5,1.0) {o};
\node (H2) at (1.0,2) {L};
\node (R2) at (2.0,3) {h};
\node (R3) at (3.0,3) {\blue{l}};
\draw [thick] (syl.north) -- (Rt.south) ;
\draw [thick] (Rt.north) -- (H.south) ;
\draw [thick] (Rt.north) -- (R.south) ;
\draw [thick] (syl.north) -- (Rt2.south) ;
\draw [thick] (Rt2.north) -- (H2.south) ;
\draw [thick] (Rt2.north) -- (R2.south) ;
\draw [dashed] (Rt2.north) -- (R3.south) ;
\end{tikzpicture}
\end{minipage}
}

% Sec. 4.2, fourth tabl., L is affected by L\% but M is not

\newcommand{\OTLInput}{
\begin{minipage}{0.17\textwidth}
\begin{tikzpicture}[xscale=\myscalex,yscale=\myscaley]
\node (tone) at (2,0) {(= L)};
\node (syl) at (0,0) {\textsigma};
\node (Rt) at (0,1) {o};
\node (H) at (-0.5,2) {L};
\node (R) at (0.5,3) {l};
\node (R2) at (2,3) {\blue{l}};
\draw [thick] (syl.north) -- (Rt.south) ;
\draw [thick] (Rt.north) -- (H.south) ;
\draw [thick] (Rt.north) -- (R.south) ;
\end{tikzpicture}
\end{minipage}
}

\newcommand{\OTLLowered}{
\begin{minipage}{0.17\textwidth}
\begin{tikzpicture}[xscale=\myscalex,yscale=\myscaley]
\node (tone) at (2,0) {(= LL)};
\node (syl) at (0,0) {\textsigma};
\node (Rt) at (0,1) {o};
\node (H) at (-0.5,2) {L};
\node (R) at (0.5,3) {l};
\node (R2) at (2,3) {\blue{l}};
\draw [thick] (syl.north) -- (Rt.south) ;
\draw [thick] (Rt.north) -- (H.south) ;
\draw [thick] (Rt.north) -- (R.south) ;
\draw [dashed] (Rt.north) -- (R2.south) ;
\end{tikzpicture}
\end{minipage}
}

\newcommand{\OTMInput}{
\begin{minipage}{0.17\textwidth}
\begin{tikzpicture}[xscale=\myscalex,yscale=\myscaley]
\node (tone) at (2,0) {(= M)};
\node (syl) at (0,0) {\textsigma};
\node (Rt) at (0,1) {o};
\node (H) at (-0.5,2) {L};
\node (R) at (0.5,3) {h};
\node (R2) at (2,3) {\blue{l}};
\draw [thick] (syl.north) -- (Rt.south) ;
\draw [thick] (Rt.north) -- (H.south) ;
\draw [thick] (Rt.north) -- (R.south) ;
\end{tikzpicture}
\end{minipage}
}

\newcommand{\OTMLowered}{
\begin{minipage}{0.17\textwidth}
\begin{tikzpicture}[xscale=\myscalex,yscale=\myscaley]
\node (tone) at (2,0) {(= ML)};
\node (syl) at (0,0) {\textsigma};
\node (Rt) at (0,1) {o};
\node (H) at (-0.5,2) {L};
\node (R) at (0.5,3) {h};
\node (R2) at (2,3) {\blue{l}};
\draw [thick] (syl.north) -- (Rt.south) ;
\draw [thick] (Rt.north) -- (H.south) ;
\draw [thick] (Rt.north) -- (R.south) ;
\draw [dashed] (Rt.north) -- (R2.south) ;
\end{tikzpicture}
\end{minipage}
}

% Sec. 4.2, fifth tableau, polar questions with level tones

\newcommand{\OTLPolIn}{
\begin{minipage}{0.20\textwidth}
\begin{tikzpicture}[xscale=\myscalex-0.05,yscale=\myscaley-0.05]
\node (tone) at (3.5,0) {(= L)};
\node (syl) at (0,0) {\textsigma};
\node (syl2) at (2,0) {\red{\textsigma}};
\node (Rt) at (0,1) {o};
\node (H) at (-0.5,2) {L};
\node (R) at (0.5,3) {l};
\node (Rt2) at (2,1) {\red{o}};
\draw [thick] (syl.north) -- (Rt.south) ;
\draw [thick,red] (syl2.north) -- (Rt2.south) ;
\draw [thick] (Rt.north) -- (H.south) ;
\draw [thick] (Rt.north) -- (R.south) ;
\end{tikzpicture}
\end{minipage}
}

\newcommand{\OTLPolDef}{
\begin{minipage}{0.20\textwidth}
\begin{tikzpicture}[xscale=\myscalex-0.05,yscale=\myscaley-0.05]
\node (tone) at (3.5,0) {(= L.M)};
\node (syl) at (0,0) {\textsigma};
\node (syl2) at (2,0) {\red{\textsigma}};
\node (Rt) at (0,1) {o};
\node (H) at (-0.5,2) {L};
\node (R) at (0.5,3) {l};
\node (H2) at (1.5,2) {\epen{L}};
\node (R2) at (2.5,3) {\epen{h}};
\node (Rt2) at (2,1) {\red{o}};
\draw [thick] (syl.north) -- (Rt.south) ;
\draw [thick,red] (syl2.north) -- (Rt2.south) ;
\draw [thick] (Rt.north) -- (H.south) ;
\draw [thick] (Rt.north) -- (R.south) ;
\draw [semithick,dashed] (Rt2.north) -- (H2.south) ;
\draw [semithick,dashed] (Rt2.north) -- (R2.south) ;
\end{tikzpicture}
\end{minipage}
}

\newcommand{\OTLPolAlt}{
\begin{minipage}{0.20\textwidth}
\begin{tikzpicture}[xscale=\myscalex-0.05,yscale=\myscaley-0.05]
\node (tone) at (3.5,0) {(= L.L)};
\node (syl) at (0,0) {\textsigma};
\node (syl2) at (2,0) {\red{\textsigma}};
\node (Rt) at (0,1) {o};
\node (H) at (-0.5,2) {L};
\node (R) at (0.5,3) {l};
\node (Rt2) at (2,1) {\red{o}};
\draw [thick] (syl.north) -- (Rt.south) ;
\draw [thick,red] (syl2.north) -- (Rt2.south) ;
\draw [thick] (Rt.north) -- (H.south) ;
\draw [thick] (Rt.north) -- (R.south) ;
\draw [semithick,dashed] (Rt2.north) -- (H.south) ;
\draw [semithick,dashed] (Rt2.north) -- (R.south) ;
\end{tikzpicture}
\end{minipage}
}

% Sec. 4.2, sixth tableau, polar questions with contour tones

\newcommand{\OTLLPolIn}{
\begin{minipage}{0.23\textwidth}
\begin{tikzpicture}[xscale=\myscalex-0.05,yscale=\myscaley-0.05]
\node (tone) at (5.2,0) {(= L)};
\node (syl) at (0,0) {\textsigma};
\node (syl3) at (3.4,0) {\red{\textsigma}};
\node (Rt) at (0,1) {o};
\node (Rt2) at (1.7,1) {o};
\node (Rt3) at (3.4,1) {\red{o}};
\node (H) at (-0.5,2) {L};
\node (R) at (0.5,3) {l};
\draw [thick] (syl.north) -- (Rt.south) ;
\draw [thick] (syl.north) -- (Rt2.south) ;
\draw [thick,red] (syl3.north) -- (Rt3.south) ;
\draw [thick] (Rt.north) -- (H.south) ;
\draw [thick] (Rt.north) -- (R.south) ;
\end{tikzpicture}
\end{minipage}
}

\newcommand{\OTLLPolDef}{
\begin{minipage}{0.23\textwidth}
\begin{tikzpicture}[xscale=\myscalex-0.05,yscale=\myscaley-0.05]
\node (tone) at (5.2,0) {(= L.M)};
\node (syl) at (0,0) {\textsigma};
\node (syl3) at (3.4,0) {\red{\textsigma}};
\node (Rt) at (0,1) {o};
\node (Rt2) at (1.7,1) {o};
\node (Rt3) at (3.4,1) {\red{o}};
\node (H) at (-0.5,2) {L};
\node (R) at (0.5,3) {l};
\node (H3) at (2.9,2) {\epen{L}};
\node (R3) at (3.9,3) {\epen{h}};
\draw [thick] (syl.north) -- (Rt.south) ;
\draw [thick] (syl.north) -- (Rt2.south) ;
\draw [thick,red] (syl3.north) -- (Rt3.south) ;
\draw [thick] (Rt.north) -- (H.south) ;
\draw [thick] (Rt.north) -- (R.south) ;
\draw [dashed] (Rt3.north) -- (H3.south) ;
\draw [dashed] (Rt3.north) -- (R3.south) ;
\end{tikzpicture}
\end{minipage}
}

\newcommand{\OTLLPolSkip}{
\begin{minipage}{0.23\textwidth}
\begin{tikzpicture}[xscale=\myscalex-0.05,yscale=\myscaley-0.05]
\node (tone) at (5.2,0) {(= L.L)};
\node (syl) at (0,0) {\textsigma};
\node (syl3) at (3.4,0) {\red{\textsigma}};
\node (Rt) at (0,1) {o};
\node (Rt2) at (1.7,1) {o};
\node (Rt3) at (3.4,1) {\red{o}};
\node (H) at (-0.5,2) {L};
\node (R) at (0.5,3) {l};
\draw [thick] (syl.north) -- (Rt.south) ;
\draw [thick] (syl.north) -- (Rt2.south) ;
\draw [thick,red] (syl3.north) -- (Rt3.south) ;
\draw [thick] (Rt.north) -- (H.south) ;
\draw [thick] (Rt.north) -- (R.south) ;
\draw [dashed] (Rt3.north) -- (H.south) ;
\draw [dashed] (Rt3.north) -- (R.south) ;
\end{tikzpicture}
\end{minipage}
}  
  
\newcommand{\ilit}[1]{#1\il{#1}}    
\newcommand{\isit}[1]{#1\is{#1}}  

\makeatletter
\let\thetitle\@title
\let\theauthor\@author 
\makeatother

\newcommand{\togglepaper}[1][0]{ 
  \bibliography{../localbibliography}
  %% hyphenation points for line breaks
%% Normally, automatic hyphenation in LaTeX is very good
%% If a word is mis-hyphenated, add it to this file
%%
%% add information to TeX file before \begin{document} with:
%% %% hyphenation points for line breaks
%% Normally, automatic hyphenation in LaTeX is very good
%% If a word is mis-hyphenated, add it to this file
%%
%% add information to TeX file before \begin{document} with:
%% \include{localhyphenation}
\hyphenation{
affri-ca-te
affri-ca-tes
com-ple-ments
par-a-digm
Sha-ron
Kings-ton
phe-nom-e-non
Daul-ton
Abu-ba-ka-ri
Ngo-nya-ni
Clem-ents 
King-ston
Tru-cken-brodt
Tab-leau
cophono-logies
mark-edness
Ti-gri-nya
a-mong
Car-stens
Lu-bu-ku-su
}
\hyphenation{
affri-ca-te
affri-ca-tes
com-ple-ments
par-a-digm
Sha-ron
Kings-ton
phe-nom-e-non
Daul-ton
Abu-ba-ka-ri
Ngo-nya-ni
Clem-ents 
King-ston
Tru-cken-brodt
Tab-leau
cophono-logies
mark-edness
Ti-gri-nya
a-mong
Car-stens
Lu-bu-ku-su
}
  \papernote{\scriptsize\normalfont
    \theauthor.
    \thetitle. 
    To appear in: 
    Emily Clem,   Peter Jenks \& Hannah Sande.
    Theory and description in African Linguistics: Selected papers from the 47th Annual Conference on African Linguistics.
    Berlin: Language Science Press. [preliminary page numbering]
  }
  \pagenumbering{roman}
  \setcounter{chapter}{#1}
  \addtocounter{chapter}{-1}
}

\newcommand{\upstep}{\textupstep}


% \newcounter{tableauxcounter}

\renewcommand{\textltailn}{ɲ}
\renewcommand{\textbardotlessj}{ɟ}

\newcommand{\emphkh}[1]{\textit{#1}} %originally \textbf, banned by the guidelines



\definecolor{lsDOIGray}{cmyk}{0,0,0,0.45}


\newcommand{\xuparrow}[1]{%
  {\left\uparrow\vbox to #1{}\right.\kern-\nulldelimiterspace}
}
\renewcommand \textupstep[1]{\char"A71B#1}
\renewcommand \textdownstep[1]{\char"A71C#1}
 
 \newcommand{\ꜛ}{\textsf{ꜛ}}
 
\def\biberror{\undefined}


\newcommand{\OTbox}[1]{\resizebox{.88\textwidth}{!}{#1}}
 
  \togglepaper[13]
}{}


\begin{document}
\maketitle

\section{Introduction} %no long sections withouth title allowed
The purpose of this paper is to investigate the functions fulfilled by the Aorist and the Perfect\footnote{Following \citet{khachhasp}, grammatical labels with an initial capital refer to language-specific categories (the \ili{Mano} Aorist and Perfect), while lower-case spelling is used for comparative concepts of \isi{aorist} and \isi{perfect}.} constructions in \ili{Mano} (< \ili{Mande}). \ili{Mano} is a \ili{Mande} language spoken in Guinea and Liberia by approximately 400,000 speakers. The data for this paper comes from Östen Dahl’s questionnaire on \isi{perfect} \citep{khachdahl2000}, as well as from spontaneous texts of various genres: routine exchanges; narratives; oral Bible translations; traditional ritual speech. The examples are marked according to the speech genre: \textit{el.} for elicitation, \textit{conv.} for routine exchanges, \textit{narr.} for narratives and \textit{rit.} for ritual speech. The excerpts from the oral Bible translations\footnote{Bible verses in \ili{English} are taken from the \citealt{khachNIV} with few exceptions.} and the excerpts from the Dahl's questionnaire are made recognizable by an explicit reference to the source. All elicitation and speech data were collected during fieldwork among the \ili{Mano} in 2009--2016.

A note on terminology will be helpful at the outset. I divide TAMP constructions in \ili{Mano} into perfective, imperfective and aspectually unspecified. The term ``perfective" is thus used here not to label a specific construction, but as a general classificatory term bringing together several aspectual constructions, including the Aorist and the Perfect, which are the \isi{focus} of the present paper. Although descriptive and typological works often classify \isi{perfect} as a category apart, it is useful to consider the \ili{Mano} Perfect a type of perfective construction in contrast with the Aorist. The two constructions clearly belong to the same family of constructions: as we will see in \sectref{khachsecneg}, negative Perfect is formed on the basis of the negative Aorist construction with addition of specific adverbs. Similarly, the term ``Aorist'' is rare in the literature and was clearly dispreferred by \citet{khachcomrie1976} (in contrast with the European tradition represented by \citet{khachplung2016} or \citet{khachmaisak2016}). However, it seemed useful to use the term ``Aorist'' as a label of a specific construction characterized by the perfective \isi{aspect}, to avoid confusion with perfective as a generic term. 

This paper is organized as follows. I begin by presenting a summary of \ili{Mano} tense, \isi{aspect}, modality and polarity system in \sectref{khachsect1} giving special attention to the constructions with perfective meaning. \sectref{khachsaorist} is dedicated to the \isi{aorist} construction. \sectref{khachsprf} focuses on the functions of the \isi{perfect} construction. \sectref{khachs4} and \sectref{khachs5} explore two secondary functions of the Perfect and the Aorist, namely, anticipation and transposition. Finally, \sectref{khachs6} is a discussion of the Aorist--Perfect opposition in a typological perspective. 

\section{Perfective constructions in Mano}
\label{khachsect1}
\subsection{Structure of Mano TAMP system}

TAMP distinctions in \ili{Mano} do not show up at the level of any one specific marker, but rather at the level of a construction which includes an \isi{auxiliary} or a \isi{copula}, a verb in a specific form, and, in certain cases, some other elements, such as adverbs or \isi{auxiliary} verbs.


There are two types of TAMP constructions in \ili{Mano}: constructions featuring a \isi{copula} and constructions featuring an \isi{auxiliary} marker (AUX). The \isi{auxiliary} markers (AUX) index the \isi{subject}’s person and number; these markers are organized in series expressing tense, \isi{aspect}, modality and polarity. \ili{Mano} counts eleven series of auxiliaries: \isi{perfect}, past, existential, imperfective, \isi{conjoint}, negative, conjunctive, prohibitive, subjunctive, prospective, and dubitative. The word order in constructions with auxiliaries is: S – AUX – (O) – V. The word order in \isi{copula} constructions is: S - (O) -- COP. For a full description of the \ili{Mano} aspectual system, see \citet{khachgramm}.


Table \ref{khachMPP} presents the \isi{perfect} and the past auxiliaries.
\begin{table}
\caption{Past and perfect auxiliary series in Mano}
\label{khachMPP}
\begin{tabular}{lcccccc}
\lsptoprule
&1\sc{sg}&2\sc{sg}&3\sc{sg}&1\sc{pl}&2\sc{pl}&3\sc{pl}\\ \midrule
past &\textit{ŋ̄ (mā)}&\textit{ī (ɓā)}&\textit{ē (ā)}&\textit{kō (kɔ̄ā)}&\textit{kā}&\textit{ō (wā)}\\
\isi{perfect} &\textit{māà}&\textit{ɓāà}&\textit{āà}&\textit{kɔ̄āà}&\textit{kāà}&\textit{wāà}\\
\lspbottomrule
\end{tabular}
\end{table}

The \isi{direct object} of transitive verbs is obligatorily expressed by a \isi{noun phrase} or a \isi{pronoun} of the basic (non-\isi{subject}) series. Past auxiliaries distinguish between a simple and a portemanteau form. The latter is used if the \isi{direct object} is a 3\textsuperscript{rd} person \textsc{sg} \isi{pronoun}; such markers are put in brackets in  Table \ref{khachMPP}. For \isi{perfect} auxiliaries there is no distinction between a simple and a portemanteau form. Compare the following two examples: in the first example, a simple and a portemanteau form are contrasted. Note the absence of this contrast in a similar context in the second example.

\begin{exe} \ex
\label{khachexpst}
\begin{xlist} \ex
\gll \textbf{ē}	ló.\\
\textbf{3\textsc{sg}.\textsc{pst}} go\\
\glt ‘(S)he left.’ (narr.)
\ex
\gll \textbf{ā}		zɛ̄.	\\
\textbf{3\textsc{sg}.\textsc{pst}>3\textsc{sg}}		kill \\
\glt ‘S(h)e killed him.’ (narr.)

\end{xlist}
\end{exe}

\begin{exe} \ex
\label{khachexprf}
\begin{xlist} \ex
\gll \textbf{āà}	ló.	\\
\textbf{3\textsc{sg}.\textsc{prf}}	go	\\
\glt ‘(S)he has left.’ (narr.)
\ex
\gll \textbf{āà}		zɛ̄.\\
\textbf{3\textsc{sg}.\textsc{prf}>3\textsc{sg}}		kill \\
\glt  ‘(S)he has killed him.’ (narr.)
\end{xlist}
\end{exe}


The verb can bear segmental and/or tonal morphemes. Note the example below with the imperfective construction, where the verb \textit{ló} `go' is used in the imperfective form, \textit{lō} `go:\textsc{ipfv}'.

\begin{exe}\ex
\gll lɛ́ɛ̀	lō.	\\
	3\textsc{sg}.\textsc{ipfv} go:\textsc{ipfv}\\
\glt ‘(S)he leaves.’ (narr.)
\end{exe}

\subsection{Affirmative perfective constructions}
\label{khachsecaffp}

The \isi{aorist} construction is formed with the \isi{auxiliary} of the past series (\textsc{pst}) and a verb in its lexical form, see \REF{khachexpst}\footnote{The past \isi{auxiliary} series is aspectually neutral, because the series is used not only in the \isi{aorist} construction, but also in the past imperfective construction which is not formed parallel to the imperfective construction, but rather parallel to the durative construction, see \citet[195-196]{khachgramm}.}

The \isi{perfect} construction is formed with the \isi{auxiliary} of the \isi{perfect} series (\textsc{prf}) and a verb in its lexical form, see \REF{khachexprf}.


The experiential value is expressed by the \isi{perfect} construction with the adverb \textit{dō} ‘(at least) once; never’:

\begin{exe}\ex
\gll	\textbf{kɔ̄āà}	mā	\textbf{dō}.	\\
	\textbf{1\textsc{pl}.\textsc{prf}>3\textsc{sg}}	hear	\textbf{once}	\\			
\glt ‘We have heard (about) it.’ (conv.)
\end{exe}

Other perfective constructions in \ili{Mano} include: resultative construction and recent past construction.

Like many African languages \citep{khachcarlson1992}, \ili{Mano} has a \isi{consecutive} construction. It is formed with an \isi{auxiliary} of the \isi{conjoint} series (\textsc{jnt}) and a verb in the \isi{conjoint} form. As its central function is to convey events on the main narrative event line, it functions like a perfective construction (and can often be replaced by the \isi{aorist} construction).

\begin{exe} \ex
\gll \textbf{ē}      lɛ̀      ā vòlò, \textbf{áà}  yílí     vɔ̀.\\
 \textbf{3\textsc{sg.pst}} place      \textsc{dem} stub.out \textbf{3\textsc{sg}.\textsc{jnt}} tree   fell:\textsc{jnt}\\
\glt `He cleared the field and felled the trees.' (narr.)
\end{exe}

\subsection{Negative perfective constructions}
\label{khachsecneg}
The negative \isi{aorist} construction is formed with the negative \isi{auxiliary} (\textsc{neg}) and the \isi{negative particle} \textit{gbā} preceding the \isi{direct object}; the verb is in the lexical form.

\begin{exe} \ex
\gll lɛ̀ɛ́	gbāā	gɛ̰̀.\\
	3\textsc{sg}.\textsc{neg}	\textsc{neg}>3\textsc{sg}	see\\
\glt ‘He didn’t see her.’ (narr.)
\end{exe} 

The negative \isi{perfect} construction is formed with the negative \isi{auxiliary} and the particle \textit{nɛ́ŋ̀} ‘yet’, following the verb in the lexical form.

\begin{exe}\ex
\gll ŋ̀	sɔ̄	dò	lɛ̀ɛ́	kɔ̀ɔ̀	nɛ́ŋ̀.	\\
	1\textsc{sg}.\textsc{poss}	cloth	\textsc{indef}	3\textsc{sg}.\textsc{neg}	dry	yet	\\
\glt ‘Some of my clothes have not dried yet.’ (el.)
\end{exe}

The negative experiential construction is formed like the \isi{perfect} construction; the difference is that the particle \textit{nɛ́ŋ̀} is replaced by the particle \textit{dō} ‘once, never’.

\begin{exe} \ex
\gll 	kòó	mīī	dò	gɛ̰̀	zèē	dō.	\\
	1\textsc{pl}.\textsc{neg}	person	\textsc{indef}	see	here	once	\\
\glt ‘We have never seen anyone here.’ (el.)
\end{exe}

\ili{Mano} also has negative resultative construction.

The present paper will be limited to the constructions of the Aorist and the Perfect, although a full analysis should include all affirmative and negative perfective constructions, including the resultative constructions, the analytic construction of recent past, and the \isi{consecutive} construction. For some details on the distribution between the Aorist and the \isi{consecutive} construction in the narrative, see \sectref{khachnarr}.
 
\section{Aorist}
\label{khachsaorist}
\subsection{Narrative}
\label{khachnarr}

The Aorist is the default tense in the narrative. The \isi{consecutive} construction has a limited distribution, usually occurring when the \isi{subject} is coreferential to the \isi{subject} of the previous clause, or with the speech verbs. Moreover, the \isi{consecutive} construction does not occur if the reported events occurred in the recent past. As for the \isi{perfect} construction, when used within the narrative, it is usually limited to the \isi{direct speech} or to the coda of the narrative (see \sectref{khachsprf}).  

\begin{exe}
\ex
\gll \textbf{ŋ̄} táá lūú, \textbf{ŋ̄} ɓálá mɛ̀nɛ̄ là, \textbf{ē} ŋ̄ sɔ̰́ɔ̰́ dɔ̄, \textbf{ŋ̄} gɛ̀lɛ̀ sí, \textbf{mā} pá á ká, \textbf{ē} gā. \\
\textbf{3\textsc{sg}.\textsc{pst}} walk bushes \textbf{1\textsc{sg}.\textsc{pst}} step snake on \textbf{3\textsc{sg}.\textsc{pst}} 1\textsc{sg} tooth stop \textbf{1\textsc{sg}.\textsc{pst}} stone take \textbf{1\textsc{sg}.\textsc{sg}>3\textsc{sg}} strike 3\textsc{sg} with \textbf{3\textsc{sg}.\textsc{pst}} die\\
\glt ‘I walked in the bushes, I stepped on a snake, it bit me, I took a stone, I hit it with it, it died.' (adapted from \citealt[801, ex. 8]{khachdahl2000})
\end{exe}

\subsection{Temporal adverbs}
\label{khachsectempadv}

The Aorist, as opposed to the Perfect, freely combines with temporal adverbs (see also \REF{khachexdeeka}).

\begin{exe} \ex
\gll	\textbf{ī}	yī	zɛ̄	pɛ̰́	sɛ̀?	 \\	
	\textbf{3\textsc{sg}.\textsc{pst}}	sleep	kill	yesterday.night	well\\	
\glt ‘(Question asked in the early morning) Did you sleep well last night?’ (conv.)
\end{exe}

\subsection{Annulled result}
\label{khachsectanres}
The Aorist is the only perfective form that can be used in the contexts with annulled result.

\begin{exe} \ex
\gll \textbf{kɔ̄ā}	dà	yéíŋwɔ̀	yí,	\textit{mais}	yéíŋwɔ̀	wáá	ká.	\\
\textbf{1\textsc{pl}.\textsc{pst}>3\textsc{sg}}	fall	joke	in	but	joke	\textsc{neg.cop}>3\textsc{sg}	with	 \\
\glt ‘We considered it a joke (lit.: we fell in a joke), but it isn’t a joke.’ (conv.)
\end{exe}

Only the Aorist is possible in combination with the verb \textit{pē} ‘fail to do something’ (be engaged, voluntarily or involuntarily, in an action that was interrupted before its natural termination):

\begin{exe} \ex
\gll	à	gɔ̄ɔ̄	\textbf{ē}	\textbf{pē}	é	ló	yíí	wì	kpà̰á̰	gbínīī	yāā	ká.\\
	\textsc{ref}	boat	\textbf{3\textsc{sg}.\textsc{pst}}	\textbf{fail}	3\textsc{sg}.\textsc{conj}	go	water	under	fish	heavy	\textsc{dem}	with\\
\glt ‘The boats did not sink (lit.: failed to sink), loaded with fish (lit.: with the heavy fish).' (and they filled both boats so that they began to sink, NIV, Lc 5:7). 
\end{exe}

Note the \isi{aorist} construction followed by the \isi{perfect} construction, the latter expressing an event which annulled the result of the former:

\begin{exe} \ex
\gll ŋ̀	kálémɔ̀	\textbf{ē}	nī	ŋ̄	ká,	\textbf{māà}	gɛ̰̀. \\
	1\textsc{sg}.\textsc{poss}	house	\textbf{3\textsc{sg}.\textsc{pst}}	forget	1\textsc{sg}	with	\textbf{1\textsc{sg}.\textsc{prf}>3\textsc{sg}}	see\\
\glt `I lost my house, but (now) I have seen it'. (conv.)
\end{exe}

\section{Perfect}
\label{khachsprf}
\subsection{Recent past}

The \isi{perfect} construction is extremely frequent in the everyday \textsc{routine exchanges}:

\begin{exe} \ex
\gll	\textbf{ɓāà}	ɓū	ɓèlè?					\\
	\textbf{2\textsc{sg}.\textsc{prf}}	rice	eat					\\
\glt ‘Have you eaten (rice, typical food)?’ (conv.)
\end{exe}

The Perfect combines with a \emphkh{very restricted set of temporal adverbs}, which even excludes some adverbs denoting recent past. Thus, the adverb \textit{pɛ́nɛ̄ɛ̄} `today' can combine with both the Aorist and the Perfect, while only the Aorist can combine with the adverb \textit{dɛ̄ɛ̄ká} ‘recently, now’.

\begin{exe} \ex
\begin{xlist} \ex
\label{khachexdeeka}
\gll	\textbf{ŋ̄}/*māà	nū	dɛ̄ɛ̄ká.	\\
		\textbf{1\textsc{sg}.\textsc{pst}}/1\textsc{sg}.\textsc{prf}	come	recently	\\
\glt ‘I have just arrived.’ (conv.)
\ex
\label{khachexpenee}
\gll \textbf{ŋ̄/māà}	nū	pɛ́nɛ̄ɛ̄.		\\
		1\textbf{1\textsc{sg}.\textsc{pst}/1\textsc{sg}.\textsc{prf}}	come	today \\	
\glt ‘I have arrived today.’ (conv.)
\end{xlist}
\end{exe}

When the \isi{perfect} construction appears in the narratives, it is most frequently used in \emphkh{direct} (\ref{khachexquot1}) and \emphkh{indirect speech} (\ref{khachexquot2}).

\begin{exe} \ex
\label{khachexquot1}
\gll	áà	gèè:	”\textbf{māà}	mā,	ŋ̀ŋ́	ló	gbāā à	gbɛ̄ɛ̄	kɛ̄-ɛ̀.”\\
3\textsc{sg}.\textsc{jnt}>3\textsc{sg}	say:\textsc{jnt}	\textbf{1\textsc{sg}.\textsc{prf}>3\textsc{sg}}	hear	1\textsc{sg}.\textsc{neg}	go	now 3\textsc{sg}	another	    do-\textsc{ger}\\
\glt ‘He says: I understand (lit.: I've understood), I won’t do it anymore.’ (narr.)
\end{exe}

\begin{exe} \ex
\label{khachexquot2}
\gll 	tó	ké	mā	ɓō	dàá	nɔ́	wɛ̄, láà	gèè	kélɛ̀	\textbf{māà}	gā.		\\
	stay	like.this	\textbf{1\textsc{sg}.\textsc{pst}>3\textsc{sg}}	implement fall.\textsc{ger}.with	only	\textsc{dem}	3\textsc{sg}.\textsc{ipfv}>3\textsc{sg}	say	that	\textbf{1\textsc{sg}.\textsc{prf}}	die\\
\glt ‘(A person relating his accident when he was hit by a motorbike and fainted). As I\textsubscript{i} stayed like this, fallen down, she\textsubscript{j} said that I\textsubscript{i} had died (lit.: have died).’ (narr.)
\end{exe}

Quotation and indirect speech fall apart from the narrative line; it may be suggested that quotes and indirect speech imitate the routine conversation, which would explain the usage of the Perfect. 

A piece of evidence supporting this explanation is that the Perfect is frequent in the \isi{direct speech} in oral Bible translations performed during the Sunday service. Again, it may be seen as an imitation of the routine conversation practice, where the Perfect is common. (Note that Östen \citet{khachdahl2014} chose to study \isi{direct speech} in the Bible separately to get an idea of the routinely spoken language, as opposed to its usage in the narrative.) The influence of \ili{French} can be minimized: in the \ili{French} source the \textit{passé composé} form was used, which in modern \ili{French} does not have the \isi{perfect} function anymore. Moreover, \ili{Mano}, including \ili{Mano} translators, are not fluent in \ili{French} and it is unlikely that \ili{French} exercises grammatical interference. Note the usage of the \isi{pronoun} of the 2\textsuperscript{nd} person \textit{ī} `your', through which it can be seen that the speech is indeed addressed -- in this case, to the city of Jerusalem:

\begin{exe}\ex
\gll kō ŋwūmɛ́ɓōmì \textbf{āà} ī yókò kɛ̄ áà lò gbèkènī ī ká.\\
\textsc{2sg} savior \textbf{\textsc{3sg.prf}} \textsc{2sg} enemy do \textsc{3sg.jnt} go:\textsc{jnt} far \textsc{2sg} with\\
\glt `Our savior has made your enemies go far from you.' (he has turned back your enemy, NIV, Ze 3:15)
\end{exe}


\subsection{Relevant past}

The \isi{perfect} construction can be used to relate a past event regardless of the time when it happened, provided it is still relevant (specifically, if there has not been any intervening event that annulled the effect of the event in question, in which case the Aorist is used, see \sectref{khachsectanres}). 

\begin{exe}
\ex
\gll \textbf{māà} mīnīɔ̰̄ɔ̰̀  pèèlɛ̄ kpɔ́ ŋ̄ sɔ́nɔ́.\\
\textbf{1\textsc{sg}.\textsc{prf}} million two put 1\textsc{sg} near\\
\glt [Question: I was told you are collecting money for your new motorbike. How much money have you collected so far?] `I have collected two million (Guinean francs).' (adapted from \citealt[803, ex. 42]{khachdahl2000})
\end{exe}

The Aorist is somewhat acceptable in these contexts. It becomes unacceptable when the event is in \isi{focus}, which happens when the assertion of the event is made as a response to a yes/no question or in contrast to what has been said before.

\begin{exe} \ex
\gll gbāō, \textbf{āà}/*ē gā.\\
no \textbf{3\textsc{sg.prf}}/3\textsc{sg.pst} die\\
\glt `[Question: Is the chief still alive?] No, he has died.' (adapted from \citealt{khachdahl2000}:801, ex. 3)
\end{exe}

The contrast between the Perfect and the Aorist can be seen when the description of some past events serves to explain the current situation. The following example is an adopted example 46 from Dahl's questionnaire \citep[803]{khachdahl2000}. The stimulus question was [A is setting out on a long journey on an old motorbike. B asks: What if something goes wrong with your motorbike on the way? A responds:]

\begin{exe}\ex
\label{khachexmoto}
\gll \textbf{māà}/*ŋ̄ pàà lɔ́, \textbf{māà}/*ŋ̄ sɛ́ɛ̀nè lɔ́. \\
\textbf{1\textsc{sg.prf}}/1\textsc{sg.pst} piece buy \textbf{1\textsc{sg.prf}}/1\textsc{sg.pst} chain buy\\
\glt `I've bought (spare) parts, I have bought a chain.' (I can replace them if needed.)
\end{exe}

Here the response is configured as a little narrative. However, it is intended to answer B's question and serves as an explanation of how A got prepared for his trip, and not just to relate past events. Had the Aorist been the construction used in this context, it would not have had any relation to the question asked, and the answer would have sounded odd. The key semantic contribution of the marking with the Perfect, then, is that it underlines the relevance of the actions A undertook for the current (and future) situation.


The \isi{perfect} construction may combine with adverbs like \textit{pɛ́lɛ̀} ‘1. early, 2. a while ago’, meaning that the action took place in the relatively remote past, but assuming it is still relevant:

\begin{exe} \ex
\gll 	\textbf{āà}	gbɛ̰̀	à	mɔ̀	pɛ́lɛ̀	ŋwɔ́	yīè	kɛ̄-ɛ̀	ká.\\
	\textbf{3\textsc{sg}.\textsc{prf}}	put	3\textsc{sg}	on	early	thing	good	do-\textsc{ger}	with\\
\glt ‘He started doing good things a long time ago.’ (narr.)
\end{exe}


The prophecies, especially those of the Old Testament, are often translated with the \isi{perfect} construction, which conveys their eternal relevance. 

\begin{exe} \ex
\gll 	kò	nɛ́	dɛ̄ɛ̄	\textbf{wāà}	nɔ̄	kō	lɛ̀ɛ̄.\\
	1\textsc{pl}.\textsc{poss}	child	new	\textbf{3\textsc{pl}.\textsc{prf}>3\textsc{sg}}	give	1\textsc{pl}	to	\\
\glt ‘A new child of ours, they have given him to us.’ (For to us a child is born, NIV, Is 9:6).
\end{exe}

\subsection{Coda of a narrative}
\label{khachcodanarr}
The Perfect often marks the concluding sentence in a narrative or other type of text describing a sequence of events. Thus, the \emphkh{descriptions of procedures} are often concluded by the \isi{perfect} construction, as in the following description of how to make an aluminum kettle:

\begin{exe}\ex
\gll {\ldots}wā	ɲɛ̀ɛ̄sɛ́lɛ́	bɛ̰̀ɛ̰̄	ɓō	yī,	kɛ̄	\textbf{wāà}	gbɔ̄ɔ̄	bɛ̀ī.\\
3\textsc{pl}.\textsc{pst}>3\textsc{sg}	sand	too	take.off	there at.that.moment	\textbf{3\textsc{pl}.\textsc{prf}} kettle	make\\
\glt ‘{\ldots}you also took away the sand, so you’ve made a kettle.’ (narr.)
\end{exe}

In narratives proper, the \isi{perfect} construction often marks \emphkh{concluding events}, as in the following three propositions closing a fairy tale:

\begin{exe} \ex
\begin{xlist} \ex
\gll	sìī	lé	\textbf{āà}	pā.\\
		spider	mouth	\textbf{3\textsc{sg}.\textsc{prf}}	fill\\
\glt ‘Spider was surprised.’
\ex
\gll yé	\textbf{wāà}	gá̰á̰	fɛ̰̀ɛ̰̀	ɛ̰̄	sìī	gí	\textbf{āà}	fɔ̰́.\\
		when	\textbf{3\textsc{pl}.\textsc{prf}>3\textsc{sg}}	drag	long	\textsc{top}	spider stomach	\textbf{3\textsc{sg}.\textsc{prf}}	pierce \\
\glt ‘After they dragged him for a long time, Spider’s stomach pierced.’
\ex	
\label{khachexspider}
\gll sìī	\textbf{āà}	gā,	là	nɔ́ɔ̀	\textbf{wāà}	ŋwɛ̀ŋ̄	lɛ́ɛ́	là.\\
		spider	\textbf{3\textsc{sg}.\textsc{prf}}	die	3\textsc{sg}.\textsc{poss}	child.\textsc{pl}	\textbf{3\textsc{pl}.\textsc{prf}}	disperse	leave	on\\
\glt  ‘Spider died and its children dispersed on the leaves.’ (narr.)
\end{xlist}
\end{exe}

When I asked my language assistant to explain this sequence of \isi{perfect} constructions, he said that the narrator took his time finishing the story, otherwise one \isi{perfect} construction as in \REF{khachexspider} would be enough.

Similarly, the Perfect can be used (although rarely) to mark an \emphkh{intermediate coda} ending a subepisode in the narrative. 

\begin{exe} \ex
\gll 	tó	ē	nɛ́fú	ɓɛ̄	mɛ̀,	ē	ē	léyíí	sùò	à	là. à	mɛ́	\textbf{āà}	ɓā.\\
	then	3\textsc{sg}.\textsc{pst}	child	\textsc{dem}	beat	3\textsc{sg}.\textsc{pst}	3\textsc{sg}.\textsc{refl}	saliva	spit	3\textsc{sg}	on 3\textsc{sg}	surface	\textbf{3\textsc{sg}.\textsc{prf}}	cover.with.wounds\\
\glt ‘Then she drew him down, then she beat the child and spit on him. He became all covered with wounds.’ (narr.)
\end{exe}



Typical situations expressed by the Aorist are either 1. atemporal, as in the case of narratives, 2. embedded in the time frame indicated by the temporal adverbs and detached from the moment of enunciation, or 3. irrelevant for the present situation, as in the case of the contexts with annulled result. The Perfect, on the contrary, is typically used when the described situation  is closely related to the moment of enunciation:  by bearing relevant consequences, including (in some cases) by being recent.

In what follows, I will describe two secondary functions of the Perfect and the Aorist, namely, anticipation and transposition, in which their basic aspectial characteristics will be supported.

\section{Anticipation}
\label{khachs4}
Both the Perfect and the Aorist can be used with an anticipatory function, when a future event is expressed as if it has already happened \citep[cf.][224]{khachhanks1990}. The Perfect is usually used when the event is expected to occur in the nearest future:

\begin{exe} \ex
\gll	\textbf{māà}	nū!\\
	\textbf{1\textsc{sg}.\textsc{prf}}	come\\
\glt ‘I’ll be right back! (lit.: I have come!)’ (conv.)
\end{exe}

The Aorist can also be used with an anticipatory function. Firstly, it can replace the imperfective or the future construction in a sequence of events in the Imperfective/Future:

\begin{exe} \ex
\gll	\textbf{íì}	lō,	ī	nā	\textbf{ē}	ló,	ɓà	nɔ́ɔ̀	yààkā	\textbf{ō}	ló.\\
	\textbf{2\textsc{sg}.\textsc{ipfv}}	go:\textsc{ipfv}	2\textsc{sg}	wife	\textbf{2\textsc{sg}.\textsc{pst}}	go	2\textsc{sg}.\textsc{poss}	child	three	\textbf{3\textsc{pl}.\textsc{pst}}	go\\
\glt ‘You will go, your wife will go, three of your children will go.’ (conv.)
\end{exe}

Secondly, it is used in ritual formulas of benediction (\ref{khachexaorant2}). Importantly, the action is not necessarily supposed to be realized immediately (although it may).

\begin{exe} \ex
\label{khachexaorant2}
\gll 	\textbf{kɔ̄ā}	lɛ̀	gɛ̰̀	zòkpɔ́là	àɲɛ̀nɛ́zɛ̀	kó	ɓɔ̄	yī!\\
	\textbf{1\textsc{pl}.\textsc{pst}>3\textsc{sg}}	place	see	peace	for.that	1\textsc{pl}.\textsc{conj}	arrive	there\\
\glt ‘(Ritual formula framing a benediction) We will see it in peace, in order for it to obtain, let us arrive there.’ (rit.)
\end{exe}

\section{Transposition}
\label{khachs5}
By \textit{transposition} (\citealt[217-223]{khachhanks1990}) I understand the function in which the reference point does not coincide with the moment of enunciation, but is transposed on the time scale: in the case of \ili{Mano}, the reference point is usually transposed to the past. A term most often used for the forms fulfilling this function is ``anterior" \citep{khachbybee1994} or ``pluperfect" \citep{khachsich2013}. This function is typically associated with \isi{perfect} forms \citep{khachklein1992, khachklein1994} to the point that perfects themselves are sometimes called ``anteriors". However, as I will make clear below, in \ili{Mano} both the Aorist and the Perfect can function as ``anteriors".

The following two examples are taken from narratives; the events of the main narrative line are expressed by the \isi{aorist} construction. The background events which occurred immediately prior to the events of the main story line are expressed by the \isi{perfect} construction:

\begin{exe} \ex
\gll	ɓūwɛ́lɛ́	nì,	ɓáá	nì,	dìì	nì	nɛ́	\textbf{wāà}	zɛ̄ tɛ̀kɛ́tɛ̀kɛ́	ɛ̄	\textbf{ē}	tó	gbāā	tíé.\\
	rice	\textsc{pl}	sheep	\textsc{pl}	cow	\textsc{pl}	\textsc{rel}	\textbf{3\textsc{pl}.\textsc{prf}>3\textsc{sg}}	kill completely	\textsc{top}	\textbf{3\textsc{sg}.\textsc{pst}}	stay	now	fire	\\	
\glt ‘The rice, the sheep, the cows that they had killed completely, they were cooking now (lit.: they stayed on the fire now).’ (narr.)
\end{exe}

%\begin{exe} \ex
%\gll \textbf{ō}	nū	zèē	ā	kɛ̄	zèē	\textbf{āà}	dɔ̄	dḭ̄ .\\
%	\textbf{3\textsc{pl}.\textsc{pst}}	come	here	\textsc{top}	at.that.moment	here	\textbf{3\textsc{sg}.\textsc{prf}}	stop	silently\\
%\glt ‘When they came here, there was silence (lit.: here had stopped in silence).’
%\end{exe}

If, however, the background event happened long before the reference point, the \isi{aorist} construction is used.

\begin{exe} \ex
\gll \textbf{wā}	gèē	à	lɛ̀ɛ̄	é	nū	Moise	là	tɔ́ŋ̀	sɛ̀ɓɛ̀ yā	ká	tɛ́	kō	ŋwūmɛ́ɓōmì	\textbf{ē}	à	dɔ̀kɛ̄	Israël	mìà mɔ́ɔ̄ŋwɔ̀	yā. \\
	\textbf{3\textsc{pl}.\textsc{pst}>3\textsc{sg}}	say	3\textsc{sg}	to	3\textsc{sg}.\textsc{conj}	come	Moses	3\textsc{sg}.\textsc{poss}	law	book:\textsc{cs} \textsc{dem}	with	\textsc{rel}	1\textsc{pl}	savior	\textbf{3\textsc{sg}.\textsc{pst}}	3\textsc{sg}	give	Israel	person.\textsc{pl}:\textsc{cs} because.of	\textsc{top} \\
\glt ‘They told him to come with the book of the Law of Moses that our savior gave because of the people of Israel’ (They told Ezra the teacher of the Law to bring out the Book of the Law of Moses, which the Lord had commanded for Israel, NIV, Ne 8:1).
\end{exe}

The Perfect can also be used in temporal clauses with habitual meaning, or with reference to the future, as well as in the real protasis of \isi{conditional} clauses. The construction is the same in both cases.\footnote{In the \ili{French} spoken by \ili{Mano}, especially by children, \textit{quand} ‘when’ and \textit{si} ‘if’ are often confused: \textit{Tantie, \textbf{si} tu finis de travailler, on va lire?}  ‘Aunty, if you finish working, we will read?’.} The protasis is closely tied to the apodosis by the causal relation, so the Aorist can never be used in this position.  

\begin{exe} \ex
\gll yé	\textbf{āà}	ɓɔ̄	nɔ́	yílí	gbùò	yā	bḭ́	mɔ̀	ɔ̄, lɛ́ɛ̀	wàà	gbāā	yílí	gbùò	ɓɛ̄	gáná	yí. \\
	when	\textbf{3\textsc{sg}.\textsc{prf}}	arrive	only	tree	big	\textsc{dem}	shadow	on	\textsc{top} 3\textsc{sg}.\textsc{ipfv}	enter:\textsc{ipfv}	now	tree	big	\textsc{dem}	root	in\\
\glt ‘When she enters under the shadow of this big tree, she gets inside its root.’ (narr.)
\end{exe}

\section{Discussion}
\label{khachs6}

As suggested in the foundational works by \citet{khachcomrie1976}, \citet{khachmccoard1978} and \citet{khachdahl1985}, the general positive definition of \isi{perfect} is the continuing relevance of a previous situation \citep[56]{khachcomrie1976}. This definition seems to match the Perfect in \ili{Mano} quite closely.

The resultative meaning is often considered the core meaning of \isi{perfect} for semantic reasons (because the result is viewed as the clearest manifestation of the relevance of the situation), but also for diachronic reasons (as \isi{perfect} is often a grammaticalized resultative construction, \citealt{khachplung2016}). In contrast, the \ili{Mano} Perfect, formed with an \isi{auxiliary} and a verbal root, is no more analytical than any other TAMP construction in \ili{Mano} and it is unlikely that it grammaticalized from a resultative construction. 

Östen Dahl's cross-linguistic study of parallel corpora (Bible translations into several European languages, \citealt{khachdahl2014}) confirms that the prototypical contexts for the \isi{perfect} involve event relevance (cf.: `{}``Take heart, daughter,'' he said, ``your faith has healed you.''{}', NIV, Ma 9:22). In these contexts, the target Bible verse was systematically translated with the use of a \isi{perfect} construction. Note, however, that a different parallel corpus study focusing on a smaller corpus, consisting of translations of \textit{Alice in Wonderland} and \textit{Winnie the Pooh}, but on a larger linguistic sample, including the languages of the Balkans (Greek, Bulgarian, Macedonian), came to a different conclusion: that the semantic core of the European \isi{perfect} is not current relevance, but experiential meaning \citep{khachsich2016}.\footnote{Note also that Dahl's study which included several translations into the same language identified significant intralinguistic variation: the variation between translations into one language is often comparable in extent to that between languages.}

Whether or not current relevance is at the core of the semantics of the European \isi{perfect}, it seems to be the main parameter defining the Perfect and distinguishing it from the Aorist in \ili{Mano}. The context of annulled result (\sectref{khachsectanres}) is a good test for this parameter as it yields strict complementarity: if the result of some action was overruled by some consequent action, the \isi{perfect} construction cannot be used.\footnote{This test, however, is not universally applicable across languages. Events that have been overruled by some other event can still be relevant, as in the ex. 37 of Dahl's questionnaire \citep[803]{khachdahl2000}: "[It is cold in the room. The window is closed.] Question: You OPEN the window (and closed it again)?". Thus, while \ili{Mano} prohibits the usage of the Perfect in this context, in the \ili{Nij} dialect of \ili{Udi} the Perfect is grammatical \citep{khachmaisak2016}.} On the contrary, when a certain past event is explicitly presented as a justification of a current or a future situation, as in example \REF{khachexmoto}, the \isi{perfect} construction is clearly preferred to the Aorist.

A relevant event expressed by the Perfect in \ili{Mano} does not have to be recent. In routine conversation many relevant events are recent: moreover, when a pair of examples with the Aorist and the Perfect are evaluated by native speakers, they tend to analyze the latter as being more recent. Cross-linguistically, however, recency seems to be more of an implicature rather than part of the semantics. Non-recent perfects are very common: experiential \isi{perfect} is typically not recent. In general, Dahl and Hedin \citep{khachdahlhedin2000} analyze the "hot news" value as a late semantic development of perfects.

Perfects cross-linguistically often show important restrictions in combinations with temporal adverbs, as temporal specification "somehow detracts from the focusing on the  result {\ldots} perhaps by transferring the attention to the time of the past event" \citep[395]{khachdahlhedin2000}. This is also the case for \ili{Mano} (see \sectref{khachsectempadv}). Compatibility with temporal adverbs, however, is also very idiosyncratic: some perfects combine freely with temporal adverbs of any kind \citep{khachmaisak2016}. 

Another function of the Perfect in \ili{Mano} is marking the coda of a narrative (\sectref{khachcodanarr}). For William \citet[65]{khachlabov2001}, the function of the coda is indicating the “termination of the narrative by returning the time frame to the present”. The narrative can be put back into relationship with the present by dissociating the narrative time and the present time. This is the strategy used in \ili{Totela}, a \ili{Bantu} language \citep{khachcrane2015}, where the narrative coda is marked by a prehodiernal affix which signals that the situation is excluded from (and is prior to) the temporal domain of "now". The \ili{Mano} strategy is different: it uses the \isi{perfect} construction which shifts the coda sentence from the domain of narrative past and associates it with the present. This is also the strategy used in the \ili{Nij} dialect of \ili{Udi} (Nakh-Dagestanian, \citealt{khachmaisak2016}).%\footnote{Note that this distinction also applies to the opening of the narrative: while \ili{Totela} also uses prehodiernal form, \ili{Nij} dialect of \ili{Udi} uses \isi{perfect} form.}


The strongest cross-linguistically valid definition of \isi{perfect}, surprisingly, is a negative one, namely, the property of \textit{not} being a narrative tense \citep{khachlind2000}. Narrative function is the "anti-prototype" of \isi{perfect}, that is, "a set of uses that are left untouched until the final end of the grammaticalization process by which perfects expand into general pasts", as occurred in spoken \ili{French} \citep[280]{khachdahl2014}. Narrative forms are not always perfective in a language (cf. narrative present), but when they are, their usage in this context can serve as a distinction between a narrative perfective tense (which is often called \isi{aorist}) and a \isi{perfect} tense \citep[cf.][]{khachmaisak2016}. This distinction is strongly supported in \ili{Mano}.

Let us now turn to the transposition and anticipation functions of the Aorist and the Perfect. In the function of anticipation, the basic semantic opposition between the Aorist and the Perfect as constructions expressing "remote" and "recent" events is preserved: as the Aorist typically expresses more remote past events, the predictions framed in it are also expected to occur in the non-immediate future; meanwhile, since the Perfect expresses recent past events, the anticipated event described by the Perfect is seen as close at hand. As for the transposition function, when the reference point is transposed from the moment of enunciation to a certain moment (typically) in the past, again, the choice between the Aorist and the Perfect conforms to exactly the same tendencies as in regular occurrences: whether the \isi{focus} event happened long or not so long before the reference point, whether it was still relevant at the reference point, etc.

\section{Conclusion}
This paper investigates the semantic differences between two perfective constructions in the \ili{Mano} language, the Aorist and the Perfect. The paper uses various sources of data, including Östen Dahl's classic questionnaire, but also spontaneous speech: narratives, routine conversations, and ritual speech, Christian and traditional. The \ili{Mano} Perfect shares with the (much disputed) cross-linguistic prototype the function of expressing event relevance. At the same time, it shares the property of not being a narrative tense, which is a cross-linguistic "anti-prototype" of \isi{perfect} and is in \ili{Mano} reserved to the \isi{aorist} construction, as well as the \isi{consecutive} construction, which remained out of the \isi{scope} of this paper. More interestingly, \ili{Mano} Aorist and Perfect have two secondary functions, namely, anticipation of future events and transposition to the past. It turns out that the secondary functions confirm the basic distinction between the Aorist and the Perfect, the latter maintaining a closer connection with the reference point.


\section*{Acknowledgements}

I am grateful to the speakers of the \ili{Mano} language and inhabitants of the city of Nzérékoré, as well as the villages of Godi, Yei and Bheleton, Guinea. I am especially indebted to Pé Mamy, Cé Simmy, Émile Loua, Aimé Simmy, \textsuperscript{†}Alfred Sandy and \textsuperscript{†}Éli Sandy, for teaching me the language and for being always available and always welcoming. I am grateful to William F. Hanks and to Valentin Vydrin for the comments on various aspects of this paper. I am thankful to the anonymous reviewer for helpful revisions and suggestions. The research, including the fieldwork, was supported by the Fyssen Foundation and by the LLACAN, CNRS, France.

\section*{Abbreviations}


\begin{tabularx}{.45\textwidth}{>{\scshape}lQ}
1 & 1st person \\
2 & 2nd person   \\
3 &  3rd person  \\
conj &  conjunctive  \\
cop &   \isi{copula} \\
cs &  construct state  \\
dem &  demonstrative  \\
ger &  gerund  \\
indef &  indefinite  \\
ipfv &  imperfect  \\
jnt &  \isi{conjoint}  \\
\end{tabularx}
\begin{tabularx}{.45\textwidth}{>{\scshape}lQ}
neg &  negative  \\
pl &  plural  \\
poss &  possessive  \\
prf &  \isi{perfect}  \\
pst &  past  \\
ref &  referential  \\
refl &  reflexive  \\
rel &   relative \\
sg &  singular  \\
top &   topic \\
\end{tabularx}

\sloppy
\printbibliography[heading=subbibliography,notkeyword=this]

\end{document}