\documentclass[output=paper,newtxmath,modfonts,nonflat,hidelinks]{langsci/langscibook} 
\ChapterDOI{10.5281/zenodo.3367177}
\author{Meredith Landman \affiliation{Pomona College}}
\title{Nominal quantification in Kipsigis} 

\abstract{In this paper, I examine the syntax and semantics of nominal quantification in Kipsigis, a Nilotic language spoken in western Kenya. I present a compositional analysis of quantificational nominals and discuss how the Kipsigis patterns relate to previous crosslinguistic work on quantification.}


\IfFileExists{../localcommands.tex}{%hack to check whether this is being compiled as part of a collection or standalone
  \usepackage{pifont}
\usepackage{savesym}

\savesymbol{downingtriple}
\savesymbol{downingdouble}
\savesymbol{downingquad}
\savesymbol{downingquint}
\savesymbol{suph}
\savesymbol{supj}
\savesymbol{supw}
\savesymbol{sups}
\savesymbol{ts}
\savesymbol{tS}
\savesymbol{devi}
\savesymbol{devu}
\savesymbol{devy}
\savesymbol{deva}
\savesymbol{N}
\savesymbol{Z}
\savesymbol{circled}
\savesymbol{sem}
\savesymbol{row}
\savesymbol{tipa}
\savesymbol{tableauxcounter}
\savesymbol{tabhead}
\savesymbol{inp}
\savesymbol{inpno}
\savesymbol{g}
\savesymbol{hanl}
\savesymbol{hanr}
\savesymbol{kuku}
\savesymbol{ip}
\savesymbol{lipm}
\savesymbol{ripm}
\savesymbol{lipn}
\savesymbol{ripn} 
% \usepackage{amsmath} 
% \usepackage{multicol}
\usepackage{qtree} 
\usepackage{tikz-qtree,tikz-qtree-compat}
% \usepackage{tikz}
\usepackage{upgreek}


%%%%%%%%%%%%%%%%%%%%%%%%%%%%%%%%%%%%%%%%%%%%%%%%%%%%
%%%                                              %%%
%%%           Examples                           %%%
%%%                                              %%%
%%%%%%%%%%%%%%%%%%%%%%%%%%%%%%%%%%%%%%%%%%%%%%%%%%%%
% remove the percentage signs in the following lines
% if your book makes use of linguistic examples
\usepackage{tipa}  
\usepackage{pstricks,pst-xkey,pst-asr}

%for sande et al
\usepackage{pst-jtree}
\usepackage{pst-node}
%\usepackage{savesym}


% \usepackage{subcaption}
\usepackage{multirow}  
\usepackage{./langsci/styles/langsci-optional} 
\usepackage{./langsci/styles/langsci-lgr} 
\usepackage{./langsci/styles/langsci-glyphs} 
\usepackage[normalem]{ulem}
%% if you want the source line of examples to be in italics, uncomment the following line
% \def\exfont{\it}
\usetikzlibrary{arrows.meta,topaths,trees}
\usepackage[linguistics]{forest}
\forestset{
	fairly nice empty nodes/.style={
		delay={where content={}{shape=coordinate,for parent={
					for children={anchor=north}}}{}}
}}
\usepackage{soul}
\usepackage{arydshln}
% \usepackage{subfloat}
\usepackage{langsci/styles/langsci-gb4e} 
   
% \usepackage{linguex}
\usepackage{vowel}

\usepackage{pifont}% http://ctan.org/pkg/pifont
\newcommand{\cmark}{\ding{51}}%
\newcommand{\xmark}{\ding{55}}%
 
 
 %Lamont
 \makeatletter
\g@addto@macro\@floatboxreset\centering
\makeatother

\usepackage{newfloat} 
\DeclareFloatingEnvironment[fileext=tbx,name=Tableau]{tableau}
  %add all your local new commands to this file
\newcommand{\downingquad}[4]{\parbox{2.5cm}{#1}\parbox{3.5cm}{#2}\parbox{2.5cm}{#3}\parbox{3.5cm}{#4}}
\newcommand{\downingtriple}[3]{\parbox{4.5cm}{#1}\parbox{3cm}{#2}\parbox{3cm}{#3}}
\newcommand{\downingdouble}[2]{\parbox{4.5cm}{#1}\parbox{6cm}{#2}}
\newcommand{\downingquint}[5]{\parbox{1.75cm}{#1}\parbox{2.25cm}{#2}\parbox{2cm}{#3}\parbox{3cm}{#4}\parbox{2cm}{#5}}
\newcolumntype{Y}{>{\centering\arraybackslash}X}
\newcolumntype{T}{>{\centering\arraybackslash}m{2cm}}

%commands for Kusmer paper below
\newcommand{\ip}{$\upiota$}
\newcommand{\lipm}{(\_{\ip-Max}}
\newcommand{\ripm}{)\_{\ip-Max}}
\newcommand{\lipn}{(\_{\ip}}
\newcommand{\ripn}{)\_{\ip}}
\renewcommand{\_}[1]{\textsubscript{#1}}


%commands for Pillion paper below
\newcommand{\suph}{\textipa{\super h}}
\newcommand{\supj}{\textipa{\super j}}
\newcommand{\supw}{\textipa{\super w}}
\newcommand{\ts}{\textipa{\t{ts}}}
\newcommand{\tS}{\textipa{\t{tS}}}
\newcommand{\devi}{\textipa{\r*i}}
\newcommand{\devu}{\textipa{\r*u}}
\newcommand{\devy}{\textipa{\r*y}}
\newcommand{\deva}{\textipa{\r*a}}
\renewcommand{\N}{\textipa{N}}
\newcommand{\Z}{\textipa{Z}}
% 

%commands for Diercks paper below
\newcommand{\circled}[1]{\begin{tikzpicture}[baseline=(word.base)]
\node[draw, rounded corners, text height=8pt, text depth=2pt, inner sep=2pt, outer sep=0pt, use as bounding box] (word) {#1};
\end{tikzpicture}
}

%commands for Pesetsky paper below
% \newcommand{\sem}[2][]{\mbox{$[\![ $\textbf{#2}$ ]\!]^{#1}$}}
\newcommand{\sem}[2][]{\mbox{$[[ $\textbf{#2}$ ]]^{#1}$}}

% \newcommand{\ripn}{{\color{red}ripn}}%this is used but never defined. Please update the definition



%commands for Lamont paper below
\newcommand{\row}[4]{
	#1. & 
    /{#2}/ & 
    [{#3}] & 
    `#4' \\ 
}
%\newcounter{tableauxcounter}
\newcommand{\tabhead}[2]{
%     \captionsetup{labelformat=empty}
%     \stepcounter{tableauxcounter}
%     \addtocounter{table}{-1}
% 	\centering
% 	\caption{Tableau \thetableauxcounter: #1}
	\caption{#1}
	\label{#2}
}
\newcommand{\candref}[2]{{(\ref{#1}#2)}}
\newcommand{\tableauref}[1]{{Tableau~\ref{#1}}}
% tableaux
\newcommand{\inp}[1]{\multicolumn{2}{|l||}{{#1}}}
\newcommand{\inpno}[1]{\multicolumn{2}{|l||}{#1}}
\newcommand{\g}{\cellcolor{lightgray}}
\newcommand{\hanl}{\HandLeft}
\newcommand{\hanr}{\HandRight}
\newcommand{\kuku}{Kuk\'{u}}

% \newcommand{\nocaption}[1]{{\color{red} Please provide a caption}}

% \providecommand{\biberror}[1]{{\color{red}#1}}

\definecolor{RED}{cmyk}{0.05,1,0.8,0}


\newfontfamily\amharicfont[Script = Ethiopic, Scale = 1.0]{AbyssinicaSIL}
\newcommand{\amh}[1]{{\amharicfont #1}}

% 
% %Gjersoe
\usepackage{textgreek}
% 
\newcommand{\viol}{\fontfamily{MinionPro-OsF}\selectfont\rotatebox{60}{$\star$}}
\newcommand{\myscalex}{0.45}
\newcommand{\myscaley}{0.65}
%\newcommand{\red}[1]{\textcolor{red}{#1}}
%\newcommand{\blue}[1]{\textcolor{blue}{#1}}
\newcommand{\epen}[1]{\colorbox{jgray}{#1}}
\newcommand{\hand}{{\normalsize \ding{43}}}
\definecolor{jgray}{gray}{0.8} 
\usetikzlibrary{positioning}
\usetikzlibrary{matrix}
\newcommand{\mora}{\textmu\xspace}
\newcommand{\si}{\textsigma\xspace}
\newcommand{\ft}{\textPhi\xspace}
\newcommand{\tone}{\texttau\xspace}
\newcommand{\word}{\textomega\xspace}
% \newcommand{\ts}{\texttslig}
\newcommand{\fns}{\footnotesize}
\newcommand{\ns}{\normalsize}
\newcommand{\vs}{\vspace{1em}}
\newcommand{\bs}{\textbackslash}   % backslash
\newcommand{\cmd}[1]{{\bf \color{red}#1}}   % highlights command
\newcommand{\scell}[2][l]{\begin{tabular}[#1]{@{}c@{}}#2\end{tabular}}
% \interfootnotelinepenalty=10000

% --- Snider Representations --- %

\newcommand{\RepLevelHh}{
\begin{minipage}{0.10\textwidth}
\begin{tikzpicture}[xscale=\myscalex,yscale=\myscaley]
%\node (syl) at (0,0) {Hi};
\node (Rt) at (0,1) {o};
\node (H) at (-0.5,2) {H};
\node (R) at (0.5,3) {h};
%\draw [thick] (syl.north) -- (Rt.south) ;
\draw [thick] (Rt.north) -- (H.south) ;
\draw [thick] (Rt.north) -- (R.south) ;
\end{tikzpicture}
\end{minipage}
}

\newcommand{\RepLevelLh}{
\begin{minipage}{0.10\textwidth}
\begin{tikzpicture}[xscale=\myscalex,yscale=\myscaley]
%\node (syl) at (0,0) {Mid2};
\node (Rt) at (0,1) {o};
\node (H) at (-0.5,2) {L};
\node (R) at (0.5,3) {h};
%\draw [thick] (syl.north) -- (Rt.south) ;
\draw [thick] (Rt.north) -- (H.south) ;
\draw [thick] (Rt.north) -- (R.south) ;
\end{tikzpicture}
\end{minipage}
}

\newcommand{\RepLevelHl}{
\begin{minipage}{0.10\textwidth}
\begin{tikzpicture}[xscale=\myscalex,yscale=\myscaley]
%\node (syl) at (0,0) {Mid1};
\node (Rt) at (0,1) {o};
\node (H) at (-0.5,2) {H};
\node (R) at (0.5,3) {l};
%\draw [thick] (syl.north) -- (Rt.south) ;
\draw [thick] (Rt.north) -- (H.south) ;
\draw [thick] (Rt.north) -- (R.south) ;
\end{tikzpicture}
\end{minipage}
}

\newcommand{\RepLevelLl}{
\begin{minipage}{0.10\textwidth}
\begin{tikzpicture}[xscale=\myscalex,yscale=\myscaley]
%\node (syl) at (0,0) {Lo};
\node (Rt) at (0,1) {o};
\node (H) at (-0.5,2) {L};
\node (R) at (0.5,3) {l};
%\draw [thick] (syl.north) -- (Rt.south) ;
\draw [thick] (Rt.north) -- (H.south) ;
\draw [thick] (Rt.north) -- (R.south) ;
\end{tikzpicture}
\end{minipage}
}

% --- Representations --- %

\newcommand{\RepLevel}{
\begin{minipage}{0.10\textwidth}
\begin{tikzpicture}[xscale=\myscalex,yscale=\myscaley]
\node (syl) at (0,0) {\textsigma};
\node (Rt) at (0,1) {o};
\node (H) at (-0.5,2) {\texttau};
\node (R) at (0.5,3) {\textrho};
\draw [thick] (syl.north) -- (Rt.south) ;
\draw [thick] (Rt.north) -- (H.south) ;
\draw [thick] (Rt.north) -- (R.south) ;
\end{tikzpicture}
\end{minipage}
}

\newcommand{\RepContour}{
\begin{minipage}{0.10\textwidth}
\begin{tikzpicture}[xscale=\myscalex,yscale=\myscaley]
\node (syl) at (0,0) {\textsigma};
\node (Rt) at (0,1) {o};
\node (H) at (-0.5,2) {\texttau};
\node (R) at (0.5,3) {\textrho};
\node (Rt2) at (1.5,1.0) {o};
%\node (H2) at (1.0,2) {$\tau$};
%\node (R2) at (2.0,2.5) {R};
\draw [thick] (syl.north) -- (Rt.south) ;
\draw [thick] (Rt.north) -- (H.south) ;
\draw [thick] (Rt.north) -- (R.south) ;
\draw [thick] (syl.north) -- (Rt2.south) ;
%\draw [thick] (Rt2.north) -- (H2.south) ;
%\draw [thick] (Rt2.north) -- (R2.south) ;
\end{tikzpicture}
\end{minipage}
}


% --- OT constraints --- %

\newcommand{\IllustrationDown}{
\begin{minipage}{0.09\textwidth}
\begin{tikzpicture}[xscale=0.7,yscale=0.45]
\node (reg) at (0,0.75) {{\small \textalpha}};
\node (arrow) at (0,0) {{\fns $\downarrow$}};
\node (Rt) at (0,-0.75) {{\small \textbeta}};
\end{tikzpicture}
\end{minipage}
}

\newcommand{\IllustrationUp}{
\begin{minipage}{0.09\textwidth}
\begin{tikzpicture}[xscale=0.7,yscale=0.45]
\node (reg) at (0,0.75) {{\small \textalpha}};
\node (arrow) at (0,0) {{\fns $\uparrow$}};
\node (Rt) at (0,-0.75) {{\small \textbeta}};
\end{tikzpicture}
\end{minipage}
}

\newcommand{\MaxAB}{
\begin{minipage}{0.09\textwidth}
\begin{tikzpicture}[xscale=0.6,yscale=0.4]
\node (max) at (0,0) {{\small \textsc{Max}}};
\node (reg) at (0.75,0.5) {{\fns \textalpha}};
\node (arrow) at (0.75,0) {{\tiny $\downarrow$}};
\node (Rt) at (0.75,-0.5) {{\fns \textbeta}};
\end{tikzpicture}
\end{minipage}
}

\newcommand{\DepAB}{
\begin{minipage}{0.09\textwidth}
\begin{tikzpicture}[xscale=0.6,yscale=0.4]
\node (max) at (0,0) {{\small \textsc{Dep}}};
\node (reg) at (0.75,0.5) {{\fns \textalpha}};
\node (arrow) at (0.75,0) {{\tiny $\downarrow$}};
\node (Rt) at (0.75,-0.5) {{\fns \textbeta}};
\end{tikzpicture}
\end{minipage}
}

\newcommand{\DepHReg}{
\begin{minipage}{0.055\textwidth}
\begin{tikzpicture}[xscale=0.6,yscale=0.4]
\node (dep) at (0,0) {{\small \textsc{Dep}}};
\node (reg) at (0,-1.0) {{\small h}};
\end{tikzpicture}
\end{minipage}
}

\newcommand{\DepLReg}{
\begin{minipage}{0.055\textwidth}
\begin{tikzpicture}[xscale=0.6,yscale=0.4]
\node (dep) at (0,0) {{\small \textsc{Dep}}};
\node (reg) at (0,-1.0) {{\small l}};
\end{tikzpicture}
\end{minipage}
}

\newcommand{\DepReg}{
\begin{minipage}{0.055\textwidth}
\begin{tikzpicture}[xscale=0.6,yscale=0.4]
\node (dep) at (0,0) {{\small \textsc{Dep}}};
\node (reg) at (0,-1.0) {{\small \textrho}};
\end{tikzpicture}
\end{minipage}
}

\newcommand{\DepTRt}{
\begin{minipage}{0.1\textwidth}
\begin{tikzpicture}[xscale=0.6,yscale=0.4]
\node (dep) at (0,0) {{\small \textsc{Dep}}};
\node (t) at (0.75,0.5) {{\fns \texttau}};
\node (arrow) at (0.75,0) {{\tiny $\downarrow$}};
\node (Rt) at (0.75,-0.5) {{\fns o}};
\end{tikzpicture}
\end{minipage}
}

\newcommand{\MaxRegRt}{
\begin{minipage}{0.1\textwidth}
\begin{tikzpicture}[xscale=0.6,yscale=0.4]
\node (max) at (0,0) {{\small \textsc{Max}}};
\node (arrow) at (0.75,0) {{\tiny $\downarrow$}};
\node (Rt) at (0.75,-0.5) {{\fns o}};
\node (reg) at (0.75,0.5) {{\fns \textrho}};
\end{tikzpicture}
\end{minipage}
}

\newcommand{\RegToneByRt}{
\begin{minipage}{0.06\textwidth}
\begin{tikzpicture}[xscale=0.6,yscale=0.5]
\node[rotate=20] (arrow1) at (-0.15,0) {{\fns $\uparrow$}};
\node[rotate=340] (arrow2) at (0.15,0) {{\fns $\uparrow$}};
\node (Rt) at (0,-0.55) {{\small o}};
\node (reg) at (0.4,0.55) {{\small \textrho}};
\node (tone) at (-0.4,0.55) {{\small \texttau}};
\end{tikzpicture}
\end{minipage}
}

\newcommand{\RegToneBySyl}{
\begin{minipage}{0.06\textwidth}
\begin{tikzpicture}[xscale=0.6,yscale=0.5]
\node[rotate=20] (arrow1) at (-0.15,0) {{\fns $\uparrow$}};
\node[rotate=340] (arrow2) at (0.15,0) {{\fns $\uparrow$}};
\node (Rt) at (0,-0.55) {{\small \textsigma}};
\node (reg) at (0.4,0.55) {{\small \textrho}};
\node (tone) at (-0.4,0.55) {{\small \texttau}};
\end{tikzpicture}
\end{minipage}
}

\newcommand{\DepTone}{
\begin{minipage}{0.055\textwidth}
\begin{tikzpicture}[xscale=0.6,yscale=0.4]
\node (dep) at (0,0) {{\small \textsc{Dep}}};
\node (tone) at (0,-1.0) {{\small \texttau}};
\end{tikzpicture}
\end{minipage}
}

\newcommand{\DepTonalRt}{
\begin{minipage}{0.055\textwidth}
\begin{tikzpicture}[xscale=0.6,yscale=0.4]
\node (dep) at (0,0) {{\small \textsc{Dep}}};
\node (tone) at (0,-1.0) {{\small o}};
\end{tikzpicture}
\end{minipage}
}

\newcommand{\DepL}{
\begin{minipage}{0.055\textwidth}
\begin{tikzpicture}[xscale=0.6,yscale=0.4]
\node (dep) at (0,0) {{\small \textsc{Dep}}};
\node (tone) at (0,-1.0) {{\small L}};
\end{tikzpicture}
\end{minipage}
}

\newcommand{\DepH}{
\begin{minipage}{0.055\textwidth}
\begin{tikzpicture}[xscale=0.6,yscale=0.4]
\node (dep) at (0,0) {{\small \textsc{Dep}}};
\node (tone) at (0,-1.0) {{\small H}};
\end{tikzpicture}
\end{minipage}
}

\newcommand{\NoMultDiff}{{\small *loh}}
\newcommand{\Alt}{{\small \textsc{Alt}}}
\newcommand{\NoSkip}{{\small \scell{\textsc{No}\\\textsc{Skip}}}}


\newcommand{\RegDomRt}{
\begin{minipage}{0.030\textwidth}
\begin{tikzpicture}[xscale=0.6,yscale=0.5]
\node (arrow) at (0,0) {{\fns $\downarrow$}};
\node (Rt) at (0,-0.55) {{\small o}};
\node (reg) at (0,0.55) {{\small \textrho}};
\end{tikzpicture}
\end{minipage}
}

\newcommand{\DepRegRt}{
\begin{minipage}{0.1\textwidth}
\begin{tikzpicture}[xscale=0.6,yscale=0.4]
\node (dep) at (0,0) {{\small \textsc{Dep}}};
\node (arrow) at (0.75,0) {{\tiny $\downarrow$}};
\node (Rt) at (0.75,-0.5) {{\fns o}};
\node (reg) at (0.75,0.5) {{\fns \textrho}};
\end{tikzpicture}
\end{minipage}
}

% unused

\newcommand{\ToneByRt}{
\begin{minipage}{0.05\textwidth}
\begin{tikzpicture}[xscale=0.6,yscale=0.5]
\node (arrow) at (0,0) {{\fns $\uparrow$}};
\node (Rt) at (0,-0.55) {{\small o}};
\node (tone) at (0,0.55) {{\small \texttau}};
\end{tikzpicture}
\end{minipage}
}

\newcommand{\RegByRt}{
\begin{minipage}{0.05\textwidth}
\begin{tikzpicture}[xscale=0.6,yscale=0.5]
\node (arrow) at (0,0) {{\fns $\uparrow$}};
\node (Rt) at (0,-0.55) {{\small o}};
\node (reg) at (0,0.55) {{\small \textrho}};
\end{tikzpicture}
\end{minipage}
}

\newcommand{\ToneDomRt}{
\begin{minipage}{0.05\textwidth}
\begin{tikzpicture}[xscale=0.6,yscale=0.5]
\node (arrow) at (0,0) {{\fns $\downarrow$}};
\node (Rt) at (0,-0.55) {{\small o}};
\node (tone) at (0,0.55) {{\small \texttau}};
\end{tikzpicture}
\end{minipage}
}

% --- OT tableaus --- %

% Sec. 3.2, first tabl.

\newcommand{\OTHLInput}{
\begin{minipage}{0.17\textwidth}
\begin{tikzpicture}[xscale=\myscalex,yscale=\myscaley]
\node (tone) at (2,0) {(= H)};
\node (syl) at (0,0) {\textsigma};
\node (Rt) at (0,1) {o};
\node (H) at (-0.5,2) {H};
\node (R) at (0.5,3) {h};
\node (Rt2) at (1.5,1.0) {o};
%\node (H2) at (1.0,2) {\epen{L}};
\node (R2) at (2.0,3) {\blue{l}};
\draw [thick] (syl.north) -- (Rt.south) ;
\draw [thick] (Rt.north) -- (H.south) ;
\draw [thick] (Rt.north) -- (R.south) ;
\draw [thick] (syl.north) -- (Rt2.south) ;
%\draw [dashed] (Rt2.north) -- (H2.south) ;
%\draw [dashed] (Rt2.north) -- (R2.south) ;
\end{tikzpicture}
\end{minipage}
}

\newcommand{\OTHLWinner}{
\begin{minipage}{0.17\textwidth}
\begin{tikzpicture}[xscale=\myscalex,yscale=\myscaley]
\node (tone) at (2,0) {(= HL)};
\node (syl) at (0,0) {\textsigma};
\node (Rt) at (0,1) {o};
\node (H) at (-0.5,2) {H};
\node (R) at (0.5,3) {h};
\node (Rt2) at (1.5,1.0) {o};
\node (H2) at (1.0,2) {\epen{L}};
\node (R2) at (2.0,3) {\blue{l}};
\draw [thick] (syl.north) -- (Rt.south) ;
\draw [thick] (Rt.north) -- (H.south) ;
\draw [thick] (Rt.north) -- (R.south) ;
\draw [thick] (syl.north) -- (Rt2.south) ;
\draw [dashed] (Rt2.north) -- (H2.south) ;
\draw [dashed] (Rt2.north) -- (R2.south) ;
\end{tikzpicture}
\end{minipage}
}

\newcommand{\OTHLSpreadingHOnly}{
\begin{minipage}{0.17\textwidth}
\begin{tikzpicture}[xscale=\myscalex,yscale=\myscaley]
\node (tone) at (2,0) {(= HM)};
\node (syl) at (0,0) {\textsigma};
\node (Rt) at (0,1) {o};
\node (H) at (-0.5,2) {H};
\node (R) at (0.5,3) {h};
\node (Rt2) at (1.5,1.0) {o};
%\node (H2) at (1.0,2) {\epen{L}};
\node (R2) at (2.0,3) {\blue{l}};
\draw [thick] (syl.north) -- (Rt.south) ;
\draw [thick] (Rt.north) -- (H.south) ;
\draw [thick] (Rt.north) -- (R.south) ;
\draw [thick] (syl.north) -- (Rt2.south) ;
\draw [dashed] (Rt2.north) -- (R2.south) ;
\draw [dashed] (Rt2.north) -- (H.south) ;
\end{tikzpicture}
\end{minipage}
}

\newcommand{\OTHLInsertH}{
\begin{minipage}{0.17\textwidth}
\begin{tikzpicture}[xscale=\myscalex,yscale=\myscaley]
\node (tone) at (2,0) {(= HM)};
\node (syl) at (0,0) {\textsigma};
\node (Rt) at (0,1) {o};
\node (H) at (-0.5,2) {H};
\node (R) at (0.5,3) {h};
\node (Rt2) at (1.5,1.0) {o};
\node (H2) at (1.0,2) {\epen{H}};
\node (R2) at (2.0,3) {\blue{l}};
\draw [thick] (syl.north) -- (Rt.south) ;
\draw [thick] (Rt.north) -- (H.south) ;
\draw [thick] (Rt.north) -- (R.south) ;
\draw [thick] (syl.north) -- (Rt2.south) ;
\draw [dashed] (Rt2.north) -- (H2.south) ;
\draw [dashed] (Rt2.north) -- (R2.south) ;
\end{tikzpicture}
\end{minipage}
}

\newcommand{\OTHLOverwriting}{
\begin{minipage}{0.17\textwidth}
\begin{tikzpicture}[xscale=\myscalex,yscale=\myscaley]
\node (syl) at (0,0) {\textsigma};
\node (Rt) at (0,1) {o};
\node (H) at (-0.5,2) {H};
\node (R) at (0.5,3) {h};
\node (Rt2) at (1.5,1.0) {o};
%\node (H2) at (1.0,2) {\epen{L}};
\node (R2) at (2.0,3) {\blue{l}};
\draw [thick] (syl.north) -- (Rt.south) ;
\draw [thick] (Rt.north) -- (H.south) ;
\draw [thick] (Rt.north) -- (R.south) ;
\draw [thick] (syl.north) -- (Rt2.south) ;
%\draw [dashed] (Rt2.north) -- (H2.south) ;
\draw [dashed] (Rt.north) -- (R2.south) ;
\node (del) at (0.3,1.9) {\textbf{=}};
\end{tikzpicture}
\end{minipage}
}

\newcommand{\OTHLSpreading}{
\begin{minipage}{0.17\textwidth}
\begin{tikzpicture}[xscale=\myscalex,yscale=\myscaley]
\node (syl) at (0,0) {\textsigma};
\node (Rt) at (0,1) {o};
\node (H) at (-0.5,2) {H};
\node (R) at (0.5,3) {h};
\node (Rt2) at (1.5,1.0) {o};
%\node (H2) at (1.0,2) {\epen{L}};
\node (R2) at (2.0,3) {\blue{l}};
\draw [thick] (syl.north) -- (Rt.south) ;
\draw [thick] (Rt.north) -- (H.south) ;
\draw [thick] (Rt.north) -- (R.south) ;
\draw [thick] (syl.north) -- (Rt2.south) ;
%\draw [dashed] (Rt2.north) -- (H2.south) ;
\draw [dashed] (Rt2.north) -- (H.south) ;
\draw [dashed] (Rt2.north) -- (R.south) ;
\end{tikzpicture}
\end{minipage}
}

% Sec. 4.2, second tabl.: phrase-medial position

\newcommand{\OTHnoLInput}{
\begin{minipage}{0.17\textwidth}
\begin{tikzpicture}[xscale=\myscalex,yscale=\myscaley]
\node (tone) at (2,0) {(= H)};
\node (syl) at (0,0) {\textsigma};
\node (Rt) at (0,1) {o};
\node (H) at (-0.5,2) {H};
\node (R) at (0.5,3) {h};
\node (Rt2) at (1.5,1.0) {o};
%\node (H2) at (1.0,2) {\epen{L}};
%\node (R2) at (2.0,3) {\blue{l}};
\draw [thick] (syl.north) -- (Rt.south) ;
\draw [thick] (Rt.north) -- (H.south) ;
\draw [thick] (Rt.north) -- (R.south) ;
\draw [thick] (syl.north) -- (Rt2.south) ;
\end{tikzpicture}
\end{minipage}
}

\newcommand{\OTHnoLEpenth}{
\begin{minipage}{0.17\textwidth}
\begin{tikzpicture}[xscale=\myscalex,yscale=\myscaley]
\node (tone) at (2,0) {(= HM)};
\node (syl) at (0,0) {\textsigma};
\node (Rt) at (0,1) {o};
\node (H) at (-0.5,2) {H};
\node (R) at (0.5,3) {h};
\node (Rt2) at (1.5,1.0) {o};
\node (H2) at (1.0,2) {\epen{L}};
\node (R2) at (2.0,3) {\epen{h}};
\draw [thick] (syl.north) -- (Rt.south) ;
\draw [thick] (Rt.north) -- (H.south) ;
\draw [thick] (Rt.north) -- (R.south) ;
\draw [thick] (syl.north) -- (Rt2.south) ;
\draw [dashed] (Rt2.north) -- (H2.south) ;
\draw [dashed] (Rt2.north) -- (R2.south) ;
\end{tikzpicture}
\end{minipage}
}

\newcommand{\OTHnoLSpreading}{
\begin{minipage}{0.17\textwidth}
\begin{tikzpicture}[xscale=\myscalex,yscale=\myscaley]
\node (tone) at (2,0) {(= HH)};
\node (syl) at (0,0) {\textsigma};
\node (Rt) at (0,1) {o};
\node (H) at (-0.5,2) {H};
\node (R) at (0.5,3) {h};
\node (Rt2) at (1.5,1.0) {o};
%\node (H2) at (1.0,2) {\epen{L}};
%\node (R2) at (2.0,3) {\blue{l}};
\draw [thick] (syl.north) -- (Rt.south) ;
\draw [thick] (Rt.north) -- (H.south) ;
\draw [thick] (Rt.north) -- (R.south) ;
\draw [thick] (syl.north) -- (Rt2.south) ;
\draw [dashed] (Rt2.north) -- (H.south) ;
\draw [dashed] (Rt2.north) -- (R.south) ;
\end{tikzpicture}
\end{minipage}
}

% Sec. 4.2, third tabl., LM is unaffected by L\%

\newcommand{\OTLMInput}{
\begin{minipage}{0.2\textwidth}
\begin{tikzpicture}[xscale=\myscalex,yscale=\myscaley]
\node (tone) at (2,0) {(= LM)};
\node (syl) at (0,0) {\textsigma};
\node (Rt) at (0,1) {o};
\node (H) at (-0.5,2) {L};
\node (R) at (0.5,3) {l};
\node (Rt2) at (1.5,1.0) {o};
\node (H2) at (1.0,2) {L};
\node (R2) at (2.0,3) {h};
\node (R3) at (3.0,3) {\blue{l}};
\draw [thick] (syl.north) -- (Rt.south) ;
\draw [thick] (Rt.north) -- (H.south) ;
\draw [thick] (Rt.north) -- (R.south) ;
\draw [thick] (syl.north) -- (Rt2.south) ;
\draw [thick] (Rt2.north) -- (H2.south) ;
\draw [thick] (Rt2.north) -- (R2.south) ;
\end{tikzpicture}
\end{minipage}
}

\newcommand{\OTLMReplace}{
\begin{minipage}{0.2\textwidth}
\begin{tikzpicture}[xscale=\myscalex,yscale=\myscaley]
\node (tone) at (2,0) {(= LL)};
\node (syl) at (0,0) {\textsigma};
\node (Rt) at (0,1) {o};
\node (H) at (-0.5,2) {L};
\node (R) at (0.5,3) {l};
\node (Rt2) at (1.5,1.0) {o};
\node (H2) at (1.0,2) {L};
\node (R2) at (2.0,3) {h};
\node (R3) at (3.0,3) {\blue{l}};
\draw [thick] (syl.north) -- (Rt.south) ;
\draw [thick] (Rt.north) -- (H.south) ;
\draw [thick] (Rt.north) -- (R.south) ;
\draw [thick] (syl.north) -- (Rt2.south) ;
\draw [thick] (Rt2.north) -- (H2.south) ;
\draw [thick] (Rt2.north) -- (R2.south) ;
\draw [dashed] (Rt2.north) -- (R3.south) ;
\node (del) at (1.8,2.1) {\textbf{=}};
\end{tikzpicture}
\end{minipage}
}

\newcommand{\OTLMTwoReg}{
\begin{minipage}{0.2\textwidth}
\begin{tikzpicture}[xscale=\myscalex,yscale=\myscaley]
\node (tone) at (2,0) {(= LML)};
\node (syl) at (0,0) {\textsigma};
\node (Rt) at (0,1) {o};
\node (H) at (-0.5,2) {L};
\node (R) at (0.5,3) {l};
\node (Rt2) at (1.5,1.0) {o};
\node (H2) at (1.0,2) {L};
\node (R2) at (2.0,3) {h};
\node (R3) at (3.0,3) {\blue{l}};
\draw [thick] (syl.north) -- (Rt.south) ;
\draw [thick] (Rt.north) -- (H.south) ;
\draw [thick] (Rt.north) -- (R.south) ;
\draw [thick] (syl.north) -- (Rt2.south) ;
\draw [thick] (Rt2.north) -- (H2.south) ;
\draw [thick] (Rt2.north) -- (R2.south) ;
\draw [dashed] (Rt2.north) -- (R3.south) ;
\end{tikzpicture}
\end{minipage}
}

% Sec. 4.2, fourth tabl., L is affected by L\% but M is not

\newcommand{\OTLInput}{
\begin{minipage}{0.17\textwidth}
\begin{tikzpicture}[xscale=\myscalex,yscale=\myscaley]
\node (tone) at (2,0) {(= L)};
\node (syl) at (0,0) {\textsigma};
\node (Rt) at (0,1) {o};
\node (H) at (-0.5,2) {L};
\node (R) at (0.5,3) {l};
\node (R2) at (2,3) {\blue{l}};
\draw [thick] (syl.north) -- (Rt.south) ;
\draw [thick] (Rt.north) -- (H.south) ;
\draw [thick] (Rt.north) -- (R.south) ;
\end{tikzpicture}
\end{minipage}
}

\newcommand{\OTLLowered}{
\begin{minipage}{0.17\textwidth}
\begin{tikzpicture}[xscale=\myscalex,yscale=\myscaley]
\node (tone) at (2,0) {(= LL)};
\node (syl) at (0,0) {\textsigma};
\node (Rt) at (0,1) {o};
\node (H) at (-0.5,2) {L};
\node (R) at (0.5,3) {l};
\node (R2) at (2,3) {\blue{l}};
\draw [thick] (syl.north) -- (Rt.south) ;
\draw [thick] (Rt.north) -- (H.south) ;
\draw [thick] (Rt.north) -- (R.south) ;
\draw [dashed] (Rt.north) -- (R2.south) ;
\end{tikzpicture}
\end{minipage}
}

\newcommand{\OTMInput}{
\begin{minipage}{0.17\textwidth}
\begin{tikzpicture}[xscale=\myscalex,yscale=\myscaley]
\node (tone) at (2,0) {(= M)};
\node (syl) at (0,0) {\textsigma};
\node (Rt) at (0,1) {o};
\node (H) at (-0.5,2) {L};
\node (R) at (0.5,3) {h};
\node (R2) at (2,3) {\blue{l}};
\draw [thick] (syl.north) -- (Rt.south) ;
\draw [thick] (Rt.north) -- (H.south) ;
\draw [thick] (Rt.north) -- (R.south) ;
\end{tikzpicture}
\end{minipage}
}

\newcommand{\OTMLowered}{
\begin{minipage}{0.17\textwidth}
\begin{tikzpicture}[xscale=\myscalex,yscale=\myscaley]
\node (tone) at (2,0) {(= ML)};
\node (syl) at (0,0) {\textsigma};
\node (Rt) at (0,1) {o};
\node (H) at (-0.5,2) {L};
\node (R) at (0.5,3) {h};
\node (R2) at (2,3) {\blue{l}};
\draw [thick] (syl.north) -- (Rt.south) ;
\draw [thick] (Rt.north) -- (H.south) ;
\draw [thick] (Rt.north) -- (R.south) ;
\draw [dashed] (Rt.north) -- (R2.south) ;
\end{tikzpicture}
\end{minipage}
}

% Sec. 4.2, fifth tableau, polar questions with level tones

\newcommand{\OTLPolIn}{
\begin{minipage}{0.20\textwidth}
\begin{tikzpicture}[xscale=\myscalex-0.05,yscale=\myscaley-0.05]
\node (tone) at (3.5,0) {(= L)};
\node (syl) at (0,0) {\textsigma};
\node (syl2) at (2,0) {\red{\textsigma}};
\node (Rt) at (0,1) {o};
\node (H) at (-0.5,2) {L};
\node (R) at (0.5,3) {l};
\node (Rt2) at (2,1) {\red{o}};
\draw [thick] (syl.north) -- (Rt.south) ;
\draw [thick,red] (syl2.north) -- (Rt2.south) ;
\draw [thick] (Rt.north) -- (H.south) ;
\draw [thick] (Rt.north) -- (R.south) ;
\end{tikzpicture}
\end{minipage}
}

\newcommand{\OTLPolDef}{
\begin{minipage}{0.20\textwidth}
\begin{tikzpicture}[xscale=\myscalex-0.05,yscale=\myscaley-0.05]
\node (tone) at (3.5,0) {(= L.M)};
\node (syl) at (0,0) {\textsigma};
\node (syl2) at (2,0) {\red{\textsigma}};
\node (Rt) at (0,1) {o};
\node (H) at (-0.5,2) {L};
\node (R) at (0.5,3) {l};
\node (H2) at (1.5,2) {\epen{L}};
\node (R2) at (2.5,3) {\epen{h}};
\node (Rt2) at (2,1) {\red{o}};
\draw [thick] (syl.north) -- (Rt.south) ;
\draw [thick,red] (syl2.north) -- (Rt2.south) ;
\draw [thick] (Rt.north) -- (H.south) ;
\draw [thick] (Rt.north) -- (R.south) ;
\draw [semithick,dashed] (Rt2.north) -- (H2.south) ;
\draw [semithick,dashed] (Rt2.north) -- (R2.south) ;
\end{tikzpicture}
\end{minipage}
}

\newcommand{\OTLPolAlt}{
\begin{minipage}{0.20\textwidth}
\begin{tikzpicture}[xscale=\myscalex-0.05,yscale=\myscaley-0.05]
\node (tone) at (3.5,0) {(= L.L)};
\node (syl) at (0,0) {\textsigma};
\node (syl2) at (2,0) {\red{\textsigma}};
\node (Rt) at (0,1) {o};
\node (H) at (-0.5,2) {L};
\node (R) at (0.5,3) {l};
\node (Rt2) at (2,1) {\red{o}};
\draw [thick] (syl.north) -- (Rt.south) ;
\draw [thick,red] (syl2.north) -- (Rt2.south) ;
\draw [thick] (Rt.north) -- (H.south) ;
\draw [thick] (Rt.north) -- (R.south) ;
\draw [semithick,dashed] (Rt2.north) -- (H.south) ;
\draw [semithick,dashed] (Rt2.north) -- (R.south) ;
\end{tikzpicture}
\end{minipage}
}

% Sec. 4.2, sixth tableau, polar questions with contour tones

\newcommand{\OTLLPolIn}{
\begin{minipage}{0.23\textwidth}
\begin{tikzpicture}[xscale=\myscalex-0.05,yscale=\myscaley-0.05]
\node (tone) at (5.2,0) {(= L)};
\node (syl) at (0,0) {\textsigma};
\node (syl3) at (3.4,0) {\red{\textsigma}};
\node (Rt) at (0,1) {o};
\node (Rt2) at (1.7,1) {o};
\node (Rt3) at (3.4,1) {\red{o}};
\node (H) at (-0.5,2) {L};
\node (R) at (0.5,3) {l};
\draw [thick] (syl.north) -- (Rt.south) ;
\draw [thick] (syl.north) -- (Rt2.south) ;
\draw [thick,red] (syl3.north) -- (Rt3.south) ;
\draw [thick] (Rt.north) -- (H.south) ;
\draw [thick] (Rt.north) -- (R.south) ;
\end{tikzpicture}
\end{minipage}
}

\newcommand{\OTLLPolDef}{
\begin{minipage}{0.23\textwidth}
\begin{tikzpicture}[xscale=\myscalex-0.05,yscale=\myscaley-0.05]
\node (tone) at (5.2,0) {(= L.M)};
\node (syl) at (0,0) {\textsigma};
\node (syl3) at (3.4,0) {\red{\textsigma}};
\node (Rt) at (0,1) {o};
\node (Rt2) at (1.7,1) {o};
\node (Rt3) at (3.4,1) {\red{o}};
\node (H) at (-0.5,2) {L};
\node (R) at (0.5,3) {l};
\node (H3) at (2.9,2) {\epen{L}};
\node (R3) at (3.9,3) {\epen{h}};
\draw [thick] (syl.north) -- (Rt.south) ;
\draw [thick] (syl.north) -- (Rt2.south) ;
\draw [thick,red] (syl3.north) -- (Rt3.south) ;
\draw [thick] (Rt.north) -- (H.south) ;
\draw [thick] (Rt.north) -- (R.south) ;
\draw [dashed] (Rt3.north) -- (H3.south) ;
\draw [dashed] (Rt3.north) -- (R3.south) ;
\end{tikzpicture}
\end{minipage}
}

\newcommand{\OTLLPolSkip}{
\begin{minipage}{0.23\textwidth}
\begin{tikzpicture}[xscale=\myscalex-0.05,yscale=\myscaley-0.05]
\node (tone) at (5.2,0) {(= L.L)};
\node (syl) at (0,0) {\textsigma};
\node (syl3) at (3.4,0) {\red{\textsigma}};
\node (Rt) at (0,1) {o};
\node (Rt2) at (1.7,1) {o};
\node (Rt3) at (3.4,1) {\red{o}};
\node (H) at (-0.5,2) {L};
\node (R) at (0.5,3) {l};
\draw [thick] (syl.north) -- (Rt.south) ;
\draw [thick] (syl.north) -- (Rt2.south) ;
\draw [thick,red] (syl3.north) -- (Rt3.south) ;
\draw [thick] (Rt.north) -- (H.south) ;
\draw [thick] (Rt.north) -- (R.south) ;
\draw [dashed] (Rt3.north) -- (H.south) ;
\draw [dashed] (Rt3.north) -- (R.south) ;
\end{tikzpicture}
\end{minipage}
}  
  
\newcommand{\ilit}[1]{#1\il{#1}}    
\newcommand{\isit}[1]{#1\is{#1}}  

\makeatletter
\let\thetitle\@title
\let\theauthor\@author 
\makeatother

\newcommand{\togglepaper}[1][0]{ 
  \bibliography{../localbibliography}
  %% hyphenation points for line breaks
%% Normally, automatic hyphenation in LaTeX is very good
%% If a word is mis-hyphenated, add it to this file
%%
%% add information to TeX file before \begin{document} with:
%% %% hyphenation points for line breaks
%% Normally, automatic hyphenation in LaTeX is very good
%% If a word is mis-hyphenated, add it to this file
%%
%% add information to TeX file before \begin{document} with:
%% \include{localhyphenation}
\hyphenation{
affri-ca-te
affri-ca-tes
com-ple-ments
par-a-digm
Sha-ron
Kings-ton
phe-nom-e-non
Daul-ton
Abu-ba-ka-ri
Ngo-nya-ni
Clem-ents 
King-ston
Tru-cken-brodt
Tab-leau
cophono-logies
mark-edness
Ti-gri-nya
a-mong
Car-stens
Lu-bu-ku-su
}
\hyphenation{
affri-ca-te
affri-ca-tes
com-ple-ments
par-a-digm
Sha-ron
Kings-ton
phe-nom-e-non
Daul-ton
Abu-ba-ka-ri
Ngo-nya-ni
Clem-ents 
King-ston
Tru-cken-brodt
Tab-leau
cophono-logies
mark-edness
Ti-gri-nya
a-mong
Car-stens
Lu-bu-ku-su
}
  \papernote{\scriptsize\normalfont
    \theauthor.
    \thetitle. 
    To appear in: 
    Emily Clem,   Peter Jenks \& Hannah Sande.
    Theory and description in African Linguistics: Selected papers from the 47th Annual Conference on African Linguistics.
    Berlin: Language Science Press. [preliminary page numbering]
  }
  \pagenumbering{roman}
  \setcounter{chapter}{#1}
  \addtocounter{chapter}{-1}
}

\newcommand{\upstep}{\textupstep}


% \newcounter{tableauxcounter}

\renewcommand{\textltailn}{ɲ}
\renewcommand{\textbardotlessj}{ɟ}

\newcommand{\emphkh}[1]{\textit{#1}} %originally \textbf, banned by the guidelines



\definecolor{lsDOIGray}{cmyk}{0,0,0,0.45}


\newcommand{\xuparrow}[1]{%
  {\left\uparrow\vbox to #1{}\right.\kern-\nulldelimiterspace}
}
\renewcommand \textupstep[1]{\char"A71B#1}
\renewcommand \textdownstep[1]{\char"A71C#1}
 
 \newcommand{\ꜛ}{\textsf{ꜛ}}
 
\def\biberror{\undefined}


\newcommand{\OTbox}[1]{\resizebox{.88\textwidth}{!}{#1}}
 
  %% hyphenation points for line breaks
%% Normally, automatic hyphenation in LaTeX is very good
%% If a word is mis-hyphenated, add it to this file
%%
%% add information to TeX file before \begin{document} with:
%% %% hyphenation points for line breaks
%% Normally, automatic hyphenation in LaTeX is very good
%% If a word is mis-hyphenated, add it to this file
%%
%% add information to TeX file before \begin{document} with:
%% %% hyphenation points for line breaks
%% Normally, automatic hyphenation in LaTeX is very good
%% If a word is mis-hyphenated, add it to this file
%%
%% add information to TeX file before \begin{document} with:
%% \include{localhyphenation}
\hyphenation{
affri-ca-te
affri-ca-tes
com-ple-ments
par-a-digm
Sha-ron
Kings-ton
phe-nom-e-non
Daul-ton
Abu-ba-ka-ri
Ngo-nya-ni
Clem-ents 
King-ston
Tru-cken-brodt
Tab-leau
cophono-logies
mark-edness
Ti-gri-nya
a-mong
Car-stens
Lu-bu-ku-su
}
\hyphenation{
affri-ca-te
affri-ca-tes
com-ple-ments
par-a-digm
Sha-ron
Kings-ton
phe-nom-e-non
Daul-ton
Abu-ba-ka-ri
Ngo-nya-ni
Clem-ents 
King-ston
Tru-cken-brodt
Tab-leau
cophono-logies
mark-edness
Ti-gri-nya
a-mong
Car-stens
Lu-bu-ku-su
}
\hyphenation{
affri-ca-te
affri-ca-tes
com-ple-ments
par-a-digm
Sha-ron
Kings-ton
phe-nom-e-non
Daul-ton
Abu-ba-ka-ri
Ngo-nya-ni
Clem-ents 
King-ston
Tru-cken-brodt
Tab-leau
cophono-logies
mark-edness
Ti-gri-nya
a-mong
Car-stens
Lu-bu-ku-su
} 
  \togglepaper[25]
}{}
 


\begin{document}
\maketitle

\section{Introduction}\label{sec:landman:introduction}

In this paper, I examine the syntax and semantics of nominal \isi{quantification} in \ili{Kipsigis}, a \ili{Nilotic} language spoken by roughly 2 million people in western Kenya. I \isi{focus} on nominals that contain the \isi{universal quantifier} \textit{tugul}, as in \REF{ex:landman:1}:\footnote{All data are from my own field notes collected through elicitation interviews with Robert Kipkemoi Langat, a native \ili{Kipsigis} speaker in his early 20s.}

\ea \label{ex:landman:1}
	\gll ru-e lagok tugul\\
       	 sleep-\textsc{prs} child.\textsc{pl} all\\
  \glt ‘All the children are sleeping.’
\z

\noindent Such nominals pose a compositional puzzle, as although \textit{tugul} may combine with a plural noun, as in \REF{ex:landman:1}, \textit{tugul} may not combine with a singular noun unless the morpheme \textit{age} is also present, in which case the resulting interpretation is ‘every, any’, as in \REF{ex:landman:2a}; \textit{age} on its own translates as ‘some, (an)other’, as in \REF{ex:landman:2b}.\footnote{For brevity, I gloss \textit{age} as ‘some’ throughout.}

\ea
  \ea \label{ex:landman:2a}
    \gll ru-e lakwet *(age) tugul\\
       sleep-\textsc{prs} child.\textsc{sg} *(some.\textsc{sg}) all\\
    \glt ‘Every child is sleeping.’
  \ex \label{ex:landman:2b}
    \gll ru-e lakwet age\\
       	 sleep-\textsc{prs} child.\textsc{sg} some.\textsc{sg}\\
    \glt ‘Some/another child is sleeping.’
  \z
\z

\noindent This pattern raises two analytical questions. First, what semantic (and syntactic) contribution does \textit{age} make, to allow \textit{tugul} to attach to a singular nominal? Second, how is the resulting universal interpretation compositionally derived, given that \textit{age} on its own means ‘some, (an)other’?

I will motivate an account of this pattern according to which the quantifier \textit{tugul} heads a QP and is sister to an individual-denoting DP, i.e., a DP of type \textit{e} (as \citealt{Matthewson:2001} argues for quantificational nominals in \ili{Lillooet Salish}):

\ea \label{ex:landman:4first}
\begin{forest}
[QP[DP\textsubscript{e}] [Q[tugul]]]\end{forest}
\z

\noindent Further, \textit{age} is an indefinite determiner that denotes a variable over Skolemized choice functions (as in \citealt{Kratzer:1998}; see also \citealt{Reinhart:1997}; \citealt{Winter:1997}; \citealt{Matthewson:1999,Matthewson:2001}; among many others); \textit{age} thus attaches to an NP of type <e, t> and yields a DP of type \textit{e}, in effect creating a suitable argument for \textit{tugul} and restricting its domain (as in \posscitet{Matthewson:2001} analysis of \ili{Salish}):\footnote{I thank the anonymous reviewers for suggesting an analysis of \textit{age} along these lines.}

\ea \label{ex:landman:4second}
\begin{forest}
[QP[DP\textsubscript{e} [NP\textsubscript{<e, t>}] [D [age]]][Q[tugul]]]\end{forest}
\z

\newpage 
\noindent Singular nouns on their own are of the basic predicative type <e, t> and so cannot serve as arguments to \textit{tugul}.\footnote{As I will show in \sectref{sec:landman:barenouns}, bare singular nouns appear in argument positions, where they permit definite interpretations; because definite singulars are standardly taken to denote individuals, they may incorrectly be expected to occur with \textit{tugul}. I address this point in \sectref{sec:landman:qnominals}.} 

This paper thus contributes to the growing body of work on \isi{quantification} in African languages, as well as across languages more generally, by (a) providing a description of the structure and interpretation of nominal \isi{quantification} in \ili{Kipsigis}, which to my knowledge has not previously been published; (b) presenting a compositional analysis of those structures; and (c) discussing how the \ili{Kipsigis} patterns relate to previous crosslinguistic work on \isi{quantification}. 

The remainder of this paper is organized as follows. In \sectref{sec:landman:background}, I provide relevant background on the structure of \ili{Kipsigis}. In \sectref{sec:landman:barenouns}, I discuss the syntax and semantics of bare nouns, and in \sectref{sec:landman:qnominals}, I present a compositional account of quantificational nominals. Finally, \sectref{sec:landman:conclusion} concludes the paper. 

\section{Background on Kipsigis}\label{sec:landman:background} 

The basic word order of \ili{Kipsigis} is verb initial, with both VSO and VOS occurring as possible variants:\footnote{\ili{Kipsigis} nominals are case-marked by \isi{tone}, where subjects bear a lower \isi{tone} than their nonsubject counterparts (\citealt{Jake:1979}; see also \citealt{Creider:1989,Creider:2003} for the closely related dialect \ili{Nandi}). I leave out \isi{tone} in my transcriptions here.}\textsuperscript{,}\footnote{See \citealt{Diercks:2016b} for a description and analysis of \ili{Kipsigis} word order.} 

\ea
  \settowidth\jamwidth{(VSO)}  
  \ea \gll	ko-e Kiprono peek\\
			\textsc{pst}-drink Kiprono water\\ \jambox{(VSO)}
      \glt ‘Kiprono drank water.’\label{ex:landman:VSO}
  \ex \gll ko-e peek Kiprono\\
           \textsc{pst}-drink water Kiprono\\ \jambox{(VOS)}
      \glt ‘Kiprono drank water.’\label{ex:landman:VOS}
  \z
\z
 
Within nominals, the head noun appears first. Nouns are inflected for number, and demonstratives \REF{ex:landman:6a} and possessives \REF{ex:landman:6b} appear as suffixes on the head noun:

\ea
  \settowidth\jamwidth{(demonstrative)}  
  \ea \label{ex:landman:6a}
    \gll ko-ibut lakwa-ni \\
      \textsc{pst}-fall child.\textsc{sg}-this\\                 \jambox{(demonstrative)}
      \glt ‘This child fell.’
  \ex\label{ex:landman:6b}
    \gll ko-ibut lakwe-nyin \\
      \textsc{pst}-fall child.\textsc{sg}-her\\                   \jambox{(possessive)}
      \glt ‘Her child fell.’
  \z
\z 

\noindent Adnominal modifiers must follow the head noun, as \REF{ex:landman:7a} shows for various types of modifiers (viz., a quantifier, numeral, possessive phrase, and \isi{relative clause}): 

%\ea
\ea \label{ex:landman:7a}
  \gll ru-e lagok somog-u ap Kiprono tugul ne-mingen\\
       sleep-\textsc{prs} child.\textsc{pl} three-\textsc{nom} of Kiprono all \textsc{rel}-small\\
  \glt ‘All three of Kiprono's children that are small are sleeping.’
%\ex[*]{\label{ex:landman:7b}
%  \gll ru-e somog-u ap Kiprono tugul ne-mingen lagok\\
%       sleep-\textsc{prs}  three-\textsc{nom} of Kiprono all %\textsc{rel}-small child.\textsc{pl}\\}
\z
%\z

\noindent Postnominal word order is highly flexible, so that the modifiers in \REF{ex:landman:7a}, for example, may occur in any order with respect to one another.

\section{Bare nouns} 
\label{sec:landman:barenouns}

This section discusses the syntax and semantics of bare nouns in \ili{Kipsigis}; this is a necessary step in understanding the composition of quantificational nominals, because bare nouns serve as building blocks for them. I look at the various interpretations of bare nouns in \sectref{sec:landman:barenounsinterpretations} and discuss the semantic contribution of number in \sectref{sec:landman:number}. 

\subsection{Indefinite, definite, and generic interpretations}
\label{sec:landman:barenounsinterpretations}
Bare nouns (both singular and plural) appear in argument positions, where they permit indefinite, definite, and generic interpretations.

There is a long-standing debate regarding how to semantically characterize \isi{definiteness} (see, among many others, \citealt{Frege:1892}, \citealt{Russell:1905}, \citealt{Heim:1982}, and \citealt{schwarz:2009}). I will assume here that definites have two characteristic properties: (a) they are felicitious only in contexts in which their referents are both familiar and unique, and (b) they are scopeless with respect to quantifiers (such as \isi{negation}). \isi{Indefinites}, in contrast, are felicitous in novel, nonunique contexts, and can interact scopally with other \isi{quantifiers}.

With respect to these properties, bare nouns in \ili{Kipsigis} allow both definite and indefinite interpretations.\footnote{For reasons of space, I omit examples with bare plurals in \REF{ex:landman:novelunique} though \REF{ex:landman:scopesg}; however, the patterns observed for bare singulars in these examples also hold for bare plurals.} Bare nouns are felicitous in both novel and familiar contexts:\footnote{The examples in \REF{ex:landman:novelunique} and \REF{ex:landman:uniqueness} are modeled after the tests for bare nouns in \citet{Gillon:2015}.}

\ea \label{ex:landman:novelunique} 
  \settowidth \jamwidth{(nonunique)}  
  \ea \label{ex:landman:novel}
     \gll enkeny-ko ki-mi kirowgindet\\
          long-ago there-was chief.\textsc{sg}\\ \jambox{(novel)}
     \glt ‘Long ago there was a chief.’
  \ex \label{ex:landman:familiar}
     \gll ki-chamat kirwogindet piik\\
          \PASS-like chief.\textsc{sg} person.\textsc{pl}\\ \jambox{(familiar)}
     \glt ‘The chief was liked by the people.’
  \z
\z 

\noindent Bare nouns are also felicitous in both nonunique and unique contexts:

\ea \label{ex:landman:uniqueness}
\settowidth\jamwidth{(non-unique)}  
  \ea \label{ex:landman:nonunique} [Context: There are two identical cups in the cupboard.]\\
    \gll konon kikombet \\
         give.{\IMP} cup.\textsc{sg} \\ \jambox{(nonunique)}
    \glt ‘Give me a cup!’
    \ex \label{ex:landman:unique} [Context: There is just one cat and one dog, and they are fighting.]\\ 
    \gll 
      ko-suger ngokta ak paget agoi ko-labat paget\\
      {\PST}-fight dog.\textsc{sg} and cat.\textsc{sg} until {\PST}-run.away cat.\textsc{sg}\\ \jambox{(unique)}
    \glt ‘The dog and the cat fought until the cat ran away.’
  \z
\z 

Bare nouns also appear in sluicing constructions, again indicating that they permit (existential) indefinite interpretations (see \citealt{Chung:1995} and \citealt{Reinhart:1997}):

\ea \label{ex:landman:sluicing}
  \gll ko-ger lakwet, kobaten mongen ale ainon\\	
	   {\PST}-see child.\textsc{sg} but \textsc{neg}-know.1\textsc{sg} \textsc{comp} which\\
  \glt ‘She saw a child, but I don't know which.’
\z 

Bare nouns also permit both narrow-\isi{scope} and scopeless interpretations with respect to \isi{negation}. For example, given the context set by \REF{ex:landman:context}, the continuation in \REF{ex:landman:barenounambig} is ambiguous (examples modeled after \citealt{Matthewson:2001}). On one reading, \REF{ex:landman:scopeless}, \textit{kitabut} ‘book.\textsc{sg}’ is scopeless; in this case, \textit{kitabut} corefers with the previously mentioned book (i.e., it is interpreted as a definite). On a second reading, \REF{ex:landman:negexis}, \textit{kitabut} scopes below \isi{negation} (i.e., it is interpreted as a narrow-\isi{scope} existential indefinite). 

\ea \label{ex:landman:context}
  \gll ko-tach Kipto kitabut ak chaik\\	
	   {\PST}-receive Kipto book.\textsc{sg} and tea\\
\glt ‘Kipto received a book and tea.’
\z 
   
\begin{exe}
\ex \label{ex:landman:barenounambig}
  \gll mo-cham kitabut\\	
       \textsc{neg}-like book.\textsc{sg}\\
  \begin{xlist}
  \ex \label{ex:landman:scopeless} ‘She doesn't like the book.’  (scopeless)
  \ex \label{ex:landman:negexis} ‘She doesn't like books.’ (Neg > $\exists$)
  \end{xlist}   
\end{exe}

 In fact, \ili{Kipsigis}, like many other languages, has no nominal expression corresponding to the \ili{English} determiner \textit{no}; instead, nominal \isi{negation} can be expressed using a bare noun in combination with verbal \isi{negation}, further illustrating that bare nouns permit narrow-\isi{scope} existential interpretations:

\ea \label{ex:landman:negbarenoun}
  \gll ma-ibut chita\\	
	   \textsc{neg}-fall person.\textsc{sg}\\
\glt ‘No one fell.’
\z 

\ili{Kipsigis} bare nouns do not, however, permit wide-\isi{scope} existential interpretations (i.e., they are nonspecific indefinites). For example, \REF{ex:landman:scopesg} can only be interpreted as in \REF{ex:landman:narrow}, where the second instance of \textit{chita} ‘person’ scopes below \isi{negation} (my consultant reported \REF{ex:landman:narrow} as \textquotedblleft contradictory\textquotedblright, but as the only interpretation available); in contrast, \REF{ex:landman:wide}, in which the second instance of \textit{chita} ‘person’ scopes above \isi{negation}, is not an available interpretation.

\begin{exe}
\ex \label{ex:landman:scopesg}
  \gll ko-ibut chita ako ma-ibut chita\\	
	   \textsc{pst}-fall person.\textsc{sg} and \textsc{pst}-fall person.\textsc{sg}\\
  \begin{xlist}
  \ex \label{ex:landman:narrow}‘Someone fell and no one fell.’ (Neg > $\exists$)
  \ex \label{ex:landman:wide} *‘Someone fell and someone (else) did not fall.’ (*$\exists$ > Neg)
  \end{xlist}   
\end{exe}

Finally, in addition to definite and nonspecific indefinite interpretations, singular and plural bare nouns can also be interpreted generically:

\ea   
  \ea \label{ex:landman:gensg}
     \gll tinye paget saroriet\\
          have cat.\textsc{sg} tail.\textsc{sg}\\ 
     \glt ‘A cat has a tail.’
  \ex \label{ex:landman:genpl}
     \gll tinye pagok sarurek\\
          have cat.\textsc{pl} tail.\textsc{pl}\\ 
     \glt ‘Cats have tails.’
  \z
\z 

To account for the various (i.e., definite, nonspecific indefinite, and generic) interpretations of bare nouns, I will assume -- as is standard -- that bare nouns have the basic predicative type <e, t>. Different semantic mechanisms (i.e., type shifting rules or modes of composition) then derive their different interpretations. Specifically, to derive nonspecific indefinite interpretations, bare nouns may combine with a transitive verb via predicate restriction (\citealt{Chung:2004}; see also \citealt{Carlson:1977}). To derive definite interpretations, bare nouns may be type-shifted via iota-shift \citep{Partee:1987}. Finally, to yield generic interpretations, bare nouns may be bound by a covert generic operator \citep{Krifka:1995}. 

\subsection{The interpretation of number}
\label{sec:landman:number}
This section provides background on the number interpretation of bare nouns. Plural nouns in \ili{Kipsigis} appear to be number-neutral (i.e., compatible with a singular or plural interpretation; see \citealt{Link:1983} and \citealt{Corbett:2000}), as the question in \REF{ex:landman:barepluralquestion} can be answered with either a singular or plural \REF{ex:landman:answers} (diagnostic from \citealt{Link:1983}):

 \ea \label{ex:landman:15}
  \ea\label{ex:landman:barepluralquestion}
     \gll ko-ger tuga i\\
          \textsc{pst}-see cow.\textsc{pl} \textsc{q}\\ 
     \glt Q: ‘Did he see cows?’
\ex\label{ex:landman:answers}
     \gll ee, ko-ger \{teta agenge / tuga somog\}\\
             yes, \textsc{pst}-see \{cow.\textsc{sg} one / cow.\textsc{pl} three\}\\
      \glt A: ‘Yes, he saw \{one cow/three cows\}.’
\z 
\z

In contrast, singular nouns are not number-neutral, but rather necessarily semantically singular. For example, singular nouns are ungrammatical in combination with numerals greater than one: 

\ea[*]{\label{ex:landman:singulars}
     \gll rue lakwet somog-u\\
            sleep child.\textsc{sg} three-\textsc{nom}\\}
\z

Given these observations, I will adopt a semantic analysis of number in \ili{Kipsigis} as in \citealt{Link:1983}, whereby a singular noun denotes a set of atomic individuals (atoms), and a plural noun denotes a set of both atomic and plural individuals.

\section{Quantificational nominals}
\label{sec:landman:qnominals}

\subsection{The universal quantifier \textit{tugul}}
Returning now to the patterns observed for universally quantified nominals observed in \sectref{sec:landman:introduction}, recall that the quantifier \textit{tugul} expresses universal \isi{quantification}: 

\begin{exe}
  \exr{ex:landman:1}
     \gll ru-e lagok tugul \\
          sleep-\textsc{prs} child.\textsc{pl} all\\ 
     \glt ‘All the children are sleeping.’
\end{exe}

 In the following two subsections, I present a syntax (\sectref{sec:landman:tugulsyn}) and semantics (\sectref{sec:landman:tugulsem}) for \textit{tugul}.

\subsubsection{The syntax of \textit{tugul}}
\label{sec:landman:tugulsyn}
\noindent I adopt the following syntax for \textit{tugul}, in which it heads a QP and is sister to DP:  

\ea \label{ex:landman:tugultree}
%\begin{forest}
[\textsubscript{QP} DP [\textsubscript{Q} tugul]]  %\end{forest}
\z

 Evidence that \textit{tugul} is sister to DP comes from \REF{ex:landman:tugulpronoun}, which shows that \textit{tugul} may combine directly with a \isi{pronoun}; pronouns are standardly taken to be DPs, as they appear on their own in argument positions.

\ea \label{ex:landman:tugulpronoun}
    \gll ko-gitiense echek tugul\\ 
	     {\PST}-sing[1\PL] we all\\ 
    \glt ‘All of us sang.’
\z

 In addition, \textit{tugul} may appear on its own, as long as the reference of the head noun is clear from the context:

\ea \label{ex:landman:tugulonown}
    \gll ko-ger tugul\\ 
	     {\PST}-see all\\
    \glt ‘He saw all.’
\z 

These facts suggests that \textit{tugul} licenses DP ellipsis (in contrast, NP ellipsis appears to be ungrammatical in \ili{Kipsigis}, as I show in \sectref{sec:landman:agesyn}).

\subsubsection{The semantics of \textit{tugul}}\label{sec:landman:tugulsem}
Descriptively, \textit{tugul} is a nondistributive \isi{universal quantifier} (i.e., it permits both distributive and collective interpretations). Consider \REF{ex:landman:distcoll}, for example, which is ambiguous between a distributive and collective reading:
 
\newpage  
\ea \label{ex:landman:distcoll}
    \gll ko-yot bokisinik somok lagok tugul\\
         \textsc{pst}-lift box.\textsc{pl} three child.\textsc{pl} all\\ 
    \glt 
      \begin{xlist}
      \ex \label{ex:landman:dist} ‘The children each lifted three boxes.’ (\textit{distributive})
      \ex \label{ex:landman:coll} ‘The children collectively lifted three boxes.’ (\textit{collective})
      \end{xlist}
\z

The semantics of \textit{tugul} can accordingly be modeled as a function that maps an individual (the denotation of DP) to a generalized quantifier (the denotation of QP; as in \citealt{Matthewson:2001}):\footnote{This formalism comes directly from \citet{Zimmermann:2014}, which is based on \citet{Matthewson:2001}.} 

\ea \label{ex:landman:tugulden}
\textlbrackdbl\textit{tugul}\textrbrackdbl = \textlambda\textit{x}\textsubscript{\textit{e}} .  \textlambda\textit{f}\textsubscript{<\textit{e}, \textit{t}>} . $\forall$\textit{y}[\textit{y}≤\textit{x} $\rightarrow$ \textit{f}(\textit{x})]
\z

\noindent This semantics for \textit{tugul} allows for both distributive and collective interpretations, as the subpart relation (≤) holds for atoms as well as collections. A distributive interpretation results when \textit{tugul} quantifies over atomic subparts of the individual denoted by DP, and a collective interpretation results when there is only one subpart (i.e., \textit{x} = \textit{y}). 

The proposed syntax and semantics for \textit{tugul} explains why \textit{tugul} cannot combine directly with a singular noun, as observed in \sectref{sec:landman:introduction}:

\ea[*]{
\gll ru-e lakwet tugul\\
       sleep-\textsc{prs} child.\textsc{sg} all\\
  \glt ‘Every child is sleeping.’}
\z

\noindent At the NP level, a singular noun has neither the right syntax (it is not a DP) nor semantics (it is not of type \textit{e}) to combine with \textit{tugul}.\footnote{However, as shown in \sectref{sec:landman:barenouns}, bare singulars permit definite interpretations, and so the analysis may incorrectly predict that a bare singular that is type-shifted to a definite could serve as an argument to \textit{tugul}. I will assume that the combination of a definite singular with \textit{tugul} is ruled out on pragmatic grounds: Attaching \textit{tugul} to a definite singular would result in universal \isi{quantification} over a single individual, which is equivalent to the denotation of the definite.}   

\subsection{The morpheme \textit{age}}
As also observed in \sectref{sec:landman:introduction}, \textit{tugul} can combine with a singular nominal just in case the morpheme \textit{age} is also present:

\ea \label{ex:landman:agetugul}
\gll ru-e lakwet *(age) tugul\\
       sleep-\textsc{prs} child.\textsc{sg} *(some.\textsc{sg}) all\\
  \glt ‘Every child is sleeping.’
\z

\noindent The \isi{plural form} of \textit{age}, namely, \textit{alak}, may also occur with \textit{tugul}, in which case \isi{quantification} is over groups (or kinds):

\ea \label{ex:landman:alaktugul}
\gll ru-e lagok alak tugul\\
       sleep-\textsc{prs} child.\textsc{pl} some.\textsc{pl} all\\
  \glt ‘All (or any groups of) children are sleeping.’
\z
 
\noindent Both \textit{age} and \textit{alak} translate as ‘some, (an)other’ when used on their own:\footnote{Because \textit{alak} is simply the \isi{plural form} of \textit{age}, I will henceforth use \textit{age} to refer to both \textit{age} and \textit{alak}, unless otherwise noted.}

\ea \label{ex:agealak}  
\ea
\gll ko-bua lakwet age\\
	   \textsc{pst}-come.by child.\textsc{sg} some.\textsc{sg}\\
  \glt ‘Some/another child came by.’
\ex 
\gll ko-bua lagok alak\\
	   \textsc{pst}-come.by child.\textsc{pl} some.\textsc{pl}\\
  \glt ‘Some/other children came by.’
  \z
\z

This raises the question of what the semantic and syntactic contribution of \textit{age} is, to allow \textit{tugul} to combine with a singular DP and yield a universal (and in some cases free-choice) interpretation. In the following two subsections, I will present evidence that \textit{age} is semantically an indefinite (\sectref{sec:landman:agesem}) and syntactically a determiner (\sectref{sec:landman:agesyn}).

\subsubsection{The semantics of \textit{age}}
\label{sec:landman:agesem}
There are several ways in which \textit{age} behaves semantically like an indefinite (tests for indefiniteness are from \citealt{Matthewson:1999}). First, \textit{age} permits sluicing:

\ea \label{ex:landman:sluicingage}
    \gll ko-ger lakwet age, kobaten mo-ngen ale ainon\\
         \textsc{pst}-see child.\textsc{sg} some.\textsc{sg}, but \textsc{neg}-know[1.\textsc{sg}] \textsc{comp} which\\
    \glt ‘She saw another child, but I do not know which.’
\z

 Second, \textit{age} may introduce new discourse referents:

\ea
  \gll ko-bua chita age\\
       \textsc{pst}-come.by person.\textsc{sg} some.\textsc{sg}\\
  \glt{‘{Some/another} person came by.’}     
\z

\newpage  
 Third, \textit{age} interacts scopally with other quantifiers, such as modals and \isi{negation}. Unlike bare nouns, \textit{age} permits both narrow- and wide-\isi{scope} existential interpretations with respect to \isi{negation}:\footnote{It is possible that the wide-\isi{scope} interpretation here is actually a definite interpretation; see the discussion of definite interpretations for \textit{age} below.}

\begin{exe}
\ex \label{ex:landman:scopepl}
  \gll ko-bua piik alak ako ma-bua piik alak\\	
	   \textsc{pst}-come.by person.\textsc{pl} some.\textsc{pl} and \textsc{neg}-come.by person.\textsc{pl} some.\textsc{pl}\\
  \begin{xlist}
  \ex \label{ex:landman:narrowpl}‘Some people came by and no other people came by.’ (Neg > $\exists$)
  \ex \label{ex:landman:widepl} ‘Some people came by and other people did not come by.’  ($\exists$  > Neg)  
  \end{xlist}   
\end{exe}

 Interestingly, \textit{age} only permits narrow-\isi{scope} interpretations with respect to modals; for example, \REF{ex:landman:29} only permits the narrow-\isi{scope} interpretation in \REF{ex:landman:30} (cf. the wide-\isi{scope} interpretation in \REF{ex:landman:31}).

\begin{exe}
\ex \label{ex:landman:29} [Context: Kipto wants to marry Kiprono.]\\
 \gll moch-e	ko-tun	chepkeleliot age\\	
want-\textsc{prs} \textsc{inf}-marry girl.\textsc{sg} some.\textsc{sg}\\
\begin{xlist}
  \ex \label{ex:landman:30}‘He wants to marry another girl (it doesn't matter who).’
  \ex \label{ex:landman:31} *‘He wants to marry another girl in particular (say Chepto).’ 
  \end{xlist}   
\end{exe}

Summarizing, these examples suggest that \textit{age} is an indefinite that permits both narrow-\isi{scope} (i.e. nonspecific) interpretations and, unlike bare nouns, wide-\isi{scope} existential interpretations (at least with respect to \isi{negation}). It should be noted, however, that there are some uses of \textit{age} that appear to be definite, as the referent of an \textit{age}-DP may be familiar:

\ea \label{ex:landman:modalage} [Context: Two children came by.]\\
\gll angen lakwet agenge, ako m-angen lakwet age\\
     know[\textsc{1.sg}] child.\textsc{sg} one, and \textsc{neg}-know[\textsc{1.sg}] child.\textsc{sg} some.\textsc{sg}\\
\glt ‘I knew one child but I did not know the other child.’ 
\z

\largerpage[-2]
\noindent Such examples may indicate that \textit{age} is not an indefinite determiner, but rather an adnominal modifier positioned within a bare plural that, like other bare plurals, permits indefinite or definite interpretations. What, then, would be the semantic contribution of \textit{age}? As already noted, \textit{age} is associated with free-choice interpretations, as my consultant often offered ‘any’ as a translation for \textit{age} in combination with \textit{tugul}.\footnote{The free-choice interpretation of \textit{age} in combination with \textit{tugul} is made clear in \textit{yes-no} questions. Consider, e.g., \REF{ex:landman:freechoice}, which is ambiguous: This question can ask whether all of the children sang (a universal interpretation) or whether any of the children sang (a free-choice interpretation). In contrast, \textit{tugul} on its own can only be interpreted as a non-free-choice universal, as in \REF{ex:landman:tugulquestion}.

\ea \label{ex:landman:freechoice}  
\gll ko-tien lakwet  age tugul i?\\
	   \textsc{pst}-sing child.\textsc{sg} some.\textsc{sg} all \textsc{q}\\
  \glt ‘Did \{every/any\} child sing?’
\z
\ea \label{ex:landman:tugulquestion}
\gll ko-tien lagok tugul i?\\
	   \textsc{pst}-sing child.\textsc{pl} all \textsc{q}\\
  \glt ‘Did all the children sing?’
\z
} If free-choice interpretations are derived via domain widening (as in \citealt{Kadmon:1993} and \citealt{Kratzer:2002}, among others), then \textit{age} may be widening the domain of the NP it modifies; \textit{tugul} then quantifies over the widened domain. However, an analysis that treats \textit{age} as a modifier within a bare plural would fail to account for the apparent wide-\isi{scope} existential interpretations available to \textit{age}, as in \REF{ex:landman:widepl}, which bare plurals do not permit.\footnote{In addition, an anonymous reviewer points out that the ‘other’ interpretation is a pervasive feature of indefinites across West Chadic (see \citealt{Zimmermann:2008} for \ili{Hausa}, and \citealt{Grubic:2015} for \ili{Ngamo}).} I conclude that \textit{age} encodes indefiniteness (and allows wide-\isi{scope} existential interpretations) and set aside its definite and free-choice interpretations as issues for future research.

As an indefinite, \textit{age} can be analyzed semantically as introducing a variable over Skolemized choice functions (as in \citealt{Kratzer:1998}). A \isi{choice function} is a function that maps an  nonempty set of individuals to a unique individual in that set \citep{Reinhart:1997}. A Skolemized \isi{choice function} has additional \isi{implicit argument}. Thus, \textit{age} (henceforth represented as \textit{age}\textsubscript{\textit{i}}, where the subscript \textit{i} represents its \isi{implicit argument}) maps an individual (its \isi{implicit argument}) to a function from a nonempty set (the denotation of NP) to an individual (the denotation of DP). More specifically (i.e., taking into account the contribution of number), \textit{age} maps a singular NP to an atom, whereas \textit{alak} maps a plural NP to an atomic or plural individual.

\newpage 
\subsubsection{The syntax of \textit{age}}
\label{sec:landman:agesyn}
I adopt the syntax in \REF{ex:landman:agetugultreefirst} for \textit{age}, in which it heads a DP and is sister to NP:

\ea \label{ex:landman:agetugultreefirst}
\begin{forest}
[QP[DP [NP] [D [age\textsubscript{\textit{i}}]]][Q[tugul]]]\end{forest}
\z

  
Evidence that \textit{age} forms a subconstituent with NP within QP (to the exclusion of \textit{tugul}) comes from \REF{ex:landman:agetugulorder}, which shows that \textit{age} must precede \textit{tugul}; other modifiers, such as numerals, may precede or follow \textit{tugul}, \REF{ex:landman:modtugul}.\footnote{Note also that no modifiers may intervene between \textit{age} and \textit{tugul} (e.g., *\textit{lagok alak somogu tugul} lit. ‘child.\textsc{pl} some.\textsc{pl} three all’).
}

\ea \label{ex:landman:agetugulorderwhole} 
\ea[*]{\label{ex:landman:agetugulorder}  
\gll ru-e lakwet tugul age\\
	   \textsc{pst}-come.by child.\textsc{sg} all some.\textsc{sg}\\}
\ex \label{ex:landman:modtugul}
\gll ko-bua lagok \{somog-u  tugul / tugul somog-u\}\\
	   \textsc{pst}-come.by child.\textsc{pl} \{three-\textsc{nom}  all / all three-\textsc{nom}\}\\
  \glt ‘All three children came by.’
  \z
\z

There is also some evidence that \textit{age}, at least when combined with \textit{tugul}, occupies a determiner position. Unlike \textit{tugul}, \textit{age} may not attach to a \isi{pronoun}:\footnote{However, in the absence of \textit{tugul}, \textit{alak}, but not \textit{age}, may attach to a \isi{pronoun}:

\ea{\label{ex:landman:alakpronoun}
    \gll ko-gitiense echek alak \\ 
	     {\PST}-sing[1\PL] we some.\textsc{pl}\\
    \glt ‘Some of us sang.’}    
\z 

 \noindent Because this is a partitive, \textit{alak} may in this case be in a different, higher syntactic position than it is when it appears with \textit{tugul}, permitting it to combine with a \isi{pronoun}.}

\ea[*]{\label{ex:agetugulpronoun}
    \gll ko-gitiense echek \{age/alak\} tugul\\ 
	     {\PST}-sing[1\PL] we \{some.\textsc{sg}/some.\textsc{pl}\} all\\}
\z

 Furthermore, in combination with \textit{tugul}, \textit{age} may not appear on its own, without the head noun:\footnote{However, here too, in the absence of \textit{tugul}, \textit{age} may occur on its own (as long as the reference of the head noun is clear from the context):

\ea\label{ex:agetugulonown}
    \gll ko-ger \{age/alak\}\\ 
	     {\PST}-see \{some.\textsc{sg}/some.\textsc{pl}\} \\
     \glt ‘He saw \{another/others\}.’   
     \z }

\ea[*]{\label{ex:landman:agetugulonown}
    \gll ko-ger \{age/alak\} tugul\\ 
	     {\PST}-see \{some.\textsc{sg}/some.\textsc{pl}\} all\\ } 
\z 

These facts (i.e., that \textit{age} must precede \textit{tugul} and cannot combine with a \isi{pronoun} or occur on its own when combined with \textit{tugul}) are explained if \textit{age} is (a) in a lower position syntactically than \textit{tugul} and (b) a determiner, on the grounds that like the \ili{English} determiners \textit{a} and \textit{the}, \textit{age} cannot license NP ellipsis.

\subsection{The semantic composition of nominals containing \textit{age} and \textit{tugul}}
Having established a syntax and semantics for both \textit{age} and \textit{tugul}, consider again a nominal that contains both:

\ea
  \ea
    \gll lakwet age\textsubscript{\textit{i}} tugul\\
         child.\textsc{sg} some.\textsc{sg} all\\
    \glt ‘every child’
   \ex \label{ex:landman:agetugultreesecond}
     \begin{forest}
[QP[DP [NP [lakwet] ] [D [age\textsubscript{\textit{i}}]]][Q[tugul]]]\end{forest}
  \z
\z

\noindent The \isi{semantic composition} of such nominals would be computed as follows: The quantifier \textit{tugul} binds the \isi{implicit argument} of the \isi{choice function} denoted by \textit{age}. In effect, for any value for the \isi{implicit argument}, the \isi{choice function} output for that argument satisfies the NP predicate. This derives universal \isi{quantification} over atoms in the case that \textit{tugul} attaches to a (singular) \textit{age}-DP, and \isi{quantification} over atomic or plural individuals (i.e., groups) in the case that \textit{tugul} attaches to a (plural) \textit{alak}-DP. 

This semantics thus predicts that when \textit{tugul} combines with an \textit{age}-DP, only a distributive interpretation is possible (because \isi{quantification} occurs over atoms), and, indeed, only distributive interpretations are possible in this case: 

\ea \label{ex:landman:agedist}
  \gll ko-yot bokisinik somok lakwet age tugul\\	 
     \textsc{pst}-lift box.\textsc{pl} three child.\textsc{sg} some.\textsc{sg} all\\
  \glt
      \begin{xlist}
      \ex \label{ex:landman:distage} ‘Each child lifted three boxes.’ (\textit{distributive})
      \ex \label{ex:landman:collage} *‘All the children collectively lifted three boxes.’ (\textit{collective})
      \end{xlist}
\z
 In contrast, when \textit{tugul} combines with an \textit{alak}-DP, \isi{quantification} may occur over atomic or plural individuals, producing distributive or collective interpretations: 

\ea
  \gll ko-yot bokisinik somok lagok alak tugul\\	
       \textsc{pst}-lift box.\textsc{pl} three child.\textsc{pl} some.\textsc{pl} all\\ 
   \glt 
      \begin{xlist}
      \ex \label{ex:landman:distalak} ‘Each child lifted three boxes.’ (\textit{distributive})
      \ex \label{ex:landman:collalak} ‘All (or any groups of) of the children collectively lifted three boxes.’ (\textit{collective})
      \end{xlist}
\z

\subsection{Summary of the analysis}
Summarizing, \textit{tugul} heads a QP and combines with a DP of type \textit{e}. As a result, \textit{tugul} may attach to a \isi{pronoun} and appear on its own (i.e., it licenses DP ellipsis), and may not attach to a predicative singular noun nor, for pragmatic reasons, a singular definite. \textit{Age} is an indefinite determiner that heads a DP and denotes a Skolemized \isi{choice function} that, relative to an \isi{implicit argument}, maps an NP of <e, t> to a DP of type \textit{e}. The resulting \textit{age}-DP may then attach to \textit{tugul}, which binds the \isi{implicit argument} of \textit{age}, resulting in universal \isi{quantification}.

\section{Conclusion} \label{sec:landman:conclusion}
This paper has presented a compositional analysis of quantificational nominals in the \ili{Nilotic} language \ili{Kipsigis}. In short, \textit{tugul} is a nondistributive \isi{universal quantifier} that heads a QP and combines with a DP of type \textit{e} to create a generalized quantifier (as in \posscitet{Matthewson:2001} analysis of \ili{Salish}). The morpheme \textit{age} is an indefinite determiner that denotes a variable over Skolemized choice functions (as in \citealt{Kratzer:1998}; see also \citealt{Matthewson:1999,Matthewson:2001}); \textit{age} thus combines with a predicative NP to create a DP of type \textit{e}, and this DP can combine with \textit{tugul}. Future research may shed light on the free-choice and definite interpretations observed for \textit{age}, which remain open questions here.

\section*{Abbreviations}
\begin{tabularx}{.45\textwidth}{lQ}
\textsc{pst} & past \\
\textsc{prs} &present \\
\textsc{neg} &{negation}\\
\textsc{nom} &{nominative}\\
\end{tabularx}
\begin{tabularx}{.45\textwidth}{lQ}
\textsc{sg}  &singular\\
\textsc{pl}  &plural\\
\textsc{comp}& complementizer\\
\textsc{q}   & question marker\\
\end{tabularx}

\medskip

\noindent
 In the orthographic conventions used here, \textit{ch} represents a voiceless palatal affricate [\textteshlig], \textit{ny} a palatal nasal [\textltailn], \textit{ng} a \isi{velar nasal} [ŋ], and \textit{y} a palatal glide [j].   

\section*{Acknowledgments}
I am very grateful to my \ili{Kipsigis} consultant, Robert Kipkemoi Langat, for his diligent work on this project. Many thanks also to the two anonymous reviewers, who provided extensive feedback on a previous draft of this paper. Thanks also to Michael Diercks, Mary Paster, and the audience at \textit{ACAL 47} at Berkeley for helpful questions and comments. All mistakes are my own.

\sloppy
\printbibliography[heading=subbibliography,notkeyword=this]
\end{document}
