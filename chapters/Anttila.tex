\documentclass[output=paper,newtxmath,modfonts,nonflat,draftmode]{langsci/langscibook} 

\author{Arto Anttila\affiliation{Stanford University} \lastand Adams Bodomo\affiliation{University of Vienna}}

\title{Metrically conditioned
vowel length in Dagaare}

\abstract{There is little evidence for stress in Dagaare, but vowel length alternations in nominal and verbal morphology reveal the presence of a word-initial metrical foot. New evidence for the foot hypothesis comes from action nominals formed with the suffix /-UU/: if the root is CV, the root lengthens and the suffix shortens; if the root is CVV the suffix shortens; if the root ends in C nothing happens.  Similar length alternations appear more idiosyncratically with number and aspect suffixes. A metrical analysis provides a simple account of these vowel length alternations.}

\IfFileExists{../localcommands.tex}{%hack to check whether this is being compiled as part of a collection or standalone
  \usepackage{pifont}
\usepackage{savesym}

\savesymbol{downingtriple}
\savesymbol{downingdouble}
\savesymbol{downingquad}
\savesymbol{downingquint}
\savesymbol{suph}
\savesymbol{supj}
\savesymbol{supw}
\savesymbol{sups}
\savesymbol{ts}
\savesymbol{tS}
\savesymbol{devi}
\savesymbol{devu}
\savesymbol{devy}
\savesymbol{deva}
\savesymbol{N}
\savesymbol{Z}
\savesymbol{circled}
\savesymbol{sem}
\savesymbol{row}
\savesymbol{tipa}
\savesymbol{tableauxcounter}
\savesymbol{tabhead}
\savesymbol{inp}
\savesymbol{inpno}
\savesymbol{g}
\savesymbol{hanl}
\savesymbol{hanr}
\savesymbol{kuku}
\savesymbol{ip}
\savesymbol{lipm}
\savesymbol{ripm}
\savesymbol{lipn}
\savesymbol{ripn} 
% \usepackage{amsmath} 
% \usepackage{multicol}
\usepackage{qtree} 
\usepackage{tikz-qtree,tikz-qtree-compat}
% \usepackage{tikz}
\usepackage{upgreek}


%%%%%%%%%%%%%%%%%%%%%%%%%%%%%%%%%%%%%%%%%%%%%%%%%%%%
%%%                                              %%%
%%%           Examples                           %%%
%%%                                              %%%
%%%%%%%%%%%%%%%%%%%%%%%%%%%%%%%%%%%%%%%%%%%%%%%%%%%%
% remove the percentage signs in the following lines
% if your book makes use of linguistic examples
\usepackage{tipa}  
\usepackage{pstricks,pst-xkey,pst-asr}

%for sande et al
\usepackage{pst-jtree}
\usepackage{pst-node}
%\usepackage{savesym}


% \usepackage{subcaption}
\usepackage{multirow}  
\usepackage{./langsci/styles/langsci-optional} 
\usepackage{./langsci/styles/langsci-lgr} 
\usepackage{./langsci/styles/langsci-glyphs} 
\usepackage[normalem]{ulem}
%% if you want the source line of examples to be in italics, uncomment the following line
% \def\exfont{\it}
\usetikzlibrary{arrows.meta,topaths,trees}
\usepackage[linguistics]{forest}
\forestset{
	fairly nice empty nodes/.style={
		delay={where content={}{shape=coordinate,for parent={
					for children={anchor=north}}}{}}
}}
\usepackage{soul}
\usepackage{arydshln}
% \usepackage{subfloat}
\usepackage{langsci/styles/langsci-gb4e} 
   
% \usepackage{linguex}
\usepackage{vowel}

\usepackage{pifont}% http://ctan.org/pkg/pifont
\newcommand{\cmark}{\ding{51}}%
\newcommand{\xmark}{\ding{55}}%
 
 
 %Lamont
 \makeatletter
\g@addto@macro\@floatboxreset\centering
\makeatother

\usepackage{newfloat} 
\DeclareFloatingEnvironment[fileext=tbx,name=Tableau]{tableau}
  %add all your local new commands to this file
\newcommand{\downingquad}[4]{\parbox{2.5cm}{#1}\parbox{3.5cm}{#2}\parbox{2.5cm}{#3}\parbox{3.5cm}{#4}}
\newcommand{\downingtriple}[3]{\parbox{4.5cm}{#1}\parbox{3cm}{#2}\parbox{3cm}{#3}}
\newcommand{\downingdouble}[2]{\parbox{4.5cm}{#1}\parbox{6cm}{#2}}
\newcommand{\downingquint}[5]{\parbox{1.75cm}{#1}\parbox{2.25cm}{#2}\parbox{2cm}{#3}\parbox{3cm}{#4}\parbox{2cm}{#5}}
\newcolumntype{Y}{>{\centering\arraybackslash}X}
\newcolumntype{T}{>{\centering\arraybackslash}m{2cm}}

%commands for Kusmer paper below
\newcommand{\ip}{$\upiota$}
\newcommand{\lipm}{(\_{\ip-Max}}
\newcommand{\ripm}{)\_{\ip-Max}}
\newcommand{\lipn}{(\_{\ip}}
\newcommand{\ripn}{)\_{\ip}}
\renewcommand{\_}[1]{\textsubscript{#1}}


%commands for Pillion paper below
\newcommand{\suph}{\textipa{\super h}}
\newcommand{\supj}{\textipa{\super j}}
\newcommand{\supw}{\textipa{\super w}}
\newcommand{\ts}{\textipa{\t{ts}}}
\newcommand{\tS}{\textipa{\t{tS}}}
\newcommand{\devi}{\textipa{\r*i}}
\newcommand{\devu}{\textipa{\r*u}}
\newcommand{\devy}{\textipa{\r*y}}
\newcommand{\deva}{\textipa{\r*a}}
\renewcommand{\N}{\textipa{N}}
\newcommand{\Z}{\textipa{Z}}
% 

%commands for Diercks paper below
\newcommand{\circled}[1]{\begin{tikzpicture}[baseline=(word.base)]
\node[draw, rounded corners, text height=8pt, text depth=2pt, inner sep=2pt, outer sep=0pt, use as bounding box] (word) {#1};
\end{tikzpicture}
}

%commands for Pesetsky paper below
% \newcommand{\sem}[2][]{\mbox{$[\![ $\textbf{#2}$ ]\!]^{#1}$}}
\newcommand{\sem}[2][]{\mbox{$[[ $\textbf{#2}$ ]]^{#1}$}}

% \newcommand{\ripn}{{\color{red}ripn}}%this is used but never defined. Please update the definition



%commands for Lamont paper below
\newcommand{\row}[4]{
	#1. & 
    /{#2}/ & 
    [{#3}] & 
    `#4' \\ 
}
%\newcounter{tableauxcounter}
\newcommand{\tabhead}[2]{
%     \captionsetup{labelformat=empty}
%     \stepcounter{tableauxcounter}
%     \addtocounter{table}{-1}
% 	\centering
% 	\caption{Tableau \thetableauxcounter: #1}
	\caption{#1}
	\label{#2}
}
\newcommand{\candref}[2]{{(\ref{#1}#2)}}
\newcommand{\tableauref}[1]{{Tableau~\ref{#1}}}
% tableaux
\newcommand{\inp}[1]{\multicolumn{2}{|l||}{{#1}}}
\newcommand{\inpno}[1]{\multicolumn{2}{|l||}{#1}}
\newcommand{\g}{\cellcolor{lightgray}}
\newcommand{\hanl}{\HandLeft}
\newcommand{\hanr}{\HandRight}
\newcommand{\kuku}{Kuk\'{u}}

% \newcommand{\nocaption}[1]{{\color{red} Please provide a caption}}

% \providecommand{\biberror}[1]{{\color{red}#1}}

\definecolor{RED}{cmyk}{0.05,1,0.8,0}


\newfontfamily\amharicfont[Script = Ethiopic, Scale = 1.0]{AbyssinicaSIL}
\newcommand{\amh}[1]{{\amharicfont #1}}

% 
% %Gjersoe
\usepackage{textgreek}
% 
\newcommand{\viol}{\fontfamily{MinionPro-OsF}\selectfont\rotatebox{60}{$\star$}}
\newcommand{\myscalex}{0.45}
\newcommand{\myscaley}{0.65}
%\newcommand{\red}[1]{\textcolor{red}{#1}}
%\newcommand{\blue}[1]{\textcolor{blue}{#1}}
\newcommand{\epen}[1]{\colorbox{jgray}{#1}}
\newcommand{\hand}{{\normalsize \ding{43}}}
\definecolor{jgray}{gray}{0.8} 
\usetikzlibrary{positioning}
\usetikzlibrary{matrix}
\newcommand{\mora}{\textmu\xspace}
\newcommand{\si}{\textsigma\xspace}
\newcommand{\ft}{\textPhi\xspace}
\newcommand{\tone}{\texttau\xspace}
\newcommand{\word}{\textomega\xspace}
% \newcommand{\ts}{\texttslig}
\newcommand{\fns}{\footnotesize}
\newcommand{\ns}{\normalsize}
\newcommand{\vs}{\vspace{1em}}
\newcommand{\bs}{\textbackslash}   % backslash
\newcommand{\cmd}[1]{{\bf \color{red}#1}}   % highlights command
\newcommand{\scell}[2][l]{\begin{tabular}[#1]{@{}c@{}}#2\end{tabular}}
% \interfootnotelinepenalty=10000

% --- Snider Representations --- %

\newcommand{\RepLevelHh}{
\begin{minipage}{0.10\textwidth}
\begin{tikzpicture}[xscale=\myscalex,yscale=\myscaley]
%\node (syl) at (0,0) {Hi};
\node (Rt) at (0,1) {o};
\node (H) at (-0.5,2) {H};
\node (R) at (0.5,3) {h};
%\draw [thick] (syl.north) -- (Rt.south) ;
\draw [thick] (Rt.north) -- (H.south) ;
\draw [thick] (Rt.north) -- (R.south) ;
\end{tikzpicture}
\end{minipage}
}

\newcommand{\RepLevelLh}{
\begin{minipage}{0.10\textwidth}
\begin{tikzpicture}[xscale=\myscalex,yscale=\myscaley]
%\node (syl) at (0,0) {Mid2};
\node (Rt) at (0,1) {o};
\node (H) at (-0.5,2) {L};
\node (R) at (0.5,3) {h};
%\draw [thick] (syl.north) -- (Rt.south) ;
\draw [thick] (Rt.north) -- (H.south) ;
\draw [thick] (Rt.north) -- (R.south) ;
\end{tikzpicture}
\end{minipage}
}

\newcommand{\RepLevelHl}{
\begin{minipage}{0.10\textwidth}
\begin{tikzpicture}[xscale=\myscalex,yscale=\myscaley]
%\node (syl) at (0,0) {Mid1};
\node (Rt) at (0,1) {o};
\node (H) at (-0.5,2) {H};
\node (R) at (0.5,3) {l};
%\draw [thick] (syl.north) -- (Rt.south) ;
\draw [thick] (Rt.north) -- (H.south) ;
\draw [thick] (Rt.north) -- (R.south) ;
\end{tikzpicture}
\end{minipage}
}

\newcommand{\RepLevelLl}{
\begin{minipage}{0.10\textwidth}
\begin{tikzpicture}[xscale=\myscalex,yscale=\myscaley]
%\node (syl) at (0,0) {Lo};
\node (Rt) at (0,1) {o};
\node (H) at (-0.5,2) {L};
\node (R) at (0.5,3) {l};
%\draw [thick] (syl.north) -- (Rt.south) ;
\draw [thick] (Rt.north) -- (H.south) ;
\draw [thick] (Rt.north) -- (R.south) ;
\end{tikzpicture}
\end{minipage}
}

% --- Representations --- %

\newcommand{\RepLevel}{
\begin{minipage}{0.10\textwidth}
\begin{tikzpicture}[xscale=\myscalex,yscale=\myscaley]
\node (syl) at (0,0) {\textsigma};
\node (Rt) at (0,1) {o};
\node (H) at (-0.5,2) {\texttau};
\node (R) at (0.5,3) {\textrho};
\draw [thick] (syl.north) -- (Rt.south) ;
\draw [thick] (Rt.north) -- (H.south) ;
\draw [thick] (Rt.north) -- (R.south) ;
\end{tikzpicture}
\end{minipage}
}

\newcommand{\RepContour}{
\begin{minipage}{0.10\textwidth}
\begin{tikzpicture}[xscale=\myscalex,yscale=\myscaley]
\node (syl) at (0,0) {\textsigma};
\node (Rt) at (0,1) {o};
\node (H) at (-0.5,2) {\texttau};
\node (R) at (0.5,3) {\textrho};
\node (Rt2) at (1.5,1.0) {o};
%\node (H2) at (1.0,2) {$\tau$};
%\node (R2) at (2.0,2.5) {R};
\draw [thick] (syl.north) -- (Rt.south) ;
\draw [thick] (Rt.north) -- (H.south) ;
\draw [thick] (Rt.north) -- (R.south) ;
\draw [thick] (syl.north) -- (Rt2.south) ;
%\draw [thick] (Rt2.north) -- (H2.south) ;
%\draw [thick] (Rt2.north) -- (R2.south) ;
\end{tikzpicture}
\end{minipage}
}


% --- OT constraints --- %

\newcommand{\IllustrationDown}{
\begin{minipage}{0.09\textwidth}
\begin{tikzpicture}[xscale=0.7,yscale=0.45]
\node (reg) at (0,0.75) {{\small \textalpha}};
\node (arrow) at (0,0) {{\fns $\downarrow$}};
\node (Rt) at (0,-0.75) {{\small \textbeta}};
\end{tikzpicture}
\end{minipage}
}

\newcommand{\IllustrationUp}{
\begin{minipage}{0.09\textwidth}
\begin{tikzpicture}[xscale=0.7,yscale=0.45]
\node (reg) at (0,0.75) {{\small \textalpha}};
\node (arrow) at (0,0) {{\fns $\uparrow$}};
\node (Rt) at (0,-0.75) {{\small \textbeta}};
\end{tikzpicture}
\end{minipage}
}

\newcommand{\MaxAB}{
\begin{minipage}{0.09\textwidth}
\begin{tikzpicture}[xscale=0.6,yscale=0.4]
\node (max) at (0,0) {{\small \textsc{Max}}};
\node (reg) at (0.75,0.5) {{\fns \textalpha}};
\node (arrow) at (0.75,0) {{\tiny $\downarrow$}};
\node (Rt) at (0.75,-0.5) {{\fns \textbeta}};
\end{tikzpicture}
\end{minipage}
}

\newcommand{\DepAB}{
\begin{minipage}{0.09\textwidth}
\begin{tikzpicture}[xscale=0.6,yscale=0.4]
\node (max) at (0,0) {{\small \textsc{Dep}}};
\node (reg) at (0.75,0.5) {{\fns \textalpha}};
\node (arrow) at (0.75,0) {{\tiny $\downarrow$}};
\node (Rt) at (0.75,-0.5) {{\fns \textbeta}};
\end{tikzpicture}
\end{minipage}
}

\newcommand{\DepHReg}{
\begin{minipage}{0.055\textwidth}
\begin{tikzpicture}[xscale=0.6,yscale=0.4]
\node (dep) at (0,0) {{\small \textsc{Dep}}};
\node (reg) at (0,-1.0) {{\small h}};
\end{tikzpicture}
\end{minipage}
}

\newcommand{\DepLReg}{
\begin{minipage}{0.055\textwidth}
\begin{tikzpicture}[xscale=0.6,yscale=0.4]
\node (dep) at (0,0) {{\small \textsc{Dep}}};
\node (reg) at (0,-1.0) {{\small l}};
\end{tikzpicture}
\end{minipage}
}

\newcommand{\DepReg}{
\begin{minipage}{0.055\textwidth}
\begin{tikzpicture}[xscale=0.6,yscale=0.4]
\node (dep) at (0,0) {{\small \textsc{Dep}}};
\node (reg) at (0,-1.0) {{\small \textrho}};
\end{tikzpicture}
\end{minipage}
}

\newcommand{\DepTRt}{
\begin{minipage}{0.1\textwidth}
\begin{tikzpicture}[xscale=0.6,yscale=0.4]
\node (dep) at (0,0) {{\small \textsc{Dep}}};
\node (t) at (0.75,0.5) {{\fns \texttau}};
\node (arrow) at (0.75,0) {{\tiny $\downarrow$}};
\node (Rt) at (0.75,-0.5) {{\fns o}};
\end{tikzpicture}
\end{minipage}
}

\newcommand{\MaxRegRt}{
\begin{minipage}{0.1\textwidth}
\begin{tikzpicture}[xscale=0.6,yscale=0.4]
\node (max) at (0,0) {{\small \textsc{Max}}};
\node (arrow) at (0.75,0) {{\tiny $\downarrow$}};
\node (Rt) at (0.75,-0.5) {{\fns o}};
\node (reg) at (0.75,0.5) {{\fns \textrho}};
\end{tikzpicture}
\end{minipage}
}

\newcommand{\RegToneByRt}{
\begin{minipage}{0.06\textwidth}
\begin{tikzpicture}[xscale=0.6,yscale=0.5]
\node[rotate=20] (arrow1) at (-0.15,0) {{\fns $\uparrow$}};
\node[rotate=340] (arrow2) at (0.15,0) {{\fns $\uparrow$}};
\node (Rt) at (0,-0.55) {{\small o}};
\node (reg) at (0.4,0.55) {{\small \textrho}};
\node (tone) at (-0.4,0.55) {{\small \texttau}};
\end{tikzpicture}
\end{minipage}
}

\newcommand{\RegToneBySyl}{
\begin{minipage}{0.06\textwidth}
\begin{tikzpicture}[xscale=0.6,yscale=0.5]
\node[rotate=20] (arrow1) at (-0.15,0) {{\fns $\uparrow$}};
\node[rotate=340] (arrow2) at (0.15,0) {{\fns $\uparrow$}};
\node (Rt) at (0,-0.55) {{\small \textsigma}};
\node (reg) at (0.4,0.55) {{\small \textrho}};
\node (tone) at (-0.4,0.55) {{\small \texttau}};
\end{tikzpicture}
\end{minipage}
}

\newcommand{\DepTone}{
\begin{minipage}{0.055\textwidth}
\begin{tikzpicture}[xscale=0.6,yscale=0.4]
\node (dep) at (0,0) {{\small \textsc{Dep}}};
\node (tone) at (0,-1.0) {{\small \texttau}};
\end{tikzpicture}
\end{minipage}
}

\newcommand{\DepTonalRt}{
\begin{minipage}{0.055\textwidth}
\begin{tikzpicture}[xscale=0.6,yscale=0.4]
\node (dep) at (0,0) {{\small \textsc{Dep}}};
\node (tone) at (0,-1.0) {{\small o}};
\end{tikzpicture}
\end{minipage}
}

\newcommand{\DepL}{
\begin{minipage}{0.055\textwidth}
\begin{tikzpicture}[xscale=0.6,yscale=0.4]
\node (dep) at (0,0) {{\small \textsc{Dep}}};
\node (tone) at (0,-1.0) {{\small L}};
\end{tikzpicture}
\end{minipage}
}

\newcommand{\DepH}{
\begin{minipage}{0.055\textwidth}
\begin{tikzpicture}[xscale=0.6,yscale=0.4]
\node (dep) at (0,0) {{\small \textsc{Dep}}};
\node (tone) at (0,-1.0) {{\small H}};
\end{tikzpicture}
\end{minipage}
}

\newcommand{\NoMultDiff}{{\small *loh}}
\newcommand{\Alt}{{\small \textsc{Alt}}}
\newcommand{\NoSkip}{{\small \scell{\textsc{No}\\\textsc{Skip}}}}


\newcommand{\RegDomRt}{
\begin{minipage}{0.030\textwidth}
\begin{tikzpicture}[xscale=0.6,yscale=0.5]
\node (arrow) at (0,0) {{\fns $\downarrow$}};
\node (Rt) at (0,-0.55) {{\small o}};
\node (reg) at (0,0.55) {{\small \textrho}};
\end{tikzpicture}
\end{minipage}
}

\newcommand{\DepRegRt}{
\begin{minipage}{0.1\textwidth}
\begin{tikzpicture}[xscale=0.6,yscale=0.4]
\node (dep) at (0,0) {{\small \textsc{Dep}}};
\node (arrow) at (0.75,0) {{\tiny $\downarrow$}};
\node (Rt) at (0.75,-0.5) {{\fns o}};
\node (reg) at (0.75,0.5) {{\fns \textrho}};
\end{tikzpicture}
\end{minipage}
}

% unused

\newcommand{\ToneByRt}{
\begin{minipage}{0.05\textwidth}
\begin{tikzpicture}[xscale=0.6,yscale=0.5]
\node (arrow) at (0,0) {{\fns $\uparrow$}};
\node (Rt) at (0,-0.55) {{\small o}};
\node (tone) at (0,0.55) {{\small \texttau}};
\end{tikzpicture}
\end{minipage}
}

\newcommand{\RegByRt}{
\begin{minipage}{0.05\textwidth}
\begin{tikzpicture}[xscale=0.6,yscale=0.5]
\node (arrow) at (0,0) {{\fns $\uparrow$}};
\node (Rt) at (0,-0.55) {{\small o}};
\node (reg) at (0,0.55) {{\small \textrho}};
\end{tikzpicture}
\end{minipage}
}

\newcommand{\ToneDomRt}{
\begin{minipage}{0.05\textwidth}
\begin{tikzpicture}[xscale=0.6,yscale=0.5]
\node (arrow) at (0,0) {{\fns $\downarrow$}};
\node (Rt) at (0,-0.55) {{\small o}};
\node (tone) at (0,0.55) {{\small \texttau}};
\end{tikzpicture}
\end{minipage}
}

% --- OT tableaus --- %

% Sec. 3.2, first tabl.

\newcommand{\OTHLInput}{
\begin{minipage}{0.17\textwidth}
\begin{tikzpicture}[xscale=\myscalex,yscale=\myscaley]
\node (tone) at (2,0) {(= H)};
\node (syl) at (0,0) {\textsigma};
\node (Rt) at (0,1) {o};
\node (H) at (-0.5,2) {H};
\node (R) at (0.5,3) {h};
\node (Rt2) at (1.5,1.0) {o};
%\node (H2) at (1.0,2) {\epen{L}};
\node (R2) at (2.0,3) {\blue{l}};
\draw [thick] (syl.north) -- (Rt.south) ;
\draw [thick] (Rt.north) -- (H.south) ;
\draw [thick] (Rt.north) -- (R.south) ;
\draw [thick] (syl.north) -- (Rt2.south) ;
%\draw [dashed] (Rt2.north) -- (H2.south) ;
%\draw [dashed] (Rt2.north) -- (R2.south) ;
\end{tikzpicture}
\end{minipage}
}

\newcommand{\OTHLWinner}{
\begin{minipage}{0.17\textwidth}
\begin{tikzpicture}[xscale=\myscalex,yscale=\myscaley]
\node (tone) at (2,0) {(= HL)};
\node (syl) at (0,0) {\textsigma};
\node (Rt) at (0,1) {o};
\node (H) at (-0.5,2) {H};
\node (R) at (0.5,3) {h};
\node (Rt2) at (1.5,1.0) {o};
\node (H2) at (1.0,2) {\epen{L}};
\node (R2) at (2.0,3) {\blue{l}};
\draw [thick] (syl.north) -- (Rt.south) ;
\draw [thick] (Rt.north) -- (H.south) ;
\draw [thick] (Rt.north) -- (R.south) ;
\draw [thick] (syl.north) -- (Rt2.south) ;
\draw [dashed] (Rt2.north) -- (H2.south) ;
\draw [dashed] (Rt2.north) -- (R2.south) ;
\end{tikzpicture}
\end{minipage}
}

\newcommand{\OTHLSpreadingHOnly}{
\begin{minipage}{0.17\textwidth}
\begin{tikzpicture}[xscale=\myscalex,yscale=\myscaley]
\node (tone) at (2,0) {(= HM)};
\node (syl) at (0,0) {\textsigma};
\node (Rt) at (0,1) {o};
\node (H) at (-0.5,2) {H};
\node (R) at (0.5,3) {h};
\node (Rt2) at (1.5,1.0) {o};
%\node (H2) at (1.0,2) {\epen{L}};
\node (R2) at (2.0,3) {\blue{l}};
\draw [thick] (syl.north) -- (Rt.south) ;
\draw [thick] (Rt.north) -- (H.south) ;
\draw [thick] (Rt.north) -- (R.south) ;
\draw [thick] (syl.north) -- (Rt2.south) ;
\draw [dashed] (Rt2.north) -- (R2.south) ;
\draw [dashed] (Rt2.north) -- (H.south) ;
\end{tikzpicture}
\end{minipage}
}

\newcommand{\OTHLInsertH}{
\begin{minipage}{0.17\textwidth}
\begin{tikzpicture}[xscale=\myscalex,yscale=\myscaley]
\node (tone) at (2,0) {(= HM)};
\node (syl) at (0,0) {\textsigma};
\node (Rt) at (0,1) {o};
\node (H) at (-0.5,2) {H};
\node (R) at (0.5,3) {h};
\node (Rt2) at (1.5,1.0) {o};
\node (H2) at (1.0,2) {\epen{H}};
\node (R2) at (2.0,3) {\blue{l}};
\draw [thick] (syl.north) -- (Rt.south) ;
\draw [thick] (Rt.north) -- (H.south) ;
\draw [thick] (Rt.north) -- (R.south) ;
\draw [thick] (syl.north) -- (Rt2.south) ;
\draw [dashed] (Rt2.north) -- (H2.south) ;
\draw [dashed] (Rt2.north) -- (R2.south) ;
\end{tikzpicture}
\end{minipage}
}

\newcommand{\OTHLOverwriting}{
\begin{minipage}{0.17\textwidth}
\begin{tikzpicture}[xscale=\myscalex,yscale=\myscaley]
\node (syl) at (0,0) {\textsigma};
\node (Rt) at (0,1) {o};
\node (H) at (-0.5,2) {H};
\node (R) at (0.5,3) {h};
\node (Rt2) at (1.5,1.0) {o};
%\node (H2) at (1.0,2) {\epen{L}};
\node (R2) at (2.0,3) {\blue{l}};
\draw [thick] (syl.north) -- (Rt.south) ;
\draw [thick] (Rt.north) -- (H.south) ;
\draw [thick] (Rt.north) -- (R.south) ;
\draw [thick] (syl.north) -- (Rt2.south) ;
%\draw [dashed] (Rt2.north) -- (H2.south) ;
\draw [dashed] (Rt.north) -- (R2.south) ;
\node (del) at (0.3,1.9) {\textbf{=}};
\end{tikzpicture}
\end{minipage}
}

\newcommand{\OTHLSpreading}{
\begin{minipage}{0.17\textwidth}
\begin{tikzpicture}[xscale=\myscalex,yscale=\myscaley]
\node (syl) at (0,0) {\textsigma};
\node (Rt) at (0,1) {o};
\node (H) at (-0.5,2) {H};
\node (R) at (0.5,3) {h};
\node (Rt2) at (1.5,1.0) {o};
%\node (H2) at (1.0,2) {\epen{L}};
\node (R2) at (2.0,3) {\blue{l}};
\draw [thick] (syl.north) -- (Rt.south) ;
\draw [thick] (Rt.north) -- (H.south) ;
\draw [thick] (Rt.north) -- (R.south) ;
\draw [thick] (syl.north) -- (Rt2.south) ;
%\draw [dashed] (Rt2.north) -- (H2.south) ;
\draw [dashed] (Rt2.north) -- (H.south) ;
\draw [dashed] (Rt2.north) -- (R.south) ;
\end{tikzpicture}
\end{minipage}
}

% Sec. 4.2, second tabl.: phrase-medial position

\newcommand{\OTHnoLInput}{
\begin{minipage}{0.17\textwidth}
\begin{tikzpicture}[xscale=\myscalex,yscale=\myscaley]
\node (tone) at (2,0) {(= H)};
\node (syl) at (0,0) {\textsigma};
\node (Rt) at (0,1) {o};
\node (H) at (-0.5,2) {H};
\node (R) at (0.5,3) {h};
\node (Rt2) at (1.5,1.0) {o};
%\node (H2) at (1.0,2) {\epen{L}};
%\node (R2) at (2.0,3) {\blue{l}};
\draw [thick] (syl.north) -- (Rt.south) ;
\draw [thick] (Rt.north) -- (H.south) ;
\draw [thick] (Rt.north) -- (R.south) ;
\draw [thick] (syl.north) -- (Rt2.south) ;
\end{tikzpicture}
\end{minipage}
}

\newcommand{\OTHnoLEpenth}{
\begin{minipage}{0.17\textwidth}
\begin{tikzpicture}[xscale=\myscalex,yscale=\myscaley]
\node (tone) at (2,0) {(= HM)};
\node (syl) at (0,0) {\textsigma};
\node (Rt) at (0,1) {o};
\node (H) at (-0.5,2) {H};
\node (R) at (0.5,3) {h};
\node (Rt2) at (1.5,1.0) {o};
\node (H2) at (1.0,2) {\epen{L}};
\node (R2) at (2.0,3) {\epen{h}};
\draw [thick] (syl.north) -- (Rt.south) ;
\draw [thick] (Rt.north) -- (H.south) ;
\draw [thick] (Rt.north) -- (R.south) ;
\draw [thick] (syl.north) -- (Rt2.south) ;
\draw [dashed] (Rt2.north) -- (H2.south) ;
\draw [dashed] (Rt2.north) -- (R2.south) ;
\end{tikzpicture}
\end{minipage}
}

\newcommand{\OTHnoLSpreading}{
\begin{minipage}{0.17\textwidth}
\begin{tikzpicture}[xscale=\myscalex,yscale=\myscaley]
\node (tone) at (2,0) {(= HH)};
\node (syl) at (0,0) {\textsigma};
\node (Rt) at (0,1) {o};
\node (H) at (-0.5,2) {H};
\node (R) at (0.5,3) {h};
\node (Rt2) at (1.5,1.0) {o};
%\node (H2) at (1.0,2) {\epen{L}};
%\node (R2) at (2.0,3) {\blue{l}};
\draw [thick] (syl.north) -- (Rt.south) ;
\draw [thick] (Rt.north) -- (H.south) ;
\draw [thick] (Rt.north) -- (R.south) ;
\draw [thick] (syl.north) -- (Rt2.south) ;
\draw [dashed] (Rt2.north) -- (H.south) ;
\draw [dashed] (Rt2.north) -- (R.south) ;
\end{tikzpicture}
\end{minipage}
}

% Sec. 4.2, third tabl., LM is unaffected by L\%

\newcommand{\OTLMInput}{
\begin{minipage}{0.2\textwidth}
\begin{tikzpicture}[xscale=\myscalex,yscale=\myscaley]
\node (tone) at (2,0) {(= LM)};
\node (syl) at (0,0) {\textsigma};
\node (Rt) at (0,1) {o};
\node (H) at (-0.5,2) {L};
\node (R) at (0.5,3) {l};
\node (Rt2) at (1.5,1.0) {o};
\node (H2) at (1.0,2) {L};
\node (R2) at (2.0,3) {h};
\node (R3) at (3.0,3) {\blue{l}};
\draw [thick] (syl.north) -- (Rt.south) ;
\draw [thick] (Rt.north) -- (H.south) ;
\draw [thick] (Rt.north) -- (R.south) ;
\draw [thick] (syl.north) -- (Rt2.south) ;
\draw [thick] (Rt2.north) -- (H2.south) ;
\draw [thick] (Rt2.north) -- (R2.south) ;
\end{tikzpicture}
\end{minipage}
}

\newcommand{\OTLMReplace}{
\begin{minipage}{0.2\textwidth}
\begin{tikzpicture}[xscale=\myscalex,yscale=\myscaley]
\node (tone) at (2,0) {(= LL)};
\node (syl) at (0,0) {\textsigma};
\node (Rt) at (0,1) {o};
\node (H) at (-0.5,2) {L};
\node (R) at (0.5,3) {l};
\node (Rt2) at (1.5,1.0) {o};
\node (H2) at (1.0,2) {L};
\node (R2) at (2.0,3) {h};
\node (R3) at (3.0,3) {\blue{l}};
\draw [thick] (syl.north) -- (Rt.south) ;
\draw [thick] (Rt.north) -- (H.south) ;
\draw [thick] (Rt.north) -- (R.south) ;
\draw [thick] (syl.north) -- (Rt2.south) ;
\draw [thick] (Rt2.north) -- (H2.south) ;
\draw [thick] (Rt2.north) -- (R2.south) ;
\draw [dashed] (Rt2.north) -- (R3.south) ;
\node (del) at (1.8,2.1) {\textbf{=}};
\end{tikzpicture}
\end{minipage}
}

\newcommand{\OTLMTwoReg}{
\begin{minipage}{0.2\textwidth}
\begin{tikzpicture}[xscale=\myscalex,yscale=\myscaley]
\node (tone) at (2,0) {(= LML)};
\node (syl) at (0,0) {\textsigma};
\node (Rt) at (0,1) {o};
\node (H) at (-0.5,2) {L};
\node (R) at (0.5,3) {l};
\node (Rt2) at (1.5,1.0) {o};
\node (H2) at (1.0,2) {L};
\node (R2) at (2.0,3) {h};
\node (R3) at (3.0,3) {\blue{l}};
\draw [thick] (syl.north) -- (Rt.south) ;
\draw [thick] (Rt.north) -- (H.south) ;
\draw [thick] (Rt.north) -- (R.south) ;
\draw [thick] (syl.north) -- (Rt2.south) ;
\draw [thick] (Rt2.north) -- (H2.south) ;
\draw [thick] (Rt2.north) -- (R2.south) ;
\draw [dashed] (Rt2.north) -- (R3.south) ;
\end{tikzpicture}
\end{minipage}
}

% Sec. 4.2, fourth tabl., L is affected by L\% but M is not

\newcommand{\OTLInput}{
\begin{minipage}{0.17\textwidth}
\begin{tikzpicture}[xscale=\myscalex,yscale=\myscaley]
\node (tone) at (2,0) {(= L)};
\node (syl) at (0,0) {\textsigma};
\node (Rt) at (0,1) {o};
\node (H) at (-0.5,2) {L};
\node (R) at (0.5,3) {l};
\node (R2) at (2,3) {\blue{l}};
\draw [thick] (syl.north) -- (Rt.south) ;
\draw [thick] (Rt.north) -- (H.south) ;
\draw [thick] (Rt.north) -- (R.south) ;
\end{tikzpicture}
\end{minipage}
}

\newcommand{\OTLLowered}{
\begin{minipage}{0.17\textwidth}
\begin{tikzpicture}[xscale=\myscalex,yscale=\myscaley]
\node (tone) at (2,0) {(= LL)};
\node (syl) at (0,0) {\textsigma};
\node (Rt) at (0,1) {o};
\node (H) at (-0.5,2) {L};
\node (R) at (0.5,3) {l};
\node (R2) at (2,3) {\blue{l}};
\draw [thick] (syl.north) -- (Rt.south) ;
\draw [thick] (Rt.north) -- (H.south) ;
\draw [thick] (Rt.north) -- (R.south) ;
\draw [dashed] (Rt.north) -- (R2.south) ;
\end{tikzpicture}
\end{minipage}
}

\newcommand{\OTMInput}{
\begin{minipage}{0.17\textwidth}
\begin{tikzpicture}[xscale=\myscalex,yscale=\myscaley]
\node (tone) at (2,0) {(= M)};
\node (syl) at (0,0) {\textsigma};
\node (Rt) at (0,1) {o};
\node (H) at (-0.5,2) {L};
\node (R) at (0.5,3) {h};
\node (R2) at (2,3) {\blue{l}};
\draw [thick] (syl.north) -- (Rt.south) ;
\draw [thick] (Rt.north) -- (H.south) ;
\draw [thick] (Rt.north) -- (R.south) ;
\end{tikzpicture}
\end{minipage}
}

\newcommand{\OTMLowered}{
\begin{minipage}{0.17\textwidth}
\begin{tikzpicture}[xscale=\myscalex,yscale=\myscaley]
\node (tone) at (2,0) {(= ML)};
\node (syl) at (0,0) {\textsigma};
\node (Rt) at (0,1) {o};
\node (H) at (-0.5,2) {L};
\node (R) at (0.5,3) {h};
\node (R2) at (2,3) {\blue{l}};
\draw [thick] (syl.north) -- (Rt.south) ;
\draw [thick] (Rt.north) -- (H.south) ;
\draw [thick] (Rt.north) -- (R.south) ;
\draw [dashed] (Rt.north) -- (R2.south) ;
\end{tikzpicture}
\end{minipage}
}

% Sec. 4.2, fifth tableau, polar questions with level tones

\newcommand{\OTLPolIn}{
\begin{minipage}{0.20\textwidth}
\begin{tikzpicture}[xscale=\myscalex-0.05,yscale=\myscaley-0.05]
\node (tone) at (3.5,0) {(= L)};
\node (syl) at (0,0) {\textsigma};
\node (syl2) at (2,0) {\red{\textsigma}};
\node (Rt) at (0,1) {o};
\node (H) at (-0.5,2) {L};
\node (R) at (0.5,3) {l};
\node (Rt2) at (2,1) {\red{o}};
\draw [thick] (syl.north) -- (Rt.south) ;
\draw [thick,red] (syl2.north) -- (Rt2.south) ;
\draw [thick] (Rt.north) -- (H.south) ;
\draw [thick] (Rt.north) -- (R.south) ;
\end{tikzpicture}
\end{minipage}
}

\newcommand{\OTLPolDef}{
\begin{minipage}{0.20\textwidth}
\begin{tikzpicture}[xscale=\myscalex-0.05,yscale=\myscaley-0.05]
\node (tone) at (3.5,0) {(= L.M)};
\node (syl) at (0,0) {\textsigma};
\node (syl2) at (2,0) {\red{\textsigma}};
\node (Rt) at (0,1) {o};
\node (H) at (-0.5,2) {L};
\node (R) at (0.5,3) {l};
\node (H2) at (1.5,2) {\epen{L}};
\node (R2) at (2.5,3) {\epen{h}};
\node (Rt2) at (2,1) {\red{o}};
\draw [thick] (syl.north) -- (Rt.south) ;
\draw [thick,red] (syl2.north) -- (Rt2.south) ;
\draw [thick] (Rt.north) -- (H.south) ;
\draw [thick] (Rt.north) -- (R.south) ;
\draw [semithick,dashed] (Rt2.north) -- (H2.south) ;
\draw [semithick,dashed] (Rt2.north) -- (R2.south) ;
\end{tikzpicture}
\end{minipage}
}

\newcommand{\OTLPolAlt}{
\begin{minipage}{0.20\textwidth}
\begin{tikzpicture}[xscale=\myscalex-0.05,yscale=\myscaley-0.05]
\node (tone) at (3.5,0) {(= L.L)};
\node (syl) at (0,0) {\textsigma};
\node (syl2) at (2,0) {\red{\textsigma}};
\node (Rt) at (0,1) {o};
\node (H) at (-0.5,2) {L};
\node (R) at (0.5,3) {l};
\node (Rt2) at (2,1) {\red{o}};
\draw [thick] (syl.north) -- (Rt.south) ;
\draw [thick,red] (syl2.north) -- (Rt2.south) ;
\draw [thick] (Rt.north) -- (H.south) ;
\draw [thick] (Rt.north) -- (R.south) ;
\draw [semithick,dashed] (Rt2.north) -- (H.south) ;
\draw [semithick,dashed] (Rt2.north) -- (R.south) ;
\end{tikzpicture}
\end{minipage}
}

% Sec. 4.2, sixth tableau, polar questions with contour tones

\newcommand{\OTLLPolIn}{
\begin{minipage}{0.23\textwidth}
\begin{tikzpicture}[xscale=\myscalex-0.05,yscale=\myscaley-0.05]
\node (tone) at (5.2,0) {(= L)};
\node (syl) at (0,0) {\textsigma};
\node (syl3) at (3.4,0) {\red{\textsigma}};
\node (Rt) at (0,1) {o};
\node (Rt2) at (1.7,1) {o};
\node (Rt3) at (3.4,1) {\red{o}};
\node (H) at (-0.5,2) {L};
\node (R) at (0.5,3) {l};
\draw [thick] (syl.north) -- (Rt.south) ;
\draw [thick] (syl.north) -- (Rt2.south) ;
\draw [thick,red] (syl3.north) -- (Rt3.south) ;
\draw [thick] (Rt.north) -- (H.south) ;
\draw [thick] (Rt.north) -- (R.south) ;
\end{tikzpicture}
\end{minipage}
}

\newcommand{\OTLLPolDef}{
\begin{minipage}{0.23\textwidth}
\begin{tikzpicture}[xscale=\myscalex-0.05,yscale=\myscaley-0.05]
\node (tone) at (5.2,0) {(= L.M)};
\node (syl) at (0,0) {\textsigma};
\node (syl3) at (3.4,0) {\red{\textsigma}};
\node (Rt) at (0,1) {o};
\node (Rt2) at (1.7,1) {o};
\node (Rt3) at (3.4,1) {\red{o}};
\node (H) at (-0.5,2) {L};
\node (R) at (0.5,3) {l};
\node (H3) at (2.9,2) {\epen{L}};
\node (R3) at (3.9,3) {\epen{h}};
\draw [thick] (syl.north) -- (Rt.south) ;
\draw [thick] (syl.north) -- (Rt2.south) ;
\draw [thick,red] (syl3.north) -- (Rt3.south) ;
\draw [thick] (Rt.north) -- (H.south) ;
\draw [thick] (Rt.north) -- (R.south) ;
\draw [dashed] (Rt3.north) -- (H3.south) ;
\draw [dashed] (Rt3.north) -- (R3.south) ;
\end{tikzpicture}
\end{minipage}
}

\newcommand{\OTLLPolSkip}{
\begin{minipage}{0.23\textwidth}
\begin{tikzpicture}[xscale=\myscalex-0.05,yscale=\myscaley-0.05]
\node (tone) at (5.2,0) {(= L.L)};
\node (syl) at (0,0) {\textsigma};
\node (syl3) at (3.4,0) {\red{\textsigma}};
\node (Rt) at (0,1) {o};
\node (Rt2) at (1.7,1) {o};
\node (Rt3) at (3.4,1) {\red{o}};
\node (H) at (-0.5,2) {L};
\node (R) at (0.5,3) {l};
\draw [thick] (syl.north) -- (Rt.south) ;
\draw [thick] (syl.north) -- (Rt2.south) ;
\draw [thick,red] (syl3.north) -- (Rt3.south) ;
\draw [thick] (Rt.north) -- (H.south) ;
\draw [thick] (Rt.north) -- (R.south) ;
\draw [dashed] (Rt3.north) -- (H.south) ;
\draw [dashed] (Rt3.north) -- (R.south) ;
\end{tikzpicture}
\end{minipage}
}  
  
\newcommand{\ilit}[1]{#1\il{#1}}    
\newcommand{\isit}[1]{#1\is{#1}}  

\makeatletter
\let\thetitle\@title
\let\theauthor\@author 
\makeatother

\newcommand{\togglepaper}[1][0]{ 
  \bibliography{../localbibliography}
  %% hyphenation points for line breaks
%% Normally, automatic hyphenation in LaTeX is very good
%% If a word is mis-hyphenated, add it to this file
%%
%% add information to TeX file before \begin{document} with:
%% %% hyphenation points for line breaks
%% Normally, automatic hyphenation in LaTeX is very good
%% If a word is mis-hyphenated, add it to this file
%%
%% add information to TeX file before \begin{document} with:
%% \include{localhyphenation}
\hyphenation{
affri-ca-te
affri-ca-tes
com-ple-ments
par-a-digm
Sha-ron
Kings-ton
phe-nom-e-non
Daul-ton
Abu-ba-ka-ri
Ngo-nya-ni
Clem-ents 
King-ston
Tru-cken-brodt
Tab-leau
cophono-logies
mark-edness
Ti-gri-nya
a-mong
Car-stens
Lu-bu-ku-su
}
\hyphenation{
affri-ca-te
affri-ca-tes
com-ple-ments
par-a-digm
Sha-ron
Kings-ton
phe-nom-e-non
Daul-ton
Abu-ba-ka-ri
Ngo-nya-ni
Clem-ents 
King-ston
Tru-cken-brodt
Tab-leau
cophono-logies
mark-edness
Ti-gri-nya
a-mong
Car-stens
Lu-bu-ku-su
}
  \papernote{\scriptsize\normalfont
    \theauthor.
    \thetitle. 
    To appear in: 
    Emily Clem,   Peter Jenks \& Hannah Sande.
    Theory and description in African Linguistics: Selected papers from the 47th Annual Conference on African Linguistics.
    Berlin: Language Science Press. [preliminary page numbering]
  }
  \pagenumbering{roman}
  \setcounter{chapter}{#1}
  \addtocounter{chapter}{-1}
}

\newcommand{\upstep}{\textupstep}


% \newcounter{tableauxcounter}

\renewcommand{\textltailn}{ɲ}
\renewcommand{\textbardotlessj}{ɟ}

\newcommand{\emphkh}[1]{\textit{#1}} %originally \textbf, banned by the guidelines



\definecolor{lsDOIGray}{cmyk}{0,0,0,0.45}


\newcommand{\xuparrow}[1]{%
  {\left\uparrow\vbox to #1{}\right.\kern-\nulldelimiterspace}
}
\renewcommand \textupstep[1]{\char"A71B#1}
\renewcommand \textdownstep[1]{\char"A71C#1}
 
 \newcommand{\ꜛ}{\textsf{ꜛ}}
 
\def\biberror{\undefined}


\newcommand{\OTbox}[1]{\resizebox{.88\textwidth}{!}{#1}}
 
  \togglepaper[2]
}{}


\begin{document}
\maketitle
\section{Introduction}


\ili{Dagaare} (\ili{Gur}, Mabia; \citealt{Naden1989}, \citealt{Bodomo1997}) is a two-\isi{tone} language of northwestern Ghana.\footnote{The data represent the Jirapa district dialect of which the second author is a native speaker. Most of the data are previously unpublished; some can be found in \citep{Kennedy1966, Bodomo1997, Anttila&Bodomo2009}, which are referred to in the text. The examples are given in \pgposscitet{Bodomo1997}{37} orthography. The digraphs <ky>, <gy>, <ny> stand for IPA [tʃ], [dʒ], [ɲ], respectively.} There is little direct evidence for metrical stress, but vowel alternations in nominal and \isi{verbal morphology} suggest the presence of a word-initial metrical foot  \citep{Anttila&Bodomo2009}. New evidence for the foot hypothesis comes from \isi{vowel length} alternations in action nominals, the topic of the present paper.



\citet[9]{Kennedy1966} gives the \isi{vowel inventory} for \ili{Dagaare} word-medial syllables shown in \tabref{tab:anttila:1}. 

\begin{table}
\begin{tabularx}{\textwidth}{XXXXX}
\lsptoprule
 & \multicolumn{2}{c}{{$-$round}} & \multicolumn{2}{c}{{$+$round}}\\
 & +\textsc{atr}  & $-$\textsc{atr} & +\textsc{atr} & $-$\textsc{atr}\\
\midrule
+high, $-$low & i, ii & ɪ, ɪɪ & u, uu & ʊ, ʊʊ\\
$-$high, $-$low & e, ie & ɛ, ɪɛ & o, uo & ɔ, ʊɔ\\
$-$high, +low &  & a, aa &  & \\
\lspbottomrule
\end{tabularx}
\caption{Dagaare vowels \citep{Kennedy1966}}
\label{tab:anttila:1}
\end{table}


Vowel length is contrastive in \ili{Dagaare}. High and low vowels can be short or long, but there is a striking gap in Kennedy’s inventory: long mid vowels are missing. \citet[8]{Kennedy1966} notes that word-medially “there are high and low long vowels, but no mid long vowels” and suggests that in terms of the phonological system the diphthongs [ie], [ɪɛ], [uo], [ʊɔ] are in fact the missing long vowels /ee/, /ɛɛ/, /oo/, /ɔɔ/. This is an attractive interpretation because it makes the long vowel pattern symmetrical. 



The problem is that long mid vowels do exist on the surface. There are even near-minimal pairs that demonstrate a phonemic contrast between a long mid vowel and the corresponding diphthong: \textit{béé} ‘or’ vs. \textit{bíé} ‘child.\textsc{sg}’, \textit{gɔ̀ɔ́} `left' vs. \textit{gúɔ̀} `thorn.\textsc{sg}'. Examples of long mid vowels are shown in \tabref{tab:anttila:2}. /E/ stands for a [$-$high, $-$low, $-$round] vowel and /I/ for a [+high, $-$low, $-$round] vowel, both underspecified for {$\pm$}{\textsc{atr}}{]; /V/ stands for a [$-$high] vowel underspecified for [$\pm$back], [$\pm$round], and [$\pm$}{\textsc{atr}}{].}\footnote{Tone does not figure into the \isi{vowel length} alternations, but a brief note is warranted. Underlyingly there is a three-way contrast between H, L, and toneless; on the surface there is a three-way contrast H, \textsuperscript{!}H, and L. Toneless morphemes surface as H or L depending on the context. We mark downstep as a raised exclamation point before a H toned syllable. Downstep seems analyzable as a floating L and contour tones as combinations of H and L. The underlying \isi{tone} marking reflects our work in progress. For more details, see \citet[42-49]{Kennedy1966} and \citet{Anttila&Bodomo2000}.}

\begin{table}
\begin{tabularx}{\textwidth}{llllll}
\lsptoprule
{Underlying} & {Surface} & & {Underlying} & {Surface} &\\
\midrule
/béé/ &	béé &	‘or’ &	/bóò/ &	bóò  &	‘which’\\
/pɔg-léé/ &	pɔ̀gléé &	‘woman-\textsc{dim}’ &	/tòò-rÍ/ &	tòòrí &	‘ear-\textsc{sg}’\\
/gbɛ́-É/ &	gbɛ́ɛ̀  &	‘leg-\textsc{pl}’ &	/dɔɔ- ́/ &	dɔ́ɔ́ &	‘man-\textsc{sg}’\\
/bar-ÈÉ/ &	bàrɛ̀ɛ́ &	‘leave-\textsc{perf}’ &	/ɔɔ-rV́ / &	ɔ̀ɔ̀rɔ́ &	‘chew-\textsc{impf}’\\
/tɛ́ɛ́sɪ̀ / &	tɛ́ɛ́sɪ̀  &	‘test.\textsc{sg}’ &	/lɔ́ɔ́-rÍ/ &	lɔ́ɔ́rɪ̀  &	‘lorry-\textsc{sg}’\\
\lspbottomrule
\end{tabularx}
\caption{Long mid vowels}
\label{tab:anttila:2}
\end{table}



{However, Kennedy’s insight is nevertheless well founded: long mid vowels are phonologically special.} {The long mid vowels in \tabref{tab:anttila:2} are} {either underlying or result from the concatenation of two underlying short mid vowels; phonologically derived long mid vowels are systematically missing. In particular, the process of vowel lengthening stops short of creating long mid vowels as shown in \tabref{tab:anttila:3}.}

\begin{table}
\begin{tabularx}{\textwidth}{lXlXlX}
\lsptoprule
& {Root} & \multicolumn{2}{l}{{Suffixed form}} & \multicolumn{2}{l}{{N + A Compound}} \\
\midrule
(a) & /bi-/ & bíí-rí & ‘child-\textsc{pl’} & bì-fáá & ‘bad child’\\
& /pì-/ & pìì-rí & ‘rock-\textsc{sg’} & pì -fáá & ‘bad rock’\\
& /kù-/ & kùù-rí & ‘hoe-\textsc{sg}’ & kù-fáá & ‘bad hoe’\\
& /gʊ́-/ & gʊ́ʊ́-rɪ̀ & ‘thorn-\textsc{pl’} & gʊ́-\textsuperscript{!}fáá & ‘bad thorn’\\
\tablevspace
(b) & /pò-/ & pùò-rí & ‘back-\textsc{sg}’ & pò-fáá & ‘bad back’\\
& /nɔ́-/ & nʊ́ɔ́-rɪ̀ & ‘mouth-\textsc{sg}’ & nɔ́-\textsuperscript{!}fáá & ‘bad mouth’\\
& /dò-/ & dò-rí & ‘pig-\textsc{pl}’ & dò-fáá & ‘bad pig’\\
& /dè-/ & dè-rí & ‘room-\textsc{pl}’ & dè-fáá & ‘bad room’\\
& /lɛ̀-/ & lɛ̀-rɪ́ & ‘bead-\textsc{sg}’ & lɛ̀-fáá & ‘bad bead’\\
& /gbɛ́-/ & gbɛ́-rɪ̀ & ‘leg-\textsc{sg}’ & gbɛ́-\textsuperscript{!}fáá & ‘bad leg’\\
\lspbottomrule
\end{tabularx}
\caption{Vowel lengthening in suffixed nouns}
\label{tab:anttila:3} 
\end{table}


\break
{\tabref{tab:anttila:3} shows that the number suffix /-rÍ/ triggers vowel}{ lengthening} {i}{n high vowel stems}{, but not in mid-vowel stems where the result is either a diphthong or the vowel simply fails to lengthen, depending on the lexical item. The noun-adjective compound is given as a diagnostic for the underlying form of the noun: the nouns in \tabref{tab:anttila:3} all have a short stem vowel. In contrast, the} {long mid vowel} {in} {\textit{dɔ́ɔ́}}{ ‘man.}{\textsc{sg}}{’ given in \tabref{tab:anttila:2} is underlying:} {\textit{dɔ̀ɔ̀-}}{\textit{fáá}}{ ‘bad man’.} {Lengthening is lexically conditioned even in high vowel stems: there are words like} {\textit{bí-rì}}{ ‘seed-}{\textsc{sg’}}{ and} {\textit{yí-rì}}{ ‘house-}{\textsc{sg}}{’ where lengthening does not happen. Finally,} {the data illustrate a characteristic}{ \isi{aspect} of \ili{Dagaare} number morphology: /-r}{Í}{/ may mean either singular or plural depending on the stem, an instance of “polarity morphology” that has attracted the attention of semanticists \citep{Grimm2012}.} 


\newpage 
{Vowel lengthening also occurs in singular forms with no overt suffix. \citet{Anttila&Bodomo2009} propose that in such cases the root vowel lengthens in order to satisfy a bimoraic foot template.}

\begin{table}
\begin{tabularx}{\textwidth}{lXlXlX}
\lsptoprule
& {Root} & \multicolumn{2}{l}{{Suffixed form}}  & \multicolumn{2}{l}{{N + A Compound}} \\
\midrule
(a) & /bi-/ & bíé & ‘child.\textsc{sg}’ & bì-fáá & ‘bad child’\\
& /gʊ́-/ & gʊ́ɔ̀ & ‘thorn.\textsc{sg}’ & gʊ́-\textsuperscript{!}fáá & ‘bad thorn’\\
\tablevspace
(b) & /dè-/ & dìé & ‘room.\textsc{sg}’ & dè-fáá & ‘bad room’\\
& /dò-/ & dùó & ‘pig.\textsc{sg}’ & dò-fáá & ‘bad pig’\\
\lspbottomrule
\end{tabularx}
\caption{Vowel lengthening in unsuffixed nouns}
\label{tab:anttila:4}
\end{table}
 

{Here is the reasoning: the singular form is a phonological word; therefore it must contain at least one foot; therefore it must be minimally bimoraic \citep{McCarthyPrince1996}. In \ili{Dagaare} this generalization holds for almost all nouns.}\footnote{We are aware of four monomoraic (CV) nouns: \textit{bâ} ‘father.\textsc{sg}’, \textit{mǎ} ‘mother.\textsc{sg}’, \textit{nû} ‘hand.\textsc{sg’,} \textit{zû} ‘head.\textsc{sg}’.}{  In contrast, function words, weak forms of pronouns, and citation forms of verbs can be monomoraic. The question is why the vowel does not simply lengthen, yielding *}{\textit{bíí}}{, *}{\textit{g}}{\textit{ʊ́ʊ̀}}{,} {\textit{*}}{\textit{dèé}}{, and *}{\textit{dòó}}{.} {} \citet{Anttila&Bodomo2009} propose that this is due to two constraints: {*}{\textit{bíí}}{ and *}{\textit{g}}{\textit{ʊ́ʊ̀}}{} {are blocked by a constraint against word-final high vowels;} {\textit{*}}{\textit{dèé}}{, and *}{\textit{dòó} }{are blocked by} {a constraint against long mid vowels. Crucially, both constraints only apply in phonologically derived environments. The optimal outcome is} {a rising diphthong:} {\textit{bíé}}{,} {\textit{g}}{\textit{ʊ́ɔ̀}}{,} {\textit{dìé}}{, and} {\textit{dùó}}{.} 



{In sum, we have seen that all the nine vowels of \ili{Dagaare} can be underlyingly either short or long \citep{Kennedy1966}. There are also underlying diphthongs, such as} {\textit{t\`ɪ\`ɛ}}{ ‘shoot’,} {\textit{pùòrì}}{ ‘thank’,} {\textit{yíélì}}{ ‘sing’,} {\textit{lʊ́ɔ́r-áá}}{ ‘lion-}{\textsc{sg}}{’. However, long mid vowels} {[ee], [ɛɛ], [oo], [ɔɔ]} {are special in that they cannot be the result of lengthening.}



{This system of \isi{vowel length} may seem complicated and one can reasonably question whether it has anything to do with foot structure. We will now provide new evidence suggesting that it indeed does. We first show that verbs exhibit parallel length alternations, complete with parallel exceptions. Particularly interesting is the action nominal paradigm where the length alternations are entirely regular and the foot template triggers both vowel lengthening and \isi{vowel shortening}.}


\section{Length alternations in verbs}


{The key alternations in the verbal paradigm are illustrated in \tabref{tab:anttila:5}.}{} 

\begin{table}
\begin{tabularx}{\textwidth}{lllllX}
\lsptoprule
& {Root} & {Cit. form} & {Imperf.} & {Nominal} & \\
\midrule
(a) & /ba-/ & bà & bàà-rá & báá-ʊ́ & ‘stick into the ground’\\
& /baa-/ & bàà & bàà-rá & báá-ʊ́ & ‘grow (of child)’\\
\tablevspace
(b) & /bar-/ & bàrɪ̀ & bà-rá & bár-ʊ́ʊ́ & ‘leave’\\
& /bá\`{r}r-/ & bárrɪ̀ & bár-\textsuperscript{!}rá & bár\textsuperscript{!}r-ʊ́ʊ́ & ‘bargain’\\
& /báàr-/ & báárɪ̀ & báá-\textsuperscript{!}rá & báá\textsuperscript{!}r-ʊ́ʊ́ & ‘finish’\\
\lspbottomrule
\end{tabularx}
\caption{Vowel length alternations in Dagaare verbs}
\label{tab:anttila:5}
\end{table}

\textup{The root and} \textup{the citation form are identical except that consonant-final roots acquire a final epenthetic vowel in the citation form, either /i/ or /ɪ/ depending on \textsc{atr}-harmony. This is because a \ili{Dagaare} word must end in a vowel or in the \isi{velar nasal} [ŋ]; in the latter case vowel epenthesis seems optional.}\footnote{This word-final epenthetic /i/ or /ɪ/ is a systematic counterexample to the ban on word-final derived high vowels. It seems that the ban only holds in the lexical phonology and that these epenthetic vowels are postlexical.} \textup{The \isi{imperfective suffix} /-r\'{V}/ copies its \isi{vowel quality} from the root. Our main \isi{focus} is on the action nominals where} \textup{both roots and suffixes alternate. We assume that the underlying form of the suffix is /-ÚÚ/}\textup{, where /U/ stands for a [+high, $-$low, +round] vowel underspecified for [$\pm$}\textsc{atr}\textup{]. Here are the key generalizations. First, a short root vowel lengthens before the suffix, e.g., /ba/ ‘stick into the ground’ becomes} \textit{báá-ʊ}́ \textup{(long root vowel). Second, the suffix vowel is short after vowel-final roots, but long after consonant-final roots, e.g., /ba/ ‘stick into the ground’ yields} \textit{báá-ʊ́} \textup{(short suffix vowel), but /bar/ ‘leave’ yields} \textit{bár-ʊ́ʊ́} \textup{(long suffix vowel).}\footnote{There exists another nominalizing suffix /-bÚ/, which results in doublets such as \textit{dííú {\textasciitilde} dííbú} ‘eating’, \textit{ɪ́ŋʊ́ʊ́ {\textasciitilde} ɪ́mmʊ́} ‘putting’, \textit{wóŋúú {\textasciitilde} wómmú} ‘understanding’, and \textit{zɪ́ŋʊ́ʊ́ {\textasciitilde} zɪ́mmʊ́} ‘sitting’. More examples can be found in \citet{Durand1953}. We have not conducted a systematic study of this suffix variation, but we speculate that it may depend on dialect and speech rate. The variation is not completely free: some verbs allow /-ÚÚ/, but not /-bÚ/, e.g., \textit{pɪ́ɪ́rʊ́ʊ́}/*\textit{pɪ́ɪ́rɪ́bʊ́} ‘sweep’, \textit{sɪ́ɪ́rʊ́ʊ́}/*\textit{sɪ́ɪ́rɪ́\textsuperscript{!}bʊ́} ‘touch’.}

{Tables 6 and 7 illustrate \isi{vowel length} alternations in CV verbs. The above generalizations hold without exception in action nominals: the root vowel is always long and the suffix vowel is always short. Vowel height matters to root vowel lengthening: low and high root vowels lengthen (\tabref{tab:anttila:6}), e.g., /bà/,} {\textit{báá-ʊ}}{ ‘stick into the ground’ and /dì/,} {\textit{díí-ú}}{ ‘eat’, whereas mid root vowels diphthongize (\tabref{tab:anttila:7}), e.g., /kyɛ/,} {\textit{kyɪɛ-ʊ}}{ ‘cut’ and /bɔ/,} {\textit{bʊɔ-ʊ}}{ ‘want, look for’. The verbs are further divided into two sets (a) and (b) based on \isi{vowel length} in the imperfective. We will return to the imperfective shortly.}

\begin{table}
\begin{tabularx}{\textwidth}{lllllX}
\lsptoprule
& {Root} & {Cit. form} & {Imperf.} & {Nominal} & \\
\midrule
(a) & /ba-/ & bà & bàà-rá & báá-ʊ́ & ‘stick into the ground’\\
& /da-/ & dà & dàà-rá & dáá-ʊ́ & ‘buy’\\
& /wa-/ & wà & wàà-ná & wáá-ʊ́ & ‘come’\\
& /kpá-/ & kpá  & kpáá-rà  & kpáá-ʊ̀  & ‘boil’\\
& /la-/ & là  & làà-rá  & láá-ʊ́  & ‘laugh’\\
& /mí-/ & mí  & míí-rè	& míí-ù  & ‘rain’\\
& /bʊ́-/ & bʊ́ & bʊ́ʊ́-rɔ̀ &	bʊ́ʊ́-ʊ̀ & ‘come (of rain)’\\
& /bú-/ & bú & búú-rò	&	búú-ù	 & ‘measure, calculate’\\
& /nyṵ́-/ & nyṵ́ & nyṵ́ṵ́rò̰ &	nyṵ́ṵ́-ṵ̀\footnotemark[contrastive]& ‘drink’\\
& /zú{} -/ & z\'u{}  & zúú-rò &	zúú-ù & ‘steal’\\
\tablevspace
(b) & /tá-/ & tá & tá-rà &	táá-ʊ̀ & ‘reach’\\
& /ɪ-/ & ɪ̀ & ɪ̀-rɛ́ &	ɪ́ɪ́-ʊ́ & ‘do’\\
& /dɪ̂-/ & dɪ̂ & dɪ́-\textsuperscript{!}rɛ́ & dɪ́ɪ́-\textsuperscript{!}ʊ́  & ‘take’\\
& /di-/ & dì & dì-ré & díí-ú & ‘eat’\\
& /kʊ-/ & kʊ̀ & kʊ̀-rɔ́ &	kʊ́ʊ́-ʊ́ & ‘give, offer’\\
& /yí-/ & yí{}  & yí-rè &	yíí-ù  & ‘divorce a male’\\
\lspbottomrule
\end{tabularx} 
\caption{CV verbs, low and high vowel roots}
\label{tab:anttila:6}
\end{table}

\footnotetext[contrastive]{We mark contrastive nasalization with a subscript tilde to avoid clutter. The interpretation of nasalized vowels is controversial. \citet[12]{Kennedy1966} derives them via absolute neutralization from vowel-/m/ sequences, e.g., /fààm/ $\rightarrow$\textit{fà̰à}̰ ‘fail’: “There is a clear hole in the final nasal pattern. Though n and ŋ occur word final, m does not. Therefore nasalized vowels which are not contiguous to nasals are interpreted as vowel-m sequences.\textit{”} \citet[9]{Bodomo1997} assumes that nasalization is phonemic and notes that it is mostly found in long vowels.}

\begin{table}
\begin{tabularx}{\textwidth}{lllllX}
\lsptoprule
& {Root} & {Cit. form} & {Imperf.} & {Nominal} & \\
\midrule
(a) &	/kyɛ-/	&kyɛ̀	&kyɪ̀ɛ̀-rɛ́	&kyɪ́ɛ́-ʊ́	&‘cut’\\
	&/kpɛ-/&	kpɛ̀	&	kpɪ̀ɛ̀-rɛ́	&kpɪ́ɛ́-ʊ́	&‘enter’\\
	&/gyɛ́-/	&gyɛ́	&gyɪ́ɛ́-rɛ̀	&gyɪ́ɛ́-ʊ̀	&‘refuse to take’\\
	&/ŋmɛ-/	&ŋmɛ̀	&ŋmɪ̀ɛ̀-rɛ́	&ŋmɪ́ɛ́-ʊ́	&‘beat’\\
	&/gbe-/&	gbè	&gbìè-ré	&gbíé-ú	&‘grind roughly’\\
	&/bɔ́-/	&bɔ́	&bʊ́ɔ́-rɔ̀	&bʊ́ɔ́-ʊ̀	&‘want, look for’\\
	&/kɔ́-/	&kɔ́	&kʊ́ɔ́-rɔ̀	&kʊ́ɔ́-ʊ̀	&‘farm’\\
	&/yɔ́-/	&yɔ́	&yʊ́ɔ́-rɔ̀ &	yʊ́ɔ́-ʊ̀	&‘roam’\\
\tablevspace
(b)	&/ko-/	&kò	&kò-ró	&kúó-ú	&‘dry’\\
	&/kó-/	&kó	&kó-rò	&kúó-ù	&‘get ready for rain’\\
	&/tɛ́-/	&tɛ́	&tɛ́-rɛ̀	&tɪ́ɛ́-ʊ̀	&‘display’\\
	&/zo-/&	zò	&zò-ró		&zóó-ú\footnotemark[action] 	&‘run’\\
	&/nyɛ́-/&	nyɛ́	&nyɛ́-rɛ̀ &	nyáá-ʊ̀\footnotemark[verb] &	‘see, understand’ \\
\lspbottomrule
\end{tabularx}
\caption{CV verbs, mid vowel roots}
\label{tab:anttila:7}
\end{table}
\footnotetext[action]{The action nominalization \textit{zóó-ú} is a counterexample to our generalization that there are no derived long mid vowels.  Another such verb is /go-/: \textit{gò, gò-ró, góó-ú} ‘wait for, keep watch’.}
\footnotetext[verb]{With this verb, vowel lengthening results in [áá], not in the expected [ɪ́ɛ́].}

     The imperfective paradigm is more complicated. The suffix /-r\'{V}/ copies the root vowel except that a high vowel becomes mid, reflecting the constraint\linebreak against word-final derived high vowels, e.g., /di/, \textit{dì-ré} ‘eat-\textsc{impf}’. The verbs are further divided into two sets (a) and (b) based on whether the root vowel undergoes lengthening and/or diphthongization. The choice is phonologically unpredictable: we have vowel lengthening in /ba/ \textit{bàà-rá} ‘stick into the ground-\textsc{impf}’, but not in /tá/ \textit{tá-rà} ‘reach-\textsc{impf}’ (\tabref{tab:anttila:6});  we have diphthongization in /gyɛ́-/  \textit{gyɪ́ɛ́-rɛ̀ } ‘refuse to take’, but not in /nyɛ́-/ \textit{nyɛ́-rɛ̀}  ‘see, understand’ (\tabref{tab:anttila:7}). This makes the \isi{imperfective suffix} /-r{V́}/ look rather similar to the number suffix /-rÍ/ which also exhibits lexically conditioned vowel lengthening.

\begin{table}
\todo[inline]{check empty and double table cells}
\begin{tabularx}{\textwidth}{lllllX}
\lsptoprule
& {Root} & {Cit. form} & {Imperf.} & {Nominal} & \\
\midrule
(a)	&/baa-/	&bàà	&bàà-rá	&báá-ʊ́	&‘grow (of child)’\\
	&/fáà-/ &    	fáà &	fáá-\textsuperscript{!}rá	& fáá-\textsuperscript{!}ʊ́	&‘seize’\\
	&	&	wàá	& wàà-rá	&wáá-ʊ́	& ‘be’\\
	&/tɪɛ-/&	tɪ̀ɛ̀	& tɪ̀ɛ̀-rɛ́	&tɪ́ɛ́-ʊ́	& ‘shoot’\\
	&/fɪ̰ɛ̰-/&	fɪ̰̀ɛ̰̀	& fɪ̰̀ɛ̰̀-rɛ̰́	&fɪ̰́ɛ̰́-ʊ̰́	& ‘whip’\\
	&/dɪ̰ɛ̰-/&	dɪ̰̀ɛ̰̀	& dɪ̰̀ɛ̰̀-nɛ̰́	&dɪ̰́ɛ̰́-ʊ̰́	& ‘play’\\
	&/yuo-/	&yùò	& yùò-ró	&yúó-ú	& ‘open’\\
\tablevspace
(b)	&&tàá	& tá-\textsuperscript{!}rá	& táá-ʊ́ & ‘have, own’\\
	&/gaa-/	&gàà	& gɛ̀-rɛ́	&gáá-ʊ́	& ‘go’\\
\lspbottomrule
\end{tabularx}
\caption{CVV verbs}
\label{tab:anttila:8}
\end{table}

 \tabref{tab:anttila:8} illustrates the same paradigms in CVV verbs. The pattern in action nominals is the same as with CV verbs: the root vowel is long and the suffix vowel is short. In imperfectives the root vowel typically remains long, but there is an interesting minor pattern: some verbs undergo vowel \textit{shortening} in the imperfective, e.g., \textit{tá-\textsuperscript{!}}\textit{rá} ‘have-\textsc{impf}’ and \textit{gɛ̀-rɛ́} ‘go-\textsc{impf’}.\footnotemark[ablaut] These verbs provide evidence for a process of root \isi{vowel shortening} which was not visible in CV verbs where we could only see root vowel lengthening. The verbs ‘be’ and ‘have’ are tonally idiosyncratic and given our uncertainty about the analysis we do not give underlying forms for them.

\footnotetext[ablaut]{The ablaut in \textit{gɛ̀-rɛ́} ‘go-\textsc{impf}’ is specific to this lexical item.}


\begin{table}
\begin{tabularx}{\textwidth}{l lXlX X} 
\lsptoprule
& {Root} & {Cit. form} & {Imperf.} & {Nominal} & \\
\midrule
(a)&	/bɔŋ-/&	bɔ̀ŋɪ̀	&bɔ̀n-nɔ́ &	bɔ́ŋ-ʊ́ʊ́&	‘know’\\
&	/dʊ́g-/&	dʊ́gɪ́	&dʊ́g-rɔ̀	&	dʊ́g-ʊ̀ʊ̀	&‘boil, brew’\\
&	/ɪŋ-/&	ɪ̀ŋɪ̀	&ɪ̀ŋ-nɛ́	&ɪ́ŋ-ʊ́ʊ́&	‘put’\\
&	/biŋ-/&	bìŋì	&bìn-né{\textasciitilde}bìŋ-né &	bíŋ-úú	&‘put down’\\
&	/sɪ̂ŋ-/&	sɪ́ŋɪ̀	&sɪ́ŋ-\textsuperscript{!}nɛ́&	sɪ́\textsuperscript{!}ŋ-ʊ́ʊ́&	‘equal’\\
&	/pɔg-/&	pɔ̀gɪ̀	&pɔ̀g-rɔ́&	pɔ́g-ʊ́ʊ́	&‘(en)close’\\
&	/sag-/&	sàgɪ̀	&sàg-rá&	ság-ʊ́ʊ́	&‘answer’\\
&	/sɛ́g-/&	sɛ́gɪ́	&sɛ́g-rɛ̀&	sɛ́g-ʊ̀ʊ̀&	‘write’\\
&	/sʊŋ-/&	sʊ̀ŋɪ̀	&sʊ̀ŋ-nɔ́	&sʊ́ŋ-ʊ́ʊ́	&‘help’\\
\tablevspace
(b)	&/bar-/&	bàrɪ̀	&bà-rá	&bár-ʊ́ʊ́	&‘leave’\\
&	/bur-/&	bùrì	&bù-ró	& búr-úú		&‘soak’\\
&	/ɛ̂r-/	&ɛ́rɪ̀	&ɛ́-\textsuperscript{!}rɛ́	&ɛ́\textsuperscript{!}r-ʊ́ʊ́	&‘grind’\\
&	/mar-/&	màrɪ̀	&mà-rá&	már-ʊ́ʊ́	&‘paste’\\
&	/sar-/	&sàrɪ̀	&sà-rá&	sár-ʊ́ʊ́	&‘slip’\\
&	/sɔ́r-/	&sɔ́rɪ́	&sɔ́-rɔ̀	&sɔ́r-ʊ̀ʊ̀	&‘count’\\
&	/woŋ-/&	wòŋì	&wò-nó	&wóŋ-úú&	‘understand’\\
&	/yel-/&	yèlì	&yè-lé	&yél-úú&	‘speak’\\
&	/zɪŋ-/	&zɪ̀ŋɪ̀	&zɪ̀-nɛ́	&zɪ́ŋ-ʊ́ʊ́	&‘sit’\\
\tablevspace
(c)	&/bal-/&	bàlɪ̀	&bàl-lá{\textasciitilde}bàl-á	&bál-ʊ́ʊ́		&‘be tired’\\
\lspbottomrule
\end{tabularx}
\caption{CVC verbs}
\label{tab:anttila:9}
\end{table}

\newpage 
We now turn to consonant-final roots. \tabref{tab:anttila:9} illustrates the same paradigms in CVC roots. Here the action nominal suffix vowel is always long. The imperfective paradigm shows mixed behavior of the familiar kind: the initial syllable may be heavy (CVC.CV) as in (a) or light (CV.CV) as in (b), depending on the verb. One and the same verb may even allow both forms as in (c): /bal-r\'{V}/ ‘be.tired-\textsc{impf’} may come out either as \textit{bàl-lá} or \textit{bàl-á}. Minimal pairs like /bɔŋ{}-r\'{V}/, \textit{bɔ̀n-nɔ́} ‘know-\textsc{impf}’ with a heavy initial syllable and /wòŋ{}-r\'{V}/ \textit{wò-nó} ‘hear-\textsc{impf}’ with a light initial syllable suggest that the choice between the two is lexical. Note that the suffixal /r/ assimilates in place and/or manner to the root-final consonant; the details will be set aside here.\footnote{The CVC verb /gb\^{i}r-/ ‘sleep’ has the exceptional paradigm \textit{gbi }\textit{rì}, \textit{gbí}\textit{\textsuperscript{!}}\textit{ré{} }, \textit{gʊ́ɔ́}\textit{\textsuperscript{!}}\textit{ʊ́}. The action nominal is exceptional in having a short suffix vowel, but since it differs segmentally from the root in several ways, including its [\textsc{atr}] value, we suspect it is probably based on a different lexeme.}
     
     The same paradigms for CVCC verbs are shown in \tabref{tab:anttila:10}. Again, the vowel in the action nominal suffix is always long.  This time even the imperfective paradigm is uniform: the initial syllable is always heavy (CVC.CV), with no free or lexical variation.

\begin{table}
\begin{tabularx}{\textwidth}{llXXX} 
\lsptoprule
{Root} & {Cit. form} & {Imperf.} & {Nominal} & \\
\midrule
/bârr-/&	bárrɪ̀&	bár-\textsuperscript{!}rá&	bár\textsuperscript{!}r-ʊ́ʊ́&	‘bargain’\\
/bɛll-/&	bɛ̀llɪ̀&	bɛ̀l-lɛ́&	bɛ́ll-ʊ́ʊ́&	‘deceive’\\
/gɔll-/&	gɔ̀llɪ̀&	gɔ̀l-lɔ́&	gɔ́ll-ʊ́ʊ́&	‘go around’\\
/kann-/&	kànnɪ̀&	kàn-ná&	kánn-ʊ́ʊ́&	‘learn’\\
/kyɛll-/&	kyɛ̀llɪ̀&	kyɛ̀l-lɛ́&	kyɛ́ll-ʊ́ʊ́&	‘listen’\\
/mánn-/&	mánnɪ̀&	mán-\textsuperscript{!}ná&	mán\textsuperscript{!}n-ʊ́ʊ́&	‘measure’\\
/nyunn-/&	nyùnnɪ̀&	nyùn-nó&	nyúnn-úú&	‘smell’\\
/pɛgl-/&	pɛ̀glɪ̀&	pɛ̀g-lɛ́&	pɛ́gl-ʊ́ʊ́&	‘carry’\\
/pɛnn-/&	pɛ̀nnɪ̀&	pɛ̀n-nɛ́&	pɛ́nn-ʊ́ʊ́&	‘rest’\\
/sɪ̂ll-/&	sɪ́llɪ̀&	sɪ́l-\textsuperscript{!}lɛ́&	sɪ́l\textsuperscript{!}l-ʊ́ʊ́&	‘tell stories’\\
/tall-/&	tàllɪ̀&	tàl-lá&	táll-ʊ́ʊ́&	‘walk fast’\\
\lspbottomrule
\end{tabularx}
\caption{CVCC verbs}
\label{tab:anttila:10}
\end{table}

     Finally, \tabref{tab:anttila:11} illustrates CVVC verbs. The action nominal suffix vowel is again always long and the imperfective paradigm is uniformly CVV.CV, with no variation.

\begin{table}
\begin{tabularx}{\textwidth}{llXXl}
\lsptoprule
 {Root} & {Cit. form} & {Imperf.} & {Nominal} & \\
\midrule
/báàr-/&	báárɪ̀&	báá-\textsuperscript{!}rá&	báá\textsuperscript{!}r-ʊ́ʊ́&	‘finish’\\
/naan-/&	nàànɪ̀&	nàà-ná	&	náán-ʊ́ʊ́&	‘get ready, develop’\\
/saal-/&	sààlɪ̀&	sààl-á&	sáál-ʊ́ʊ́ &	‘sharpen’\\
/sa̰a̰ŋ-/&	sà̰à̰&	sà̰à̰-ná̰&	sá̰á̰ŋ-ʊ̰́ʊ̰́&	‘spoil’\\
/piir-/&	pììrì&	pìì-ré&	píír-úú&	‘discover’\\
/pɪɪr-/&	pɪ̀ɪ̀rɪ̀&	pɪ̀ɪ̀-rɛ́&	pɪ́ɪ́r-ʊ́ʊ́&	‘sweep’\\
/sɪ́ɪ̀r-/&	sɪ́ɪ́rɪ̀&	sɪ́ɪ́-\textsuperscript{!}rɛ́&	sɪ́ɪ́\textsuperscript{!}r-ʊ́ʊ́&	‘touch’\\
/yíèl-/&	yíélì&	yíé-\textsuperscript{!}lé&	yíé\textsuperscript{!}l-úú&	‘sing’\\
/gíèr-/&	gíérì&	gíé-\textsuperscript{!}ré&	gíé\textsuperscript{!}r-úú&	‘belch’\\
/fúòr-/&	fúórì&	fúó-\textsuperscript{!}ró&	fúó\textsuperscript{!}r-úú&	‘sip’\\
/puor-/&	pùòrì&	pùò-ró	&	púór-úú&	‘thank, greet, pray’\\
/kɔɔr-/&	kɔ̀ɔ̀rɪ̀&	kɔ̀ɔ̀-rɔ́&	kɔ́ɔ́r-ʊ́ʊ́&	‘delay’\\
/ɔɔr-/&	ɔ̀ɔ̀rɪ̀&	ɔ̀ɔ̀-rɔ́&	ɔ́ɔ́r-ʊ́ʊ́&	‘chew’\\
\lspbottomrule
\end{tabularx}
\caption{CVVC verbs}
\label{tab:anttila:11}
\end{table}


Having the overtly vowel-final \textit{sà̰à̰} ‘spoil’ listed among CVVC verbs deserves a comment. The citation form is clearly vowel-final, i.e., CVV, but there is good evidence that the root is underlyingly /saaŋ/: the \isi{velar nasal} surfaces in the action nominal \textit{sá̰á̰ŋ-ʊ̰́ʊ̰́}. It is as if the root-final /ŋ/ were present when the suffix \isi{vowel length} is determined and then deleted leaving its nasal component behind, resulting in \textit{sà̰à̰}. The coronal nasal in the imperfective \textit{sà̰à̰-ná̰} results from place assimilation with the initial coronal consonant of the \isi{imperfective suffix} /-r\'{V}/. Parallel examples from nouns include \textit{kʊ̰̀ɔ̰́} ‘water’, underlyingly /kɔ̀ŋ-/, as in \textit{kɔ̀ŋ-fáá} ‘bad water’. In the free form the velar stop deletes leaving nasalization behind and the mid vowel diphthongizes to fill the foot template, resulting in (\textit{kʊ̰̀ɔ̰́}).



Not all verbs with nasal vowels behave in the same way. Compare \textit{sà̰à̰} ‘spoil’ to \textit{dɪ̰̀ɛ̰̀} ‘play’ and \textit{fɪ̰̀ɛ̰̀} ‘whip’. Unlike \textit{sà̰à̰}, the latter two must be underlyingly vowel-final since the corresponding action nominals are \textit{dɪ̰́ɛ̰́-ʊ̰́} and \textit{fɪ̰́ɛ̰́-ʊ̰́}, with a short suffix vowel. However, the two differ in the imperfective: in \textit{dɪ̰̀ɛ̰̀-nɛ̰́} the coronal stop of the \isi{imperfective suffix} /-r\'{V}/ becomes a nasal, whereas in \textit{fɪ̰̀ɛ̰̀-rɛ̰́} it does not. We do not have a satisfactory analysis to offer and must leave the topic with these preliminary remarks.


\section{Proposal}


\textup{Our claim is that these \isi{vowel length} alternations serve to optimize metrical structure. The key assumption is that the action nominal suffix /ÚÚ/} \textup{subcategorizes for a foot: the left edge of /-ÚÚ/ strives to be aligned with the \isi{right edge} of a foot. This demands a well-formed foot that respects alignment. Vowel length adjustments are a way to achieve this goal: a short root vowel lengthens to make up a minimal foot and a long suffix vowel shortens because it is unstressed.} \\
\textup{We illustrate the analysis in \tabref{tab:anttila:12} with two vowel-final verbs: /ba/ ‘stick into the ground’ and /baa/ ‘grow (of child)’. The processes are described in terms of informal ordered rules. Foot boundaries are marked with parentheses and imply syllable boundaries.}\\

\begin{table}
\begin{tabularx}{\textwidth}{XXXl}
\lsptoprule
{Process} & /ba-\'ʊ\'ʊ / & /baa-\'ʊ\'ʊ / & {Motivation}\\
\midrule
Footing & (bá)ʊ́ʊ́ & (báá)ʊ́ʊ́ & Initial foot needed\\
V lengthening & (báá)ʊ́ʊ́ & -- & No degenerate feet\\
V shortening & (báá)ʊ́ & (báá)ʊ́ & No unstressed VV\\
& [bááʊ́] & [bááʊ́] & \\
\lspbottomrule
\end{tabularx}
\caption{The derivation of vowel length in V-final roots}
\label{tab:anttila:12}
\end{table}

/ba-ʊ́ʊ́/ undergoes both root vowel lengthening and suffix \isi{vowel shortening}; /baa-ʊ́ʊ́/ only undergoes suffix \isi{vowel shortening}. In both cases, the outcome is (\textit{báá})\textit{ʊ́}, where the syllable containing the suffix vowel falls outside the foot, i.e., it is extrametrical. \citet[4]{Kennedy1966} calls such word-final light syllables \textit{secondary syllables}. Their \isi{prosodic structure} is illustrated in \REF{ex:anttila:1} below.

% Edwin: Changed \citet[4]{Kennedy1996} to \citet[4]{Kennedy1966} based on other similar citations in this paper

\newpage 

\ea\label{ex:anttila:1} A phonological word with a secondary syllable: (\textit{báá})\textit{ʊ́}\\
\vspace*{-1.5cm}       
%\label{bkm:Ref482774629}
%      $\omega $      word
%
%    $\varphi $         foot
%
%             $\sigma $     $\sigma $          syllable
%
%   $\mu $ $\mu $   $\mu $        mora
%
%         b      a    ʊ       melody
%
%        [b     a :   ʊ]     phonetics
\begin{tikzpicture}[sibling distance=2mm]
\Tree 
[.\node[fill=white] (pseudoroot) {}; %add a "super node" to make the tree tiers align vertically
  \edge[draw=none];
  [.\node (rootfill) {};
  	\edge[draw=none];	
	[.\node (omega) {$\omega$};
        [.\node (phi) {$\varphi$}; 
          [.\node (sigma) {$\sigma$};
          	\edge[draw=none];
          	[.\node (filler) {}; 
          		\edge[draw=none];
          		[.\node (b) {b}; 
          			\edge[draw=none];
          			[.{\big[}b ]
          		]
          	]
          	[.\node (mu1) {$\mu$}; 
          		\edge[draw=none]; {}
          		[.\node (a) {a}; 
          			\edge[draw=none];
          			[.{aː} ]
          		]
          	]
          	[.\node (mu2) {$\mu$}; ]
          ]
        ]
      ]
    \edge[draw=none];
    [.\node (blank1) {};
      \edge[draw=none];
      [.\node (blank2) {}; 
      	\edge[draw=none];
      	[.\node (sigma2) {$\sigma$}; 
      		[.\node (mu3) {$\mu$};
      			[.\node (oops1) {ʊ};
      				[.\node (oops2) {ʊ{\big]}}; ]
      			]
      		]
      	]
      ]
    ]
  ] 
  \edge[draw=none];
    [.\node (blank1) {};
      \edge[draw=none];
      [.\node (word) {word}; 
      	[.\node (foot) {foot}; 
      	  [.\node (syll) {syllable};
      		[.\node (mora) {mora};
      		  [.\node (mel) {melody}; 
      		  	[.\node (phon) {phonetics}; ]
      		  ]
      		]
      	  ]
      	]
      ]
   	]      	
   	]
\draw (mu2.south) -- (a.north);  
\draw (sigma.south) -- (b.north); 
\end{tikzpicture} 

\z

Consonant-final roots are different. Consider /bar/ ‘leave’: if suffix alignment were all that counts the input /bar-ʊ́ʊ́/ should be footed *(\textit{bár})\textit{ʊ́ʊ́}, but that is not possible because it implies the syllabification *\textit{bá}\textit{r.ʊ́ʊ́} which is illegal in \ili{Dagaare}. Suffix alignment and word prosody are driven into conflict and word prosody wins: the solution is (\textit{bá.rʊ́})\textit{ʊ́} where the long suffix vowel is split into two light syllables: the first is incorporated into the foot and the second remains extrametrical. This implies the syllabification CV.CV.V which is legal in \ili{Dagaare} \citep[3-4]{Kennedy1966}. \tabref{tab:anttila:13} illustrates this for the consonant-final verbs /bar/ ‘leave’, /bár̀r/ ‘bargain’ and /báàr/ ‘finish’ in terms of informal ordered rules. The \isi{prosodic structure} of \textit{bárʊ́ʊ́} is shown in \REF{ex:anttila:2} below.

\begin{table}
\begin{tabularx}{\textwidth}{lllll}
\lsptoprule
{Process} & /bar-ʊ́ʊ́/ & /bá{r̀}r-ʊ́ʊ́/ & /báàr-ʊ́ʊ́/ & {Motivation}\\
\midrule
Footing & (bá.rʊ́)ʊ́ & (bár.\textsuperscript{!}rʊ́)ʊ́ & (báá.\textsuperscript{!}rʊ́)ʊ́ & Initial foot needed\\
V lengthening & -- & -- & -- & No degenerate feet\\
V shortening & -- & -- & -- & No unstressed VV\\
& [bárʊ́ʊ́] & [bár\textsuperscript{!}rʊ́ʊ́] & [báá\textsuperscript{!}rʊ́ʊ́] & \\
\lspbottomrule
\end{tabularx}
\caption{The derivation of vowel length in C-final roots}
\label{tab:anttila:13}
\end{table}

\newpage 
\ea\label{ex:anttila:2} A phonological word with a secondary syllable: (\textit{bárʊ́})\textit{ʊ́}\\
\vspace*{-1.5cm}       
%\label{bkm:Ref482782387}
%            $\omega $    word
%
%         $\varphi $         foot
%
%             $\sigma $         $\sigma $   $\sigma $   syllable
%
%    $\mu $         $\mu $   $\mu $      mora
%
%          b    a    r    ʊ   ʊ     melody
%
%        [b    a     r    ʊ   : ]   phonetics
%        
        
      
\begin{tikzpicture}[sibling distance=2mm]
\Tree 
[.\node[fill=white] (pseudoroot) {}; %add a "super node" to make the tree tiers align vertically
    \edge[draw=none];
  [.\node (rootfill) {};
  	\edge[draw=none];	
	[.\node (omega) {$\omega$};
        [.\node (phi) {$\varphi$}; 
          [.\node (sigma1) {$\sigma$};
          	\edge[draw=none];
          	[.\node (filler) {}; 
          		\edge[draw=none];
          		[.\node (b) {b}; 
          			\edge[draw=none];
          			[.{\big[}b ]
          		]
          	]
          	[.\node (mu1) {$\mu$}; 
          		[.\node (a) {a}; 
          			\edge[draw=none];
          			[.{a} ]
          		]
          	]
          ]
          [.\node (sigma2) {$\sigma$};
          	\edge[draw=none];
          	[.\node (filler) {}; 
          		\edge[draw=none];
          		[.\node (roops) {r}; 
          			\edge[draw=none];
          			[.r ]
          		]
          	]
          	[.\node (mu2) {$\mu$}; 
          		[.\node (ʊ) {ʊ}; 
          			\edge[draw=none];
          			[.ʊ ]
          		]
          	]
          ]
        ]
      ]
    \edge[draw=none];
    [.\node (blank1) {};
      \edge[draw=none];
      [.\node (blank2) {}; 
      	\edge[draw=none];
      	[.\node (sigma3) {$\sigma$}; 
      		[.\node (mu3) {$\mu$};
      			[.\node (oops1) {ʊ};
      			    \edge[draw=none];
      				[.\node (oops2) {\hspace{10pt}ː{\big]}}; ]
      			]
      		]
      	]
      ]
    ]
  ]
\edge[draw=none];  
    [.\node (blank1) {};
      \edge[draw=none];
      [.\node (word) {word}; 
      	[.\node (foot) {foot}; 
      	  [.\node (syll) {syllable};
      		[.\node (mora) {mora};
      		  [.\node (mel) {melody}; 
      		    \edge[draw=none];
      		  	[.\node (phon) {phonetics}; ]
      		  ]
      		]
      	  ]
      	]
      ]
   	]      	
]
\draw (sigma1.south) -- (b.north); 
\draw (sigma2.south) -- (roops.north); 
\end{tikzpicture} 

\z

Summarizing, \isi{vowel length} alternations in \ili{Dagaare} action nominals can be understood from a metrical perspective. The three key facts, namely vowel lengthening in CV roots, suffix \isi{vowel shortening} after vowel-final roots and absence of suffix \isi{vowel shortening} after consonant-final roots receive a unified explanation. In the next section we will outline an optimality-theoretic analysis of action nominals. 

\section{Analysis}

\subsection{Constraints}

To keep things simple we will make the following assumptions. \ili{Dagaare} words have an initial trochaic foot; feet are binary under syllabic or moraic analysis; and degenerate feet, e.g., *(\textit{ba}), and ternary feet, e.g., *(\textit{ba.rʊ.ʊ}), are excluded. At most one syllable may be extrametrical: (\textit{baa}.\textit{ʊ})\textit{ʊ} is possible, but *(\textit{baa})\textit{ʊ.ʊ} is not. Candidates that violate these high-ranking constraints will not be mentioned.

Four phonological constraints are needed to express the generalizations informally outlined in earlier sections. These constraints are given in \tabref{tab:anttila:14}.

\begin{table}
\begin{tabularx}{\textwidth}{XX}
\lsptoprule
\textsc{Weight-to-Stress Principle} & ‘No unstressed heavy syllables’\\
\textsc{Max(V)} & ‘No \isi{vowel deletion}’\\
\textsc{Dep(V)} & ‘No vowel insertion’\\
\textsc{Align(Suffix, L, Foot, R)} & ‘The left edge of a suffix coincides with the \isi{right edge} of a foot’\\
\lspbottomrule
\end{tabularx}
\caption{Four constraints}
\label{tab:anttila:14}
\end{table}

The Weight-to-Stress Principle (WSP, \citealt{Prince1990}) punishes unstressed heavy syllables. It is satisfied in (\textit{báá})\textit{ʊ́} where the suffix vowel has shortened and surfaces as the light extrametrical syllable \textit{ʊ́} that lacks an onset. It is also satisfied in (\textit{bár.rʊ́})\textit{ʊ́} where the long suffix vowel has been parsed into two light syllables: the tail of the foot \textit{rʊ́} and the light extrametrical syllable \textit{ʊ́} that lacks an onset. The latter is \posscitet{Kennedy1966} ``{secondary syllable}.'' The WSP is violated in *(\textit{báá})\textit{ʊ́ʊ́}, *(\textit{bár})\textit{rʊ́ʊ́} and *(\textit{bár.rʊ́ʊ́}) where the long suffix vowel is parsed as a single heavy syllable.\footnote{An anonymous reviewer notes that the word /dàgáárɪ̀/ ‘the \ili{Dagaare} language’ violates the WSP given a left-aligned trochee, i.e., (\textit{d\`a.g\'a\'a})\textit{r\`ɪ} and wonders why the vowel does not shorten. Two explanations seem possible. First, this could be an instance of \isi{nonderived environment blocking} \citep{Kiparsky1993}. Second, the intuitively strong syllable is the penult, suggesting the foot structure \textit{dà}(\textit{gá}\textit{á}\textit{rɪ̀}). It should be pointed out that trisyllabic and longer words in \ili{Dagaare} are often right-headed compounds with the morphological structure $\sigma $+$\sigma \sigma $, e.g., \textit{lá}\textit{bɪ́}\textit{rɪ̀} ‘small axe’ from \textit{lá}\textit{rɪ̀} + \textit{bɪ́}\textit{rɪ̀} ‘axe-\textsc{sg} + seed-\textsc{sg}’. It is possible that /dágáárɪ̀/ is etymologically a compound, i.e., /dá+gáárɪ̀/, although synchronically opaque.} 

\subsection{Deriving vowel length}

The four constraints in \tabref{tab:anttila:14} allow us to derive the \isi{vowel length} alternations in action nominals. We start with CV stems. Tableau \REF{ex:anttila:3} establishes the crucial rankings. To simplify presentation, we have omitted \isi{tone} and simply assume the correct \isi{vowel harmony} (\textsc{atr}, rounding). Candidates with ternary feet, degenerate feet, and multiple extrametrical syllables are systematically omitted.

\ea 
Vowel length with CV roots \label{ex:anttila:3}
%\label{bkm:Ref483251457}

\begin{tabularx}{.9\textwidth}{lll|| C:C|C:C|}
% \lsptoprule
\hline\hline
\multicolumn{3}{l||}{/ba-ʊʊ/} & WSP & \textsc{Align} & \textsc{Dep(V)} & \textsc{Max(V)}\\
\hline\hline
(a)	&\ding{43}		&(baa)ʊ 	&  	&  	&\shadecell 1 	&\shadecell 1\\
(b)	&   			&(ba.ʊ)ʊ 	&  	& 1! 	&\shadecell  	&\shadecell \\
(c)	&   			&(ba.ʊʊ) 	&\shadecell 1 	&\shadecell 1 	&\shadecell  	&\shadecell \\
(d)	&   			&(baa)ʊʊ 	& 1! 	&  	&\shadecell 1 	&\shadecell \\
(e)	&   			&(baa.ʊʊ) 	&\shadecell 1! 	&\shadecell 1 	&\shadecell 1 	&\shadecell \\
(f)	&   			&(ba.ʊ) 	&\shadecell  	&\shadecell 1 	&\shadecell  	&\shadecell 1\\
\hline\hline
\end{tabularx}
\z
\textup{The winner (a) exhibits both suffix \isi{vowel shortening} and root vowel lengthening}\textup{. T}\textup{he faithful candidate (b) is \isi{perfect} in every way except that it fatally misaligns the suffix and foot boundaries. Since} \textsc{Align} \textup{dominates both faithfulness constraints,} \textsc{Max(V)} \textup{and} \textsc{Dep(V)}\textup{, the result is a double adjustment of \isi{vowel shortening} and vowel lengthening. Candidates (c), (e), and (f) are grayed out to show that they are harmonically bounded: they can never win no matter how the constraints are ranked.} \\
\textup{We now turn to CVV roots illustrated in Tableau} (4)\textup{. In this case, only suffix \isi{vowel shortening} is needed in order to satisfy the} \textsc{WSP}\textup{:}\\

\ea  Vowel length with CVV roots \label{ex:anttila:4}
%\label{bkm:Ref483918070}
\begin{tabularx}{.9\textwidth}{lll|| C:C|C:C|}
% \lsptoprule
\hline\hline
\multicolumn{3}{l||}{/baa-ʊʊ/} & WSP & \textsc{Align} & \textsc{Dep(V)} & \textsc{Max(V)}\\
\hline\hline
(a)& \ding{43}  &(baa)ʊ 	&  	&  	&\shadecell  	&\shadecell 1\\
(b)&   		&(ba.ʊ)ʊ 	&\shadecell  	&\shadecell 1 	&\shadecell  	&\shadecell 1\\
(c)&   		&(ba.ʊʊ) 	&\shadecell 1 	&\shadecell 1 	&\shadecell  	&\shadecell 1\\
(d)&   		&(baa)ʊʊ 	& 1! 	&  	&\shadecell  	&\shadecell \\
(e)&   		&(baa.ʊʊ) 	&\shadecell 1! 	&\shadecell 1 	&\shadecell  	&\shadecell \\
(f)&   		&(ba.ʊ) 	&\shadecell  	&\shadecell 1 	&\shadecell  	&\shadecell 2\\
\hline\hline
\end{tabularx}
\z

Consonant-final roots behave differently. What sets them apart from vowel-final roots is that they inevitably violate \textsc{Align} when combined with a vowel-initial suffix. Given the input /CVC-VV/ the best-aligned candidate is (\textit{CVC})\textit{VV} where the suffix boundary is crisply aligned with the foot boundary. But this foot structure entails the syllabification *\textit{CVC.VV} which is illegal in \ili{Dagaare}.\footnote{A full analysis of \ili{Dagaare} \isi{syllable structure} cannot be undertaken here. Here we simply assume an undominated locally conjoined constraint \textsc{Onset} \& \textsubscript{L}\textsc{*Coda} that is violated by the syllabification C.V where the first syllable has a coda and the second syllable has no onset. Other analyses are no doubt possible.} We need a better syllabification, but that will inevitably violate \textsc{Align}.{} This makes alignment irrelevant with consonant-final roots because it will have to be violated no matter what. We illustrate this for CVC roots in Tableau \REF{ex:anttila:5}. The winner (\textit{ba.rʊ})\textit{ʊ} has the \isi{syllable structure} CV.CV.V which is  legal in \ili{Dagaare}.

\ea  Vowel length with CVC roots  \label{ex:anttila:5}
\begin{tabularx}{.9\textwidth}{lll|| C:C|C:C|}
\hline\hline
\multicolumn{3}{l||}{/bar-ʊʊ/} & WSP & \textsc{Align} & \textsc{Dep(V)} & \textsc{Max(V)}\\
\hline\hline
(a)& \ding{43} &(ba.rʊ)ʊ	 &  		& 1		 & 		 & \\
(b)&  		&(ba.rʊʊ) 	&\shadecell 1 	&\shadecell 1 	&\shadecell  	&\shadecell \\
(c)&  		&(baa.rʊ)ʊ 	&\shadecell  	&\shadecell 1 	&\shadecell 1 	&\shadecell \\
(d)&  		&(baa)rʊʊ 	&\shadecell 1 	&\shadecell 1 	&\shadecell 1 	&\shadecell \\
(e)&  		&(ba.rʊ) 	&\shadecell  	&\shadecell 1 	&\shadecell  	&\shadecell 1\\
\hline\hline
\end{tabularx}
\z

The following question raised by a reviewer is best quoted verbatim:
\begin{quote}
I see a potential inconsistency between the analyses of /ba-ʊʊ/ and /bar-ʊʊ/. If foot structure can make the suffix split across foot edges, why does /ba-ʊʊ/ need vowel lengthening? The structure (\textit{baʊ})\textit{ʊ} has no degenerate foot and no unstressed VV. It doesn’t have -\textit{ʊʊ} attaching to a foot, but then neither does (\textit{ba.rʊ})\textit{ʊ}.
\end{quote}

The answer is characteristically optimality-theoretic: grammaticality is determined by competition. In the case of /ba-ʊʊ/, the candidate *(\textit{baʊ})\textit{ʊ} loses because there is a better candidate available: the winner (\textit{baa})\textit{ʊ} that satisfies \textsc{Align}. In the case of /bar-ʊʊ/ we have no such luxury: all candidates violate \textsc{Align} and therefore we must settle for the suffix-splitting (\textit{ba.rʊ})\textit{ʊ}. 

We conclude by showing the tableaux for CVVC and CVCC roots. They behave analogously and present no additional complications.

\ea\label{ex:anttila:6} Vowel length with CVCC roots
\begin{tabularx}{.9\textwidth}{lll|| C:C|C:C|}
\hline\hline
\multicolumn{3}{l||}{/barr-ʊʊ/} & WSP & \textsc{Align} & \textsc{Dep(V)} & \textsc{Max(V)}\\
\hline\hline
(a)& \ding{43}	&(bar.rʊ)ʊ 	&  	& 1 	&  	& \\
(b)&  		&(bar.rʊʊ) 	&\shadecell 1 	&\shadecell 1 	&\shadecell  	&\shadecell \\
(c)&  		&(baar.rʊ)ʊ 	&\shadecell  	&\shadecell 1 	&\shadecell 1 	&\shadecell \\
(d)&  		&(bar.rʊ) 	&\shadecell  	&\shadecell 1 	&\shadecell  	&\shadecell 1\\
\hline\hline
\end{tabularx}
\z

\ea\label{ex:anttila:7} Vowel length with CVVC roots
\begin{tabularx}{.9\textwidth}{lllXXXX}
\hline\hline
\multicolumn{3}{l}{/baar-ʊʊ/} & WSP & \textsc{Align} & \textsc{Dep(V)} & \textsc{Max(V)}\\
(a)& \ding{43} &(baa.rʊ)ʊ &  & 1 &  & \\
(b)&   			&(ba.rʊ)ʊ & \shadecell  & \shadecell1 & \shadecell & \shadecell1\\
(c)&   			&(baa)rʊʊ & \shadecell1 & \shadecell1 &\shadecell  & \shadecell \\
(d)&   			&(baa.rʊ) & \shadecell & \shadecell1 & \shadecell & \shadecell1\\
\hline\hline
\end{tabularx}
\z

\subsection{Lexically conditioned length}

Our metrical analysis of \ili{Dagaare} action nominals is relatively straightforward. Much more intriguing are the number and imperfective paradigms. \tabref{tab:anttila:15} below illustrates lexically conditioned length alternations with the \isi{imperfective suffix} /-r\'{V}/.

\begin{table}
\begin{tabularx}{\textwidth}{lXXXX}
\lsptoprule
& {Underlying} & {Imperfective} & {Alternation} & \\
\midrule
(a) & /da-r\'{V}/ & dàà-rá & lengthening & ‘buy’\\
& /tá{}-r\'{V}/ & tá-rà & -- & ‘reach’\\
\tablevspace
(b) & /fáà-r\'{V}/ & fáá-\textsuperscript{!}rá & -- & ‘seize’\\
& /gaa{}-r\'{V}/ & gɛ̀-rɛ́ & shortening & ‘go’\\
\tablevspace
(c) & /bɔŋ-r\'{V}/ & bɔ̀n-nɔ́ -- & ‘know’\\
& /woŋ-r\'{V}/ & wò-nó & C-deletion & ‘understand’\\
& /bal-r\'{V}/ & bàl-lá {\textasciitilde} bàl-á & variation & ‘be tired’\\
\lspbottomrule
\end{tabularx}
\caption{Lexical conditioning in the imperfective}
\label{tab:anttila:15}
\end{table}\largerpage[2]

In CV-roots the vowel lengthens or stays short; in CVV-roots the vowel stays long or shortens; in CVC-roots the suffix creates a CC cluster /CVC-r\'{V}/ which either survives or shortens, sometimes variably within a single lexical item. Why are length alternations so uniform in the action nominal paradigm, but riddled with lexical exceptions in the number and imperfective paradigms? To answer this question with any degree of confidence would require a deeper understanding of \ili{Dagaare} morphophonology than we have at the moment. However, one is immediately struck by the observation that it is the \textit{vowel-initial} suffixes that tend to have uniform paradigms. In addition to the action nominal /-\'{U}\'{U}/, the perfective /-\`{E}\'{E} / and the plural /-\'{V} / seem fairly regular. It is the \textit{consonant-initial} suffixes that permit exceptions, in particular the number /-rÍ/ and the imperfective \textit{/}{}-r\'{V}/.\footnote{Space does not permit a discussion of the perfective /-\`{E}É/ and the plural /-\'{V}/ here. We hope to return to the topic in a more complete exposition in the future.} Trying to explain these apparent suffix-related regularities is an interesting project, but must be left for future work.

\section{Summary}

We have provided new evidence for metrical structure in \ili{Dagaare} based on \isi{vowel length} alternations in action nominals. If the root is CV the root lengthens and the suffix shortens; if the root is CVV the suffix shortens; if the root ends in C nothing happens. Similar length alternations appear more idiosyncratically with number and \isi{aspect} suffixes. We have proposed a metrical analysis that explains the length alternations in action nominals and lends further support to the metrical analysis of \isi{vowel length} proposed in \citealt{Anttila&Bodomo2009} for \ili{Dagaare} nouns.
 
\section*{Acknowledgements}

This paper has benefited from presentations at the UC Berkeley Phonetics and Phonology Forum (April 30, 2012), the Stanford Phonology Workshop (May 18, 2012), and the 47\textsuperscript{th} Annual Conference on African Linguistics at UC Berkeley (March 26, 2016). We thank Luca Iacoponi, David Odden, and two anonymous reviewers for helpful comments. We are responsible for any errors.

\section*{Abbreviations}

\begin{tabularx}{.45\textwidth}{ll}
\textsc{dim} &  diminutive \\
\textsc{impf} &  imperfective \isi{aspect} \\ 
\textsc{perf} &  perfective \isi{aspect}\\
\end{tabularx}
\begin{tabularx}{.43\textwidth}{ll}
\textsc{pl} & plural\\ 
\textsc{sg} & singular\\
\\
\end{tabularx}


\sloppy
\printbibliography[heading=subbibliography,notkeyword=this] 
 
\end{document}