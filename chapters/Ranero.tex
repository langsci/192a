\documentclass[output=paper,newtxmath,modfonts,nonflat,hidelinks]{langsci/langscibook} 

\title{Deriving an object dislocation asymmetry in Luganda}

\author{Rodrigo Ranero\affiliation{University of Maryland}}

\abstract{In this paper, I document and analyze an object-dislocation asymmetry in Luganda (Bantu: JE15) that becomes apparent only upon comparing double object left-dislo\-ca\-tion versus double object right-dislocation. If two objects are left-dislocated, the object markers (OMs) on the verb are strictly ordered OM\textsubscript{THEME} > OM\textsubscript{GOAL/BEN} and the dislocated objects are ordered freely, either \textsc{goal/ben} > \textsc{theme} or \textsc{theme} > \textsc{goal/ben}. In contrast, if two objects are right-dislocated, the objects cannot be freely ordered — two right-dislocated objects must be ordered \textsc{goal/ben > theme}. However, in double object right-dislocation, the OMs must also be ordered OM\textsubscript{THEME} > OM\textsubscript{GOAL/BEN}. I propose that this asymmetry can be captured if left-dislocated objects are base generated in their surface position, whereas right-dislocated objects are derived via movement. Several predictions concerning binding and superiority effects are borne out, providing support for the analysis. 
}

\IfFileExists{../localcommands.tex}{%hack to check whether this is being compiled as part of a collection or standalone
  \usepackage{pifont}
\usepackage{savesym}

\savesymbol{downingtriple}
\savesymbol{downingdouble}
\savesymbol{downingquad}
\savesymbol{downingquint}
\savesymbol{suph}
\savesymbol{supj}
\savesymbol{supw}
\savesymbol{sups}
\savesymbol{ts}
\savesymbol{tS}
\savesymbol{devi}
\savesymbol{devu}
\savesymbol{devy}
\savesymbol{deva}
\savesymbol{N}
\savesymbol{Z}
\savesymbol{circled}
\savesymbol{sem}
\savesymbol{row}
\savesymbol{tipa}
\savesymbol{tableauxcounter}
\savesymbol{tabhead}
\savesymbol{inp}
\savesymbol{inpno}
\savesymbol{g}
\savesymbol{hanl}
\savesymbol{hanr}
\savesymbol{kuku}
\savesymbol{ip}
\savesymbol{lipm}
\savesymbol{ripm}
\savesymbol{lipn}
\savesymbol{ripn} 
% \usepackage{amsmath} 
% \usepackage{multicol}
\usepackage{qtree} 
\usepackage{tikz-qtree,tikz-qtree-compat}
% \usepackage{tikz}
\usepackage{upgreek}


%%%%%%%%%%%%%%%%%%%%%%%%%%%%%%%%%%%%%%%%%%%%%%%%%%%%
%%%                                              %%%
%%%           Examples                           %%%
%%%                                              %%%
%%%%%%%%%%%%%%%%%%%%%%%%%%%%%%%%%%%%%%%%%%%%%%%%%%%%
% remove the percentage signs in the following lines
% if your book makes use of linguistic examples
\usepackage{tipa}  
\usepackage{pstricks,pst-xkey,pst-asr}

%for sande et al
\usepackage{pst-jtree}
\usepackage{pst-node}
%\usepackage{savesym}


% \usepackage{subcaption}
\usepackage{multirow}  
\usepackage{./langsci/styles/langsci-optional} 
\usepackage{./langsci/styles/langsci-lgr} 
\usepackage{./langsci/styles/langsci-glyphs} 
\usepackage[normalem]{ulem}
%% if you want the source line of examples to be in italics, uncomment the following line
% \def\exfont{\it}
\usetikzlibrary{arrows.meta,topaths,trees}
\usepackage[linguistics]{forest}
\forestset{
	fairly nice empty nodes/.style={
		delay={where content={}{shape=coordinate,for parent={
					for children={anchor=north}}}{}}
}}
\usepackage{soul}
\usepackage{arydshln}
% \usepackage{subfloat}
\usepackage{langsci/styles/langsci-gb4e} 
   
% \usepackage{linguex}
\usepackage{vowel}

\usepackage{pifont}% http://ctan.org/pkg/pifont
\newcommand{\cmark}{\ding{51}}%
\newcommand{\xmark}{\ding{55}}%
 
 
 %Lamont
 \makeatletter
\g@addto@macro\@floatboxreset\centering
\makeatother

\usepackage{newfloat} 
\DeclareFloatingEnvironment[fileext=tbx,name=Tableau]{tableau}
  %add all your local new commands to this file
\newcommand{\downingquad}[4]{\parbox{2.5cm}{#1}\parbox{3.5cm}{#2}\parbox{2.5cm}{#3}\parbox{3.5cm}{#4}}
\newcommand{\downingtriple}[3]{\parbox{4.5cm}{#1}\parbox{3cm}{#2}\parbox{3cm}{#3}}
\newcommand{\downingdouble}[2]{\parbox{4.5cm}{#1}\parbox{6cm}{#2}}
\newcommand{\downingquint}[5]{\parbox{1.75cm}{#1}\parbox{2.25cm}{#2}\parbox{2cm}{#3}\parbox{3cm}{#4}\parbox{2cm}{#5}}
\newcolumntype{Y}{>{\centering\arraybackslash}X}
\newcolumntype{T}{>{\centering\arraybackslash}m{2cm}}

%commands for Kusmer paper below
\newcommand{\ip}{$\upiota$}
\newcommand{\lipm}{(\_{\ip-Max}}
\newcommand{\ripm}{)\_{\ip-Max}}
\newcommand{\lipn}{(\_{\ip}}
\newcommand{\ripn}{)\_{\ip}}
\renewcommand{\_}[1]{\textsubscript{#1}}


%commands for Pillion paper below
\newcommand{\suph}{\textipa{\super h}}
\newcommand{\supj}{\textipa{\super j}}
\newcommand{\supw}{\textipa{\super w}}
\newcommand{\ts}{\textipa{\t{ts}}}
\newcommand{\tS}{\textipa{\t{tS}}}
\newcommand{\devi}{\textipa{\r*i}}
\newcommand{\devu}{\textipa{\r*u}}
\newcommand{\devy}{\textipa{\r*y}}
\newcommand{\deva}{\textipa{\r*a}}
\renewcommand{\N}{\textipa{N}}
\newcommand{\Z}{\textipa{Z}}
% 

%commands for Diercks paper below
\newcommand{\circled}[1]{\begin{tikzpicture}[baseline=(word.base)]
\node[draw, rounded corners, text height=8pt, text depth=2pt, inner sep=2pt, outer sep=0pt, use as bounding box] (word) {#1};
\end{tikzpicture}
}

%commands for Pesetsky paper below
% \newcommand{\sem}[2][]{\mbox{$[\![ $\textbf{#2}$ ]\!]^{#1}$}}
\newcommand{\sem}[2][]{\mbox{$[[ $\textbf{#2}$ ]]^{#1}$}}

% \newcommand{\ripn}{{\color{red}ripn}}%this is used but never defined. Please update the definition



%commands for Lamont paper below
\newcommand{\row}[4]{
	#1. & 
    /{#2}/ & 
    [{#3}] & 
    `#4' \\ 
}
%\newcounter{tableauxcounter}
\newcommand{\tabhead}[2]{
%     \captionsetup{labelformat=empty}
%     \stepcounter{tableauxcounter}
%     \addtocounter{table}{-1}
% 	\centering
% 	\caption{Tableau \thetableauxcounter: #1}
	\caption{#1}
	\label{#2}
}
\newcommand{\candref}[2]{{(\ref{#1}#2)}}
\newcommand{\tableauref}[1]{{Tableau~\ref{#1}}}
% tableaux
\newcommand{\inp}[1]{\multicolumn{2}{|l||}{{#1}}}
\newcommand{\inpno}[1]{\multicolumn{2}{|l||}{#1}}
\newcommand{\g}{\cellcolor{lightgray}}
\newcommand{\hanl}{\HandLeft}
\newcommand{\hanr}{\HandRight}
\newcommand{\kuku}{Kuk\'{u}}

% \newcommand{\nocaption}[1]{{\color{red} Please provide a caption}}

% \providecommand{\biberror}[1]{{\color{red}#1}}

\definecolor{RED}{cmyk}{0.05,1,0.8,0}


\newfontfamily\amharicfont[Script = Ethiopic, Scale = 1.0]{AbyssinicaSIL}
\newcommand{\amh}[1]{{\amharicfont #1}}

% 
% %Gjersoe
\usepackage{textgreek}
% 
\newcommand{\viol}{\fontfamily{MinionPro-OsF}\selectfont\rotatebox{60}{$\star$}}
\newcommand{\myscalex}{0.45}
\newcommand{\myscaley}{0.65}
%\newcommand{\red}[1]{\textcolor{red}{#1}}
%\newcommand{\blue}[1]{\textcolor{blue}{#1}}
\newcommand{\epen}[1]{\colorbox{jgray}{#1}}
\newcommand{\hand}{{\normalsize \ding{43}}}
\definecolor{jgray}{gray}{0.8} 
\usetikzlibrary{positioning}
\usetikzlibrary{matrix}
\newcommand{\mora}{\textmu\xspace}
\newcommand{\si}{\textsigma\xspace}
\newcommand{\ft}{\textPhi\xspace}
\newcommand{\tone}{\texttau\xspace}
\newcommand{\word}{\textomega\xspace}
% \newcommand{\ts}{\texttslig}
\newcommand{\fns}{\footnotesize}
\newcommand{\ns}{\normalsize}
\newcommand{\vs}{\vspace{1em}}
\newcommand{\bs}{\textbackslash}   % backslash
\newcommand{\cmd}[1]{{\bf \color{red}#1}}   % highlights command
\newcommand{\scell}[2][l]{\begin{tabular}[#1]{@{}c@{}}#2\end{tabular}}
% \interfootnotelinepenalty=10000

% --- Snider Representations --- %

\newcommand{\RepLevelHh}{
\begin{minipage}{0.10\textwidth}
\begin{tikzpicture}[xscale=\myscalex,yscale=\myscaley]
%\node (syl) at (0,0) {Hi};
\node (Rt) at (0,1) {o};
\node (H) at (-0.5,2) {H};
\node (R) at (0.5,3) {h};
%\draw [thick] (syl.north) -- (Rt.south) ;
\draw [thick] (Rt.north) -- (H.south) ;
\draw [thick] (Rt.north) -- (R.south) ;
\end{tikzpicture}
\end{minipage}
}

\newcommand{\RepLevelLh}{
\begin{minipage}{0.10\textwidth}
\begin{tikzpicture}[xscale=\myscalex,yscale=\myscaley]
%\node (syl) at (0,0) {Mid2};
\node (Rt) at (0,1) {o};
\node (H) at (-0.5,2) {L};
\node (R) at (0.5,3) {h};
%\draw [thick] (syl.north) -- (Rt.south) ;
\draw [thick] (Rt.north) -- (H.south) ;
\draw [thick] (Rt.north) -- (R.south) ;
\end{tikzpicture}
\end{minipage}
}

\newcommand{\RepLevelHl}{
\begin{minipage}{0.10\textwidth}
\begin{tikzpicture}[xscale=\myscalex,yscale=\myscaley]
%\node (syl) at (0,0) {Mid1};
\node (Rt) at (0,1) {o};
\node (H) at (-0.5,2) {H};
\node (R) at (0.5,3) {l};
%\draw [thick] (syl.north) -- (Rt.south) ;
\draw [thick] (Rt.north) -- (H.south) ;
\draw [thick] (Rt.north) -- (R.south) ;
\end{tikzpicture}
\end{minipage}
}

\newcommand{\RepLevelLl}{
\begin{minipage}{0.10\textwidth}
\begin{tikzpicture}[xscale=\myscalex,yscale=\myscaley]
%\node (syl) at (0,0) {Lo};
\node (Rt) at (0,1) {o};
\node (H) at (-0.5,2) {L};
\node (R) at (0.5,3) {l};
%\draw [thick] (syl.north) -- (Rt.south) ;
\draw [thick] (Rt.north) -- (H.south) ;
\draw [thick] (Rt.north) -- (R.south) ;
\end{tikzpicture}
\end{minipage}
}

% --- Representations --- %

\newcommand{\RepLevel}{
\begin{minipage}{0.10\textwidth}
\begin{tikzpicture}[xscale=\myscalex,yscale=\myscaley]
\node (syl) at (0,0) {\textsigma};
\node (Rt) at (0,1) {o};
\node (H) at (-0.5,2) {\texttau};
\node (R) at (0.5,3) {\textrho};
\draw [thick] (syl.north) -- (Rt.south) ;
\draw [thick] (Rt.north) -- (H.south) ;
\draw [thick] (Rt.north) -- (R.south) ;
\end{tikzpicture}
\end{minipage}
}

\newcommand{\RepContour}{
\begin{minipage}{0.10\textwidth}
\begin{tikzpicture}[xscale=\myscalex,yscale=\myscaley]
\node (syl) at (0,0) {\textsigma};
\node (Rt) at (0,1) {o};
\node (H) at (-0.5,2) {\texttau};
\node (R) at (0.5,3) {\textrho};
\node (Rt2) at (1.5,1.0) {o};
%\node (H2) at (1.0,2) {$\tau$};
%\node (R2) at (2.0,2.5) {R};
\draw [thick] (syl.north) -- (Rt.south) ;
\draw [thick] (Rt.north) -- (H.south) ;
\draw [thick] (Rt.north) -- (R.south) ;
\draw [thick] (syl.north) -- (Rt2.south) ;
%\draw [thick] (Rt2.north) -- (H2.south) ;
%\draw [thick] (Rt2.north) -- (R2.south) ;
\end{tikzpicture}
\end{minipage}
}


% --- OT constraints --- %

\newcommand{\IllustrationDown}{
\begin{minipage}{0.09\textwidth}
\begin{tikzpicture}[xscale=0.7,yscale=0.45]
\node (reg) at (0,0.75) {{\small \textalpha}};
\node (arrow) at (0,0) {{\fns $\downarrow$}};
\node (Rt) at (0,-0.75) {{\small \textbeta}};
\end{tikzpicture}
\end{minipage}
}

\newcommand{\IllustrationUp}{
\begin{minipage}{0.09\textwidth}
\begin{tikzpicture}[xscale=0.7,yscale=0.45]
\node (reg) at (0,0.75) {{\small \textalpha}};
\node (arrow) at (0,0) {{\fns $\uparrow$}};
\node (Rt) at (0,-0.75) {{\small \textbeta}};
\end{tikzpicture}
\end{minipage}
}

\newcommand{\MaxAB}{
\begin{minipage}{0.09\textwidth}
\begin{tikzpicture}[xscale=0.6,yscale=0.4]
\node (max) at (0,0) {{\small \textsc{Max}}};
\node (reg) at (0.75,0.5) {{\fns \textalpha}};
\node (arrow) at (0.75,0) {{\tiny $\downarrow$}};
\node (Rt) at (0.75,-0.5) {{\fns \textbeta}};
\end{tikzpicture}
\end{minipage}
}

\newcommand{\DepAB}{
\begin{minipage}{0.09\textwidth}
\begin{tikzpicture}[xscale=0.6,yscale=0.4]
\node (max) at (0,0) {{\small \textsc{Dep}}};
\node (reg) at (0.75,0.5) {{\fns \textalpha}};
\node (arrow) at (0.75,0) {{\tiny $\downarrow$}};
\node (Rt) at (0.75,-0.5) {{\fns \textbeta}};
\end{tikzpicture}
\end{minipage}
}

\newcommand{\DepHReg}{
\begin{minipage}{0.055\textwidth}
\begin{tikzpicture}[xscale=0.6,yscale=0.4]
\node (dep) at (0,0) {{\small \textsc{Dep}}};
\node (reg) at (0,-1.0) {{\small h}};
\end{tikzpicture}
\end{minipage}
}

\newcommand{\DepLReg}{
\begin{minipage}{0.055\textwidth}
\begin{tikzpicture}[xscale=0.6,yscale=0.4]
\node (dep) at (0,0) {{\small \textsc{Dep}}};
\node (reg) at (0,-1.0) {{\small l}};
\end{tikzpicture}
\end{minipage}
}

\newcommand{\DepReg}{
\begin{minipage}{0.055\textwidth}
\begin{tikzpicture}[xscale=0.6,yscale=0.4]
\node (dep) at (0,0) {{\small \textsc{Dep}}};
\node (reg) at (0,-1.0) {{\small \textrho}};
\end{tikzpicture}
\end{minipage}
}

\newcommand{\DepTRt}{
\begin{minipage}{0.1\textwidth}
\begin{tikzpicture}[xscale=0.6,yscale=0.4]
\node (dep) at (0,0) {{\small \textsc{Dep}}};
\node (t) at (0.75,0.5) {{\fns \texttau}};
\node (arrow) at (0.75,0) {{\tiny $\downarrow$}};
\node (Rt) at (0.75,-0.5) {{\fns o}};
\end{tikzpicture}
\end{minipage}
}

\newcommand{\MaxRegRt}{
\begin{minipage}{0.1\textwidth}
\begin{tikzpicture}[xscale=0.6,yscale=0.4]
\node (max) at (0,0) {{\small \textsc{Max}}};
\node (arrow) at (0.75,0) {{\tiny $\downarrow$}};
\node (Rt) at (0.75,-0.5) {{\fns o}};
\node (reg) at (0.75,0.5) {{\fns \textrho}};
\end{tikzpicture}
\end{minipage}
}

\newcommand{\RegToneByRt}{
\begin{minipage}{0.06\textwidth}
\begin{tikzpicture}[xscale=0.6,yscale=0.5]
\node[rotate=20] (arrow1) at (-0.15,0) {{\fns $\uparrow$}};
\node[rotate=340] (arrow2) at (0.15,0) {{\fns $\uparrow$}};
\node (Rt) at (0,-0.55) {{\small o}};
\node (reg) at (0.4,0.55) {{\small \textrho}};
\node (tone) at (-0.4,0.55) {{\small \texttau}};
\end{tikzpicture}
\end{minipage}
}

\newcommand{\RegToneBySyl}{
\begin{minipage}{0.06\textwidth}
\begin{tikzpicture}[xscale=0.6,yscale=0.5]
\node[rotate=20] (arrow1) at (-0.15,0) {{\fns $\uparrow$}};
\node[rotate=340] (arrow2) at (0.15,0) {{\fns $\uparrow$}};
\node (Rt) at (0,-0.55) {{\small \textsigma}};
\node (reg) at (0.4,0.55) {{\small \textrho}};
\node (tone) at (-0.4,0.55) {{\small \texttau}};
\end{tikzpicture}
\end{minipage}
}

\newcommand{\DepTone}{
\begin{minipage}{0.055\textwidth}
\begin{tikzpicture}[xscale=0.6,yscale=0.4]
\node (dep) at (0,0) {{\small \textsc{Dep}}};
\node (tone) at (0,-1.0) {{\small \texttau}};
\end{tikzpicture}
\end{minipage}
}

\newcommand{\DepTonalRt}{
\begin{minipage}{0.055\textwidth}
\begin{tikzpicture}[xscale=0.6,yscale=0.4]
\node (dep) at (0,0) {{\small \textsc{Dep}}};
\node (tone) at (0,-1.0) {{\small o}};
\end{tikzpicture}
\end{minipage}
}

\newcommand{\DepL}{
\begin{minipage}{0.055\textwidth}
\begin{tikzpicture}[xscale=0.6,yscale=0.4]
\node (dep) at (0,0) {{\small \textsc{Dep}}};
\node (tone) at (0,-1.0) {{\small L}};
\end{tikzpicture}
\end{minipage}
}

\newcommand{\DepH}{
\begin{minipage}{0.055\textwidth}
\begin{tikzpicture}[xscale=0.6,yscale=0.4]
\node (dep) at (0,0) {{\small \textsc{Dep}}};
\node (tone) at (0,-1.0) {{\small H}};
\end{tikzpicture}
\end{minipage}
}

\newcommand{\NoMultDiff}{{\small *loh}}
\newcommand{\Alt}{{\small \textsc{Alt}}}
\newcommand{\NoSkip}{{\small \scell{\textsc{No}\\\textsc{Skip}}}}


\newcommand{\RegDomRt}{
\begin{minipage}{0.030\textwidth}
\begin{tikzpicture}[xscale=0.6,yscale=0.5]
\node (arrow) at (0,0) {{\fns $\downarrow$}};
\node (Rt) at (0,-0.55) {{\small o}};
\node (reg) at (0,0.55) {{\small \textrho}};
\end{tikzpicture}
\end{minipage}
}

\newcommand{\DepRegRt}{
\begin{minipage}{0.1\textwidth}
\begin{tikzpicture}[xscale=0.6,yscale=0.4]
\node (dep) at (0,0) {{\small \textsc{Dep}}};
\node (arrow) at (0.75,0) {{\tiny $\downarrow$}};
\node (Rt) at (0.75,-0.5) {{\fns o}};
\node (reg) at (0.75,0.5) {{\fns \textrho}};
\end{tikzpicture}
\end{minipage}
}

% unused

\newcommand{\ToneByRt}{
\begin{minipage}{0.05\textwidth}
\begin{tikzpicture}[xscale=0.6,yscale=0.5]
\node (arrow) at (0,0) {{\fns $\uparrow$}};
\node (Rt) at (0,-0.55) {{\small o}};
\node (tone) at (0,0.55) {{\small \texttau}};
\end{tikzpicture}
\end{minipage}
}

\newcommand{\RegByRt}{
\begin{minipage}{0.05\textwidth}
\begin{tikzpicture}[xscale=0.6,yscale=0.5]
\node (arrow) at (0,0) {{\fns $\uparrow$}};
\node (Rt) at (0,-0.55) {{\small o}};
\node (reg) at (0,0.55) {{\small \textrho}};
\end{tikzpicture}
\end{minipage}
}

\newcommand{\ToneDomRt}{
\begin{minipage}{0.05\textwidth}
\begin{tikzpicture}[xscale=0.6,yscale=0.5]
\node (arrow) at (0,0) {{\fns $\downarrow$}};
\node (Rt) at (0,-0.55) {{\small o}};
\node (tone) at (0,0.55) {{\small \texttau}};
\end{tikzpicture}
\end{minipage}
}

% --- OT tableaus --- %

% Sec. 3.2, first tabl.

\newcommand{\OTHLInput}{
\begin{minipage}{0.17\textwidth}
\begin{tikzpicture}[xscale=\myscalex,yscale=\myscaley]
\node (tone) at (2,0) {(= H)};
\node (syl) at (0,0) {\textsigma};
\node (Rt) at (0,1) {o};
\node (H) at (-0.5,2) {H};
\node (R) at (0.5,3) {h};
\node (Rt2) at (1.5,1.0) {o};
%\node (H2) at (1.0,2) {\epen{L}};
\node (R2) at (2.0,3) {\blue{l}};
\draw [thick] (syl.north) -- (Rt.south) ;
\draw [thick] (Rt.north) -- (H.south) ;
\draw [thick] (Rt.north) -- (R.south) ;
\draw [thick] (syl.north) -- (Rt2.south) ;
%\draw [dashed] (Rt2.north) -- (H2.south) ;
%\draw [dashed] (Rt2.north) -- (R2.south) ;
\end{tikzpicture}
\end{minipage}
}

\newcommand{\OTHLWinner}{
\begin{minipage}{0.17\textwidth}
\begin{tikzpicture}[xscale=\myscalex,yscale=\myscaley]
\node (tone) at (2,0) {(= HL)};
\node (syl) at (0,0) {\textsigma};
\node (Rt) at (0,1) {o};
\node (H) at (-0.5,2) {H};
\node (R) at (0.5,3) {h};
\node (Rt2) at (1.5,1.0) {o};
\node (H2) at (1.0,2) {\epen{L}};
\node (R2) at (2.0,3) {\blue{l}};
\draw [thick] (syl.north) -- (Rt.south) ;
\draw [thick] (Rt.north) -- (H.south) ;
\draw [thick] (Rt.north) -- (R.south) ;
\draw [thick] (syl.north) -- (Rt2.south) ;
\draw [dashed] (Rt2.north) -- (H2.south) ;
\draw [dashed] (Rt2.north) -- (R2.south) ;
\end{tikzpicture}
\end{minipage}
}

\newcommand{\OTHLSpreadingHOnly}{
\begin{minipage}{0.17\textwidth}
\begin{tikzpicture}[xscale=\myscalex,yscale=\myscaley]
\node (tone) at (2,0) {(= HM)};
\node (syl) at (0,0) {\textsigma};
\node (Rt) at (0,1) {o};
\node (H) at (-0.5,2) {H};
\node (R) at (0.5,3) {h};
\node (Rt2) at (1.5,1.0) {o};
%\node (H2) at (1.0,2) {\epen{L}};
\node (R2) at (2.0,3) {\blue{l}};
\draw [thick] (syl.north) -- (Rt.south) ;
\draw [thick] (Rt.north) -- (H.south) ;
\draw [thick] (Rt.north) -- (R.south) ;
\draw [thick] (syl.north) -- (Rt2.south) ;
\draw [dashed] (Rt2.north) -- (R2.south) ;
\draw [dashed] (Rt2.north) -- (H.south) ;
\end{tikzpicture}
\end{minipage}
}

\newcommand{\OTHLInsertH}{
\begin{minipage}{0.17\textwidth}
\begin{tikzpicture}[xscale=\myscalex,yscale=\myscaley]
\node (tone) at (2,0) {(= HM)};
\node (syl) at (0,0) {\textsigma};
\node (Rt) at (0,1) {o};
\node (H) at (-0.5,2) {H};
\node (R) at (0.5,3) {h};
\node (Rt2) at (1.5,1.0) {o};
\node (H2) at (1.0,2) {\epen{H}};
\node (R2) at (2.0,3) {\blue{l}};
\draw [thick] (syl.north) -- (Rt.south) ;
\draw [thick] (Rt.north) -- (H.south) ;
\draw [thick] (Rt.north) -- (R.south) ;
\draw [thick] (syl.north) -- (Rt2.south) ;
\draw [dashed] (Rt2.north) -- (H2.south) ;
\draw [dashed] (Rt2.north) -- (R2.south) ;
\end{tikzpicture}
\end{minipage}
}

\newcommand{\OTHLOverwriting}{
\begin{minipage}{0.17\textwidth}
\begin{tikzpicture}[xscale=\myscalex,yscale=\myscaley]
\node (syl) at (0,0) {\textsigma};
\node (Rt) at (0,1) {o};
\node (H) at (-0.5,2) {H};
\node (R) at (0.5,3) {h};
\node (Rt2) at (1.5,1.0) {o};
%\node (H2) at (1.0,2) {\epen{L}};
\node (R2) at (2.0,3) {\blue{l}};
\draw [thick] (syl.north) -- (Rt.south) ;
\draw [thick] (Rt.north) -- (H.south) ;
\draw [thick] (Rt.north) -- (R.south) ;
\draw [thick] (syl.north) -- (Rt2.south) ;
%\draw [dashed] (Rt2.north) -- (H2.south) ;
\draw [dashed] (Rt.north) -- (R2.south) ;
\node (del) at (0.3,1.9) {\textbf{=}};
\end{tikzpicture}
\end{minipage}
}

\newcommand{\OTHLSpreading}{
\begin{minipage}{0.17\textwidth}
\begin{tikzpicture}[xscale=\myscalex,yscale=\myscaley]
\node (syl) at (0,0) {\textsigma};
\node (Rt) at (0,1) {o};
\node (H) at (-0.5,2) {H};
\node (R) at (0.5,3) {h};
\node (Rt2) at (1.5,1.0) {o};
%\node (H2) at (1.0,2) {\epen{L}};
\node (R2) at (2.0,3) {\blue{l}};
\draw [thick] (syl.north) -- (Rt.south) ;
\draw [thick] (Rt.north) -- (H.south) ;
\draw [thick] (Rt.north) -- (R.south) ;
\draw [thick] (syl.north) -- (Rt2.south) ;
%\draw [dashed] (Rt2.north) -- (H2.south) ;
\draw [dashed] (Rt2.north) -- (H.south) ;
\draw [dashed] (Rt2.north) -- (R.south) ;
\end{tikzpicture}
\end{minipage}
}

% Sec. 4.2, second tabl.: phrase-medial position

\newcommand{\OTHnoLInput}{
\begin{minipage}{0.17\textwidth}
\begin{tikzpicture}[xscale=\myscalex,yscale=\myscaley]
\node (tone) at (2,0) {(= H)};
\node (syl) at (0,0) {\textsigma};
\node (Rt) at (0,1) {o};
\node (H) at (-0.5,2) {H};
\node (R) at (0.5,3) {h};
\node (Rt2) at (1.5,1.0) {o};
%\node (H2) at (1.0,2) {\epen{L}};
%\node (R2) at (2.0,3) {\blue{l}};
\draw [thick] (syl.north) -- (Rt.south) ;
\draw [thick] (Rt.north) -- (H.south) ;
\draw [thick] (Rt.north) -- (R.south) ;
\draw [thick] (syl.north) -- (Rt2.south) ;
\end{tikzpicture}
\end{minipage}
}

\newcommand{\OTHnoLEpenth}{
\begin{minipage}{0.17\textwidth}
\begin{tikzpicture}[xscale=\myscalex,yscale=\myscaley]
\node (tone) at (2,0) {(= HM)};
\node (syl) at (0,0) {\textsigma};
\node (Rt) at (0,1) {o};
\node (H) at (-0.5,2) {H};
\node (R) at (0.5,3) {h};
\node (Rt2) at (1.5,1.0) {o};
\node (H2) at (1.0,2) {\epen{L}};
\node (R2) at (2.0,3) {\epen{h}};
\draw [thick] (syl.north) -- (Rt.south) ;
\draw [thick] (Rt.north) -- (H.south) ;
\draw [thick] (Rt.north) -- (R.south) ;
\draw [thick] (syl.north) -- (Rt2.south) ;
\draw [dashed] (Rt2.north) -- (H2.south) ;
\draw [dashed] (Rt2.north) -- (R2.south) ;
\end{tikzpicture}
\end{minipage}
}

\newcommand{\OTHnoLSpreading}{
\begin{minipage}{0.17\textwidth}
\begin{tikzpicture}[xscale=\myscalex,yscale=\myscaley]
\node (tone) at (2,0) {(= HH)};
\node (syl) at (0,0) {\textsigma};
\node (Rt) at (0,1) {o};
\node (H) at (-0.5,2) {H};
\node (R) at (0.5,3) {h};
\node (Rt2) at (1.5,1.0) {o};
%\node (H2) at (1.0,2) {\epen{L}};
%\node (R2) at (2.0,3) {\blue{l}};
\draw [thick] (syl.north) -- (Rt.south) ;
\draw [thick] (Rt.north) -- (H.south) ;
\draw [thick] (Rt.north) -- (R.south) ;
\draw [thick] (syl.north) -- (Rt2.south) ;
\draw [dashed] (Rt2.north) -- (H.south) ;
\draw [dashed] (Rt2.north) -- (R.south) ;
\end{tikzpicture}
\end{minipage}
}

% Sec. 4.2, third tabl., LM is unaffected by L\%

\newcommand{\OTLMInput}{
\begin{minipage}{0.2\textwidth}
\begin{tikzpicture}[xscale=\myscalex,yscale=\myscaley]
\node (tone) at (2,0) {(= LM)};
\node (syl) at (0,0) {\textsigma};
\node (Rt) at (0,1) {o};
\node (H) at (-0.5,2) {L};
\node (R) at (0.5,3) {l};
\node (Rt2) at (1.5,1.0) {o};
\node (H2) at (1.0,2) {L};
\node (R2) at (2.0,3) {h};
\node (R3) at (3.0,3) {\blue{l}};
\draw [thick] (syl.north) -- (Rt.south) ;
\draw [thick] (Rt.north) -- (H.south) ;
\draw [thick] (Rt.north) -- (R.south) ;
\draw [thick] (syl.north) -- (Rt2.south) ;
\draw [thick] (Rt2.north) -- (H2.south) ;
\draw [thick] (Rt2.north) -- (R2.south) ;
\end{tikzpicture}
\end{minipage}
}

\newcommand{\OTLMReplace}{
\begin{minipage}{0.2\textwidth}
\begin{tikzpicture}[xscale=\myscalex,yscale=\myscaley]
\node (tone) at (2,0) {(= LL)};
\node (syl) at (0,0) {\textsigma};
\node (Rt) at (0,1) {o};
\node (H) at (-0.5,2) {L};
\node (R) at (0.5,3) {l};
\node (Rt2) at (1.5,1.0) {o};
\node (H2) at (1.0,2) {L};
\node (R2) at (2.0,3) {h};
\node (R3) at (3.0,3) {\blue{l}};
\draw [thick] (syl.north) -- (Rt.south) ;
\draw [thick] (Rt.north) -- (H.south) ;
\draw [thick] (Rt.north) -- (R.south) ;
\draw [thick] (syl.north) -- (Rt2.south) ;
\draw [thick] (Rt2.north) -- (H2.south) ;
\draw [thick] (Rt2.north) -- (R2.south) ;
\draw [dashed] (Rt2.north) -- (R3.south) ;
\node (del) at (1.8,2.1) {\textbf{=}};
\end{tikzpicture}
\end{minipage}
}

\newcommand{\OTLMTwoReg}{
\begin{minipage}{0.2\textwidth}
\begin{tikzpicture}[xscale=\myscalex,yscale=\myscaley]
\node (tone) at (2,0) {(= LML)};
\node (syl) at (0,0) {\textsigma};
\node (Rt) at (0,1) {o};
\node (H) at (-0.5,2) {L};
\node (R) at (0.5,3) {l};
\node (Rt2) at (1.5,1.0) {o};
\node (H2) at (1.0,2) {L};
\node (R2) at (2.0,3) {h};
\node (R3) at (3.0,3) {\blue{l}};
\draw [thick] (syl.north) -- (Rt.south) ;
\draw [thick] (Rt.north) -- (H.south) ;
\draw [thick] (Rt.north) -- (R.south) ;
\draw [thick] (syl.north) -- (Rt2.south) ;
\draw [thick] (Rt2.north) -- (H2.south) ;
\draw [thick] (Rt2.north) -- (R2.south) ;
\draw [dashed] (Rt2.north) -- (R3.south) ;
\end{tikzpicture}
\end{minipage}
}

% Sec. 4.2, fourth tabl., L is affected by L\% but M is not

\newcommand{\OTLInput}{
\begin{minipage}{0.17\textwidth}
\begin{tikzpicture}[xscale=\myscalex,yscale=\myscaley]
\node (tone) at (2,0) {(= L)};
\node (syl) at (0,0) {\textsigma};
\node (Rt) at (0,1) {o};
\node (H) at (-0.5,2) {L};
\node (R) at (0.5,3) {l};
\node (R2) at (2,3) {\blue{l}};
\draw [thick] (syl.north) -- (Rt.south) ;
\draw [thick] (Rt.north) -- (H.south) ;
\draw [thick] (Rt.north) -- (R.south) ;
\end{tikzpicture}
\end{minipage}
}

\newcommand{\OTLLowered}{
\begin{minipage}{0.17\textwidth}
\begin{tikzpicture}[xscale=\myscalex,yscale=\myscaley]
\node (tone) at (2,0) {(= LL)};
\node (syl) at (0,0) {\textsigma};
\node (Rt) at (0,1) {o};
\node (H) at (-0.5,2) {L};
\node (R) at (0.5,3) {l};
\node (R2) at (2,3) {\blue{l}};
\draw [thick] (syl.north) -- (Rt.south) ;
\draw [thick] (Rt.north) -- (H.south) ;
\draw [thick] (Rt.north) -- (R.south) ;
\draw [dashed] (Rt.north) -- (R2.south) ;
\end{tikzpicture}
\end{minipage}
}

\newcommand{\OTMInput}{
\begin{minipage}{0.17\textwidth}
\begin{tikzpicture}[xscale=\myscalex,yscale=\myscaley]
\node (tone) at (2,0) {(= M)};
\node (syl) at (0,0) {\textsigma};
\node (Rt) at (0,1) {o};
\node (H) at (-0.5,2) {L};
\node (R) at (0.5,3) {h};
\node (R2) at (2,3) {\blue{l}};
\draw [thick] (syl.north) -- (Rt.south) ;
\draw [thick] (Rt.north) -- (H.south) ;
\draw [thick] (Rt.north) -- (R.south) ;
\end{tikzpicture}
\end{minipage}
}

\newcommand{\OTMLowered}{
\begin{minipage}{0.17\textwidth}
\begin{tikzpicture}[xscale=\myscalex,yscale=\myscaley]
\node (tone) at (2,0) {(= ML)};
\node (syl) at (0,0) {\textsigma};
\node (Rt) at (0,1) {o};
\node (H) at (-0.5,2) {L};
\node (R) at (0.5,3) {h};
\node (R2) at (2,3) {\blue{l}};
\draw [thick] (syl.north) -- (Rt.south) ;
\draw [thick] (Rt.north) -- (H.south) ;
\draw [thick] (Rt.north) -- (R.south) ;
\draw [dashed] (Rt.north) -- (R2.south) ;
\end{tikzpicture}
\end{minipage}
}

% Sec. 4.2, fifth tableau, polar questions with level tones

\newcommand{\OTLPolIn}{
\begin{minipage}{0.20\textwidth}
\begin{tikzpicture}[xscale=\myscalex-0.05,yscale=\myscaley-0.05]
\node (tone) at (3.5,0) {(= L)};
\node (syl) at (0,0) {\textsigma};
\node (syl2) at (2,0) {\red{\textsigma}};
\node (Rt) at (0,1) {o};
\node (H) at (-0.5,2) {L};
\node (R) at (0.5,3) {l};
\node (Rt2) at (2,1) {\red{o}};
\draw [thick] (syl.north) -- (Rt.south) ;
\draw [thick,red] (syl2.north) -- (Rt2.south) ;
\draw [thick] (Rt.north) -- (H.south) ;
\draw [thick] (Rt.north) -- (R.south) ;
\end{tikzpicture}
\end{minipage}
}

\newcommand{\OTLPolDef}{
\begin{minipage}{0.20\textwidth}
\begin{tikzpicture}[xscale=\myscalex-0.05,yscale=\myscaley-0.05]
\node (tone) at (3.5,0) {(= L.M)};
\node (syl) at (0,0) {\textsigma};
\node (syl2) at (2,0) {\red{\textsigma}};
\node (Rt) at (0,1) {o};
\node (H) at (-0.5,2) {L};
\node (R) at (0.5,3) {l};
\node (H2) at (1.5,2) {\epen{L}};
\node (R2) at (2.5,3) {\epen{h}};
\node (Rt2) at (2,1) {\red{o}};
\draw [thick] (syl.north) -- (Rt.south) ;
\draw [thick,red] (syl2.north) -- (Rt2.south) ;
\draw [thick] (Rt.north) -- (H.south) ;
\draw [thick] (Rt.north) -- (R.south) ;
\draw [semithick,dashed] (Rt2.north) -- (H2.south) ;
\draw [semithick,dashed] (Rt2.north) -- (R2.south) ;
\end{tikzpicture}
\end{minipage}
}

\newcommand{\OTLPolAlt}{
\begin{minipage}{0.20\textwidth}
\begin{tikzpicture}[xscale=\myscalex-0.05,yscale=\myscaley-0.05]
\node (tone) at (3.5,0) {(= L.L)};
\node (syl) at (0,0) {\textsigma};
\node (syl2) at (2,0) {\red{\textsigma}};
\node (Rt) at (0,1) {o};
\node (H) at (-0.5,2) {L};
\node (R) at (0.5,3) {l};
\node (Rt2) at (2,1) {\red{o}};
\draw [thick] (syl.north) -- (Rt.south) ;
\draw [thick,red] (syl2.north) -- (Rt2.south) ;
\draw [thick] (Rt.north) -- (H.south) ;
\draw [thick] (Rt.north) -- (R.south) ;
\draw [semithick,dashed] (Rt2.north) -- (H.south) ;
\draw [semithick,dashed] (Rt2.north) -- (R.south) ;
\end{tikzpicture}
\end{minipage}
}

% Sec. 4.2, sixth tableau, polar questions with contour tones

\newcommand{\OTLLPolIn}{
\begin{minipage}{0.23\textwidth}
\begin{tikzpicture}[xscale=\myscalex-0.05,yscale=\myscaley-0.05]
\node (tone) at (5.2,0) {(= L)};
\node (syl) at (0,0) {\textsigma};
\node (syl3) at (3.4,0) {\red{\textsigma}};
\node (Rt) at (0,1) {o};
\node (Rt2) at (1.7,1) {o};
\node (Rt3) at (3.4,1) {\red{o}};
\node (H) at (-0.5,2) {L};
\node (R) at (0.5,3) {l};
\draw [thick] (syl.north) -- (Rt.south) ;
\draw [thick] (syl.north) -- (Rt2.south) ;
\draw [thick,red] (syl3.north) -- (Rt3.south) ;
\draw [thick] (Rt.north) -- (H.south) ;
\draw [thick] (Rt.north) -- (R.south) ;
\end{tikzpicture}
\end{minipage}
}

\newcommand{\OTLLPolDef}{
\begin{minipage}{0.23\textwidth}
\begin{tikzpicture}[xscale=\myscalex-0.05,yscale=\myscaley-0.05]
\node (tone) at (5.2,0) {(= L.M)};
\node (syl) at (0,0) {\textsigma};
\node (syl3) at (3.4,0) {\red{\textsigma}};
\node (Rt) at (0,1) {o};
\node (Rt2) at (1.7,1) {o};
\node (Rt3) at (3.4,1) {\red{o}};
\node (H) at (-0.5,2) {L};
\node (R) at (0.5,3) {l};
\node (H3) at (2.9,2) {\epen{L}};
\node (R3) at (3.9,3) {\epen{h}};
\draw [thick] (syl.north) -- (Rt.south) ;
\draw [thick] (syl.north) -- (Rt2.south) ;
\draw [thick,red] (syl3.north) -- (Rt3.south) ;
\draw [thick] (Rt.north) -- (H.south) ;
\draw [thick] (Rt.north) -- (R.south) ;
\draw [dashed] (Rt3.north) -- (H3.south) ;
\draw [dashed] (Rt3.north) -- (R3.south) ;
\end{tikzpicture}
\end{minipage}
}

\newcommand{\OTLLPolSkip}{
\begin{minipage}{0.23\textwidth}
\begin{tikzpicture}[xscale=\myscalex-0.05,yscale=\myscaley-0.05]
\node (tone) at (5.2,0) {(= L.L)};
\node (syl) at (0,0) {\textsigma};
\node (syl3) at (3.4,0) {\red{\textsigma}};
\node (Rt) at (0,1) {o};
\node (Rt2) at (1.7,1) {o};
\node (Rt3) at (3.4,1) {\red{o}};
\node (H) at (-0.5,2) {L};
\node (R) at (0.5,3) {l};
\draw [thick] (syl.north) -- (Rt.south) ;
\draw [thick] (syl.north) -- (Rt2.south) ;
\draw [thick,red] (syl3.north) -- (Rt3.south) ;
\draw [thick] (Rt.north) -- (H.south) ;
\draw [thick] (Rt.north) -- (R.south) ;
\draw [dashed] (Rt3.north) -- (H.south) ;
\draw [dashed] (Rt3.north) -- (R.south) ;
\end{tikzpicture}
\end{minipage}
}  
  
\newcommand{\ilit}[1]{#1\il{#1}}    
\newcommand{\isit}[1]{#1\is{#1}}  

\makeatletter
\let\thetitle\@title
\let\theauthor\@author 
\makeatother

\newcommand{\togglepaper}[1][0]{ 
  \bibliography{../localbibliography}
  %% hyphenation points for line breaks
%% Normally, automatic hyphenation in LaTeX is very good
%% If a word is mis-hyphenated, add it to this file
%%
%% add information to TeX file before \begin{document} with:
%% %% hyphenation points for line breaks
%% Normally, automatic hyphenation in LaTeX is very good
%% If a word is mis-hyphenated, add it to this file
%%
%% add information to TeX file before \begin{document} with:
%% \include{localhyphenation}
\hyphenation{
affri-ca-te
affri-ca-tes
com-ple-ments
par-a-digm
Sha-ron
Kings-ton
phe-nom-e-non
Daul-ton
Abu-ba-ka-ri
Ngo-nya-ni
Clem-ents 
King-ston
Tru-cken-brodt
Tab-leau
cophono-logies
mark-edness
Ti-gri-nya
a-mong
Car-stens
Lu-bu-ku-su
}
\hyphenation{
affri-ca-te
affri-ca-tes
com-ple-ments
par-a-digm
Sha-ron
Kings-ton
phe-nom-e-non
Daul-ton
Abu-ba-ka-ri
Ngo-nya-ni
Clem-ents 
King-ston
Tru-cken-brodt
Tab-leau
cophono-logies
mark-edness
Ti-gri-nya
a-mong
Car-stens
Lu-bu-ku-su
}
  \papernote{\scriptsize\normalfont
    \theauthor.
    \thetitle. 
    To appear in: 
    Emily Clem,   Peter Jenks \& Hannah Sande.
    Theory and description in African Linguistics: Selected papers from the 47th Annual Conference on African Linguistics.
    Berlin: Language Science Press. [preliminary page numbering]
  }
  \pagenumbering{roman}
  \setcounter{chapter}{#1}
  \addtocounter{chapter}{-1}
}

\newcommand{\upstep}{\textupstep}


% \newcounter{tableauxcounter}

\renewcommand{\textltailn}{ɲ}
\renewcommand{\textbardotlessj}{ɟ}

\newcommand{\emphkh}[1]{\textit{#1}} %originally \textbf, banned by the guidelines



\definecolor{lsDOIGray}{cmyk}{0,0,0,0.45}


\newcommand{\xuparrow}[1]{%
  {\left\uparrow\vbox to #1{}\right.\kern-\nulldelimiterspace}
}
\renewcommand \textupstep[1]{\char"A71B#1}
\renewcommand \textdownstep[1]{\char"A71C#1}
 
 \newcommand{\ꜛ}{\textsf{ꜛ}}
 
\def\biberror{\undefined}


\newcommand{\OTbox}[1]{\resizebox{.88\textwidth}{!}{#1}}
 
  \togglepaper[31]
}{}


\begin{document}
\maketitle

\section{Introduction}\label{sec:ranero:1}

In this paper, I investigate the syntax of \isi{object dislocation} in \ili{Luganda} (\ili{Bantu}: JE15). Example (1a) below exemplifies object \isi{left-dislocation} (OLD) and (1b) \isi{object right-dislocation} (ORD):\footnote{All data come from my field notes except where indicated. Tone is not marked in the data. When one object/OM is relevant, it is bolded; when two objects/OMs are relevant, one is bolded as well — this does not carry any significance beyond helping the reader identify the relevant aspects of each example. In the orthography used, a <j> corresponds to a voiced palato-alveolar affricate [ʤ], a <g> before an <i> a voiced palato-alveolar affricate [ʤ], a <k> before an <i> a voiceless post-alveolar affricate [ʧ], a <ny> a palatal nasal [ɲ], a <y> a palatal approximant [j]. All others correspond to their IPA counterparts. A double vowel represents a long vowel and a double consonant a geminate. The notation {\textbar}{\textbar} between two elements indicates that they are freely ordered.  All translations are given in neutral word order, since I have not tried to replicate in \ili{English} any of the pragmatic aspects of the \ili{Luganda} data.}

\ea\label{ex:ranero:1} 
\ili{Luganda}
\ea\label{ex:ranero:1a} Object \isi{left-dislocation} (OLD)\\
\gll \textbf{A-m-envu} o-mw-ana y-a-*(\textbf{ga)}{}-gul-a.\\          
\textbf{6\textsc{aug}}\textbf{{}-6-banana} 1\textsc{aug}{}-1-child 1\textsc{sa}{}-\textsc{pst}{}-6\textbf{\textsc{om}}{}-buy\textsc{{}-fv}\\
\glt ‘The child bought the banana.’   

\ex\label{ex:ranero:1b} Object right-dislocation (ORD)\\
\gll O-mw-ana      y-a-\textbf{ga}{}-gul-a       luli,       \textbf{a-m-envu.}\\
1\textsc{aug}{}-1-child \textsc{1sa-pst}\textbf{{}-6}\textbf{\textsc{om}}{}-buy-\textsc{fv} the.other.day \textbf{6\textsc{aug}}\textbf{{}-6-banana}\\
\glt ‘The child bought the banana the other day.’ 
\z
\z

Empirically, I document the possible syntactic configurations related to object left and right-dislocation in the language, emphasizing in particular an asymmetry that becomes apparent only in ditransitive constructions. From a theoretical perspective, I propose an analysis inspired by \citet{Cecchetto1999} and \citet{Zeller2015} to capture the phenomenon. Given the complexity of the data, a number of standing issues are left for future investigation. The paper is structured as follows — in \sectref{sec:ranero:2}, I briefly discuss \isi{object dislocation} cross-linguistically and in the \ili{Bantu} family. In \sectref{sec:ranero:3}, I describe the pattern of \isi{object dislocation} in \ili{Luganda}. In \sectref{sec:ranero:4}, I present my analysis, establishing that OLD and ORD are each derived differently — OLD via base generation and ORD via \isi{movement}. \sectref{sec:ranero:5} lays out the predictions made by the proposal. Finally, \sectref{sec:ranero:6} concludes and points out areas for future research.

\section{Object marking and dislocation in Bantu}\label{sec:ranero:2}
\largerpage[2]
The analysis of \isi{object dislocation} has received significant attention \newline cross-linguistically, with a particularly rich body of work concerning the phenomenon in \ili{Romance} languages (\citealt{anagnostopouloutoappear} and references therein). Examples \REF{ex:ranero:2}a-b below show instances of \isi{object dislocation}. Note that in both examples, the \isi{direct object} is not in its canonical position (as evidenced by the prosodic break) and that the object co-occurs with a co-indexed clitic agreeing in $\varphi $-features with the object. The latter observation has led researchers to name the phenomenon clitic \isi{left-dislocation} and clitic right-dislocation, respectively:\footnote{Throughout the paper, I will use the neutral term object-dislocation for the \ili{Luganda} data. Note however, that object markers in \ili{Bantu} have been argued to be clitics \citep{diercks2015}, so the clitic \isi{left-dislocation} and clitic right-dislocation terminology might be appropriate for \ili{Bantu} as well. I leave for future research determining whether OMs in \ili{Luganda} should also be treated as clitics.}

\ea\label{ex:ranero:2}
\ili{Italian} \citep{Cecchetto1999}
\ea\label{ex:ranero:2a} Clitic \isi{left-dislocation}\\
\gll \textbf{Gianni}, io \textbf{lo}    odio. \\ 
\textbf{Gianni} I   \textbf{him} hate      \\
\glt ‘I hate Gianni.’ 

\ex\label{ex:ranero:2b} Clitic right-dislocation\\
\gll Io \textbf{lo}    odio, \textbf{Gianni}. \\
I   \textbf{him} hate  \textbf{Gianni}.\\
\glt ‘I hate Gianni.’ 
\z
\z

Object dislocation has also been investigated in the \ili{Bantu} languages. First, note that across the family, it is possible to pronominalize an object with an \isi{object marker} (henceforth OM) on the verb. This is shown below:\footnote{See \citet{Marlo2015} for an overview of OMing in \ili{Bantu}.}

\ea\label{ex:ranero:3}
\ili{Kuria} \citep{diercks2015}
\ea\label{ex:ranero:3a}
\gll n-aa-tɛm-ér-é       ómo-gámbi\\
\textsc{foc}.1sg\textsc{sa}{}-\textsc{pst}{}-hit-\textsc{perf-fv} 1-king  \\
\glt ‘I hit the king.’
\ex\label{ex:ranero:3b}
\gll n-aa-\textbf{mó}{}-tɛm-ér-e \\
\textsc{foc}.1sg\textsc{sa}{}-\textsc{pst}{}-\textbf{1\textsc{om}}{}-hit-\textsc{perf-fv}\\
\glt ‘I hit him.’
\z
\z

Of particular interest has been whether an OM can co-occur with an \textit{in-situ} object (henceforth OM doubling).\footnote{The distinction between \ili{Bantu} languages that allow OM doubling versus those that only allow an OM to co-occur with a dislocated object mirrors the long tradition of distinguishing between languages that allow clitic doubling versus those that do not—see section 4 for relevant references.} For instance, \citet{Bresnan1987} analyze OMs in Chiche\^wa as co-occurring with objects outside their canonical position (hence dislocated); in contrast OMs in \ili{Sambaa} can co-occur with \textit{in-situ} objects. \REF{ex:ranero:4} shows data from Chiche\^wa and \REF{ex:ranero:5} from \ili{Sambaa}:

\ea\label{ex:ranero:4}
Chiche\^wa \citep{Bresnan1987}\\
\gll Njûchi zi-ná-\textbf{wá}{}-lum-a            \textbf{alenje}.\\
bees     \textsc{sa-past-}\textbf{\textsc{om}}{}-bite-\textsc{indic} \textbf{hunters}\\
\glt 'The bees bit them, the hunters.’
\z

\ea\label{ex:ranero:5}
\ili{Sambaa} \citep{Riedel2009}\\
\gll N-za-\textbf{ch}i-m{}-nka             ng’wana \textbf{kitabu.}   \\
1sg\textsc{sa-perf.dj}{}-\textbf{\textsc{7om}}\textsc{{}-}\textsc{1om}{}-give 1child     \textbf{7book}\\
\glt ‘I gave the child a book.’
\z



A \ili{Bantu} language in which \isi{object dislocation} has been studied in some depth is \ili{Zulu} (\citealt{vanderSpuy1993}, \citealt{ChengDowning2009}, \citealt{zeller2009,Zeller2015}, \citealt{Halpert2015}); \REF{ex:ranero:6}a below shows an instance of \isi{left-dislocation}; \REF{ex:ranero:6}b exemplifies right dislocation:

\ea\label{ex:ranero:6}
\ili{Zulu} (\citealt{zeller2009}; \citealt{Zeller2015} respectively)\\
\ea\label{ex:ranero:6a} Object \isi{left-dislocation} (OLD)\\
\gll \textbf{UJohn}   intombazana i-\textbf{m}{}-qabul-ile.\\
\textbf{John1a}  girl9      \textsc{sa-o}\textbf{\textsc{m}}\textbf{1a}{}-kiss-\textsc{perf}\\
\glt ‘John, the girl kissed (him).’

\ex\label{ex:ranero:6b}  Object right-dislocation (ORD)\\
\gll Ngi-ya-\textbf{yi}{}-theng-a     \textbf{i-moto}\\
1\textsc{sa}{}-\textsc{dj}{}-\textbf{9\textsc{om}}{}-buy-\textsc{fv} \textbf{\textsc{aug}}\textbf{{}-9.car}\\
\glt ‘I bought (it), the car.’
\z
\z
With this background in mind, we can now turn to the pattern of OMing and \isi{object dislocation} in \ili{Luganda}.

\section{Patterns of object-dislocation in Luganda}\label{sec:ranero:3}
\subsection{Object marking in Luganda}\label{sec:ranero:3.1}
\largerpage

In this section, I describe the basic distribution of OMs and \isi{object dislocation} in \ili{Luganda}. The generalization that will arise is the following:

\ea\label{ex:ranero:7}
Object Dislocation and Object Marking (OMing) Generalization in \ili{Luganda}
\ea\label{ex:ranero:7a}
When one object is dislocated:\\
\ea\label{ex:ranero:7ai}  It must co-occur with an OM both in OLD and ORD.\footnote{Although see section 6, where I note that an object can be right-dislocated without the appearance of an OM.}
\z
\ex\label{ex:ranero:7b}  
When two objects are dislocated:\\
\ea\label{ex:ranero:7bi}  The dislocated objects occur in any order in OLD\\

\ex\label{ex:ranero:7bii}  The dislocated objects must occur in the order \textsc{goal/ben > theme} in ORD

\ex\label{ex:ranero:7biii}  In both OLD and ORD, the objects co-occur with OMs and the order of OMs is always OM\textsc{theme} > OM\textsc{goal}
\z
\z
\z

In the interest of brevity, I will not describe in detail the pragmatic interpretation of dislocated objects in \ili{Luganda}, since they align with broader cross-linguistic patterns of the phenomenon—(i) weakly quantified objects cannot dislocate, (ii) dislocated objects are interpreted as specific, and (iii) dislocated objects cannot be focused (see \citealt{Hyman1993} and \citealt{vanderwal2016} for \isi{focus} marking strategies in the language). Dislocated objects can function as a variety of topics (in the sense of \citealt{Reinhart1981}; see \citealt{ranero2015} for discussion), with some differences between left or right-dislocation. Particularly, right-dislocated objects can be exploited as afterthoughts—corrective statements to clarify part of an utterance to the interlocutor (\citealt{grosz1998}; \citealt{Villalba2000}). 

As shown in the previous section for \ili{Bantu} more broadly, objects in \ili{Luganda} can be marked on the verbal stem through an OM that agrees in \isi{noun class} with its corresponding object. I exemplify this below with a lexical ditransitive; note that \ili{Luganda} is an SVO language and the order of postverbal objects in the ditransitive examples is strictly \textsc{goal/ben > theme}:
% TODO Check that ex * thing?!?!?!?!
\ea\label{ex:ranero:8}
\ea\label{ex:ranero:8a}
\gll O-mu-sajja    y-a-w-a          a-ba-kazi   ssente.\\
\textsc{1aug-}1-man \textsc{1sa-pst}{}-give-\textsc{fv} 2\textsc{aug}{}-2-woman 9a.money\\
\glt ‘The man gave the women money.’

\ex\label{ex:ranero:8b}*O-mu-sajja y-a-w-a ssente a-ba-kazi.
\z
\z
Either of the objects can be OMed on the verb \REF{ex:ranero:9}a-b; both objects can be OMed on the verb as well, but the OMs must follow a strict ordering—OM\textsubscript{THEME} > OM\textsubscript{GOAL/BEN} \REF{ex:ranero:9}c. The reverse ordering OM\textsubscript{GOAL/BEN} >\textsubscript{} OM\textsubscript{THEME} is unacceptable \REF{ex:ranero:9}d:

\ea\label{ex:ranero:9}
\ea\label{ex:ranero:9a}
\gll O-mu-sajja    y-a-\textbf{ba}{}-w-a                  ssente.\\
1\textsc{aug}{}-1-man \textsc{1sa-pst-}\textbf{\textsc{2om}}{}-give-\textsc{fv} 9a.money\\
\glt ‘The man gave them money.’

\ex\label{ex:ranero:9b}
\gll O-mu-sajja    y-a-\textbf{zi}{}-w-a            a-ba-kazi.\\
1\textsc{aug}{}-1-man 1\textsc{sa}{}-\textsc{pst}{}-\textbf{9a\textsc{om}}{}-give-\textsc{fv} 2\textsc{aug}{}-2-woman\\
\glt ‘The man gave the women it.’

\ex\label{ex:ranero:9c}
\gll O-mu-sajja  y-a-\textbf{zi-ba}{}-w-a. \\       
\textsc{aug-}1-man \textsc{1sa-pst}{}-\textbf{\textsc{9aom-2om}}{}-give-\textsc{fv}\\
\glt ‘The man gave them it.’

\ex\label{ex:ranero:9d}*O-mu-sajja y-a-\textbf{ba-zi}{}-w-a.
\z
\z

As noted in the introduction, it has long been a concern in the \ili{Bantu} literature whether OM doubling configurations are licit in particular languages. In \ili{Luganda}, it is impossible for an OM to co-occur with an \textit{in-situ} object, as evidenced by several diagnostics.\footnote{Some diagnostics used in the \ili{Bantu} literature to diagnose object-dislocation are not applicable to \ili{Luganda}. These include the \isi{conjoint}/\isi{disjoint} alternation in languages like \ili{Zulu} \citep{Zeller2015} and penultimate vowel lengthening to indicate the edge of a phrase (also in \ili{Zulu}; \citealt{ChengDowning2009}). I leave for future investigation the applicability of tonal diagnostics to determine the edge of phrases in \ili{Luganda} (as in Chiche\^wa; \citealt{Bresnan1987}).} First, a prosodic pause is obligatory before an object in the right-periphery if it co-occurs with an OM on the verb, suggesting that the object is \textit{ex-situ}. This diagnostic has been extensively used in the \ili{Romance} literature (for instance \citealt{Cecchetto1999}, \citealt{Cruschina2011}, \citealt{anagnostopouloutoappear}).\footnote{This diagnostic is a one-way diagnostic—that is, the presence of a pause shows that the object is \textit{ex-situ}, but the absence of a pause is not definitive evidence that the object is \textit{in-situ} (see \citealt{dierckstoappear} for \ili{Lubukusu}; \citealt{diercks2015} for \ili{Kuria}). An anonymous reviewer asks to define more precisely what I mean by “prosodic pause” here. What I mean is that there is a short break in my consultant’s flow of speech before the right-dislocated object. I acknowledge that it would be useful to investigate what the acoustic correlates of this break are and whether there are other effects related to melodic contours, vowel lengthening, or tonal processes. I leave this for future research.} An example is shown below; note the obligatory pause before the object:\footnote{Note that here an OM co-occurs with the right-dislocated object. In the final section, I point out the existence of a construction in which an object is right-dislocated but no OM appears.}

\ea\label{ex:ranero:10}
\gll Aisha    y-a-\textbf{bi}{}-lab-a        luli           *(,) \textbf{e-bi-nyonyi}.\\
1.Aisha \textsc{1sa-pst-}\textbf{\textsc{8om}}{}-see-\textsc{fv} the.other.day   {}  \textbf{8\textsc{aug}}\textbf{{}-8-bird}\\
\glt ‘Aisha saw the birds the other day.’
\z

Second, the placement of temporal adverbs to demarcate the edge of the verb phrase has been used by others to diagnose OM doubling in \ili{Bantu} (\citealt{Henderson2006}; \citealt{Riedel2009} for \ili{Sambaa}; \citealt{bax2012} for \ili{Manyika}; \citealt{dierckstoappear} for \ili{Lubukusu}; \citealt{zeller2009,Zeller2015} for \ili{Zulu}). If an object is to the left of the \isi{temporal adverb}, it is \textit{in-situ}, whereas an object to the right of the \isi{temporal adverb} is in a dislocated position. In \ili{Luganda}, if the object occurs to the left of the \isi{temporal adverb} \textit{luli,} an OM corresponding to the object cannot appear:

\ea\label{ex:ranero:11}
\gll *O-m-wana    y-a-\textbf{ga}{}-gul-a               \textbf{a-m-envu}           luli.\\
1\textsc{aug}{}-1-child \textsc{1sa-pst-}\textbf{\textsc{6om}}{}-buy-\textsc{fv} \textbf{6\textsc{aug}}\textbf{{}-6-banana} the.other.day\\
\glt Intended: ‘The child bought the banana the other day.’
\z

In contrast, if the object is to the right of the \isi{temporal adverb}, the OM can appear on the verb. I take this to mean that OMs in \ili{Luganda} can only co-occur with dislocated objects:\footnote{An anonymous reviewer asks whether using manner adverbials would be a better diagnostic to demarcate the edge of the verbal phrase, since temporal adverbs could be adjoined as high as TP. Data using manner adverbs were also collected and the pattern is the same as with temporal adverbs. Examples with a manner adverb are shown in (34) and (37).} 

\ea\label{ex:ranero:12}
\gll O-m-wana     y-a-\textbf{ga}{}-gul-a         luli,           \textbf{a-m-envu}.\\
1\textsc{aug}{}-1-child  \textsc{1sa-pst}{}-\textbf{\textsc{6om}}\textsc{{}-}buy-\textsc{fv}  the.other.day  \textbf{6\textsc{aug}}\textbf{{}-6-banana}\\
\glt ‘The child bought the banana the other day.’
\z

Finally, we can construct a ditransitive utterance in which one of the objects is clearly \textit{in-situ}; attempting to double the object with an OM is unacceptable. Consider the following example, where the \textsc{goal/ben} is to the left of a weakly quantified object. Weakly quantified objects function as indefinites and as such cannot be topics (see \citealt{Diesing1992} on indefinites and \citealt{Reinhart1981} on why quantificational phrases cannot be interpreted as topics). Given that dislocated positions in \ili{Luganda} are reserved for topics, we expect weakly quantified objects to be \textit{in-situ} rather than dislocated. Since the \textsc{goal/ben} argument is to the left of the weakly quantified object, it must also be \textit{in-situ}:

\ea\label{ex:ranero:13}
\gll *Nakayiza  y-a-\textbf{mu}{}-w-a                   \textbf{Lukwaago}    e-bi-rabo            bitono\\
1.Nakayiza \textsc{1sa-pst-}\textbf{\textsc{1om}}{}-give-\textsc{fv} \textbf{1.Lukwaago} 8\textsc{aug}{}-8-present 8.few\\
\glt Intended: ‘Nakayiza gave Lukwaago few gifts.’ (Jenneke van der Wal field notes)
\z

Given the previous discussion, we arrive at the following generalization—an OM can never double an \textit{in-situ} object in \ili{Luganda}, but it can co-occur with a dislocated object.

\subsection{Object left-dislocation}\label{sec:ranero:3.2}

As shown before, OMs can only co-occur with an object in \ili{Luganda} if the object has been dislocated. Let us first explore the pattern of OLD. An object in \ili{Luganda} can be dislocated to a pre-verbal position—the left-dislocated object can either precede or follow the \isi{subject}, as shown by the examples in \REF{ex:ranero:14}a-b below.\footnote{An anonymous reviewer asks whether dislocation of the external argument was studied as well. Note that in \REF{ex:ranero:14}b, the \isi{subject} must be left-dislocated, since it precedes the left-dislocated object. In \sectref{sec:ranero:5}, \isi{subject} left and right-dislocation are used to test the predictions of the analysis. However, I leave for future research a full investigation of how dislocating the external argument interacts with \isi{object dislocation}.} Note crucially that OMing the object is obligatory and failing to do so is unacceptable:\footnote{A comma indicates a prosodic pause. A pause after a left-dislocated object is optional.}\textsuperscript{,}\footnote{Throughout all the dislocation examples, I will maintain a neutral translation that does not attempt to reflect the \isi{information structure} considerations that render these constructions licit; I briefly discuss these \isi{information structure} constraints, but refer the reader to \citet{ranero2015} for a more complete discussion.}

\ea\label{ex:ranero:14}
\ea\label{ex:ranero:14a}
\gll \textbf{A-m-envu},         o-m-wana       y-a-*(\textbf{ga}{}-)gul-a.\\
\textbf{6\textsc{aug}}\textbf{{}-6-banana} 1\textsc{aug}{}-1-child \textsc{1sa-pst-}\textbf{\textsc{6om}}{}-buy-\textsc{fv}\\
\glt ‘The child bought the bananas.’

\ex\label{ex:ranero:14b} O-m-wana \textbf{a-m-envu}   y-a-*(\textbf{ga-})gul-a.
\z
\z

An object lacking the augment vowel cannot be left-dislocated, regardless of whether it is OMed or not. Augmentless nouns are in \isi{focus} \citep{Hyman1993}, so this suggests that dislocated objects cannot be focused. An example with an augmentless noun is shown below in \REF{ex:ranero:15}:\footnote{This relates to the observation before regarding the \isi{information structure} constraints on dislocated objects, which can only function as topics.}

\ea[*]  
{\gll \textbf{M-envu} o-m-wana      y-a-(\textbf{ga}{}-)gul-a.\\
\textbf{6-banana} 1\textsc{aug}{}-1-child \textsc{1sa-pst}{}-\textbf{6\textsc{om}}{}-buy-\textsc{fv}\\
\glt Intended: ‘The child bought the bananas.’ \label{ex:ranero:15}}
\z

In lexical ditransitives, either of the objects can be left-dislocated. As with previous OLD examples, OMing the dislocated object is obligatory; this is shown in \REF{ex:ranero:17}a-b:

\ea\label{ex:ranero:16}
\gll Aizaka y-a-w-a           a-ba-kazi     e-ki-rabo.\\
1.Isaac \textsc{1sa-pst}{}-give-\textsc{fv} 2\textsc{aug}{}-2-woman 7\textsc{aug}{}-7-gift\\
\glt ‘Isaac gave the women a gift.’
\z

\ea\label{ex:ranero:17}
\ea\label{ex:ranero:17a}
\gll \textbf{E-ki-rabo}    Aizaka y-a-*(\textbf{ki-})w-a        a-ba-kazi.\\
\textbf{7\textsc{aug}}\textbf{{}-7-gift} 1.Isaac \textsc{1sa-pst}{}-\textbf{7\textsc{om}}{}-give-\textsc{fv} 2\textsc{aug}{}-2-woman\\
\glt ‘Isaac gave the women a gift.’

\ex\label{ex:ranero:17b}
\gll \textbf{A-ba-kazi}          Aizaka y-a-*(\textbf{ba}{}-)w-a              e-ki-rabo.\\
\textbf{2\textsc{aug}}\textbf{{}-2-woman} 1.Isaac \textsc{1sa-pst}{}-\textbf{2\textsc{om}}{}-give-\textsc{fv} 7\textsc{aug}{}-7-gift\\
\glt ‘Isaac gave the women a gift.’
\z
\z

Both objects can be left-dislocated in either order. If both objects are left-dislocated—regardless of the ordering in which they are dislocated—the OMs on the verb must follow the OM\textsubscript{THEME} > OM\textsubscript{GOAL/BEN} order. This is shown in \REF{ex:ranero:18}a-b below:

\ea\label{ex:ranero:18}
\ea\label{ex:ranero:18a}
\gll \textbf{E-ki-rabo} a-ba-kazi           Aizaka y-a-\textbf{ki}{}-ba{}-w-a.\\
\textbf{7\textsc{aug}}\textbf{{}-7-gift} 2\textsc{aug}{}-2-woman 1.Isaac \textsc{1sa-pst-}\textbf{\textsc{7om}}\textsc{{}-}\textsc{2om}{}-give-\textsc{fv}\\
\glt ‘Isaac gave the women a gift.’

\ex\label{ex:ranero:18b}
\gll A-ba-kazi          \textbf{e-ki-rabo} Aizaka y-a-\textbf{ki}-ba-w-a.\\
\textsc{2aug}-2-woman \textbf{\textsc{7aug}}\textbf{-7-gift} 1.Isaac \textsc{1sa-pst-}\textbf{\textsc{7om}}-\textsc{2om}-give-\textsc{fv}\\
\glt ‘Isaac gave the women a gift.’
\z
\z

In contrast, if the ordering of OMs on the verb is OM\textsubscript{GOAL/BEN} > OM\textsubscript{THEME}, left-dislocating both objects in either order is unacceptable, showing that the ordering of OMs must be strictly OM\textsubscript{THEME} > OM\textsubscript{GOAL/BEN}:

\ea\label{ex:ranero:19}
\ea\label{ex:ranero:19a}
*Ekirabo abakazi Aizaka ya\textbf{baki}wa.  

\ex\label{ex:ranero:19b}
*Abakazi ekirabo  Aizaka ya\textbf{baki}wa.
\z
\z

If neither or only one of the left-dislocated objects is OMed, the utterance is unacceptable, as shown below in (\ref{ex:ranero:20}a-f):

\ea\label{ex:ranero:20} 
\ea\label{ex:ranero:20a}
*Ekirabo abakazi Aizaka yawa.

\ex\label{ex:ranero:20b}
*\textbf{Ekirabo} abakazi Aizaka ya\textbf{ki}wa.  

\ex\label{ex:ranero:20c}
*Ekirabo \textbf{abakazi} Aizaka ya\textbf{ba}wa.  

\ex\label{ex:ranero:20d}
*Abakazi ekirabo Aizaka yawa.

\ex\label{ex:ranero:20e}
*\textbf{Abakazi} ekirabo Aizaka ya\textbf{ba}wa.

\ex\label{ex:ranero:20f}
*Abakazi \textbf{ekirabo} Aizaka ya\textbf{ki}wa.
\z
\z

All the patterns described here are replicated with applicative and causative constructions (see \citealt{ranero2015}). The essential observation of OLD for the purposes of the upcoming analysis is the following: in ditransitive constructions, either or both objects can be left-dislocated in either order, but the ordering of OMs is strictly OM\textsubscript{THEME} > OM\textsubscript{GOAL/BEN}. 

\subsection{Object right-dislocation}
\label{sec:ranero:3.3}

An object in \ili{Luganda} can be dislocated to a position in the right periphery; an example in a monotransitive clause is shown below. Recall that objects to the right of a \isi{temporal adverb} are dislocated:\footnote{Further evidence for this claim comes from the observation that weakly quantified objects cannot appear to the right of a \isi{temporal adverb}

\ea[*]{
\gll Aisha   y-a-(bi-)lab-a             \textbf{luli},                e-bi-wugulu  bitono.\\
	1.Aisha  \textsc{1sa-pst-8om}{}-see-\textsc{fv} \textbf{the.other.day} 8\textsc{aug}{}-8-owl  8.few\\
	\glt Intended: ‘Aisha saw few owls the other day.’
	}
\zlast
} and note that an OM co-occurs with the dislocated object: 

 
\ea[]{
\gll Aisha      y-a-\textbf{bi}{}-lab-a        luli,        \textbf{e-bi-nyonyi}.\\
1.Aisha \textsc{1sa-pst-}\textbf{\textsc{8om}}{}-see-\textsc{fv} the.other.day \textbf{8\textsc{aug}}\textbf{{}-8-bird}\\
\glt ‘Aisha saw the birds the other day.’  \label{ex:ranero:21}
}
\z

As with OLD, an augmentless object cannot be right-dislocated:\footnote{Regardless of whether the OM is present or not; see \sectref{sec:ranero:6} for an example of ORD without an OM.}

\ea[*]{
\gll Aisha  y-a-\textbf{bi}{}-lab-a         luli,         \textbf{bi-nyonyi.}\\
1.Aisha \textsc{1sa-pst-}\textbf{8\textsc{om}}-see\textsc{-fv}  the.other.day \textbf{8-bird}\\
\glt Intended: ‘Aisha saw the birds the other day.’ 
 \label{ex:ranero:22}}
\z

In ditransitive constructions, either the \textsc{goal/ben} or \textsc{theme} argument can be right-dislocated; note that an OM co-occurs with the right-dislocated object:\footnote{I exemplify throughout with an applicative construction, although the pattern is replicated as well with lexical ditransitives (see \citealt{ranero2015}).}

\ea    \label{ex:ranero:23}
\gll Namugga    y-a-\textbf{ba-}fumb-ir-a                     e-n-gege         luli,            \textbf{a-ba-ana.}\\
1.Namugga \textsc{1sa-pst-}\textbf{\textsc{2om}}{}-cook-\textsc{appl-fv}9 \textsc{aug}{}-9-tilapia the.other.day \textbf{2\textsc{aug}}\textbf{{}-2-child}\\
\glt ‘Namugga cooked the tilapia for the children the other day.’
\z

\ea\label{ex:ranero:24}
\gll Namugga    y-a-\textbf{gi-}fumb-ir-a        a-ba-ana        luli,    \textbf{e-n-gege}.\\
1.Namugga \textsc{1sa-pst-}\textbf{\textsc{9om}}{}-cook-\textsc{appl-fv} 2\textsc{aug}{}-2-child the.other.day \textbf{9\textsc{aug}}\textbf{{}-9-tilapia}\\
\glt ‘Namugga cooked the tilapia for the children the other day.’
\z

Both objects can be right-dislocated in a ditransitive construction. The objects must be dislocated in the order \textsc{goal/ben > theme} and the OMs on the verb must be strictly ordered OM\textsubscript{THEME} > OM\textsubscript{GOAL/BEN}:


\ea\label{ex:ranero:25}
\gll Namugga     y-a-gi{}-\textbf{ba-}fumb-ir-a                     luli,                \textbf{a-ba-ana}      e-n-gege.\\
1.Namugga \textsc{1sa-pst-9om}\textsc{{}-}\textbf{\textsc{2om}}{}-cook-\textsc{appl-fv} the.other.day \textbf{2\textsc{aug}}\textbf{{}-2-child} 9\textsc{aug}{}-9-tilapia\\
\glt ‘Namugga cooked the tilapia for the children the other day.’
\z

Right dislocating the objects in the order \textsc{theme > goal/ben} is unacceptable, as in \REF{ex:ranero:26}; OMing in the order OM\textsubscript{GOAL/BEN} > OM\textsubscript{THEME} is unacceptable regardless of the ordering of the right-dislocated objects, as in (\ref{ex:ranero:27}a-b):

\ea\label{ex:ranero:26}
* Namugga y-a-gi{}-\textbf{ba}{}-fumb-ir-a luli, e-n-gege \textbf{a-ba-ana.}
\z

\ea\label{ex:ranero:27}
\ea\label{ex:ranero:27a}

*Namugga y-a\textbf{{}-ba-}gi{}-fumb-ir-a luli, \textbf{a-ba-ana} e-n-gege.

\ex\label{ex:ranero:27b}
*Namugga y-a-\textbf{ba-}gi{}-fumb-ir-a luli, e-n-gege \textbf{a-ba-ana}.
\z
\z

The essential aspects of ORD are the following: in ditransitives, if both objects are right-dislocated, not only is the ordering of OMs strictly OM\textsubscript{THEME} > OM\textsubscript{GOAL/BEN} (as with the \isi{left-dislocation} pattern), but the ordering of the dislocated objects is also strict—\textsc{goal/ben > theme}. 

\section{Analysis}\label{sec:ranero:4}
\largerpage 
The literature on generative approaches to the syntax of object-dislocation is extensive. In particular, debates have centered on whether dislocated objects surface in their position through base generation or \isi{movement}, a distinction that I will argue allows us to explain the asymmetry we observed regarding dislocation of both objects in ditransitives in OLD vs. ORD. While it is not my purpose to review the literature in detail, the following are representative of different approaches. Analyzing \isi{left-dislocation} as base generation, \citet{Cinque1990}, \citet{iatridou1995}, \citet{Anagnostopoulou1994}, \citet{Suñer2006}, \citet{De2007} are representative; analyzing the phenomenon as the result of \isi{movement}, \citet{kayne1994}, \citet{Zubizarreta1998}, and \citet{zeller2009}. Moving on to right-dislocation, \citet{kayne1994} and \citet{Cardinaletti2002} treat the phenomenon as base generation, while \citet{Kayne1995}, \citet{Cecchetto1999}, \citet{Zeller2015} and \citet{Samek-Lodovici2016} treat it as \isi{movement}. Given the variety of possible analyses, I will make my proposal and explore its predictions. In so doing, I bring \ili{Luganda} to bear on the issue of the analysis of these phenomena, while also highlighting another instance of a left vs. right periphery asymmetry that deserves further investigation.

First, let us summarize the core of the proposal:

\ea\label{ex:ranero:28}
Object-dislocation in \ili{Luganda} 

\ea\label{ex:ranero:28a}
Object \isi{left-dislocation} and right-dislocation in \ili{Luganda} are not derived through the same mechanism.
\ex\label{ex:ranero:28b}
Left-dislocated objects are base generated.
\ex\label{ex:ranero:28c}
Right-dislocated objects arise in their surface position via \isi{movement}.
\z
\z

This proposal is similar in spirit to an argument made for the analysis of dislocation in \ili{Romance} languages in \citet{Cecchetto1999}, which rejected the hypothesis from \citet{Vallduví1992} that clitic right-dislocation is simply the “mirror image” of clitic \isi{left-dislocation}. Let us now turn to the analytical assumptions which lead me to propose \REF{ex:ranero:28}. I take a Minimalist approach couched in the Agree based system (\citealt{Chomsky2000} and subsequent work). I assume the operation Merge to come in (at least) two flavors: External Merge, which is when an object not previously introduced into the derivation is taken from the Numeration and merged, and Internal Merge, which involves taking an item previously introduced into the derivation and merging it, resulting in Movement. I assume that Internal Merge (Movement) is driven by an operation Agree, which involves feature-valuation between a Probe and Goal:

\ea\label{ex:ranero:29}
Agree
\z
\begin{quote}Operation in which a Probe enters into a relation with a Goal it c-commands. The operation applies when a Probe bears an unvalued feature [uF] and enters into an Agree relation with a Goal bearing a valued feature [iF].	
\end{quote}

\largerpage 
Unvalued features must be valued in the course of the narrow syntactic derivation in order to avoid a crash—that is, unvalued features may not arrive at LF without having been valued through the Agree operation. An additional ingredient to Movement involves an EPP feature on the Probe. An EPP feature dictates that \isi{movement} must occur, so the Goal raises locally to the specifier of the Probe head. An illustration of \isi{movement} in the context of \textit{wh}{}-features is observed below; notice crucially that the probe is looking for a Goal with the relevant feature (in this case \textit{wh}{}-features); if there were an intervening DP that did not possess the relevant feature, the probe would ignore it and no intervention effect would arise:  

\begin{figure}
%\includegraphics{figures/fig-ranero-1.png}	
\begin{forest} fairly nice empty nodes
	[CP
		[DP] [
		[C, name=c] [
			[{[}\st{uWh,} EPP{]}] [\st{DP}\\{[}Wh{]},name=Wh,base=top,align=center]
			]
		]
	]
\draw[dashed] (c) to [bend right=50, looseness=.7] node[near start,below] {Agree} (Wh);	
\end{forest}
\caption{Agree and movement}
\label{fig:ranero:1}
\end{figure}  

As can be observed from \figref{fig:ranero:1} as well, I assume that moved elements leave behind a copy—thus I also assume the Copy Theory of Movement \citep{Chomsky1995}. Copies that are left behind from \isi{movement} are readable at LF and contribute to the interpretation of the utterance. If there are several copies of an element in the derivation that is shipped to LF, then LF has a choice as to which copy to interpret, thus accounting for sentences where several readings are possible. As will be observed later on, the existence of these copies make predictions regarding the interpretation of sentences where I analyze that \isi{movement} has taken place. Furthermore, I also assume that in carrying out the Agree operation, Locality is essential. I define Locality below (see \citet{Zeller2015} for a similar definition):

\ea\label{ex:ranero:30}
Locality
\z
\begin{quote}
A Probe P with an unvalued feature [uF] enters into an Agree relation with a Goal G if G is the closest element bearing a valued Feature [iF]. If there are two Goals G and G$'$ in P’s c-command domain, then G is closer to P than G$'$ if G asymmetrically c-commands G$'$.
\end{quote}

Another assumption I will make is that copies of moved elements do not intervene between a Probe and a Goal for Locality purposes. When there are two potential Goals with a relevant Feature, a Probe (P) with an [EPP] feature searches its c-command domain and Agrees with the closest Goal (G). Once this Goal (G) has been moved, a second Probe (P$'$) can then search its c-command domain and reach another Goal (G$'$). The copy left behind by G between this second Probe (P$'$) and second Goal (G$'$) does not count as an intervener. This is illustrated below:


\begin{figure}
% \includegraphics{figures/fig-ranero-2.png}	
\begin{forest} nice empty nodes 
  [~
    [G$'$,minimum width=2cm] 
      [
      [P$'$,name=pprime,minimum width=2cm] 
      [
        [G,minimum width=2cm]
        [
          [P,name=p,minimum width=2cm]
          [
            {\ldots}
            [\st{G},name=g,minimum width=2cm] 
            [
              [\st{G$'$}, name=gprime,minimum width=2cm] 
              [~]
            ]
          ]
        ]
      ]
    ]
  ]
\draw[dashed] (pprime) to [bend right=90] (gprime);	
\draw[dashed] (p.south) to [out=260, in=190] (g.south);
\end{forest}	
\caption{Locality and intervention}
\label{fig:ranero:2}
\end{figure}    
    

{With these assumptions in place, we can move to the specifics of the analysis. I propose following \citet{Zeller2015} that right-dislocated objects that co-occur with an OM on the verb move to the right-branching specifier of an optional projection immediately above} v{, which is labeled TopP in what follows}.\footnote{Right-branching specifiers have been proposed to account for word order in a variety of languages. For instance, \citet{Chung1998} provides an array of diagnostics showing that specifiers branch rightwards in \ili{Chamorro} (Austronesian), while \citet{Aissen1992} accounts for VOS order in \ili{Mayan} languages through the \isi{subject} occupying a right-branching specifier.}{ The \isi{movement} of the object is triggered by an Agree operation between the head of the projection Top, which is specified for an unvalued topic feature [uTop]}\footnote{\citet{Zeller2015} calls this feature “anti-\isi{focus}”, primarily because non-focused DPs in \ili{Zulu} must vacate the vP. Given that this does not apply to \ili{Luganda}, I use [Top] as the relevant feature, given the interpretation of the dislocated objects.} {and unvalued $\varphi $-features [u$\varphi $], and a Goal bearing valued topic [iTop] and valued $\varphi $-features [i$\varphi $]}.\footnote{An anonymous reviewer asks why the external argument does not intervene. I assume that the external argument does not carry an [iTop], so it cannot be an intervener for the Top that is searching for this specific feature—the object is the first relevant DP carrying the feature. Whether features relevant to information-structure considerations are active in the narrow syntax is an issue of ongoing debate in the literature, particularly among proponents and critics of the cartographic approach (\citealt{rizzi1997} and subsequent work); see for instance 
\citet{landmantoappear} for a proposal in favor of such an architecture in \ili{Bantu} and \citet{horvath2007} for a contrary position to the general idea.}{ It is crucial for our analysis that the main probe is the [uTop] and the [u$\varphi $] is parasitic on the main probe; we thus ensure that OMs never double an} in-situ {object, but only topicalized dislocated ones}.\footnote{An anonymous reviewer asks what we mean by the [$\varphi $] features being parasitic on [Top]. I simply mean to capture the fact that OMs never occur unless the Top head is merged; this head then enters into an Agree relation with an object that is a topic and the $\varphi $-agreement is spelled-out as the OM. Note that Top enters into an Agree relation with pro and an OM is spelled out in cases where there is no overt object at all—see \REF{ex:ranero:9}a-c.}{ When the head of the projection Top acts as a Probe and searches its c-command domain, it finds a DP with valued topic features [iTop], triggering an Agree relation}.\footnote{I crucially assume the Weak Phase Impenetrability Condition; the complement of the v phase does not become unavailable for syntactic computation until the higher C phase head is merged \citep{Citko2014}.}{ The head Top carries an [EPP] feature that causes the DP object with which it agrees to move to a right-branching specifier, resulting in a right-dislocation configuration. The Agreement operation also results in the spell-out of the valued $\varphi $-features on the head Top as the \isi{object marker} OM, which then joins with the verb as the verb moves up through the structure to reach its final landing place, accounting for the morpheme order}.\footnote{An alternative placement for the Topic projection would be high in the left-periphery. However, note that the placement of the OM immediately before the root should reflect the syntactic configuration, in adherence to the Mirror Principle \citep{Baker1985}. Therefore, I propose the existence of the low Top position in \ili{Luganda}.}{ Given space considerations, I do not illustrate the analysis with monotransitives, but move directly to the most complex case, with two objects. An illustration of double \isi{object right-dislocation} is shown below in \figref{fig:ranero:3}. The curved line indicates an Agreement relation and the arrow indicates \isi{movement}:}

\largerpage
% TODO Ask what to do about the picture being inside the ex
\ea\label{ex:ranero:31}
\gll Namugga     y-a-gi{}-\textbf{ba-}fumb-ir-a                     luli,                \textbf{a-ba-ana}      e-n-gege.\\
1.Namugga \textsc{1sa-pst-9om}\textsc{{}-}\textbf{\textsc{2om}}{}-cook-\textsc{appl-fv} the.other.day \textbf{2\textsc{aug}}\textbf{{}-2-child} 9\textsc{aug}{}-9-tilapia\\
\glt ‘Namugga cooked the tilapia for the children the other day.’ (repeated from \REF{ex:ranero:25})
\z

\newpage 
\begin{figure}
%\includegraphics{figures/fig-ranero-3.png}
\resizebox{.8\textwidth}{!}{
\begin{forest} fairly nice empty nodes
	[TopP
		[
			[Top\\OM\textsubscript{\scshape theme},name=TopOMTheme1,align=center,base=top] [TopP 
				[
					[Top\\OM\textsubscript{\scshape goal\slash ben},name=topomgoalben,align=center,base=top] [\textit{v}P
						[DP\textsubscript{\scshape subj}] [
							[\textit{v}] [VP
								[\st{DP}\textsubscript{\scshape \st{goal\slash ben}},name=dpgoalben2] [
									[V] [\st{DP}\textsubscript{\scshape \st{theme}},name=dptheme]
								]
							]
						]
					]
				] [DP\textsubscript{\scshape goal\slash ben},name=dpgoalben]
			]
		] [DP\textsubscript{\scshape theme},name=dptheme1]
	]
\draw (TopOMTheme1.south)	to [bend right=90,looseness=1.5] (dptheme);
\draw (topomgoalben.south) to [bend right] (dpgoalben2.west);
\begin{pgfinterruptboundingbox}
\draw[-{Triangle[]}] (dptheme.60) |- (dptheme1);
\draw[-{Triangle[]}] (dpgoalben2.south) |- ++(2em,-4em) -| ++(8em,2em) |- (dpgoalben);
\end{pgfinterruptboundingbox}
\end{forest}	
}
\caption{Double object right-dislocation}
\label{fig:ranero:3}
\end{figure}   

 


Let us summarize the essential steps in the derivation above. The first Top head merges above \textit{v}P and searches its c-command domain—given Locality, it finds the DP\textsc{goal/ben}, which moves to a rightward specifier. When a second Top is merged (given proper discourse configurations), it searches its c-command domain for a goal and finds the DP\textsc{theme}, which moves as well. Therefore, when two DPs carry a Topic feature, the DP\textsc{goal/ben} will raise to SpecTopP of the lower TopP, while the DP\textsc{theme} will raise to SpecTopP of the higher TopP; we have thus derived the strict ordering of dislocated DPs in right-dislocation.\footnote{This immediately highlights the virtue of this analysis over one that would assume the antisymmetric program \citep{kayne1994}, which bans rightward specifiers. Under such an approach, right dislocation would have to be derived in \ili{Luganda} via \isi{movement} of the DP objects to leftward-specifiers, followed by remnant \isi{movement} of the \textit{v}P above them — however, note that that account would predict the wrong strict ordering of the dislocated objects (DP\textsc{theme} > DP\textsc{goal/ben}). Given this strikingly inaccurate prediction, we do not take such an approach, noting additionally that the antisymmetric program has been called into question for independent reasons \citep{Abels2009}.} Crucially, we have also accounted for the ordering of the OMs—given our analysis, the OM\textsubscript{GOAL/BEN} surfaces closer to the verb root. Since the right-dislocated object is outside the \textit{v}P, which I take to be a prosodic domain, we can also straightforwardly account for the obligatory presence of a pause between \textit{v}P internal elements and the right-dislocated objects.

\largerpage[2]
{Let us now turn to OLD. In contrast to the previous discussion, I propose that a left-dislocated object is base generated in its surface position in the specifier of an XP}\footnote{I could have called this TopP as well, but I call it XP to avoid confusion with ORD.} {projection above TP. The obligatory OM in \isi{left-dislocation} constructions arises via an Agree relation between the head Top that searches its c-command domain for a Goal bearing an unvalued Top feature [iTop]. The Goal that Top finds is a} pro {argument that is co-referential with the DP base generated in left-dislocated position; the left-dislocated object binds the null} pro\footnote{Given that pro is phonetically null, it is irrelevant for our purposes whether Top carries an [EPP] feature in examples like these and pro raises to the right-branching specifier of Top. An anonymous reviewer asks how we ensure that \isi{left-dislocation} does not co-occur with an overt object in base position. In other languages that allow object \isi{left-dislocation}, having an object in base position as well is unacceptable:

\ea
	\gll  *A Juan, yo lo  vi     a Juan. \\
	 a  Juan  I    \textsc{cl} saw a Juan \\
	 \glt  Intended: ‘I saw Juan.’ (Spanish)
\z
There certainly exist phenomena where multiple links in a chain are realized \citep{Nunes2004}, but my analysis of OLD does not involve \isi{movement}. There could be two reasons then for a left-dislocated object not co-occurring with an overt object in base position: (i) as a result of the base generation analysis versus a \isi{movement} one, or (ii) pragmatic reasons that have nothing to do with the syntax—repetition is simply dispreffered. I leave for future research exploring whether a base generation analysis of object-dislocation excludes the pronunciation of the dislocated object and an identical object in base position due to syntactic or extra-grammatical reasons.}{. The \isi{subject} raises to SpecTP, accounting for the observed word order. Given space considerations, I illustrate the analysis with a \isi{double object} construction outright:}

\ea\label{ex:ranero:32}
\gll \textbf{E-ki-rabo} a-ba-kazi           Aizaka y-a-\textbf{ki}{}-ba{}-w-a.\\
\textbf{7\textsc{aug}}\textbf{{}-7-gift} 2\textsc{aug}{}-2-woman 1.Isaac \textsc{1sa-pst-}\textbf{\textsc{7om}}\textsc{{}-}\textsc{2om}{}-give-\textsc{fv}\\
\glt ‘Isaac gave the women a gift.’ (repeated from \REF{ex:ranero:18}a)
\z
\newpage 
\begin{figure}
%\includegraphics{figures/fig-ranero-4.png}
\begin{forest}
	fairly nice empty nodes, for tree={
		l sep=.1cm,
		s sep=0.1cm}
	[XP
		[DP\textsubscript{\scshape theme}] [
			[X] [XP
				[DP\textsubscript{\scshape goal\slash ben}] [
					[X] [TP
						[DP\textsubscript{\scshape subj}] [
							[T] [\ldots
								[] [TopP [
									[Top\\OM\textsubscript{\scshape theme},name=omtheme,align=center,base=top] [TopP [
										[Top\\OM\textsubscript{\scshape goal\slash ben},name=Topgoalben, align=center, base=top] [\textit{v}P
											[\st{DP}\textsubscript{\st{\scshape subj}}] [
												[\textit{v}] [
													[\textit{pro}\textsubscript{\scshape goal\slash ben}, name=progoalben] [VP
														[V] [\textit{pro}\textsubscript{\scshape theme},name=protheme]
													]
												]
											]
										]
						]	[~]		]
						]	[~] 	] 
							]
						]
					]
				]
			]
		]
	]
\draw (omtheme) to [bend right=80] (protheme);
\draw (Topgoalben) to [bend right=45] (progoalben);
\end{forest}	
\caption{Double object left-dislocation}
\label{fig:ranero:4}
\end{figure}


{Given Locality, the Top merged first will find the DP}\textsc{goal/ben} {argument and Agree with it, resulting in the spell-out of an OM. The Top merged above it will then search its c-command domain and find the DP}\textsc{theme} {argument, resulting in the spell-out of the second OM. Base generation allows for the left-dislocated objects to be ordered freely, so the position of the dislocated DP objects could be swapped, accounting for the two data points in \REF{ex:ranero:18}a-b. Note crucially that the way we derive the OMs is the same between object left and right-dislocation, thus accounting for their identical ordering in both constructions. We therefore derive the strict ordering of the OMs, while also deriving the free ordering of both objects in \isi{left-dislocation} and the strict ordering of both objects in right-dislocation. In the next section, I show that several predictions made by the analysis are borne out.}\footnote{The analysis presented here contrasts with \ili{Zulu} in two ways. First, \ili{Zulu} allows for double-\isi{object dislocation}, but only for one OM on the verb (though \citealt{Adams2010} claims that a second OM in double object-dislocation constructions is phonologically null; see \citealt{Zeller2015} for discussion); second, \citet{zeller2009} claims that OLD is derived via \isi{movement}, even if both left-dislocated objects are ordered freely (see fn.27). Given that OMing in other languages such as Chiche\^wa is restricted thematically, we do not delve into the details of their analysis, though see \citet{Bresnan1987} for a seminal treatment of objects and OMs in that language.}


\section{Predictions of the analysis}\label{sec:ranero:5}
\subsection{Principle C violations}\label{sec:ranero:5.1}

{In this section, I show that three predictions made by my account are borne out, suggesting that the base generation vs. \isi{movement} approach to left and right object-dislocation in \ili{Luganda} is on the right track}.\footnote{The three diagnostics presented in this section follow \citet{zeller2009}, which explores OLD in \ili{Zulu}. Applied to \ili{Zulu}, the diagnostics in \sectref{sec:ranero:5.1} and \sectref{sec:ranero:5.2} yield the opposite result to \ili{Luganda}, suggesting that left-dislocated objects in \ili{Zulu} are derived via \isi{movement}.} 



{First, the base generation analysis for \isi{left-dislocation} predicts that an \linebreak R-expression in a left-dislocated position should be able to co-refer with a \isi{pronoun} in the \isi{main clause}.}\footnote{This follows from the Copy Theory of Movement, which proposes that a moved phrase leaves behind a copy in A-bar \isi{movement} configurations (unpronounced at PF) that is relevant for interpretation at LF. If the left dislocated object were generated from inside the VP and moved to its base position in the left periphery, we would expect that the lower copy of the object R-expression would be bound by the \isi{pronoun} at LF and a Principle C violation would result.} {Given that a left-dislocated object does not move out of a} v{P internal position, no Principle C}\footnote{Principle C: An R-expression (an expression that introduces a referent) must be free; it cannot be c-commanded by a co-indexed category at LF.} {violation should be incurred throughout the derivation. This is exactly what we find. Consider the following examples: in the canonical sentence in \REF{ex:ranero:33}a, a Principle C violation occurs, resulting in an unacceptable sentence if ‘she’ is co-indexed and c-commands ‘Aisha’; contrast with (33)b, where both a free and bound reading are available if the object is left-dislocated:}


\ea\label{ex:ranero:33}
\ea\label{ex:ranero:33a}
\gll Ye  y-a-lab-a           a-ba-wala      ba        Aisha.\\
she \textsc{1sa-pst}{}-see\textsc{{}-fv} 2\textsc{aug}{}-2-daughter 2.\textsc{poss} 1.Aisha\\
\glt ‘*She\textsubscript{i} saw Aisha’s\textsubscript{i} daughters.’ (bound) / ‘She\textsubscript{i} saw Aisha’s\textsubscript{j} daughters.’ (free)

\ex\label{ex:ranero:33b}
\gll \textbf{A-ba-wala} \textbf{ba}        \textbf{Aisha},   ye   y-a-\textbf{ba}{}-lab-a.\\
\textbf{2\textsc{aug}}\textbf{{}-2-daughter} \textbf{2.}\textbf{\textsc{poss}} \textbf{1.Aisha} 3sg \textsc{1sa-pst-}\textbf{\textsc{2om}}{}-see-\textsc{fv}\\
\glt ‘She\textsubscript{i} saw Aisha’s\textsubscript{i} daughters.’(bound) / ‘She\textsubscript{i} saw Aisha’s\textsubscript{j} daughters.’ (free)
\z
\z

{In contrast, the analysis predicts that the equivalent of sentence \REF{ex:ranero:33}b in a right-dislocated context should not have two possible readings. If a right-dislocated R-expression moves out of the VP to its surface position, the lower copy should be bound by the \isi{subject pronoun} at LF and a Principle C violation would result. This is exactly what we find. Notice that in both the canonical sentence in \REF{ex:ranero:34}a and the example with a right-dislocated object in (34)b, the bound reading is impossible}:\footnote{An anonymous reviewer wonders given \REF{ex:ranero:34} why an \ili{English} example like ‘Which of Sophie’s\textsubscript{1} daughters did she\textsubscript{1} send a care package to?’ is not ungrammatical, since the \isi{subject} c-commands the lower copy of Sophie. Note that the example offered by the reviewer is not exactly parallel to the \ili{Luganda} data, since the R-expression is more deeply embedded in the \ili{English} sentence. The degree of embedding seems relevant for examples involving topicalization in \ili{English}:
\ea
*Sophie\textsubscript{1}, she\textsubscript{1} saw <Sophie\textsubscript{1}>.\\Intended: Sophie saw herself.
\z
The example above seems to involve obligatory reconstruction, resulting in the Principle C Violation; this contrasts with the acceptable example raised by the reviewer. I leave for future investigation whether there are cases in \ili{Luganda} where reconstruction is not obligatory (similar to the example offered by the reviewer), resulting in acceptable examples involving ORD that contrast with the result in \REF{ex:ranero:34}.}


\ea\label{ex:ranero:34}
\ea\label{ex:ranero:34a}
\gll Ye  y-a-vug-a             e-mmottoka   ya          Babirye   bulunji.\\
3sg \textsc{1sa-pst}{}-drive-\textsc{fv} 9a\textsc{aug}{}-9a.car 9a.\textsc{poss} 1.Babirye well\\
\glt *‘She\textsubscript{i} drove Babirye’s\textsubscript{i} car well.’ (bound) / ‘She\textsubscript{i} drove Babirye’s\textsubscript{j} car well.’ (free)
\ex\label{ex:ranero:34b}
\gll Ye  y-a-\textbf{gi}{}-vug-a          bulunji, \textbf{e-mottoka}       \textbf{ya}          \textbf{Babirye.}\\
3sg \textsc{1sa-pst}{}-\textbf{9a\textsc{om}}{}-drive-\textsc{fv} well       \textbf{9a\textsc{aug}}\textbf{{}-9a.car} \textbf{9a.}\textbf{\textsc{poss}} \textbf{1.Babirye}\\
\glt *‘She\textsubscript{i} drove Babirye’s\textsubscript{i} car well.’ (bound) / ‘She\textsubscript{i} drove Babirye’s\textsubscript{j} car well.’ (free)
\z
\z

\subsection{Binding of variables}\label{sec:ranero:5.2}

{Another prediction made by the analysis concerns the binding of variables. If we assume that bound pronouns must be bound at LF by a quantified phrase \citep[see][]{Hornstein1990}, then my analysis would predict that in left-dislocating an object, only a free reading should be possible. This follows from the observation that under a base generation analysis for left-dislocated objects, there is no copy of the object at LF that can be bound by a quantified \isi{subject}. This prediction is indeed borne out: contrast the readings available for the canonical sentence in \REF{ex:ranero:35} below with the unavailability of a bound reading in the sentence in \REF{ex:ranero:36}, where the object is left-dislocated:}


\ea\label{ex:ranero:35}
\gll Buli   mu-yiizi   y-a-buuz-a     o-mu-somesa     we.\\
every 1-student  \textsc{1sa-pst}{}-greet-\textsc{fv} \textsc{1aug-}1-teacher 1.\textsc{poss}\\
\glt ‘Every student greeted his teacher.’\\ 
\-\hspace{1cm}For every student x, x greeted x’s teacher. = \textsc{available}\\
\-\hspace{1cm}For every student x, x greeted y’s teacher. = \textsc{available}
\z

\ea\label{ex:ranero:36}
\gll \textbf{O-mu-somesa  we}         buli    mu-yiizi   y-a-\textbf{mu}{}-buuz-a.\\
\textbf{1\textsc{aug}}\textbf{{}-1-teacher 1.}\textbf{\textsc{poss}} every 1-student \textsc{1sa-pst-}\textbf{\textsc{1om}}\textbf{{}-}greet-\textsc{fv}\\
\glt ‘Every student greeted his teacher.’\\ 
\-\hspace{1cm}For every student x, x greeted x’s teacher. = \textsc{unavailable}\\
\-\hspace{1cm}For every student x, x greeted y’s teacher. = \textsc{available}
\z

In contrast, I also predict that a bound reading should be available in the context of right-dislocation, given that there is a copy in base position. This is exactly what we find, as shown by the example below:\footnote{An anonymous reviewer asks how \isi{movement} facilitates binding in ORD. I clarify that it’s not the \isi{movement} itself that facilitates binding, but the existence of the VP internal copy of the dislocated object in ORD. In contrast, such a copy does not exist in OLD.}
% TODO Should I do the strikethrough?
\ea\label{ex:ranero:37}
\gll Buli   mu-yiizi  y-a-\textbf{mu}{}-buuz-a <\textbf{o-mu-somesa}    \textbf{we}> bulunji, \textbf{o-mu-somesa}     \textbf{we}.\\
every 1-student \textsc{1sa-pst-1om}{}-greet-\textsc{fv} <\textbf{1\textsc{aug}}\textbf{{}-1-teacher} \textbf{1.\textsc{poss>}} well \textbf{1\textsc{aug}}\textbf{{}-1-teacher} \textbf{1.\textsc{poss}}\\        
\glt ‘Every student greeted his teacher well.’ \\

\-\hspace{1cm}For every student x, x greeted x’s teacher well. = AVAILABLE\\
\-\hspace{1cm}For every student x, x greeted y’s teacher well. = AVAILABLE
\z

Since right-dislocated objects are the product of \isi{movement}, the \isi{pronoun} contained in the right-dislocated phrase above can be bound by the quantifier \isi{subject} covertly at LF. Thus, we can see that further evidence for the analysis comes from the behavior of bound variables with respect to left and right object-dislocation.

\subsection{Superiority effects}\label{sec:ranero:5.3}

{A final prediction concerns superiority effects. When two phrases undergo A’-\isi{movement}, the structural hierarchy from which they are extracted affects the linear order in which they appear following \isi{movement}. If this superiority condition is an inviolable constraint, we expect that in dislocated constructions that are derived via A’-\isi{movement}, superiority effects would emerge. In contrast, if dislocated phrases are not the result of A’-\isi{movement}, but are rather base generated in their surface positions, then we predict that no superiority effects would arise. The latter case is exactly what we find in \ili{Luganda} OLD: no superiority effects arise. Consider first the canonical utterance below:}


\ea\label{ex:ranero:38}
\gll O-mu-somesa    a-kkakas-a                 nti        a-ba-yiizi           ba-a-soma     e-ki-tabo.\\
1\textsc{aug}{}-1-teacher \textsc{1sa.prs}{}-believe-\textsc{fv} \textsc{comp} \textsc{2aug-2-}student \textsc{2sa-pst}{}-read 7\textsc{aug}{}-7-book\\
\glt ‘The teacher believes that the students read the book.’
\z

In left-dislocating both the embedded \isi{subject} and object in the sentence above, a \isi{movement} approach to \isi{left-dislocation} would predict that the ordering would have to be fixed and mirror the structural relations between the arguments—that is, the dislocated \isi{subject} would have to precede and c-command the dislocated object. However, in dislocating both embedded \isi{subject} and object, we find that their ordering is free:

\ea\label{ex:ranero:39}
\gll \textbf{A-ba-yiizi}           \textbf{{\textbar}{\textbar}} \textbf{e-ki-tabo}        o-mu-somesa     a-kkakas-a            nti  ba-a-ki-som-a.\\
\textbf{2\textsc{aug}}\textbf{{}-2-student}  {}  \textbf{7}\textbf{\textsc{aug}}\textbf{{}-7-book} 1\textsc{aug}{}-1-teacher \textsc{1sa.prs}{}-believe-\textsc{fv} \textsc{comp} \textsc{2sa-pst-7om}{}-read\textsc{{}-fv}\\
\glt \textsc{‘}The teacher believes that the students read the book.’
\z

{In contrast, superiority effects arise in right-dislocation contexts. Consider first the sentence below:}

\ea\label{ex:ranero:40}
\gll A-ba-yiizi       ba-a-som-a  e-ki-tabo   luli.\\
2\textsc{aug}{}-2-student \textsc{2sa-pst}{}-read-\textsc{fv} 7\textsc{aug}{}-7-book the.other.day\\
\glt ‘The students read the book.’
\z

If both \isi{subject} and object are right-dislocated, only one ordering is permitted. In \REF{ex:ranero:41}a, observe that the dislocated-object precedes the dislocated \isi{subject}. Attempting the opposite ordering as in \REF{ex:ranero:41}b is unacceptable:

\ea\label{ex:ranero:41}
\ea\label{ex:ranero:41a}
\gll Ba-a-\textbf{ki}{}-som-a      luli,        \textbf{e-ki-tabo} a-ba-yiizi.\\
\textsc{2sa-pst-}\textbf{\textsc{7om}}{}-read-\textsc{fv} the.other.day \textbf{\textsc{7aug-7}}\textbf{{}-book} \textsc{2aug}{}-2-student\\
\glt ‘The students read the book.’

\ex\label{ex:ranero:41b}
* Ba-a-\textbf{ki}{}-som-a  luli,   abayiizi,   \textbf{ekitabo}.
\z
\z

{I take these facts to be evidence that a \isi{movement} analysis for right-dislocation is on the right track, while a base-generation analysis for \isi{left-dislocation} also makes the correct predictions.}


\section{Conclusions and future directions}\label{sec:ranero:6}

{In this paper, I have achieved the following: empirically, I have documented an asymmetry concerning left vs. right object-dislocation in \ili{Luganda}, therefore contributing to our knowledge on the language and the patterning of these phenomena cross-linguistically; from a theoretical perspective, I have shown that an approach treating these two constructions as arising from different syntactic configurations is on the right track. Several questions remain, which cannot be addressed in this short paper, though they are described in \citet{ranero2015} and are left for future investigation. First, causative ditransitives do not show the asymmetry we described for ORD — if two objects are right-dislocated in a causative construction, they are ordered freely. Second, there exists a very limited construction in which an object is right-dislocated, but no OMing is triggered. Observe the example below: since the object that is not OMed on the verb occurs to the right of a dislocated object that is OMed, then it must also be right-dislocated:}


\ea\label{ex:ranero:42}
\gll Namugga    y-a-ba{}-fumb-ir-a           luli,             a-ba-ana         \textbf{e-n-gege.}\\
1.Namugga \textsc{1sa-pst-2om}{}-cook-\textsc{appl-fv} the.other.day \textsc{2aug-}2-child \textbf{\textsc{9aug-}}\textbf{9-tilapia}\\
\glt ‘Namuga cooked the tilapia for the children the other day.’
\z

Objects that are right-dislocated but not OMed are very restricted pragmatically, being limited exclusively to given topics. Due to space considerations, I leave their derivation for future investigation. Third, my analysis makes predictions regarding island effects \citep{Boeckx2012}: right-dislocated objects should be \isi{subject} to island restrictions, while left-dislocated ones should not. However, this is not consistently the case. For instance, right-dislocating an object out of a coordinated structure is banned (as predicted), but so is left-dislocating the object, contrary to our expectations:

\ea\label{ex:ranero:43}
\gll *Aisha  y-a-fumb-a          naye ye      Aizaka  y-a-(\textbf{ki-})som-a            luli,              \textbf{e-ki-tabo.} \\
1.Aisha \textsc{1sa-pst}{}-cook-\textsc{fv} but   1.\textsc{foc} 1.Isaac  \textsc{1sa-pst}{}-\textbf{\textsc{7om}}{}-read-\textsc{fv} the.other.day \textbf{7\textsc{aug}}\textbf{{}-7-book}\\
\glt Intended: ‘Aisha cooked but Isaac read a book the other day.’
\z

\ea\label{ex:ranero:44}
\gll *\textbf{E-ki-tabo}     Aisha    y-a-fumb-a   naye ye       Aizaka y-a-\textbf{ki}{}-som-a.\\
\textbf{7\textsc{aug}}\textbf{{}-7-book} 1.Aisha \textsc{1sa-pst}{}-cook\textsc{{}-fv} but   1.\textsc{foc} 1.Isaac \textsc{1sa-pst}{}-\textbf{7\textsc{om}}{}-read-\textsc{fv}\\
\glt Intended: ‘Aisha cooked but Isaac read a book.’
\z

{While such data are puzzling, I note that there exist approaches to \linebreak \isi{left-dislocation} that take a base generation approach regardless of island restrictions, such as \citet{Cinque1990} and \citet{iatridou1995}. Given that the study of islands in \ili{Luganda} has not yet been undertaken in depth, I leave whether these data can be accommodated into our analysis for future investigation as well. Finally, it is necessary to point out avenues for future research in this area of \ili{Bantu} syntax. As \citet{Zeller2015} notes, while the syntax of object marking in the family has received extensive attention, double object-dislocation constructions specifically have been restricted to few studies (e.g. \citealt{Adams2010}, \citealt{zeller2009}, and \citealt{Zeller2015} for \ili{Zulu}). Further, the pattern reported here has not been described for other \ili{Bantu} languages, as far as I know. A first step for future investigation would involve studying double object-dislocation constructions in other \ili{Bantu} languages that also permit two OMs on the verb. \citet{Marlo2015} points out that the following languages allow for this: Bemba, Dciriku, Ha, Jita, Lungu, Lwena, Nyambo, Nyole, Ruri, Saamia, Taabwa, Tiriki, Ruwund, and \ili{Umbundu}. Replicating the \ili{Luganda} data would be a fruitful area of research, both to increase our knowledge of the typology of these constructions, and to explore whether the syntactic principles used here to account for the \ili{Luganda} patterns can be applied more broadly.}


\section*{Acknowledgements}

I thank Saudah Namyalo for her patience in providing the acceptability judgements reported here and Jenneke van der Wal for her support throughout the development of this project. Thank you as well to the audience at ACAL 47, Gesoel Mendes, and Ted Levin for helpful comments. This paper summarizes and expands upon \citet{ranero2015}; for questions, contact the author at rranero@umd.edu.

\section*{Abbreviations}

Numbers indicate \ili{Bantu} \isi{noun class}, following \citet{Hyman1990}.\medskip

\noindent
\begin{tabularx}{.45\textwidth}{lQ}
\textsc{appl} & applicative\\
\textsc{aug} & augment \\
\textsc{caus} & causative \\
\textsc{comp} & complementizer \\
\textsc{dj} &  {disjoint}\\
\textsc{foc} &  {focus}\\
\textsc{fut} & future\\
\textsc{fv} &  {final vowel}\\
\end{tabularx}
\begin{tabularx}{.45\textwidth}{lQ}
\textsc{indic} & indicative\\
\textsc{om} &  {object marker}\\
\textsc{perf} & perfective\\
\textsc{poss} & possessive\\
\textsc{prs} & present\\
\textsc{pst} & past\\
\textsc{sa} &  {subject agreement} \\
\\
\end{tabularx}

\sloppy
\printbibliography[heading=subbibliography,notkeyword=this]

\end{document}
