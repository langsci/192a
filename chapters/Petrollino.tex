\documentclass[output=paper,modfonts,nonflat,hidelinks]{langsci/langscibook} 

\author{Sara Petrollino
\affiliation{Laboratoire Dynamique Du Langage, Lyon and Leiden University}
}

\title{Between tone and stress in Hamar}

\abstract{This paper provides a preliminary description of the word-prosodic system of \linebreak Hamar, a South Omotic language spoken in South West Ethiopia. The prosodic system of Hamar shows properties of both stress accent and tone: accent is lexically contrastive in nouns, but not in verbs, where it has a grammatical function. Post-lexical tonal oppositions arise when lexical accent and grammatical accent interact in both nouns and verbs. The prosodic behaviour of Hamar nouns and verbs is in line with the pattern proposed by \cite{Smith2011}, whereby nouns are higher than verbs in a hierarchy of phonological privilege.} 

\IfFileExists{../localcommands.tex}{%hack to check whether this is being compiled as part of a collection or standalone
  \usepackage{pifont}
\usepackage{savesym}

\savesymbol{downingtriple}
\savesymbol{downingdouble}
\savesymbol{downingquad}
\savesymbol{downingquint}
\savesymbol{suph}
\savesymbol{supj}
\savesymbol{supw}
\savesymbol{sups}
\savesymbol{ts}
\savesymbol{tS}
\savesymbol{devi}
\savesymbol{devu}
\savesymbol{devy}
\savesymbol{deva}
\savesymbol{N}
\savesymbol{Z}
\savesymbol{circled}
\savesymbol{sem}
\savesymbol{row}
\savesymbol{tipa}
\savesymbol{tableauxcounter}
\savesymbol{tabhead}
\savesymbol{inp}
\savesymbol{inpno}
\savesymbol{g}
\savesymbol{hanl}
\savesymbol{hanr}
\savesymbol{kuku}
\savesymbol{ip}
\savesymbol{lipm}
\savesymbol{ripm}
\savesymbol{lipn}
\savesymbol{ripn} 
% \usepackage{amsmath} 
% \usepackage{multicol}
\usepackage{qtree} 
\usepackage{tikz-qtree,tikz-qtree-compat}
% \usepackage{tikz}
\usepackage{upgreek}


%%%%%%%%%%%%%%%%%%%%%%%%%%%%%%%%%%%%%%%%%%%%%%%%%%%%
%%%                                              %%%
%%%           Examples                           %%%
%%%                                              %%%
%%%%%%%%%%%%%%%%%%%%%%%%%%%%%%%%%%%%%%%%%%%%%%%%%%%%
% remove the percentage signs in the following lines
% if your book makes use of linguistic examples
\usepackage{tipa}  
\usepackage{pstricks,pst-xkey,pst-asr}

%for sande et al
\usepackage{pst-jtree}
\usepackage{pst-node}
%\usepackage{savesym}


% \usepackage{subcaption}
\usepackage{multirow}  
\usepackage{./langsci/styles/langsci-optional} 
\usepackage{./langsci/styles/langsci-lgr} 
\usepackage{./langsci/styles/langsci-glyphs} 
\usepackage[normalem]{ulem}
%% if you want the source line of examples to be in italics, uncomment the following line
% \def\exfont{\it}
\usetikzlibrary{arrows.meta,topaths,trees}
\usepackage[linguistics]{forest}
\forestset{
	fairly nice empty nodes/.style={
		delay={where content={}{shape=coordinate,for parent={
					for children={anchor=north}}}{}}
}}
\usepackage{soul}
\usepackage{arydshln}
% \usepackage{subfloat}
\usepackage{langsci/styles/langsci-gb4e} 
   
% \usepackage{linguex}
\usepackage{vowel}

\usepackage{pifont}% http://ctan.org/pkg/pifont
\newcommand{\cmark}{\ding{51}}%
\newcommand{\xmark}{\ding{55}}%
 
 
 %Lamont
 \makeatletter
\g@addto@macro\@floatboxreset\centering
\makeatother

\usepackage{newfloat} 
\DeclareFloatingEnvironment[fileext=tbx,name=Tableau]{tableau}
  %add all your local new commands to this file
\newcommand{\downingquad}[4]{\parbox{2.5cm}{#1}\parbox{3.5cm}{#2}\parbox{2.5cm}{#3}\parbox{3.5cm}{#4}}
\newcommand{\downingtriple}[3]{\parbox{4.5cm}{#1}\parbox{3cm}{#2}\parbox{3cm}{#3}}
\newcommand{\downingdouble}[2]{\parbox{4.5cm}{#1}\parbox{6cm}{#2}}
\newcommand{\downingquint}[5]{\parbox{1.75cm}{#1}\parbox{2.25cm}{#2}\parbox{2cm}{#3}\parbox{3cm}{#4}\parbox{2cm}{#5}}
\newcolumntype{Y}{>{\centering\arraybackslash}X}
\newcolumntype{T}{>{\centering\arraybackslash}m{2cm}}

%commands for Kusmer paper below
\newcommand{\ip}{$\upiota$}
\newcommand{\lipm}{(\_{\ip-Max}}
\newcommand{\ripm}{)\_{\ip-Max}}
\newcommand{\lipn}{(\_{\ip}}
\newcommand{\ripn}{)\_{\ip}}
\renewcommand{\_}[1]{\textsubscript{#1}}


%commands for Pillion paper below
\newcommand{\suph}{\textipa{\super h}}
\newcommand{\supj}{\textipa{\super j}}
\newcommand{\supw}{\textipa{\super w}}
\newcommand{\ts}{\textipa{\t{ts}}}
\newcommand{\tS}{\textipa{\t{tS}}}
\newcommand{\devi}{\textipa{\r*i}}
\newcommand{\devu}{\textipa{\r*u}}
\newcommand{\devy}{\textipa{\r*y}}
\newcommand{\deva}{\textipa{\r*a}}
\renewcommand{\N}{\textipa{N}}
\newcommand{\Z}{\textipa{Z}}
% 

%commands for Diercks paper below
\newcommand{\circled}[1]{\begin{tikzpicture}[baseline=(word.base)]
\node[draw, rounded corners, text height=8pt, text depth=2pt, inner sep=2pt, outer sep=0pt, use as bounding box] (word) {#1};
\end{tikzpicture}
}

%commands for Pesetsky paper below
% \newcommand{\sem}[2][]{\mbox{$[\![ $\textbf{#2}$ ]\!]^{#1}$}}
\newcommand{\sem}[2][]{\mbox{$[[ $\textbf{#2}$ ]]^{#1}$}}

% \newcommand{\ripn}{{\color{red}ripn}}%this is used but never defined. Please update the definition



%commands for Lamont paper below
\newcommand{\row}[4]{
	#1. & 
    /{#2}/ & 
    [{#3}] & 
    `#4' \\ 
}
%\newcounter{tableauxcounter}
\newcommand{\tabhead}[2]{
%     \captionsetup{labelformat=empty}
%     \stepcounter{tableauxcounter}
%     \addtocounter{table}{-1}
% 	\centering
% 	\caption{Tableau \thetableauxcounter: #1}
	\caption{#1}
	\label{#2}
}
\newcommand{\candref}[2]{{(\ref{#1}#2)}}
\newcommand{\tableauref}[1]{{Tableau~\ref{#1}}}
% tableaux
\newcommand{\inp}[1]{\multicolumn{2}{|l||}{{#1}}}
\newcommand{\inpno}[1]{\multicolumn{2}{|l||}{#1}}
\newcommand{\g}{\cellcolor{lightgray}}
\newcommand{\hanl}{\HandLeft}
\newcommand{\hanr}{\HandRight}
\newcommand{\kuku}{Kuk\'{u}}

% \newcommand{\nocaption}[1]{{\color{red} Please provide a caption}}

% \providecommand{\biberror}[1]{{\color{red}#1}}

\definecolor{RED}{cmyk}{0.05,1,0.8,0}


\newfontfamily\amharicfont[Script = Ethiopic, Scale = 1.0]{AbyssinicaSIL}
\newcommand{\amh}[1]{{\amharicfont #1}}

% 
% %Gjersoe
\usepackage{textgreek}
% 
\newcommand{\viol}{\fontfamily{MinionPro-OsF}\selectfont\rotatebox{60}{$\star$}}
\newcommand{\myscalex}{0.45}
\newcommand{\myscaley}{0.65}
%\newcommand{\red}[1]{\textcolor{red}{#1}}
%\newcommand{\blue}[1]{\textcolor{blue}{#1}}
\newcommand{\epen}[1]{\colorbox{jgray}{#1}}
\newcommand{\hand}{{\normalsize \ding{43}}}
\definecolor{jgray}{gray}{0.8} 
\usetikzlibrary{positioning}
\usetikzlibrary{matrix}
\newcommand{\mora}{\textmu\xspace}
\newcommand{\si}{\textsigma\xspace}
\newcommand{\ft}{\textPhi\xspace}
\newcommand{\tone}{\texttau\xspace}
\newcommand{\word}{\textomega\xspace}
% \newcommand{\ts}{\texttslig}
\newcommand{\fns}{\footnotesize}
\newcommand{\ns}{\normalsize}
\newcommand{\vs}{\vspace{1em}}
\newcommand{\bs}{\textbackslash}   % backslash
\newcommand{\cmd}[1]{{\bf \color{red}#1}}   % highlights command
\newcommand{\scell}[2][l]{\begin{tabular}[#1]{@{}c@{}}#2\end{tabular}}
% \interfootnotelinepenalty=10000

% --- Snider Representations --- %

\newcommand{\RepLevelHh}{
\begin{minipage}{0.10\textwidth}
\begin{tikzpicture}[xscale=\myscalex,yscale=\myscaley]
%\node (syl) at (0,0) {Hi};
\node (Rt) at (0,1) {o};
\node (H) at (-0.5,2) {H};
\node (R) at (0.5,3) {h};
%\draw [thick] (syl.north) -- (Rt.south) ;
\draw [thick] (Rt.north) -- (H.south) ;
\draw [thick] (Rt.north) -- (R.south) ;
\end{tikzpicture}
\end{minipage}
}

\newcommand{\RepLevelLh}{
\begin{minipage}{0.10\textwidth}
\begin{tikzpicture}[xscale=\myscalex,yscale=\myscaley]
%\node (syl) at (0,0) {Mid2};
\node (Rt) at (0,1) {o};
\node (H) at (-0.5,2) {L};
\node (R) at (0.5,3) {h};
%\draw [thick] (syl.north) -- (Rt.south) ;
\draw [thick] (Rt.north) -- (H.south) ;
\draw [thick] (Rt.north) -- (R.south) ;
\end{tikzpicture}
\end{minipage}
}

\newcommand{\RepLevelHl}{
\begin{minipage}{0.10\textwidth}
\begin{tikzpicture}[xscale=\myscalex,yscale=\myscaley]
%\node (syl) at (0,0) {Mid1};
\node (Rt) at (0,1) {o};
\node (H) at (-0.5,2) {H};
\node (R) at (0.5,3) {l};
%\draw [thick] (syl.north) -- (Rt.south) ;
\draw [thick] (Rt.north) -- (H.south) ;
\draw [thick] (Rt.north) -- (R.south) ;
\end{tikzpicture}
\end{minipage}
}

\newcommand{\RepLevelLl}{
\begin{minipage}{0.10\textwidth}
\begin{tikzpicture}[xscale=\myscalex,yscale=\myscaley]
%\node (syl) at (0,0) {Lo};
\node (Rt) at (0,1) {o};
\node (H) at (-0.5,2) {L};
\node (R) at (0.5,3) {l};
%\draw [thick] (syl.north) -- (Rt.south) ;
\draw [thick] (Rt.north) -- (H.south) ;
\draw [thick] (Rt.north) -- (R.south) ;
\end{tikzpicture}
\end{minipage}
}

% --- Representations --- %

\newcommand{\RepLevel}{
\begin{minipage}{0.10\textwidth}
\begin{tikzpicture}[xscale=\myscalex,yscale=\myscaley]
\node (syl) at (0,0) {\textsigma};
\node (Rt) at (0,1) {o};
\node (H) at (-0.5,2) {\texttau};
\node (R) at (0.5,3) {\textrho};
\draw [thick] (syl.north) -- (Rt.south) ;
\draw [thick] (Rt.north) -- (H.south) ;
\draw [thick] (Rt.north) -- (R.south) ;
\end{tikzpicture}
\end{minipage}
}

\newcommand{\RepContour}{
\begin{minipage}{0.10\textwidth}
\begin{tikzpicture}[xscale=\myscalex,yscale=\myscaley]
\node (syl) at (0,0) {\textsigma};
\node (Rt) at (0,1) {o};
\node (H) at (-0.5,2) {\texttau};
\node (R) at (0.5,3) {\textrho};
\node (Rt2) at (1.5,1.0) {o};
%\node (H2) at (1.0,2) {$\tau$};
%\node (R2) at (2.0,2.5) {R};
\draw [thick] (syl.north) -- (Rt.south) ;
\draw [thick] (Rt.north) -- (H.south) ;
\draw [thick] (Rt.north) -- (R.south) ;
\draw [thick] (syl.north) -- (Rt2.south) ;
%\draw [thick] (Rt2.north) -- (H2.south) ;
%\draw [thick] (Rt2.north) -- (R2.south) ;
\end{tikzpicture}
\end{minipage}
}


% --- OT constraints --- %

\newcommand{\IllustrationDown}{
\begin{minipage}{0.09\textwidth}
\begin{tikzpicture}[xscale=0.7,yscale=0.45]
\node (reg) at (0,0.75) {{\small \textalpha}};
\node (arrow) at (0,0) {{\fns $\downarrow$}};
\node (Rt) at (0,-0.75) {{\small \textbeta}};
\end{tikzpicture}
\end{minipage}
}

\newcommand{\IllustrationUp}{
\begin{minipage}{0.09\textwidth}
\begin{tikzpicture}[xscale=0.7,yscale=0.45]
\node (reg) at (0,0.75) {{\small \textalpha}};
\node (arrow) at (0,0) {{\fns $\uparrow$}};
\node (Rt) at (0,-0.75) {{\small \textbeta}};
\end{tikzpicture}
\end{minipage}
}

\newcommand{\MaxAB}{
\begin{minipage}{0.09\textwidth}
\begin{tikzpicture}[xscale=0.6,yscale=0.4]
\node (max) at (0,0) {{\small \textsc{Max}}};
\node (reg) at (0.75,0.5) {{\fns \textalpha}};
\node (arrow) at (0.75,0) {{\tiny $\downarrow$}};
\node (Rt) at (0.75,-0.5) {{\fns \textbeta}};
\end{tikzpicture}
\end{minipage}
}

\newcommand{\DepAB}{
\begin{minipage}{0.09\textwidth}
\begin{tikzpicture}[xscale=0.6,yscale=0.4]
\node (max) at (0,0) {{\small \textsc{Dep}}};
\node (reg) at (0.75,0.5) {{\fns \textalpha}};
\node (arrow) at (0.75,0) {{\tiny $\downarrow$}};
\node (Rt) at (0.75,-0.5) {{\fns \textbeta}};
\end{tikzpicture}
\end{minipage}
}

\newcommand{\DepHReg}{
\begin{minipage}{0.055\textwidth}
\begin{tikzpicture}[xscale=0.6,yscale=0.4]
\node (dep) at (0,0) {{\small \textsc{Dep}}};
\node (reg) at (0,-1.0) {{\small h}};
\end{tikzpicture}
\end{minipage}
}

\newcommand{\DepLReg}{
\begin{minipage}{0.055\textwidth}
\begin{tikzpicture}[xscale=0.6,yscale=0.4]
\node (dep) at (0,0) {{\small \textsc{Dep}}};
\node (reg) at (0,-1.0) {{\small l}};
\end{tikzpicture}
\end{minipage}
}

\newcommand{\DepReg}{
\begin{minipage}{0.055\textwidth}
\begin{tikzpicture}[xscale=0.6,yscale=0.4]
\node (dep) at (0,0) {{\small \textsc{Dep}}};
\node (reg) at (0,-1.0) {{\small \textrho}};
\end{tikzpicture}
\end{minipage}
}

\newcommand{\DepTRt}{
\begin{minipage}{0.1\textwidth}
\begin{tikzpicture}[xscale=0.6,yscale=0.4]
\node (dep) at (0,0) {{\small \textsc{Dep}}};
\node (t) at (0.75,0.5) {{\fns \texttau}};
\node (arrow) at (0.75,0) {{\tiny $\downarrow$}};
\node (Rt) at (0.75,-0.5) {{\fns o}};
\end{tikzpicture}
\end{minipage}
}

\newcommand{\MaxRegRt}{
\begin{minipage}{0.1\textwidth}
\begin{tikzpicture}[xscale=0.6,yscale=0.4]
\node (max) at (0,0) {{\small \textsc{Max}}};
\node (arrow) at (0.75,0) {{\tiny $\downarrow$}};
\node (Rt) at (0.75,-0.5) {{\fns o}};
\node (reg) at (0.75,0.5) {{\fns \textrho}};
\end{tikzpicture}
\end{minipage}
}

\newcommand{\RegToneByRt}{
\begin{minipage}{0.06\textwidth}
\begin{tikzpicture}[xscale=0.6,yscale=0.5]
\node[rotate=20] (arrow1) at (-0.15,0) {{\fns $\uparrow$}};
\node[rotate=340] (arrow2) at (0.15,0) {{\fns $\uparrow$}};
\node (Rt) at (0,-0.55) {{\small o}};
\node (reg) at (0.4,0.55) {{\small \textrho}};
\node (tone) at (-0.4,0.55) {{\small \texttau}};
\end{tikzpicture}
\end{minipage}
}

\newcommand{\RegToneBySyl}{
\begin{minipage}{0.06\textwidth}
\begin{tikzpicture}[xscale=0.6,yscale=0.5]
\node[rotate=20] (arrow1) at (-0.15,0) {{\fns $\uparrow$}};
\node[rotate=340] (arrow2) at (0.15,0) {{\fns $\uparrow$}};
\node (Rt) at (0,-0.55) {{\small \textsigma}};
\node (reg) at (0.4,0.55) {{\small \textrho}};
\node (tone) at (-0.4,0.55) {{\small \texttau}};
\end{tikzpicture}
\end{minipage}
}

\newcommand{\DepTone}{
\begin{minipage}{0.055\textwidth}
\begin{tikzpicture}[xscale=0.6,yscale=0.4]
\node (dep) at (0,0) {{\small \textsc{Dep}}};
\node (tone) at (0,-1.0) {{\small \texttau}};
\end{tikzpicture}
\end{minipage}
}

\newcommand{\DepTonalRt}{
\begin{minipage}{0.055\textwidth}
\begin{tikzpicture}[xscale=0.6,yscale=0.4]
\node (dep) at (0,0) {{\small \textsc{Dep}}};
\node (tone) at (0,-1.0) {{\small o}};
\end{tikzpicture}
\end{minipage}
}

\newcommand{\DepL}{
\begin{minipage}{0.055\textwidth}
\begin{tikzpicture}[xscale=0.6,yscale=0.4]
\node (dep) at (0,0) {{\small \textsc{Dep}}};
\node (tone) at (0,-1.0) {{\small L}};
\end{tikzpicture}
\end{minipage}
}

\newcommand{\DepH}{
\begin{minipage}{0.055\textwidth}
\begin{tikzpicture}[xscale=0.6,yscale=0.4]
\node (dep) at (0,0) {{\small \textsc{Dep}}};
\node (tone) at (0,-1.0) {{\small H}};
\end{tikzpicture}
\end{minipage}
}

\newcommand{\NoMultDiff}{{\small *loh}}
\newcommand{\Alt}{{\small \textsc{Alt}}}
\newcommand{\NoSkip}{{\small \scell{\textsc{No}\\\textsc{Skip}}}}


\newcommand{\RegDomRt}{
\begin{minipage}{0.030\textwidth}
\begin{tikzpicture}[xscale=0.6,yscale=0.5]
\node (arrow) at (0,0) {{\fns $\downarrow$}};
\node (Rt) at (0,-0.55) {{\small o}};
\node (reg) at (0,0.55) {{\small \textrho}};
\end{tikzpicture}
\end{minipage}
}

\newcommand{\DepRegRt}{
\begin{minipage}{0.1\textwidth}
\begin{tikzpicture}[xscale=0.6,yscale=0.4]
\node (dep) at (0,0) {{\small \textsc{Dep}}};
\node (arrow) at (0.75,0) {{\tiny $\downarrow$}};
\node (Rt) at (0.75,-0.5) {{\fns o}};
\node (reg) at (0.75,0.5) {{\fns \textrho}};
\end{tikzpicture}
\end{minipage}
}

% unused

\newcommand{\ToneByRt}{
\begin{minipage}{0.05\textwidth}
\begin{tikzpicture}[xscale=0.6,yscale=0.5]
\node (arrow) at (0,0) {{\fns $\uparrow$}};
\node (Rt) at (0,-0.55) {{\small o}};
\node (tone) at (0,0.55) {{\small \texttau}};
\end{tikzpicture}
\end{minipage}
}

\newcommand{\RegByRt}{
\begin{minipage}{0.05\textwidth}
\begin{tikzpicture}[xscale=0.6,yscale=0.5]
\node (arrow) at (0,0) {{\fns $\uparrow$}};
\node (Rt) at (0,-0.55) {{\small o}};
\node (reg) at (0,0.55) {{\small \textrho}};
\end{tikzpicture}
\end{minipage}
}

\newcommand{\ToneDomRt}{
\begin{minipage}{0.05\textwidth}
\begin{tikzpicture}[xscale=0.6,yscale=0.5]
\node (arrow) at (0,0) {{\fns $\downarrow$}};
\node (Rt) at (0,-0.55) {{\small o}};
\node (tone) at (0,0.55) {{\small \texttau}};
\end{tikzpicture}
\end{minipage}
}

% --- OT tableaus --- %

% Sec. 3.2, first tabl.

\newcommand{\OTHLInput}{
\begin{minipage}{0.17\textwidth}
\begin{tikzpicture}[xscale=\myscalex,yscale=\myscaley]
\node (tone) at (2,0) {(= H)};
\node (syl) at (0,0) {\textsigma};
\node (Rt) at (0,1) {o};
\node (H) at (-0.5,2) {H};
\node (R) at (0.5,3) {h};
\node (Rt2) at (1.5,1.0) {o};
%\node (H2) at (1.0,2) {\epen{L}};
\node (R2) at (2.0,3) {\blue{l}};
\draw [thick] (syl.north) -- (Rt.south) ;
\draw [thick] (Rt.north) -- (H.south) ;
\draw [thick] (Rt.north) -- (R.south) ;
\draw [thick] (syl.north) -- (Rt2.south) ;
%\draw [dashed] (Rt2.north) -- (H2.south) ;
%\draw [dashed] (Rt2.north) -- (R2.south) ;
\end{tikzpicture}
\end{minipage}
}

\newcommand{\OTHLWinner}{
\begin{minipage}{0.17\textwidth}
\begin{tikzpicture}[xscale=\myscalex,yscale=\myscaley]
\node (tone) at (2,0) {(= HL)};
\node (syl) at (0,0) {\textsigma};
\node (Rt) at (0,1) {o};
\node (H) at (-0.5,2) {H};
\node (R) at (0.5,3) {h};
\node (Rt2) at (1.5,1.0) {o};
\node (H2) at (1.0,2) {\epen{L}};
\node (R2) at (2.0,3) {\blue{l}};
\draw [thick] (syl.north) -- (Rt.south) ;
\draw [thick] (Rt.north) -- (H.south) ;
\draw [thick] (Rt.north) -- (R.south) ;
\draw [thick] (syl.north) -- (Rt2.south) ;
\draw [dashed] (Rt2.north) -- (H2.south) ;
\draw [dashed] (Rt2.north) -- (R2.south) ;
\end{tikzpicture}
\end{minipage}
}

\newcommand{\OTHLSpreadingHOnly}{
\begin{minipage}{0.17\textwidth}
\begin{tikzpicture}[xscale=\myscalex,yscale=\myscaley]
\node (tone) at (2,0) {(= HM)};
\node (syl) at (0,0) {\textsigma};
\node (Rt) at (0,1) {o};
\node (H) at (-0.5,2) {H};
\node (R) at (0.5,3) {h};
\node (Rt2) at (1.5,1.0) {o};
%\node (H2) at (1.0,2) {\epen{L}};
\node (R2) at (2.0,3) {\blue{l}};
\draw [thick] (syl.north) -- (Rt.south) ;
\draw [thick] (Rt.north) -- (H.south) ;
\draw [thick] (Rt.north) -- (R.south) ;
\draw [thick] (syl.north) -- (Rt2.south) ;
\draw [dashed] (Rt2.north) -- (R2.south) ;
\draw [dashed] (Rt2.north) -- (H.south) ;
\end{tikzpicture}
\end{minipage}
}

\newcommand{\OTHLInsertH}{
\begin{minipage}{0.17\textwidth}
\begin{tikzpicture}[xscale=\myscalex,yscale=\myscaley]
\node (tone) at (2,0) {(= HM)};
\node (syl) at (0,0) {\textsigma};
\node (Rt) at (0,1) {o};
\node (H) at (-0.5,2) {H};
\node (R) at (0.5,3) {h};
\node (Rt2) at (1.5,1.0) {o};
\node (H2) at (1.0,2) {\epen{H}};
\node (R2) at (2.0,3) {\blue{l}};
\draw [thick] (syl.north) -- (Rt.south) ;
\draw [thick] (Rt.north) -- (H.south) ;
\draw [thick] (Rt.north) -- (R.south) ;
\draw [thick] (syl.north) -- (Rt2.south) ;
\draw [dashed] (Rt2.north) -- (H2.south) ;
\draw [dashed] (Rt2.north) -- (R2.south) ;
\end{tikzpicture}
\end{minipage}
}

\newcommand{\OTHLOverwriting}{
\begin{minipage}{0.17\textwidth}
\begin{tikzpicture}[xscale=\myscalex,yscale=\myscaley]
\node (syl) at (0,0) {\textsigma};
\node (Rt) at (0,1) {o};
\node (H) at (-0.5,2) {H};
\node (R) at (0.5,3) {h};
\node (Rt2) at (1.5,1.0) {o};
%\node (H2) at (1.0,2) {\epen{L}};
\node (R2) at (2.0,3) {\blue{l}};
\draw [thick] (syl.north) -- (Rt.south) ;
\draw [thick] (Rt.north) -- (H.south) ;
\draw [thick] (Rt.north) -- (R.south) ;
\draw [thick] (syl.north) -- (Rt2.south) ;
%\draw [dashed] (Rt2.north) -- (H2.south) ;
\draw [dashed] (Rt.north) -- (R2.south) ;
\node (del) at (0.3,1.9) {\textbf{=}};
\end{tikzpicture}
\end{minipage}
}

\newcommand{\OTHLSpreading}{
\begin{minipage}{0.17\textwidth}
\begin{tikzpicture}[xscale=\myscalex,yscale=\myscaley]
\node (syl) at (0,0) {\textsigma};
\node (Rt) at (0,1) {o};
\node (H) at (-0.5,2) {H};
\node (R) at (0.5,3) {h};
\node (Rt2) at (1.5,1.0) {o};
%\node (H2) at (1.0,2) {\epen{L}};
\node (R2) at (2.0,3) {\blue{l}};
\draw [thick] (syl.north) -- (Rt.south) ;
\draw [thick] (Rt.north) -- (H.south) ;
\draw [thick] (Rt.north) -- (R.south) ;
\draw [thick] (syl.north) -- (Rt2.south) ;
%\draw [dashed] (Rt2.north) -- (H2.south) ;
\draw [dashed] (Rt2.north) -- (H.south) ;
\draw [dashed] (Rt2.north) -- (R.south) ;
\end{tikzpicture}
\end{minipage}
}

% Sec. 4.2, second tabl.: phrase-medial position

\newcommand{\OTHnoLInput}{
\begin{minipage}{0.17\textwidth}
\begin{tikzpicture}[xscale=\myscalex,yscale=\myscaley]
\node (tone) at (2,0) {(= H)};
\node (syl) at (0,0) {\textsigma};
\node (Rt) at (0,1) {o};
\node (H) at (-0.5,2) {H};
\node (R) at (0.5,3) {h};
\node (Rt2) at (1.5,1.0) {o};
%\node (H2) at (1.0,2) {\epen{L}};
%\node (R2) at (2.0,3) {\blue{l}};
\draw [thick] (syl.north) -- (Rt.south) ;
\draw [thick] (Rt.north) -- (H.south) ;
\draw [thick] (Rt.north) -- (R.south) ;
\draw [thick] (syl.north) -- (Rt2.south) ;
\end{tikzpicture}
\end{minipage}
}

\newcommand{\OTHnoLEpenth}{
\begin{minipage}{0.17\textwidth}
\begin{tikzpicture}[xscale=\myscalex,yscale=\myscaley]
\node (tone) at (2,0) {(= HM)};
\node (syl) at (0,0) {\textsigma};
\node (Rt) at (0,1) {o};
\node (H) at (-0.5,2) {H};
\node (R) at (0.5,3) {h};
\node (Rt2) at (1.5,1.0) {o};
\node (H2) at (1.0,2) {\epen{L}};
\node (R2) at (2.0,3) {\epen{h}};
\draw [thick] (syl.north) -- (Rt.south) ;
\draw [thick] (Rt.north) -- (H.south) ;
\draw [thick] (Rt.north) -- (R.south) ;
\draw [thick] (syl.north) -- (Rt2.south) ;
\draw [dashed] (Rt2.north) -- (H2.south) ;
\draw [dashed] (Rt2.north) -- (R2.south) ;
\end{tikzpicture}
\end{minipage}
}

\newcommand{\OTHnoLSpreading}{
\begin{minipage}{0.17\textwidth}
\begin{tikzpicture}[xscale=\myscalex,yscale=\myscaley]
\node (tone) at (2,0) {(= HH)};
\node (syl) at (0,0) {\textsigma};
\node (Rt) at (0,1) {o};
\node (H) at (-0.5,2) {H};
\node (R) at (0.5,3) {h};
\node (Rt2) at (1.5,1.0) {o};
%\node (H2) at (1.0,2) {\epen{L}};
%\node (R2) at (2.0,3) {\blue{l}};
\draw [thick] (syl.north) -- (Rt.south) ;
\draw [thick] (Rt.north) -- (H.south) ;
\draw [thick] (Rt.north) -- (R.south) ;
\draw [thick] (syl.north) -- (Rt2.south) ;
\draw [dashed] (Rt2.north) -- (H.south) ;
\draw [dashed] (Rt2.north) -- (R.south) ;
\end{tikzpicture}
\end{minipage}
}

% Sec. 4.2, third tabl., LM is unaffected by L\%

\newcommand{\OTLMInput}{
\begin{minipage}{0.2\textwidth}
\begin{tikzpicture}[xscale=\myscalex,yscale=\myscaley]
\node (tone) at (2,0) {(= LM)};
\node (syl) at (0,0) {\textsigma};
\node (Rt) at (0,1) {o};
\node (H) at (-0.5,2) {L};
\node (R) at (0.5,3) {l};
\node (Rt2) at (1.5,1.0) {o};
\node (H2) at (1.0,2) {L};
\node (R2) at (2.0,3) {h};
\node (R3) at (3.0,3) {\blue{l}};
\draw [thick] (syl.north) -- (Rt.south) ;
\draw [thick] (Rt.north) -- (H.south) ;
\draw [thick] (Rt.north) -- (R.south) ;
\draw [thick] (syl.north) -- (Rt2.south) ;
\draw [thick] (Rt2.north) -- (H2.south) ;
\draw [thick] (Rt2.north) -- (R2.south) ;
\end{tikzpicture}
\end{minipage}
}

\newcommand{\OTLMReplace}{
\begin{minipage}{0.2\textwidth}
\begin{tikzpicture}[xscale=\myscalex,yscale=\myscaley]
\node (tone) at (2,0) {(= LL)};
\node (syl) at (0,0) {\textsigma};
\node (Rt) at (0,1) {o};
\node (H) at (-0.5,2) {L};
\node (R) at (0.5,3) {l};
\node (Rt2) at (1.5,1.0) {o};
\node (H2) at (1.0,2) {L};
\node (R2) at (2.0,3) {h};
\node (R3) at (3.0,3) {\blue{l}};
\draw [thick] (syl.north) -- (Rt.south) ;
\draw [thick] (Rt.north) -- (H.south) ;
\draw [thick] (Rt.north) -- (R.south) ;
\draw [thick] (syl.north) -- (Rt2.south) ;
\draw [thick] (Rt2.north) -- (H2.south) ;
\draw [thick] (Rt2.north) -- (R2.south) ;
\draw [dashed] (Rt2.north) -- (R3.south) ;
\node (del) at (1.8,2.1) {\textbf{=}};
\end{tikzpicture}
\end{minipage}
}

\newcommand{\OTLMTwoReg}{
\begin{minipage}{0.2\textwidth}
\begin{tikzpicture}[xscale=\myscalex,yscale=\myscaley]
\node (tone) at (2,0) {(= LML)};
\node (syl) at (0,0) {\textsigma};
\node (Rt) at (0,1) {o};
\node (H) at (-0.5,2) {L};
\node (R) at (0.5,3) {l};
\node (Rt2) at (1.5,1.0) {o};
\node (H2) at (1.0,2) {L};
\node (R2) at (2.0,3) {h};
\node (R3) at (3.0,3) {\blue{l}};
\draw [thick] (syl.north) -- (Rt.south) ;
\draw [thick] (Rt.north) -- (H.south) ;
\draw [thick] (Rt.north) -- (R.south) ;
\draw [thick] (syl.north) -- (Rt2.south) ;
\draw [thick] (Rt2.north) -- (H2.south) ;
\draw [thick] (Rt2.north) -- (R2.south) ;
\draw [dashed] (Rt2.north) -- (R3.south) ;
\end{tikzpicture}
\end{minipage}
}

% Sec. 4.2, fourth tabl., L is affected by L\% but M is not

\newcommand{\OTLInput}{
\begin{minipage}{0.17\textwidth}
\begin{tikzpicture}[xscale=\myscalex,yscale=\myscaley]
\node (tone) at (2,0) {(= L)};
\node (syl) at (0,0) {\textsigma};
\node (Rt) at (0,1) {o};
\node (H) at (-0.5,2) {L};
\node (R) at (0.5,3) {l};
\node (R2) at (2,3) {\blue{l}};
\draw [thick] (syl.north) -- (Rt.south) ;
\draw [thick] (Rt.north) -- (H.south) ;
\draw [thick] (Rt.north) -- (R.south) ;
\end{tikzpicture}
\end{minipage}
}

\newcommand{\OTLLowered}{
\begin{minipage}{0.17\textwidth}
\begin{tikzpicture}[xscale=\myscalex,yscale=\myscaley]
\node (tone) at (2,0) {(= LL)};
\node (syl) at (0,0) {\textsigma};
\node (Rt) at (0,1) {o};
\node (H) at (-0.5,2) {L};
\node (R) at (0.5,3) {l};
\node (R2) at (2,3) {\blue{l}};
\draw [thick] (syl.north) -- (Rt.south) ;
\draw [thick] (Rt.north) -- (H.south) ;
\draw [thick] (Rt.north) -- (R.south) ;
\draw [dashed] (Rt.north) -- (R2.south) ;
\end{tikzpicture}
\end{minipage}
}

\newcommand{\OTMInput}{
\begin{minipage}{0.17\textwidth}
\begin{tikzpicture}[xscale=\myscalex,yscale=\myscaley]
\node (tone) at (2,0) {(= M)};
\node (syl) at (0,0) {\textsigma};
\node (Rt) at (0,1) {o};
\node (H) at (-0.5,2) {L};
\node (R) at (0.5,3) {h};
\node (R2) at (2,3) {\blue{l}};
\draw [thick] (syl.north) -- (Rt.south) ;
\draw [thick] (Rt.north) -- (H.south) ;
\draw [thick] (Rt.north) -- (R.south) ;
\end{tikzpicture}
\end{minipage}
}

\newcommand{\OTMLowered}{
\begin{minipage}{0.17\textwidth}
\begin{tikzpicture}[xscale=\myscalex,yscale=\myscaley]
\node (tone) at (2,0) {(= ML)};
\node (syl) at (0,0) {\textsigma};
\node (Rt) at (0,1) {o};
\node (H) at (-0.5,2) {L};
\node (R) at (0.5,3) {h};
\node (R2) at (2,3) {\blue{l}};
\draw [thick] (syl.north) -- (Rt.south) ;
\draw [thick] (Rt.north) -- (H.south) ;
\draw [thick] (Rt.north) -- (R.south) ;
\draw [dashed] (Rt.north) -- (R2.south) ;
\end{tikzpicture}
\end{minipage}
}

% Sec. 4.2, fifth tableau, polar questions with level tones

\newcommand{\OTLPolIn}{
\begin{minipage}{0.20\textwidth}
\begin{tikzpicture}[xscale=\myscalex-0.05,yscale=\myscaley-0.05]
\node (tone) at (3.5,0) {(= L)};
\node (syl) at (0,0) {\textsigma};
\node (syl2) at (2,0) {\red{\textsigma}};
\node (Rt) at (0,1) {o};
\node (H) at (-0.5,2) {L};
\node (R) at (0.5,3) {l};
\node (Rt2) at (2,1) {\red{o}};
\draw [thick] (syl.north) -- (Rt.south) ;
\draw [thick,red] (syl2.north) -- (Rt2.south) ;
\draw [thick] (Rt.north) -- (H.south) ;
\draw [thick] (Rt.north) -- (R.south) ;
\end{tikzpicture}
\end{minipage}
}

\newcommand{\OTLPolDef}{
\begin{minipage}{0.20\textwidth}
\begin{tikzpicture}[xscale=\myscalex-0.05,yscale=\myscaley-0.05]
\node (tone) at (3.5,0) {(= L.M)};
\node (syl) at (0,0) {\textsigma};
\node (syl2) at (2,0) {\red{\textsigma}};
\node (Rt) at (0,1) {o};
\node (H) at (-0.5,2) {L};
\node (R) at (0.5,3) {l};
\node (H2) at (1.5,2) {\epen{L}};
\node (R2) at (2.5,3) {\epen{h}};
\node (Rt2) at (2,1) {\red{o}};
\draw [thick] (syl.north) -- (Rt.south) ;
\draw [thick,red] (syl2.north) -- (Rt2.south) ;
\draw [thick] (Rt.north) -- (H.south) ;
\draw [thick] (Rt.north) -- (R.south) ;
\draw [semithick,dashed] (Rt2.north) -- (H2.south) ;
\draw [semithick,dashed] (Rt2.north) -- (R2.south) ;
\end{tikzpicture}
\end{minipage}
}

\newcommand{\OTLPolAlt}{
\begin{minipage}{0.20\textwidth}
\begin{tikzpicture}[xscale=\myscalex-0.05,yscale=\myscaley-0.05]
\node (tone) at (3.5,0) {(= L.L)};
\node (syl) at (0,0) {\textsigma};
\node (syl2) at (2,0) {\red{\textsigma}};
\node (Rt) at (0,1) {o};
\node (H) at (-0.5,2) {L};
\node (R) at (0.5,3) {l};
\node (Rt2) at (2,1) {\red{o}};
\draw [thick] (syl.north) -- (Rt.south) ;
\draw [thick,red] (syl2.north) -- (Rt2.south) ;
\draw [thick] (Rt.north) -- (H.south) ;
\draw [thick] (Rt.north) -- (R.south) ;
\draw [semithick,dashed] (Rt2.north) -- (H.south) ;
\draw [semithick,dashed] (Rt2.north) -- (R.south) ;
\end{tikzpicture}
\end{minipage}
}

% Sec. 4.2, sixth tableau, polar questions with contour tones

\newcommand{\OTLLPolIn}{
\begin{minipage}{0.23\textwidth}
\begin{tikzpicture}[xscale=\myscalex-0.05,yscale=\myscaley-0.05]
\node (tone) at (5.2,0) {(= L)};
\node (syl) at (0,0) {\textsigma};
\node (syl3) at (3.4,0) {\red{\textsigma}};
\node (Rt) at (0,1) {o};
\node (Rt2) at (1.7,1) {o};
\node (Rt3) at (3.4,1) {\red{o}};
\node (H) at (-0.5,2) {L};
\node (R) at (0.5,3) {l};
\draw [thick] (syl.north) -- (Rt.south) ;
\draw [thick] (syl.north) -- (Rt2.south) ;
\draw [thick,red] (syl3.north) -- (Rt3.south) ;
\draw [thick] (Rt.north) -- (H.south) ;
\draw [thick] (Rt.north) -- (R.south) ;
\end{tikzpicture}
\end{minipage}
}

\newcommand{\OTLLPolDef}{
\begin{minipage}{0.23\textwidth}
\begin{tikzpicture}[xscale=\myscalex-0.05,yscale=\myscaley-0.05]
\node (tone) at (5.2,0) {(= L.M)};
\node (syl) at (0,0) {\textsigma};
\node (syl3) at (3.4,0) {\red{\textsigma}};
\node (Rt) at (0,1) {o};
\node (Rt2) at (1.7,1) {o};
\node (Rt3) at (3.4,1) {\red{o}};
\node (H) at (-0.5,2) {L};
\node (R) at (0.5,3) {l};
\node (H3) at (2.9,2) {\epen{L}};
\node (R3) at (3.9,3) {\epen{h}};
\draw [thick] (syl.north) -- (Rt.south) ;
\draw [thick] (syl.north) -- (Rt2.south) ;
\draw [thick,red] (syl3.north) -- (Rt3.south) ;
\draw [thick] (Rt.north) -- (H.south) ;
\draw [thick] (Rt.north) -- (R.south) ;
\draw [dashed] (Rt3.north) -- (H3.south) ;
\draw [dashed] (Rt3.north) -- (R3.south) ;
\end{tikzpicture}
\end{minipage}
}

\newcommand{\OTLLPolSkip}{
\begin{minipage}{0.23\textwidth}
\begin{tikzpicture}[xscale=\myscalex-0.05,yscale=\myscaley-0.05]
\node (tone) at (5.2,0) {(= L.L)};
\node (syl) at (0,0) {\textsigma};
\node (syl3) at (3.4,0) {\red{\textsigma}};
\node (Rt) at (0,1) {o};
\node (Rt2) at (1.7,1) {o};
\node (Rt3) at (3.4,1) {\red{o}};
\node (H) at (-0.5,2) {L};
\node (R) at (0.5,3) {l};
\draw [thick] (syl.north) -- (Rt.south) ;
\draw [thick] (syl.north) -- (Rt2.south) ;
\draw [thick,red] (syl3.north) -- (Rt3.south) ;
\draw [thick] (Rt.north) -- (H.south) ;
\draw [thick] (Rt.north) -- (R.south) ;
\draw [dashed] (Rt3.north) -- (H.south) ;
\draw [dashed] (Rt3.north) -- (R.south) ;
\end{tikzpicture}
\end{minipage}
}  
  
\newcommand{\ilit}[1]{#1\il{#1}}    
\newcommand{\isit}[1]{#1\is{#1}}  

\makeatletter
\let\thetitle\@title
\let\theauthor\@author 
\makeatother

\newcommand{\togglepaper}[1][0]{ 
  \bibliography{../localbibliography}
  %% hyphenation points for line breaks
%% Normally, automatic hyphenation in LaTeX is very good
%% If a word is mis-hyphenated, add it to this file
%%
%% add information to TeX file before \begin{document} with:
%% %% hyphenation points for line breaks
%% Normally, automatic hyphenation in LaTeX is very good
%% If a word is mis-hyphenated, add it to this file
%%
%% add information to TeX file before \begin{document} with:
%% \include{localhyphenation}
\hyphenation{
affri-ca-te
affri-ca-tes
com-ple-ments
par-a-digm
Sha-ron
Kings-ton
phe-nom-e-non
Daul-ton
Abu-ba-ka-ri
Ngo-nya-ni
Clem-ents 
King-ston
Tru-cken-brodt
Tab-leau
cophono-logies
mark-edness
Ti-gri-nya
a-mong
Car-stens
Lu-bu-ku-su
}
\hyphenation{
affri-ca-te
affri-ca-tes
com-ple-ments
par-a-digm
Sha-ron
Kings-ton
phe-nom-e-non
Daul-ton
Abu-ba-ka-ri
Ngo-nya-ni
Clem-ents 
King-ston
Tru-cken-brodt
Tab-leau
cophono-logies
mark-edness
Ti-gri-nya
a-mong
Car-stens
Lu-bu-ku-su
}
  \papernote{\scriptsize\normalfont
    \theauthor.
    \thetitle. 
    To appear in: 
    Emily Clem,   Peter Jenks \& Hannah Sande.
    Theory and description in African Linguistics: Selected papers from the 47th Annual Conference on African Linguistics.
    Berlin: Language Science Press. [preliminary page numbering]
  }
  \pagenumbering{roman}
  \setcounter{chapter}{#1}
  \addtocounter{chapter}{-1}
}

\newcommand{\upstep}{\textupstep}


% \newcounter{tableauxcounter}

\renewcommand{\textltailn}{ɲ}
\renewcommand{\textbardotlessj}{ɟ}

\newcommand{\emphkh}[1]{\textit{#1}} %originally \textbf, banned by the guidelines



\definecolor{lsDOIGray}{cmyk}{0,0,0,0.45}


\newcommand{\xuparrow}[1]{%
  {\left\uparrow\vbox to #1{}\right.\kern-\nulldelimiterspace}
}
\renewcommand \textupstep[1]{\char"A71B#1}
\renewcommand \textdownstep[1]{\char"A71C#1}
 
 \newcommand{\ꜛ}{\textsf{ꜛ}}
 
\def\biberror{\undefined}


\newcommand{\OTbox}[1]{\resizebox{.88\textwidth}{!}{#1}}
 
  \togglepaper[17]
}{}


\begin{document}
\maketitle 

\section{Introduction} \label{sec:Petrollino:1}

\ili{Hamar} is spoken in South-West Ethiopia by approximately 47,500 people \citep{SimonsFenning2017} and it is commonly classified within the South \ili{Omotic} branch of the \ili{Omotic} family. The internal and external classification of \ili{Omotic} is still unsettled and the affiliation of South \ili{Omotic} languages to either the \ili{Afro-Asiatic} or the \ili{Nilo-Saharan} phylum is debated, see \citet{Zaborski2004}, \citet{Blažek2008}, \citet{Bender2000, Bender2003}, \citet{Hayward2003}, \citet{Fleming1974}, and \citet{Azeb2012}. The \ili{Hamar} live in the lower Omo valley, in the Ethiopian administrative zone referred to as Southern Nations, Nationalities, and People's region (SNNPR). The neighbours of the \ili{Hamar} are the \ili{Aari} people to the north (\ili{Aari} is a South \ili{Omotic} language), the \ili{Arbore} (Lowland East \ili{Cushitic}) to the east, the \ili{Dhaasanac} (Lowland East \ili{Cushitic}) to the south, the \ili{Nyangatom} (Eastern \ili{Sudanic}, \ili{Nilotic}) and the \ili{Kara} (South \ili{Omotic}) to the west. \ili{Hamar}, together with \ili{Banna} and \ili{Bashaɗɗa}, forms a linguistic unit which is usually referred to as the \ili{Hamar}-\ili{Banna} cluster. The three languages are mutually intelligible and show only minor variations in the lexicon and in the phonology. This paper presents a preliminary description of the word-level prosodic system of the \ili{Hamar} variety, and it is based on the analysis of circa 200 \ili{Hamar} words uttered in isolation and in context. These have been extracted from a larger corpus of first-hand data collected in \ili{Hamar} territories between 2013 and 2014 for the compilation of the \ili{Hamar} grammar, see \citet{Petrollino2016}.\footnote{For the phonological analysis, speakers were asked to repeat three tokens of each word in isolation and in carrier phrases. Some of the speakers were used to utter words in sequence as if they were individual, separate utterances, and words in isolation were always compared to words uttered in carrier phrases in order to exclude list intonation.} An overview of the main phonological features of \ili{Hamar} is given in \sectref{sec:Petrollino:2}; the word-prosodic system is illustrated in \sectref{sec:Petrollino:3}, followed by concluding remarks in \sectref{sec:Petrollino:4}.  

\section{Phonological preliminaries} \label{sec:Petrollino:2}

\ili{Hamar} displays phonological features which are typical of the “Ethiopian Linguistic Area”, such as the implosive /ɗ/, the ejective consonants and the replacement of /p/ with /f/ (or vice versa) \citep{Ferguson1970,Ferguson1976,CrassMeyer2008}. Various assimilatory processes attested in neighbouring \ili{Omotic} and \ili{Cushitic} languages, such as translaryngeal harmony and sibilant harmony \citep{Hayward1988} occur also in \ili{Hamar}. Sibilant harmony in \ili{Hamar} is a root-structure condition but it extends also across morpheme boundaries; the sibilant consonants in a word do not need to be identical but must agree in place of articulation. The word-prosodic system of \ili{Hamar} is not uncommon among \ili{Omotic} and \ili{Cushitic} languages, even though these language families show great variation in terms of prosodic systems (see \citealt{Mous2012} and \citealt{Azeb2012} for a \ili{Cushitic} and \ili{Omotic} overview). According to \citet[438]{Azeb2012} the languages located in the southern and eastern parts of the \ili{Omotic} area are characterised by “pitch-accent” systems, while highly tonal systems are usually found in the northern and western parts (Bench, for instance, is an \ili{Omotic} language with five level tones and a rising \isi{tone}, see \citealt{Rapold2006}).

This section offers an overview of the phonemic inventories, including vowel realization \sectref{sec:Petrollino:Phonemicinventories}, and the \isi{syllable structure} \sectref{sec:Petrollino:Syllablestructure} of \ili{Hamar}. \ili{Hamar} examples are written in a surface-phonemic transcription. The following modifications to the International Phonetic Alphabet have been adopted: /j/ for the palato-alveolar affricate [ʤ]; /c/ for the voiceless palato-alveolar [ʧ]; /cʼ/ for the palato-alveolar ejective affricate [tʃʼ]; /y/ for the glide [j]; /h/ for the breathy-voiced glottal approximant [ɦ]; /sh/ for the palato-alveolar [ʃ]. Long vowels and geminated consonants are indicated by doubling the vowel or the consonant symbol. Word initial glottal stop is not written in surface-phonemic transcription. An asterisk * is used for ungrammatical forms and unattested stages, whereas the diacritics v́ and v̂ indicate high and falling pitch, respectively. The absence of a diacritic on vowels indicate accent-less vowels and syllables, which are usually realized with a low pitch. On \isi{consecutive} (long) vowels, however, the high pitch is written only on the first vowel, i.e. /v́v/ is realized as [v́v́] and not as [v́v̀].   

\subsection{Phonemic inventories} \label{sec:Petrollino:Phonemicinventories}

The \isi{phonemic inventory} of \ili{Hamar} has 26 consonant phonemes (\tabref{tab:Petrollino:1}), seven vowel qualities (\tabref{tab:Petrollino:2}) and five diphthongs (/ai/, /au/, /ei/, /oi/, /ia/). 
The voiceless bilabial, alveolar and velar stops are aspirated in word initial position, but aspiration is not phonemic. The velar implosive /ɠ/ is marginal as it occurs only in the lexeme \textit{ɠiá} ʻhitʼ where it contrasts with the velar stop /g/ in the lexeme \textit{giá} ʻtellʼ. Ejective consonants cannot be geminated. The glides /w/, /y/, /ʔ/, /h/ form a natural class in that they undergo the same morpho-phonological rule and get deleted in specific contexts.
Consonant gemination is distinctive \REF{ex:Petrollino:gemination} and it can arise grammatically \REF{ex:Petrollino:causative}:

\begin{exe}
	\ex \label{ex:Petrollino:gemination} \begin{xlist}
	\ex \textit{kumá}\hspace{10mm} ʻdrink milkʼ
    \ex \textit{kummá}\hspace{8mm}ʻeatʼ \label{ex:Petrollino:kumá}
   	\end{xlist}
\end{exe}

\begin{exe}
\ex  \label{ex:Petrollino:causative}\begin{xlist}
\ex \textit{raatá}\hspace{12mm}ʻsleepʼ
\ex \textit{rattá}\hspace{13mm}ʻmake sb. sleepʼ (causative derived form)\footnote{The \isi{vowel shortening} in \textit{rattá} occurs to avoid CVVC.CV word structure, see section \sectref{sec:Petrollino:Syllablestructure}} \label{ex:Petrollino:rattá}
\ex \textit{afála}\hspace{13mm}ʻblanketʼ
\ex \textit{afálla}\hspace{12mm}ʻblanketsʼ (blanket:\textsc{pl})
\end{xlist}
\end{exe}

\begin{table}
\caption{Consonant phonemes}
\label{tab:Petrollino:1}
 \begin{tabular}{lcccccc}
  \lsptoprule
            & Bilabial & Alveolar & Palato-alveolar & Velar & Uvular & Glottal\\ %table header
  \midrule
Stops		& p\footnote{The bilabial stop /p/ can be realized as [p] or [ɸ] (a common feature found in the languages of Ethiopia): a word like /payá/ ʻgoodʼ can be realized as [payá] or [ɸayá], thus both p and f will be used in surface-phonemic transcriptions.}  b	& t d & c j & k g & q &  \\
Implosives	& ɓ 	&  ɗ  &  	& 	(ɠ) &   &  \\
Ejectives	&   	&  tʼ &  cʼ	&  	  &   &  \\
Fricatives	&    	& s z &  sh &  x  &   &  \\
Nasals		& m   	& n   & ɲ   & 	  &   &  \\
Liquids		&    	& l, r 	  &     &     &   &  \\
Glides		&    w	&  	  &   y  &     &   & h,ʔ \\
  \lspbottomrule
 \end{tabular}
\end{table}

\begin{table}
\caption{Vowel phonemes}
\label{tab:Petrollino:2}
 \begin{tabular}{lccc} 
  \lsptoprule
            & Front & Central & Back\\ 
  \midrule
high		& i\quad ii	&  & u\quad uu \\
mid-high	& e\quad ee	&  & o\quad oo  \\
mid-low		& ɛ\quad ɛɛ  & &  ɔ\quad ɔɔ \\
low			&    	&\quad a &  \\
  \lspbottomrule
 \end{tabular}
\end{table}

Vowel quantity is also distinctive as illustrated in \REF{ex:Petrollino:vowelgemination}. Vowel length is further discussed in \sectref{sec:Petrollino:stress}.  

\begin{exe}
	\ex \label{ex:Petrollino:vowelgemination} \begin{xlist}
    \ex \textit{ɛ́na}\hspace{15mm} ʻpastʼ
    \ex \textit{ɛ́ɛna}\hspace{14mm} ʻpeopleʼ
    \ex \textit{gobá}\hspace{14mm} ʻrunʼ \label{ex:Petrollino:run}
    \ex \textit{goobá}\hspace{12mm} ʻdecorateʼ \label{ex:Petrollino:decorate}
   	\end{xlist}
\end{exe}

Vowel realization can be affected by accent. Word-final unaccented vowels can be devoiced or partially devoiced depending on the rate of speech and on whether they occur in utterance-final position:

\begin{exe}
	\ex \textit{róqo}\hspace{21mm}ʻtamardind treeʼ\hspace{6mm} [róqo] or [róqo̥] 
\end{exe}
Word-final accented vowels can be phonetically breathy:

\begin{exe}
	\ex \textit{meté}\hspace{21mm}ʻheadʼ\hspace{21mm} [meté] or [meté\super h]
\end{exe}
The mid-low vowels are phonemic as illustrated in the minimal pair below:

\begin{exe}
 \ex \begin{xlist}
	\ex \textit{ɛdá}\hspace{13mm} ʻluckʼ 	\label{ex:Petrollino:ɛdá1}	  		  
    \ex \textit{edá}\hspace{13mm} ʻseparate' \label{ex:Petrollino:edá}
\end{xlist}
\end{exe}
Mid-low vowels, however, can also be in \isi{complementary distribution} with the mid-high vowels /e/ and /o/: except for some idiosyncratic exceptions illustrated in \REF{ex:Petrollino:coobar} and \REF{ex:Petrollino:ɛda}, accented mid vowels followed by the low vowel /a/ are usually realized as mid-low, see \REF{ex:Petrollino:bone marrow} and \REF{ex:Petrollino:roof}; unaccented mid vowels are not affected by the following low vowel /a/ and they are realized as mid-high, see \REF{ex:Petrollino:show} and \REF{ex:Petrollino:hold} below:

\begin{exe}
	\ex \begin{xlist}
	\ex \textit{ɗɔ́ya}\hspace{10mm} ʻbone marrowʼ \label{ex:Petrollino:bone marrow}
    \ex \textit{ɗoyá}\hspace{10mm} ʻshowʼ \label{ex:Petrollino:show}
    \end{xlist}
\end{exe}

\begin{exe}
\ex \begin{xlist}
    \ex \textit{yɛ́ɛla}\hspace{10mm} ʻroofʼ \label{ex:Petrollino:roof}
    \ex \textit{yedá}\hspace{11mm} ʻholdʼ \label{ex:Petrollino:hold}
\end{xlist}
\end{exe}
The relationship between mid vowels and accent cannot always be used as a cue to determine the location of stress in a given word since there are several exceptions to the pattern illustrated in the examples above. First of all, the realization of mid vowels can vary across speakers and within the same speaker's speech: in \REF{ex:Petrollino:keda} and \REF{ex:Petrollino:oshala} below, for instance, there is \isi{free variation} and none of the two realizations is preferred over the other. 

A few words (less then ten items) have an idiosyncratic pronunciation and allow accented mid-high vowels followed by the low vowel /a/ \REF{ex:Petrollino:coobar}, or vice versa, unaccented mid-low vowels \REF{ex:Petrollino:ɛda}:

\begin{exe}
	\ex \begin{xlist}
	\ex \textit{kéda}\hspace{8mm}ʻthenʼ\hspace{19mm} [kéda], [kɛ́da] \label{ex:Petrollino:keda}
    \ex \textit{oshála}\hspace{5mm} ʻafter two daysʼ\hspace{3mm} [ʔoʃála], [ʔɔʃála] \label{ex:Petrollino:oshala}
    \end{xlist}
\end{exe}

\begin{exe}
	\ex \begin{xlist} \label{ex:Petrollino:coobar} 
    \ex \textit{cóobar}\hspace{5mm} ʻdown thereʼ\hspace{7mm} [tʃó:bar] 
    \ex \textit{zéega}\hspace{7mm} ʻbird of prey sp.ʼ \hspace{1mm}[zé:ga] 
    \end{xlist}
\end{exe}
    
\begin{exe}
    \ex \textit{ɛdá}\hspace{17mm} ʻluckʼ\hspace{18mm} [ʔɛdá] \label{ex:Petrollino:ɛda}
\end{exe}

\largerpage
Mid-low vowels have a high \isi{functional load} since they arise grammatically. The realization of \isi{masculine} \isi{gender}, for instance, can be signalled by the presence of mid-low vowels:

 \begin{exe}
 	\ex \begin{xlist}
 	\ex \textit{segeré}\hspace{7mm} ʻdik-dik' (non inflected form)\footnote{\ili{Hamar} nouns can be marked for \isi{gender} depending on the syntactic context and on the semantic functions. This means that nouns can be marked for \isi{gender}, as in \REF{ex:Petrollino:dikdik} and \REF{ex:Petrollino:lion} but they can also be used in the uninflected form, which is non-specific for \isi{gender}. This is called “general form” and it corresponds to the citation form of nouns, see \citet{Petrollino2016} for further details.}
    \ex \textit{sɛgɛrɛ̂}\hspace{7mm} ʻmale dik-dik' \label{ex:Petrollino:dikdik} (dik-dik:M)
 	\end{xlist}
 \end{exe}
 
\begin{exe}
 	\ex \begin{xlist}
 	\ex \textit{zóbo}\hspace{10mm} ʻlion' (non inflected form)
    \ex \textit{zɔbɔ̂}\hspace{10mm} ʻmale lion' \label{ex:Petrollino:lion} (lion:M)
 	\end{xlist}
 \end{exe}
In the examples above, the \isi{masculine} suffix \textit{-â} merges with the \isi{final vowel} of the noun and triggers lowering of root-internal mid-high vowels. More examples of nouns marked for \isi{masculine} \isi{gender} can be found in section \sectref{sec:Petrollino:masculinenouns}.

\subsection{Syllable structure} \label{sec:Petrollino:Syllablestructure}

\ili{Hamar} nouns and verbs are mainly disyllabic. Trisyllabic words are more rare. There are four possible syllable types: CV, CVV\footnote{Long vowels are restricted to the first syllable of a word, but the behaviour of accent (discussed in the next section) does not allow a trochaic analysis. Further investigation into vowel distribution is needed in order to better understand foot structure.}, CVC and CVVC. The latter is found only in monosyllabic nouns, and in order to avoid CVVC.CV word types, the long vowel of CVVC nouns is shortened when inflectional and derivational suffixes are attached, see example \REF{ex:Petrollino:rattá} above and \REF{ex:Petrollino:arm} and \REF{ex:Petrollino:yiir} below.

\begin{exe}
	\ex \textit{áan}\hspace{9mm} ʻarmʼ\hspace{15mm}*aan-ta \ >\hspace{8mm} \textit{antâ}\hspace{8mm} ʻarm:Mʼ \label{ex:Petrollino:arm}
    \ex \textit{yíir}\hspace{9mm} ʻupper armʼ\hspace{5mm}*yiir-na >\hspace{7mm} \textit{yírna}\hspace{7mm} ʻupper arm:\textsc{pl}ʼ \label{ex:Petrollino:yiir}
\end{exe}

Onsetless syllables and consonant clusters in onset or in coda position are not permitted. Recall that glottal stop in word-initial position is not written, thus the noun for ʻarmʼ in \REF{ex:Petrollino:arm} has a CVVC structure. Geminate consonants are ambisyllabic segments filling the coda of a syllable and the onset of the following syllable:

\begin{exe}
	\ex \textit{qul.lá}\hspace{8mm}ʻgoatsʼ (goat:\textsc{pl}) \label{ex:Petrollino:qul.lá}
\end{exe}

Closed syllables tend to end in a sonorant consonant. Obstruent segments in coda position are rare and are found in monosyllabic words or in word final syllables. If consonant clusters arise where an obstruent occurs as the first segment of the cluster, metathesis and assimilation rules apply, see the examples below in which the plural marker \textit{-na} is suffixed to consonant-final nouns:  

\begin{exe}
	\ex \textit{atáɓ}\hspace{9mm} ʻtongueʼ\hspace{10mm}*atáɓ-na \ \ >\hspace{6mm} \textit{atámɓa}\hspace{4mm} ʻtongue:\textsc{pl}ʼ
    \ex \textit{cʼagáj}\hspace{7mm} ʻgreenʼ\hspace{12mm}*cʼagáj-na >\hspace{6mm} \textit{cʼagáɲa}\hspace{4mm} ʻgreen:\textsc{pl}ʼ
\end{exe}

\section{Word prosody} \label{sec:Petrollino:3}

This section outlines the prosodic properties of \ili{Hamar} nouns and verbs. Accented syllables in both nouns and verbs are obligatory and culminative \REF{ex:Petrollino:obligatorinessculminativity}. These properties, together with the fact that the syllable, rather than the mora, is the TBU \REF{ex:Petrollino:nocontour}, correspond to the definitional characteristics of stress accent \citep[231]{Hyman2006}. However, the \ili{Hamar} word-prosodic type can be analysed also as a \isi{tone} system after \possciteauthor{Hyman2001} broad definition (\citeyear{Hyman2001}), whereby “an indication of pitch enters into the lexical realisation of at least some morphemes" \citep[1367]{Hyman2001}. Accent in \ili{Hamar} has both lexical and grammatical functions; grammatical functions are observable in particular in some verbal inflections and in \isi{masculine} nouns. The interaction between lexical and \isi{grammatical accent} is discussed in \sectref{sec:Petrollino:interaction}. 

\subsection{Prosodic properties of nouns and verbs} \label{sec:Petrollino:stress}   

There is only one prominent syllable per word in \ili{Hamar} (\ref{ex:Petrollino:σ}, \ref{ex:Petrollino:σσ}), and accent-less words are not attested \REF{ex:Petrollino:σσσ}:

\begin{exe}
\ex  \begin{xlist} \label{ex:Petrollino:obligatorinessculminativity}
\ex  \textit{σ́.σ}  ,  \textit{σ.σ́} \label{ex:Petrollino:σ}
\ex *\textit{σ́.σ́}  \label{ex:Petrollino:σσ}
\ex *\textit{σ.σ} \label{ex:Petrollino:σσσ}
\end{xlist}
\end{exe}

Prominent syllables are perceptually louder, longer and with a higher pitch than neighbouring syllables; instrumental measurements show increased values for F0, duration and intensity on accented syllables. Long vowels, which can be distinctive as shown in example \REF{ex:Petrollino:vowelgemination} above, carry one and the same pitch: rising or falling pitches are not attested on long vowels \REF{ex:Petrollino:nocontour}.

\begin{exe}
	\ex \begin{xlist}
	\ex \textit{háada}\hspace{7mm}[ˈháádà]\hspace{4mm}ʻropeʼ\hspace{10mm}*[háàda] *[hàáda]
    \ex \textit{zíini}\hspace{10mm}[ˈzíínì]\hspace{6mm}ʻmosquitoʼ\hspace{3mm}
    \ex \textit{déer}\hspace{11mm}[déér]\hspace{7mm}ʻredʼ\hspace{12mm}
	\ex \textit{doobí}\hspace{9mm}[dòòbí]\hspace{5mm}ʻrainʼ\hspace{12mm}
    \end{xlist} \label{ex:Petrollino:nocontour}
\end{exe}

The measurements given in \tabref{tab:Petrollino:3} below show that phonemically long vowels are phonetically long, and long vowels are phonetically longer than short vowels in accented syllables. VL1 in \tabref{tab:Petrollino:3} refers to the \isi{vowel length} of the first syllable measured in seconds. The unaccented long vowel in \textit{goobá} ʻdecorateʼ is longer than the short accented vowel in \textit{góro} ʻColobus monkeyʼ.\footnote{The words were  elicited in isolation and the speakers were asked to repeat three tokens of each word. The examples in \tabref{tab:Petrollino:3} report the measurements of the first tokens.}

\begin{table}
\caption{Vowel length measurements}
\label{tab:Petrollino:3}
 \begin{tabular}{llll}
  \lsptoprule
    & Word & Meaning & VL1\\
  \midrule
    &   \textit{góro} & Colobus monkey &  0.091\\
    &   \textit{gobá} & run & 0.070 \\
    &   \textit{góodo} & termite eater & 0.151\\
    &	\textit{goobá} & decorate & 0.130\\
  \lspbottomrule
 \end{tabular}
\end{table}

The position of the accent is not sensitive to syllable weight: the heavy syllables CVV and CVC in the bisyllabic words in \REF{ex:Petrollino:lexicalstress} do not always attract accent.

\begin{exe}
	\ex \label{ex:Petrollino:lexicalstress} \begin{xlist}
	\ex \textit{shaa.lá}\hspace{7mm}ʻceilingʼ
    \ex \textit{zíi.ga}\hspace{10mm}ʻspinal cordʼ
    \ex \textit{síl.qa}\hspace{10mm}ʻknuckleʼ
    \ex \textit{gur.dá}\hspace{9mm}ʻvillageʼ
    \end{xlist}
\end{exe}
In trisyllabic nouns accent is found on the antepenultimate, penultimate and final syllable:

\begin{exe}
	\ex \begin{xlist}
	\ex \textit{gɛ́.da.qa}\hspace{4mm}ʻplant sp.ʼ
    \ex \textit{gu.gá.na}\hspace{4mm}ʻlightningʼ
    \ex \textit{gi.gi.rí}\hspace{7mm}ʻmolar teethʼ
    \end{xlist}
\end{exe}
Accent in nouns is thus unpredictable and lexically distinctive: 

\begin{exe}
	\ex \begin{xlist}
	\ex \textit{átti}\hspace{12mm}ʻbirdʼ\hspace{10mm}\textit{attí}\hspace{10mm}ʻfermented sorghumʼ
    \ex \textit{hámmo}\hspace{6mm}ʻfield:Fʼ\hspace{6mm}\textit{hammó}\hspace{5mm}ʻwhich:Fʼ
    \ex \textit{ásho}\hspace{11mm}ʻslopeʼ\hspace{8mm}\textit{ashó}\hspace{9mm}ʻplant sp.ʼ
\end{xlist} \label{ex:Petrollino:minimalpairs1}
\end{exe}

\begin{exe}
	\ex \begin{xlist}
	\ex \textit{ánqasi}\hspace{8mm}ʻbeeʼ\hspace{11mm}\textit{anqási}\hspace{6mm}ʻlambʼ
    \ex \textit{shékini}\hspace{8mm}ʻquartzʼ\hspace{6mm}\textit{shekíni}\hspace{5mm}ʻbeadsʼ
    \ex \textit{bagáde}\hspace{8mm}ʻloinʼ\hspace{10mm}\textit{bagadé}\hspace{5mm}ʻcooked bloodʼ
\end{xlist} \label{ex:Petrollino:minimalpairs2}
\end{exe}

\largerpage
Suffixation of nominal markers, such as the plural marker \textit{-na} or the \isi{feminine} \isi{gender} marker \textit{-no}, does not affect accent placement even when suffixation results in longer words:

\begin{exe} 
	\ex \label{ex:Petrollino:nominflections} \begin{xlist}
	\ex \textit{meté}\hspace{12mm}ʻheadʼ\hspace{9mm}\textit{meté-na}\hspace{4mm}ʻhead-\textsc{pl}ʼ
    \ex \textit{kárcʼa}\hspace{10mm}ʻcheekʼ\hspace{8mm}\textit{kárcʼa-na}\hspace{2mm}ʻcheek-\textsc{pl}ʼ
    \ex \textit{góro}\hspace{13mm}ʻmonkeyʼ\hspace{4mm}\textit{góro-no}\hspace{5mm}ʻmonkey-Fʼ
    \ex \textit{qulí}\hspace{13mm}ʻgoatʼ\hspace{10mm}\textit{qullá}\hspace{9mm}ʻgoat:\textsc{pl}ʼ\label{ex:Petrollino:qullá}
\end{xlist}
\end{exe}
In the plural noun \textit{qullá} in example \REF{ex:Petrollino:qullá}, the plural marker \textit{-na} does not attach to the terminal vowel of the noun \textit{qulí}, but it is suffixed directly to the root, assimilating to the preceding \isi{liquid} segment (*qul-na).\footnote{This phonological rule occurs when the terminal vowels of nouns are not stable. Terminal vowels in \ili{Hamar} (and in other \ili{Omotic} languages) can be “unstable” in the sense that they can be dropped and ignored with the suffixation of some morphemes. Stable and unstable terminal vowels determine different types of nominal declensions in \ili{Hamar} (see \citealt[73-77]{Petrollino2016}; \citealt{Hayward1987} and \citealt{Azeb2012} for terminal vowels in \ili{Omotic} languages).} The position of the accent thus does not change in the case of assimilation, metathesis, or other phonological processes.

Different from nouns, accent is not lexical in verbs. \ili{Hamar} verb roots are accent-less but they always occur with verbal suffixes which bear the culminative accent on the verbal word. This means that the accent is always found on the verbal suffix and never on the verb root. The singular addressee of the imperative mood for instance is formed by suffixing \textit{-á} to the verb root. This form is also used as the citation form of the verb. Prominence is therefore found on the right-most edge of the citation form of any verb:

\begin{exe}
\ex \begin{xlist}
\ex CV.CV́ \hspace{12mm}\textit{pug-á}\hspace{12mm}ʻblow!ʼ (blow-\textsc{imp}.2\textsc{sg})
\ex CVC.CV́ \hspace{9mm}\textit{ashk-á}\hspace{11mm}ʻdo!ʼ (do-\textsc{imp}.2\textsc{sg})
\ex CV.CVC.CV́ \hspace{3mm}\textit{ukuns-á}\hspace{10mm}ʻrest!ʼ (rest-\textsc{imp}.2\textsc{sg})
\end{xlist}
\end{exe}
The final accented \textit{-á} of the citation form of the verb can be substituted with other verbal suffixes of different syllabic structure:

\begin{exe}
\ex \begin{xlist}
\ex \textit{pug-é}\hspace{13mm}ʻblow!ʼ (blow-\textsc{imp}.2\textsc{pl}) \label{ex:Petrollino:pugé}
\ex \textit{ashk-íma}\hspace{7mm}ʻwithout doingʼ (do-\textsc{neg}.\textsc{sub})
\ex \textit{ukuns-énka}\hspace{4mm}ʻwhile restingʼ (rest-\textsc{cnv}) \label{ex:Petrollino:uskenka}
\ex \textit{bul-idí}\hspace{12mm}ʻopenedʼ (open-\textsc{pf})
\ex \textit{gob-áise}\hspace{10mm}ʻrunningʼ (run-\textsc{sub})
\end{xlist} \label{ex:Petrollino:grammaticalstress}
\end{exe}

Verbal suffixes cannot be combined: a single verb word cannot contain more than one verbal suffix. Adding pronominal \isi{subject} clitics to the verb word does not affect accent placement, cf. \REF{ex:Petrollino:pugé} with \REF{ex:Petrollino:kopugé} and \REF{ex:Petrollino:uskenka} with \REF{ex:Petrollino:konuskenka}:

\begin{exe}
\ex \begin{xlist}
\ex \textit{ko=pug-é}\hspace{11mm}ʻlet her blow!ʼ (3\textsc{f}=blow-\textsc{juss}) \label{ex:Petrollino:kopugé}
\ex \textit{kon=ukuns-énka}\hspace{1mm}ʻwhile she restedʼ (3\textsc{f}=rest-\textsc{cnv}) \label{ex:Petrollino:konuskenka}
\end{xlist}
\end{exe}

Some verbal tenses are distinguished only by accent placement: cf. the negative past in \REF{ex:Petrollino:NEGPast} with the negative present in \REF{ex:Petrollino:NEGPres}. 

\begin{exe}
\ex \begin{xlist}
\ex \textit{qan-átine}\hspace{8mm}ʻI did not hitʼ (hit-\textsc{past}.\textsc{neg}.1\textsc{sg}) 
\ex \textit{qan-átane}\hspace{8mm}ʻYou did not hitʼ (hit-\textsc{past}.\textsc{neg}.2\textsc{sg})
\end{xlist}\label{ex:Petrollino:NEGPast}
\end{exe} 

\begin{exe}
\ex \begin{xlist}
\ex \textit{qan-atíne}\hspace{8mm}ʻI do not hitʼ (hit-\textsc{pres}.\textsc{neg}.1\textsc{sg}) 
\ex \textit{qan-atáne}\hspace{8mm}ʻYou do not hitʼ (hit-\textsc{pres}.\textsc{neg}.2\textsc{sg}) 
\end{xlist}\label{ex:Petrollino:NEGPres}
\end{exe} 

The inflectional verb suffix used in the \isi{third person} of the negative present is realized with a final falling pitch \textit{-ê}: this contrasts with the final accent of the imperative mood which is realized with a high pitch:

\begin{exe}
\ex \begin{xlist} \label{ex:Petrollino:IMPNEG}
\ex \textit{pug-é}\hspace{15mm}ʻblow!ʼ\hspace{40mm}(blow-\textsc{imp}.2\textsc{pl})
\ex \textit{pug-ê}\hspace{15mm}ʻhe/she does not blowʼ\hspace{16mm}(blow-\textsc{pres}.\textsc{neg}.3)
\end{xlist}
\end{exe}

\begin{exe}
\ex \begin{xlist}
\ex \textit{qan-é}\hspace{15mm}ʻhit!ʼ\hspace{44mm}(hit-\textsc{imp}.2\textsc{pl})
\ex \textit{qan-ê}\hspace{15mm}ʻhe/she does not hitʼ\hspace{20mm}(hit-\textsc{pres}.\textsc{neg}.3)
\end{xlist}
\end{exe}

\begin{exe}
\ex \begin{xlist}
\ex \textit{ukuns-é}\hspace{12mm}ʻrest!ʼ\hspace{43mm}(rest-\textsc{imp}.2\textsc{pl})
\ex \textit{ukuns-ê}\hspace{12mm}ʻhe/she does not restʼ\hspace{19mm}(rest-\textsc{pres}.\textsc{neg}.3)
\end{xlist}
\end{exe}

The negative suffix \textit{-ê} is found also in the negative \isi{copula} which contrasts with the locative case \REF{ex:Petrollino:negativecopula}; a similar opposition is found in the negative existential predicator which contrasts with its interrogative counterpart \REF{ex:Petrollino:negative existential}:

\begin{exe}
\ex \label{ex:Petrollino:negativecopula}\begin{xlist}
\ex \textit{tê}\hspace{15mm}ʻis notʼ
\ex \textit{te}\hspace{15mm}ʻinsideʼ
\end{xlist}
\end{exe}

\begin{exe}
\ex \label{ex:Petrollino:negative existential} \begin{xlist}
\ex \textit{qolê}\hspace{12mm}ʻthere is notʼ
\ex \textit{qóle}\hspace{12mm}ʻwhere is?ʼ
\end{xlist}
\end{exe}

\largerpage
There are a few verb-noun pairs which can be distinguished only prosodically. This contrast is illustrated in \REF{ex:Petrollino:qanábulá} and \REF{ex:Petrollino:qánabúla}: the citation form of the verb has always final accent, whereas in the segmentally identical noun accent falls on the first syllable. These examples are important to understand the interaction between grammatical and lexical accent in \ili{Hamar}, and will be re-proposed later on in \sectref{sec:Petrollino:interaction}:

\begin{exe}
\ex \label{ex:Petrollino:qanábulá} \begin{xlist}
\ex \textit{qaná}\hspace{15mm}ʻhit!ʼ
\ex \textit{ɓulá}\hspace{16mm}ʻjump!ʼ
\end{xlist}
\end{exe}

\begin{exe}
\ex \label{ex:Petrollino:qánabúla} \begin{xlist}
\ex \textit{qána}\hspace{15mm}ʻstreamʼ
\ex \textit{ɓúla}\hspace{16mm}ʻeggʼ
\end{xlist}
\end{exe}

The examples illustrated so far show that accent is unpredictable and lexical in nouns as shown in \REF{ex:Petrollino:lexicalstress}, \REF{ex:Petrollino:minimalpairs1}, \REF{ex:Petrollino:minimalpairs2}. The accentual system of \ili{Hamar} verbs, on the other hand, is more predictable as accent is found always on function morphemes. The examples in \REF{ex:Petrollino:grammaticalstress}, \REF{ex:Petrollino:NEGPast} and \REF{ex:Petrollino:NEGPres} show the \isi{functional load} of accent on verbs. Imperative and negative verbs, moreover, display an opposition between high and falling pitch on the last syllable \REF{ex:Petrollino:IMPNEG}.  

\subsection{Masculine nouns} \label{sec:Petrollino:masculinenouns}

It was illustrated earlier that \isi{feminine} \isi{gender} and plural number suffixes do not affect the position of the accent, see examples under \REF{ex:Petrollino:nominflections} above. Different from the \isi{feminine} and the plural suffixes, the \isi{masculine} suffix \textit{-â} affects the prosody of the word as well as the realization of the vowels: nouns marked by \isi{masculine} \isi{gender} are realized with a falling pitch on the \isi{final vowel} as shown in \REF{ex:Petrollino:masculine}; the \isi{masculine} \isi{gender} marker \textit{-â}, moreover, triggers height harmony, lowering the mid-high vowels /e/ and /o/ \REF{ex:Petrollino:heightharmony}. The lowering of the mid-high vowels in \REF{ex:Petrollino:heightharmony} is the same morpho-phonological rule which was introduced in \sectref{sec:Petrollino:Phonemicinventories} for examples \REF{ex:Petrollino:dikdik} and \REF{ex:Petrollino:lion}.

\begin{exe}
\ex \label{ex:Petrollino:masculine} \begin{xlist}
\ex \textit{bankár}\hspace{10mm}ʻarrowʼ\hspace{10mm}\textit{bankarâ}\hspace{10mm}ʻarrow:Mʼ\label{ex:Petrollino:bankár} 
\ex \textit{jagá}\hspace{15mm}ʻsparrowʼ\hspace{6mm}\textit{jagâ}\hspace{16mm}ʻsparrow:Mʼ\hspace{5mm}[dʒaˈgâ]\label{ex:Petrollino:jagá}
\ex \textit{qása} [ˈqásḁ]\hspace{4mm}ʻlouseʼ\hspace{10mm}\textit{qasâ}\hspace{16mm}ʻlouse:Mʼ\hspace{10mm}[qaˈsâ]\label{ex:Petrollino:qasa}
\ex \textit{háɲa} [ˈháɲḁ]\hspace{3mm}ʻsheepʼ\hspace{9mm}\textit{haɲâ}\hspace{15mm}ʻsheep:Mʼ\hspace{9mm}[haˈɲâ] \label{ex:Petrollino:haɲa}
\end{xlist}
\end{exe}

\begin{exe}
\ex \label{ex:Petrollino:heightharmony} \begin{xlist}
\ex \textit{ási}\hspace{17mm}ʻtoothʼ\hspace{11mm}\textit{asɛ̂}\hspace{17mm}ʻtooth:Mʼ\label{ex:Petrollino:asi}
\ex \textit{ooní}\hspace{15mm}ʻhouseʼ\hspace{10mm}\textit{ɔɔnɛ̂}\hspace{15mm}ʻhouse:Mʼ\label{ex:Petrollino:house}
\ex \textit{meté}\hspace{15mm}ʻheadʼ\hspace{11mm}\textit{mɛtɛ̂}\hspace{15mm}ʻhead:Mʼ\label{ex:Petrollino:mɛtɛ}
\end{xlist}
\end{exe}
The final falling pitch of \isi{masculine} nouns is clearly audible when nouns are uttered in isolation or before a pause. The difference can however be lost in connected and allegro speech, so the falling pitch of \isi{masculine} nouns is sometimes realized as a final high pitch. Tokens of the same \isi{masculine} noun in connected speech can be uttered with both a final falling pitch or a final high pitch, so the final falling pitch on \isi{masculine} nouns cannot be analysed as a final high \isi{tone} followed by a low boundary \isi{tone} before a pause.

On the prosodic level there are two possible outcomes for nouns marked by \isi{masculine} \isi{gender}. If the uninflected noun has lexical accent on the final syllable, the derived \isi{masculine} noun is realized with a final falling \isi{tone} as in (\ref{ex:Petrollino:bankár}, \ref{ex:Petrollino:jagá}), (\ref{ex:Petrollino:house}, \ref{ex:Petrollino:mɛtɛ}). In nouns with lexical accent on the first syllable, prominence shifts to the final syllable, and a falling \isi{tone} is realized on the \isi{final vowel} of nouns such as those in (\ref{ex:Petrollino:qasa}, \ref{ex:Petrollino:haɲa}) and \REF{ex:Petrollino:asi} above. This outcome is summarized below:

\begin{exe}
\ex \begin{xlist} \label{ex:Petrollino:35}
\ex \label{CV.CV̂} CV.ˈCV́ > CV.ˈCV̂ \label{ex:Petrollino:35a}
\ex ˈCV́.CV > CV.ˈCV̂ \label{ex:Petrollino:35b}
\end{xlist}
\end{exe}
Example \REF{ex:Petrollino:35a} shows a high vs. falling opposition on the last syllable, whereas \REF{ex:Petrollino:35b} shows a low vs. falling opposition on the last syllable. In \isi{masculine} nouns which follow the pattern in \REF{ex:Petrollino:35}, \isi{grammatical accent} is culminative and obligatory; however, not all nouns follow this pattern, and exceptions to culminativity can be attested when the \isi{grammatical accent} interacts with the lexical accent of nouns. These interactions are described in the following section.

\subsection{Interaction between lexical and grammatical accent} \label{sec:Petrollino:interaction}

Nouns with lexical accent on the first syllable, like those schematised in \REF{ex:Petrollino:35b} can show variation in the prosodic realization of the \isi{masculine} form. When inflected, nouns like \textit{qása} in \REF{ex:Petrollino:qasa} or \textit{háɲa} in \REF{ex:Petrollino:haɲa} can retain their lexical accent on the first syllable together with the \isi{grammatical accent} of the \isi{masculine} suffix. In other words, the outcome for CV́.CV nouns can be CV.CV̂ or CV́.CV̂ after suffixation of the \isi{masculine} \isi{gender} marker. The variation is highly irregular and it is attested across speakers and within the same speaker's speech. Nouns like those in \REF{ex:Petrollino:irregular} do not constitute a special class of nouns; they rather belong to the most common nominal declension which represents the majority of \ili{Hamar} nouns, see \citet[74]{Petrollino2016}. 
\begin{exe}
\ex \label{ex:Petrollino:irregular}\begin{xlist}
\ex \textit{qasâ}\hspace{16mm}ʻlouse:Mʼ\hspace{10mm}[qàˈsâ] or [qáˈsâ]
\ex \textit{haɲâ}\hspace{16mm}ʻsheep:Mʼ\hspace{9mm}[hàˈɲâ] or [háˈɲâ]
\ex \textit{ɓulâ}\hspace{17mm}ʻegg:Mʼ\hspace{13mm}[ɓùˈlâ] or [ɓúˈlâ]
\end{xlist}
\end{exe}

The realization of the lexical accent on the first syllable of \isi{masculine} nouns can be fundamental to distinguish nominal stems from nominalized stems. The \isi{masculine} suffix \textit{-â}, in fact, can be suffixed also to verb roots to form relativized nouns with \isi{masculine} agreement. Since verb roots are always accent-less, \isi{masculine} relativized verbs always result as CV.CV̂ words:
\begin{exe}
\ex \label{ex:Petrollino:relativized verbs} \begin{xlist}
\ex \textit{qaná}\hspace{15mm}ʻhit!ʼ\hspace{8mm}\textit{qanâ}\hspace{8mm}ʻthe one (M) who hitsʼ\hspace{5mm} [qàˈnâ]
\ex \textit{ɓulá}\hspace{16mm}ʻjump!ʼ\hspace{4mm}\textit{ɓulâ}\hspace{9mm}ʻthe one (M) who jumpsʼ\hspace{2mm}[ɓùˈlâ]
\end{xlist}
\end{exe}

Nominalized verbs with \isi{masculine} agreement pattern like nouns with lexical accent on the final syllable, see \tabref{tab:Petrollino:4} below: uninflected nouns in the first column are paired with the respective \isi{masculine} form in the second column; verbs are paired with their \isi{masculine} nominalized form. Both nouns and verbs display a H vs. HL opposition on the final syllable:

\begin{table}
\caption{Tonal opposition 1}
\label{tab:Petrollino:4}
 \begin{tabular}{lll} 
  \lsptoprule
            & CV̀.ˈCV́ & CV̀.ˈCV̂\\ 
  \midrule
nouns		& \textit{jagá} ʻsparrowʼ & \textit{jagâ} ʻsparrow:Mʼ  \\
			& \textit{mirjá} ʻkuduʼ & \textit{mirjâ} ʻkudu:Mʼ\\ 
verbs		& \textit{pugá} ʻblow!ʼ & \textit{pugâ} ʻthe one (M) who blowsʼ \\
			& \textit{qaná} ʻhit!ʼ & \textit{qanâ} ʻthe one (M) who hitsʼ\\
  \lspbottomrule
 \end{tabular}
\end{table}
When the \isi{masculine} marker \textit{-â} is suffixed to nouns and verbs which are segmentally identical, such as those in \REF{ex:Petrollino:qanábulá} and \REF{ex:Petrollino:qánabúla} above, a H or a L \isi{tone} on the first syllable of the noun/verb root plays a crucial distinctive role: the \isi{nominalized verb} always has a L.HL melody, whereas the segmentally identical \isi{masculine} noun is realized as H.HL. Contrast is maintained between segmentally identical nouns and verbs through the accent (\isi{tone}) system, so these noun/verb pairs show a H vs L tonal opposition on the first syllable as illustrated in \tabref{tab:Petrollino:5}. 

\begin{table}
\caption{Tonal opposition 2}
\label{tab:Petrollino:5}
 \begin{tabular}{lll} 
  \lsptoprule
            & CV́.ˈCV̂ & CV̀.ˈCV̂\\ 
  \midrule
		& \textit{qanâ} ʻstream:Mʼ [qánâ] & \textit{qanâ} ʻthe one (M) who hitsʼ [qànâ]\\
		& \textit{ɓulâ} ʻegg:Mʼ [ɓúlâ] & \textit{ɓulâ} ʻthe one (M) who jumpsʼ [ɓùlâ]\\
  \lspbottomrule
 \end{tabular}
\end{table}

\section{Conclusions} \label{sec:Petrollino:4}
The \ili{Hamar} prosodic system represents an “intermediate” type in Hyman's word-prosodic typology \citep{Hyman2006,Hyman2009}, in the sense that it displays properties of both stress and \isi{tone}. On nouns and verbs accent is culminative and obligatory, showing stress-like properties. Accent is lexically contrastive in any word position in nouns, whereas it is grammatical in verbs. Tone-like properties can be observed in the verbal domain, where a H vs. HL opposition is found on the last syllable of the imperative and negative form of the verb \REF{ex:Petrollino:IMPNEG}, but also when the \isi{grammatical accent} of the \isi{masculine} \isi{gender} marker interacts with the lexical accent of verb roots and nouns. In this case, paradigmatic tonal contrasts arise on the first syllable (\tabref{tab:Petrollino:5}) and the last syllable (\tabref{tab:Petrollino:4}) of both nouns and verbs.
This preliminary analysis shows also the category-specific phonological effects which distinguish \ili{Hamar} nouns from verbs: as illustrated in \sectref{sec:Petrollino:stress}, \ili{Hamar} nouns allow more contrastive prosodic choices than verbs; this phenomenon is described by \citet{Smith2011} in terms of greater “phonological privilege” of nominal categories over verbs. Phonological processes can be sensitive to parts of speech, and according to Smith's typological study parts of speech tend to conform to the following hierarchy of phonological privilege: nouns > adjectives > verbs; the majority of category-specific phonological effects involves mainly suprasegmental and prosodic phenomena, rather than segmental phenomena \citep[2448]{Smith2011}. Nouns' phonological privilege in \ili{Hamar} is also supported by the fact that \isi{vowel harmony}, which gives rise to mid-low vowels, takes place only in nouns and not in verbs.

\section*{Abbreviations}
\begin{tabularx}{.45\textwidth}{>{\scshape}lQ}
1 & {first person}\\
 2 & {second person}\\
 3 & {third person}\\
 m & {masculine}\\
 f & {feminine}\\
 pl & plural\\
 sg & singular\\ 
\end{tabularx}
\begin{tabularx}{.45\textwidth}{>{\scshape}lQ}
 imp & imperative\\
 neg & negative\\
 sub & subordinative\\
 cnv & converb\\
 pf & perfective\\
 juss & jussive\\
 past & {past tense}\\
 pres & present tense\\ 
\end{tabularx}


\section*{Acknowledgments}
I am grateful to the LABEX ASLAN (ANR-10-LABX-0081), Université de
Lyon, for the financial support (ANR-11-IDEX-0007, “Investissements d’Avenir” program operated by the \ili{French} National Research Agency). I wish to thank the anonymous reviewers for the thoughtful comments on the original paper. 

\sloppy
\printbibliography[heading=subbibliography,notkeyword=this]

\end{document}