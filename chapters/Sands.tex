\documentclass[output=paper,newtxmath,modfonts,nonflat,draftmode]{langsci/langscibook} 

\title{Clicks on the fringes of the Kalahari Basin Area}

\author{Bonny Sands\affiliation{Northern Arizona University} \lastand Hilde Gunnink\affiliation{Ghent University}}


\abstract{How do we define the limits of a linguistic area? Typologically rare features may spill out beyond the bounds of an otherwise well-defined linguistic area. Rather than viewing the “fuzzy” boundary of a linguistic area as a problem, it can instead be seen to be an integral part of the structure of the linguistic area which may include a core, “depleted core”, fringe and even areas beyond the fringe. Clicks are a typical feature of the Kalahari Basin linguistic area, but their patterning on the fringes of this area is not so well-known. Clicks have been borrowed into Bantu languages spoken on the fringes of the area, but their functional load, as measured by the number of click phonemes and frequency of clicks in the lexicon, is lower than in the languages of the core of the area. Clicks have also been borrowed into Bantu languages spoken beyond the fringe of the area, but the functional load of clicks in these ultimate recipients is very low. Processes of click loss, both in Bantu languages and Khoisan languages on the fringe, show the same geographical patterning. The geographical distribution of clicks in southern Africa can be compared to the situation in eastern Africa, where there is evidence for an old linguistic area including Hadza and Sandawe in its core and Dahalo in its fringe. }

%\keywords{areal linguistics, borrowing, functional load, phonological change, linguistic divergence}

\IfFileExists{../localcommands.tex}{%hack to check whether this is being compiled as part of a collection or standalone
  \usepackage{pifont}
\usepackage{savesym}

\savesymbol{downingtriple}
\savesymbol{downingdouble}
\savesymbol{downingquad}
\savesymbol{downingquint}
\savesymbol{suph}
\savesymbol{supj}
\savesymbol{supw}
\savesymbol{sups}
\savesymbol{ts}
\savesymbol{tS}
\savesymbol{devi}
\savesymbol{devu}
\savesymbol{devy}
\savesymbol{deva}
\savesymbol{N}
\savesymbol{Z}
\savesymbol{circled}
\savesymbol{sem}
\savesymbol{row}
\savesymbol{tipa}
\savesymbol{tableauxcounter}
\savesymbol{tabhead}
\savesymbol{inp}
\savesymbol{inpno}
\savesymbol{g}
\savesymbol{hanl}
\savesymbol{hanr}
\savesymbol{kuku}
\savesymbol{ip}
\savesymbol{lipm}
\savesymbol{ripm}
\savesymbol{lipn}
\savesymbol{ripn} 
% \usepackage{amsmath} 
% \usepackage{multicol}
\usepackage{qtree} 
\usepackage{tikz-qtree,tikz-qtree-compat}
% \usepackage{tikz}
\usepackage{upgreek}


%%%%%%%%%%%%%%%%%%%%%%%%%%%%%%%%%%%%%%%%%%%%%%%%%%%%
%%%                                              %%%
%%%           Examples                           %%%
%%%                                              %%%
%%%%%%%%%%%%%%%%%%%%%%%%%%%%%%%%%%%%%%%%%%%%%%%%%%%%
% remove the percentage signs in the following lines
% if your book makes use of linguistic examples
\usepackage{tipa}  
\usepackage{pstricks,pst-xkey,pst-asr}

%for sande et al
\usepackage{pst-jtree}
\usepackage{pst-node}
%\usepackage{savesym}


% \usepackage{subcaption}
\usepackage{multirow}  
\usepackage{./langsci/styles/langsci-optional} 
\usepackage{./langsci/styles/langsci-lgr} 
\usepackage{./langsci/styles/langsci-glyphs} 
\usepackage[normalem]{ulem}
%% if you want the source line of examples to be in italics, uncomment the following line
% \def\exfont{\it}
\usetikzlibrary{arrows.meta,topaths,trees}
\usepackage[linguistics]{forest}
\forestset{
	fairly nice empty nodes/.style={
		delay={where content={}{shape=coordinate,for parent={
					for children={anchor=north}}}{}}
}}
\usepackage{soul}
\usepackage{arydshln}
% \usepackage{subfloat}
\usepackage{langsci/styles/langsci-gb4e} 
   
% \usepackage{linguex}
\usepackage{vowel}

\usepackage{pifont}% http://ctan.org/pkg/pifont
\newcommand{\cmark}{\ding{51}}%
\newcommand{\xmark}{\ding{55}}%
 
 
 %Lamont
 \makeatletter
\g@addto@macro\@floatboxreset\centering
\makeatother

\usepackage{newfloat} 
\DeclareFloatingEnvironment[fileext=tbx,name=Tableau]{tableau}
  %add all your local new commands to this file
\newcommand{\downingquad}[4]{\parbox{2.5cm}{#1}\parbox{3.5cm}{#2}\parbox{2.5cm}{#3}\parbox{3.5cm}{#4}}
\newcommand{\downingtriple}[3]{\parbox{4.5cm}{#1}\parbox{3cm}{#2}\parbox{3cm}{#3}}
\newcommand{\downingdouble}[2]{\parbox{4.5cm}{#1}\parbox{6cm}{#2}}
\newcommand{\downingquint}[5]{\parbox{1.75cm}{#1}\parbox{2.25cm}{#2}\parbox{2cm}{#3}\parbox{3cm}{#4}\parbox{2cm}{#5}}
\newcolumntype{Y}{>{\centering\arraybackslash}X}
\newcolumntype{T}{>{\centering\arraybackslash}m{2cm}}

%commands for Kusmer paper below
\newcommand{\ip}{$\upiota$}
\newcommand{\lipm}{(\_{\ip-Max}}
\newcommand{\ripm}{)\_{\ip-Max}}
\newcommand{\lipn}{(\_{\ip}}
\newcommand{\ripn}{)\_{\ip}}
\renewcommand{\_}[1]{\textsubscript{#1}}


%commands for Pillion paper below
\newcommand{\suph}{\textipa{\super h}}
\newcommand{\supj}{\textipa{\super j}}
\newcommand{\supw}{\textipa{\super w}}
\newcommand{\ts}{\textipa{\t{ts}}}
\newcommand{\tS}{\textipa{\t{tS}}}
\newcommand{\devi}{\textipa{\r*i}}
\newcommand{\devu}{\textipa{\r*u}}
\newcommand{\devy}{\textipa{\r*y}}
\newcommand{\deva}{\textipa{\r*a}}
\renewcommand{\N}{\textipa{N}}
\newcommand{\Z}{\textipa{Z}}
% 

%commands for Diercks paper below
\newcommand{\circled}[1]{\begin{tikzpicture}[baseline=(word.base)]
\node[draw, rounded corners, text height=8pt, text depth=2pt, inner sep=2pt, outer sep=0pt, use as bounding box] (word) {#1};
\end{tikzpicture}
}

%commands for Pesetsky paper below
% \newcommand{\sem}[2][]{\mbox{$[\![ $\textbf{#2}$ ]\!]^{#1}$}}
\newcommand{\sem}[2][]{\mbox{$[[ $\textbf{#2}$ ]]^{#1}$}}

% \newcommand{\ripn}{{\color{red}ripn}}%this is used but never defined. Please update the definition



%commands for Lamont paper below
\newcommand{\row}[4]{
	#1. & 
    /{#2}/ & 
    [{#3}] & 
    `#4' \\ 
}
%\newcounter{tableauxcounter}
\newcommand{\tabhead}[2]{
%     \captionsetup{labelformat=empty}
%     \stepcounter{tableauxcounter}
%     \addtocounter{table}{-1}
% 	\centering
% 	\caption{Tableau \thetableauxcounter: #1}
	\caption{#1}
	\label{#2}
}
\newcommand{\candref}[2]{{(\ref{#1}#2)}}
\newcommand{\tableauref}[1]{{Tableau~\ref{#1}}}
% tableaux
\newcommand{\inp}[1]{\multicolumn{2}{|l||}{{#1}}}
\newcommand{\inpno}[1]{\multicolumn{2}{|l||}{#1}}
\newcommand{\g}{\cellcolor{lightgray}}
\newcommand{\hanl}{\HandLeft}
\newcommand{\hanr}{\HandRight}
\newcommand{\kuku}{Kuk\'{u}}

% \newcommand{\nocaption}[1]{{\color{red} Please provide a caption}}

% \providecommand{\biberror}[1]{{\color{red}#1}}

\definecolor{RED}{cmyk}{0.05,1,0.8,0}


\newfontfamily\amharicfont[Script = Ethiopic, Scale = 1.0]{AbyssinicaSIL}
\newcommand{\amh}[1]{{\amharicfont #1}}

% 
% %Gjersoe
\usepackage{textgreek}
% 
\newcommand{\viol}{\fontfamily{MinionPro-OsF}\selectfont\rotatebox{60}{$\star$}}
\newcommand{\myscalex}{0.45}
\newcommand{\myscaley}{0.65}
%\newcommand{\red}[1]{\textcolor{red}{#1}}
%\newcommand{\blue}[1]{\textcolor{blue}{#1}}
\newcommand{\epen}[1]{\colorbox{jgray}{#1}}
\newcommand{\hand}{{\normalsize \ding{43}}}
\definecolor{jgray}{gray}{0.8} 
\usetikzlibrary{positioning}
\usetikzlibrary{matrix}
\newcommand{\mora}{\textmu\xspace}
\newcommand{\si}{\textsigma\xspace}
\newcommand{\ft}{\textPhi\xspace}
\newcommand{\tone}{\texttau\xspace}
\newcommand{\word}{\textomega\xspace}
% \newcommand{\ts}{\texttslig}
\newcommand{\fns}{\footnotesize}
\newcommand{\ns}{\normalsize}
\newcommand{\vs}{\vspace{1em}}
\newcommand{\bs}{\textbackslash}   % backslash
\newcommand{\cmd}[1]{{\bf \color{red}#1}}   % highlights command
\newcommand{\scell}[2][l]{\begin{tabular}[#1]{@{}c@{}}#2\end{tabular}}
% \interfootnotelinepenalty=10000

% --- Snider Representations --- %

\newcommand{\RepLevelHh}{
\begin{minipage}{0.10\textwidth}
\begin{tikzpicture}[xscale=\myscalex,yscale=\myscaley]
%\node (syl) at (0,0) {Hi};
\node (Rt) at (0,1) {o};
\node (H) at (-0.5,2) {H};
\node (R) at (0.5,3) {h};
%\draw [thick] (syl.north) -- (Rt.south) ;
\draw [thick] (Rt.north) -- (H.south) ;
\draw [thick] (Rt.north) -- (R.south) ;
\end{tikzpicture}
\end{minipage}
}

\newcommand{\RepLevelLh}{
\begin{minipage}{0.10\textwidth}
\begin{tikzpicture}[xscale=\myscalex,yscale=\myscaley]
%\node (syl) at (0,0) {Mid2};
\node (Rt) at (0,1) {o};
\node (H) at (-0.5,2) {L};
\node (R) at (0.5,3) {h};
%\draw [thick] (syl.north) -- (Rt.south) ;
\draw [thick] (Rt.north) -- (H.south) ;
\draw [thick] (Rt.north) -- (R.south) ;
\end{tikzpicture}
\end{minipage}
}

\newcommand{\RepLevelHl}{
\begin{minipage}{0.10\textwidth}
\begin{tikzpicture}[xscale=\myscalex,yscale=\myscaley]
%\node (syl) at (0,0) {Mid1};
\node (Rt) at (0,1) {o};
\node (H) at (-0.5,2) {H};
\node (R) at (0.5,3) {l};
%\draw [thick] (syl.north) -- (Rt.south) ;
\draw [thick] (Rt.north) -- (H.south) ;
\draw [thick] (Rt.north) -- (R.south) ;
\end{tikzpicture}
\end{minipage}
}

\newcommand{\RepLevelLl}{
\begin{minipage}{0.10\textwidth}
\begin{tikzpicture}[xscale=\myscalex,yscale=\myscaley]
%\node (syl) at (0,0) {Lo};
\node (Rt) at (0,1) {o};
\node (H) at (-0.5,2) {L};
\node (R) at (0.5,3) {l};
%\draw [thick] (syl.north) -- (Rt.south) ;
\draw [thick] (Rt.north) -- (H.south) ;
\draw [thick] (Rt.north) -- (R.south) ;
\end{tikzpicture}
\end{minipage}
}

% --- Representations --- %

\newcommand{\RepLevel}{
\begin{minipage}{0.10\textwidth}
\begin{tikzpicture}[xscale=\myscalex,yscale=\myscaley]
\node (syl) at (0,0) {\textsigma};
\node (Rt) at (0,1) {o};
\node (H) at (-0.5,2) {\texttau};
\node (R) at (0.5,3) {\textrho};
\draw [thick] (syl.north) -- (Rt.south) ;
\draw [thick] (Rt.north) -- (H.south) ;
\draw [thick] (Rt.north) -- (R.south) ;
\end{tikzpicture}
\end{minipage}
}

\newcommand{\RepContour}{
\begin{minipage}{0.10\textwidth}
\begin{tikzpicture}[xscale=\myscalex,yscale=\myscaley]
\node (syl) at (0,0) {\textsigma};
\node (Rt) at (0,1) {o};
\node (H) at (-0.5,2) {\texttau};
\node (R) at (0.5,3) {\textrho};
\node (Rt2) at (1.5,1.0) {o};
%\node (H2) at (1.0,2) {$\tau$};
%\node (R2) at (2.0,2.5) {R};
\draw [thick] (syl.north) -- (Rt.south) ;
\draw [thick] (Rt.north) -- (H.south) ;
\draw [thick] (Rt.north) -- (R.south) ;
\draw [thick] (syl.north) -- (Rt2.south) ;
%\draw [thick] (Rt2.north) -- (H2.south) ;
%\draw [thick] (Rt2.north) -- (R2.south) ;
\end{tikzpicture}
\end{minipage}
}


% --- OT constraints --- %

\newcommand{\IllustrationDown}{
\begin{minipage}{0.09\textwidth}
\begin{tikzpicture}[xscale=0.7,yscale=0.45]
\node (reg) at (0,0.75) {{\small \textalpha}};
\node (arrow) at (0,0) {{\fns $\downarrow$}};
\node (Rt) at (0,-0.75) {{\small \textbeta}};
\end{tikzpicture}
\end{minipage}
}

\newcommand{\IllustrationUp}{
\begin{minipage}{0.09\textwidth}
\begin{tikzpicture}[xscale=0.7,yscale=0.45]
\node (reg) at (0,0.75) {{\small \textalpha}};
\node (arrow) at (0,0) {{\fns $\uparrow$}};
\node (Rt) at (0,-0.75) {{\small \textbeta}};
\end{tikzpicture}
\end{minipage}
}

\newcommand{\MaxAB}{
\begin{minipage}{0.09\textwidth}
\begin{tikzpicture}[xscale=0.6,yscale=0.4]
\node (max) at (0,0) {{\small \textsc{Max}}};
\node (reg) at (0.75,0.5) {{\fns \textalpha}};
\node (arrow) at (0.75,0) {{\tiny $\downarrow$}};
\node (Rt) at (0.75,-0.5) {{\fns \textbeta}};
\end{tikzpicture}
\end{minipage}
}

\newcommand{\DepAB}{
\begin{minipage}{0.09\textwidth}
\begin{tikzpicture}[xscale=0.6,yscale=0.4]
\node (max) at (0,0) {{\small \textsc{Dep}}};
\node (reg) at (0.75,0.5) {{\fns \textalpha}};
\node (arrow) at (0.75,0) {{\tiny $\downarrow$}};
\node (Rt) at (0.75,-0.5) {{\fns \textbeta}};
\end{tikzpicture}
\end{minipage}
}

\newcommand{\DepHReg}{
\begin{minipage}{0.055\textwidth}
\begin{tikzpicture}[xscale=0.6,yscale=0.4]
\node (dep) at (0,0) {{\small \textsc{Dep}}};
\node (reg) at (0,-1.0) {{\small h}};
\end{tikzpicture}
\end{minipage}
}

\newcommand{\DepLReg}{
\begin{minipage}{0.055\textwidth}
\begin{tikzpicture}[xscale=0.6,yscale=0.4]
\node (dep) at (0,0) {{\small \textsc{Dep}}};
\node (reg) at (0,-1.0) {{\small l}};
\end{tikzpicture}
\end{minipage}
}

\newcommand{\DepReg}{
\begin{minipage}{0.055\textwidth}
\begin{tikzpicture}[xscale=0.6,yscale=0.4]
\node (dep) at (0,0) {{\small \textsc{Dep}}};
\node (reg) at (0,-1.0) {{\small \textrho}};
\end{tikzpicture}
\end{minipage}
}

\newcommand{\DepTRt}{
\begin{minipage}{0.1\textwidth}
\begin{tikzpicture}[xscale=0.6,yscale=0.4]
\node (dep) at (0,0) {{\small \textsc{Dep}}};
\node (t) at (0.75,0.5) {{\fns \texttau}};
\node (arrow) at (0.75,0) {{\tiny $\downarrow$}};
\node (Rt) at (0.75,-0.5) {{\fns o}};
\end{tikzpicture}
\end{minipage}
}

\newcommand{\MaxRegRt}{
\begin{minipage}{0.1\textwidth}
\begin{tikzpicture}[xscale=0.6,yscale=0.4]
\node (max) at (0,0) {{\small \textsc{Max}}};
\node (arrow) at (0.75,0) {{\tiny $\downarrow$}};
\node (Rt) at (0.75,-0.5) {{\fns o}};
\node (reg) at (0.75,0.5) {{\fns \textrho}};
\end{tikzpicture}
\end{minipage}
}

\newcommand{\RegToneByRt}{
\begin{minipage}{0.06\textwidth}
\begin{tikzpicture}[xscale=0.6,yscale=0.5]
\node[rotate=20] (arrow1) at (-0.15,0) {{\fns $\uparrow$}};
\node[rotate=340] (arrow2) at (0.15,0) {{\fns $\uparrow$}};
\node (Rt) at (0,-0.55) {{\small o}};
\node (reg) at (0.4,0.55) {{\small \textrho}};
\node (tone) at (-0.4,0.55) {{\small \texttau}};
\end{tikzpicture}
\end{minipage}
}

\newcommand{\RegToneBySyl}{
\begin{minipage}{0.06\textwidth}
\begin{tikzpicture}[xscale=0.6,yscale=0.5]
\node[rotate=20] (arrow1) at (-0.15,0) {{\fns $\uparrow$}};
\node[rotate=340] (arrow2) at (0.15,0) {{\fns $\uparrow$}};
\node (Rt) at (0,-0.55) {{\small \textsigma}};
\node (reg) at (0.4,0.55) {{\small \textrho}};
\node (tone) at (-0.4,0.55) {{\small \texttau}};
\end{tikzpicture}
\end{minipage}
}

\newcommand{\DepTone}{
\begin{minipage}{0.055\textwidth}
\begin{tikzpicture}[xscale=0.6,yscale=0.4]
\node (dep) at (0,0) {{\small \textsc{Dep}}};
\node (tone) at (0,-1.0) {{\small \texttau}};
\end{tikzpicture}
\end{minipage}
}

\newcommand{\DepTonalRt}{
\begin{minipage}{0.055\textwidth}
\begin{tikzpicture}[xscale=0.6,yscale=0.4]
\node (dep) at (0,0) {{\small \textsc{Dep}}};
\node (tone) at (0,-1.0) {{\small o}};
\end{tikzpicture}
\end{minipage}
}

\newcommand{\DepL}{
\begin{minipage}{0.055\textwidth}
\begin{tikzpicture}[xscale=0.6,yscale=0.4]
\node (dep) at (0,0) {{\small \textsc{Dep}}};
\node (tone) at (0,-1.0) {{\small L}};
\end{tikzpicture}
\end{minipage}
}

\newcommand{\DepH}{
\begin{minipage}{0.055\textwidth}
\begin{tikzpicture}[xscale=0.6,yscale=0.4]
\node (dep) at (0,0) {{\small \textsc{Dep}}};
\node (tone) at (0,-1.0) {{\small H}};
\end{tikzpicture}
\end{minipage}
}

\newcommand{\NoMultDiff}{{\small *loh}}
\newcommand{\Alt}{{\small \textsc{Alt}}}
\newcommand{\NoSkip}{{\small \scell{\textsc{No}\\\textsc{Skip}}}}


\newcommand{\RegDomRt}{
\begin{minipage}{0.030\textwidth}
\begin{tikzpicture}[xscale=0.6,yscale=0.5]
\node (arrow) at (0,0) {{\fns $\downarrow$}};
\node (Rt) at (0,-0.55) {{\small o}};
\node (reg) at (0,0.55) {{\small \textrho}};
\end{tikzpicture}
\end{minipage}
}

\newcommand{\DepRegRt}{
\begin{minipage}{0.1\textwidth}
\begin{tikzpicture}[xscale=0.6,yscale=0.4]
\node (dep) at (0,0) {{\small \textsc{Dep}}};
\node (arrow) at (0.75,0) {{\tiny $\downarrow$}};
\node (Rt) at (0.75,-0.5) {{\fns o}};
\node (reg) at (0.75,0.5) {{\fns \textrho}};
\end{tikzpicture}
\end{minipage}
}

% unused

\newcommand{\ToneByRt}{
\begin{minipage}{0.05\textwidth}
\begin{tikzpicture}[xscale=0.6,yscale=0.5]
\node (arrow) at (0,0) {{\fns $\uparrow$}};
\node (Rt) at (0,-0.55) {{\small o}};
\node (tone) at (0,0.55) {{\small \texttau}};
\end{tikzpicture}
\end{minipage}
}

\newcommand{\RegByRt}{
\begin{minipage}{0.05\textwidth}
\begin{tikzpicture}[xscale=0.6,yscale=0.5]
\node (arrow) at (0,0) {{\fns $\uparrow$}};
\node (Rt) at (0,-0.55) {{\small o}};
\node (reg) at (0,0.55) {{\small \textrho}};
\end{tikzpicture}
\end{minipage}
}

\newcommand{\ToneDomRt}{
\begin{minipage}{0.05\textwidth}
\begin{tikzpicture}[xscale=0.6,yscale=0.5]
\node (arrow) at (0,0) {{\fns $\downarrow$}};
\node (Rt) at (0,-0.55) {{\small o}};
\node (tone) at (0,0.55) {{\small \texttau}};
\end{tikzpicture}
\end{minipage}
}

% --- OT tableaus --- %

% Sec. 3.2, first tabl.

\newcommand{\OTHLInput}{
\begin{minipage}{0.17\textwidth}
\begin{tikzpicture}[xscale=\myscalex,yscale=\myscaley]
\node (tone) at (2,0) {(= H)};
\node (syl) at (0,0) {\textsigma};
\node (Rt) at (0,1) {o};
\node (H) at (-0.5,2) {H};
\node (R) at (0.5,3) {h};
\node (Rt2) at (1.5,1.0) {o};
%\node (H2) at (1.0,2) {\epen{L}};
\node (R2) at (2.0,3) {\blue{l}};
\draw [thick] (syl.north) -- (Rt.south) ;
\draw [thick] (Rt.north) -- (H.south) ;
\draw [thick] (Rt.north) -- (R.south) ;
\draw [thick] (syl.north) -- (Rt2.south) ;
%\draw [dashed] (Rt2.north) -- (H2.south) ;
%\draw [dashed] (Rt2.north) -- (R2.south) ;
\end{tikzpicture}
\end{minipage}
}

\newcommand{\OTHLWinner}{
\begin{minipage}{0.17\textwidth}
\begin{tikzpicture}[xscale=\myscalex,yscale=\myscaley]
\node (tone) at (2,0) {(= HL)};
\node (syl) at (0,0) {\textsigma};
\node (Rt) at (0,1) {o};
\node (H) at (-0.5,2) {H};
\node (R) at (0.5,3) {h};
\node (Rt2) at (1.5,1.0) {o};
\node (H2) at (1.0,2) {\epen{L}};
\node (R2) at (2.0,3) {\blue{l}};
\draw [thick] (syl.north) -- (Rt.south) ;
\draw [thick] (Rt.north) -- (H.south) ;
\draw [thick] (Rt.north) -- (R.south) ;
\draw [thick] (syl.north) -- (Rt2.south) ;
\draw [dashed] (Rt2.north) -- (H2.south) ;
\draw [dashed] (Rt2.north) -- (R2.south) ;
\end{tikzpicture}
\end{minipage}
}

\newcommand{\OTHLSpreadingHOnly}{
\begin{minipage}{0.17\textwidth}
\begin{tikzpicture}[xscale=\myscalex,yscale=\myscaley]
\node (tone) at (2,0) {(= HM)};
\node (syl) at (0,0) {\textsigma};
\node (Rt) at (0,1) {o};
\node (H) at (-0.5,2) {H};
\node (R) at (0.5,3) {h};
\node (Rt2) at (1.5,1.0) {o};
%\node (H2) at (1.0,2) {\epen{L}};
\node (R2) at (2.0,3) {\blue{l}};
\draw [thick] (syl.north) -- (Rt.south) ;
\draw [thick] (Rt.north) -- (H.south) ;
\draw [thick] (Rt.north) -- (R.south) ;
\draw [thick] (syl.north) -- (Rt2.south) ;
\draw [dashed] (Rt2.north) -- (R2.south) ;
\draw [dashed] (Rt2.north) -- (H.south) ;
\end{tikzpicture}
\end{minipage}
}

\newcommand{\OTHLInsertH}{
\begin{minipage}{0.17\textwidth}
\begin{tikzpicture}[xscale=\myscalex,yscale=\myscaley]
\node (tone) at (2,0) {(= HM)};
\node (syl) at (0,0) {\textsigma};
\node (Rt) at (0,1) {o};
\node (H) at (-0.5,2) {H};
\node (R) at (0.5,3) {h};
\node (Rt2) at (1.5,1.0) {o};
\node (H2) at (1.0,2) {\epen{H}};
\node (R2) at (2.0,3) {\blue{l}};
\draw [thick] (syl.north) -- (Rt.south) ;
\draw [thick] (Rt.north) -- (H.south) ;
\draw [thick] (Rt.north) -- (R.south) ;
\draw [thick] (syl.north) -- (Rt2.south) ;
\draw [dashed] (Rt2.north) -- (H2.south) ;
\draw [dashed] (Rt2.north) -- (R2.south) ;
\end{tikzpicture}
\end{minipage}
}

\newcommand{\OTHLOverwriting}{
\begin{minipage}{0.17\textwidth}
\begin{tikzpicture}[xscale=\myscalex,yscale=\myscaley]
\node (syl) at (0,0) {\textsigma};
\node (Rt) at (0,1) {o};
\node (H) at (-0.5,2) {H};
\node (R) at (0.5,3) {h};
\node (Rt2) at (1.5,1.0) {o};
%\node (H2) at (1.0,2) {\epen{L}};
\node (R2) at (2.0,3) {\blue{l}};
\draw [thick] (syl.north) -- (Rt.south) ;
\draw [thick] (Rt.north) -- (H.south) ;
\draw [thick] (Rt.north) -- (R.south) ;
\draw [thick] (syl.north) -- (Rt2.south) ;
%\draw [dashed] (Rt2.north) -- (H2.south) ;
\draw [dashed] (Rt.north) -- (R2.south) ;
\node (del) at (0.3,1.9) {\textbf{=}};
\end{tikzpicture}
\end{minipage}
}

\newcommand{\OTHLSpreading}{
\begin{minipage}{0.17\textwidth}
\begin{tikzpicture}[xscale=\myscalex,yscale=\myscaley]
\node (syl) at (0,0) {\textsigma};
\node (Rt) at (0,1) {o};
\node (H) at (-0.5,2) {H};
\node (R) at (0.5,3) {h};
\node (Rt2) at (1.5,1.0) {o};
%\node (H2) at (1.0,2) {\epen{L}};
\node (R2) at (2.0,3) {\blue{l}};
\draw [thick] (syl.north) -- (Rt.south) ;
\draw [thick] (Rt.north) -- (H.south) ;
\draw [thick] (Rt.north) -- (R.south) ;
\draw [thick] (syl.north) -- (Rt2.south) ;
%\draw [dashed] (Rt2.north) -- (H2.south) ;
\draw [dashed] (Rt2.north) -- (H.south) ;
\draw [dashed] (Rt2.north) -- (R.south) ;
\end{tikzpicture}
\end{minipage}
}

% Sec. 4.2, second tabl.: phrase-medial position

\newcommand{\OTHnoLInput}{
\begin{minipage}{0.17\textwidth}
\begin{tikzpicture}[xscale=\myscalex,yscale=\myscaley]
\node (tone) at (2,0) {(= H)};
\node (syl) at (0,0) {\textsigma};
\node (Rt) at (0,1) {o};
\node (H) at (-0.5,2) {H};
\node (R) at (0.5,3) {h};
\node (Rt2) at (1.5,1.0) {o};
%\node (H2) at (1.0,2) {\epen{L}};
%\node (R2) at (2.0,3) {\blue{l}};
\draw [thick] (syl.north) -- (Rt.south) ;
\draw [thick] (Rt.north) -- (H.south) ;
\draw [thick] (Rt.north) -- (R.south) ;
\draw [thick] (syl.north) -- (Rt2.south) ;
\end{tikzpicture}
\end{minipage}
}

\newcommand{\OTHnoLEpenth}{
\begin{minipage}{0.17\textwidth}
\begin{tikzpicture}[xscale=\myscalex,yscale=\myscaley]
\node (tone) at (2,0) {(= HM)};
\node (syl) at (0,0) {\textsigma};
\node (Rt) at (0,1) {o};
\node (H) at (-0.5,2) {H};
\node (R) at (0.5,3) {h};
\node (Rt2) at (1.5,1.0) {o};
\node (H2) at (1.0,2) {\epen{L}};
\node (R2) at (2.0,3) {\epen{h}};
\draw [thick] (syl.north) -- (Rt.south) ;
\draw [thick] (Rt.north) -- (H.south) ;
\draw [thick] (Rt.north) -- (R.south) ;
\draw [thick] (syl.north) -- (Rt2.south) ;
\draw [dashed] (Rt2.north) -- (H2.south) ;
\draw [dashed] (Rt2.north) -- (R2.south) ;
\end{tikzpicture}
\end{minipage}
}

\newcommand{\OTHnoLSpreading}{
\begin{minipage}{0.17\textwidth}
\begin{tikzpicture}[xscale=\myscalex,yscale=\myscaley]
\node (tone) at (2,0) {(= HH)};
\node (syl) at (0,0) {\textsigma};
\node (Rt) at (0,1) {o};
\node (H) at (-0.5,2) {H};
\node (R) at (0.5,3) {h};
\node (Rt2) at (1.5,1.0) {o};
%\node (H2) at (1.0,2) {\epen{L}};
%\node (R2) at (2.0,3) {\blue{l}};
\draw [thick] (syl.north) -- (Rt.south) ;
\draw [thick] (Rt.north) -- (H.south) ;
\draw [thick] (Rt.north) -- (R.south) ;
\draw [thick] (syl.north) -- (Rt2.south) ;
\draw [dashed] (Rt2.north) -- (H.south) ;
\draw [dashed] (Rt2.north) -- (R.south) ;
\end{tikzpicture}
\end{minipage}
}

% Sec. 4.2, third tabl., LM is unaffected by L\%

\newcommand{\OTLMInput}{
\begin{minipage}{0.2\textwidth}
\begin{tikzpicture}[xscale=\myscalex,yscale=\myscaley]
\node (tone) at (2,0) {(= LM)};
\node (syl) at (0,0) {\textsigma};
\node (Rt) at (0,1) {o};
\node (H) at (-0.5,2) {L};
\node (R) at (0.5,3) {l};
\node (Rt2) at (1.5,1.0) {o};
\node (H2) at (1.0,2) {L};
\node (R2) at (2.0,3) {h};
\node (R3) at (3.0,3) {\blue{l}};
\draw [thick] (syl.north) -- (Rt.south) ;
\draw [thick] (Rt.north) -- (H.south) ;
\draw [thick] (Rt.north) -- (R.south) ;
\draw [thick] (syl.north) -- (Rt2.south) ;
\draw [thick] (Rt2.north) -- (H2.south) ;
\draw [thick] (Rt2.north) -- (R2.south) ;
\end{tikzpicture}
\end{minipage}
}

\newcommand{\OTLMReplace}{
\begin{minipage}{0.2\textwidth}
\begin{tikzpicture}[xscale=\myscalex,yscale=\myscaley]
\node (tone) at (2,0) {(= LL)};
\node (syl) at (0,0) {\textsigma};
\node (Rt) at (0,1) {o};
\node (H) at (-0.5,2) {L};
\node (R) at (0.5,3) {l};
\node (Rt2) at (1.5,1.0) {o};
\node (H2) at (1.0,2) {L};
\node (R2) at (2.0,3) {h};
\node (R3) at (3.0,3) {\blue{l}};
\draw [thick] (syl.north) -- (Rt.south) ;
\draw [thick] (Rt.north) -- (H.south) ;
\draw [thick] (Rt.north) -- (R.south) ;
\draw [thick] (syl.north) -- (Rt2.south) ;
\draw [thick] (Rt2.north) -- (H2.south) ;
\draw [thick] (Rt2.north) -- (R2.south) ;
\draw [dashed] (Rt2.north) -- (R3.south) ;
\node (del) at (1.8,2.1) {\textbf{=}};
\end{tikzpicture}
\end{minipage}
}

\newcommand{\OTLMTwoReg}{
\begin{minipage}{0.2\textwidth}
\begin{tikzpicture}[xscale=\myscalex,yscale=\myscaley]
\node (tone) at (2,0) {(= LML)};
\node (syl) at (0,0) {\textsigma};
\node (Rt) at (0,1) {o};
\node (H) at (-0.5,2) {L};
\node (R) at (0.5,3) {l};
\node (Rt2) at (1.5,1.0) {o};
\node (H2) at (1.0,2) {L};
\node (R2) at (2.0,3) {h};
\node (R3) at (3.0,3) {\blue{l}};
\draw [thick] (syl.north) -- (Rt.south) ;
\draw [thick] (Rt.north) -- (H.south) ;
\draw [thick] (Rt.north) -- (R.south) ;
\draw [thick] (syl.north) -- (Rt2.south) ;
\draw [thick] (Rt2.north) -- (H2.south) ;
\draw [thick] (Rt2.north) -- (R2.south) ;
\draw [dashed] (Rt2.north) -- (R3.south) ;
\end{tikzpicture}
\end{minipage}
}

% Sec. 4.2, fourth tabl., L is affected by L\% but M is not

\newcommand{\OTLInput}{
\begin{minipage}{0.17\textwidth}
\begin{tikzpicture}[xscale=\myscalex,yscale=\myscaley]
\node (tone) at (2,0) {(= L)};
\node (syl) at (0,0) {\textsigma};
\node (Rt) at (0,1) {o};
\node (H) at (-0.5,2) {L};
\node (R) at (0.5,3) {l};
\node (R2) at (2,3) {\blue{l}};
\draw [thick] (syl.north) -- (Rt.south) ;
\draw [thick] (Rt.north) -- (H.south) ;
\draw [thick] (Rt.north) -- (R.south) ;
\end{tikzpicture}
\end{minipage}
}

\newcommand{\OTLLowered}{
\begin{minipage}{0.17\textwidth}
\begin{tikzpicture}[xscale=\myscalex,yscale=\myscaley]
\node (tone) at (2,0) {(= LL)};
\node (syl) at (0,0) {\textsigma};
\node (Rt) at (0,1) {o};
\node (H) at (-0.5,2) {L};
\node (R) at (0.5,3) {l};
\node (R2) at (2,3) {\blue{l}};
\draw [thick] (syl.north) -- (Rt.south) ;
\draw [thick] (Rt.north) -- (H.south) ;
\draw [thick] (Rt.north) -- (R.south) ;
\draw [dashed] (Rt.north) -- (R2.south) ;
\end{tikzpicture}
\end{minipage}
}

\newcommand{\OTMInput}{
\begin{minipage}{0.17\textwidth}
\begin{tikzpicture}[xscale=\myscalex,yscale=\myscaley]
\node (tone) at (2,0) {(= M)};
\node (syl) at (0,0) {\textsigma};
\node (Rt) at (0,1) {o};
\node (H) at (-0.5,2) {L};
\node (R) at (0.5,3) {h};
\node (R2) at (2,3) {\blue{l}};
\draw [thick] (syl.north) -- (Rt.south) ;
\draw [thick] (Rt.north) -- (H.south) ;
\draw [thick] (Rt.north) -- (R.south) ;
\end{tikzpicture}
\end{minipage}
}

\newcommand{\OTMLowered}{
\begin{minipage}{0.17\textwidth}
\begin{tikzpicture}[xscale=\myscalex,yscale=\myscaley]
\node (tone) at (2,0) {(= ML)};
\node (syl) at (0,0) {\textsigma};
\node (Rt) at (0,1) {o};
\node (H) at (-0.5,2) {L};
\node (R) at (0.5,3) {h};
\node (R2) at (2,3) {\blue{l}};
\draw [thick] (syl.north) -- (Rt.south) ;
\draw [thick] (Rt.north) -- (H.south) ;
\draw [thick] (Rt.north) -- (R.south) ;
\draw [dashed] (Rt.north) -- (R2.south) ;
\end{tikzpicture}
\end{minipage}
}

% Sec. 4.2, fifth tableau, polar questions with level tones

\newcommand{\OTLPolIn}{
\begin{minipage}{0.20\textwidth}
\begin{tikzpicture}[xscale=\myscalex-0.05,yscale=\myscaley-0.05]
\node (tone) at (3.5,0) {(= L)};
\node (syl) at (0,0) {\textsigma};
\node (syl2) at (2,0) {\red{\textsigma}};
\node (Rt) at (0,1) {o};
\node (H) at (-0.5,2) {L};
\node (R) at (0.5,3) {l};
\node (Rt2) at (2,1) {\red{o}};
\draw [thick] (syl.north) -- (Rt.south) ;
\draw [thick,red] (syl2.north) -- (Rt2.south) ;
\draw [thick] (Rt.north) -- (H.south) ;
\draw [thick] (Rt.north) -- (R.south) ;
\end{tikzpicture}
\end{minipage}
}

\newcommand{\OTLPolDef}{
\begin{minipage}{0.20\textwidth}
\begin{tikzpicture}[xscale=\myscalex-0.05,yscale=\myscaley-0.05]
\node (tone) at (3.5,0) {(= L.M)};
\node (syl) at (0,0) {\textsigma};
\node (syl2) at (2,0) {\red{\textsigma}};
\node (Rt) at (0,1) {o};
\node (H) at (-0.5,2) {L};
\node (R) at (0.5,3) {l};
\node (H2) at (1.5,2) {\epen{L}};
\node (R2) at (2.5,3) {\epen{h}};
\node (Rt2) at (2,1) {\red{o}};
\draw [thick] (syl.north) -- (Rt.south) ;
\draw [thick,red] (syl2.north) -- (Rt2.south) ;
\draw [thick] (Rt.north) -- (H.south) ;
\draw [thick] (Rt.north) -- (R.south) ;
\draw [semithick,dashed] (Rt2.north) -- (H2.south) ;
\draw [semithick,dashed] (Rt2.north) -- (R2.south) ;
\end{tikzpicture}
\end{minipage}
}

\newcommand{\OTLPolAlt}{
\begin{minipage}{0.20\textwidth}
\begin{tikzpicture}[xscale=\myscalex-0.05,yscale=\myscaley-0.05]
\node (tone) at (3.5,0) {(= L.L)};
\node (syl) at (0,0) {\textsigma};
\node (syl2) at (2,0) {\red{\textsigma}};
\node (Rt) at (0,1) {o};
\node (H) at (-0.5,2) {L};
\node (R) at (0.5,3) {l};
\node (Rt2) at (2,1) {\red{o}};
\draw [thick] (syl.north) -- (Rt.south) ;
\draw [thick,red] (syl2.north) -- (Rt2.south) ;
\draw [thick] (Rt.north) -- (H.south) ;
\draw [thick] (Rt.north) -- (R.south) ;
\draw [semithick,dashed] (Rt2.north) -- (H.south) ;
\draw [semithick,dashed] (Rt2.north) -- (R.south) ;
\end{tikzpicture}
\end{minipage}
}

% Sec. 4.2, sixth tableau, polar questions with contour tones

\newcommand{\OTLLPolIn}{
\begin{minipage}{0.23\textwidth}
\begin{tikzpicture}[xscale=\myscalex-0.05,yscale=\myscaley-0.05]
\node (tone) at (5.2,0) {(= L)};
\node (syl) at (0,0) {\textsigma};
\node (syl3) at (3.4,0) {\red{\textsigma}};
\node (Rt) at (0,1) {o};
\node (Rt2) at (1.7,1) {o};
\node (Rt3) at (3.4,1) {\red{o}};
\node (H) at (-0.5,2) {L};
\node (R) at (0.5,3) {l};
\draw [thick] (syl.north) -- (Rt.south) ;
\draw [thick] (syl.north) -- (Rt2.south) ;
\draw [thick,red] (syl3.north) -- (Rt3.south) ;
\draw [thick] (Rt.north) -- (H.south) ;
\draw [thick] (Rt.north) -- (R.south) ;
\end{tikzpicture}
\end{minipage}
}

\newcommand{\OTLLPolDef}{
\begin{minipage}{0.23\textwidth}
\begin{tikzpicture}[xscale=\myscalex-0.05,yscale=\myscaley-0.05]
\node (tone) at (5.2,0) {(= L.M)};
\node (syl) at (0,0) {\textsigma};
\node (syl3) at (3.4,0) {\red{\textsigma}};
\node (Rt) at (0,1) {o};
\node (Rt2) at (1.7,1) {o};
\node (Rt3) at (3.4,1) {\red{o}};
\node (H) at (-0.5,2) {L};
\node (R) at (0.5,3) {l};
\node (H3) at (2.9,2) {\epen{L}};
\node (R3) at (3.9,3) {\epen{h}};
\draw [thick] (syl.north) -- (Rt.south) ;
\draw [thick] (syl.north) -- (Rt2.south) ;
\draw [thick,red] (syl3.north) -- (Rt3.south) ;
\draw [thick] (Rt.north) -- (H.south) ;
\draw [thick] (Rt.north) -- (R.south) ;
\draw [dashed] (Rt3.north) -- (H3.south) ;
\draw [dashed] (Rt3.north) -- (R3.south) ;
\end{tikzpicture}
\end{minipage}
}

\newcommand{\OTLLPolSkip}{
\begin{minipage}{0.23\textwidth}
\begin{tikzpicture}[xscale=\myscalex-0.05,yscale=\myscaley-0.05]
\node (tone) at (5.2,0) {(= L.L)};
\node (syl) at (0,0) {\textsigma};
\node (syl3) at (3.4,0) {\red{\textsigma}};
\node (Rt) at (0,1) {o};
\node (Rt2) at (1.7,1) {o};
\node (Rt3) at (3.4,1) {\red{o}};
\node (H) at (-0.5,2) {L};
\node (R) at (0.5,3) {l};
\draw [thick] (syl.north) -- (Rt.south) ;
\draw [thick] (syl.north) -- (Rt2.south) ;
\draw [thick,red] (syl3.north) -- (Rt3.south) ;
\draw [thick] (Rt.north) -- (H.south) ;
\draw [thick] (Rt.north) -- (R.south) ;
\draw [dashed] (Rt3.north) -- (H.south) ;
\draw [dashed] (Rt3.north) -- (R.south) ;
\end{tikzpicture}
\end{minipage}
}  
  
\newcommand{\ilit}[1]{#1\il{#1}}    
\newcommand{\isit}[1]{#1\is{#1}}  

\makeatletter
\let\thetitle\@title
\let\theauthor\@author 
\makeatother

\newcommand{\togglepaper}[1][0]{ 
  \bibliography{../localbibliography}
  %% hyphenation points for line breaks
%% Normally, automatic hyphenation in LaTeX is very good
%% If a word is mis-hyphenated, add it to this file
%%
%% add information to TeX file before \begin{document} with:
%% %% hyphenation points for line breaks
%% Normally, automatic hyphenation in LaTeX is very good
%% If a word is mis-hyphenated, add it to this file
%%
%% add information to TeX file before \begin{document} with:
%% \include{localhyphenation}
\hyphenation{
affri-ca-te
affri-ca-tes
com-ple-ments
par-a-digm
Sha-ron
Kings-ton
phe-nom-e-non
Daul-ton
Abu-ba-ka-ri
Ngo-nya-ni
Clem-ents 
King-ston
Tru-cken-brodt
Tab-leau
cophono-logies
mark-edness
Ti-gri-nya
a-mong
Car-stens
Lu-bu-ku-su
}
\hyphenation{
affri-ca-te
affri-ca-tes
com-ple-ments
par-a-digm
Sha-ron
Kings-ton
phe-nom-e-non
Daul-ton
Abu-ba-ka-ri
Ngo-nya-ni
Clem-ents 
King-ston
Tru-cken-brodt
Tab-leau
cophono-logies
mark-edness
Ti-gri-nya
a-mong
Car-stens
Lu-bu-ku-su
}
  \papernote{\scriptsize\normalfont
    \theauthor.
    \thetitle. 
    To appear in: 
    Emily Clem,   Peter Jenks \& Hannah Sande.
    Theory and description in African Linguistics: Selected papers from the 47th Annual Conference on African Linguistics.
    Berlin: Language Science Press. [preliminary page numbering]
  }
  \pagenumbering{roman}
  \setcounter{chapter}{#1}
  \addtocounter{chapter}{-1}
}

\newcommand{\upstep}{\textupstep}


% \newcounter{tableauxcounter}

\renewcommand{\textltailn}{ɲ}
\renewcommand{\textbardotlessj}{ɟ}

\newcommand{\emphkh}[1]{\textit{#1}} %originally \textbf, banned by the guidelines



\definecolor{lsDOIGray}{cmyk}{0,0,0,0.45}


\newcommand{\xuparrow}[1]{%
  {\left\uparrow\vbox to #1{}\right.\kern-\nulldelimiterspace}
}
\renewcommand \textupstep[1]{\char"A71B#1}
\renewcommand \textdownstep[1]{\char"A71C#1}
 
 \newcommand{\ꜛ}{\textsf{ꜛ}}
 
\def\biberror{\undefined}


\newcommand{\OTbox}[1]{\resizebox{.88\textwidth}{!}{#1}}
 
  \togglepaper[35]
}{}


\begin{document}
\maketitle

\section{Introduction}\label{sec:sands:1}

It is well known that linguistic features may cluster in particular geographic regions. We argue that the \isi{functional load} of a linguistic feature may also exhibit geographical patterning. The traditional reliance on binary feature oppositions in areal linguistics limits the amount of linguistic patterning that may be detected. By looking at \isi{functional load}, as we do here, or at inter-speaker variability in the use of a feature (as done by \citealt{Kulkarni-Joshi2016}), more information about the historical processes of linguistic convergence and divergence in a particular geographical region can be revealed. 

Clicks are an oft-cited example of a cross-linguistically rare feature that is shared across multiple language families. Clicks are one characteristic feature of the Kalahari Basin Area (KBA) which has been established as a \isi{linguistic area} on the basis of morphosyntactic as well as phonological features (\citealt{Güldemann1998}; \citeyear{güldemann2013}; \citealt{naumann2015}). Clicks have also spread from the core of the KBA to certain languages spoken on the fringe of the area. We estimate the \isi{functional load} of clicks in the phoneme inventory and in the lexicon of languages of the KBA core and fringe and show that functional loads are lower in the fringe than in the core. We look at newly attested cases of \isi{click loss}, showing that there is a geographical patterning to this process as well. Finally, we discuss the \isi{functional load} of clicks in East African languages, which can be interpreted as evidence for an old \isi{linguistic area}, where continued contact with clickless languages has resulted in a reduction of the \isi{functional load} of clicks. By focusing on the fringes of a \isi{linguistic area}, we gain insight into the processes that may characterize the area over both space and time. 

\section{Comparison of functional load of clicks: Core vs. fringe}\label{sec:sands:2}

The Kalahari Basin Area (KBA) includes languages from three different families, Kx’a, \ili{Tuu} and \ili{Khoe} (formerly referred to as “\ili{Khoisan}”). Geographically speaking, the area of the KBA is also infiltrated by \ili{Bantu} languages, as well as \ili{English} and \ili{Afrikaans}. None of these are part of the \isi{linguistic area}; although the \ili{Bantu} languages encroaching on the KBA share some of its features, the similarities are too small to consider them true members of the area \citep[18]{güldemanntoappear}. The core of the KBA is situated in south-eastern Botswana and the adjacent area in Namibia. The fringe of the area can be roughly defined as the zone geographically adjacent to the core, which contains languages belonging to two or more families which participate in the \isi{linguistic area}, as well as many \ili{Bantu} languages. The fringe of the KBA encompasses most of southern Africa, excluding eastern Zimbabwe and Mozambique (see \figref{fig:sands:1}). Clicks, as one of the features of the KBA, occur in certain \ili{Bantu} languages on the fringes of the KBA. Two main clusters of \ili{Bantu} \isi{click} languages can be distinguished on the fringes of the KBA: the South-West \ili{Bantu} (SWB) \isi{click} languages, spoken on the southwestern edge of the \ili{Bantu}-speaking area, and the South-East \ili{Bantu} (SEB) \isi{click} languages, spoken on the southeastern edge of the \ili{Bantu}-speaking area \citep{Pakendorf2017}. The SWB languages are \ili{Fwe}, Manyo, \ili{Mbukushu}, Kwangali and \ili{Yeyi}, spoken on the border of Botswana, Zambia, Namibia and Angola, which is the northern fringe of the KBA. The SEB languages include the \ili{Nguni} languages \ili{Zulu}, \ili{Xhosa}, \ili{Swati}, \ili{Ndebele} and \ili{Phuthi}, and the \ili{Sotho} language Southern \ili{Sotho}, and are spoken in South Africa, Swaziland, Lesotho and in western Zimbabwe, which is part of the southeastern fringe of the KBA. Certain \ili{Bantu} languages are also spoken inside the core of the KBA, such as \ili{Tswana}, \ili{Kgalagadi} and \ili{Herero}, though none of these make use of clicks as a regular phoneme.\footnote{For \ili{Kgalagadi}, marginal clicks have been reported (\citealt[298]{Dickens1987}, \citealt[10]{Lukusa2008}), as well as for the \ili{Ngwato} dialect of \ili{Tswana} \citep[209-210]{Tlale2005}. It is possible that these languages used to have more substantial \isi{click} inventories in the past, but more research is needed to verify this possibility \citep{Pakendorf2017}.} 

It has long been recognized that clicks in \ili{Bantu} languages are the result of contact with \ili{Khoisan} languages \citep{Bleek1862}. For the SEB languages, the acquisition of clicks appears to be the result of contact with \ili{Khoekhoe} mainly, but possibly also with one or more \ili{Tuu} languages \citep{Pakendorf2017}. For the SWB languages, contact has mainly taken place with \ili{Ju} varieties and with \ili{Khwe} \citep{Gunnink2015}. There are different processes that have led to the incorporation of clicks: for the SEB languages, it has been argued that the borrowing of clicks was facilitated by the practice of \textit{hlonipha}, a taboo for married women to pronounce words that resemble the names of their male in-laws \citep{Herbert1990}. Among speakers of the SWB languages, however, the practice of \textit{hlonipha} is unknown: for these languages, the incorporation of clicks may have been motivated by sound symbolism \citep{BostoenSands2012}. Language shift from \ili{Khoisan} to \ili{Bantu} has also played a role, specifically from \ili{Khoisan}-speaking women marrying into \ili{Bantu} society \citep{Pakendorf2011}, coupled with a certain prestige attached to language of the \ili{Khoisan} speakers, and the use of clicks to flag a separate identity \citep{Gunnink2015}. 

That the \isi{functional load} of clicks in the \isi{phonemic inventory} and in the lexicon of different \isi{click} languages varies widely across languages has been previously noted \citep{Güldemann2008}. \citet{naumann2015} show that the presence of clicks, and of an inventory of more than three basic \isi{click} types is characteristic of the KBA. We use different metrics to measure \isi{functional load} and how it varies between languages of the core vs. those of the fringe of the Kalahari Basin Area, as described below. 

\largerpage
Languages of the core of the KBA typically use at least four different \isi{click} types, i.e.\ dental, alveolar, palatal and lateral. Some also use a fifth \isi{click} type, the bilabial. This contrasts sharply with the fringe languages, many of which only use one \isi{click} type, most commonly the dental; other fringe languages use two or three \isi{contrastive click} types. Botswana \ili{Yeyi} is the fringe language that is geographically closest to the core of the KBA and also the only fringe language to use four \isi{contrastive click} types. (See \tabref{tab:sands:1} for an overview.) 

The number of \isi{click} consonants in a language depends on the contrasts made involving \isi{click} types with various \isi{click} accompaniments, i.e.\ particular laryngeal, nasal and dorsal release features). We follow a unitary analysis of clicks whereby, /ǀ, ǀq, ǀqʰ/, for example, are considered to be three distinct consonants, rather than a cluster analysis which would see these as a single \isi{click} (/ǀ/) which may occur in clusters with obstruents /q/ and /qʰ/. See \citet{Bradfield2014} for a discussion of unitary vs. cluster analyses. 

\begin{table}
\caption{The functional load of clicks in core and fringe languages. Numbers are rounded to the nearest integer.}
\label{tab:sands:1}
\begin{tabularx}{\textwidth}{l>{\raggedleft}p{1.1cm}YY>{\raggedleft}p{2.5cm}@{}p{0mm}}
\lsptoprule
Language & \isi{click} types & \isi{click} phonemes & percentage of lexicon & percentage of basic lexicon&\\
\midrule
\multicolumn{5}{l}{\textbf{Core}}&\\
\midrule
{ǂHoan} & 5 & 75 & 64\% & 52\%&\\
{\ili{Ju}ǀ}\textbf{’hoan} & 4 & 47 & 69\% & 68\%&\\
{Khoekhoe} & 4 & 20 & 72\% & 66\%&\\
{Naro} & 4 & 28 & 64\% & 62\%&\\
{N{\textbar}uu} & 5 & 45 & 86\% & 77\%&\\
{!Xoon} & 5 & 80 & 73\% & 82\%&\\
{G{\textbar}ui} & 4 & 52 & 71\% & 56\%&\\
{Kua} & 4 & 26 & 58\% & 55\%&\\
{Shua} & 4 & 29 & \footnote{As no full lexicon for \ili{Shua} is available, the percentage of clicks in the lexicon cannot be given.} & 33\%&\\
{Tsua} & 4 & 34 & 56\% & 37\%&\\\midrule

\multicolumn{5}{l}{\textbf{Southern fringe (SEB)}}&\\
\midrule
{Zulu} & 3 & 15 & 14\% & 7\%&\\
{Xhosa} & 3 & 18 & 17\% & 10\%&\\
{Southern Ndebele} & 2 & 8 & 7\% & 5\%&\\
{Zimbabwean Ndebele} & 3 & 15 & 8\% & 6\%&\\
{Swati} & 1 & 4 & 12\% & 5\%&\\
{Phuti} & 3 & 12 & 8\% & 8\%&\\
{Southern Sotho} & 1 & 3 & 3-5\% & 0\%&\\\midrule

\multicolumn{5}{l}{\textbf{Northern fringe (SWB)}}&\\
\midrule
{Namibian Yeyi} & 2 & 12 & 10\% & 6\%&\\
{Botswana Yeyi} & 4 & 22 & 15\% & 8\%&\\
{Manyo} & 1 & 5 & 1\% & 1\%&\\
{Kwangali} & 1 & 5 & 2\% & 0\%&\\
{Mbukushu} & 1 & 4 & <1\% & 0\%&\\
{Fwe} & 1 & 4 & <1\% & 1\% &\\
\lspbottomrule
\end{tabularx}
\parbox{\textwidth}{\raggedright\footnotesize
Sources: SEB and SWB languages: \citet{Pakendorf2017}, and sources cited therein. ǂHoan: \citet{Collins2014}, \citet{Gruber1975}. \ili{Ju}ǀ’hoan: \citet{Dickens1994}, \citet{Miller-Ockhuizen2003}. \ili{Khoekhoe}: \citet[47]{Brugman2009}, \citet{Haacke2002}. \ili{Naro}: \citeauthor{Visser2001} (\citeyear{Visser2001}; \citeyear{Visser2013}). Nǀuu: \citet{Miller2014}, \citet{Miller2009}, \citet{Miller2007}. !Xoon: \citet{Traill1985,Traill1994}. Gǀui: \citeauthor{Nakagawa1996} (\citeyear{Nakagawa1996}; \citeyear{Nakagawa2013}). Kua: \citet{Chebanne2014}. Tsua: \citet{Mathes2016}, \citet{Chebanne2013}. \ili{Shua}: Fehn, personal communication, \citet{Vossen2013}. }
\end{table}  


In core languages we see as many as 75–80 \isi{click} phonemes (ǂHoan and !Xoon, respectively). Within the core languages, there are differences in the size of the \isi{click} inventories of different languages: Kua and \ili{Shua}, spoken on the eastern edge of the core area, use between 20 and 30 \isi{click} phonemes, and \ili{Khoekhoe}, spoken on the western edge of the core area, uses only 20 \isi{click} phonemes. Despite these differences within the core area, \isi{click} inventories of fringe languages are significantly smaller. Many fringe languages use fewer than 10 \isi{click} phonemes; between 10 and 20 \isi{click} phonemes are found in \ili{Namibian Yeyi} and the \ili{Nguni} languages. Southern \ili{Sotho} only has three \isi{click} phonemes, which may be related to the hypothesis that Southern \ili{Sotho} did not acquire clicks directly from \ili{Khoisan} languages, but as a result of contact with \ili{Nguni} languages, as many Southern \ili{Sotho} \isi{click} words are shared with \ili{Nguni} (\citealt{bourquin1951}; \citealt[23]{Doke1957}).\footnote{\ili{Sotho}-\ili{Tswana} peoples are believed to have migrated to southern Africa more recently than \ili{Nguni}-speaking populations \citep[31]{Pakendorf2017} and thus would have had a shorter period of contact with speakers of \isi{click} languages, perhaps accounting for the smaller \isi{functional load} of clicks in \ili{Sotho}-\ili{Tswana} languages than \ili{Nguni}, despite their relative proximity to the KBA.} The largest \isi{click} inventory is found in Botswana \ili{Yeyi}, which uses 22 contrastive clicks. It should be noted, however, that Botswana \ili{Yeyi} is a moribund language displaying some phonetic irregularity, and firm evidence for the phonemic status of all 22 clicks cannot be given \citep{Fulop2003}. The differences in the size of the \isi{click} inventory between core and fringe languages listed in \tabref{tab:sands:1} is illustrated in \figref{fig:sands:1}. 


\begin{figure}
\includegraphics[height=.35\textheight]{figures/sands-fig1.png}
\caption{Map showing the relative sizes of click inventories of languages of the core and fringe of the Kalahari Basin Area (based on data in \tabref{tab:sands:1}). (The sizes of filled circles are proportional to the number of click phonemes in each language. The fringe is enclosed by a dotted line, the core by a thicker dashed line. Bantu languages are shown with striped circles; other languages are shown with solid circles.)} 
\label{fig:sands:1}
\end{figure}

Another parameter by which the \isi{functional load} of clicks can be measured, the occurrence in the lexicon, also yields different results for core and fringe languages. In \ili{Bantu} fringe languages, the percentage of the lexicon in which clicks occur ranges from 1 to 17\%. In all the core languages, more than 50\% of the lexicon contains a \isi{click}. This difference is illustrated in \figref{fig:sands:2}. The \isi{functional load} of clicks may also be estimated, as \citet{idiatov2016} do for labial-velar stops. They compare the expected occurrence of each consonant with the actual occurrence, presupposing that each C phoneme occurs with equal frequency. This measure tends to heighten differences between the core and the fringe, e.g. 63\% of Nǀuu consonants are clicks but they occur in 86\% of the lexicon; while 29\% of \ili{Zulu} consonants are clicks, they occur in only 14\% of the lexicon. 


\begin{figure}
\includegraphics[height=.35\textheight]{figures/sands-fig2.png}
\caption{Map with pie charts showing the functional load of clicks in the lexicon in languages of the core of the KBA and on its fringe. The percentage of clicks is shown by the solid dark color. Circles representing Bantu languages have a stippled pattern.}
\label{fig:sands:2}
\end{figure}

\largerpage
The percentage of clicks in the basic lexicon also differs between core and fringe languages, as shown in \figref{fig:sands:3}. Using a version of the Swadesh-100 list of \isi{basic vocabulary} \citep{Holman2008}, we counted a much higher percentage of \isi{click} words in \isi{basic vocabulary} in core languages than in fringe languages. Furthermore, in the core languages, the percentage of clicks in the overall lexicon and the percentage of clicks in \isi{basic vocabulary} does not differ significantly, whereas in some of the fringe languages, i.e.\ \ili{Zulu}, \ili{Xhosa} and Botswana \ili{Yeyi}, the percentage of \isi{click} words in the \isi{basic vocabulary} is significantly lower than in the overall lexicon. This is probably the result of \isi{lexical borrowing}, which is less likely to affect \isi{basic vocabulary}. In the SWB languages, borrowings from \ili{Khoisan} languages are mainly found in restricted, specialized semantic domains related to the natural environment and a foraging lifestyle \citep{Gunnink2015}. 


\begin{figure}
\includegraphics[height=.35\textheight]{figures/sands-fig3.png}
\caption{Map with pie charts showing the functional load of clicks in the basic vocabulary of languages of the core of the KBA and on its fringe. The percentage of clicks is shown by the solid dark color. Circles representing Bantu languages have a stippled pattern.} 
\label{fig:sands:3}
\end{figure}

\clearpage
The relative \isi{functional load} of a feature can be a strong indicator of the source language(s) of the feature. The \isi{functional load} of clicks in \ili{Bantu} languages is much lower than it is in \ili{Khoe}, Kx'a and \ili{Tuu} languages. The average percentage of words with clicks is more than 8 times as high in the lexicons of core KBA languages (68\%) as it is in the languages of the fringe (8\%) listed in \tabref{tab:sands:1}. Differences in percentages of clicks in lexicons of core KBA language families are relatively minimal, i.e. Kx'a (67\%), \ili{Tuu} (80\%) and \ili{Khoe} (64\%). Another example of features borrowed across language families are labial-velar stops, e.g. from \ili{Ubangian} into \ili{Bantu}. These phonemes also have a higher \isi{functional load} in the source languages than in the recipient languages: the percentage of words with labial-velar stops twice as high in the lexicons of \ili{Ubangian} languages \ili{Ngbaka} (18\%) and \ili{Ngbandi} (17\%) as it is in the neighboring \ili{Bantu} language \ili{Lingombe} (9\%) (\citealt{BostoenDonzo2013}). 

\section{Click loss in fringe languages}\label{sec:sands:3}
\largerpage[2]
The \isi{functional load} of clicks not only differs from one language to the next, but variation can also occur across dialects of a single language. We now discuss a number of cases of \ili{Bantu} languages on the fringe of the KBA where one of their varieties has undergone \isi{click loss}, leading either to the complete loss of the feature of clicks or to a reduction in its \isi{functional load}. 

\ili{Fwe} is one of the SWB \isi{click} languages spoken on the northern fringe of the KBA. Clicks in \ili{Fwe} have a low \isi{functional load}; only four \isi{click} phonemes are distinguished, and clicks have so far been found in about 80 vocabulary items, none of which are \isi{basic vocabulary}. \ili{Fwe} has a northern variety, spoken in the Sinjembela area of Zambia, and a southern variety, spoken in the Zambezi region (formerly known as Caprivi strip) in Namibia. Clicks only occur in the southern variety of \ili{Fwe}. The northern variety does not use clicks, but uses a velar consonant where the southern variety uses a \isi{click}.\footnote{Many \ili{Bantu} languages do not use IPA symbols in their official orthographies, but transcribe clicks with the letters <c>, <q> and <x>. Throughout this paper, we transcribe all clicks using the IPA symbols, even where this deviates from the source or the official orthography of the language.}

\ea  \label{ex:sands:1}
\begin{tabular}[t]{ll}
{kù-\textsuperscript{ŋ}{\textbar}ânk-à} &  Southern \ili{Fwe}\\
{kù-ŋânk-à} &  Northern \ili{Fwe}\\
\multicolumn{2}{l}{‘to shell groundnuts’}\\  
\end{tabular} 

\ex \label{ex:sands:2}
\begin{tabular}[t]{lll}
{rù-{\textbar}ɔ\'{} mà} &  Southern \ili{Fwe}\\
{rù-kɔ́mà} & Northern \ili{Fwe}\\
\multicolumn{3}{l}{‘papyrus’}\\
\end{tabular}
\z 

The correspondences in (\ref{ex:sands:1}-\ref{ex:sands:2}) could be explained as either \isi{click loss} in the northern variety or as \isi{click} insertion in the southern variety. \citet{gunninktoappear} argues that \isi{click loss} is the more likely explanation, as can be seen from the form of lexemes that use a \isi{click} in Southern \ili{Fwe}, but have a \ili{Bantu} reconstruction without a \isi{click}. The original consonant has been replaced by a \isi{click} at some point in the history of \ili{Fwe}, such as the southern \ili{Fwe} word \textit{- ŋ{\textbar}ùm-ùn-à} ‘to pull out, uproot’. This word is of \ili{Bantu} origin, as attested by the reconstruction \textit{*-túmʊd-} ‘take firewood from fire, tear asunder’ \citep{Bastin2002}, and reflexes in \ili{Bantu} languages related to \ili{Fwe} such as \ili{Tonga} \textit{-fum-un-a} ‘pull out as grass from thatch’ \citep[117]{Torrend1931}. The expected reflex in \ili{Fwe} would be \textit{-sùm-ùn-à}, as Proto-\ili{Bantu} *t followed by a high back vowel regularly changes to /s/ in \ili{Fwe} \citep[118]{Bostoen2009}. In northern \ili{Fwe}, however, this word is realized as \textit{-ŋùm-ùn-à}. The use of /ŋ/ rather than /s/ in the northern \ili{Fwe} form can only be explained as a change from the nasal \isi{click}. This shows that northern \ili{Fwe}, too, must have used clicks in the past, but lost them later, probably as the result of the lack of contact with speakers of other \isi{click} languages. Northern \ili{Fwe} is in extensive contact with \ili{Lozi}, a clickless \ili{Bantu} language, as well as \ili{Kwamashi} and \ili{Shanjo}, also \ili{Bantu} languages that do not use clicks. Southern \ili{Fwe}, however, is in contact with \ili{Yeyi}, a \ili{Bantu} language in which clicks have a higher \isi{functional load} than in \ili{Fwe}, and also with the \ili{Khoe}-\ili{Kwadi} language (Caprivi-)\ili{Khwe}. The continued contact between southern \ili{Fwe} and languages in which clicks have a high \isi{functional load} has helped this variety to maintain its clicks. 

Another example of \isi{click loss} is seen in \ili{Yeyi}, a \ili{Bantu} \isi{click} language spoken on the northern fringe of the KBA. Like \ili{Fwe}, \ili{Yeyi} has two varieties, a Namibian variety spoken in the Zambezi region (former Caprivi strip), and a Botswana variety spoken in Ngamiland. Although both varieties use clicks, the \isi{functional load} of clicks in Botswana \ili{Yeyi} is higher than in \ili{Namibian Yeyi} (\tabref{tab:sands:2}).

\begin{table}
\caption{Functional load of clicks in Yeyi}
\label{tab:sands:2}
\begin{tabularx}{\textwidth}{lp{1.5cm}p{1.5cm}p{2cm}p{3cm}}
\lsptoprule
 & \# of \isi{click} types & \# of \isi{click} phonemes & \% of clicks in vocabulary & \% of clicks in \isi{basic vocabulary}\\
\midrule
Botswana \ili{Yeyi} & 4 & 22 & 15\% & 10.4\%\\
\ili{Namibian Yeyi} & 2 & 12 & 10\% & 5.6\%\\
\lspbottomrule
\end{tabularx}
\end{table} 

As \ili{Namibian Yeyi} has fewer \isi{click} types than Botswana \ili{Yeyi}, it has merged certain \isi{click} types: examples (\ref{ex:sands:3}--\ref{ex:sands:4}) show that both palatal and dental clicks in Botswana \ili{Yeyi} correspond to dental clicks in \ili{Namibian Yeyi}. 

\pagebreak

\ea \label{ex:sands:3}
\begin{tabular}[t]{ll}
\textit{kù-ǀ}\textit{hàkà} &    Botswana \ili{Yeyi} \\
\textit{kù-ǀ}\textit{hàkà} &  {Namibian Yeyi} \\
\multicolumn{2}{l}{‘to chop’ (\citealt[10]{Lukusa2009}; \citealt[41]{Seidel2008})} \\
\end{tabular}

\ex \label{ex:sands:4}
\begin{tabular}[t]{ll}
\textit{kù-í-ǂ}\textit{hòà} &   Botswana \ili{Yeyi}\\
\textit{ku-i-ǀ}\textit{hoa} &  {Namibian Yeyi}\\
\multicolumn{2}{l}{‘to slap’ (\citealt[41]{Seidel2008}; \citealt[34]{Sommer1992})} \\
 \end{tabular}
 \z 

Click loss, where clicks in Botswana \ili{Yeyi} correspond to non-clicks in \ili{Namibian Yeyi}, is also attested, as shown in example \REF{ex:sands:5}, which shows that a \isi{click} in Botswana \ili{Yeyi} can correspond to a non-\isi{click} velar in \ili{Namibian Yeyi}.

\ea \label{ex:sands:5}
\begin{tabular}[t]{ll}  
\textit{kù-ì-ɡǃ}\textit{ámánì} &    Botswana \ili{Yeyi} \\
\textit{kù-ì-khyàmínà}   & {Namibian Yeyi} \\
\multicolumn{2}{l}{‘to throw’ (\citealt[43]{Seidel2008}; \citealt[32]{Sommer1992})} \\
 \end{tabular}
 \z 

Botswana \ili{Yeyi} is spoken much closer to the core of the Kalahari basin area than \ili{Namibian Yeyi}, and as such is in contact with languages where clicks have a high \isi{functional load}; this may have helped the language maintain its \isi{click} inventory. \ili{Namibian Yeyi}, on the other hand, is mainly in contact with \ili{Bantu} languages with fewer clicks, such as \ili{Fwe} and \ili{Mbukushu}, or no clicks, such as \ili{Lozi}, Subiya and \ili{Totela}. This contact situation may have prompted \ili{Namibian Yeyi} to simplify its \isi{click} inventory.

Click loss also occurs in \ili{Bantu} \isi{click} languages spoken on the southeastern fringe of the KBA. The \ili{Nguni} language \ili{Ndebele} has three varieties: southern \ili{Ndebele}, spoken in the Mpumalanga province of South Africa, Zimbabwean \ili{Ndebele}, spoken in eastern Zimbabwe, and northern \ili{Ndebele}, spoken in the Limpopo province of South Africa. Southern and Zimbabwean \ili{Ndebele} use clicks, but clicks have been lost in northern \ili{Ndebele}, where they have been replaced by velar non-\isi{click} consonants. This \isi{click loss} must have taken place recently: at the time of \posscitet{Ziervogel1959} research, some speakers of northern \ili{Ndebele} still used clicks in certain plant names, but a later study \citep{Skhosana2009} found that these too had been replaced by velar non-clicks. Recently, however, northern \ili{Ndebele} appears to have reborrowed clicks, probably as a result of contact with \ili{Zulu} \citep{schulzlaine2016}. 

Another case where contact did not lead to the loss of clicks, but to a decrease in their \isi{functional load}, is seen in the variety of \ili{Zulu} spoken in Soweto. Soweto is an urban area south of Johannesburg where extensive \isi{language contact}, especially between \ili{Zulu} and \ili{Sotho}, has led to the creation of an urban \isi{register} that deviates in certain aspects from the standard language. One of these deviations is the simplification of its \isi{click} inventory, specifically the loss of contrast between dental and postalveolar clicks. These \isi{click} types are contrastive in standard \ili{Zulu}, but are used as free allophones in \ili{Sowetan Zulu}. For example, the word \textit{-ǀela} ‘request’, which has a dental \isi{click} in standard \ili{Zulu}, can be realized as either \textit{\-ǀela} or \textit{-ǃela} in \ili{Sowetan Zulu}; similarly, the word \textit{-ǃala} ‘start’, which has a postalveolar \isi{click} in standard \ili{Zulu}, can be realized as either \textit{-ǀala} or \textit{-ǃala} in \ili{Sowetan Zulu} \citep[164-165]{Gunnink2014}. This merger is likely to be motivated by contact with \ili{Sotho}, which has only one \isi{click} type, the postalveolar \isi{click}. Contact with other, clickless \ili{Bantu} languages may also have played a role, such as \ili{Pedi} and \ili{Tswana}. \ili{Sowetan Zulu} is widely spoken as a second language by migrants with very diverse linguistic backgrounds, including many languages with no or fewer \isi{click} contrasts than standard \ili{Zulu}, which may also have played a role in the reduction of the \isi{functional load} of its \isi{click} inventory.

Although \isi{click loss} may occur as the result of regular sound change, as is attested in for instance the loss of a contrastive retroflex \isi{click} type in northern and southern \ili{Ju} languages \citep[cf.][]{Sands2010}, \isi{language contact} seems to play the crucial role in these \ili{Bantu} languages. Just as \ili{Bantu} languages have acquired clicks through contact with languages in which clicks have a higher \isi{functional load}, in the same way, they appear to reduce or lose their \isi{click} inventories when they come in contact with languages in which clicks have a lower \isi{functional load}, or are absent altogether. In addition to contact, however, prestige also plays a role: clicks may be discarded in areas where these sounds are associated with \ili{Khoisan} speakers, who generally have a much lower social position than \ili{Bantu} speakers \citep{Wilmsen1990}. 

\section{Clicks beyond the fringe of the Kalahari Basin Area}\label{sec:sands:4}

Clicks have not only spread from the core of the KBA to its fringe, but from the fringe to languages yet more geographically removed from the KBA, as shown in \figref{fig:sands:4}. The \isi{functional load} of clicks in \ili{Bantu} languages of eastern Zimbabwe, Mozambique and Malawi is low. They occur mainly in borrowings and ideophones. The \ili{Changana} variety of \ili{Tsonga} has three \isi{click} phonemes and 142 words with clicks \citep{Sitoe1996}. Other lects seem to have fewer \isi{click} words. Certain varieties of \ili{Karanga}, spoken in the Midlands of Zimbabwe, are reported to have a small number \isi{click} words, such as \textit{mùǀìrò} ‘whip’, \textit{-ǀùb̤à} ‘rinse mouth’ and \textit{mà-{\textbar}ìmb̤í} ‘edible caterpillars’ \citep{Pongweni1990}, but the total number of words in the lexicon with clicks is unknown. 

\begin{figure}
\includegraphics[height=.35\textheight]{figures/sands-fig4.png}
\caption{Map showing Bantu languages immediately outside of the Kalahari Basin Area fringe in which clicks occur as (marginal) phonemes}
\label{fig:sands:4}
\end{figure}

\newpage 
In the \ili{Mzimba} variety of \ili{Tumbuka}, spoken in Malawi, clicks occur in certain place names. These clicks correspond to alveolar ejectives in the \ili{Nkhamanga} variety of \ili{Tumbuka}, which lacks clicks: the \ili{Mzimba} place name \textit{!aba} is known as \textit{t’afa} in \ili{Nkhamanga}, and the \ili{Mzimba} name \textit{Enguǀwini} as \textit{Ngut’wini} in \ili{Nkhamanga} \citep{Moyo1995}. 

In \ili{Ndau} as described by \citet{Borland1970}, certain words with clicks can be found, most of which are traceable to \ili{Zulu}, such as \textit{ku-ɡǁoka} ‘wear clothes’ \citep[32]{Borland1970}, from \ili{Zulu} \textit{-ɡǁoka} ‘wear, put on’ \citep[85]{Doke1958}. There is some instability in the pronunciation of clicks in \ili{Ndau}: lateral clicks alternate with dental and alveolar clicks, i.e. \textit{chi-ǀembo} {\textasciitilde} \textit{chi-ǁembo} ‘spoon’, or \textit{n!wadi} {\textasciitilde} \textit{nǁwadi} ‘book’. Clicks also alternate with velar non-clicks, i.e. \textit{chi-ɡǁoɡo} {\textasciitilde} \textit{chi-ɡoɡo} ‘hat’ \citep[30]{Borland1970}. Other descriptions of \ili{Ndau}, such as \citet{Doke1931}, do not mention clicks, suggesting either that they are recently acquired or only found in specific dialects. 

Clicks in these \ili{Bantu} languages beyond the fringe of the KBA are not the result of direct contact with core languages, but of contact with fringe languages. The \isi{functional load} of clicks in \ili{Bantu} languages beyond the fringe is even lower than in fringe languages, showing that with each transmission, the \isi{functional load} of clicks was reduced. In many languages, the relatively high prestige of the donor language may have facilitated the adoption of clicks. 

The donor languages are likely to be \ili{Nguni} languages: many \isi{click} words have \ili{Nguni} etymologies, and contact is either ongoing or historically attested. In the case of \ili{Ndau}, \ili{Tsonga}, Chopi and \ili{Ronga}, the likely donor language appears to be \ili{Zulu}, a language with more than 10 million native speakers and an equal number of second language speakers, and a relatively high prestige. This prestige may have facilitated the introduction of clicks in certain languages. In the case of \ili{Karanga}, clicks are likely to be the result of contact with Zimbabwean \ili{Ndebele}, the main language of western Zimbabwe. For \ili{Tumbuka}, the use of clicks appears to be the result of contact with \ili{Ngoni}, the language of the former ruling class of the \ili{Tumbuka}. \ili{Ngoni} was a \ili{Nguni} language spoken by a group of migrants that fled South Africa in the nineteenth century as a result of the political upheaval of the Mfecane. They ultimately settled in eastern Africa, where they came into contact with \ili{Tumbuka} speakers. Although the \ili{Ngoni} language is no longer spoken in Malawi today, its influence on some varieties of \ili{Tumbuka} is still seen in the use of clicks, as well as other phonological features \citep{Moyo1995}. 

\section{Clicks in Khoisan fringe languages}\label{sec:sands:5}

Up to now, we have emphasized the relatively low \isi{functional load} of clicks in \ili{Bantu} languages as compared to languages of the core of the Kalahari Basin Area. In this section, we show cases of \isi{click loss} in non-\ili{Bantu} languages. Click loss has been documented primarily on the fringes of the KBA, but has affected each of the three families which participate in the \isi{linguistic area} (\ili{Khoe}-\ili{Kwadi}, \ili{Tuu}, Kx'a), as shown in \figref{fig:sands:5}. We are primarily concerned here with the loss of \isi{contrastive click} types, as this determines the number of \isi{click} types and \isi{click} phonemes in each language. Because the lexical documentation of these languages is very uneven, we will not attempt a comparison of the \isi{functional load} of clicks in their lexicons. 

Many \ili{Khoe}-\ili{Kwadi} languages have been affected by \isi{click loss} \citep{Traill1997}. \ili{Kwadi}, just beyond the fringe of the KBA,\footnote{We have placed \ili{Kwadi} just outside the fringe because it is geographically further from the other languages and also because the \isi{functional load} of clicks is comparatively low.} has lost all Proto-\ili{Khoe}-\ili{Kwadi} \isi{click} types but the dental \citep{fehn-toappear-a}. East Kalahari \ili{Khoe} languages such as \ili{Tshwao} and \ili{Shua} have lost both palatal and alveolar \isi{click} types, while \ili{Khwe} has lost only alveolar clicks \citep{fehn-toappear-a}. Tsua has full sets of accompaniments for dental and lateral clicks (11 phonemes per \isi{click} type) but only 5 alveolar and 7 palatal \isi{click} phonemes \citep{Mathes2016}. In contrast, Gǀui and \ili{Naro} in the core of the KBA have retained all Proto-\ili{Khoe} \isi{click} types, and all \isi{click} types occur with the same set of accompaniments. Click loss is sporadic but affects all \isi{click} types in \ili{Sesfontein Damara} \citep{Job2014}, a dialect of \ili{Khoekhoe}. Interestingly, \isi{click loss} was previously reported to occur in Sesfontein in an undocumented \ili{San} language known as \ili{Kubun} (ǁUbun) \citep[45]{vanwarmelo1951}. 

Click loss in the Kx'a languages \ili{Ju}ǀ'hoan and ǂHoan is much less extensive than than seen in Mupa ǃXuun. Proto-Kx'a is reconstructed with a contrastive retroflex \isi{click} *ǃǃ which has been lost in all daughter languages apart from Central \ili{Ju} lects \citep{Heine2010,Sands2010}. In addition to the loss of *ǃǃ, Mupa ǃXuun is in the process of losing most palatal and alveolar clicks (with the exception of those with nasalized, glottalized, delayed aspirated accompaniments which are generally retained) \citep{fehn-toappear-b}. Palatal clicks and alveolar clicks are replaced by alveolar and velar non-clicks, respectively \citep{fehn-toappear-b}. Click loss in the speech of young people speaking varieties of \linebreak ǃXuun in southern Angola appears to go back some generations \citep{Bleek1928,Traill1997}. 

\begin{figure}
\includegraphics[height=.35\textheight]{figures/sands-fig5.png}
\caption{Map showing Non-Bantu languages which have lost click contrasts: Kwadi, Sesfontein Damara, Khwe, Shua, Tshwao (Khoe-Kwadi); Mupa !Xuun, Juǀ'hoan, ǂHoan (Kx'a); ǀXam, ǁXegwi (Tuu)} 
\label{fig:sands:5}
\end{figure}

In the southern fringe of the KBA, some \ili{Tuu} languages of the ǃUi subgroup show signs of \isi{click loss}. ǁXegwi lost Proto-ǃUi palatal and alveolar clicks, but reborrowed the latter from \ili{Swati} \citep{sands2014, Traill1997}. ǀXam merged some or all Proto-ǃUi palatal clicks with alveolars, but reborrowed palatal clicks from \ili{Khoe} \citep{sands2014}. 

In these non-\ili{Bantu} languages, loss of clicks generally increases with distance from the core of the KBA, suggesting the process may be accelerated by contact with non-KBA languages. Languages in the north came into contact with \ili{Bantu} languages earlier than those to the south, and we see a higher rate of \isi{click loss} in the north as compared to the south. Click loss need not be indicative of divergence from the KBA; the loss of retroflex clicks in core languages \ili{Ju}ǀ'hoan and ǂHoan may be considered a convergence toward the KBA, since \ili{Khoe} and \ili{Tuu} languages do not have retroflex clicks. Different types of \isi{click loss} must be attributed to different historical contact patterns. 

The presence of clicks outside of the KBA in the non-\ili{Bantu} languages raises the likelihood that the geographical extent of the KBA was once greater than it is today. We distinguish the former presence of a larger \isi{linguistic area} outside of the present-day core and label it a depleted core. In the case of \ili{Bantu} languages on the fringe of the KBA, the presence of clicks appears to be a feature which has bled out from the core. With the depleted core languages, clicks have shown signs of fading away with greater distance from the core, particularly to the north of the present-day core. Thus, a geographical fringe may be comprised of both a depleted region and an overlapping region into which a feature has spread. 

\section{Clicks in East Africa}\label{sec:sands:6}

There are three \isi{click} languages in East Africa, as shown in \figref{fig:sands:6}: \ili{Hadza} (isolate), \ili{Dahalo} (\ili{Cushitic}) and \ili{Sandawe} (which has a tentative link to \ili{Khoe}-\ili{Kwadi}, \citealt{Güldemann2010}). We look at the \isi{functional load} of clicks in these languages and compare them to the languages of the Kalahari Basin. 


\begin{figure}
\includegraphics[height=.35\textheight]{figures/sands-fig6.png}
\caption{Map of East African click languages showing the percentage of clicks in basic vocabulary. The proposed (depleted) core is enclosed by a heavy dashed line, the fringe by a dotted line.}
\label{fig:sands:6}
\end{figure}

With three \isi{contrastive click} types, \ili{Hadza} and \ili{Sandawe} are similar to KBA fringe languages \ili{Zulu} and \ili{Xhosa}; \ili{Dahalo} has only one \isi{contrastive click} type, similar to fringe languages such as \ili{Fwe}. The number of \isi{click} phonemes in these languages is also comparable to those of the KBA fringe, ranging from 4 phonemes (/ŋ̊ǀ, ŋǀ, ŋ̊ǀʷ, ŋǀʷ/) in \ili{Dahalo} \citep{Maddieson1993}, to 12 in \ili{Hadza} \citep{Miller2012} and 15 in \ili{Sandawe} \citep{Elderkin2013,hunziker2008}. 

The frequency of clicks in the lexicon is similar in \ili{Sandawe} (21\%) and \ili{Hadza} (18\%), but much lower in \ili{Dahalo} (3\%) (based on hand counts of words in \citealt{Miller2012}, \citealt{Ten2012}, \citealt{Tosco1991}). These frequencies are similar to frequencies seen in the fringe of the KBA. Rates of clicks in \isi{basic vocabulary} are shown in \figref{fig:sands:6}. The \isi{functional load} of clicks in the \isi{basic vocabulary} of \ili{Sandawe} (37\%) and \ili{Hadza} (16\%) however, is higher than that seen in any \ili{Bantu} language. 

\newpage 
Unlike most languages of the KBA fringe, populations speaking these languages have been isolated from speakers of other \isi{click} languages for multiple generations, as shown by genetic evidence (cf. \citealt{Schlebush2012}, \citealt{Soi2015}). It seems likely that clicks in all of these languages once had a higher \isi{functional load} than they do at present, and that continued contact with clickless languages has reduced their \isi{functional load}, similar to what is seen in southern Africa. If East African \isi{click} languages once formed a \isi{linguistic area}, the \isi{functional load} of clicks suggests that \ili{Hadza} and \ili{Sandawe} are part of a depleted core and \ili{Dahalo} is part of its fringe. 

\section{Conclusion}\label{sec:sands:7}

In this paper, we have examined the distribution of clicks, one of the features of the Kalahari Basin \isi{linguistic area}, on the fringes of the area. By considering the \isi{functional load} of this feature, rather than merely its presence or absence, we have been able to reveal considerable substructuring of the \isi{linguistic area}, distinguishing a core of the area, a depleted core, and a fringe. Weak signals of the area can even be detected beyond the fringe. The \isi{functional load} of the feature of clicks diminishes with distance from the core of the area, and appears to diminish with each transmission. 

We have discussed cases where clicks are used in a specific variety of a language, but are absent in others, or where different varieties differ in the \isi{functional load} of clicks. Clicks can be acquired through contact with languages where clicks have a higher \isi{functional load}, but clicks can also be lost through contact with languages where clicks are absent or have a lower \isi{functional load}. Furthermore, the differences in \isi{click} inventory between closely-related varieties of the same languages underscore the need for dialect studies, which may elucidate the processes by which these features are acquired and lost. 

Finally, we have suggested that differences in the \isi{functional load} of a linguistic feature may be useful in identifying old linguistic areas. Outside of the Kalahari Basin Area, we have seen that the \isi{functional load} of clicks is relatively higher in \ili{Hadza} and \ili{Sandawe} than it is in \ili{Dahalo}, a pattern that is reminiscent of the relative \isi{functional load} of clicks in the core vs. the fringe of the Kalahari Basin Area. 

\section*{Acknowledgments}

We thank Koen Bostoen, Brigitte Pakendorf, Anne-Marie Fehn, Richard Bailey, Kirk Miller, and Will Grundy for their assistance. 

\todo{please address red items in bibliography}
\sloppy
\printbibliography[heading=subbibliography,notkeyword=this]

\end{document}
