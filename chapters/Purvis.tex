\documentclass[output=paper,newtxmath,modfonts,nonflat,hidelinks]{langsci/langscibook}  

\title{Hausa chat jargon: Semantic extension versus borrowing}
\author{Tristan Purvis\affiliation{American University of Nigeria}}


\IfFileExists{../localcommands.tex}{%hack to check whether this is being compiled as part of a collection or standalone
  \usepackage{pifont}
\usepackage{savesym}

\savesymbol{downingtriple}
\savesymbol{downingdouble}
\savesymbol{downingquad}
\savesymbol{downingquint}
\savesymbol{suph}
\savesymbol{supj}
\savesymbol{supw}
\savesymbol{sups}
\savesymbol{ts}
\savesymbol{tS}
\savesymbol{devi}
\savesymbol{devu}
\savesymbol{devy}
\savesymbol{deva}
\savesymbol{N}
\savesymbol{Z}
\savesymbol{circled}
\savesymbol{sem}
\savesymbol{row}
\savesymbol{tipa}
\savesymbol{tableauxcounter}
\savesymbol{tabhead}
\savesymbol{inp}
\savesymbol{inpno}
\savesymbol{g}
\savesymbol{hanl}
\savesymbol{hanr}
\savesymbol{kuku}
\savesymbol{ip}
\savesymbol{lipm}
\savesymbol{ripm}
\savesymbol{lipn}
\savesymbol{ripn} 
% \usepackage{amsmath} 
% \usepackage{multicol}
\usepackage{qtree} 
\usepackage{tikz-qtree,tikz-qtree-compat}
% \usepackage{tikz}
\usepackage{upgreek}


%%%%%%%%%%%%%%%%%%%%%%%%%%%%%%%%%%%%%%%%%%%%%%%%%%%%
%%%                                              %%%
%%%           Examples                           %%%
%%%                                              %%%
%%%%%%%%%%%%%%%%%%%%%%%%%%%%%%%%%%%%%%%%%%%%%%%%%%%%
% remove the percentage signs in the following lines
% if your book makes use of linguistic examples
\usepackage{tipa}  
\usepackage{pstricks,pst-xkey,pst-asr}

%for sande et al
\usepackage{pst-jtree}
\usepackage{pst-node}
%\usepackage{savesym}


% \usepackage{subcaption}
\usepackage{multirow}  
\usepackage{./langsci/styles/langsci-optional} 
\usepackage{./langsci/styles/langsci-lgr} 
\usepackage{./langsci/styles/langsci-glyphs} 
\usepackage[normalem]{ulem}
%% if you want the source line of examples to be in italics, uncomment the following line
% \def\exfont{\it}
\usetikzlibrary{arrows.meta,topaths,trees}
\usepackage[linguistics]{forest}
\forestset{
	fairly nice empty nodes/.style={
		delay={where content={}{shape=coordinate,for parent={
					for children={anchor=north}}}{}}
}}
\usepackage{soul}
\usepackage{arydshln}
% \usepackage{subfloat}
\usepackage{langsci/styles/langsci-gb4e} 
   
% \usepackage{linguex}
\usepackage{vowel}

\usepackage{pifont}% http://ctan.org/pkg/pifont
\newcommand{\cmark}{\ding{51}}%
\newcommand{\xmark}{\ding{55}}%
 
 
 %Lamont
 \makeatletter
\g@addto@macro\@floatboxreset\centering
\makeatother

\usepackage{newfloat} 
\DeclareFloatingEnvironment[fileext=tbx,name=Tableau]{tableau}
  %add all your local new commands to this file
\newcommand{\downingquad}[4]{\parbox{2.5cm}{#1}\parbox{3.5cm}{#2}\parbox{2.5cm}{#3}\parbox{3.5cm}{#4}}
\newcommand{\downingtriple}[3]{\parbox{4.5cm}{#1}\parbox{3cm}{#2}\parbox{3cm}{#3}}
\newcommand{\downingdouble}[2]{\parbox{4.5cm}{#1}\parbox{6cm}{#2}}
\newcommand{\downingquint}[5]{\parbox{1.75cm}{#1}\parbox{2.25cm}{#2}\parbox{2cm}{#3}\parbox{3cm}{#4}\parbox{2cm}{#5}}
\newcolumntype{Y}{>{\centering\arraybackslash}X}
\newcolumntype{T}{>{\centering\arraybackslash}m{2cm}}

%commands for Kusmer paper below
\newcommand{\ip}{$\upiota$}
\newcommand{\lipm}{(\_{\ip-Max}}
\newcommand{\ripm}{)\_{\ip-Max}}
\newcommand{\lipn}{(\_{\ip}}
\newcommand{\ripn}{)\_{\ip}}
\renewcommand{\_}[1]{\textsubscript{#1}}


%commands for Pillion paper below
\newcommand{\suph}{\textipa{\super h}}
\newcommand{\supj}{\textipa{\super j}}
\newcommand{\supw}{\textipa{\super w}}
\newcommand{\ts}{\textipa{\t{ts}}}
\newcommand{\tS}{\textipa{\t{tS}}}
\newcommand{\devi}{\textipa{\r*i}}
\newcommand{\devu}{\textipa{\r*u}}
\newcommand{\devy}{\textipa{\r*y}}
\newcommand{\deva}{\textipa{\r*a}}
\renewcommand{\N}{\textipa{N}}
\newcommand{\Z}{\textipa{Z}}
% 

%commands for Diercks paper below
\newcommand{\circled}[1]{\begin{tikzpicture}[baseline=(word.base)]
\node[draw, rounded corners, text height=8pt, text depth=2pt, inner sep=2pt, outer sep=0pt, use as bounding box] (word) {#1};
\end{tikzpicture}
}

%commands for Pesetsky paper below
% \newcommand{\sem}[2][]{\mbox{$[\![ $\textbf{#2}$ ]\!]^{#1}$}}
\newcommand{\sem}[2][]{\mbox{$[[ $\textbf{#2}$ ]]^{#1}$}}

% \newcommand{\ripn}{{\color{red}ripn}}%this is used but never defined. Please update the definition



%commands for Lamont paper below
\newcommand{\row}[4]{
	#1. & 
    /{#2}/ & 
    [{#3}] & 
    `#4' \\ 
}
%\newcounter{tableauxcounter}
\newcommand{\tabhead}[2]{
%     \captionsetup{labelformat=empty}
%     \stepcounter{tableauxcounter}
%     \addtocounter{table}{-1}
% 	\centering
% 	\caption{Tableau \thetableauxcounter: #1}
	\caption{#1}
	\label{#2}
}
\newcommand{\candref}[2]{{(\ref{#1}#2)}}
\newcommand{\tableauref}[1]{{Tableau~\ref{#1}}}
% tableaux
\newcommand{\inp}[1]{\multicolumn{2}{|l||}{{#1}}}
\newcommand{\inpno}[1]{\multicolumn{2}{|l||}{#1}}
\newcommand{\g}{\cellcolor{lightgray}}
\newcommand{\hanl}{\HandLeft}
\newcommand{\hanr}{\HandRight}
\newcommand{\kuku}{Kuk\'{u}}

% \newcommand{\nocaption}[1]{{\color{red} Please provide a caption}}

% \providecommand{\biberror}[1]{{\color{red}#1}}

\definecolor{RED}{cmyk}{0.05,1,0.8,0}


\newfontfamily\amharicfont[Script = Ethiopic, Scale = 1.0]{AbyssinicaSIL}
\newcommand{\amh}[1]{{\amharicfont #1}}

% 
% %Gjersoe
\usepackage{textgreek}
% 
\newcommand{\viol}{\fontfamily{MinionPro-OsF}\selectfont\rotatebox{60}{$\star$}}
\newcommand{\myscalex}{0.45}
\newcommand{\myscaley}{0.65}
%\newcommand{\red}[1]{\textcolor{red}{#1}}
%\newcommand{\blue}[1]{\textcolor{blue}{#1}}
\newcommand{\epen}[1]{\colorbox{jgray}{#1}}
\newcommand{\hand}{{\normalsize \ding{43}}}
\definecolor{jgray}{gray}{0.8} 
\usetikzlibrary{positioning}
\usetikzlibrary{matrix}
\newcommand{\mora}{\textmu\xspace}
\newcommand{\si}{\textsigma\xspace}
\newcommand{\ft}{\textPhi\xspace}
\newcommand{\tone}{\texttau\xspace}
\newcommand{\word}{\textomega\xspace}
% \newcommand{\ts}{\texttslig}
\newcommand{\fns}{\footnotesize}
\newcommand{\ns}{\normalsize}
\newcommand{\vs}{\vspace{1em}}
\newcommand{\bs}{\textbackslash}   % backslash
\newcommand{\cmd}[1]{{\bf \color{red}#1}}   % highlights command
\newcommand{\scell}[2][l]{\begin{tabular}[#1]{@{}c@{}}#2\end{tabular}}
% \interfootnotelinepenalty=10000

% --- Snider Representations --- %

\newcommand{\RepLevelHh}{
\begin{minipage}{0.10\textwidth}
\begin{tikzpicture}[xscale=\myscalex,yscale=\myscaley]
%\node (syl) at (0,0) {Hi};
\node (Rt) at (0,1) {o};
\node (H) at (-0.5,2) {H};
\node (R) at (0.5,3) {h};
%\draw [thick] (syl.north) -- (Rt.south) ;
\draw [thick] (Rt.north) -- (H.south) ;
\draw [thick] (Rt.north) -- (R.south) ;
\end{tikzpicture}
\end{minipage}
}

\newcommand{\RepLevelLh}{
\begin{minipage}{0.10\textwidth}
\begin{tikzpicture}[xscale=\myscalex,yscale=\myscaley]
%\node (syl) at (0,0) {Mid2};
\node (Rt) at (0,1) {o};
\node (H) at (-0.5,2) {L};
\node (R) at (0.5,3) {h};
%\draw [thick] (syl.north) -- (Rt.south) ;
\draw [thick] (Rt.north) -- (H.south) ;
\draw [thick] (Rt.north) -- (R.south) ;
\end{tikzpicture}
\end{minipage}
}

\newcommand{\RepLevelHl}{
\begin{minipage}{0.10\textwidth}
\begin{tikzpicture}[xscale=\myscalex,yscale=\myscaley]
%\node (syl) at (0,0) {Mid1};
\node (Rt) at (0,1) {o};
\node (H) at (-0.5,2) {H};
\node (R) at (0.5,3) {l};
%\draw [thick] (syl.north) -- (Rt.south) ;
\draw [thick] (Rt.north) -- (H.south) ;
\draw [thick] (Rt.north) -- (R.south) ;
\end{tikzpicture}
\end{minipage}
}

\newcommand{\RepLevelLl}{
\begin{minipage}{0.10\textwidth}
\begin{tikzpicture}[xscale=\myscalex,yscale=\myscaley]
%\node (syl) at (0,0) {Lo};
\node (Rt) at (0,1) {o};
\node (H) at (-0.5,2) {L};
\node (R) at (0.5,3) {l};
%\draw [thick] (syl.north) -- (Rt.south) ;
\draw [thick] (Rt.north) -- (H.south) ;
\draw [thick] (Rt.north) -- (R.south) ;
\end{tikzpicture}
\end{minipage}
}

% --- Representations --- %

\newcommand{\RepLevel}{
\begin{minipage}{0.10\textwidth}
\begin{tikzpicture}[xscale=\myscalex,yscale=\myscaley]
\node (syl) at (0,0) {\textsigma};
\node (Rt) at (0,1) {o};
\node (H) at (-0.5,2) {\texttau};
\node (R) at (0.5,3) {\textrho};
\draw [thick] (syl.north) -- (Rt.south) ;
\draw [thick] (Rt.north) -- (H.south) ;
\draw [thick] (Rt.north) -- (R.south) ;
\end{tikzpicture}
\end{minipage}
}

\newcommand{\RepContour}{
\begin{minipage}{0.10\textwidth}
\begin{tikzpicture}[xscale=\myscalex,yscale=\myscaley]
\node (syl) at (0,0) {\textsigma};
\node (Rt) at (0,1) {o};
\node (H) at (-0.5,2) {\texttau};
\node (R) at (0.5,3) {\textrho};
\node (Rt2) at (1.5,1.0) {o};
%\node (H2) at (1.0,2) {$\tau$};
%\node (R2) at (2.0,2.5) {R};
\draw [thick] (syl.north) -- (Rt.south) ;
\draw [thick] (Rt.north) -- (H.south) ;
\draw [thick] (Rt.north) -- (R.south) ;
\draw [thick] (syl.north) -- (Rt2.south) ;
%\draw [thick] (Rt2.north) -- (H2.south) ;
%\draw [thick] (Rt2.north) -- (R2.south) ;
\end{tikzpicture}
\end{minipage}
}


% --- OT constraints --- %

\newcommand{\IllustrationDown}{
\begin{minipage}{0.09\textwidth}
\begin{tikzpicture}[xscale=0.7,yscale=0.45]
\node (reg) at (0,0.75) {{\small \textalpha}};
\node (arrow) at (0,0) {{\fns $\downarrow$}};
\node (Rt) at (0,-0.75) {{\small \textbeta}};
\end{tikzpicture}
\end{minipage}
}

\newcommand{\IllustrationUp}{
\begin{minipage}{0.09\textwidth}
\begin{tikzpicture}[xscale=0.7,yscale=0.45]
\node (reg) at (0,0.75) {{\small \textalpha}};
\node (arrow) at (0,0) {{\fns $\uparrow$}};
\node (Rt) at (0,-0.75) {{\small \textbeta}};
\end{tikzpicture}
\end{minipage}
}

\newcommand{\MaxAB}{
\begin{minipage}{0.09\textwidth}
\begin{tikzpicture}[xscale=0.6,yscale=0.4]
\node (max) at (0,0) {{\small \textsc{Max}}};
\node (reg) at (0.75,0.5) {{\fns \textalpha}};
\node (arrow) at (0.75,0) {{\tiny $\downarrow$}};
\node (Rt) at (0.75,-0.5) {{\fns \textbeta}};
\end{tikzpicture}
\end{minipage}
}

\newcommand{\DepAB}{
\begin{minipage}{0.09\textwidth}
\begin{tikzpicture}[xscale=0.6,yscale=0.4]
\node (max) at (0,0) {{\small \textsc{Dep}}};
\node (reg) at (0.75,0.5) {{\fns \textalpha}};
\node (arrow) at (0.75,0) {{\tiny $\downarrow$}};
\node (Rt) at (0.75,-0.5) {{\fns \textbeta}};
\end{tikzpicture}
\end{minipage}
}

\newcommand{\DepHReg}{
\begin{minipage}{0.055\textwidth}
\begin{tikzpicture}[xscale=0.6,yscale=0.4]
\node (dep) at (0,0) {{\small \textsc{Dep}}};
\node (reg) at (0,-1.0) {{\small h}};
\end{tikzpicture}
\end{minipage}
}

\newcommand{\DepLReg}{
\begin{minipage}{0.055\textwidth}
\begin{tikzpicture}[xscale=0.6,yscale=0.4]
\node (dep) at (0,0) {{\small \textsc{Dep}}};
\node (reg) at (0,-1.0) {{\small l}};
\end{tikzpicture}
\end{minipage}
}

\newcommand{\DepReg}{
\begin{minipage}{0.055\textwidth}
\begin{tikzpicture}[xscale=0.6,yscale=0.4]
\node (dep) at (0,0) {{\small \textsc{Dep}}};
\node (reg) at (0,-1.0) {{\small \textrho}};
\end{tikzpicture}
\end{minipage}
}

\newcommand{\DepTRt}{
\begin{minipage}{0.1\textwidth}
\begin{tikzpicture}[xscale=0.6,yscale=0.4]
\node (dep) at (0,0) {{\small \textsc{Dep}}};
\node (t) at (0.75,0.5) {{\fns \texttau}};
\node (arrow) at (0.75,0) {{\tiny $\downarrow$}};
\node (Rt) at (0.75,-0.5) {{\fns o}};
\end{tikzpicture}
\end{minipage}
}

\newcommand{\MaxRegRt}{
\begin{minipage}{0.1\textwidth}
\begin{tikzpicture}[xscale=0.6,yscale=0.4]
\node (max) at (0,0) {{\small \textsc{Max}}};
\node (arrow) at (0.75,0) {{\tiny $\downarrow$}};
\node (Rt) at (0.75,-0.5) {{\fns o}};
\node (reg) at (0.75,0.5) {{\fns \textrho}};
\end{tikzpicture}
\end{minipage}
}

\newcommand{\RegToneByRt}{
\begin{minipage}{0.06\textwidth}
\begin{tikzpicture}[xscale=0.6,yscale=0.5]
\node[rotate=20] (arrow1) at (-0.15,0) {{\fns $\uparrow$}};
\node[rotate=340] (arrow2) at (0.15,0) {{\fns $\uparrow$}};
\node (Rt) at (0,-0.55) {{\small o}};
\node (reg) at (0.4,0.55) {{\small \textrho}};
\node (tone) at (-0.4,0.55) {{\small \texttau}};
\end{tikzpicture}
\end{minipage}
}

\newcommand{\RegToneBySyl}{
\begin{minipage}{0.06\textwidth}
\begin{tikzpicture}[xscale=0.6,yscale=0.5]
\node[rotate=20] (arrow1) at (-0.15,0) {{\fns $\uparrow$}};
\node[rotate=340] (arrow2) at (0.15,0) {{\fns $\uparrow$}};
\node (Rt) at (0,-0.55) {{\small \textsigma}};
\node (reg) at (0.4,0.55) {{\small \textrho}};
\node (tone) at (-0.4,0.55) {{\small \texttau}};
\end{tikzpicture}
\end{minipage}
}

\newcommand{\DepTone}{
\begin{minipage}{0.055\textwidth}
\begin{tikzpicture}[xscale=0.6,yscale=0.4]
\node (dep) at (0,0) {{\small \textsc{Dep}}};
\node (tone) at (0,-1.0) {{\small \texttau}};
\end{tikzpicture}
\end{minipage}
}

\newcommand{\DepTonalRt}{
\begin{minipage}{0.055\textwidth}
\begin{tikzpicture}[xscale=0.6,yscale=0.4]
\node (dep) at (0,0) {{\small \textsc{Dep}}};
\node (tone) at (0,-1.0) {{\small o}};
\end{tikzpicture}
\end{minipage}
}

\newcommand{\DepL}{
\begin{minipage}{0.055\textwidth}
\begin{tikzpicture}[xscale=0.6,yscale=0.4]
\node (dep) at (0,0) {{\small \textsc{Dep}}};
\node (tone) at (0,-1.0) {{\small L}};
\end{tikzpicture}
\end{minipage}
}

\newcommand{\DepH}{
\begin{minipage}{0.055\textwidth}
\begin{tikzpicture}[xscale=0.6,yscale=0.4]
\node (dep) at (0,0) {{\small \textsc{Dep}}};
\node (tone) at (0,-1.0) {{\small H}};
\end{tikzpicture}
\end{minipage}
}

\newcommand{\NoMultDiff}{{\small *loh}}
\newcommand{\Alt}{{\small \textsc{Alt}}}
\newcommand{\NoSkip}{{\small \scell{\textsc{No}\\\textsc{Skip}}}}


\newcommand{\RegDomRt}{
\begin{minipage}{0.030\textwidth}
\begin{tikzpicture}[xscale=0.6,yscale=0.5]
\node (arrow) at (0,0) {{\fns $\downarrow$}};
\node (Rt) at (0,-0.55) {{\small o}};
\node (reg) at (0,0.55) {{\small \textrho}};
\end{tikzpicture}
\end{minipage}
}

\newcommand{\DepRegRt}{
\begin{minipage}{0.1\textwidth}
\begin{tikzpicture}[xscale=0.6,yscale=0.4]
\node (dep) at (0,0) {{\small \textsc{Dep}}};
\node (arrow) at (0.75,0) {{\tiny $\downarrow$}};
\node (Rt) at (0.75,-0.5) {{\fns o}};
\node (reg) at (0.75,0.5) {{\fns \textrho}};
\end{tikzpicture}
\end{minipage}
}

% unused

\newcommand{\ToneByRt}{
\begin{minipage}{0.05\textwidth}
\begin{tikzpicture}[xscale=0.6,yscale=0.5]
\node (arrow) at (0,0) {{\fns $\uparrow$}};
\node (Rt) at (0,-0.55) {{\small o}};
\node (tone) at (0,0.55) {{\small \texttau}};
\end{tikzpicture}
\end{minipage}
}

\newcommand{\RegByRt}{
\begin{minipage}{0.05\textwidth}
\begin{tikzpicture}[xscale=0.6,yscale=0.5]
\node (arrow) at (0,0) {{\fns $\uparrow$}};
\node (Rt) at (0,-0.55) {{\small o}};
\node (reg) at (0,0.55) {{\small \textrho}};
\end{tikzpicture}
\end{minipage}
}

\newcommand{\ToneDomRt}{
\begin{minipage}{0.05\textwidth}
\begin{tikzpicture}[xscale=0.6,yscale=0.5]
\node (arrow) at (0,0) {{\fns $\downarrow$}};
\node (Rt) at (0,-0.55) {{\small o}};
\node (tone) at (0,0.55) {{\small \texttau}};
\end{tikzpicture}
\end{minipage}
}

% --- OT tableaus --- %

% Sec. 3.2, first tabl.

\newcommand{\OTHLInput}{
\begin{minipage}{0.17\textwidth}
\begin{tikzpicture}[xscale=\myscalex,yscale=\myscaley]
\node (tone) at (2,0) {(= H)};
\node (syl) at (0,0) {\textsigma};
\node (Rt) at (0,1) {o};
\node (H) at (-0.5,2) {H};
\node (R) at (0.5,3) {h};
\node (Rt2) at (1.5,1.0) {o};
%\node (H2) at (1.0,2) {\epen{L}};
\node (R2) at (2.0,3) {\blue{l}};
\draw [thick] (syl.north) -- (Rt.south) ;
\draw [thick] (Rt.north) -- (H.south) ;
\draw [thick] (Rt.north) -- (R.south) ;
\draw [thick] (syl.north) -- (Rt2.south) ;
%\draw [dashed] (Rt2.north) -- (H2.south) ;
%\draw [dashed] (Rt2.north) -- (R2.south) ;
\end{tikzpicture}
\end{minipage}
}

\newcommand{\OTHLWinner}{
\begin{minipage}{0.17\textwidth}
\begin{tikzpicture}[xscale=\myscalex,yscale=\myscaley]
\node (tone) at (2,0) {(= HL)};
\node (syl) at (0,0) {\textsigma};
\node (Rt) at (0,1) {o};
\node (H) at (-0.5,2) {H};
\node (R) at (0.5,3) {h};
\node (Rt2) at (1.5,1.0) {o};
\node (H2) at (1.0,2) {\epen{L}};
\node (R2) at (2.0,3) {\blue{l}};
\draw [thick] (syl.north) -- (Rt.south) ;
\draw [thick] (Rt.north) -- (H.south) ;
\draw [thick] (Rt.north) -- (R.south) ;
\draw [thick] (syl.north) -- (Rt2.south) ;
\draw [dashed] (Rt2.north) -- (H2.south) ;
\draw [dashed] (Rt2.north) -- (R2.south) ;
\end{tikzpicture}
\end{minipage}
}

\newcommand{\OTHLSpreadingHOnly}{
\begin{minipage}{0.17\textwidth}
\begin{tikzpicture}[xscale=\myscalex,yscale=\myscaley]
\node (tone) at (2,0) {(= HM)};
\node (syl) at (0,0) {\textsigma};
\node (Rt) at (0,1) {o};
\node (H) at (-0.5,2) {H};
\node (R) at (0.5,3) {h};
\node (Rt2) at (1.5,1.0) {o};
%\node (H2) at (1.0,2) {\epen{L}};
\node (R2) at (2.0,3) {\blue{l}};
\draw [thick] (syl.north) -- (Rt.south) ;
\draw [thick] (Rt.north) -- (H.south) ;
\draw [thick] (Rt.north) -- (R.south) ;
\draw [thick] (syl.north) -- (Rt2.south) ;
\draw [dashed] (Rt2.north) -- (R2.south) ;
\draw [dashed] (Rt2.north) -- (H.south) ;
\end{tikzpicture}
\end{minipage}
}

\newcommand{\OTHLInsertH}{
\begin{minipage}{0.17\textwidth}
\begin{tikzpicture}[xscale=\myscalex,yscale=\myscaley]
\node (tone) at (2,0) {(= HM)};
\node (syl) at (0,0) {\textsigma};
\node (Rt) at (0,1) {o};
\node (H) at (-0.5,2) {H};
\node (R) at (0.5,3) {h};
\node (Rt2) at (1.5,1.0) {o};
\node (H2) at (1.0,2) {\epen{H}};
\node (R2) at (2.0,3) {\blue{l}};
\draw [thick] (syl.north) -- (Rt.south) ;
\draw [thick] (Rt.north) -- (H.south) ;
\draw [thick] (Rt.north) -- (R.south) ;
\draw [thick] (syl.north) -- (Rt2.south) ;
\draw [dashed] (Rt2.north) -- (H2.south) ;
\draw [dashed] (Rt2.north) -- (R2.south) ;
\end{tikzpicture}
\end{minipage}
}

\newcommand{\OTHLOverwriting}{
\begin{minipage}{0.17\textwidth}
\begin{tikzpicture}[xscale=\myscalex,yscale=\myscaley]
\node (syl) at (0,0) {\textsigma};
\node (Rt) at (0,1) {o};
\node (H) at (-0.5,2) {H};
\node (R) at (0.5,3) {h};
\node (Rt2) at (1.5,1.0) {o};
%\node (H2) at (1.0,2) {\epen{L}};
\node (R2) at (2.0,3) {\blue{l}};
\draw [thick] (syl.north) -- (Rt.south) ;
\draw [thick] (Rt.north) -- (H.south) ;
\draw [thick] (Rt.north) -- (R.south) ;
\draw [thick] (syl.north) -- (Rt2.south) ;
%\draw [dashed] (Rt2.north) -- (H2.south) ;
\draw [dashed] (Rt.north) -- (R2.south) ;
\node (del) at (0.3,1.9) {\textbf{=}};
\end{tikzpicture}
\end{minipage}
}

\newcommand{\OTHLSpreading}{
\begin{minipage}{0.17\textwidth}
\begin{tikzpicture}[xscale=\myscalex,yscale=\myscaley]
\node (syl) at (0,0) {\textsigma};
\node (Rt) at (0,1) {o};
\node (H) at (-0.5,2) {H};
\node (R) at (0.5,3) {h};
\node (Rt2) at (1.5,1.0) {o};
%\node (H2) at (1.0,2) {\epen{L}};
\node (R2) at (2.0,3) {\blue{l}};
\draw [thick] (syl.north) -- (Rt.south) ;
\draw [thick] (Rt.north) -- (H.south) ;
\draw [thick] (Rt.north) -- (R.south) ;
\draw [thick] (syl.north) -- (Rt2.south) ;
%\draw [dashed] (Rt2.north) -- (H2.south) ;
\draw [dashed] (Rt2.north) -- (H.south) ;
\draw [dashed] (Rt2.north) -- (R.south) ;
\end{tikzpicture}
\end{minipage}
}

% Sec. 4.2, second tabl.: phrase-medial position

\newcommand{\OTHnoLInput}{
\begin{minipage}{0.17\textwidth}
\begin{tikzpicture}[xscale=\myscalex,yscale=\myscaley]
\node (tone) at (2,0) {(= H)};
\node (syl) at (0,0) {\textsigma};
\node (Rt) at (0,1) {o};
\node (H) at (-0.5,2) {H};
\node (R) at (0.5,3) {h};
\node (Rt2) at (1.5,1.0) {o};
%\node (H2) at (1.0,2) {\epen{L}};
%\node (R2) at (2.0,3) {\blue{l}};
\draw [thick] (syl.north) -- (Rt.south) ;
\draw [thick] (Rt.north) -- (H.south) ;
\draw [thick] (Rt.north) -- (R.south) ;
\draw [thick] (syl.north) -- (Rt2.south) ;
\end{tikzpicture}
\end{minipage}
}

\newcommand{\OTHnoLEpenth}{
\begin{minipage}{0.17\textwidth}
\begin{tikzpicture}[xscale=\myscalex,yscale=\myscaley]
\node (tone) at (2,0) {(= HM)};
\node (syl) at (0,0) {\textsigma};
\node (Rt) at (0,1) {o};
\node (H) at (-0.5,2) {H};
\node (R) at (0.5,3) {h};
\node (Rt2) at (1.5,1.0) {o};
\node (H2) at (1.0,2) {\epen{L}};
\node (R2) at (2.0,3) {\epen{h}};
\draw [thick] (syl.north) -- (Rt.south) ;
\draw [thick] (Rt.north) -- (H.south) ;
\draw [thick] (Rt.north) -- (R.south) ;
\draw [thick] (syl.north) -- (Rt2.south) ;
\draw [dashed] (Rt2.north) -- (H2.south) ;
\draw [dashed] (Rt2.north) -- (R2.south) ;
\end{tikzpicture}
\end{minipage}
}

\newcommand{\OTHnoLSpreading}{
\begin{minipage}{0.17\textwidth}
\begin{tikzpicture}[xscale=\myscalex,yscale=\myscaley]
\node (tone) at (2,0) {(= HH)};
\node (syl) at (0,0) {\textsigma};
\node (Rt) at (0,1) {o};
\node (H) at (-0.5,2) {H};
\node (R) at (0.5,3) {h};
\node (Rt2) at (1.5,1.0) {o};
%\node (H2) at (1.0,2) {\epen{L}};
%\node (R2) at (2.0,3) {\blue{l}};
\draw [thick] (syl.north) -- (Rt.south) ;
\draw [thick] (Rt.north) -- (H.south) ;
\draw [thick] (Rt.north) -- (R.south) ;
\draw [thick] (syl.north) -- (Rt2.south) ;
\draw [dashed] (Rt2.north) -- (H.south) ;
\draw [dashed] (Rt2.north) -- (R.south) ;
\end{tikzpicture}
\end{minipage}
}

% Sec. 4.2, third tabl., LM is unaffected by L\%

\newcommand{\OTLMInput}{
\begin{minipage}{0.2\textwidth}
\begin{tikzpicture}[xscale=\myscalex,yscale=\myscaley]
\node (tone) at (2,0) {(= LM)};
\node (syl) at (0,0) {\textsigma};
\node (Rt) at (0,1) {o};
\node (H) at (-0.5,2) {L};
\node (R) at (0.5,3) {l};
\node (Rt2) at (1.5,1.0) {o};
\node (H2) at (1.0,2) {L};
\node (R2) at (2.0,3) {h};
\node (R3) at (3.0,3) {\blue{l}};
\draw [thick] (syl.north) -- (Rt.south) ;
\draw [thick] (Rt.north) -- (H.south) ;
\draw [thick] (Rt.north) -- (R.south) ;
\draw [thick] (syl.north) -- (Rt2.south) ;
\draw [thick] (Rt2.north) -- (H2.south) ;
\draw [thick] (Rt2.north) -- (R2.south) ;
\end{tikzpicture}
\end{minipage}
}

\newcommand{\OTLMReplace}{
\begin{minipage}{0.2\textwidth}
\begin{tikzpicture}[xscale=\myscalex,yscale=\myscaley]
\node (tone) at (2,0) {(= LL)};
\node (syl) at (0,0) {\textsigma};
\node (Rt) at (0,1) {o};
\node (H) at (-0.5,2) {L};
\node (R) at (0.5,3) {l};
\node (Rt2) at (1.5,1.0) {o};
\node (H2) at (1.0,2) {L};
\node (R2) at (2.0,3) {h};
\node (R3) at (3.0,3) {\blue{l}};
\draw [thick] (syl.north) -- (Rt.south) ;
\draw [thick] (Rt.north) -- (H.south) ;
\draw [thick] (Rt.north) -- (R.south) ;
\draw [thick] (syl.north) -- (Rt2.south) ;
\draw [thick] (Rt2.north) -- (H2.south) ;
\draw [thick] (Rt2.north) -- (R2.south) ;
\draw [dashed] (Rt2.north) -- (R3.south) ;
\node (del) at (1.8,2.1) {\textbf{=}};
\end{tikzpicture}
\end{minipage}
}

\newcommand{\OTLMTwoReg}{
\begin{minipage}{0.2\textwidth}
\begin{tikzpicture}[xscale=\myscalex,yscale=\myscaley]
\node (tone) at (2,0) {(= LML)};
\node (syl) at (0,0) {\textsigma};
\node (Rt) at (0,1) {o};
\node (H) at (-0.5,2) {L};
\node (R) at (0.5,3) {l};
\node (Rt2) at (1.5,1.0) {o};
\node (H2) at (1.0,2) {L};
\node (R2) at (2.0,3) {h};
\node (R3) at (3.0,3) {\blue{l}};
\draw [thick] (syl.north) -- (Rt.south) ;
\draw [thick] (Rt.north) -- (H.south) ;
\draw [thick] (Rt.north) -- (R.south) ;
\draw [thick] (syl.north) -- (Rt2.south) ;
\draw [thick] (Rt2.north) -- (H2.south) ;
\draw [thick] (Rt2.north) -- (R2.south) ;
\draw [dashed] (Rt2.north) -- (R3.south) ;
\end{tikzpicture}
\end{minipage}
}

% Sec. 4.2, fourth tabl., L is affected by L\% but M is not

\newcommand{\OTLInput}{
\begin{minipage}{0.17\textwidth}
\begin{tikzpicture}[xscale=\myscalex,yscale=\myscaley]
\node (tone) at (2,0) {(= L)};
\node (syl) at (0,0) {\textsigma};
\node (Rt) at (0,1) {o};
\node (H) at (-0.5,2) {L};
\node (R) at (0.5,3) {l};
\node (R2) at (2,3) {\blue{l}};
\draw [thick] (syl.north) -- (Rt.south) ;
\draw [thick] (Rt.north) -- (H.south) ;
\draw [thick] (Rt.north) -- (R.south) ;
\end{tikzpicture}
\end{minipage}
}

\newcommand{\OTLLowered}{
\begin{minipage}{0.17\textwidth}
\begin{tikzpicture}[xscale=\myscalex,yscale=\myscaley]
\node (tone) at (2,0) {(= LL)};
\node (syl) at (0,0) {\textsigma};
\node (Rt) at (0,1) {o};
\node (H) at (-0.5,2) {L};
\node (R) at (0.5,3) {l};
\node (R2) at (2,3) {\blue{l}};
\draw [thick] (syl.north) -- (Rt.south) ;
\draw [thick] (Rt.north) -- (H.south) ;
\draw [thick] (Rt.north) -- (R.south) ;
\draw [dashed] (Rt.north) -- (R2.south) ;
\end{tikzpicture}
\end{minipage}
}

\newcommand{\OTMInput}{
\begin{minipage}{0.17\textwidth}
\begin{tikzpicture}[xscale=\myscalex,yscale=\myscaley]
\node (tone) at (2,0) {(= M)};
\node (syl) at (0,0) {\textsigma};
\node (Rt) at (0,1) {o};
\node (H) at (-0.5,2) {L};
\node (R) at (0.5,3) {h};
\node (R2) at (2,3) {\blue{l}};
\draw [thick] (syl.north) -- (Rt.south) ;
\draw [thick] (Rt.north) -- (H.south) ;
\draw [thick] (Rt.north) -- (R.south) ;
\end{tikzpicture}
\end{minipage}
}

\newcommand{\OTMLowered}{
\begin{minipage}{0.17\textwidth}
\begin{tikzpicture}[xscale=\myscalex,yscale=\myscaley]
\node (tone) at (2,0) {(= ML)};
\node (syl) at (0,0) {\textsigma};
\node (Rt) at (0,1) {o};
\node (H) at (-0.5,2) {L};
\node (R) at (0.5,3) {h};
\node (R2) at (2,3) {\blue{l}};
\draw [thick] (syl.north) -- (Rt.south) ;
\draw [thick] (Rt.north) -- (H.south) ;
\draw [thick] (Rt.north) -- (R.south) ;
\draw [dashed] (Rt.north) -- (R2.south) ;
\end{tikzpicture}
\end{minipage}
}

% Sec. 4.2, fifth tableau, polar questions with level tones

\newcommand{\OTLPolIn}{
\begin{minipage}{0.20\textwidth}
\begin{tikzpicture}[xscale=\myscalex-0.05,yscale=\myscaley-0.05]
\node (tone) at (3.5,0) {(= L)};
\node (syl) at (0,0) {\textsigma};
\node (syl2) at (2,0) {\red{\textsigma}};
\node (Rt) at (0,1) {o};
\node (H) at (-0.5,2) {L};
\node (R) at (0.5,3) {l};
\node (Rt2) at (2,1) {\red{o}};
\draw [thick] (syl.north) -- (Rt.south) ;
\draw [thick,red] (syl2.north) -- (Rt2.south) ;
\draw [thick] (Rt.north) -- (H.south) ;
\draw [thick] (Rt.north) -- (R.south) ;
\end{tikzpicture}
\end{minipage}
}

\newcommand{\OTLPolDef}{
\begin{minipage}{0.20\textwidth}
\begin{tikzpicture}[xscale=\myscalex-0.05,yscale=\myscaley-0.05]
\node (tone) at (3.5,0) {(= L.M)};
\node (syl) at (0,0) {\textsigma};
\node (syl2) at (2,0) {\red{\textsigma}};
\node (Rt) at (0,1) {o};
\node (H) at (-0.5,2) {L};
\node (R) at (0.5,3) {l};
\node (H2) at (1.5,2) {\epen{L}};
\node (R2) at (2.5,3) {\epen{h}};
\node (Rt2) at (2,1) {\red{o}};
\draw [thick] (syl.north) -- (Rt.south) ;
\draw [thick,red] (syl2.north) -- (Rt2.south) ;
\draw [thick] (Rt.north) -- (H.south) ;
\draw [thick] (Rt.north) -- (R.south) ;
\draw [semithick,dashed] (Rt2.north) -- (H2.south) ;
\draw [semithick,dashed] (Rt2.north) -- (R2.south) ;
\end{tikzpicture}
\end{minipage}
}

\newcommand{\OTLPolAlt}{
\begin{minipage}{0.20\textwidth}
\begin{tikzpicture}[xscale=\myscalex-0.05,yscale=\myscaley-0.05]
\node (tone) at (3.5,0) {(= L.L)};
\node (syl) at (0,0) {\textsigma};
\node (syl2) at (2,0) {\red{\textsigma}};
\node (Rt) at (0,1) {o};
\node (H) at (-0.5,2) {L};
\node (R) at (0.5,3) {l};
\node (Rt2) at (2,1) {\red{o}};
\draw [thick] (syl.north) -- (Rt.south) ;
\draw [thick,red] (syl2.north) -- (Rt2.south) ;
\draw [thick] (Rt.north) -- (H.south) ;
\draw [thick] (Rt.north) -- (R.south) ;
\draw [semithick,dashed] (Rt2.north) -- (H.south) ;
\draw [semithick,dashed] (Rt2.north) -- (R.south) ;
\end{tikzpicture}
\end{minipage}
}

% Sec. 4.2, sixth tableau, polar questions with contour tones

\newcommand{\OTLLPolIn}{
\begin{minipage}{0.23\textwidth}
\begin{tikzpicture}[xscale=\myscalex-0.05,yscale=\myscaley-0.05]
\node (tone) at (5.2,0) {(= L)};
\node (syl) at (0,0) {\textsigma};
\node (syl3) at (3.4,0) {\red{\textsigma}};
\node (Rt) at (0,1) {o};
\node (Rt2) at (1.7,1) {o};
\node (Rt3) at (3.4,1) {\red{o}};
\node (H) at (-0.5,2) {L};
\node (R) at (0.5,3) {l};
\draw [thick] (syl.north) -- (Rt.south) ;
\draw [thick] (syl.north) -- (Rt2.south) ;
\draw [thick,red] (syl3.north) -- (Rt3.south) ;
\draw [thick] (Rt.north) -- (H.south) ;
\draw [thick] (Rt.north) -- (R.south) ;
\end{tikzpicture}
\end{minipage}
}

\newcommand{\OTLLPolDef}{
\begin{minipage}{0.23\textwidth}
\begin{tikzpicture}[xscale=\myscalex-0.05,yscale=\myscaley-0.05]
\node (tone) at (5.2,0) {(= L.M)};
\node (syl) at (0,0) {\textsigma};
\node (syl3) at (3.4,0) {\red{\textsigma}};
\node (Rt) at (0,1) {o};
\node (Rt2) at (1.7,1) {o};
\node (Rt3) at (3.4,1) {\red{o}};
\node (H) at (-0.5,2) {L};
\node (R) at (0.5,3) {l};
\node (H3) at (2.9,2) {\epen{L}};
\node (R3) at (3.9,3) {\epen{h}};
\draw [thick] (syl.north) -- (Rt.south) ;
\draw [thick] (syl.north) -- (Rt2.south) ;
\draw [thick,red] (syl3.north) -- (Rt3.south) ;
\draw [thick] (Rt.north) -- (H.south) ;
\draw [thick] (Rt.north) -- (R.south) ;
\draw [dashed] (Rt3.north) -- (H3.south) ;
\draw [dashed] (Rt3.north) -- (R3.south) ;
\end{tikzpicture}
\end{minipage}
}

\newcommand{\OTLLPolSkip}{
\begin{minipage}{0.23\textwidth}
\begin{tikzpicture}[xscale=\myscalex-0.05,yscale=\myscaley-0.05]
\node (tone) at (5.2,0) {(= L.L)};
\node (syl) at (0,0) {\textsigma};
\node (syl3) at (3.4,0) {\red{\textsigma}};
\node (Rt) at (0,1) {o};
\node (Rt2) at (1.7,1) {o};
\node (Rt3) at (3.4,1) {\red{o}};
\node (H) at (-0.5,2) {L};
\node (R) at (0.5,3) {l};
\draw [thick] (syl.north) -- (Rt.south) ;
\draw [thick] (syl.north) -- (Rt2.south) ;
\draw [thick,red] (syl3.north) -- (Rt3.south) ;
\draw [thick] (Rt.north) -- (H.south) ;
\draw [thick] (Rt.north) -- (R.south) ;
\draw [dashed] (Rt3.north) -- (H.south) ;
\draw [dashed] (Rt3.north) -- (R.south) ;
\end{tikzpicture}
\end{minipage}
}  
  
\newcommand{\ilit}[1]{#1\il{#1}}    
\newcommand{\isit}[1]{#1\is{#1}}  

\makeatletter
\let\thetitle\@title
\let\theauthor\@author 
\makeatother

\newcommand{\togglepaper}[1][0]{ 
  \bibliography{../localbibliography}
  %% hyphenation points for line breaks
%% Normally, automatic hyphenation in LaTeX is very good
%% If a word is mis-hyphenated, add it to this file
%%
%% add information to TeX file before \begin{document} with:
%% %% hyphenation points for line breaks
%% Normally, automatic hyphenation in LaTeX is very good
%% If a word is mis-hyphenated, add it to this file
%%
%% add information to TeX file before \begin{document} with:
%% \include{localhyphenation}
\hyphenation{
affri-ca-te
affri-ca-tes
com-ple-ments
par-a-digm
Sha-ron
Kings-ton
phe-nom-e-non
Daul-ton
Abu-ba-ka-ri
Ngo-nya-ni
Clem-ents 
King-ston
Tru-cken-brodt
Tab-leau
cophono-logies
mark-edness
Ti-gri-nya
a-mong
Car-stens
Lu-bu-ku-su
}
\hyphenation{
affri-ca-te
affri-ca-tes
com-ple-ments
par-a-digm
Sha-ron
Kings-ton
phe-nom-e-non
Daul-ton
Abu-ba-ka-ri
Ngo-nya-ni
Clem-ents 
King-ston
Tru-cken-brodt
Tab-leau
cophono-logies
mark-edness
Ti-gri-nya
a-mong
Car-stens
Lu-bu-ku-su
}
  \papernote{\scriptsize\normalfont
    \theauthor.
    \thetitle. 
    To appear in: 
    Emily Clem,   Peter Jenks \& Hannah Sande.
    Theory and description in African Linguistics: Selected papers from the 47th Annual Conference on African Linguistics.
    Berlin: Language Science Press. [preliminary page numbering]
  }
  \pagenumbering{roman}
  \setcounter{chapter}{#1}
  \addtocounter{chapter}{-1}
}

\newcommand{\upstep}{\textupstep}


% \newcounter{tableauxcounter}

\renewcommand{\textltailn}{ɲ}
\renewcommand{\textbardotlessj}{ɟ}

\newcommand{\emphkh}[1]{\textit{#1}} %originally \textbf, banned by the guidelines



\definecolor{lsDOIGray}{cmyk}{0,0,0,0.45}


\newcommand{\xuparrow}[1]{%
  {\left\uparrow\vbox to #1{}\right.\kern-\nulldelimiterspace}
}
\renewcommand \textupstep[1]{\char"A71B#1}
\renewcommand \textdownstep[1]{\char"A71C#1}
 
 \newcommand{\ꜛ}{\textsf{ꜛ}}
 
\def\biberror{\undefined}


\newcommand{\OTbox}[1]{\resizebox{.88\textwidth}{!}{#1}}
 
  \togglepaper[30]
}{}



\abstract{A corpus of WhatsApp chats reveals how Hausa-speaking youth have adopted and spread homegrown Hausa terms, via semantic extension, for the actions (e.g. chatting, forwarding), objects (e.g. image) and space (e.g. group, online/offline) associated with computer-mediated communication rather than strictly borrowing from English chat jargon.  Along with other contextual factors, this study reviews the linguistic forms (including source language), range of terminology, and frequency of occurrence of specialized chat terminology found in this corpus, representing 56 different interlocutors in 40 different dyads of chat excerpts.}


\begin{document}

\maketitle

\section{Introduction and background}

This study analyzes the vocabulary that \ili{Hausa}-speaking chat participants adopt when consciously referring to the chat environment itself. In particular, I analyze the extent to which chatters either draw on \ili{English}-based chat jargon or employ equivalent \ili{Hausa} terms for this purpose. Observations are drawn from a freshly developed corpus of WhatsApp chats between \ili{Hausa} speakers. The corpus includes 40 different dyads of chats involving 56 different interlocutors. Sixty terms (lemma), including 22 inherent \ili{Hausa} items and 38 instances of \ili{English} loanwords or code-mixing, were tracked as terms used in reference to the actions (e.g. \textit{chat(ting)}, \textit{forward(ing)}), objects (e.g., \textit{image}), and space (e.g. \textit{group}, \textit{online/offline}). Results reveal members of the \ili{Hausa}-speaking community to be quite innovative when it comes to drawing on their language’s own lexical resources for use as chat terminology rather than strictly borrowing from popularly known \ili{English} chat jargon.

\section{Background}
\subsection{Increasingly multilingual cyberspace}

\ili{English} has long been recognized as the dominant, established lingua franca of the Internet \citep{danet2007} as well as SMS communication. Nonetheless, as smartphones and wireless technology spread to the remotest areas of the world, more and more languages have been adapted for computer-mediated communication (CMC), and by now the Internet and cybersphere can truly be recognized as a relatively diversified, multilingual environment. 

But what does it take to truly adapt to this medium? To the extent that online chat and SMS messaging, presumably the most widely used applications of CMC, are similar to spoken conversation, one might think that adapting to the new technology is a simple matter of typing words as they are spoken. However, this naturally comes with various challenges, and I would argue the outcome is that \ili{English}’s influence in computer-mediated communication is partly reinforced by these obstacles. 

First of all, of course, users must be literate and share some basic standards or \isi{common ground} of orthographic conventions with their interlocutors. For languages lacking an established literate tradition, bilingual speakers may end up preferring to use \ili{English}, thus reinforcing the continued dominance of \ili{English} as the language of the Internet. For example, when recruiting contributors for the corpus of \ili{Hausa}-based texts presented in this paper, numerous fluent, mother-tongue speakers of \ili{Hausa} who otherwise use \ili{Hausa} frequently in various spoken contexts admitted that they tended to text in \ili{English}, not \ili{Hausa}. Likewise, from among those who agreed to participate, several contributions for the corpus building were rejected on the grounds that the majority of texting was in \ili{English}. 

Furthermore, languages using non-\ili{Latin} scripts face challenges. Although Internet and cell-phone technology has accommodated different language scripts, we still find users adapting their native language to \ili{Latin} scripts. For example, “Greeklish” is a \ili{Latin} script-based rendering of \ili{Greek} that was developed as soon as Internet came to \ili{Greek} society \citep{androutsopoulos2012}. Similarly, \citet{palfreyman2007} have studied the use of a so-called “ASCII-ized \ili{Arabic}” — where \ili{Latin} characters along with numerals and other symbols represent different \ili{Arabic} letters — among college students in UAE. So, even though the language of communication may not be \ili{English}, the implicit hegemony of \ili{English} as the language of the Internet is still reflected in the choice of script. 

\largerpage 
Third, in the online chat environment at least, it is desirable to express oneself as rapidly as possible. This is largely facilitated by the development of abbreviated forms such as the iconic trends seen in the \ili{English}-speaking world of CMC with phrases like \textit{y r u so l8} (in place of the 15-character phrase \textit{Why are you so late?}). While any given language can be used for online chatting without such abbreviations, certain bilingual speakers again might opt for \ili{English} as the language that gives them a ready-made, established medium for rapid, not to mention playful, communication. 

\subsection{Chat jargon (terminology)}

Even where a language has successfully adapted to the CMC environment, there is yet another area where one might expect to see remnant signs of the dominance of \ili{English} as the global language of technology — namely, in the use of specialized chat terminology. Though meant to mirror in many ways spoken conversation, chatters must on occasion refer to actions, objects, and space that are unique to the computer-mediated medium. In fact, presence in the chat environment often serves as a topic of conversation, as chatters make reference to \textit{profile pictures} that they have \textit{uploaded} to their \textit{account} and request one another \textit{forward} \textit{snapshots}, for example. Thus, inevitably, chat participants will have a need and desire for jargon of this nature for conscious reference to the virtual electronic environment itself — terms like \textit{email}, \textit{attachment}, \textit{profile}, \textit{upload}, and \textit{online} found in \ili{English}. 

With such chat terminology logically taking cues from the field of information technology and with online chat being a product of globalization in its own right, one might expect, to begin with, bilingual chatters to resort to code-mixing in \ili{English} (as the dominant language of globalization and IT). Furthermore, even monolingual chatters would be influenced by the multilingual community, and languages might fully adopt (borrow/code-mix) \ili{English}-based loanwords for such terms as \textit{chat}, \textit{forward}, and \textit{online}.

Indeed, technical communication is often cited among the motivations for code-switching (bilingual speakers switching back and forth between different languages) and among bases for code-mixing (i.e., linguistic borrowing). In general, technological terms, such as these, are prone to spread from the originating or dominant language to other cultures where they get adopted as loanwords. For example, when checking for translation equivalents for the word \textit{computer} in Google Translate, 76\% (77 of 101) of the languages supported present a word that is clearly derived from the \ili{Latin}-cum-\ili{English} term. \citet{daulton2012} further confirms that “the most borrowed words refer to technology (e.g. engine) and names for new artifacts (e.g. taxi).”

\subsection{Alternatives to English loanwords}

The use of chat jargon might be inevitable, but the spread of terminology as loanwords is not. After all, the \ili{English} language itself has drawn on various word-building strategies in the development of jargon dealing with computer technology — from reviving an old term like \textit{cursor} (which itself had been borrowed earlier from \ili{Latin} like so many \ili{English} words) to repurposing common words like \textit{mouse} and \textit{web} via \isi{semantic extension} to use of acronyms like \textit{PC}. Similarly, other languages can draw on their own resources.

In many cases, when languages are found using intrinsic strategies for technological lexical development, it is understood as a conscious effort to defend linguistic purity \citep{blommaert2002,haspelmath2009}. For example, the Académie française has long been active with moderating the development and documentation of new \ili{French} terms, with moderate success thanks to government backing in matters of broadcasting and publication. Examples include recommending the use of \textit{logiciel} and \textit{courriel} in place of \textit{software} and \textit{e-mail} \citep{daulton2012}. Similar efforts for linguistic purification can be seen with other languages of the world such as \ili{Korean} and various Eastern European languages \citep{haspelmath2009}.

\subsection{Hausa}

\ili{Hausa}, an Afro-asiatic language spoken widely in West Africa, is an example of a language that has successfully been adapted for CMC. For one thing it does have an established, printed literary tradition using a \ili{Latin}-based script.\footnote{Although the \ili{Latin}-based script was only introduced early in the 20th century, it has overtaken Ajami (an \ili{Arabic}-based script whose use with \ili{Hausa} dates back to the 15th century) as the dominant orthographic standard.}  While many speakers might not be familiar with official standards of orthography, they get by well enough with predictable pronunciation and influence from mixed levels of literacy in \ili{English}. Secondly, regarding the desire for rapid communication, within the corpus of \ili{Hausa} chats described in this article, the \ili{Hausa} speakers do collectively use a variety of abbreviated forms such as \textit{wlh} for \textit{wallahi} (‘by God’) and \textit{ya kk} for \textit{yaya kake/kike/kuke} (‘How are you?’ – covering \isi{masculine}, \isi{feminine}, and plural forms of second-person reference in \ili{Hausa} grammar).

But what about specialized chat terminology in \ili{Hausa}? Returning to the discussion in the preceding section, we can first observe that the \ili{Hausa} community is not documented as one that is prone to language purification efforts. First of all, the \ili{Hausa} language has frequently drawn upon languages in contact for expanding its lexicon. For example, words like \textit{burodi} (‘bread’), \textit{tebur} (‘table’), and \textit{famfo} (‘pump’) have come from \ili{English}, while terms like \textit{albarka} (‘blessing’), \textit{hankali} (‘wisdom’), and \textit{wallahi} (‘by God’) come from \ili{Arabic}. Some words traced to these two languages were transmitted to \ili{Hausa} via yet other languages — such as \textit{tasha} (‘station’) coming into \ili{Hausa} from \ili{Yoruba} (or possibly other languages spoken south of \ili{Hausa} speaking areas) and \textit{kasuwa} (‘market’) having been introduced via another language of northern Nigeria, \ili{Kanuri}, which had its own \isi{lexical borrowing} from the \ili{Arabic} word \textit{suq} \citep{newman2000comp}. Furthermore, and more directly relevant to this study, many of the \ili{Hausa} speakers in the \ili{Hausa} chat corpus (all bilingual) frequently code-switch between \ili{Hausa} and \ili{English} (and less frequently, \ili{Arabic}, Fulfulde, and \ili{Kanuri}) in addition to using \ili{English} code-mixing within \ili{Hausa} texts. That is, on average, they are clearly not inclined towards so-called linguistic purity.

So, as a language open to \isi{lexical borrowing}, one might expect these bilingual chatters to naturally draw on established \ili{English} terms for chat jargon. Indeed, many do draw on \ili{English} both for emotive jargon (like 206 instances of \textit{lol} and 3 instances of \textit{l8r} ‘later’), which is not analyzed in this study, and for the specialized terminology referring to the chat environment, analyzed in this paper. Yet, interestingly, within this relatively new and modern medium, young \ili{Hausa} speakers appear to have spontaneously adopted and spread numerous homegrown terms, via \isi{semantic extension} or metaphor, for the actions or processes (e.g. chatting, forwarding), objects (e.g. image) and space (e.g. group, online/offline) associated with phone-based and Internet-based communication. \ili{Hausa} still shows itself to be a language with robust \isi{semantic extension}, among other strategies for lexical expansion.

\section{Methodology}\label{sec:purvis:3}
\subsection{Corpus development}
\subsubsection{Data collection}

The corpus was originally targeted as a database of SMS texts with the goal of collecting a minimum of 60 texts from at least 50 participants.\footnote{This objective came from University of Maryland Center for Advanced Study of Language (CASL), who conceived of and funded the project.} WhatsApp chats were ultimately adopted with the following justification:

\begin{itemize}
\item more widely used for extended communication than SMS; 

\item more practical to collect;

\item largely comparable to SMS texting in form and context.
\end{itemize}

University students and some other community members shared excerpts of chats for which their interlocutors also agreed for the texts to be used in the database. To meet the originally targeted volume of data, chats were collected such that the contribution from each participant was at least 4200 characters (based on an estimated average SMS length of 70 characters) — although for 6 additional participants included in the study the volume of texts fell short of 4200 characters. At the time of this study, the corpus included 56 participants (representing excerpts for 40 conversations between two individuals). The total volume of the corpus has reached 21,693 lines (about 90,000 words or 380,000 characters).

A short survey of sociolinguistic\slash contextual information was collected for\linebreak each participant, the details of which are summarized in \tabref{tab:purvis:1}. As can be seen from these demographics, the majority of participants are university students (85.7\%) in their early 20s. Although some claim a language other than \ili{Hausa} as their mother tongue, a majority (48.2\%) consider \ili{Hausa} as their mother tongue, and all are fluent in \ili{Hausa}. In addition to the details shown in \tabref{tab:purvis:1}, all participants claim to speak \ili{English}, with a handful of them claiming fluency in other languages as well. As noted earlier, the participants are all bilingual, essentially fluent speakers of both \ili{Hausa} and \ili{English} (Nigerian standard, which is largely based on British standard).

\begin{table}[p]
\footnotesize
\begin{tabularx}{\textwidth}{lQ}
\lsptoprule
 Factor &  Details\\
\midrule
Gender: & 24 Females, 32 Males\\
\tablevspace
Age: & Average=22; Mode=20; Range of 14-35\\
\tablevspace
Education: & Mostly undergraduate; but range from H.S. to Masters\\
\tablevspace
Occupation: & Student (48); Teacher/Lecturer (2); Nurse (1); Entrepreneur (1); Music producer/singer (1); Film maker (1); Music artist (1); Unemployed (2)\\
\tablevspace
Origin(/birthplace): & Adamawa 10; Borno 1 (5); Gombe 2 (1); Jigawa 2(1); Kaduna 4 (5); Kano 20 (19); Katsina 7; Kogi 0 (1); Niger (1); Sokoto 1 (0); Taraba 2(1); Yobe 6 (5)\\
\tablevspace
Residence: & Adamawa 22; Borno 2; Gombe 1; Jigawa 2; Kaduna 6; Kano 10; Katsina 4; Yobe 4; Sudan 2\\
\tablevspace
Mother tongue: & \ilit{Hausa} (27), Fulfulde (16), \ilit{Kanuri} (6), \ilit{Yoruba} (1), Margi (1) , \ilit{Nupe} (1), Other: 5\\
\tablevspace
Language spoken at home: & \ilit{Hausa} (45), Fulfulde (9), \ilit{English} (1), \ilit{Yoruba} (1), \ilit{Kanuri} (2)\\
\tablevspace
Relationship to interlocutor: & (Close/Best/Family) Friend 29, Brother 3, Sister 3, Cousin 3, Uncle 1, Colleague 3\\
\lspbottomrule
\end{tabularx}  
\caption{Chat participant demographics.}
\label{tab:purvis:1}
\end{table} 

\begin{figure}[p]
\includegraphics[width=.8\textwidth]{figures/purvis-fig1.png}
\caption{Data annotation example}
\label{fig:purvis:1}
\end{figure}

  
\subsubsection{Data processing}

Each line of chat was annotated for standardized spelling, word translation, parts-of-speech, language (in case of code-switching) and a free translation of entire comment. This was facilitated through the use of the Linguist’s Toolbox (SIL), as illustrated in \figref{fig:purvis:1}.

 

 

\subsection{Data preparation}

To analyze the use of chat jargon, Search \& Replace software (Funduc, Inc.) using Regular Expressions scripts was used to search for targeted keywords dealing with the chat environment and presumed to be potential candidates for chat terminology used by this \isi{speech community}. An example of such a word appears in \figref{fig:purvis:1}: \textit{sauka} (a \ili{Hausa} verb that literally means ‘to descend or get down,’ and which has been extended to refer to ‘logging off or going offline’). In order to achieve a relatively exhaustive list of appropriate terms, \ili{English} equivalents of common chat terms were also searched in the translation field. The set of words ultimately included in the study (i.e., for which at least 1 instance was found to occur in the texts) is presented in \tabref{tab:purvis:2}. As seen in the table, jargon was categorized by field of use (Theme group) to help track patterns of choice between \ili{Hausa} terms and \ili{English} code-mixing or code-switching. 

\begin{table}
\begin{tabularx}{\textwidth}{lQ}
\lsptoprule
{ Theme group} &  Jargon terms\\
\midrule
Group A (‘talk’): & {\textit{chat(ting)}, \textit{gist} (Nigerian \ili{English} term for casual/playful chat), \textit{talk(ing)}, \textit{hira}, \textit{magana}, \textit{surutu}, \textit{taɗi}, \textit{[kuke] whatsapp)}}\\
\tablevspace
Group B (‘message’): & {\textit{answer}, \textit{comment}, \textit{link}, \textit{mail}, \textit{message}, \textit{reply(ing)}, \textit{respond(ing)/response}, \textit{text}, \textit{ping}, \textit{amsa(wa)}, \textit{saƙo}, \textit{taɓa(wa)}}\\
Group C (‘send’): & {\textit{email}, \textit{forward(ing)}, \textit{send(ing)}, \textit{transfer(ing)}, \textit{tura(wa)}, \textit{turo(wa)}}\\
\tablevspace
Group D (‘file operations’): & {\textit{attach(ing/ment)}, \textit{copy(ing), download(ing), screenshot}, \textit{snapping}, \textit{delete}, \textit{saving}, \textit{goge(wa)}}\\
\tablevspace
Group E (‘image’): & {\textit{image}, \textit{(display/profile) picture} (\textit{dp}/\textit{pp, pic/pix}), \textit{photo}, \textit{hoto}}\\
\tablevspace
Group F (‘post’): & {\textit{post(ing)}, \textit{upload(ing)}, \textit{sa/saka(wa)}}\\
\tablevspace
Group G (‘enter’): & {\textit{enter}, \textit{launch}, \textit{buɗe(wa)}, \textit{shiga}}\\
\tablevspace
Group H (‘online/offline’): & {\textit{offline}, \textit{online}, \textit{[tana] on}, \textit{fita}, \textit{hau/hawa}, \textit{sauka}}\\
\tablevspace
Group I (‘group’): & {\textit{account}, \textit{group}, \textit{shafuffukan yaɗa zumunta}, \textit{azure}}\\
\tablevspace
Group J (‘Internet’) & {\textit{Internet}, \textit{network}, \textit{website}, \textit{yanar gizo-gizo}}\\
\lspbottomrule
\end{tabularx} 
\caption{List of words tracked (that appear in the corpus)\\{}[See Appendix A for brief translations of Hausa terms]}
\label{tab:purvis:2}
\end{table} 
 

A total of 1655 instances of the targeted terms were found to occur in the \ili{Hausa} chat database. This initial tally included all instances, whether used as specialized chat terminology or polysemous terms used in other senses (as in an \ili{English} chatter referring to an actual spider web or a web of lies as opposed to the [world wide] web.) Although the Toolbox software used for initial data entry and processing has a concordancing feature, this was not a practical means to complete the next step of data processing — to verify which instances of targeted words were actually used as chat jargon as opposed to other senses (e.g. \textit{sauka} meaning ‘to get off a bus’ versus \textit{sauka} meaning ‘to go offline’). A simple means to facilitate this task, allowing \ili{English} translations to be viewed alongside the original contextual occurrences in \ili{Hausa}, was to import the corpus into an Excel spreadsheet (as illustrated in \figref{fig:purvis:2}).

  

\begin{figure}
\includegraphics[width=\textwidth]{figures/purvis-fig2.png}
\caption{Excel table used to verify chat jargon usage}
\label{fig:purvis:2}
\end{figure} 

Each occurrence of the targeted terms was tagged for the following contextual features: (1) Usage \& language choice (\ili{Hausa} chat jargon versus other use of \ili{Hausa} term, and \ili{English} loanword versus \ili{English} term used in full instance of code-switching); (2) part-of-speech (Noun, Verb, Gerund/Verbal-noun, Adjective); (3) field of use (Action, Object, Space); (4) \ili{Hausa} suffixes appearing on words; and (5) whether or not the instance was a typo, correction, or immediate repetition of a previous instance. 

\section{Results}\label{sec:purvis:4}
\subsection{Tally of chat jargon terms}

Of the 1655 instances of the target terms, 824 were identified as being used as chat jargon within \ili{Hausa} texts. The remaining instances were excluded on one of the following grounds: (a) the term was not used as a chat term in the particular context (for example, as in the literal use of \textit{sauka} in the sense of ‘to descend or alight’ — as opposed to going offline — as seen in the first two lines of \figref{fig:purvis:2} presented earlier), (b) the term appeared in a full instance of code-switching — i.e., a text entirely or predominantly expressed in \ili{English} or, more rarely, some other language, (c) the term appeared as a correction to a typing error (thus already counted in an immediately preceding instance). Tables \ref{tab:purvis:3}--\ref{tab:purvis:12} present the results of these tallies for each of the 10 theme groups. Each group is presented and discussed in turn.

\subsection{Group A: ‘Talk’}

Admittedly, the notion of \textit{chat} or \textit{talk} is a relatively problematic theme to track as a jargon term since communication (and thus terms referring to verbal exchange) is a natural part of the chat environment. In any case, as seen in \tabref{tab:purvis:3}, the \ili{Hausa} chatters in this corpus draw predominantly on \ili{Hausa} vocabulary — using \ili{Hausa} terms over twice as frequently as corresponding loanwords from \ili{English}.

\begin{table}
\begin{tabularx}{\textwidth}{Qrr}
\lsptoprule
 Word & \multicolumn{1}{c}{Total instances} &  \multicolumn{1}{r}{Used as jargon in Hausa}\\
\midrule
\textit{chat}     & 54 (16.5\%) & 39   (19.8\%)\\
\textit{chatting} & 23 (7\%)    & 22   (11.2\%)\\
\textit{gist}     & 4 (1.2\%)   & 0   (0.0\%)\\
\textit{talk}     & 14 (4.3\%)  & 0   (0.0\%)\\
\textit{talking}  & 1 (0.3\%)   & 0   (0.0\%)\\
\textit{[kuke] whatsapp} (‘you guys are on WhatsApp’) & 1 (0.3\%) & 1  (0.5\%)\\\midrule
& & N=62   (31.5\%)\\\midrule
\textit{hira} (‘chat’; lit. ‘gist, informal chat of the evening’) & 48 (14.7\%)  & 41 (20.8\%)\\
\textit{magana} (‘talk, chat’; lit. ‘talking, matter, issue’)     & 160 (48.9\%) & 80 (40.6\%)\\
\textit{surutu} (‘chatting’)                                      & 6 (1.8\%)    & 2  (1.0\%)\\
\textit{taɗi} (‘chatting’)                                        & 14 (4.3\%)   & 12 (6.1\%)\\
\textit{zance} (‘talk, chat’)                                     & 2 (0.6\%)    & 0  (0.0\%)\\\midrule
& & N=135  (68.5\%)\\
\lspbottomrule
\end{tabularx} 
\caption{Frequency of occurrence for words in Group A: ‘Talk’}
\label{tab:purvis:3}
\end{table} 

The frequency of use of these \ili{Hausa} terms might actually be a bit higher than what is represented here. I was relatively conservative in inclusion of instances of the word \textit{magana}, which can carry the sense of ‘matter, issue’ in addition to ‘talk, discussion’ (the latter often in combination with the verb \textit{yi} (‘do’)). Where the interpretation wasn’t clear, I treated it as ‘matter’ and excluded it from the chat jargon tally. Though appearing less frequently than \textit{magana} overall, the word \textit{hira} comes across as the principle \ili{Hausa} word used as jargon to refer to ‘chat.’  While \textit{magana} is a frequently occurring word in \ili{Hausa} in any context, \textit{hira} has a more specialized original meaning: ‘chat of an evening’ (i.e. speakers making a special point to take time to chat casually), and nowadays it refers to chatting in more general terms. In a similar vein, online forums for chatting present a space for very purposeful yet casual discussion between individuals, and thus the term \textit{hira} must have been a natural choice for \isi{semantic extension} for referring to this act. A relatively higher frequency of occurrence of \textit{hira} in these chats compared to spoken communication (according to informal input from \ili{Hausa} speakers) underscores its use as jargon.  

\subsection{Group B: ‘Message’}

Group B includes a wider range of terms — various forms or methods of messaging by which chat users communicate with one another. In this case, it is the use of \ili{English} code-mixing that is over twice as frequent as seen in \tabref{tab:purvis:4}. I speculate this is due to the readily distinguishable nuances available with the well-established the \ili{English} terms. 

\begin{table}
\begin{tabularx}{\textwidth}{Qrr}
\lsptoprule
Word & Total instances & Used as jargon in \ili{Hausa}\\
\midrule
\textit{answer}           & 10 (6\%)  & 2 (4.1\%)\\
\textit{comment}          & 3 (1.8\%) & 2 (4.1\%)\\
\textit{link}             & 1 (0.6\%) & 1 (2.0\%)\\
\textit{mail}             & 9 (5.4\%) & 8 (16.3\%)\\
\textit{message}          & 17 (10.2\%) & 7 (14.3\%)\\
\textit{reply(ing)}       & 12 (7.2\%) & 3 (6.1\%)\\
\textit{respon(ding/nse)} & 5 (3\%) & 5 (10.2\%)\\
\textit{text}             & 16 (9.6\%) & 8 (16.3\%)\\
\textit{ping}             & 3 (1.8\%) & 0 (0.0\%)\\\midrule
& & N=36 (73.5\%)\\\midrule
\textit{amsa(wa)} (‘reply’)                & 10 (6\%) & 2 (4.1\%)\\
\textit{saƙo} (‘message’)                  & 9 (5.4\%) & 9 (18.4\%)\\
\textit{taɓa(wa)} (‘poke’?; lit. ‘touch’) & 71 (42.8\%) & 2 (4.0\%)\\\midrule
& & N=13 (26.5\%)\\
\lspbottomrule
\end{tabularx} 
\caption{Frequency of occurrence for words in Group B: ‘Message’}
\label{tab:purvis:4}
\end{table} 

Among the \ili{Hausa} terms found in use, \textit{amsa} (‘respond’/‘response’) and \textit{saƙo} (‘message’) are relatively general terms. Though it was hard to tell the exact intended sense of the instances of \textit{taɓa} (verb) and \textit{taɓawa} (gerund), judging from the basic meaning of this term (‘touch’), it seems likely that this is a budding extension of this term to refer to something like ‘poking’ as used in social media platforms. 

\subsection{Group C: ‘Send’}

Compared to the various \textit{formats} of message represented in Group B, the \textit{means} of conveying them is more or less constant. Although \ili{English} has various terms like \textit{send}, \textit{forward}, \textit{email}, and \textit{transfer}, these terms all boil down to basically sending. Incidentally, it is a \ili{Hausa} word (\textit{tura(wa)/turo(wa)}) that is overwhelmingly the term of choice when referring to the action of sending as seen in \tabref{tab:purvis:5}. 

\begin{table}
\begin{tabularx}{\textwidth}{lrr}
\lsptoprule
Word & Total instances & Used as jargon in \ili{Hausa}\\
\midrule
\textit{email}        & 9 (4.8\%) & 3 (2.1\%)\\
\textit{forward}      & 1 (0.5\%) & 0 (0.0\%)\\
\textit{forwarding}   & 2 (1.1\%) & 2 (1.4\%)\\
\textit{send}         & 15 (8\%) & 1 (0.7\%)\\
\textit{sending}      & 4 (2.1\%) & 3 (2.1\%)\\
\textit{transfer}     & 3 (1.6\%) & 3 (2.1\%)\\
\textit{transferring} & 1 (0.5\%) & 1 (0.7\%)\\\midrule
& & N=13 (9.2\%)\\\midrule
\textit{tura} (‘send’; lit. ‘push (out)’)     & 55 (29.4\%) & 47 (33.1\%)\\
\textit{turawa} (‘sending’; lit. ‘pushing’)   & 4 (2.1\%) & 3 (2.1\%)\\
\textit{turo} (‘send’; lit. ‘push (hither)’)  & 90 (48.1\%) & 76 (53.5\%)\\
\textit{turowa} (‘sending’; lit. ‘pushing’)   & 3 (1.6\%) & 3 (2.1\%)\\\midrule
& & N=129 (90.8\%)\\
\lspbottomrule
\end{tabularx}
\caption{Frequency of occurrence for words in Group C: ‘Send’}
\label{tab:purvis:5}
\end{table} 

The adoption of this term also illustrates a noteworthy case of \isi{semantic extension}. The term \textit{tura} literally means ‘to push.’ (The difference between \textit{tura} and \textit{turo} is that of directionality (‘push away’ vs. ‘push towards,’ respectively); and the –\textit{wa} suffix creates a nominalized form of the verb or gerund as pointed out earlier with \textit{taɓawa}.) Outside of the chat environment, the term already carries an extended meaning of sending packages physically. So, again, it is a logical choice for conveying the notion of sending messages, pictures, attachments, etc. by electronic means. 

\subsection{Group D: ‘File-operations’}

Compared to sending, which is a straightforward and common action regardless of what we call it, the chat environment involves numerous other specialized file operations. This is an area where we do find the \ili{Hausa} speakers almost exclusively code-mixing in \ili{English} as shown in \tabref{tab:purvis:6}.  

The only specialized file operation for which a \ili{Hausa} term is found to be used is the notion of deleting (a picture/file), which is expressed by the word \textit{goge} (literally meaning ‘to rub, wipe’ and with an extended meaning of ‘erase’). Next to the 4 instances of \textit{goge}, the only instance of the \ili{English} word \textit{delete} occurs where a speaker has fully shifted to a full \ili{English} utterance. All other distinctive file operations referenced in this corpus (attaching, copying, downloading, taking a screenshot, snapping (a picture), saving) draw on \ili{English} terms.

\begin{table}
\begin{tabularx}{\textwidth}{Qrr}
\lsptoprule
Word & Total instances & Used as jargon in \ili{Hausa}\\
\midrule
\textit{attachment}       & 3 (7.1\%) & 2 (5.9\%)\\
\textit{attached}         & 1 (2.4\%) & 1 (2.9\%)\\
\textit{attaching}        & 1 (2.4\%) & 1 (2.9\%)\\
\textit{copy} (and paste) & 6 (14.3\%) & 5 (14.7\%)\\
\textit{copying}          & 3 (7.1\%) & 3 (8.8\%)\\
\textit{download}         & 2 (4.8\%) & 0 (0.0\%)\\
\textit{downloading}      & 5 (11.9\%) & 5 (14.7\%)\\
\textit{screenshot}       & 3 (7.1\%) & 3 (8.8\%)\\
\textit{snapping}         & 3 (7.1\%) & 3 (8.8\%)\\
\textit{delete}           & 1 (2.4\%) & 0 (0.0\%)\\
\textit{saving}           & 8 (19\%) & 7 (20.6\%)\\\midrule
& & N=30 (88.2\%)\\\midrule
\textit{goge(wa)} (‘delete’; lit. ‘rub clean, polish’) & 6 (14.3\%) & N=4 (11.8\%)\\
\lspbottomrule
\end{tabularx}
\caption{Frequency of occurrence for words in Group D: ‘File-operations’}
\label{tab:purvis:6}
\end{table} 

\subsection{Group E: ‘Image’}

The most prominent object discussed in the WhatsApp environment is the image — especially the so-called \textit{dp} (display picture) on a user’s profile, but also other images that are shared. In this case, abbreviated \ili{English} forms \textit{pic} (and related forms like \textit{pix}) and \textit{dp} are extremely ubiquitous, accounting for 61.7\% of references to images (\tabref{tab:purvis:7}).

\begin{table}
\begin{tabularx}{\textwidth}{Qrr}
\lsptoprule
 Word & Total instances &  Used as jargon in \ili{Hausa}\\
\midrule
\textit{image}                                    & 5 (1.8\%)   & 5 (2.4\%)\\
\textit{pic} \& related forms (e.g. \textit{pix}) & 89 (32.6\%) & 72 (35.0\%)\\
\textit{dp} (display pic)                         & 98 (35.9\%) & 55 (26.7\%)\\
\textit{pp} (profile pic)                         & 3 (1.1\%)   & 1 (0.5\%)\\
\textit{photo}                                    & 4 (1.5\%)   & 2 (1.0\%)\\\midrule
& & N=135 (65.5\%)\\\midrule
\textit{hoto/foto}  (‘photo, picture’) & 74 (27.1\%)\footnote{(including 7 spelled as \textit{photo})} & N=71 (34.5\%)\\
\lspbottomrule
\end{tabularx}
\caption{Frequency of occurrence for words in Group E: ‘Image’}
\label{tab:purvis:7}
\end{table} 

However, the \ili{Hausa} term for picture (\textit{hoto/foto}) appears about as frequently as the most common \ili{English} term (\textit{pic}). Obviously, the \ili{Hausa} term is already an \ili{English} borrowing; yet, here we are dealing with a loanword that entered the \ili{Hausa} language over 80 years ago at least \citep{bargery1934} in reference to physical photographs and has since been fully adopted as a \ili{Hausa} term carrying the same general \isi{scope} as the \ili{English} term \textit{picture}. Included within the tally of \ili{Hausa} \textit{hoto} (alternative spelling \textit{foto}) are a handful of instances that had been spelled as \textit{photo} but that otherwise pattern as the \ili{Hausa} word based on clues like use of the Class II plural ending (as in \textit{photuna}, compared to \textit{hotuna}) and the definite marker \textit{{}-n} (as in photon (‘the image’)).

\subsection{Group F: ‘Post’}

A specialized operation not included in Group D deals more specifically with images as opposed to other file types: posting. For this operation, which again is both common and straightforward (there being no nuanced ways to post an image), a \ili{Hausa} term is almost exclusively used: \textit{sa(ka)}. This verb has the basic meaning of ‘put, place.’ The short form, \textit{sa}, is also used in common expressions like \textit{Me ya sa?} (‘What happened?’) and is a very frequently occurring word in general — 289 total instances in this corpus (as shown in \tabref{tab:purvis:8}), of which 30 refer to posting in the chat environment. 

\begin{table}
\begin{tabularx}{\textwidth}{Qrr}
\lsptoprule
Word & Total instances & Used as jargon in \ili{Hausa}\\
\midrule
\textit{post(ing)}   & 2 (0.6\%) & 1 (2.1\%)\\
\textit{upload(ing)} & 3 (0.9\%) & 1 (2.1\%)\\\midrule
& & N=2 (4.2\%)\\\midrule
\textit{sa} (‘post’; lit. ‘put, place’)   & 289 (89.2\%) & 30 (63.8\%)\\
\textit{saka} (‘post’; lit. ‘put, place’) & 26 (8\%) & 13 (27.7\%)\\
\textit{sakawa} (‘placing, posting’)      & 4 (1.2\%) & 2 (4.3\%)\\\midrule
& & N=45 (95.8\%)\\
\lspbottomrule
\end{tabularx}
\caption{Frequency of occurrence for words in Group F: ‘Post’}
\label{tab:purvis:8}
\end{table} 

Technically, \textit{sa} is just a reduced form of \textit{saka}, but in practice the full form is used more rarely and (according to informal input from \ili{Hausa} speakers) it tends to be used in reference to a very deliberate act like placing a poster or sign on a wall or bulletin board, for example. Given that \textit{saka} is also heard more rarely in speech (based on impressions of \ili{Hausa} speakers consulted on the difference between \textit{sa} and \textit{saka}), it seems the 1:2 frequency in this corpus relative to the more common short form \textit{sa} is noteworthy — potentially indicative of its status as chat jargon.

\subsection{Group G: ‘Enter’}

Another type of action that is referenced in the chat environment has to do with navigating the space, as in clicking on a link. Somewhat surprisingly, the \ili{English} term \textit{click} (seemingly a likely candidate for jargon loanword in the IT environment) is not found to be used at all — only appearing in shared links (copied text from some other source). As shown in \tabref{tab:purvis:9}, the only other \ili{English} terms found anywhere are 2 instances of \textit{launch} and 1 of \textit{enter} used only when fully switching to \ili{English}. 

\begin{table}
\begin{tabularx}{\textwidth}{Qrr}
\lsptoprule
Word & Total instances & Used as jargon in \ili{Hausa}\\
\midrule
\textit{enter}  & 1 (1.2\%) & 0 (0.0\%)\\
\textit{launch} & 2 (2.3\%) & 0 (0.0\%)\\\midrule
& & N=0 (0\%)\\\midrule
\textit{buɗe(wa)} (‘open’) & 18 (20.9\%) & 7 (33.3\%)\\
\textit{shiga} (‘enter’)   & 65 (75.6\%) & 14 (66.7\%)\\\midrule
& & N=21 (100\%)\\
\lspbottomrule
\end{tabularx}
\caption{Frequency of occurrence for words in Group G: ‘Enter’}
\label{tab:purvis:9}
\end{table} 

All reference to navigating the WhatsApp space (as in guiding an interlocutor through account settings) is carried out with two \ili{Hausa} terms: 14 instances of \textit{shiga} (‘enter’) and 7 instances of \textit{buɗe} (‘open’).

\subsection{Group H: ‘On/offline’}

Another concept that comes immediately to mind as a likely candidate for borrowing from among ubiquitous \ili{English} chat jargon is the notion of being online or offline. In this case, as seen in \tabref{tab:purvis:10}, the \ili{English} term \textit{online} is indeed frequently used along with a couple instances of \textit{offline}. However, these terms get competition from \ili{Hausa} equivalents, with the \ili{Hausa} terms being slightly favored (55.3\% versus 44.7\%).

\begin{table}
\begin{tabularx}{\textwidth}{Qrr}
\lsptoprule
Word & Total instances & Used as jargon in \ili{Hausa}\\
\midrule
\textit{offline}                             & 2 (1.6\%) & 2 (5.3\%)\\
\textit{online}                              & 20 (15.5\%) & 14 (36.8\%)\\
\textit{[tana] on} (i.e.‘[she is] on[line]’) & 1 (0.8\%) & 1 (2.6\%)\\\midrule
& & N=17 (44.7)\\\midrule
\textit{fita} (‘enter’)                            & 63 (48.8\%) & 1 (2.6\%)\\
\textit{hau/hawa} (‘go(ing) online’; lit. ‘mount’) & 34 (26.4\%) & 16 (42.1\%)\\
\textit{sauka} (‘go offline’; lit. ‘descend’)      & 9 (7\%) & 4 (10.5\%)\\\midrule
& & N=21 (55.3\%)\\
\lspbottomrule
\end{tabularx}
\caption{Frequency of occurrence for words in Group H: ‘On/offline’}
\label{tab:purvis:10}
\end{table} 

The word for offline (\textit{sauka}) and its original meaning of ‘to descend’ was introduced earlier with the examples of data processing presented in \sectref{sec:purvis:3} Similarly, the concept of being online draws on \ili{Hausa}’s antonym for \textit{sauka}: \textit{hau} (‘to mount, climb’). These two terms are rather clearly on their way to being spread as the principle \ili{Hausa} chat jargon terms for online/offline. However, in one instance the verb \textit{fita} (‘to exit/go out’) was used in reference to going offline.

\subsection{Groups I \& J: ‘Group’ \& ‘Internet’}

The remaining two theme groups involve direct reference to virtual spaces: from one’s personal account, to exclusive online groups, to the broader Internet itself. Frequency data for relevant jargon terms found in this corpus are presented in \tabref{tab:purvis:11} (Group I - ‘Group’) and \tabref{tab:purvis:12} (Group J - ‘Internet’).

\begin{table}
\begin{tabularx}{\textwidth}{Qrr}
\lsptoprule
Word & Total instances & Used as jargon in \ili{Hausa}\\
\midrule
\textit{account} & 10 (50\%) & 3 (30.0\%)\\
\textit{group}   & 8 (40\%) & 5 (50.0\%)\\\midrule
& & N=8 (80\%)\\\midrule
\textit{shafuffukan yaɗa zumunta} (‘social network’)      & 1 (5\%) & 1 (10.0\%)\\
\textit{zaure} (‘group’; lit. ‘entry hall to a compound’) & 1 (5\%) & 1 (10.0\%)\\\midrule
& & N=2 (20\%)\\
\lspbottomrule
\end{tabularx}
\caption{Frequency of occurrence for words in Group I: ‘Group’}
\label{tab:purvis:11}
\end{table} 

\begin{table}
\begin{tabularx}{\textwidth}{Qrr}
\lsptoprule
Word &  Total instances & Used as jargon in \ili{Hausa}\\
\midrule
\textit{internet} & 1 (4.5\%) & 1 (20.0\%)\\
\textit{network}  & 18 (81.9\%) & 2 (40.0\%)\\
\textit{website}  & 2 (9.1\%) & 1 (20.0\%)\\\midrule
& & N=4 (80\%)\\\midrule
\textit{yanar gizo-gizo} (‘Internet’) & 1 (4.5\%) & N=1 (20.0\%)\\
\lspbottomrule
\end{tabularx}
\caption{Frequency of occurrence for words in Group J: ‘Internet’}
\label{tab:purvis:12}
\end{table} 

Two similar observations can be made for the two theme groups represented here. First, in both instances, \ili{English} terms are more frequently drawn upon, but \ili{Hausa} equivalents also appear. Secondly, the number of occurrences of any term is quite low, so the relevance of relative frequency between \ili{English} versus \ili{Hausa} terms is less conclusive. The fact that the \ili{Hausa} alternatives exist means that they could conceivably be or become more widely spread, especially if there is a trend to continue to draw on indigenous terms to fill the role of chat jargon. 

The \ili{Hausa} terms adopted in these cases are especially creative. The word for group (\textit{zaure}) comes from the word for entry hall in the traditional \ili{Hausa} housing compound where guests wait to be received by the host. This ends up being a fitting extension of this particular word, if not as obvious of a choice as jargon terms like \textit{hira} (‘chat’) and \textit{sa(ka)} (‘post’). Its simple one-word format also makes it a good candidate to catch on as a chat term. The other creative \ili{Hausa} terms in these groups are built from compounding. The phrase \textit{shafuffukan yaɗa zumunta} was used in place of the term ‘social media.’ The breakdown in meaning is as follows: \textit{Shafuffukan} is the \isi{plural form} of the word \textit{shafi} (along with the linking suffix \textit{–n}). \textit{Shafi} has a variety of senses having to do with a sheet of something (lining of cloth in a garment, page of a book, coat of paint); \textit{yaɗa} is a verb meaning ‘to spread (news, info, rumors)’; and \textit{zumunta} means ‘close relations, intimacy.’ So, the literal translation is ‘sheets (media) for spreading good relationships.’ Surely, a phrase of this length is not so likely to catch on without an abbreviated form, which is somewhat hard to imagine from this particular complex phrase. Similarly, the term for the Internet, clearly a calque of sorts of \ili{English} \textit{web}, is a relatively lengthy compound: \textit{yanar gizo-gizo} (‘spider web’). In the latter case, however, it is conceivable that this term could be reduced to \textit{yana}, for example, even though in its original sense \textit{yana} on its own refers to a film or scum covering a surface and does not convey the sense of ‘web’ without being combined with the word \textit{gizo-gizo} (‘spider’). For the younger generation, the sense of ‘web’ comes more readily. 

\section{Discussion and summary}
\subsection{Summary of findings}

From the presentation of results, we see that \ili{Hausa}-speaking chat users are employing a mixture of \ili{English} code-mixing and \ili{Hausa} words as chat jargon. That bilingual speakers (or even non-\ili{English} speakers in a multilingual \isi{speech community}) end up using \ili{English} loanwords from the IT field is not surprising. It is, however, somewhat striking to see the degree to which \ili{Hausa} terms have quickly been adapted for use as chat jargon in a relatively new medium that otherwise tends to be dominated by the \ili{English} language globally. 

When organizing the results by theme groups, we see that the likelihood of finding an \ili{English} term versus a \ili{Hausa} alternative is not entirely random. First, a number of \ili{Hausa} terms emerge as natural candidates to fill the role of key chat jargon where the referenced meaning is clear, having a literal sense or applying only a light metaphorical extension: \textit{hira} (‘chat’), \textit{tura} (‘send’), \textit{hoto} (‘image’), \textit{sa} or \textit{saka} (‘place’ = ‘post’), and a combination of \textit{shiga} (‘enter’) and \textit{buɗe} (‘open’) for clicking on links. In the case of \textit{tura}, \textit{sa} and \textit{shiga/ buɗe} (or variant forms), the \ili{Hausa} terms are used almost exclusively. 

With a number of other terms, a wider leap of \isi{semantic extension} is called upon to repurpose \ili{Hausa} words to expand the \ili{Hausa}-based chat jargon. For example, the notion of going or being online and offline is aptly equated to climbing on and descending, employing the \ili{Hausa} verbs \textit{hau} and \textit{sauka} (and variant forms), respectively. Though extremely rare in this corpus (and thus not substantial enough to draw meaningful conclusions about relative frequency of use), we also find innovative \isi{semantic extension} with terms for online group and Internet, as well as an innovative compound term to refer to social media: \textit{zaure} (‘entry hall’ = ‘group’), \textit{yanar gizo-gizo} (‘spider web’ = ‘Internet’), and \textit{shafuffukan yaɗa zumunta} (= ‘social media’).

Where \ili{English} still dominates to a great extent are areas where the widely established \ili{English} IT terms account for important distinctions or nuances in specialized actions and objects — including various file operations (like attaching, copying, downloading, deleting, and saving) and message types (like comment, response, link, and text). Nonetheless, we do find speakers drawing on \ili{Hausa} resources for purposes of this sort — such as \textit{buɗe} (‘open’), mentioned above as a logical choice for clicking a link or opening a file and \textit{goge} (literally ‘rub, wipe’) being used in reference to deletion of a virtual object. It may just be a matter of time before the innovative \ili{Hausa}-speaking community repurposes other \ili{Hausa} words for these more specialized IT concepts.  

\subsection{Future directions}

When it comes to analyzing lexical choices by bilingual speakers, we should also account more fully for different sociolinguistic factors. In terms of \isi{gender} differences, the relatively homogenous nature of this corpus (mostly composed of college students around 20 years old), has actually been beneficial, roughly controlling for most other factors. That is, the corpus is relatively balanced (24 females \& 32 males as shown in \tabref{tab:purvis:1}, with 70\% of the chat jargon terms coming from females and 30\% coming from males). So, I can briefly report that females are found to prefer a combination of code-mixing (41.5\%) and code-switching (19.6\%) to \ili{Hausa}-based jargon (38.9\%), compared to their male counterparts: 46.5\% \ili{Hausa} terms versus 36.2\% \ili{English} code-mixing and 17.2\% code-switching (Chi-square = 4.284; ~\textit{p}{}-value = .038473., significant at~\textit{p}~< .05) — incidentally confirming findings in other studies that female speakers tend to code-mix and code-switch more than men \citep{ahmeh2015,hamdani2012,wong2006}. In any case, however, it will be of interest to pursue a fuller, more systematic account of the relation between different sociolinguistic factors and use of chat jargon, collecting data from a broader demographic set, if possible. 

Another important question to address more systematically is the relation between the chat jargon terms and the use of the same words in various other contexts. For example, while still focusing on chat space: how do the dynamics of a chat group (instead of just one-on-one exchanges) affect word choices and the promotion of particular jargon terms? To what extent are the various IT jargon terms found elsewhere on the Internet? Can we get a more accurate estimate of the relative frequency of the target terms in spoken communication versus online communication? (In the presentation of results in \sectref{sec:purvis:4}, I relied on impressions from native speakers for rough judgments.) 

Finally, this article necessarily attributes the spread of \ili{Hausa} chat jargon to the \ili{Hausa}-speaking chat participants. But where has this community drawn its inspiration? For example, the term \textit{yana(r gizo-gizo)} had been documented as referring to the Internet as early as 2007 \citep{newman2007}. Recently, this word has even been used as the title of a “Kannywood”\footnote{The hub of the \ili{Hausa} film industry is the city of Kano (hence “Kannywood”).} film in which use of social media is the \isi{focus}: “Yanar Gizo” (A.Y.A. Media, Nigeria). By nature of most Kannywood films, the word also features in song and in multiple film installments — all of which is likely to reinforce or spread its use among \ili{Hausa} speakers. Other chat conventions might be traced to popular \ili{Hausa} literature. For example, several speakers use the sequence \textit{mtsw} as an \isi{ideophone} for a lip-pursing/inward sucking sound used to express disapproval, and one of the users claimed this spelling convention can be traced to \ili{Hausa} romance novels. While it is quite conceivable that many innovations have and will continue to come directly from within the chat community itself, inspiration by and reinforcement in other media will surely help spread the fuller development of a \ili{Hausa}-based chat jargon that already appears to be robust based on patterns found in the corpus presented in this study. 

\section*{Abbreviations}
Forms ending in \textit{{}-wa} after verb entries are the nominalized forms (akin to gerunds).

\noindent\begin{tabularx}{\textwidth}{@{}lQ@{}}
n & noun\\
v & verb\\ 
pers./asp.& person/\isi{aspect} complex (i.e. \isi{pronoun} + tense/\isi{aspect} encoding)
\end{tabularx}

\section*{Appendix A – Glossary of \ili{Hausa} Terms}
\noindent
\begin{tabularx}{\textwidth}{lQ}
\textit{amsa (amsawa)} & \textit{v.} answer, reply \\
\textit{buɗe (buɗewa)} & \textit{v.} open \\
\textit{fita} & \textit{v.} go out\\
\textit{goge (gogewa)} & \textit{v.} rub clean, polish\\
\textit{hau (hawa)} & \textit{v.} mount, climb, ride (figuratively used in the texts in this corpus to refer to going online) \\
\textit{hira} & \textit{n.} chatting, conversation\\
\textit{hoto} (alternative spelling: foto) & \textit{n.} photograph, picture\\
\textit{kuke} (in \textit{kuke whatsapp}) & \textit{pers./asp.} 2\textsuperscript{nd} person plural relative imperfective (i.e. ‘(that) you all are …’)\\
\textit{magana} & \textit{n.} speech, talk; matter, affair\\
\textit{sa} & \textit{v.} put, place; wear; appoint (often used in the texts in this corpus in reference to posting)\\
\textit{saka (sakawa)} & \textit{v.} put, place, arrange (often used in the texts in this corpus in reference to posting)\\
\textit{saƙo} & \textit{n.} message\\
\textit{sauka} & \textit{v.} descend, come down (figuratively used in the texts in this corpus in reference to going offline)\\
\end{tabularx}

\noindent
\begin{tabularx}{\textwidth}{lQ}
\textit{shafuffukan yaɗa zumunta} & \textit{n.} social media (relatively new coinage, literally meaning pages spreading close relations)\\
\textit{shiga} & \textit{v.} enter, go in (sometimes used in the texts in this corpus in reference to clicking/selecting)\\
\textit{surutu} & \textit{n.} talkativeness, chattering\\
\textit{taɓa (taɓawa)} & \textit{v.} touch, feel; affect; have ever done something (used in one text in this corpus in reference to texting or possibly akin to the notion of “poking” in cyberspace?) \\
\textit{taɗi} & \textit{n.} conversation, chatting\\
\textit{tana} (in \textit{tana on}) & \textit{pers./asp.} 3\textsuperscript{rd} person singular \isi{feminine} imperfective\\
\textit{tura (turawa)} & \textit{v.} push; send (out)~(often used in the texts in this corpus in reference to sending)\\
\textit{turo (turowa)} & \textit{v.} push; send (this way) (often used in the texts in this corpus in reference to sending)\\
\textit{yanar gizo-gizo} & \textit{n.} Internet, World Wide Web \\
\textit{zance} & \textit{n.} talk, conversation; \isi{subject}, matter \\
\textit{zaure} & \textit{n.} entry hall to a compound (figuratively used to refer to a chat group in this corpus)\\
\end{tabularx}


\sloppy
\printbibliography[heading=subbibliography,notkeyword=this]

\end{document}
