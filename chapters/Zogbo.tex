\documentclass[output=paper,newtxmath,modfonts,nonflat,draft]{langsci/langscibook}  

\title{Central vowels in the Kru language family: Innovation and areal spreading}

 
%%please move the includegraphics inside the {figure} environment
%%\includegraphics[width=\textwidth]{a2017versioncentralvowelsinEKruBerkeley2016-img1.jpg}

 
%%please move the includegraphics inside the {figure} environment
%%\includegraphics[width=\textwidth]{a2017versioncentralvowelsinEKruBerkeley2016-img2.jpg}

\author{Lynell Marchese Zogbo\affiliation{University of the Free State; Institut de Linguistique Appliquée, Abidjan}}
\abstract{While Proto Kru and many languages on both sides of the East-West divide today show a set of 9 oral vowels, a subset of Eastern Kru languages attests a much higher inventory, with up to five distinctive central vowels, resulting in a thirteen vowel +ATR set. The locus for central vowel innovation appears to be in the Godié-Guibéroua region, with neighboring languages at varying stages of innovation. In this paper we attempt to document vocalic inventories, point to developing systems, and speculate on how such innovations occurred, including proximity to resonant liquids (especially in a CV 1 LV 2 environment where V 1 is reduced in various contexts) and to suffixal morpheme boundaries. In some languages, co-existing lexical variation (mʊ $\sim$ mɤ
‘go’, Kagbʊwalɩ dialect of Godié) is one clear pathway to phonological change. Pressure for “rounding out” vocalic systems may also play a role in the unusually high number of innovated central vowels. Interestingly one Western language, Bakwé \citep{Marchese1989}, also has a full set
of central vowels, an apparent case of areal spreading. \citeauthor{Vydrine2009}'s (\citeyear{Vydrine2009}) hypothesis of a wider cross-family spread of central vowels into southern Mande is also discussed. While this article only scratches the surface of this complicated puzzle, evidence points to intricate interaction
between phonological change and areal spreading.}

\IfFileExists{../localcommands.tex}{%hack to check whether this is being compiled as part of a collection or standalone
  \usepackage{pifont}
\usepackage{savesym}

\savesymbol{downingtriple}
\savesymbol{downingdouble}
\savesymbol{downingquad}
\savesymbol{downingquint}
\savesymbol{suph}
\savesymbol{supj}
\savesymbol{supw}
\savesymbol{sups}
\savesymbol{ts}
\savesymbol{tS}
\savesymbol{devi}
\savesymbol{devu}
\savesymbol{devy}
\savesymbol{deva}
\savesymbol{N}
\savesymbol{Z}
\savesymbol{circled}
\savesymbol{sem}
\savesymbol{row}
\savesymbol{tipa}
\savesymbol{tableauxcounter}
\savesymbol{tabhead}
\savesymbol{inp}
\savesymbol{inpno}
\savesymbol{g}
\savesymbol{hanl}
\savesymbol{hanr}
\savesymbol{kuku}
\savesymbol{ip}
\savesymbol{lipm}
\savesymbol{ripm}
\savesymbol{lipn}
\savesymbol{ripn} 
% \usepackage{amsmath} 
% \usepackage{multicol}
\usepackage{qtree} 
\usepackage{tikz-qtree,tikz-qtree-compat}
% \usepackage{tikz}
\usepackage{upgreek}


%%%%%%%%%%%%%%%%%%%%%%%%%%%%%%%%%%%%%%%%%%%%%%%%%%%%
%%%                                              %%%
%%%           Examples                           %%%
%%%                                              %%%
%%%%%%%%%%%%%%%%%%%%%%%%%%%%%%%%%%%%%%%%%%%%%%%%%%%%
% remove the percentage signs in the following lines
% if your book makes use of linguistic examples
\usepackage{tipa}  
\usepackage{pstricks,pst-xkey,pst-asr}

%for sande et al
\usepackage{pst-jtree}
\usepackage{pst-node}
%\usepackage{savesym}


% \usepackage{subcaption}
\usepackage{multirow}  
\usepackage{./langsci/styles/langsci-optional} 
\usepackage{./langsci/styles/langsci-lgr} 
\usepackage{./langsci/styles/langsci-glyphs} 
\usepackage[normalem]{ulem}
%% if you want the source line of examples to be in italics, uncomment the following line
% \def\exfont{\it}
\usetikzlibrary{arrows.meta,topaths,trees}
\usepackage[linguistics]{forest}
\forestset{
	fairly nice empty nodes/.style={
		delay={where content={}{shape=coordinate,for parent={
					for children={anchor=north}}}{}}
}}
\usepackage{soul}
\usepackage{arydshln}
% \usepackage{subfloat}
\usepackage{langsci/styles/langsci-gb4e} 
   
% \usepackage{linguex}
\usepackage{vowel}

\usepackage{pifont}% http://ctan.org/pkg/pifont
\newcommand{\cmark}{\ding{51}}%
\newcommand{\xmark}{\ding{55}}%
 
 
 %Lamont
 \makeatletter
\g@addto@macro\@floatboxreset\centering
\makeatother

\usepackage{newfloat} 
\DeclareFloatingEnvironment[fileext=tbx,name=Tableau]{tableau}
  %add all your local new commands to this file
\newcommand{\downingquad}[4]{\parbox{2.5cm}{#1}\parbox{3.5cm}{#2}\parbox{2.5cm}{#3}\parbox{3.5cm}{#4}}
\newcommand{\downingtriple}[3]{\parbox{4.5cm}{#1}\parbox{3cm}{#2}\parbox{3cm}{#3}}
\newcommand{\downingdouble}[2]{\parbox{4.5cm}{#1}\parbox{6cm}{#2}}
\newcommand{\downingquint}[5]{\parbox{1.75cm}{#1}\parbox{2.25cm}{#2}\parbox{2cm}{#3}\parbox{3cm}{#4}\parbox{2cm}{#5}}
\newcolumntype{Y}{>{\centering\arraybackslash}X}
\newcolumntype{T}{>{\centering\arraybackslash}m{2cm}}

%commands for Kusmer paper below
\newcommand{\ip}{$\upiota$}
\newcommand{\lipm}{(\_{\ip-Max}}
\newcommand{\ripm}{)\_{\ip-Max}}
\newcommand{\lipn}{(\_{\ip}}
\newcommand{\ripn}{)\_{\ip}}
\renewcommand{\_}[1]{\textsubscript{#1}}


%commands for Pillion paper below
\newcommand{\suph}{\textipa{\super h}}
\newcommand{\supj}{\textipa{\super j}}
\newcommand{\supw}{\textipa{\super w}}
\newcommand{\ts}{\textipa{\t{ts}}}
\newcommand{\tS}{\textipa{\t{tS}}}
\newcommand{\devi}{\textipa{\r*i}}
\newcommand{\devu}{\textipa{\r*u}}
\newcommand{\devy}{\textipa{\r*y}}
\newcommand{\deva}{\textipa{\r*a}}
\renewcommand{\N}{\textipa{N}}
\newcommand{\Z}{\textipa{Z}}
% 

%commands for Diercks paper below
\newcommand{\circled}[1]{\begin{tikzpicture}[baseline=(word.base)]
\node[draw, rounded corners, text height=8pt, text depth=2pt, inner sep=2pt, outer sep=0pt, use as bounding box] (word) {#1};
\end{tikzpicture}
}

%commands for Pesetsky paper below
% \newcommand{\sem}[2][]{\mbox{$[\![ $\textbf{#2}$ ]\!]^{#1}$}}
\newcommand{\sem}[2][]{\mbox{$[[ $\textbf{#2}$ ]]^{#1}$}}

% \newcommand{\ripn}{{\color{red}ripn}}%this is used but never defined. Please update the definition



%commands for Lamont paper below
\newcommand{\row}[4]{
	#1. & 
    /{#2}/ & 
    [{#3}] & 
    `#4' \\ 
}
%\newcounter{tableauxcounter}
\newcommand{\tabhead}[2]{
%     \captionsetup{labelformat=empty}
%     \stepcounter{tableauxcounter}
%     \addtocounter{table}{-1}
% 	\centering
% 	\caption{Tableau \thetableauxcounter: #1}
	\caption{#1}
	\label{#2}
}
\newcommand{\candref}[2]{{(\ref{#1}#2)}}
\newcommand{\tableauref}[1]{{Tableau~\ref{#1}}}
% tableaux
\newcommand{\inp}[1]{\multicolumn{2}{|l||}{{#1}}}
\newcommand{\inpno}[1]{\multicolumn{2}{|l||}{#1}}
\newcommand{\g}{\cellcolor{lightgray}}
\newcommand{\hanl}{\HandLeft}
\newcommand{\hanr}{\HandRight}
\newcommand{\kuku}{Kuk\'{u}}

% \newcommand{\nocaption}[1]{{\color{red} Please provide a caption}}

% \providecommand{\biberror}[1]{{\color{red}#1}}

\definecolor{RED}{cmyk}{0.05,1,0.8,0}


\newfontfamily\amharicfont[Script = Ethiopic, Scale = 1.0]{AbyssinicaSIL}
\newcommand{\amh}[1]{{\amharicfont #1}}

% 
% %Gjersoe
\usepackage{textgreek}
% 
\newcommand{\viol}{\fontfamily{MinionPro-OsF}\selectfont\rotatebox{60}{$\star$}}
\newcommand{\myscalex}{0.45}
\newcommand{\myscaley}{0.65}
%\newcommand{\red}[1]{\textcolor{red}{#1}}
%\newcommand{\blue}[1]{\textcolor{blue}{#1}}
\newcommand{\epen}[1]{\colorbox{jgray}{#1}}
\newcommand{\hand}{{\normalsize \ding{43}}}
\definecolor{jgray}{gray}{0.8} 
\usetikzlibrary{positioning}
\usetikzlibrary{matrix}
\newcommand{\mora}{\textmu\xspace}
\newcommand{\si}{\textsigma\xspace}
\newcommand{\ft}{\textPhi\xspace}
\newcommand{\tone}{\texttau\xspace}
\newcommand{\word}{\textomega\xspace}
% \newcommand{\ts}{\texttslig}
\newcommand{\fns}{\footnotesize}
\newcommand{\ns}{\normalsize}
\newcommand{\vs}{\vspace{1em}}
\newcommand{\bs}{\textbackslash}   % backslash
\newcommand{\cmd}[1]{{\bf \color{red}#1}}   % highlights command
\newcommand{\scell}[2][l]{\begin{tabular}[#1]{@{}c@{}}#2\end{tabular}}
% \interfootnotelinepenalty=10000

% --- Snider Representations --- %

\newcommand{\RepLevelHh}{
\begin{minipage}{0.10\textwidth}
\begin{tikzpicture}[xscale=\myscalex,yscale=\myscaley]
%\node (syl) at (0,0) {Hi};
\node (Rt) at (0,1) {o};
\node (H) at (-0.5,2) {H};
\node (R) at (0.5,3) {h};
%\draw [thick] (syl.north) -- (Rt.south) ;
\draw [thick] (Rt.north) -- (H.south) ;
\draw [thick] (Rt.north) -- (R.south) ;
\end{tikzpicture}
\end{minipage}
}

\newcommand{\RepLevelLh}{
\begin{minipage}{0.10\textwidth}
\begin{tikzpicture}[xscale=\myscalex,yscale=\myscaley]
%\node (syl) at (0,0) {Mid2};
\node (Rt) at (0,1) {o};
\node (H) at (-0.5,2) {L};
\node (R) at (0.5,3) {h};
%\draw [thick] (syl.north) -- (Rt.south) ;
\draw [thick] (Rt.north) -- (H.south) ;
\draw [thick] (Rt.north) -- (R.south) ;
\end{tikzpicture}
\end{minipage}
}

\newcommand{\RepLevelHl}{
\begin{minipage}{0.10\textwidth}
\begin{tikzpicture}[xscale=\myscalex,yscale=\myscaley]
%\node (syl) at (0,0) {Mid1};
\node (Rt) at (0,1) {o};
\node (H) at (-0.5,2) {H};
\node (R) at (0.5,3) {l};
%\draw [thick] (syl.north) -- (Rt.south) ;
\draw [thick] (Rt.north) -- (H.south) ;
\draw [thick] (Rt.north) -- (R.south) ;
\end{tikzpicture}
\end{minipage}
}

\newcommand{\RepLevelLl}{
\begin{minipage}{0.10\textwidth}
\begin{tikzpicture}[xscale=\myscalex,yscale=\myscaley]
%\node (syl) at (0,0) {Lo};
\node (Rt) at (0,1) {o};
\node (H) at (-0.5,2) {L};
\node (R) at (0.5,3) {l};
%\draw [thick] (syl.north) -- (Rt.south) ;
\draw [thick] (Rt.north) -- (H.south) ;
\draw [thick] (Rt.north) -- (R.south) ;
\end{tikzpicture}
\end{minipage}
}

% --- Representations --- %

\newcommand{\RepLevel}{
\begin{minipage}{0.10\textwidth}
\begin{tikzpicture}[xscale=\myscalex,yscale=\myscaley]
\node (syl) at (0,0) {\textsigma};
\node (Rt) at (0,1) {o};
\node (H) at (-0.5,2) {\texttau};
\node (R) at (0.5,3) {\textrho};
\draw [thick] (syl.north) -- (Rt.south) ;
\draw [thick] (Rt.north) -- (H.south) ;
\draw [thick] (Rt.north) -- (R.south) ;
\end{tikzpicture}
\end{minipage}
}

\newcommand{\RepContour}{
\begin{minipage}{0.10\textwidth}
\begin{tikzpicture}[xscale=\myscalex,yscale=\myscaley]
\node (syl) at (0,0) {\textsigma};
\node (Rt) at (0,1) {o};
\node (H) at (-0.5,2) {\texttau};
\node (R) at (0.5,3) {\textrho};
\node (Rt2) at (1.5,1.0) {o};
%\node (H2) at (1.0,2) {$\tau$};
%\node (R2) at (2.0,2.5) {R};
\draw [thick] (syl.north) -- (Rt.south) ;
\draw [thick] (Rt.north) -- (H.south) ;
\draw [thick] (Rt.north) -- (R.south) ;
\draw [thick] (syl.north) -- (Rt2.south) ;
%\draw [thick] (Rt2.north) -- (H2.south) ;
%\draw [thick] (Rt2.north) -- (R2.south) ;
\end{tikzpicture}
\end{minipage}
}


% --- OT constraints --- %

\newcommand{\IllustrationDown}{
\begin{minipage}{0.09\textwidth}
\begin{tikzpicture}[xscale=0.7,yscale=0.45]
\node (reg) at (0,0.75) {{\small \textalpha}};
\node (arrow) at (0,0) {{\fns $\downarrow$}};
\node (Rt) at (0,-0.75) {{\small \textbeta}};
\end{tikzpicture}
\end{minipage}
}

\newcommand{\IllustrationUp}{
\begin{minipage}{0.09\textwidth}
\begin{tikzpicture}[xscale=0.7,yscale=0.45]
\node (reg) at (0,0.75) {{\small \textalpha}};
\node (arrow) at (0,0) {{\fns $\uparrow$}};
\node (Rt) at (0,-0.75) {{\small \textbeta}};
\end{tikzpicture}
\end{minipage}
}

\newcommand{\MaxAB}{
\begin{minipage}{0.09\textwidth}
\begin{tikzpicture}[xscale=0.6,yscale=0.4]
\node (max) at (0,0) {{\small \textsc{Max}}};
\node (reg) at (0.75,0.5) {{\fns \textalpha}};
\node (arrow) at (0.75,0) {{\tiny $\downarrow$}};
\node (Rt) at (0.75,-0.5) {{\fns \textbeta}};
\end{tikzpicture}
\end{minipage}
}

\newcommand{\DepAB}{
\begin{minipage}{0.09\textwidth}
\begin{tikzpicture}[xscale=0.6,yscale=0.4]
\node (max) at (0,0) {{\small \textsc{Dep}}};
\node (reg) at (0.75,0.5) {{\fns \textalpha}};
\node (arrow) at (0.75,0) {{\tiny $\downarrow$}};
\node (Rt) at (0.75,-0.5) {{\fns \textbeta}};
\end{tikzpicture}
\end{minipage}
}

\newcommand{\DepHReg}{
\begin{minipage}{0.055\textwidth}
\begin{tikzpicture}[xscale=0.6,yscale=0.4]
\node (dep) at (0,0) {{\small \textsc{Dep}}};
\node (reg) at (0,-1.0) {{\small h}};
\end{tikzpicture}
\end{minipage}
}

\newcommand{\DepLReg}{
\begin{minipage}{0.055\textwidth}
\begin{tikzpicture}[xscale=0.6,yscale=0.4]
\node (dep) at (0,0) {{\small \textsc{Dep}}};
\node (reg) at (0,-1.0) {{\small l}};
\end{tikzpicture}
\end{minipage}
}

\newcommand{\DepReg}{
\begin{minipage}{0.055\textwidth}
\begin{tikzpicture}[xscale=0.6,yscale=0.4]
\node (dep) at (0,0) {{\small \textsc{Dep}}};
\node (reg) at (0,-1.0) {{\small \textrho}};
\end{tikzpicture}
\end{minipage}
}

\newcommand{\DepTRt}{
\begin{minipage}{0.1\textwidth}
\begin{tikzpicture}[xscale=0.6,yscale=0.4]
\node (dep) at (0,0) {{\small \textsc{Dep}}};
\node (t) at (0.75,0.5) {{\fns \texttau}};
\node (arrow) at (0.75,0) {{\tiny $\downarrow$}};
\node (Rt) at (0.75,-0.5) {{\fns o}};
\end{tikzpicture}
\end{minipage}
}

\newcommand{\MaxRegRt}{
\begin{minipage}{0.1\textwidth}
\begin{tikzpicture}[xscale=0.6,yscale=0.4]
\node (max) at (0,0) {{\small \textsc{Max}}};
\node (arrow) at (0.75,0) {{\tiny $\downarrow$}};
\node (Rt) at (0.75,-0.5) {{\fns o}};
\node (reg) at (0.75,0.5) {{\fns \textrho}};
\end{tikzpicture}
\end{minipage}
}

\newcommand{\RegToneByRt}{
\begin{minipage}{0.06\textwidth}
\begin{tikzpicture}[xscale=0.6,yscale=0.5]
\node[rotate=20] (arrow1) at (-0.15,0) {{\fns $\uparrow$}};
\node[rotate=340] (arrow2) at (0.15,0) {{\fns $\uparrow$}};
\node (Rt) at (0,-0.55) {{\small o}};
\node (reg) at (0.4,0.55) {{\small \textrho}};
\node (tone) at (-0.4,0.55) {{\small \texttau}};
\end{tikzpicture}
\end{minipage}
}

\newcommand{\RegToneBySyl}{
\begin{minipage}{0.06\textwidth}
\begin{tikzpicture}[xscale=0.6,yscale=0.5]
\node[rotate=20] (arrow1) at (-0.15,0) {{\fns $\uparrow$}};
\node[rotate=340] (arrow2) at (0.15,0) {{\fns $\uparrow$}};
\node (Rt) at (0,-0.55) {{\small \textsigma}};
\node (reg) at (0.4,0.55) {{\small \textrho}};
\node (tone) at (-0.4,0.55) {{\small \texttau}};
\end{tikzpicture}
\end{minipage}
}

\newcommand{\DepTone}{
\begin{minipage}{0.055\textwidth}
\begin{tikzpicture}[xscale=0.6,yscale=0.4]
\node (dep) at (0,0) {{\small \textsc{Dep}}};
\node (tone) at (0,-1.0) {{\small \texttau}};
\end{tikzpicture}
\end{minipage}
}

\newcommand{\DepTonalRt}{
\begin{minipage}{0.055\textwidth}
\begin{tikzpicture}[xscale=0.6,yscale=0.4]
\node (dep) at (0,0) {{\small \textsc{Dep}}};
\node (tone) at (0,-1.0) {{\small o}};
\end{tikzpicture}
\end{minipage}
}

\newcommand{\DepL}{
\begin{minipage}{0.055\textwidth}
\begin{tikzpicture}[xscale=0.6,yscale=0.4]
\node (dep) at (0,0) {{\small \textsc{Dep}}};
\node (tone) at (0,-1.0) {{\small L}};
\end{tikzpicture}
\end{minipage}
}

\newcommand{\DepH}{
\begin{minipage}{0.055\textwidth}
\begin{tikzpicture}[xscale=0.6,yscale=0.4]
\node (dep) at (0,0) {{\small \textsc{Dep}}};
\node (tone) at (0,-1.0) {{\small H}};
\end{tikzpicture}
\end{minipage}
}

\newcommand{\NoMultDiff}{{\small *loh}}
\newcommand{\Alt}{{\small \textsc{Alt}}}
\newcommand{\NoSkip}{{\small \scell{\textsc{No}\\\textsc{Skip}}}}


\newcommand{\RegDomRt}{
\begin{minipage}{0.030\textwidth}
\begin{tikzpicture}[xscale=0.6,yscale=0.5]
\node (arrow) at (0,0) {{\fns $\downarrow$}};
\node (Rt) at (0,-0.55) {{\small o}};
\node (reg) at (0,0.55) {{\small \textrho}};
\end{tikzpicture}
\end{minipage}
}

\newcommand{\DepRegRt}{
\begin{minipage}{0.1\textwidth}
\begin{tikzpicture}[xscale=0.6,yscale=0.4]
\node (dep) at (0,0) {{\small \textsc{Dep}}};
\node (arrow) at (0.75,0) {{\tiny $\downarrow$}};
\node (Rt) at (0.75,-0.5) {{\fns o}};
\node (reg) at (0.75,0.5) {{\fns \textrho}};
\end{tikzpicture}
\end{minipage}
}

% unused

\newcommand{\ToneByRt}{
\begin{minipage}{0.05\textwidth}
\begin{tikzpicture}[xscale=0.6,yscale=0.5]
\node (arrow) at (0,0) {{\fns $\uparrow$}};
\node (Rt) at (0,-0.55) {{\small o}};
\node (tone) at (0,0.55) {{\small \texttau}};
\end{tikzpicture}
\end{minipage}
}

\newcommand{\RegByRt}{
\begin{minipage}{0.05\textwidth}
\begin{tikzpicture}[xscale=0.6,yscale=0.5]
\node (arrow) at (0,0) {{\fns $\uparrow$}};
\node (Rt) at (0,-0.55) {{\small o}};
\node (reg) at (0,0.55) {{\small \textrho}};
\end{tikzpicture}
\end{minipage}
}

\newcommand{\ToneDomRt}{
\begin{minipage}{0.05\textwidth}
\begin{tikzpicture}[xscale=0.6,yscale=0.5]
\node (arrow) at (0,0) {{\fns $\downarrow$}};
\node (Rt) at (0,-0.55) {{\small o}};
\node (tone) at (0,0.55) {{\small \texttau}};
\end{tikzpicture}
\end{minipage}
}

% --- OT tableaus --- %

% Sec. 3.2, first tabl.

\newcommand{\OTHLInput}{
\begin{minipage}{0.17\textwidth}
\begin{tikzpicture}[xscale=\myscalex,yscale=\myscaley]
\node (tone) at (2,0) {(= H)};
\node (syl) at (0,0) {\textsigma};
\node (Rt) at (0,1) {o};
\node (H) at (-0.5,2) {H};
\node (R) at (0.5,3) {h};
\node (Rt2) at (1.5,1.0) {o};
%\node (H2) at (1.0,2) {\epen{L}};
\node (R2) at (2.0,3) {\blue{l}};
\draw [thick] (syl.north) -- (Rt.south) ;
\draw [thick] (Rt.north) -- (H.south) ;
\draw [thick] (Rt.north) -- (R.south) ;
\draw [thick] (syl.north) -- (Rt2.south) ;
%\draw [dashed] (Rt2.north) -- (H2.south) ;
%\draw [dashed] (Rt2.north) -- (R2.south) ;
\end{tikzpicture}
\end{minipage}
}

\newcommand{\OTHLWinner}{
\begin{minipage}{0.17\textwidth}
\begin{tikzpicture}[xscale=\myscalex,yscale=\myscaley]
\node (tone) at (2,0) {(= HL)};
\node (syl) at (0,0) {\textsigma};
\node (Rt) at (0,1) {o};
\node (H) at (-0.5,2) {H};
\node (R) at (0.5,3) {h};
\node (Rt2) at (1.5,1.0) {o};
\node (H2) at (1.0,2) {\epen{L}};
\node (R2) at (2.0,3) {\blue{l}};
\draw [thick] (syl.north) -- (Rt.south) ;
\draw [thick] (Rt.north) -- (H.south) ;
\draw [thick] (Rt.north) -- (R.south) ;
\draw [thick] (syl.north) -- (Rt2.south) ;
\draw [dashed] (Rt2.north) -- (H2.south) ;
\draw [dashed] (Rt2.north) -- (R2.south) ;
\end{tikzpicture}
\end{minipage}
}

\newcommand{\OTHLSpreadingHOnly}{
\begin{minipage}{0.17\textwidth}
\begin{tikzpicture}[xscale=\myscalex,yscale=\myscaley]
\node (tone) at (2,0) {(= HM)};
\node (syl) at (0,0) {\textsigma};
\node (Rt) at (0,1) {o};
\node (H) at (-0.5,2) {H};
\node (R) at (0.5,3) {h};
\node (Rt2) at (1.5,1.0) {o};
%\node (H2) at (1.0,2) {\epen{L}};
\node (R2) at (2.0,3) {\blue{l}};
\draw [thick] (syl.north) -- (Rt.south) ;
\draw [thick] (Rt.north) -- (H.south) ;
\draw [thick] (Rt.north) -- (R.south) ;
\draw [thick] (syl.north) -- (Rt2.south) ;
\draw [dashed] (Rt2.north) -- (R2.south) ;
\draw [dashed] (Rt2.north) -- (H.south) ;
\end{tikzpicture}
\end{minipage}
}

\newcommand{\OTHLInsertH}{
\begin{minipage}{0.17\textwidth}
\begin{tikzpicture}[xscale=\myscalex,yscale=\myscaley]
\node (tone) at (2,0) {(= HM)};
\node (syl) at (0,0) {\textsigma};
\node (Rt) at (0,1) {o};
\node (H) at (-0.5,2) {H};
\node (R) at (0.5,3) {h};
\node (Rt2) at (1.5,1.0) {o};
\node (H2) at (1.0,2) {\epen{H}};
\node (R2) at (2.0,3) {\blue{l}};
\draw [thick] (syl.north) -- (Rt.south) ;
\draw [thick] (Rt.north) -- (H.south) ;
\draw [thick] (Rt.north) -- (R.south) ;
\draw [thick] (syl.north) -- (Rt2.south) ;
\draw [dashed] (Rt2.north) -- (H2.south) ;
\draw [dashed] (Rt2.north) -- (R2.south) ;
\end{tikzpicture}
\end{minipage}
}

\newcommand{\OTHLOverwriting}{
\begin{minipage}{0.17\textwidth}
\begin{tikzpicture}[xscale=\myscalex,yscale=\myscaley]
\node (syl) at (0,0) {\textsigma};
\node (Rt) at (0,1) {o};
\node (H) at (-0.5,2) {H};
\node (R) at (0.5,3) {h};
\node (Rt2) at (1.5,1.0) {o};
%\node (H2) at (1.0,2) {\epen{L}};
\node (R2) at (2.0,3) {\blue{l}};
\draw [thick] (syl.north) -- (Rt.south) ;
\draw [thick] (Rt.north) -- (H.south) ;
\draw [thick] (Rt.north) -- (R.south) ;
\draw [thick] (syl.north) -- (Rt2.south) ;
%\draw [dashed] (Rt2.north) -- (H2.south) ;
\draw [dashed] (Rt.north) -- (R2.south) ;
\node (del) at (0.3,1.9) {\textbf{=}};
\end{tikzpicture}
\end{minipage}
}

\newcommand{\OTHLSpreading}{
\begin{minipage}{0.17\textwidth}
\begin{tikzpicture}[xscale=\myscalex,yscale=\myscaley]
\node (syl) at (0,0) {\textsigma};
\node (Rt) at (0,1) {o};
\node (H) at (-0.5,2) {H};
\node (R) at (0.5,3) {h};
\node (Rt2) at (1.5,1.0) {o};
%\node (H2) at (1.0,2) {\epen{L}};
\node (R2) at (2.0,3) {\blue{l}};
\draw [thick] (syl.north) -- (Rt.south) ;
\draw [thick] (Rt.north) -- (H.south) ;
\draw [thick] (Rt.north) -- (R.south) ;
\draw [thick] (syl.north) -- (Rt2.south) ;
%\draw [dashed] (Rt2.north) -- (H2.south) ;
\draw [dashed] (Rt2.north) -- (H.south) ;
\draw [dashed] (Rt2.north) -- (R.south) ;
\end{tikzpicture}
\end{minipage}
}

% Sec. 4.2, second tabl.: phrase-medial position

\newcommand{\OTHnoLInput}{
\begin{minipage}{0.17\textwidth}
\begin{tikzpicture}[xscale=\myscalex,yscale=\myscaley]
\node (tone) at (2,0) {(= H)};
\node (syl) at (0,0) {\textsigma};
\node (Rt) at (0,1) {o};
\node (H) at (-0.5,2) {H};
\node (R) at (0.5,3) {h};
\node (Rt2) at (1.5,1.0) {o};
%\node (H2) at (1.0,2) {\epen{L}};
%\node (R2) at (2.0,3) {\blue{l}};
\draw [thick] (syl.north) -- (Rt.south) ;
\draw [thick] (Rt.north) -- (H.south) ;
\draw [thick] (Rt.north) -- (R.south) ;
\draw [thick] (syl.north) -- (Rt2.south) ;
\end{tikzpicture}
\end{minipage}
}

\newcommand{\OTHnoLEpenth}{
\begin{minipage}{0.17\textwidth}
\begin{tikzpicture}[xscale=\myscalex,yscale=\myscaley]
\node (tone) at (2,0) {(= HM)};
\node (syl) at (0,0) {\textsigma};
\node (Rt) at (0,1) {o};
\node (H) at (-0.5,2) {H};
\node (R) at (0.5,3) {h};
\node (Rt2) at (1.5,1.0) {o};
\node (H2) at (1.0,2) {\epen{L}};
\node (R2) at (2.0,3) {\epen{h}};
\draw [thick] (syl.north) -- (Rt.south) ;
\draw [thick] (Rt.north) -- (H.south) ;
\draw [thick] (Rt.north) -- (R.south) ;
\draw [thick] (syl.north) -- (Rt2.south) ;
\draw [dashed] (Rt2.north) -- (H2.south) ;
\draw [dashed] (Rt2.north) -- (R2.south) ;
\end{tikzpicture}
\end{minipage}
}

\newcommand{\OTHnoLSpreading}{
\begin{minipage}{0.17\textwidth}
\begin{tikzpicture}[xscale=\myscalex,yscale=\myscaley]
\node (tone) at (2,0) {(= HH)};
\node (syl) at (0,0) {\textsigma};
\node (Rt) at (0,1) {o};
\node (H) at (-0.5,2) {H};
\node (R) at (0.5,3) {h};
\node (Rt2) at (1.5,1.0) {o};
%\node (H2) at (1.0,2) {\epen{L}};
%\node (R2) at (2.0,3) {\blue{l}};
\draw [thick] (syl.north) -- (Rt.south) ;
\draw [thick] (Rt.north) -- (H.south) ;
\draw [thick] (Rt.north) -- (R.south) ;
\draw [thick] (syl.north) -- (Rt2.south) ;
\draw [dashed] (Rt2.north) -- (H.south) ;
\draw [dashed] (Rt2.north) -- (R.south) ;
\end{tikzpicture}
\end{minipage}
}

% Sec. 4.2, third tabl., LM is unaffected by L\%

\newcommand{\OTLMInput}{
\begin{minipage}{0.2\textwidth}
\begin{tikzpicture}[xscale=\myscalex,yscale=\myscaley]
\node (tone) at (2,0) {(= LM)};
\node (syl) at (0,0) {\textsigma};
\node (Rt) at (0,1) {o};
\node (H) at (-0.5,2) {L};
\node (R) at (0.5,3) {l};
\node (Rt2) at (1.5,1.0) {o};
\node (H2) at (1.0,2) {L};
\node (R2) at (2.0,3) {h};
\node (R3) at (3.0,3) {\blue{l}};
\draw [thick] (syl.north) -- (Rt.south) ;
\draw [thick] (Rt.north) -- (H.south) ;
\draw [thick] (Rt.north) -- (R.south) ;
\draw [thick] (syl.north) -- (Rt2.south) ;
\draw [thick] (Rt2.north) -- (H2.south) ;
\draw [thick] (Rt2.north) -- (R2.south) ;
\end{tikzpicture}
\end{minipage}
}

\newcommand{\OTLMReplace}{
\begin{minipage}{0.2\textwidth}
\begin{tikzpicture}[xscale=\myscalex,yscale=\myscaley]
\node (tone) at (2,0) {(= LL)};
\node (syl) at (0,0) {\textsigma};
\node (Rt) at (0,1) {o};
\node (H) at (-0.5,2) {L};
\node (R) at (0.5,3) {l};
\node (Rt2) at (1.5,1.0) {o};
\node (H2) at (1.0,2) {L};
\node (R2) at (2.0,3) {h};
\node (R3) at (3.0,3) {\blue{l}};
\draw [thick] (syl.north) -- (Rt.south) ;
\draw [thick] (Rt.north) -- (H.south) ;
\draw [thick] (Rt.north) -- (R.south) ;
\draw [thick] (syl.north) -- (Rt2.south) ;
\draw [thick] (Rt2.north) -- (H2.south) ;
\draw [thick] (Rt2.north) -- (R2.south) ;
\draw [dashed] (Rt2.north) -- (R3.south) ;
\node (del) at (1.8,2.1) {\textbf{=}};
\end{tikzpicture}
\end{minipage}
}

\newcommand{\OTLMTwoReg}{
\begin{minipage}{0.2\textwidth}
\begin{tikzpicture}[xscale=\myscalex,yscale=\myscaley]
\node (tone) at (2,0) {(= LML)};
\node (syl) at (0,0) {\textsigma};
\node (Rt) at (0,1) {o};
\node (H) at (-0.5,2) {L};
\node (R) at (0.5,3) {l};
\node (Rt2) at (1.5,1.0) {o};
\node (H2) at (1.0,2) {L};
\node (R2) at (2.0,3) {h};
\node (R3) at (3.0,3) {\blue{l}};
\draw [thick] (syl.north) -- (Rt.south) ;
\draw [thick] (Rt.north) -- (H.south) ;
\draw [thick] (Rt.north) -- (R.south) ;
\draw [thick] (syl.north) -- (Rt2.south) ;
\draw [thick] (Rt2.north) -- (H2.south) ;
\draw [thick] (Rt2.north) -- (R2.south) ;
\draw [dashed] (Rt2.north) -- (R3.south) ;
\end{tikzpicture}
\end{minipage}
}

% Sec. 4.2, fourth tabl., L is affected by L\% but M is not

\newcommand{\OTLInput}{
\begin{minipage}{0.17\textwidth}
\begin{tikzpicture}[xscale=\myscalex,yscale=\myscaley]
\node (tone) at (2,0) {(= L)};
\node (syl) at (0,0) {\textsigma};
\node (Rt) at (0,1) {o};
\node (H) at (-0.5,2) {L};
\node (R) at (0.5,3) {l};
\node (R2) at (2,3) {\blue{l}};
\draw [thick] (syl.north) -- (Rt.south) ;
\draw [thick] (Rt.north) -- (H.south) ;
\draw [thick] (Rt.north) -- (R.south) ;
\end{tikzpicture}
\end{minipage}
}

\newcommand{\OTLLowered}{
\begin{minipage}{0.17\textwidth}
\begin{tikzpicture}[xscale=\myscalex,yscale=\myscaley]
\node (tone) at (2,0) {(= LL)};
\node (syl) at (0,0) {\textsigma};
\node (Rt) at (0,1) {o};
\node (H) at (-0.5,2) {L};
\node (R) at (0.5,3) {l};
\node (R2) at (2,3) {\blue{l}};
\draw [thick] (syl.north) -- (Rt.south) ;
\draw [thick] (Rt.north) -- (H.south) ;
\draw [thick] (Rt.north) -- (R.south) ;
\draw [dashed] (Rt.north) -- (R2.south) ;
\end{tikzpicture}
\end{minipage}
}

\newcommand{\OTMInput}{
\begin{minipage}{0.17\textwidth}
\begin{tikzpicture}[xscale=\myscalex,yscale=\myscaley]
\node (tone) at (2,0) {(= M)};
\node (syl) at (0,0) {\textsigma};
\node (Rt) at (0,1) {o};
\node (H) at (-0.5,2) {L};
\node (R) at (0.5,3) {h};
\node (R2) at (2,3) {\blue{l}};
\draw [thick] (syl.north) -- (Rt.south) ;
\draw [thick] (Rt.north) -- (H.south) ;
\draw [thick] (Rt.north) -- (R.south) ;
\end{tikzpicture}
\end{minipage}
}

\newcommand{\OTMLowered}{
\begin{minipage}{0.17\textwidth}
\begin{tikzpicture}[xscale=\myscalex,yscale=\myscaley]
\node (tone) at (2,0) {(= ML)};
\node (syl) at (0,0) {\textsigma};
\node (Rt) at (0,1) {o};
\node (H) at (-0.5,2) {L};
\node (R) at (0.5,3) {h};
\node (R2) at (2,3) {\blue{l}};
\draw [thick] (syl.north) -- (Rt.south) ;
\draw [thick] (Rt.north) -- (H.south) ;
\draw [thick] (Rt.north) -- (R.south) ;
\draw [dashed] (Rt.north) -- (R2.south) ;
\end{tikzpicture}
\end{minipage}
}

% Sec. 4.2, fifth tableau, polar questions with level tones

\newcommand{\OTLPolIn}{
\begin{minipage}{0.20\textwidth}
\begin{tikzpicture}[xscale=\myscalex-0.05,yscale=\myscaley-0.05]
\node (tone) at (3.5,0) {(= L)};
\node (syl) at (0,0) {\textsigma};
\node (syl2) at (2,0) {\red{\textsigma}};
\node (Rt) at (0,1) {o};
\node (H) at (-0.5,2) {L};
\node (R) at (0.5,3) {l};
\node (Rt2) at (2,1) {\red{o}};
\draw [thick] (syl.north) -- (Rt.south) ;
\draw [thick,red] (syl2.north) -- (Rt2.south) ;
\draw [thick] (Rt.north) -- (H.south) ;
\draw [thick] (Rt.north) -- (R.south) ;
\end{tikzpicture}
\end{minipage}
}

\newcommand{\OTLPolDef}{
\begin{minipage}{0.20\textwidth}
\begin{tikzpicture}[xscale=\myscalex-0.05,yscale=\myscaley-0.05]
\node (tone) at (3.5,0) {(= L.M)};
\node (syl) at (0,0) {\textsigma};
\node (syl2) at (2,0) {\red{\textsigma}};
\node (Rt) at (0,1) {o};
\node (H) at (-0.5,2) {L};
\node (R) at (0.5,3) {l};
\node (H2) at (1.5,2) {\epen{L}};
\node (R2) at (2.5,3) {\epen{h}};
\node (Rt2) at (2,1) {\red{o}};
\draw [thick] (syl.north) -- (Rt.south) ;
\draw [thick,red] (syl2.north) -- (Rt2.south) ;
\draw [thick] (Rt.north) -- (H.south) ;
\draw [thick] (Rt.north) -- (R.south) ;
\draw [semithick,dashed] (Rt2.north) -- (H2.south) ;
\draw [semithick,dashed] (Rt2.north) -- (R2.south) ;
\end{tikzpicture}
\end{minipage}
}

\newcommand{\OTLPolAlt}{
\begin{minipage}{0.20\textwidth}
\begin{tikzpicture}[xscale=\myscalex-0.05,yscale=\myscaley-0.05]
\node (tone) at (3.5,0) {(= L.L)};
\node (syl) at (0,0) {\textsigma};
\node (syl2) at (2,0) {\red{\textsigma}};
\node (Rt) at (0,1) {o};
\node (H) at (-0.5,2) {L};
\node (R) at (0.5,3) {l};
\node (Rt2) at (2,1) {\red{o}};
\draw [thick] (syl.north) -- (Rt.south) ;
\draw [thick,red] (syl2.north) -- (Rt2.south) ;
\draw [thick] (Rt.north) -- (H.south) ;
\draw [thick] (Rt.north) -- (R.south) ;
\draw [semithick,dashed] (Rt2.north) -- (H.south) ;
\draw [semithick,dashed] (Rt2.north) -- (R.south) ;
\end{tikzpicture}
\end{minipage}
}

% Sec. 4.2, sixth tableau, polar questions with contour tones

\newcommand{\OTLLPolIn}{
\begin{minipage}{0.23\textwidth}
\begin{tikzpicture}[xscale=\myscalex-0.05,yscale=\myscaley-0.05]
\node (tone) at (5.2,0) {(= L)};
\node (syl) at (0,0) {\textsigma};
\node (syl3) at (3.4,0) {\red{\textsigma}};
\node (Rt) at (0,1) {o};
\node (Rt2) at (1.7,1) {o};
\node (Rt3) at (3.4,1) {\red{o}};
\node (H) at (-0.5,2) {L};
\node (R) at (0.5,3) {l};
\draw [thick] (syl.north) -- (Rt.south) ;
\draw [thick] (syl.north) -- (Rt2.south) ;
\draw [thick,red] (syl3.north) -- (Rt3.south) ;
\draw [thick] (Rt.north) -- (H.south) ;
\draw [thick] (Rt.north) -- (R.south) ;
\end{tikzpicture}
\end{minipage}
}

\newcommand{\OTLLPolDef}{
\begin{minipage}{0.23\textwidth}
\begin{tikzpicture}[xscale=\myscalex-0.05,yscale=\myscaley-0.05]
\node (tone) at (5.2,0) {(= L.M)};
\node (syl) at (0,0) {\textsigma};
\node (syl3) at (3.4,0) {\red{\textsigma}};
\node (Rt) at (0,1) {o};
\node (Rt2) at (1.7,1) {o};
\node (Rt3) at (3.4,1) {\red{o}};
\node (H) at (-0.5,2) {L};
\node (R) at (0.5,3) {l};
\node (H3) at (2.9,2) {\epen{L}};
\node (R3) at (3.9,3) {\epen{h}};
\draw [thick] (syl.north) -- (Rt.south) ;
\draw [thick] (syl.north) -- (Rt2.south) ;
\draw [thick,red] (syl3.north) -- (Rt3.south) ;
\draw [thick] (Rt.north) -- (H.south) ;
\draw [thick] (Rt.north) -- (R.south) ;
\draw [dashed] (Rt3.north) -- (H3.south) ;
\draw [dashed] (Rt3.north) -- (R3.south) ;
\end{tikzpicture}
\end{minipage}
}

\newcommand{\OTLLPolSkip}{
\begin{minipage}{0.23\textwidth}
\begin{tikzpicture}[xscale=\myscalex-0.05,yscale=\myscaley-0.05]
\node (tone) at (5.2,0) {(= L.L)};
\node (syl) at (0,0) {\textsigma};
\node (syl3) at (3.4,0) {\red{\textsigma}};
\node (Rt) at (0,1) {o};
\node (Rt2) at (1.7,1) {o};
\node (Rt3) at (3.4,1) {\red{o}};
\node (H) at (-0.5,2) {L};
\node (R) at (0.5,3) {l};
\draw [thick] (syl.north) -- (Rt.south) ;
\draw [thick] (syl.north) -- (Rt2.south) ;
\draw [thick,red] (syl3.north) -- (Rt3.south) ;
\draw [thick] (Rt.north) -- (H.south) ;
\draw [thick] (Rt.north) -- (R.south) ;
\draw [dashed] (Rt3.north) -- (H.south) ;
\draw [dashed] (Rt3.north) -- (R.south) ;
\end{tikzpicture}
\end{minipage}
}  
  
\newcommand{\ilit}[1]{#1\il{#1}}    
\newcommand{\isit}[1]{#1\is{#1}}  

\makeatletter
\let\thetitle\@title
\let\theauthor\@author 
\makeatother

\newcommand{\togglepaper}[1][0]{ 
  \bibliography{../localbibliography}
  %% hyphenation points for line breaks
%% Normally, automatic hyphenation in LaTeX is very good
%% If a word is mis-hyphenated, add it to this file
%%
%% add information to TeX file before \begin{document} with:
%% %% hyphenation points for line breaks
%% Normally, automatic hyphenation in LaTeX is very good
%% If a word is mis-hyphenated, add it to this file
%%
%% add information to TeX file before \begin{document} with:
%% \include{localhyphenation}
\hyphenation{
affri-ca-te
affri-ca-tes
com-ple-ments
par-a-digm
Sha-ron
Kings-ton
phe-nom-e-non
Daul-ton
Abu-ba-ka-ri
Ngo-nya-ni
Clem-ents 
King-ston
Tru-cken-brodt
Tab-leau
cophono-logies
mark-edness
Ti-gri-nya
a-mong
Car-stens
Lu-bu-ku-su
}
\hyphenation{
affri-ca-te
affri-ca-tes
com-ple-ments
par-a-digm
Sha-ron
Kings-ton
phe-nom-e-non
Daul-ton
Abu-ba-ka-ri
Ngo-nya-ni
Clem-ents 
King-ston
Tru-cken-brodt
Tab-leau
cophono-logies
mark-edness
Ti-gri-nya
a-mong
Car-stens
Lu-bu-ku-su
}
  \papernote{\scriptsize\normalfont
    \theauthor.
    \thetitle. 
    To appear in: 
    Emily Clem,   Peter Jenks \& Hannah Sande.
    Theory and description in African Linguistics: Selected papers from the 47th Annual Conference on African Linguistics.
    Berlin: Language Science Press. [preliminary page numbering]
  }
  \pagenumbering{roman}
  \setcounter{chapter}{#1}
  \addtocounter{chapter}{-1}
}

\newcommand{\upstep}{\textupstep}


% \newcounter{tableauxcounter}

\renewcommand{\textltailn}{ɲ}
\renewcommand{\textbardotlessj}{ɟ}

\newcommand{\emphkh}[1]{\textit{#1}} %originally \textbf, banned by the guidelines



\definecolor{lsDOIGray}{cmyk}{0,0,0,0.45}


\newcommand{\xuparrow}[1]{%
  {\left\uparrow\vbox to #1{}\right.\kern-\nulldelimiterspace}
}
\renewcommand \textupstep[1]{\char"A71B#1}
\renewcommand \textdownstep[1]{\char"A71C#1}
 
 \newcommand{\ꜛ}{\textsf{ꜛ}}
 
\def\biberror{\undefined}


\newcommand{\OTbox}[1]{\resizebox{.88\textwidth}{!}{#1}}
 
  \togglepaper[36]
}{}

\begin{document}
\maketitle 
\section{Introduction}\label{sec:zogbo:1} 

A quick inventory of vowel systems in the \ili{Kru} language family\footnote{The status of the \ili{Kru} language family within Niger-Congo is still \isi{subject} to debate, having been proposed as independent (Westermann), a branch of \ili{Kwa} (Greenberg), closely connected to \ili{Gur} (Bennett and Sterk), and of late \citep[18]{Williamson2000} an offshoot of Proto West Volta Congo.} reveals a striking diversity.  While in Western \ili{Kru}, with the exception of /a/, no phonemic central vowels are attested, in Eastern \ili{Kru}, some languages have a full set, with 5 out of 13 vowels being central (or back unrounded). Citing numerous shared features in South \ili{Mande} and \ili{Kru}, \citet{Vydrine2009} proposes that central vowels may be one of several areal features of the Ivorian forest region, cutting across genetic boundaries.  Thus in this paper, we attempt to:

\begin{itemize}
\item explore the innovation of central vowels in Eastern \ili{Kru}, examining the extent and possible means of this phonological innovation and \\[-0.75cm]
\item evaluate the viability of areal hypotheses concerning the spread of central vowels within \ili{Kru} and across its linguistic borders. 
\end{itemize}

All \ili{Kru} languages show a minimum of 9 oral vowels, featuring two sets of vowels based on the feature +ATR, and usually a strong \isi{vowel harmony} system affecting word internal stems and suffix affixation. A typical system is seen in \ili{Kouya} where vowels occur in two sets (\tabref{tab:zogbo:1}, adapted from \citealt[50]{Saunders2009}).

\begin{table}[h]
\begin{tabular}{lllllll}
\lsptoprule
& Front   &&   Central &   Back &\\
\midrule
& +ATR & −ATR   &   &  +ATR & −ATR\\

+high  &  i &	ɪ    &&   u  &  ʊ\\

−high  &  e & ɛ  &  a  &  o & ɔ\\
\lspbottomrule
\end{tabular}  
\caption{Kouya vowels}
\label{tab:zogbo:1}
\end{table}

 Words are made up of either + or – ATR vowels (\tabref{tab:zogbo:2a}) with /a/ co-occurring with both + ATR vowels (\tabref{tab:zogbo:2b}).\footnote{As is traditional in \ili{Kru} literature, in the examples to follow and throughout this paper, tones are marked by raised superscripts. Most \ili{Kru} languages show four level tones: high (1), mid-high (2), mid (3), and low (4).  Exceptionally \ili{Godié} has only three level tones (high, mid, low), with only remnants of a fourth \isi{tone} \citep{Gratrix1975}.} 

\begin{table}

\caption{Kouya $\pm$ATR sets}     \label{tab:zogbo:2a}
\begin{tabular}{llll}

\lsptoprule
\multicolumn{2}{c}{−ATR} & \multicolumn{2}{c}{+ATR}\\
\midrule
\textit{ɓʊ}{$^1$}\textit{l}ɛ$^2$ &     {‘buffalo’} & \textit{di}$^2$\textit{de}$^2$ & {‘father’}\\

\textit{tɪɓɛ}$^{33}$  &  ‘snake’   &   \textit{ɓu}$^2$\textit{ɓui}$^1$ & {‘smoke,  vapor’}\\

\textit{mɪɔ}$^{13}$  &  ‘tear’ (n.) &    \textit{ɓeli}$^{23}$  &  ‘brother’ \\

\textit{m{ʊ}}{$^3$}\textit{{m}ɔɛ}$^{44}$ & ‘smile’ (v.)  &  \textit{petu}$^{41}$  &  ‘grass’ \\

\textit{ɓʊɪ}$^4$ & ‘flower’  &  \textit{liɓo}$^{33}$    & ‘work’ (n.) \\
\lspbottomrule
\end{tabular}
\end{table}

\begin{table}
\caption{Kouya words with /a/ with both $\pm$ATR sets}
\label{tab:zogbo:2b}
  
\begin{tabular}{llll}

\lsptoprule
\multicolumn{2}{c}{−ATR} & \multicolumn{2}{c}{+ATR} \\
\midrule
\textit{kʋa}$^{11}$ & ‘bone’ & \textit{bita}$^{41}$ & ‘mat’ \\

\textit{{kp}ɛ}$^2$\textit{{l}a}$^1$ & {‘to refuse’} & \textit{te}$^2$\textit{la}$^2$ & ‘porcupine’ \\

\textit{yɪ}$^1$\textit{ɓa}$^1$ & ‘desire, want’ & \textit{gba}$^2$\textit{gbo}$^3$ & ‘partridge’ \\ 
\lspbottomrule
\end{tabular}
\end{table}

Despite its non-participatory status in \isi{vowel harmony}, /a/ usually patterns in other ways as –ATR. In terms of frequency, −ATR vowels are more frequent than +ATR, and most suffixes (verbal suffixes, \isi{noun class} markers and other nominal suffixes) are underlying –ATR. \citet{Casali2008} notes in dominant harmony languages, affix harmony involves assimilation of [$-$ATR] to [+ATR] vowels, a fact that seems to hold in our Eastern \ili{Kru} samples, for example, in \ili{Godié} where rightward assimilation frequently shows a –ATR to +ATR shift, as in the following example of object clitics in \ili{Godié}:


\ea
    \label{ex:zogbo:3}
		\ili{Godié} \citep{Marchese1975}
    \z


\downingdouble{/\textit{ɔ}$^2$ \textit{bi}$^2$\textit{bie}$^2$ \textit{ɔ}$^2$/ }{‘he begs him’ (person)}

\downingdouble{\textit{bibiǿ{ }ɔ}}{(vowel elision) }

\downingdouble{[\textit{ɔ{ }bibi{ }o}] }{(\isi{vowel harmony})}


\section{More elaborate systems}\label{sec:zogbo:2} 

Though both Western and Eastern \ili{Kru} attest the standard 9 oral vowel system, several Eastern \ili{Kru} languages (and Western \ili{Bakwé}) have much larger phonemic vowel inventories, with many additional central (or back unrounded) phonemic vowels\footnote{Researchers have used both terms.  Central vowels in \ili{Kru} are not rounded. In acoustic studies, \citet{grégoire1972} has called these vowels in \ili{Bété} of \ili{Guibéroua} \textit{central} \citep[see also][15]{Zogbo1981}. In other descriptions, \citet[61]{Werle1976} as well as \citet[7]{kipre2005} analyze them as \textit{back unrounded.} In Goprou’s more recent study of \ili{Kpɔkolo}, a \ili{Bété} dialect (\citeyear{Goprou2010}; \citeyear[177]{Goprou2014}), findings are somewhat skewed. For vowels [ɨ, θ, ʌ, and a], a female speaker shows F2 readings around 1500 Hz, indicating a clear central position, while [θ] positions itself as a back rounded vowels (under 1500 Hz), as does [ʌ] in male speakers. This issue is important but out of the \isi{scope} of this paper.}, as seen in \tabref{tab:zogbo:4}.

\begin{table}
\begin{tabularx}{\textwidth}{XXX}
\lsptoprule
\textbf{Language} & \textbf{Number of Vowels} & \textbf{Number of Central Vowels (excluding /a/)}\\
\midrule
\ili{Godié} & 13 & 4\\
\ili{Koyo} & 13 & 4\\
\ili{Guibéroua} \ili{Bété} & 13 & 4\\
\ili{Gbawale} & 13 & 4\\
\ili{Daloa} \ili{Bété} & 12 & 3\\
\ili{Kpɔkolo} & 11 & 2\\
\ili{Gaɓʊgbʊ} & 11 & 2\\
\ili{Guébie} & 10 & 1\\
\ili{Vata} & 10 & 1\\
\ili{Gbadi} & 9 & 0\\
\ili{Lakota Dida} & 9 & 0\\
\ili{Yocoboue Dida} & 9 & 0\\
\ili{Neyo} & 9 & 0\\
\ili{Kouya} & 9 & 0\\
\lspbottomrule
\end{tabularx}
\caption{Vowel inventories in Eastern Kru languages}
\label{tab:zogbo:4}
\end{table}

Within Western \ili{Kru}, no phonemic central vowels are attested, except in \ili{Bakwé}, which lies contiguous to Eastern \ili{Godié} (see Maps 1 and 2 below).  For over a century \citep{Delafosse1904}, \ili{Bakwé} has been classified as a Western \ili{Kru} language based on important isoglosses such as \textbf{t/s} (‘tree’ \textit{tu/su$^3$},); \textbf{ny/ng} (‘name’, ‘woman’); Western \textit{nɪ}$^1$ ‘water’ vs. PEK \textit{*nyu}$^1$. \citep{Marchese1989}.  Curiously \citet{lewisetal2014} classify \ili{Bakwé} as Eastern. In this language and the four Eastern languages seen at the top of the table above (\ili{Guibéroua} \ili{Bété}, Gbawʌlɪ, \ili{Godié}, \ili{Koyo}), there is a full set of five phonologically contrastive central vowels, which correspond to vowel heights of the peripheral vowels and are also defined as +ATR, as seen in \tabref{tab:zogbo:5}.

\begin{table}
\begin{tabular}{llll}
\lsptoprule
Front & Central & Back\\
\midrule
i  &  ɨ [ɯ, ï]\footnote{Differences among researchers in transcription complicate our task and it is difficult to identify the exact phonetic realization of such a variety of transcriptions, especially the symbol [ə] used as default schwa in languages without central vowels. As seen above, in languages with full central vowel sets, [ə] is a higher vowel than [ʌ] and is +ATR. In most instances, I tried to respect the author’s original transcription.} & u  &  (+high,+ATR)\\

ɪ  &  ʉ [ɤ, θ]  & ʊ  &  (+high, −ATR) \\

e  &  ə  &  o  &  (-high, +ATR)\\

ɛ  &  ʌ  &  ɔ  &  (-high , −ATR) \\

& a\\
\lspbottomrule
\end{tabular}	
\caption{Largest oral vowel system in Kru}
\label{tab:zogbo:5}
\end{table}

Despite the fact that it is hard to find \isi{perfect} sets of minimal pairs, native speakers clearly distinguish 5 central vowel qualities and can learn to read and write them without difficulty. In many languages, to establish a full set of contrasts, plural forms complete minimal pairs lists: 

\ea \ili{Guibéroua} \ili{Bété} (Werle \& Gbalehi, 1976)

\begin{tabular}{llllll}
\textit{kpə}$^1$ & ‘chair’  &  \textit{pə}$^3$ &  ‘cover’  &  \textit{pʉ}$^1$& ‘lie down’\\

\textit{kpɨ}$^1$ &  ‘chairs’ & \textit{pʌ}$^3$   & ‘throw’ & \textit{kpa}²  &  ‘mud’ \\
\end{tabular}
\z

\begin{table}
\begin{tabular}{llll}
\lsptoprule	
\textit{li}$^1$  &  ‘spear’  &  \textit{lɪ}$^2$ & ‘wealth/riches’\\

\textit{lɨ}$^2$  &  ‘eat’  &  \textit{lʉla}  &  ‘grill, fry’\\

\textit{luu}$^{12}$ &   ‘paddle’ & \textit{lʊ}$^1$ & ‘song’\\

\textit{l}V$^2$\textit{l}V$^2$ & ‘new’    & \textit{lɔ}$^3$  &  ‘law’ \\

\textit{laa}$^2$ & ‘call’ &   \textit{lʌ}$^3$ &    ‘bring’ \\
\lspbottomrule
\end{tabular}
\caption{Godié (\textit{Kagbʊwalɪ} dialect, \citealt{Association2004})}
\label{tab:zogbo:6}
\end{table}


The adjective ‘new’ appears to be inherently +ATR and agrees with the noun it modifies (lolo, lala, lələ).

\begin{table}
\caption{Bakwé \citep{centredetraduction2006}}
\label{tab:zogbo:7}
\begin{tabular}{llllll}
\lsptoprule
\textit{pa}$^3$ &   ‘enter’  &  \textit{gɔ}$^4$    & ‘to be old’ &  \textit{go}$^4$ &    ‘to dig’\\

\textit{pʌ}$^4$  &   ‘share’    & \textit{ga}$^4$ &   ‘vines’     & \textit{gʌ}$^4$ &    ‘affair’\\

\textit{bɨ}$^2$\textit{ti}$^3$ & ‘thorn’ &    \textit{gɨ}$^4$ &   ‘plants’  & \textit{gʊ}$^4$ &    ‘tail’\\

\textit{bʌ}$^3$ & ‘to be’ &    \textit{gʉ}$^4$ &   ‘eggs’  &   \textit{gə}$^4$/\textit{gɪ}$^4$ & ‘egg’\\

\textit{bə}$^3$ & ‘to tap’ &    \textit{gu}$^4$   & ‘to give birth’ &  \textit{gɛ}$^4$  &   ‘vine’\\

\textit{bʉ}$^2$ & ‘ball (of something)’\\

\textit{bɨ}$^2$ & ‘balls’ (\textsc{pl})\\

\textit{ba}$^2$\textit{li}$^2$ & ‘pick up’\\
\lspbottomrule
\end{tabular}
\end{table}

Within these systems, central vowels follow the rules of \isi{vowel harmony}, with typical +ATR word-internal constraints:   

 

\begin{table}
\caption{Guibéroua Bété \citep{Werle1976}}
\label{tab:zogbo:8}
\begin{tabular}{llll}
\lsptoprule
\multicolumn{2}{c}{−ATR} & \multicolumn{2}{c}{+ATR}\\
\midrule
\textit{kʉ}$^3$\textit{ɓʌ}$^3$ &  ‘to grab’ &      \textit{ko}$^4$\textit{su}$^2$ &   ‘fire’\\

\textit{tɪ}$^2$\textit{mʉ}$^2$ & ‘to pay the dowry’  &  \textit{wuə}\textsuperscript{2-4} & ‘all’\\

\textit{gwʌ}$^1$\textit{zɪ}$^3$ &  ‘medecine’      & \textit{nuə}\textsuperscript{1-1}  &   ‘mouth’\\
\lspbottomrule
\end{tabular}
\end{table}

\begin{table}
\label{tab:zogbo:9}
\caption{Gbawale \citep{Seri1987}}
\begin{tabular}{llll}
\lsptoprule
\multicolumn{2}{c}{−ATR} & \multicolumn{2}{c}{+ATR}\\

\textit{wʌ}$^3$\textit{lɪ}$^3$  &  ‘problem’  &     \textit{di}$^4$\textit{gbə}$^3$ & ‘mortar’ \\

\textit{kɔ}$^4$\textit{kwɛ}$^1$ & ‘chicken &      \textit{go}$^4$\textit{və}$^3$ &  ‘tree trunk’\\

\textit{sɪ}$^1$\textit{kʌ}$^1$ & ‘rice’  &       \textit{do}$^4$\textit{pe}$^1$ &  ‘proper name’ \\ 

\textit{zɪ}$^3$\textit{kpɔ}$^4$ & ‘tomorrow’   &    \textit{bi}$^2$\textit{do}$^4$ &  ‘to wash (oneself)\\

\textit{mɔ}$^4$\textit{mʉ}$^3$ & ‘you’ (indep)   &   \textit{ci}$^3$\textit{gbe}$^4$ & ‘yesterday’\\
\lspbottomrule
\end{tabular}
\end{table}

As in most of these languages, /a/ occurs with both series: 

 

\begin{table}
\label{ex:zogbo:10}
\caption{Gbawale \citep{Seri1987}}
\begin{tabular}{llll}
\multicolumn{2}{c}{−ATR} & \multicolumn{2}{c}{+ATR} \\

\textit{pɪa}  & ‘buy’ & \textit{a}$^4$\textit{zie}$^3$ & ‘proper name*’ \\ 

\textit{a}$^4$\textit{mʉ}$^1$ & ‘me’ \\

\textit{wa}$^2$\textit{mʌ}$^3$ & ‘them’ \\
\end{tabular}
\end{table}

These systems of 13 phonologically contrastive vowels constitute the largest oral vowel systems in the \ili{Kru} language family.  

\subsection{Innovation of central vowels}\label{sec:zogbo:2.1} 

Given that, with the exception of \ili{Bakwé}, no central vowels occur in Western \ili{Kru}, and that within Eastern \ili{Kru}, several languages have no central vowels other than /a/, we are assuming Proto \ili{Kru} had a basic oral 9 vowel system, as in \ili{Kouya} today, with no central vowels.  Central vowels would represent an important innovation in a defined area and/or sub-branch.  In the following map, the dark black line indicates the main West-East divide in \ili{Kru}.  Areas where full sets of 5 central vowels (darker blue) occur are distinguished from those with no central vowels (rose) and those with an incomplete set (lighter blue). As will be discussed later, the distribution of central vowels suggest an areal spread, across the West-East border, and outside of \ili{Kru} into \ili{Dan}, a \ili{Mande} language.


\begin{figure}
\includegraphics[width=0.7\textwidth]{figures/fig-zogbo-1.png}
\caption{Map 1}
\label{fig:zogbo:1}
\end{figure} 

\subsection{Languages without the full set of central vowels}\label{sec:zogbo:2.2} 

The top languages in \tabref{tab:zogbo:4} (\ili{Godié}, \ili{Guibéroua} \ili{Bété}, etc.) along with Western \ili{Bakwé} (all in darker blue) appear to be the locus of a major innovation which has not (yet?) affected some of the Eastern languages such as \ili{Neyo}, \ili{Kouya}, \ili{Gbadi}, and various dialects of \ili{Dida}. Examining those languages which have partial sets (light blue) may provide clues as to how full central series developed in certain languages. 

\textbf{\ili{Daloa} Bété} slightly east and north of \ili{Guibéroua}-\ili{Godié}-\ili{Bakwé} has three non-low central vowels (+ATR) but no low –ATR /ʌ/ (\citealt{Zogbo2005}). /a/ occurs with both sets of +ATR vowels. This system is not as symmetrical as those three mentioned above. However, as far as we know, this dialect shows no signs of developing the –ATR counterpart /ʌ/: 

\begin{table}
\begin{tabular}{ccc}
i  &  ɨ (ɯ) &   u\\

ɪ  &  ʉ (ɣ)  &  ʋ\\

e  &  ə  &  o\\

ɛ  &   &   ɔ\\

&a&\\
\end{tabular}
\caption{Kpↄkolo phonetic vowels}
\label{tab:zogbo:11}
\end{table}

\textbf{Kpɔkolo} is a dialect of \ili{Bété} spoken in 20 villages south of Gagnoa. \citeauthor{Goprou2014} (\citeyear{Goprou2010}; \citeyear[175, 179]{Goprou2014}) cites the following phonetic vowel chart: 

\begin{table}
\caption{Kpↄkolo phonemes}
\label{tab:zogbo:12}
\begin{tabular}{ccc}
I  &  ɨ  &  u\\

ɪ  &  θ  &  ʋ\\

e   & &    o\\

ɛ   & ʌ  &  ɔ\\

	& a\\
\end{tabular}	
\end{table}

He notes there are no contrastive minimal pairs for [ɨ] and [θ] except in sin\-gular-plu\-ral forms. He thus analyzes the two high central vowels as allophones of high front vowels high /i/ and /ɪ/, an analysis which might provide some insight into how central vowels develop historically. \citet{Vahoua2011}, however, provides good evidence that /ʌ/ has phonemic status in this dialect.    

\textbf{\ili{Ga}ɓʊgbʊ} spoken in Gagnoa, Lakota (to the south), and the villages in between, attests 11 oral vowels, including two high central (or back unrounded ones, \citealt[5, 9]{Gnahore2006}). 

\begin{table}
	\caption{Gaɓʊgbʊ phonemic vowels}
	\label{tab:zogbo:13}
	\begin{tabular}{ccc}
i  &  ɯ  &   u\\

ɪ    & ɣ &   ʋ\\

e    & &   o\\

ɛ    & &  ɔ\\

	& a\\
	\end{tabular}
\end{table}


If the two high central vowels are truly phonemic, this language may be one step further than \ili{Kpɔkolo} in the development of central vowels. Typical \isi{vowel harmony} is present, with /a/ classified as –ATR.\footnote{Gnahoure’s vowel chart presents the two high central vowels as –ATR (p. 9): \textit{ɔzwa ja}\textbf{\textit{ma}} ‘Ozoua became light’ and \textit{Jai nyu}\textbf{\textit{m}}\textbf{\textit{Ɣ}} ‘Jai became ugly’ (\citeyear[25]{Gnahore2006}).  In her examples /a/ combines with both +ATR:  ga$^4$ji$^1$ ‘proper name’, a$^4$mɪ$^3$ ‘1 \textsc{sg}’ (obj).  More study is needed on how central vowels and the feature ATR combine.} 

\textbf{\ili{Guébie}:} This language, on the border between \ili{Bété} and \ili{Dida}, attests only one central vowel, phonetically higher than /a/. Hannah Sande (p.c.) reports that /a/ functions as –ATR and the higher central vowel as +ATR [ə].  As in other \ili{Kru} languages, /a/ shows a tendency to occur with both + ATR.  Sande notes an /-a/ suffix remains constant, no matter the ATR feature of a verbal root. 

\begin{table}
\caption{Vata vowels according to \citet{Kaye1980}}	
\label{tab:zogbo:14}
\begin{tabular}{lllllll}
\multicolumn{3}{c}{+ATR} && \multicolumn{3}{c}{−ATR}\\ 
i  &&  u  &~&  ɪ  &&  ʊ\\

e  &&  o  &&  ɛ  &&  ɔ\\

& ə   &&&&     a \\
	\end{tabular}
\end{table}

\textbf{Vata}: Like \ili{Guébie}, \ili{Vata} shows signs of shifting to a 10 vowel system, with \citet[70]{Kaye1980} also reporting an additional central vowel as part of the +ATR series. He notes “The 10\textsuperscript{th} vowel, i.e. the advanced low vowel is not pronounced in the speech of all \ili{Vata} speakers.  Nevertheless, there are reasons to justify in every \ili{Vata} dialect, a system of 10 vowels”. In the following chart we suspect that what is marked as /ʌ/ corresponds to what most \ili{Kru} researchers would write as /ə/, a  +ATR vowel phonetically higher than /ʌ/:


\begin{table}
\caption{Vata vowels reanalyzed}	
\label{tab:zogbo:15}
\begin{tabular}{lllllll}
\multicolumn{3}{c}{+ATR} && \multicolumn{3}{c}{−ATR}\\ 
i  &&  u  &~&  ɪ  &&  ʊ\\

e  &&  o  &&  ɛ  &&  ɔ\\

& ʌ   &&&&     a \\
	\end{tabular}
\end{table}

{\noindent\textbf{Gbadi}}: Curiously, though \ili{Gbadi} is classified as \ili{Bété}, with the exception of /a/, it attests no central vowels (C. Goprou, p.c. \& H. Tebili, word list), underlining the issue of frequent mismatches between ethnic/social perceptions and linguistic classifications. 

What is striking here is that some languages seem to be developing central vowels “from the top”, with high central vowels (\ili{Gaɓʊgbʊ}, \ili{Daloa} \ili{Bété}), while others (\ili{Vata}, \ili{Guébie}) appear to be developing them “from the bottom”.  In \ili{Kpɔkolo}, it would appear a lower central vowel /ʌ/ has become phonemic, but it may be the two higher conditioned central vowels will one day become phonemic as well. 

\section{Historic sources for central vowels}\label{sec:zogbo:3}

Based on the hypothesis that Proto \ili{Kru} had a nine vowel oral system, the source of central vowels will now be examined.  Our research shows that these vowels develop from both front and back vowels as well as central /a/, but the most frequent cases involve front vowels *i, *ɪ, and *e, and central *a.  It is important to note that the emergence of central vowels in \ili{Kru} never results in the disappearance of peripheral vowels in any given vocalic system.  

Below reconstructions from Proto Eastern \ili{Kru} (PEK) are proposed and traced to their current forms mainly in \ili{Godié}, a language which shows a very high number of central vowels.  In almost all cases the central vowel in question retains the same features for vowel height and +ATR as the proto form.  Here we concentrate on \textit{sources} of innovated central vowels, being able to identify very few conditioning factors. 

However, it will be noted that a very frequent environment for central vowels to emerge is in the environment of CLV, a fact which will be discussed below.  Note that in virtually all \ili{Kru} languages, /l/ has a variety of allophones (flap n, l, r) in CLV and in some languages implosive ɗ, \citealt{Marchese1979/1983}). Dialects of \ili{Godié} are cited when known (jlʉkɔwalɪ, kagbʊwalɪ, and koyo).

%%please move \begin{table} just above \begin{tabular
\begin{table}
\caption{\textbf{*i $\rightarrow$ ɨ}  in Godié CLV}
\label{tab:zogbo:16}
\begin{tabularx}{\textwidth}{lp{1cm}lp{4.5cm}llp{2cm}}
\lsptoprule
 PEK & \textit{*ɓli}$^2$  or $^3$  & ‘fall’ & \ili{Kouya}, \ili{Gbawale}, \ili{Ga}ɓʊgbʊ & $\rightarrow$ & \textit{ɓlɨ}$^2$ & \ili{Godié}\\
 PEK & \textit{*zli/e} & ‘fish’ &  & $\rightarrow$ & \textit{zlɨ}$^2$ & kagbʊwalɪ\\
 PEK & \textit{*mli} & ‘bite’ &  & $\rightarrow$ & \textit{mlɨ}$^2$ & jlʉkɔwalɪ, kagbʊwalɪ, koyo\\\midrule
\multicolumn{7}{l}{In some CV words beginning with /l/, often pronounced as implosive ɗ (probably *ɗ)}\\\midrule
 PEK & \textit{*li}\textbf{$^2$}  & ‘eat’ & \ili{Kouya}, \ili{Dida}, \ili{Gbawale}, \ili{Ga}ɓʊgbʊ, \ili{Vata} & $\rightarrow$ & \textit{lɨ} & jlʉkɔwalɪ, kagbʊwalɪ\\
\lspbottomrule
\end{tabularx}
\end{table}

%%please move \begin{table} just above \begin{tabular
\begin{table}
\caption{\textbf{*ɪ $\rightarrow$ ʉ} in Godié}
\label{tab:zogbo:17}
\begin{tabularx}{\textwidth}{lp{1cm}lp{3.1cm}llp{1.75cm}}
\lsptoprule
 PEK & \textit{*ŋlɪ}$^1$  & ‘name’  & \ili{Neyo}, \ili{Dida} \ili{Guibéroua} \ili{Bété}, \ili{Daloa} & $\rightarrow$ & \textit{ŋlʉ}$^1$ & \ili{Godié}, \ili{Koyo} [\textit{ŋňʉ}$^1$]\\
 PEK & \textit{*dɪ}$^2$ & ‘news’ & \ili{Dida}, \ili{Daloa}  Bété; \ili{Kouya} \textit{dɪ}$^1$  ‘chat’, ‘talk’ & $\rightarrow$ & \textit{dʉ}$^1$ & \ili{Godié}, \ili{Koyo} [\textit{dɨ}$^1$]\\
 PEK & \textit{{*a}}{$^4$}\textit{{m}ɪ}$^1$ & {'1 \textsc{sg emph}'} { }{}  & {\ili{Kouya}, \ili{Ga}ɓʊgbʊ} & $\rightarrow$ & \textit{{a}}{$^3$}\textit{{m}ʉ}$^1$ & \ili{Godié}\\
 PEK & \textit{*nɪ}$^1$  & ‘and (then)’ & \ili{Guibéroua} \ili{Bété} & $\rightarrow$ & \textit{nʉ}$^1$ & \ili{Godié}\\
\lspbottomrule
\end{tabularx}
\end{table}
 
\begin{table}
\caption{\textbf{*a $\rightarrow$ʌ} in Godié}
\label{tab:zogbo:18}
\begin{tabularx}{\textwidth}{lp{1cm}lp{3.1cm}llp{2cm}}
\lsptoprule
 PEK & \textit{*mla}$^2$ & ‘swallow’ & \ili{Dida}, \ili{Koyo}, \ili{Neyo}, \ili{Guibéroua}, \ili{Daloa} & $\rightarrow$ & \textit{mlʌ}$^2$ & \ili{Godié}  [\textit{mʌňʌ}]\\
 P\ili{Kru} & \textit{*mla}\textsuperscript{1/2} & ‘drink’ & \ili{Ga}ɓʊgbʊ \textit{mla}$^3$; Tepo \textit{mna}$^2$, \ili{Nyabwa} \textit{mna}$^2$ & $\rightarrow$ & \textit{mlʌ}$^1$ & \ili{Godié} [\textit{mʌňʌ}$^1$]\\
 PEK & \textit{{*kwala}}{$^{12}$}{}  & {‘tortoise’}  & {\ili{Kouya} \textit{kwlaa}}{$^{12}$ }{; \ili{Ga}ɓʊgbʊ \textit{kwala}}{$^{12}$} & $\rightarrow$ & \textit{{kwl}ʌ}{$^{12}$} & \ili{Godié}\\
 PEK & \textit{*kpa}$^2$\textit{la}$^2$ & ‘bottle’ & \ili{Bakwé} & $\rightarrow$ & \textit{kpʌlʌ}$^1$ & \ili{Godié} (Kagbo)\\
 PEK & \textit{*sa} & ‘pick (up)’ & \ili{Dida}, \ili{Ga}ɓʊ{gbʊ}, cf. \ili{Wobe} \textit{saa} ‘choose’ & $\rightarrow$ & \textit{sʌ} & \ili{Godié}\\
 PEK & \textit{*ka$^3$} & ‘have’ & \ili{Kouya}  \textit{ka}$^3$ & $\rightarrow$ & \textit{kʌ}$^3$ & \ili{Godié}, \ili{Gbawale}   \\
 PEK & \textit{{*ga}}{$^3$}{}  & {‘to be awake’} & \ili{Kouya} & $\rightarrow$ & \textit{{g}ʌ}$^3$ & \ili{Godié}\\
\lspbottomrule
\end{tabularx}
\end{table}
\textbf{Proto back vowels} may also give central reflexes, though not as frequently and perhaps following a more complicated path \ref{sec:zogbo:3}.

%%please move \begin{table} just above \begin{tabular}
\begin{table}
\caption{\textbf{*ʊ $\rightarrow$ ʉ}}
\label{tab:zogbo:19}
\begin{tabularx}{\textwidth}{lp{1cm}lp{4.5cm}llp{2cm}}
\lsptoprule
 PEK & \textit{*zʊ} & ‘shame’ & \ili{Neyo} \textit{zʊʊ}\textsuperscript{2-3}, \textit{zʊ}$^1$, \ili{Daloa} \textit{zʊ}$^2$ & $\rightarrow$ & \textit{zʉ}$^3$ & \ili{Godié}\\
 PEK & \textit{*mʊ}$^2$ & ‘go’ & \ili{Dida} & $\rightarrow$ & \textit{mʉ}$^2$ & \ili{Godié}\\
 PEK & \textit{*ngbʊ} & {‘five’} & {Kouya} & $\rightarrow$ & \textit{n}$^3$\textit{gbʉ}$^2$ & \ili{Godié}\footnotemark{}\\
\lspbottomrule
\end{tabularx}
\end{table}
\footnotetext{ See also \textit{n$^4$}\textit{gbɤ$^3$},\textsuperscript{} Kodia (Leidenfrost, p.c.).}

Note that examples of proto back vowels *u, *o, and *ɔ giving a central reflex are rare.  One example might be PEK \textit{*ɓlo} ‘one’ $\rightarrow$ \textit{ɓlʉ}  (\ili{Godié}, jlʉkɔ dialect). Cases of low vowels *Ɛ and *ɔ giving a central vowel are equally rare.

\section{Mechanisms for central vowel development}\label{sec:zogbo:4}

The question as to how these phonologically contrastive central vowels developed from an original 9 vowel proto system is a main concern here. What caused languages to move from a seemingly stable Proto system towards a more complex one, with full or partial sets of central vowels? For the moment, putting aside the question of \isi{language contact} and areal features, we will explore possible phonetic and phonological explanations of this development. 

\subsection{Phonetically motivated centralization}\label{sec:zogbo:4.1}

Of course the development of central vowels is not unique to \ili{Kru} or to Africa. Central vowels involve less tongue displacement than peripheral vowels. Thus quite naturally many languages develop central allophones. \citeauthor{Welmers1973} (\citeyear[23, 25]{Welmers1973}) notes phonetically conditioned centralizing tendencies of both front and back vowels in certain \ili{Mande} languages, for example \ili{Kpelle} where “short front unrounded vowels /i, e, ɛ/ have centralized allophones [ɨ,ə] after most consonants in monosyllables and in some types of bisyllabic forms”. Within \ili{Kru}, \citet{Bentinck1978} notes centralized realizations in sentence final position and after labiovelars.

However, more compelling is what appears to be a universal tendency for central vowels to emerge in proximity to resonant liquids /r/ and /l/ as well as their nasal counterparts.  Well known examples are high front vowels becoming central in such environments in Middle \ili{English}, for example, with \textit{bird} losing its short “I” and evolving into a central vowel (\citeauthor{hickeyms}, MS). \citet[76]{Lynch2015} notes in Proto Oceanic a central vowel reflex in \ili{Iaai}: *o > i, ə, as when \textit{*roŋoR} ‘hear’ becomes /ləŋ/ or /liŋ/.  Though he cites no conditioning factor, the r-l connection seems clear. Closer to home, \citet{Morton2012} notes a high *ɪ gives rise to a high central /ɨ/ phoneme before liquids and nasals in \ili{Anii}, an \ili{Akan} language.  

In \ili{Kru} languages, where the typical syllable structures are V, CV, CVV, CCV (where C\textsubscript{2} is a \isi{liquid} or sonorant), many researchers note the appearance synchronically of a central \isi{transition vowel} in CLV words.  \citet[98]{Marchese1979/1983} initially described the phenomenon as following:  
\begin{quote}
In many cases, a \isi{transition vowel} appears between the first consonant and [l]. The quality of this vowel is determined by the main vowel.  If the main vowel [i.e. V2] is central or back, the \isi{transition vowel} is identical to the main vowel.  If it is a front vowel, the \isi{transition vowel} is generally a central vowel of the same height. 
\end{quote}

Obviously the vowel carries the same ATR feature as the primary vowel, as seen in \tabref{tab:zogbo:20}.

\begin{table}
\caption{Godié}
\label{tab:zogbo:20}
\begin{tabular}{llllll}
\lsptoprule
\multicolumn{3}{l}{front vowel} & \multicolumn{3}{l}{central and back vowel}\\
\midrule
/yli$^1$/  & [y\textsuperscript{ɨ}li]  &  ‘eye’  &  /ɓlɨ$^1$…kʊ$^1$/ & [ɓ\textsuperscript{ɨ}lɨ] & ‘pick up’\\

/gwlɛ/ & [gw\textsuperscript{ʌ}lɛ] & ‘remain’ & /ɓlʉ$^3$/  &  [ɓ\textsuperscript{ʉ}lʉ]  & ‘one (certain)’\\

&&& /plʌ$^2$/  &  [p\textsuperscript{ʌ}lʌ]  & ‘enter’\\
\lspbottomrule
\end{tabular} 
\end{table}

In \ili{Kouya}, an Eastern language with no contrastive central vowels, Saunders reports a phonetically predictable central \isi{transition vowel} which he writes as [ə], {usually when V is a front vowel or /a/}:\footnote{The exact nature of [ə] is not known, but Saunders (p.c.) reports there is no violation of + ATR harmony.}

\begin{table}
\caption{Kouya \citep{Saunders2009}}
\label{tab:zogbo:21}
	\begin{tabular}{p{1.5cm}ll}
\lsptoprule
{/yra}{$^3$}{/} &   {[y}{\textsuperscript{ə}}{ra]} & {‘to look at’}\\

{/plE}{$^2$}{/}  &  {[p}{\textsuperscript{ə}}{lE]} & {‘liver’} \\

{/fli}{$^{41}$}{/}  &  {[f}{\textsuperscript{ə}}{li]} & {‘forest’} \\
\lspbottomrule
	\end{tabular}
\end{table}

We note for back vowels, as in \ili{Godié}, the epenthetic vowel is identical to the primary vowel: /ɓlo/ [ɓ\textsuperscript{o}lo] ‘one’.  

For Western Glaro, where there are no central vowel phonemes, Wolfe (p.c.) reports that retracted /ɪ/ becomes central in fast speech in certain words such as /ny\textbf{ɩ}nɔ/ ‘woman/wife’. Note that here C\textsubscript{2} provides the expected liquid-nasal environment. 

Of course while current synchronic analyses vary, with some positing epenthetic vowel insertion and others an underlying dissyllabic C\textsubscript{1}V\textsubscript{1}C\textsubscript{2}V\textsubscript{2} with a subsequent reduction, it is clear that historically these sequences derived from dissyllables. Reduction to one syllable CLV is precipitated by C\textsubscript{2} being a \isi{liquid} or nasal sonorant and tones on both vowels being identical. Identical tones speed up the realization of the word, which \textit{favors} a centralized transitional vowel rather than a full one.  A difference in \isi{tone} on V\textsubscript{1} and V\textsubscript{2} blocks the reduction process.  Compare w\textbf{ɨ$^2$}li$^2$ ‘goat’ vs. w\textbf{o$^3$}lo$^4$ ‘look’ in \ili{Gbawale} \citep[20, 31]{Seri1987} or the \ili{Godié} examples in (9) to words like gɔ$^3$lʊ$^1$ ‘canoe’ and lu$^3$lu$^2$ ‘tamtam’ where no reduction occurs.   

Note, however, that in many languages, a reduced CLV functions synchronically as a single syllable (see \citealt{Gratrix1975} for \ili{Godié}){}.\footnote{Note also in all \ili{Kru} languages, when alveolars (+cor) are involved, /l/ $\rightarrow$ r, and the vowel disappears completely, for example,  t\textsuperscript{ʊ}lʊ ‘to blossom’ $\rightarrow$ [trʊ], enhancing perception as one syllable \citep{Marchese1979/1983}.} It is interesting to note, however, that linguists who are native speakers of \ili{Kru} languages often opt for C\textsubscript{1}V\textsubscript{1}C\textsubscript{2}V\textsubscript{2 .}  Thus \citet{Kipre2015} argues strongly for a synchronic underlying two \isi{syllable structure} in \ili{Daloa} \ili{Bété}. Guehoun, as well, as a native speaker of Lakota \ili{Dida}, notes in the case of CLV “the \isi{transition vowel} is predictable [but] “when enunciating the word, when they are asked to slow their speech or to pronounce the sequences of a words with insistence, they pronounce two syllables”. He also notes “a child learning to speak automatically says CLV words as CVLV, without the word becoming unintelligible.” (\citeyear[55–56]{Guehoun1993}). Thus Guehoun proposes /ngɛlɛ/ ‘odor’ for [nglɛ], and /kpo$^3$ke$^3$le$^3$/ ‘bench, chair’ for [kpokle].  

\subsection{Pathways of development}\label{sec:zogbo:4.2} 

While the above discussion shows that central vowels are phonetically predictable, it does not provide a pathway for these sounds becoming phonemic.  At this stage, considering the data, we can only suggest possible pathways.  However, \ili{Kpɔkolo} may serve as a good example of a language that appears to be currently developing central vowels.  In this language, \citet[191]{Goprou2014} notes centralization in a similar environment as outlined in the preceding section (liquid-sonorant), but with dissimilar tones. Another a native speaker linguist, he too posits identical vowels as underlying:

\ea \ili{Kpɔkolo}\\
\begin{tabular}{lll}
/bɪ$^4$lɪ$^2$/ & ‘neck’ & $\rightarrow$[bθlɪ]\\
/kɪ$^1$lɪ$^4$/ & ‘first’ & $\rightarrow$ [kθlɪ]\\
\end{tabular}
\z

He thus posits [θ] as an allophone of /ɪ/, and likewise for [ɨ] as an allophone of /i/.  He notes however that for the latter, there are some minimal pairs, but only in a singular-plural paradigm. As noted, this language has apparently developed a lower central phonemic vowel /ʌ/ \citep{Vahoua2011}. Our major problem is finding a pathway for development for these central vowels in \ili{Kpɔkolo} and other innovating languages. 

One possible pathway might be the development of a central vowel V\textsubscript{1} position and the loss of the final syllable CV\textsubscript{2}, leaving the new vowel in a contrastive CV\# position.  Unfortunately however, we have found few examples which could justify this scenario. Also arguing against this hypothesis is the fact that Western languages, showing the most word final syllable reduction, have not developed any central vowels.  Another possibility is that rightward assimilation (a common \ili{Kru} process in \isi{vowel harmony}) would affect V\textsubscript{2}, with V\textsubscript{1} taking on a central quality and then coming to dominate V\textsubscript{2}. This would give a central vowel in a primary vocalic position where it would come into contrast with peripheral vowels, for example:  \textit{kpala $\rightarrow$  kp\textsuperscript{ʌ}}\textit{la $\rightarrow$ kplʌ.}

\subsection{The effect of morpheme boundaries}\label{sec:zogbo:4.3} 

Examples above open up another possible pathway for central vowel development. \ili{Kru} languages are primarily suffixing.  Historically \isi{noun class} suffixes have interacted and often coalesced with stem final vowels. To these forms are added plural markers and, in some languages, definite suffixes \citep{Marchese1979/1983,Zogbo2017}.  Verbs as well carry object clitics but also aspectual markers, causative, and other transitivity-changing suffixes. In some of our data, these instances of vowels “coming together” at morpheme boundaries seems to effect word (and syllable?) structure, resulting in some centralizing phenomena. 

For example, the environment \textit{noun + class marker} is clearly to be reconstructed for Proto \ili{Kru}.  Did this environment create a context where central vowels emerged in a single syllable?  To give an example, current variant forms such as /kpʊ/ and /kpʉ/ ‘oil’ can be seen as deriving from *kpV + *ʊ, root + \isi{noun class} suffix. In all likelihood the form could have been *kpɪ+ *ʊ, where in some languages the first vowel was centralized, as in \ili{Godié} and \ili{Bété} (/kpʉ/). In others the initial stem vowel was lost and the \isi{noun class} marker took its place yielding (/kpʊ/).

It is worth noting that \ili{Kru} plurals—most often marked with human suffix \textit{{}-ʊa} or non human \textit{{}-ɪ}—have a peculiar feature of effecting upward centralization, a process which is hard to account for synchronically on a strictly phonological level in Eastern \ili{Kru} and \ili{Bakwé} \citep{Marchese1979/1983}. This is particularly predominant in \ili{Godié}, for example, in singular plural pairs such as li$^1$/l\textbf{ɨ}$^1$ ‘spear’, mɪɪ\textsuperscript{1-2} mʉʉ\textsuperscript{1-2} ‘boat’, kpʌ/kpʉ ‘herd’.  While mu + \textit{{}-ɪ} might give mʉʉ\textsuperscript{1-2}, it is hard to derive l\textbf{ɨ}$^1$ from li + \textit{{}-ɪ}.\footnote{According to morpho-phonological rules \textit{li + ɪ} should give lii (assimilation, \isi{vowel harmony}) and mɪɪ + ɪ, mɪɪɪ.} It is as if the mere presence of the plural morpheme boundary produces heightening and centralization. \citet{Goprou2014} also reports a similar centralization of back vowels (which he describes as unrounding, but could also be considered as fronting) in the environment of plural –ɪ. Thus \ili{Kpɔkolo} shows central vowels on the surface in plural forms but not in underlying ones:

\begin{table}
\label{tab:zogbo:22}
\caption{Kpɔkolo \citep[202-206]{Goprou2014}}                                  
\begin{tabular}{lllll}
\lsptoprule
/pʊ$^2$lʊ$^3$/ + ɪ 
& $\rightarrow$ /pʊlɪ/  
& $\rightarrow$ [pθlɪ]        
&
& ‘piece’ + \textsc{pl}   \\

/so$^4$lu$^2$/ + ɪ  
& $\rightarrow$ /solu + ɪ/ 
& $\rightarrow$ /soli/  
& $\rightarrow$ [sƔli]   
& ‘pail + \textsc{pl}’  \\

/kɔ$^2$lɪ$^2$/ + ɪ 
& $\rightarrow$ [kʌlɪ]     
& 
&     
& ‘bamboo + \textsc{pl}’   \\

/mu$^4$du$^2$/ + ɪ  
& $\rightarrow$ mudu + i  
& $\rightarrow$ mudi 
& $\rightarrow$ [m\textbf{ɨ}di]   
& ‘(finger)nail’ + \textsc{pl}\\
\lspbottomrule
\end{tabular}	
\end{table}

Note that this is basically the same CLV environment as the transition vowels in other languages\footnote{We might suspect that d in the last example is a reflex of *ɗ.}, and it is again a question of vocalic assimilation of back vowels moving front. \citet{Welmers1973} notes a similar “derounding” as well as fronting of back vowels /o/ and /ɔ/ in \ili{Kpelle} when followed by a front vowel, either directly or after an intervening /l, r, n/. As Goprou, he calls these centralized forms “allophones” of other vowels. Welmers notes however, that “native speaker reaction “strongly favors the interpretation of the underlying vowel, in this case /o/ and /ɔ/”. 

The data from \ili{Kpɔkolo} confirms yet again the “weakness” of the position of the first vowel in a CVCV [lateral/sonorant] word. Clearly the C\_LV environment lends itself to centralization in \ili{Kru} (and cross-linguistically), but the addition of a morpheme boundary seems to add “additional weight” to this tendency. For \ili{Koyo}, \citeauthor{Kokora1976} (another native speaker, \citeyear[39]{Kokora1976}) cites the form /mala+à/ [m\textbf{ɨ}lá-à] (drink-\textsc{perf} \textsc{past}) where in addition to the CVLV environment, we think the “added weight” of the rightward morpheme boundary causes the first /a/ to weaken, and here, to heighten as well. Another example comes from {\ili{Nyabwa} where no phonemic central vowels exist.} \citet[50]{Bentinck1978} reports phonetic centralizing of the vowel /e/, at the end of conjugated verbs in a CV + V environment:  /ɪn$^2$  li$^3$ e$^4$ pɪ$^2$tɛ$^4$/ (I eat-\textsc{suffix} banana) ‘I’m eating a banana’.  Word boundaries may also come into play, as seen in the following examples from Lakota \ili{Dida}, where \citet[47]{Guehoun1993} reports a phonetic /a/ $\rightarrow$ [ə] development, which seems a “change in progress”: 

\begin{table}
\caption{Lakota Dida} 
\label{tab:zogbo:23}
\fittable{
\begin{tabular}{lll}
\lsptoprule
/ɔ$^3$ sa$^3$ ka$^4$fɪ$^1$/     
& $\rightarrow$ [ɔ s\textbf{ə} k\textbf{ə}fɪ]    
& ‘He’s picking coffee’\\

/ɔ$^3$ la$^4$ du$^1$tʊ$^3$ bo$^3$du$^4$kwo$^2$/ 
& $\rightarrow$  [ɔ l\textbf{ə} dutʊ bodukwo] 
& ‘He brought a package to the village chief’  \\

/ɔ$^3$ ka$^4$ cɛ$^1$/     
& $\rightarrow$  [ɔ k\textbf{ə} cɛ]
& ‘He has noise (he’s loud)’\\
\lspbottomrule
\end{tabular}
}
\end{table}

Despite these various scenarios, we cannot say exactly \textit{how} allophones or phonetic realizations become contrastive phonologically.  Neither do we know if these changes occurred early on, i.e. high up in the Eastern \ili{Kru} tree and consequently spread, or even (though extremely less probably), whether the innovation occurred in \ili{Bakwé} and slowly spread eastward into Eastern \ili{Kru} (See discussion below). 

We do know, however, as is well attested in all types of linguistic change, that variation plays an important role in the adoption of central vowels. Indeed, in the \textit{kagbʊwalɪ} dialect of \ili{Godié}, \textit{mʊ} and \textit{mʉ} ‘go’ are in \isi{free variation}, while in the \textit{jlʉkɔ} dialect the central vowel has become the standard form. In Lakota \ili{Dida} \citet[48]{Guehoun1993} notes that /a/ and [ə] are often in free distribution, “…since a speaker can use or not use either realization without it affecting the meaning of the message.”  It would thus hardly be surprising if this dialect of \ili{Dida} develops a slightly higher phonemic central vowel to join /a/, with each occurring in its own separate harmonic set.      

\subsection{Pressure for symmetric systems}\label{sec:zogbo:4.4} 

\citet[501, 502]{Casali2008} notes that a 9 vowel systems with five [$-$ATR] and four [+ATR] vowels, where “a contrastive non-high [+ATR] counterpart of /a/…is absent” are “extremely common (numbering, by any reasonable estimate, in the hundreds) and are geographically and genetically widespread within Niger-Congo and \ili{Nilo-Saharan}”. He further notes that while 10 vowel languages are not the most common within NC, many ATR languages “have nine contrastive vowels, with a tenth vowel on the surface, a predictable [+ATR] variant of /a/ that occurs in the neighborhood of [+ATR] vowels”.  This seems to be Kaye’s mysterious 10\textsuperscript{th} vowel in \ili{Vata}.  Certainly however, while symmetry in vowel systems is not universal, it is common for languages to attempt to “round out” their vocalic systems \citep[21]{Welmers1973}. This tendency seems to be at work in \ili{Kru} today, for example, in \ili{Guébie}, where a 10\textsuperscript{th} vowel /ə/ seems to have emerged to balance out the +ATR \isi{vowel harmony} system (Sande, p.c.). 

One final observation seems important in regards to the high numbers of central vowels in some Eastern \ili{Kru} languages and \ili{Bakwé}.  It may be significant that in Western languages, where phonemic central vowels have not developed, there are full sets of nasalized vowels, whereas in the languages with central vowel phonemes, nasalized vowel phonemes do not exist or are marginal \citep{Marchese1979/1983}.  So it may be that the size of the \isi{vowel inventory} may be a factor in central vowel formation in \ili{Kru}. In Western \ili{Kru} the full \isi{vowel inventory} may have blocked the development of central vowels, due to limits on perception, while in Eastern \ili{Kru}, where nasalized vowels do not appear contrastively (and presumably may have been lost), space has been created to allow for such a development.  At this point, we cannot affirm this, but the complimentary distribution, noticed in other parts of Africa \citep{Rolle2013}, is most intriguing. Note that this explanation would work for \ili{Kru} but not for \ili{Dan} (southern \ili{Mande}) where both sets (central and nasalized) do co-occur (see below).

\section{The areal hypothesis}\label{sec:zogbo:5} 

Examining southern \ili{Mande} and \ili{Kru} languages, \citet[92, 112]{Vydrine2009} proposes an “Upper Guinean Coast Sprachband” sharing numerous features, including +ATR, \isi{vowel harmony}, a high \isi{vowel inventory} (7+), nasalized vowels, asymmetry of oral and nasalized vowels, lack of nasal consonants, at least three or more level tones, consonant homo-resonance, implosives, labiovelars, v and z, high frequency of CVV feet, locative nouns, and, importantly for this study, central or back unrounded vowels. While these observations are intriguing, it is important to note that some of the above features are not systemically shared by \textit{both} Western and Eastern \ili{Kru}. Thus, while most Western \ili{Kru} can be analyzed as having nasalized vowels with no nasal consonants, Eastern \ili{Kru} does not exhibit this behavior. And while Eastern \ili{Kru} attests central or back unrounded vowels, Western \ili{Kru} does not.  

In this section, we would like to consider the details and/or implications of areal sharing of central vowels as it affects this region. In exploring this areal hypothesis, several questions emerge:

\begin{itemize}
\item First, within \ili{Kru} itself, how much of the central vowel phenomenon is due to \textit{areal contact}?  Or are central vowels a result of \textit{genetic affiliation} (for example, an innovation in PEK occurring, say, before \ili{Guibéroua} \ili{Bété} and \ili{Godié} split)? \\[-0.75cm]
\item Regarding the \ili{Kru}-\ili{Mande} areal connection, what is the locus/source of central vowel innovation and which direction is the borrowing/\isi{language contact} going?\\[-0.75cm]
\item What factors might play a role in the spread of centralization? What are the possible scenarios and what might this tell us about the history of the \ili{Kru} peoples and their interaction with \ili{Mande} populations?
\end{itemize}

\subsection{Internal spreading of central vowels within the Kru language family}\label{sec:zogbo:5.1} 

Within Eastern \ili{Kru}, it is clear that central vowels are emerging, which may well be a case of family-internal areal spreading.  The question remains: are languages such as \ili{Guébie} and \ili{Kpɔkolo} adopting central vowels because of natural phonetic developments (internal phonological processes and pressure as described above), or rather, is this a case of \isi{language contact}? Or are both factors at work?  \ili{Kru} languages constitute complex and numerous dialect chains and when speaking, \ili{Kru} peoples regularly “switch back and forth”, adapting words to be understood by other \ili{Kru} speakers. Thus contact as well as phonological processes seem likely influences.    

Most noteworthy as a candidate for areal spreading is \ili{Bakwé}, traditionally considered a Western \ili{Kru} language.\footnote{Linguistic evidence confirms this classification, as well as strong oral tradition \citep{centredetraduction2013}.} This language seems to have acquired a full set of central vowels through \isi{language contact} or areal spreading.  Leidenfrost (p.c.) points out that the \ili{Bakwé}, who are a very small group, pride themselves in speaking other languages and in the fact that their neighbors cannot speak their language.  Though culturally they have been greatly influenced by Western Guere culture, having incorporated Guere masks (who, it turns out, must speak Guere!), their small number and sociolinguistic profile might make them susceptible to influence from adjacent and currently much larger \ili{Godié}-\ili{Guibéroua} \ili{Bété} groups to the east. Also note in \figref{fig:zogbo:1} \ili{Bakwé} is today separated from related Western languages by the huge Tai forest. However, questions remain. If this such contact and borrowing did occur, it is hard to know why \ili{Bakwé}, which is contiguous to \ili{Godié}, would borrow central vowels, while \ili{Kouya}, contiguous to \ili{Bété}, would resist incorporating them! Another hypothesis is that \ili{Bakwé} itself first innovated central vowels, which spread either to a Proto Eastern \ili{Kru} ancestor, or spread slowly (as is still happening) throughout Eastern \ili{Kru} (especially the \ili{Bété} complex), but this seems less probable.   

\subsection{Central vowels spreading across language families}\label{sec:zogbo:5.2} 

Cases of borrowing of central vowels across language families is not uncommon. M. Harley (p.c.), notes that in Western Chadic, Ywom and \ili{Goemai} with 7 vowels (including 3 central vowels), “appear to have developed a third central vowel through contact with the neighbouring \ili{Tarok} (a \ili{Benue-Congo} language), which has an identical 7-vowel system.”  Southern \ili{Mande} includes two \ili{Dan} languages with vowel systems which closely resemble \ili{Kru} systems, in that full series of central vowels are present. Eastern \ili{Dan} attests the following: 

\begin{table}
\caption{Eastern Dan vowels \citep{Vydrine2009}}
\label{tab:zogbo:24}
\begin{tabular}{llp{2cm}lll}
\lsptoprule
\multicolumn{3}{l}{Oral vowels} &     \multicolumn{3}{l}{Nasal vowels}\\
\midrule
i & ɯ & u  &  ĩ & ɯ & ũ\\

e & ɤ & o\\

ɛ & ʌ & ɔ  &  ɛ & ʌ & ɔ\\

æ & a & ɒ  &  \~{æ} & ã & ɒ
\lspbottomrule
\end{tabular}
\end{table}

With the exception of \ili{Goo}, other languages of the southern \ili{Mande} group and of other \ili{Mande} branches do not have central vowels.  Though it is possible that these languages underwent similar processes as \ili{Kru} in developing central vowels, \citet{Vydrine2009} is probably correct in assuming that these languages must have been influenced by \ili{Kru} languages through \isi{language contact}. This scenario is more likely (than the other way around, with \ili{Kru} borrowing from \ili{Mande}), since far more \ili{Kru} languages show centralization than is the case in \ili{Mande}, where, besides these 2–3 affected languages, central vowels are virtually unknown.  In the map below, we see \ili{Kru} languages with central vowels, those without and the area where they are attested in \ili{Mande} languages. 

\begin{figure}
\includegraphics[width=0.8\textwidth]{figures/fig-zogbo-2.png}
\caption{Map 2}
\end{figure}  

We note that \ili{Akye}, an \ili{Akan} language spoken by peoples who immigrated from Ghana, also attests two central vowels\footnote{Bogny, Joseph, “Typological features template for Attie”, \url{https://typecraft.org/tc2wiki/Typological_Features_Template_for_Attie}.} (ɤ and ʌ).  We have yet to investigate this link, which may point to another case of \isi{language contact} and areal spreading of central vowels in this region.

\section{Historical explanations}\label{sec:zogbo:6} 

As the above map demonstrates, one problem with the areal hypothesis concerning central vowels in subsets of southern \ili{Mande} and \ili{Kru} is that currently \ili{Dan} is separated from the centralizing Eastern languages by a huge space occupied by Western \ili{Kru}, where central vowels have not innovated. This fact suggests that historical explanations may need to come into play. If central vowels are a shared feature of \ili{Mande} and \ili{Kru}, this would suggest at some point the \ili{Dan} peoples and the ancestors of the \ili{Godié}-\ili{Bété} branch of Eastern \ili{Kru} and/or \ili{Bakwé} were geographically contiguous. Thus, in this case, linguistic evidence may help us determine certain people movements.   

Despite late oral traditions describing a \isi{movement} of \ili{Kru} peoples from west to east (i.e. from Liberia into the Ivorian forests), it is commonly accepted that the \ili{Kru} were once located much further north, and then were pushed down into the forest by the \ili{Mande} expansion. Citing Y. \citet{Person1964} and A. \citet{Schwartz1970}, S. \citet{Lafage1983} traces the \ili{Kru} immigrations towards the south in three stages:

\begin{itemize}
\item 14\textsuperscript{th} to 18\textsuperscript{th} century: Mandes and \ili{Kru} were positioned “on the Niger”, with the \ili{Mande} pushing small \ili{Kru} groups into the forest.\\[-0.75cm] 
\item 15\textsuperscript{th} century onward: the \ili{Kru} move towards the coast (in light of European trade, including the slave trade). \\[-0.75cm]
\item 18\textsuperscript{th} century: the arrival (in waves) of the \ili{Akan} from the East would have pushed the Krus further south and west. \citet[68]{kipre2005} notes as well that in the 18\textsuperscript{th} there were early \ili{Akan} infiltrations and a certain “akanization” of certain \ili{Dida} villages.
\end{itemize}

Though the individual \ili{Kru} groups appear to be fairly autonomous, Kipre also notes a high level of contact not only between \ili{Kru} themselves, but between \ili{Kru} and \ili{Mande} groups, describing a process of “compression”: 

[in Côte d’Ivoire] …several peoples were in contact with one another, interpenetrating each other, whether easily or not, certainly not without conflict.  There were frequent confrontations between \ili{Gouro} and \ili{Bété}, between \ili{Gagou} and \ili{Bété}, between \ili{Dan} and \ili{Wè} during this time frame. Also we have here a “transition zone” where several peoples are \textbf{pressed together in a kind of “metissage culturel”…} Niabwa and \ili{Nidedboua} are squeezed between \ili{Wè} and \ili{Bété}; the \ili{Bakwé}’s are squeezed in between the Southern \ili{Kru} and the \ili{Bété}….”


This kind of geographic as well as socio-cultural ‘compression’ point to conditions which could easily lead to linguistic borrowing and the development of areal features.  Kipre goes on to note (\citeyear[69]{kipre2005}) that within Côte d’Ivoire the “progressive interpenetration of peoples makes the idea of ethnic groups as ‘pure peoples’ (or races) inappropriate”.     


What do these facts tell us?  Probably that present territorial placements of various ethnic groups do not reflect past history. It is likely, for example, that the \ili{Dan} tribes came into contact at an earlier period with parts of what today is the \ili{Godié}-\ili{Bété} branch of Eastern \ili{Kru}, where central vowels were innovated. Despite the fact that the \ili{Mande} would be considered the “dominators” over the last three to four centuries (\citealt{Lafage1983}; \citealt[108]{Vydrine2009}), it is possible that the \ili{Mande} super-stratum assimilated some of the substratum language features, especially on the phonological level.  Recent scholarship suggests other “higher” areal features for a wider region such as a common S AUX O V word order may have come from the other direction, namely from \ili{Mande} to \ili{Kru} (\citealt{guld08}; Sande et. al., this volume).  Besides past historical contact and borrowing, it is clear that foreigners of all provenances (\ili{Mande}, \ili{Akan}, etc.) have penetrated and continue to penetrate into the rich and fertile \ili{Kru} territory.\footnote{\citet[54]{Lafage1983} notes for example that in Côte d’Ivoire today in \ili{Kru} regions, Krus are in the minority, for example in the prefecture of \ili{Daloa}, prior to 1980,  the following figures held: \ili{Kru}  (from the region) 27.81 \%; Non Ivoirians, 25.49 \% ; \ili{Akan}, 18.74 \%; N. \ili{Mande},13.64\%; S \ili{Mande}, 9.71\%; \ili{Gur}, 4.57.} Will such mixing lead to more language change and sharing of other linguistic features? 

\section{Conclusions}\label{sec:zogbo:7} 

In this study, we have tried to go beyond Vydrine’s initial observations (\citeyear{Vydrine2009}), to study in some detail the innovation of central vowels in a subset of Eastern \ili{Kru} languages, with the locus of initial change presumably being the \ili{Godié}-\ili{Guibéroua} \ili{Bété} complex, possibly before this group subdivided into today’s individual languages. It seems highly probable that \ili{Bakwé}, a Western \ili{Kru} language, but contiguous to \ili{Godié}, has acquired central vowels through \isi{language contact}.  It may be the case that current central vowel innovations maybe constitute cases of \isi{language contact} within the \ili{Kru} group itself. However, Western \ili{Kru} has, for whatever reasons, resisted any such innovation, perhaps due to its already very full \isi{vowel inventory}.  

In terms of the wider region, it would appear that two or three southern \ili{Mande} languages have indeed incorporated central vowels through \isi{language contact}, despite what appears to be a dominator-dominated social scenario.\footnote{Bonny Sands (p.c.) suggests that in some African cultures, speaking “differently” is a way for leaders to gain social status and upward mobility.  Could this be behind the adoption of \ili{Kru} central vowels among the \ili{Dan} dialects?} Our data might suggest that the innovation of central vowels in \ili{Godié}-\ili{Bété} occurred rather early, that the \ili{Dan}-\ili{Kru} contact occurred sometime after that, but still quite some time ago, in a linguistic and geographical setting quite different from that of today.  It is possible the \ili{Godié}-\ili{Guibéroua} \ili{Bété} were initially in closer geographic contact with \ili{Dan}-Glio than Western \ili{Wè} was (currently contiguous to \ili{Dan}), and that the \ili{Godié}-\ili{Guibéroua} \ili{Bété} group “moved on”, pushing further down into the forest into their current position, while the \ili{Wè} peoples seem to have moved in between them and their \ili{Mande} neighbors. It remains to be seen if any traditional accounts or historical evidence exists to justify such a scenario.  

The conditions and mechanisms leading to central vowel innovation are multiple and certainly have not all been identified.  The means by which areal features propagate is also not clear, but hopefully we are beginning to better understand these kinds of phenomena, and we may learn more as we continue to watch central vowels emerging (and perhaps spreading) within \ili{Kru} (and beyond).\footnote{The examination of \ili{Akye} central vowels is certainly a \isi{subject} for further research.} 

\section*{Acknowledgements}


Thanks to H. Sande, J. Singler and V. Vydrine for comments on this and/or a previous version of this paper, to Dada Jean, I. Egner, P. Saunders, C. Leidenfrost, C. Wuesthoff and L. Wolfe for information on data as well as a special thanks to B. Alvarez for maps and H. Sande for formatting. \ili{French} quotations are translated by myself.

\section*{Abbreviations}
\begin{tabularx}{.55\textwidth}{ll}
1 & \isi{first person} \\
\textsc{perf} & perfective \\
\textsc{pl} & plural \\
\textsc{sg} & singular \\
\end{tabularx}

\sloppy
\printbibliography[heading=subbibliography,notkeyword=this]

\end{document}
