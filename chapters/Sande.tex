\documentclass[output=paper,newtxmath,modfonts,nonflat,draftmode]{langsci/langscibook}
\ChapterDOI{10.5281/zenodo.3367195}

\author{Hannah Sande\affiliation{Georgetown University}\and Nico Baier\affiliation{McGill University}\lastand Peter Jenks\affiliation{UC Berkeley}}
\title{The syntactic diversity of SAuxOV in West Africa}

\IfFileExists{../localcommands.tex}{%hack to check whether this is being compiled as part of a collection or standalone
  \usepackage{pifont}
\usepackage{savesym}

\savesymbol{downingtriple}
\savesymbol{downingdouble}
\savesymbol{downingquad}
\savesymbol{downingquint}
\savesymbol{suph}
\savesymbol{supj}
\savesymbol{supw}
\savesymbol{sups}
\savesymbol{ts}
\savesymbol{tS}
\savesymbol{devi}
\savesymbol{devu}
\savesymbol{devy}
\savesymbol{deva}
\savesymbol{N}
\savesymbol{Z}
\savesymbol{circled}
\savesymbol{sem}
\savesymbol{row}
\savesymbol{tipa}
\savesymbol{tableauxcounter}
\savesymbol{tabhead}
\savesymbol{inp}
\savesymbol{inpno}
\savesymbol{g}
\savesymbol{hanl}
\savesymbol{hanr}
\savesymbol{kuku}
\savesymbol{ip}
\savesymbol{lipm}
\savesymbol{ripm}
\savesymbol{lipn}
\savesymbol{ripn} 
% \usepackage{amsmath} 
% \usepackage{multicol}
\usepackage{qtree} 
\usepackage{tikz-qtree,tikz-qtree-compat}
% \usepackage{tikz}
\usepackage{upgreek}


%%%%%%%%%%%%%%%%%%%%%%%%%%%%%%%%%%%%%%%%%%%%%%%%%%%%
%%%                                              %%%
%%%           Examples                           %%%
%%%                                              %%%
%%%%%%%%%%%%%%%%%%%%%%%%%%%%%%%%%%%%%%%%%%%%%%%%%%%%
% remove the percentage signs in the following lines
% if your book makes use of linguistic examples
\usepackage{tipa}  
\usepackage{pstricks,pst-xkey,pst-asr}

%for sande et al
\usepackage{pst-jtree}
\usepackage{pst-node}
%\usepackage{savesym}


% \usepackage{subcaption}
\usepackage{multirow}  
\usepackage{./langsci/styles/langsci-optional} 
\usepackage{./langsci/styles/langsci-lgr} 
\usepackage{./langsci/styles/langsci-glyphs} 
\usepackage[normalem]{ulem}
%% if you want the source line of examples to be in italics, uncomment the following line
% \def\exfont{\it}
\usetikzlibrary{arrows.meta,topaths,trees}
\usepackage[linguistics]{forest}
\forestset{
	fairly nice empty nodes/.style={
		delay={where content={}{shape=coordinate,for parent={
					for children={anchor=north}}}{}}
}}
\usepackage{soul}
\usepackage{arydshln}
% \usepackage{subfloat}
\usepackage{langsci/styles/langsci-gb4e} 
   
% \usepackage{linguex}
\usepackage{vowel}

\usepackage{pifont}% http://ctan.org/pkg/pifont
\newcommand{\cmark}{\ding{51}}%
\newcommand{\xmark}{\ding{55}}%
 
 
 %Lamont
 \makeatletter
\g@addto@macro\@floatboxreset\centering
\makeatother

\usepackage{newfloat} 
\DeclareFloatingEnvironment[fileext=tbx,name=Tableau]{tableau}
  %add all your local new commands to this file
\newcommand{\downingquad}[4]{\parbox{2.5cm}{#1}\parbox{3.5cm}{#2}\parbox{2.5cm}{#3}\parbox{3.5cm}{#4}}
\newcommand{\downingtriple}[3]{\parbox{4.5cm}{#1}\parbox{3cm}{#2}\parbox{3cm}{#3}}
\newcommand{\downingdouble}[2]{\parbox{4.5cm}{#1}\parbox{6cm}{#2}}
\newcommand{\downingquint}[5]{\parbox{1.75cm}{#1}\parbox{2.25cm}{#2}\parbox{2cm}{#3}\parbox{3cm}{#4}\parbox{2cm}{#5}}
\newcolumntype{Y}{>{\centering\arraybackslash}X}
\newcolumntype{T}{>{\centering\arraybackslash}m{2cm}}

%commands for Kusmer paper below
\newcommand{\ip}{$\upiota$}
\newcommand{\lipm}{(\_{\ip-Max}}
\newcommand{\ripm}{)\_{\ip-Max}}
\newcommand{\lipn}{(\_{\ip}}
\newcommand{\ripn}{)\_{\ip}}
\renewcommand{\_}[1]{\textsubscript{#1}}


%commands for Pillion paper below
\newcommand{\suph}{\textipa{\super h}}
\newcommand{\supj}{\textipa{\super j}}
\newcommand{\supw}{\textipa{\super w}}
\newcommand{\ts}{\textipa{\t{ts}}}
\newcommand{\tS}{\textipa{\t{tS}}}
\newcommand{\devi}{\textipa{\r*i}}
\newcommand{\devu}{\textipa{\r*u}}
\newcommand{\devy}{\textipa{\r*y}}
\newcommand{\deva}{\textipa{\r*a}}
\renewcommand{\N}{\textipa{N}}
\newcommand{\Z}{\textipa{Z}}
% 

%commands for Diercks paper below
\newcommand{\circled}[1]{\begin{tikzpicture}[baseline=(word.base)]
\node[draw, rounded corners, text height=8pt, text depth=2pt, inner sep=2pt, outer sep=0pt, use as bounding box] (word) {#1};
\end{tikzpicture}
}

%commands for Pesetsky paper below
% \newcommand{\sem}[2][]{\mbox{$[\![ $\textbf{#2}$ ]\!]^{#1}$}}
\newcommand{\sem}[2][]{\mbox{$[[ $\textbf{#2}$ ]]^{#1}$}}

% \newcommand{\ripn}{{\color{red}ripn}}%this is used but never defined. Please update the definition



%commands for Lamont paper below
\newcommand{\row}[4]{
	#1. & 
    /{#2}/ & 
    [{#3}] & 
    `#4' \\ 
}
%\newcounter{tableauxcounter}
\newcommand{\tabhead}[2]{
%     \captionsetup{labelformat=empty}
%     \stepcounter{tableauxcounter}
%     \addtocounter{table}{-1}
% 	\centering
% 	\caption{Tableau \thetableauxcounter: #1}
	\caption{#1}
	\label{#2}
}
\newcommand{\candref}[2]{{(\ref{#1}#2)}}
\newcommand{\tableauref}[1]{{Tableau~\ref{#1}}}
% tableaux
\newcommand{\inp}[1]{\multicolumn{2}{|l||}{{#1}}}
\newcommand{\inpno}[1]{\multicolumn{2}{|l||}{#1}}
\newcommand{\g}{\cellcolor{lightgray}}
\newcommand{\hanl}{\HandLeft}
\newcommand{\hanr}{\HandRight}
\newcommand{\kuku}{Kuk\'{u}}

% \newcommand{\nocaption}[1]{{\color{red} Please provide a caption}}

% \providecommand{\biberror}[1]{{\color{red}#1}}

\definecolor{RED}{cmyk}{0.05,1,0.8,0}


\newfontfamily\amharicfont[Script = Ethiopic, Scale = 1.0]{AbyssinicaSIL}
\newcommand{\amh}[1]{{\amharicfont #1}}

% 
% %Gjersoe
\usepackage{textgreek}
% 
\newcommand{\viol}{\fontfamily{MinionPro-OsF}\selectfont\rotatebox{60}{$\star$}}
\newcommand{\myscalex}{0.45}
\newcommand{\myscaley}{0.65}
%\newcommand{\red}[1]{\textcolor{red}{#1}}
%\newcommand{\blue}[1]{\textcolor{blue}{#1}}
\newcommand{\epen}[1]{\colorbox{jgray}{#1}}
\newcommand{\hand}{{\normalsize \ding{43}}}
\definecolor{jgray}{gray}{0.8} 
\usetikzlibrary{positioning}
\usetikzlibrary{matrix}
\newcommand{\mora}{\textmu\xspace}
\newcommand{\si}{\textsigma\xspace}
\newcommand{\ft}{\textPhi\xspace}
\newcommand{\tone}{\texttau\xspace}
\newcommand{\word}{\textomega\xspace}
% \newcommand{\ts}{\texttslig}
\newcommand{\fns}{\footnotesize}
\newcommand{\ns}{\normalsize}
\newcommand{\vs}{\vspace{1em}}
\newcommand{\bs}{\textbackslash}   % backslash
\newcommand{\cmd}[1]{{\bf \color{red}#1}}   % highlights command
\newcommand{\scell}[2][l]{\begin{tabular}[#1]{@{}c@{}}#2\end{tabular}}
% \interfootnotelinepenalty=10000

% --- Snider Representations --- %

\newcommand{\RepLevelHh}{
\begin{minipage}{0.10\textwidth}
\begin{tikzpicture}[xscale=\myscalex,yscale=\myscaley]
%\node (syl) at (0,0) {Hi};
\node (Rt) at (0,1) {o};
\node (H) at (-0.5,2) {H};
\node (R) at (0.5,3) {h};
%\draw [thick] (syl.north) -- (Rt.south) ;
\draw [thick] (Rt.north) -- (H.south) ;
\draw [thick] (Rt.north) -- (R.south) ;
\end{tikzpicture}
\end{minipage}
}

\newcommand{\RepLevelLh}{
\begin{minipage}{0.10\textwidth}
\begin{tikzpicture}[xscale=\myscalex,yscale=\myscaley]
%\node (syl) at (0,0) {Mid2};
\node (Rt) at (0,1) {o};
\node (H) at (-0.5,2) {L};
\node (R) at (0.5,3) {h};
%\draw [thick] (syl.north) -- (Rt.south) ;
\draw [thick] (Rt.north) -- (H.south) ;
\draw [thick] (Rt.north) -- (R.south) ;
\end{tikzpicture}
\end{minipage}
}

\newcommand{\RepLevelHl}{
\begin{minipage}{0.10\textwidth}
\begin{tikzpicture}[xscale=\myscalex,yscale=\myscaley]
%\node (syl) at (0,0) {Mid1};
\node (Rt) at (0,1) {o};
\node (H) at (-0.5,2) {H};
\node (R) at (0.5,3) {l};
%\draw [thick] (syl.north) -- (Rt.south) ;
\draw [thick] (Rt.north) -- (H.south) ;
\draw [thick] (Rt.north) -- (R.south) ;
\end{tikzpicture}
\end{minipage}
}

\newcommand{\RepLevelLl}{
\begin{minipage}{0.10\textwidth}
\begin{tikzpicture}[xscale=\myscalex,yscale=\myscaley]
%\node (syl) at (0,0) {Lo};
\node (Rt) at (0,1) {o};
\node (H) at (-0.5,2) {L};
\node (R) at (0.5,3) {l};
%\draw [thick] (syl.north) -- (Rt.south) ;
\draw [thick] (Rt.north) -- (H.south) ;
\draw [thick] (Rt.north) -- (R.south) ;
\end{tikzpicture}
\end{minipage}
}

% --- Representations --- %

\newcommand{\RepLevel}{
\begin{minipage}{0.10\textwidth}
\begin{tikzpicture}[xscale=\myscalex,yscale=\myscaley]
\node (syl) at (0,0) {\textsigma};
\node (Rt) at (0,1) {o};
\node (H) at (-0.5,2) {\texttau};
\node (R) at (0.5,3) {\textrho};
\draw [thick] (syl.north) -- (Rt.south) ;
\draw [thick] (Rt.north) -- (H.south) ;
\draw [thick] (Rt.north) -- (R.south) ;
\end{tikzpicture}
\end{minipage}
}

\newcommand{\RepContour}{
\begin{minipage}{0.10\textwidth}
\begin{tikzpicture}[xscale=\myscalex,yscale=\myscaley]
\node (syl) at (0,0) {\textsigma};
\node (Rt) at (0,1) {o};
\node (H) at (-0.5,2) {\texttau};
\node (R) at (0.5,3) {\textrho};
\node (Rt2) at (1.5,1.0) {o};
%\node (H2) at (1.0,2) {$\tau$};
%\node (R2) at (2.0,2.5) {R};
\draw [thick] (syl.north) -- (Rt.south) ;
\draw [thick] (Rt.north) -- (H.south) ;
\draw [thick] (Rt.north) -- (R.south) ;
\draw [thick] (syl.north) -- (Rt2.south) ;
%\draw [thick] (Rt2.north) -- (H2.south) ;
%\draw [thick] (Rt2.north) -- (R2.south) ;
\end{tikzpicture}
\end{minipage}
}


% --- OT constraints --- %

\newcommand{\IllustrationDown}{
\begin{minipage}{0.09\textwidth}
\begin{tikzpicture}[xscale=0.7,yscale=0.45]
\node (reg) at (0,0.75) {{\small \textalpha}};
\node (arrow) at (0,0) {{\fns $\downarrow$}};
\node (Rt) at (0,-0.75) {{\small \textbeta}};
\end{tikzpicture}
\end{minipage}
}

\newcommand{\IllustrationUp}{
\begin{minipage}{0.09\textwidth}
\begin{tikzpicture}[xscale=0.7,yscale=0.45]
\node (reg) at (0,0.75) {{\small \textalpha}};
\node (arrow) at (0,0) {{\fns $\uparrow$}};
\node (Rt) at (0,-0.75) {{\small \textbeta}};
\end{tikzpicture}
\end{minipage}
}

\newcommand{\MaxAB}{
\begin{minipage}{0.09\textwidth}
\begin{tikzpicture}[xscale=0.6,yscale=0.4]
\node (max) at (0,0) {{\small \textsc{Max}}};
\node (reg) at (0.75,0.5) {{\fns \textalpha}};
\node (arrow) at (0.75,0) {{\tiny $\downarrow$}};
\node (Rt) at (0.75,-0.5) {{\fns \textbeta}};
\end{tikzpicture}
\end{minipage}
}

\newcommand{\DepAB}{
\begin{minipage}{0.09\textwidth}
\begin{tikzpicture}[xscale=0.6,yscale=0.4]
\node (max) at (0,0) {{\small \textsc{Dep}}};
\node (reg) at (0.75,0.5) {{\fns \textalpha}};
\node (arrow) at (0.75,0) {{\tiny $\downarrow$}};
\node (Rt) at (0.75,-0.5) {{\fns \textbeta}};
\end{tikzpicture}
\end{minipage}
}

\newcommand{\DepHReg}{
\begin{minipage}{0.055\textwidth}
\begin{tikzpicture}[xscale=0.6,yscale=0.4]
\node (dep) at (0,0) {{\small \textsc{Dep}}};
\node (reg) at (0,-1.0) {{\small h}};
\end{tikzpicture}
\end{minipage}
}

\newcommand{\DepLReg}{
\begin{minipage}{0.055\textwidth}
\begin{tikzpicture}[xscale=0.6,yscale=0.4]
\node (dep) at (0,0) {{\small \textsc{Dep}}};
\node (reg) at (0,-1.0) {{\small l}};
\end{tikzpicture}
\end{minipage}
}

\newcommand{\DepReg}{
\begin{minipage}{0.055\textwidth}
\begin{tikzpicture}[xscale=0.6,yscale=0.4]
\node (dep) at (0,0) {{\small \textsc{Dep}}};
\node (reg) at (0,-1.0) {{\small \textrho}};
\end{tikzpicture}
\end{minipage}
}

\newcommand{\DepTRt}{
\begin{minipage}{0.1\textwidth}
\begin{tikzpicture}[xscale=0.6,yscale=0.4]
\node (dep) at (0,0) {{\small \textsc{Dep}}};
\node (t) at (0.75,0.5) {{\fns \texttau}};
\node (arrow) at (0.75,0) {{\tiny $\downarrow$}};
\node (Rt) at (0.75,-0.5) {{\fns o}};
\end{tikzpicture}
\end{minipage}
}

\newcommand{\MaxRegRt}{
\begin{minipage}{0.1\textwidth}
\begin{tikzpicture}[xscale=0.6,yscale=0.4]
\node (max) at (0,0) {{\small \textsc{Max}}};
\node (arrow) at (0.75,0) {{\tiny $\downarrow$}};
\node (Rt) at (0.75,-0.5) {{\fns o}};
\node (reg) at (0.75,0.5) {{\fns \textrho}};
\end{tikzpicture}
\end{minipage}
}

\newcommand{\RegToneByRt}{
\begin{minipage}{0.06\textwidth}
\begin{tikzpicture}[xscale=0.6,yscale=0.5]
\node[rotate=20] (arrow1) at (-0.15,0) {{\fns $\uparrow$}};
\node[rotate=340] (arrow2) at (0.15,0) {{\fns $\uparrow$}};
\node (Rt) at (0,-0.55) {{\small o}};
\node (reg) at (0.4,0.55) {{\small \textrho}};
\node (tone) at (-0.4,0.55) {{\small \texttau}};
\end{tikzpicture}
\end{minipage}
}

\newcommand{\RegToneBySyl}{
\begin{minipage}{0.06\textwidth}
\begin{tikzpicture}[xscale=0.6,yscale=0.5]
\node[rotate=20] (arrow1) at (-0.15,0) {{\fns $\uparrow$}};
\node[rotate=340] (arrow2) at (0.15,0) {{\fns $\uparrow$}};
\node (Rt) at (0,-0.55) {{\small \textsigma}};
\node (reg) at (0.4,0.55) {{\small \textrho}};
\node (tone) at (-0.4,0.55) {{\small \texttau}};
\end{tikzpicture}
\end{minipage}
}

\newcommand{\DepTone}{
\begin{minipage}{0.055\textwidth}
\begin{tikzpicture}[xscale=0.6,yscale=0.4]
\node (dep) at (0,0) {{\small \textsc{Dep}}};
\node (tone) at (0,-1.0) {{\small \texttau}};
\end{tikzpicture}
\end{minipage}
}

\newcommand{\DepTonalRt}{
\begin{minipage}{0.055\textwidth}
\begin{tikzpicture}[xscale=0.6,yscale=0.4]
\node (dep) at (0,0) {{\small \textsc{Dep}}};
\node (tone) at (0,-1.0) {{\small o}};
\end{tikzpicture}
\end{minipage}
}

\newcommand{\DepL}{
\begin{minipage}{0.055\textwidth}
\begin{tikzpicture}[xscale=0.6,yscale=0.4]
\node (dep) at (0,0) {{\small \textsc{Dep}}};
\node (tone) at (0,-1.0) {{\small L}};
\end{tikzpicture}
\end{minipage}
}

\newcommand{\DepH}{
\begin{minipage}{0.055\textwidth}
\begin{tikzpicture}[xscale=0.6,yscale=0.4]
\node (dep) at (0,0) {{\small \textsc{Dep}}};
\node (tone) at (0,-1.0) {{\small H}};
\end{tikzpicture}
\end{minipage}
}

\newcommand{\NoMultDiff}{{\small *loh}}
\newcommand{\Alt}{{\small \textsc{Alt}}}
\newcommand{\NoSkip}{{\small \scell{\textsc{No}\\\textsc{Skip}}}}


\newcommand{\RegDomRt}{
\begin{minipage}{0.030\textwidth}
\begin{tikzpicture}[xscale=0.6,yscale=0.5]
\node (arrow) at (0,0) {{\fns $\downarrow$}};
\node (Rt) at (0,-0.55) {{\small o}};
\node (reg) at (0,0.55) {{\small \textrho}};
\end{tikzpicture}
\end{minipage}
}

\newcommand{\DepRegRt}{
\begin{minipage}{0.1\textwidth}
\begin{tikzpicture}[xscale=0.6,yscale=0.4]
\node (dep) at (0,0) {{\small \textsc{Dep}}};
\node (arrow) at (0.75,0) {{\tiny $\downarrow$}};
\node (Rt) at (0.75,-0.5) {{\fns o}};
\node (reg) at (0.75,0.5) {{\fns \textrho}};
\end{tikzpicture}
\end{minipage}
}

% unused

\newcommand{\ToneByRt}{
\begin{minipage}{0.05\textwidth}
\begin{tikzpicture}[xscale=0.6,yscale=0.5]
\node (arrow) at (0,0) {{\fns $\uparrow$}};
\node (Rt) at (0,-0.55) {{\small o}};
\node (tone) at (0,0.55) {{\small \texttau}};
\end{tikzpicture}
\end{minipage}
}

\newcommand{\RegByRt}{
\begin{minipage}{0.05\textwidth}
\begin{tikzpicture}[xscale=0.6,yscale=0.5]
\node (arrow) at (0,0) {{\fns $\uparrow$}};
\node (Rt) at (0,-0.55) {{\small o}};
\node (reg) at (0,0.55) {{\small \textrho}};
\end{tikzpicture}
\end{minipage}
}

\newcommand{\ToneDomRt}{
\begin{minipage}{0.05\textwidth}
\begin{tikzpicture}[xscale=0.6,yscale=0.5]
\node (arrow) at (0,0) {{\fns $\downarrow$}};
\node (Rt) at (0,-0.55) {{\small o}};
\node (tone) at (0,0.55) {{\small \texttau}};
\end{tikzpicture}
\end{minipage}
}

% --- OT tableaus --- %

% Sec. 3.2, first tabl.

\newcommand{\OTHLInput}{
\begin{minipage}{0.17\textwidth}
\begin{tikzpicture}[xscale=\myscalex,yscale=\myscaley]
\node (tone) at (2,0) {(= H)};
\node (syl) at (0,0) {\textsigma};
\node (Rt) at (0,1) {o};
\node (H) at (-0.5,2) {H};
\node (R) at (0.5,3) {h};
\node (Rt2) at (1.5,1.0) {o};
%\node (H2) at (1.0,2) {\epen{L}};
\node (R2) at (2.0,3) {\blue{l}};
\draw [thick] (syl.north) -- (Rt.south) ;
\draw [thick] (Rt.north) -- (H.south) ;
\draw [thick] (Rt.north) -- (R.south) ;
\draw [thick] (syl.north) -- (Rt2.south) ;
%\draw [dashed] (Rt2.north) -- (H2.south) ;
%\draw [dashed] (Rt2.north) -- (R2.south) ;
\end{tikzpicture}
\end{minipage}
}

\newcommand{\OTHLWinner}{
\begin{minipage}{0.17\textwidth}
\begin{tikzpicture}[xscale=\myscalex,yscale=\myscaley]
\node (tone) at (2,0) {(= HL)};
\node (syl) at (0,0) {\textsigma};
\node (Rt) at (0,1) {o};
\node (H) at (-0.5,2) {H};
\node (R) at (0.5,3) {h};
\node (Rt2) at (1.5,1.0) {o};
\node (H2) at (1.0,2) {\epen{L}};
\node (R2) at (2.0,3) {\blue{l}};
\draw [thick] (syl.north) -- (Rt.south) ;
\draw [thick] (Rt.north) -- (H.south) ;
\draw [thick] (Rt.north) -- (R.south) ;
\draw [thick] (syl.north) -- (Rt2.south) ;
\draw [dashed] (Rt2.north) -- (H2.south) ;
\draw [dashed] (Rt2.north) -- (R2.south) ;
\end{tikzpicture}
\end{minipage}
}

\newcommand{\OTHLSpreadingHOnly}{
\begin{minipage}{0.17\textwidth}
\begin{tikzpicture}[xscale=\myscalex,yscale=\myscaley]
\node (tone) at (2,0) {(= HM)};
\node (syl) at (0,0) {\textsigma};
\node (Rt) at (0,1) {o};
\node (H) at (-0.5,2) {H};
\node (R) at (0.5,3) {h};
\node (Rt2) at (1.5,1.0) {o};
%\node (H2) at (1.0,2) {\epen{L}};
\node (R2) at (2.0,3) {\blue{l}};
\draw [thick] (syl.north) -- (Rt.south) ;
\draw [thick] (Rt.north) -- (H.south) ;
\draw [thick] (Rt.north) -- (R.south) ;
\draw [thick] (syl.north) -- (Rt2.south) ;
\draw [dashed] (Rt2.north) -- (R2.south) ;
\draw [dashed] (Rt2.north) -- (H.south) ;
\end{tikzpicture}
\end{minipage}
}

\newcommand{\OTHLInsertH}{
\begin{minipage}{0.17\textwidth}
\begin{tikzpicture}[xscale=\myscalex,yscale=\myscaley]
\node (tone) at (2,0) {(= HM)};
\node (syl) at (0,0) {\textsigma};
\node (Rt) at (0,1) {o};
\node (H) at (-0.5,2) {H};
\node (R) at (0.5,3) {h};
\node (Rt2) at (1.5,1.0) {o};
\node (H2) at (1.0,2) {\epen{H}};
\node (R2) at (2.0,3) {\blue{l}};
\draw [thick] (syl.north) -- (Rt.south) ;
\draw [thick] (Rt.north) -- (H.south) ;
\draw [thick] (Rt.north) -- (R.south) ;
\draw [thick] (syl.north) -- (Rt2.south) ;
\draw [dashed] (Rt2.north) -- (H2.south) ;
\draw [dashed] (Rt2.north) -- (R2.south) ;
\end{tikzpicture}
\end{minipage}
}

\newcommand{\OTHLOverwriting}{
\begin{minipage}{0.17\textwidth}
\begin{tikzpicture}[xscale=\myscalex,yscale=\myscaley]
\node (syl) at (0,0) {\textsigma};
\node (Rt) at (0,1) {o};
\node (H) at (-0.5,2) {H};
\node (R) at (0.5,3) {h};
\node (Rt2) at (1.5,1.0) {o};
%\node (H2) at (1.0,2) {\epen{L}};
\node (R2) at (2.0,3) {\blue{l}};
\draw [thick] (syl.north) -- (Rt.south) ;
\draw [thick] (Rt.north) -- (H.south) ;
\draw [thick] (Rt.north) -- (R.south) ;
\draw [thick] (syl.north) -- (Rt2.south) ;
%\draw [dashed] (Rt2.north) -- (H2.south) ;
\draw [dashed] (Rt.north) -- (R2.south) ;
\node (del) at (0.3,1.9) {\textbf{=}};
\end{tikzpicture}
\end{minipage}
}

\newcommand{\OTHLSpreading}{
\begin{minipage}{0.17\textwidth}
\begin{tikzpicture}[xscale=\myscalex,yscale=\myscaley]
\node (syl) at (0,0) {\textsigma};
\node (Rt) at (0,1) {o};
\node (H) at (-0.5,2) {H};
\node (R) at (0.5,3) {h};
\node (Rt2) at (1.5,1.0) {o};
%\node (H2) at (1.0,2) {\epen{L}};
\node (R2) at (2.0,3) {\blue{l}};
\draw [thick] (syl.north) -- (Rt.south) ;
\draw [thick] (Rt.north) -- (H.south) ;
\draw [thick] (Rt.north) -- (R.south) ;
\draw [thick] (syl.north) -- (Rt2.south) ;
%\draw [dashed] (Rt2.north) -- (H2.south) ;
\draw [dashed] (Rt2.north) -- (H.south) ;
\draw [dashed] (Rt2.north) -- (R.south) ;
\end{tikzpicture}
\end{minipage}
}

% Sec. 4.2, second tabl.: phrase-medial position

\newcommand{\OTHnoLInput}{
\begin{minipage}{0.17\textwidth}
\begin{tikzpicture}[xscale=\myscalex,yscale=\myscaley]
\node (tone) at (2,0) {(= H)};
\node (syl) at (0,0) {\textsigma};
\node (Rt) at (0,1) {o};
\node (H) at (-0.5,2) {H};
\node (R) at (0.5,3) {h};
\node (Rt2) at (1.5,1.0) {o};
%\node (H2) at (1.0,2) {\epen{L}};
%\node (R2) at (2.0,3) {\blue{l}};
\draw [thick] (syl.north) -- (Rt.south) ;
\draw [thick] (Rt.north) -- (H.south) ;
\draw [thick] (Rt.north) -- (R.south) ;
\draw [thick] (syl.north) -- (Rt2.south) ;
\end{tikzpicture}
\end{minipage}
}

\newcommand{\OTHnoLEpenth}{
\begin{minipage}{0.17\textwidth}
\begin{tikzpicture}[xscale=\myscalex,yscale=\myscaley]
\node (tone) at (2,0) {(= HM)};
\node (syl) at (0,0) {\textsigma};
\node (Rt) at (0,1) {o};
\node (H) at (-0.5,2) {H};
\node (R) at (0.5,3) {h};
\node (Rt2) at (1.5,1.0) {o};
\node (H2) at (1.0,2) {\epen{L}};
\node (R2) at (2.0,3) {\epen{h}};
\draw [thick] (syl.north) -- (Rt.south) ;
\draw [thick] (Rt.north) -- (H.south) ;
\draw [thick] (Rt.north) -- (R.south) ;
\draw [thick] (syl.north) -- (Rt2.south) ;
\draw [dashed] (Rt2.north) -- (H2.south) ;
\draw [dashed] (Rt2.north) -- (R2.south) ;
\end{tikzpicture}
\end{minipage}
}

\newcommand{\OTHnoLSpreading}{
\begin{minipage}{0.17\textwidth}
\begin{tikzpicture}[xscale=\myscalex,yscale=\myscaley]
\node (tone) at (2,0) {(= HH)};
\node (syl) at (0,0) {\textsigma};
\node (Rt) at (0,1) {o};
\node (H) at (-0.5,2) {H};
\node (R) at (0.5,3) {h};
\node (Rt2) at (1.5,1.0) {o};
%\node (H2) at (1.0,2) {\epen{L}};
%\node (R2) at (2.0,3) {\blue{l}};
\draw [thick] (syl.north) -- (Rt.south) ;
\draw [thick] (Rt.north) -- (H.south) ;
\draw [thick] (Rt.north) -- (R.south) ;
\draw [thick] (syl.north) -- (Rt2.south) ;
\draw [dashed] (Rt2.north) -- (H.south) ;
\draw [dashed] (Rt2.north) -- (R.south) ;
\end{tikzpicture}
\end{minipage}
}

% Sec. 4.2, third tabl., LM is unaffected by L\%

\newcommand{\OTLMInput}{
\begin{minipage}{0.2\textwidth}
\begin{tikzpicture}[xscale=\myscalex,yscale=\myscaley]
\node (tone) at (2,0) {(= LM)};
\node (syl) at (0,0) {\textsigma};
\node (Rt) at (0,1) {o};
\node (H) at (-0.5,2) {L};
\node (R) at (0.5,3) {l};
\node (Rt2) at (1.5,1.0) {o};
\node (H2) at (1.0,2) {L};
\node (R2) at (2.0,3) {h};
\node (R3) at (3.0,3) {\blue{l}};
\draw [thick] (syl.north) -- (Rt.south) ;
\draw [thick] (Rt.north) -- (H.south) ;
\draw [thick] (Rt.north) -- (R.south) ;
\draw [thick] (syl.north) -- (Rt2.south) ;
\draw [thick] (Rt2.north) -- (H2.south) ;
\draw [thick] (Rt2.north) -- (R2.south) ;
\end{tikzpicture}
\end{minipage}
}

\newcommand{\OTLMReplace}{
\begin{minipage}{0.2\textwidth}
\begin{tikzpicture}[xscale=\myscalex,yscale=\myscaley]
\node (tone) at (2,0) {(= LL)};
\node (syl) at (0,0) {\textsigma};
\node (Rt) at (0,1) {o};
\node (H) at (-0.5,2) {L};
\node (R) at (0.5,3) {l};
\node (Rt2) at (1.5,1.0) {o};
\node (H2) at (1.0,2) {L};
\node (R2) at (2.0,3) {h};
\node (R3) at (3.0,3) {\blue{l}};
\draw [thick] (syl.north) -- (Rt.south) ;
\draw [thick] (Rt.north) -- (H.south) ;
\draw [thick] (Rt.north) -- (R.south) ;
\draw [thick] (syl.north) -- (Rt2.south) ;
\draw [thick] (Rt2.north) -- (H2.south) ;
\draw [thick] (Rt2.north) -- (R2.south) ;
\draw [dashed] (Rt2.north) -- (R3.south) ;
\node (del) at (1.8,2.1) {\textbf{=}};
\end{tikzpicture}
\end{minipage}
}

\newcommand{\OTLMTwoReg}{
\begin{minipage}{0.2\textwidth}
\begin{tikzpicture}[xscale=\myscalex,yscale=\myscaley]
\node (tone) at (2,0) {(= LML)};
\node (syl) at (0,0) {\textsigma};
\node (Rt) at (0,1) {o};
\node (H) at (-0.5,2) {L};
\node (R) at (0.5,3) {l};
\node (Rt2) at (1.5,1.0) {o};
\node (H2) at (1.0,2) {L};
\node (R2) at (2.0,3) {h};
\node (R3) at (3.0,3) {\blue{l}};
\draw [thick] (syl.north) -- (Rt.south) ;
\draw [thick] (Rt.north) -- (H.south) ;
\draw [thick] (Rt.north) -- (R.south) ;
\draw [thick] (syl.north) -- (Rt2.south) ;
\draw [thick] (Rt2.north) -- (H2.south) ;
\draw [thick] (Rt2.north) -- (R2.south) ;
\draw [dashed] (Rt2.north) -- (R3.south) ;
\end{tikzpicture}
\end{minipage}
}

% Sec. 4.2, fourth tabl., L is affected by L\% but M is not

\newcommand{\OTLInput}{
\begin{minipage}{0.17\textwidth}
\begin{tikzpicture}[xscale=\myscalex,yscale=\myscaley]
\node (tone) at (2,0) {(= L)};
\node (syl) at (0,0) {\textsigma};
\node (Rt) at (0,1) {o};
\node (H) at (-0.5,2) {L};
\node (R) at (0.5,3) {l};
\node (R2) at (2,3) {\blue{l}};
\draw [thick] (syl.north) -- (Rt.south) ;
\draw [thick] (Rt.north) -- (H.south) ;
\draw [thick] (Rt.north) -- (R.south) ;
\end{tikzpicture}
\end{minipage}
}

\newcommand{\OTLLowered}{
\begin{minipage}{0.17\textwidth}
\begin{tikzpicture}[xscale=\myscalex,yscale=\myscaley]
\node (tone) at (2,0) {(= LL)};
\node (syl) at (0,0) {\textsigma};
\node (Rt) at (0,1) {o};
\node (H) at (-0.5,2) {L};
\node (R) at (0.5,3) {l};
\node (R2) at (2,3) {\blue{l}};
\draw [thick] (syl.north) -- (Rt.south) ;
\draw [thick] (Rt.north) -- (H.south) ;
\draw [thick] (Rt.north) -- (R.south) ;
\draw [dashed] (Rt.north) -- (R2.south) ;
\end{tikzpicture}
\end{minipage}
}

\newcommand{\OTMInput}{
\begin{minipage}{0.17\textwidth}
\begin{tikzpicture}[xscale=\myscalex,yscale=\myscaley]
\node (tone) at (2,0) {(= M)};
\node (syl) at (0,0) {\textsigma};
\node (Rt) at (0,1) {o};
\node (H) at (-0.5,2) {L};
\node (R) at (0.5,3) {h};
\node (R2) at (2,3) {\blue{l}};
\draw [thick] (syl.north) -- (Rt.south) ;
\draw [thick] (Rt.north) -- (H.south) ;
\draw [thick] (Rt.north) -- (R.south) ;
\end{tikzpicture}
\end{minipage}
}

\newcommand{\OTMLowered}{
\begin{minipage}{0.17\textwidth}
\begin{tikzpicture}[xscale=\myscalex,yscale=\myscaley]
\node (tone) at (2,0) {(= ML)};
\node (syl) at (0,0) {\textsigma};
\node (Rt) at (0,1) {o};
\node (H) at (-0.5,2) {L};
\node (R) at (0.5,3) {h};
\node (R2) at (2,3) {\blue{l}};
\draw [thick] (syl.north) -- (Rt.south) ;
\draw [thick] (Rt.north) -- (H.south) ;
\draw [thick] (Rt.north) -- (R.south) ;
\draw [dashed] (Rt.north) -- (R2.south) ;
\end{tikzpicture}
\end{minipage}
}

% Sec. 4.2, fifth tableau, polar questions with level tones

\newcommand{\OTLPolIn}{
\begin{minipage}{0.20\textwidth}
\begin{tikzpicture}[xscale=\myscalex-0.05,yscale=\myscaley-0.05]
\node (tone) at (3.5,0) {(= L)};
\node (syl) at (0,0) {\textsigma};
\node (syl2) at (2,0) {\red{\textsigma}};
\node (Rt) at (0,1) {o};
\node (H) at (-0.5,2) {L};
\node (R) at (0.5,3) {l};
\node (Rt2) at (2,1) {\red{o}};
\draw [thick] (syl.north) -- (Rt.south) ;
\draw [thick,red] (syl2.north) -- (Rt2.south) ;
\draw [thick] (Rt.north) -- (H.south) ;
\draw [thick] (Rt.north) -- (R.south) ;
\end{tikzpicture}
\end{minipage}
}

\newcommand{\OTLPolDef}{
\begin{minipage}{0.20\textwidth}
\begin{tikzpicture}[xscale=\myscalex-0.05,yscale=\myscaley-0.05]
\node (tone) at (3.5,0) {(= L.M)};
\node (syl) at (0,0) {\textsigma};
\node (syl2) at (2,0) {\red{\textsigma}};
\node (Rt) at (0,1) {o};
\node (H) at (-0.5,2) {L};
\node (R) at (0.5,3) {l};
\node (H2) at (1.5,2) {\epen{L}};
\node (R2) at (2.5,3) {\epen{h}};
\node (Rt2) at (2,1) {\red{o}};
\draw [thick] (syl.north) -- (Rt.south) ;
\draw [thick,red] (syl2.north) -- (Rt2.south) ;
\draw [thick] (Rt.north) -- (H.south) ;
\draw [thick] (Rt.north) -- (R.south) ;
\draw [semithick,dashed] (Rt2.north) -- (H2.south) ;
\draw [semithick,dashed] (Rt2.north) -- (R2.south) ;
\end{tikzpicture}
\end{minipage}
}

\newcommand{\OTLPolAlt}{
\begin{minipage}{0.20\textwidth}
\begin{tikzpicture}[xscale=\myscalex-0.05,yscale=\myscaley-0.05]
\node (tone) at (3.5,0) {(= L.L)};
\node (syl) at (0,0) {\textsigma};
\node (syl2) at (2,0) {\red{\textsigma}};
\node (Rt) at (0,1) {o};
\node (H) at (-0.5,2) {L};
\node (R) at (0.5,3) {l};
\node (Rt2) at (2,1) {\red{o}};
\draw [thick] (syl.north) -- (Rt.south) ;
\draw [thick,red] (syl2.north) -- (Rt2.south) ;
\draw [thick] (Rt.north) -- (H.south) ;
\draw [thick] (Rt.north) -- (R.south) ;
\draw [semithick,dashed] (Rt2.north) -- (H.south) ;
\draw [semithick,dashed] (Rt2.north) -- (R.south) ;
\end{tikzpicture}
\end{minipage}
}

% Sec. 4.2, sixth tableau, polar questions with contour tones

\newcommand{\OTLLPolIn}{
\begin{minipage}{0.23\textwidth}
\begin{tikzpicture}[xscale=\myscalex-0.05,yscale=\myscaley-0.05]
\node (tone) at (5.2,0) {(= L)};
\node (syl) at (0,0) {\textsigma};
\node (syl3) at (3.4,0) {\red{\textsigma}};
\node (Rt) at (0,1) {o};
\node (Rt2) at (1.7,1) {o};
\node (Rt3) at (3.4,1) {\red{o}};
\node (H) at (-0.5,2) {L};
\node (R) at (0.5,3) {l};
\draw [thick] (syl.north) -- (Rt.south) ;
\draw [thick] (syl.north) -- (Rt2.south) ;
\draw [thick,red] (syl3.north) -- (Rt3.south) ;
\draw [thick] (Rt.north) -- (H.south) ;
\draw [thick] (Rt.north) -- (R.south) ;
\end{tikzpicture}
\end{minipage}
}

\newcommand{\OTLLPolDef}{
\begin{minipage}{0.23\textwidth}
\begin{tikzpicture}[xscale=\myscalex-0.05,yscale=\myscaley-0.05]
\node (tone) at (5.2,0) {(= L.M)};
\node (syl) at (0,0) {\textsigma};
\node (syl3) at (3.4,0) {\red{\textsigma}};
\node (Rt) at (0,1) {o};
\node (Rt2) at (1.7,1) {o};
\node (Rt3) at (3.4,1) {\red{o}};
\node (H) at (-0.5,2) {L};
\node (R) at (0.5,3) {l};
\node (H3) at (2.9,2) {\epen{L}};
\node (R3) at (3.9,3) {\epen{h}};
\draw [thick] (syl.north) -- (Rt.south) ;
\draw [thick] (syl.north) -- (Rt2.south) ;
\draw [thick,red] (syl3.north) -- (Rt3.south) ;
\draw [thick] (Rt.north) -- (H.south) ;
\draw [thick] (Rt.north) -- (R.south) ;
\draw [dashed] (Rt3.north) -- (H3.south) ;
\draw [dashed] (Rt3.north) -- (R3.south) ;
\end{tikzpicture}
\end{minipage}
}

\newcommand{\OTLLPolSkip}{
\begin{minipage}{0.23\textwidth}
\begin{tikzpicture}[xscale=\myscalex-0.05,yscale=\myscaley-0.05]
\node (tone) at (5.2,0) {(= L.L)};
\node (syl) at (0,0) {\textsigma};
\node (syl3) at (3.4,0) {\red{\textsigma}};
\node (Rt) at (0,1) {o};
\node (Rt2) at (1.7,1) {o};
\node (Rt3) at (3.4,1) {\red{o}};
\node (H) at (-0.5,2) {L};
\node (R) at (0.5,3) {l};
\draw [thick] (syl.north) -- (Rt.south) ;
\draw [thick] (syl.north) -- (Rt2.south) ;
\draw [thick,red] (syl3.north) -- (Rt3.south) ;
\draw [thick] (Rt.north) -- (H.south) ;
\draw [thick] (Rt.north) -- (R.south) ;
\draw [dashed] (Rt3.north) -- (H.south) ;
\draw [dashed] (Rt3.north) -- (R.south) ;
\end{tikzpicture}
\end{minipage}
}  
  
\newcommand{\ilit}[1]{#1\il{#1}}    
\newcommand{\isit}[1]{#1\is{#1}}  

\makeatletter
\let\thetitle\@title
\let\theauthor\@author 
\makeatother

\newcommand{\togglepaper}[1][0]{ 
  \bibliography{../localbibliography}
  %% hyphenation points for line breaks
%% Normally, automatic hyphenation in LaTeX is very good
%% If a word is mis-hyphenated, add it to this file
%%
%% add information to TeX file before \begin{document} with:
%% %% hyphenation points for line breaks
%% Normally, automatic hyphenation in LaTeX is very good
%% If a word is mis-hyphenated, add it to this file
%%
%% add information to TeX file before \begin{document} with:
%% \include{localhyphenation}
\hyphenation{
affri-ca-te
affri-ca-tes
com-ple-ments
par-a-digm
Sha-ron
Kings-ton
phe-nom-e-non
Daul-ton
Abu-ba-ka-ri
Ngo-nya-ni
Clem-ents 
King-ston
Tru-cken-brodt
Tab-leau
cophono-logies
mark-edness
Ti-gri-nya
a-mong
Car-stens
Lu-bu-ku-su
}
\hyphenation{
affri-ca-te
affri-ca-tes
com-ple-ments
par-a-digm
Sha-ron
Kings-ton
phe-nom-e-non
Daul-ton
Abu-ba-ka-ri
Ngo-nya-ni
Clem-ents 
King-ston
Tru-cken-brodt
Tab-leau
cophono-logies
mark-edness
Ti-gri-nya
a-mong
Car-stens
Lu-bu-ku-su
}
  \papernote{\scriptsize\normalfont
    \theauthor.
    \thetitle. 
    To appear in: 
    Emily Clem,   Peter Jenks \& Hannah Sande.
    Theory and description in African Linguistics: Selected papers from the 47th Annual Conference on African Linguistics.
    Berlin: Language Science Press. [preliminary page numbering]
  }
  \pagenumbering{roman}
  \setcounter{chapter}{#1}
  \addtocounter{chapter}{-1}
}

\newcommand{\upstep}{\textupstep}


% \newcounter{tableauxcounter}

\renewcommand{\textltailn}{ɲ}
\renewcommand{\textbardotlessj}{ɟ}

\newcommand{\emphkh}[1]{\textit{#1}} %originally \textbf, banned by the guidelines



\definecolor{lsDOIGray}{cmyk}{0,0,0,0.45}


\newcommand{\xuparrow}[1]{%
  {\left\uparrow\vbox to #1{}\right.\kern-\nulldelimiterspace}
}
\renewcommand \textupstep[1]{\char"A71B#1}
\renewcommand \textdownstep[1]{\char"A71C#1}
 
 \newcommand{\ꜛ}{\textsf{ꜛ}}
 
\def\biberror{\undefined}


\newcommand{\OTbox}[1]{\resizebox{.88\textwidth}{!}{#1}}
 
  \togglepaper[34]
}{}


\abstract{Surface SAuxOV orders abound in West Africa. We demonstrate that apparent examples of this word order have important structural differences across languages. We show that SAuxOV orders in some languages are due to mixed clausal headedness, consisting of a head initial TP and head-final VP, though this order can be concealed by verb movement. Other languages are more consistently head-initial, and what appear to be SAuxOV orders arise in limited syntactic contexts due to specific syntactic constructions such as object shift or nominalized complements. Finally, we show that languages which have genuine SAuxOV, corresponding to a head-final VP, tend to exhibit head-final properties more generally. This observation supports the idea that syntactic typology is most productively framed in terms of structural analyses of languages rather than the existence of surface word orders.}

\begin{document}

\maketitle

\section{Introduction} 
 
%It has been known since at least \citet{heine76} that SAuxOV is a typologically significant property of West African languages. More recently, \citet{guld08,guld11} has suggested that S(Aux)OVX, with emphasis on X, is a property of a \isi{linguistic area} he labels the Macro-Sudan Belt, similar to the \ili{Sudanic} zone of \citet{clements08}, which stretches west to Senegal and Guinea and east to the Central African Republic.  One issue with this claim is that the distribution of S(Aux)OVX observed by \citet{guld08} is not quite coextensive with the Macro-Sudan Belt.  Compare, for example, G\"uldemann's map for the distribution of labiovelars in \figref{fig:sande:1:labiovelar}, which spans the entirety of the proposed Macro-Sudan Belt, to the distribution of S(Aux)OVX in \figref{fig:sande:2:sauxov}, which seems restricted to the Western parts of the same region.
%
%  \begin{figure} \label{fig:sande:1:labiovelar}
%    \centering
%    \includegraphics[width =0.5\textwidth]{figures/labiovelar.png}
%    \caption{Labio-velar consonants in African languages \citep[10]{guld08}}
%  \end{figure}
%
%
%  \begin{figure} 
%    \centering
%    \includegraphics[width =0.5\textwidth]{figures/SAuxOV.png}
%    \caption{Word order S(Aux)OVX in African languages \citep[15]{guld08}} \label{fig:sande:2:sauxov}
%  \end{figure}
%
%
%Another issue is the murkiness of the S(Aux)OVX label itself. There are several candidates for combinations of surface properties that we could typologize as `S(Aux)OVX'. For example, we could include languages with surface SAuxOV in any context, languages which alternate between SVO and SOV, or languages with OVX order in any context, among other possibilities. 

%We argue that only one structural incarnation of SAuxOV represents an areally significant phenomenon in West Africa, namely, structures in which TP is head initial but VP is head final. We show this area is restricted to a particular geographical area we call the Mandesphere, and that this structure is correlated with head finality, an observation that goes back to at least \citet{heine76}. We confirm this correlation with a new typological survey of the region, and propose diagnostics for the relevant structure. For other languages occurring to the east of the Mandesphere, we show that SAuxOV is a property of specific constructions which are structurally distinct from SAuxOV proper.

The order subject-auxiliary-object-verb (SAuxOV) is quite common across West Africa. At the same time, it is well-known that syntactic differences exist among the languages with this surface order \citep{Creissels2005typology}. Our goal in this paper is to identify structural differences across languages for which SAuxOV order occurs, and to show that these structural differences correlate with other word order properties of the language.

Our central observation is that there is a single \isi{clause structure} which results in SAuxOV word order as a language-wide property. The relevant structure is \textsc{mixed clausal headedness}; here, the property of having a head-initial TP and a head-final VP, resulting in SAuxOV word order whenever an overt \isi{auxiliary} is present. Such a structure is typical of the \ili{Kru} and \ili{Mande} language families. One example each from Guébie (\ili{Kru}) and \ili{Dafing} (\ili{Mande}) is provided below.
\ea 
\label{ex:1:sauxov}
\ea \langinfo{Marka Dafing}{\ili{Mande}: Burkina Faso}{Notes} \\ 
\gll wúrú-ꜝú {ꜝ{ní}} {ʃwó-ꜝó} {{ɲì mì}}  \\
{dog}-\textsc{def} \textsc{pst} meat-\textsc{def} eat\\
\glt `The dog ate the meat.'  \label{ex:1b:sauxov}
\ex \langinfo{Guébie}{\ili{Kru}: Côte d'Ivoire}{Notes} \\ 
\gll {e}$^{4}$ {ji}$^{3}$ {ɟa}$^{31}$ {li}$^{3}$\\
\textsc{1sg}.\textsc{nom} \textsc{fut} coconuts eat \\
\glt `I will eat coconuts.'  \label{ex:1a:sauxov}
\z
\z
This structure occurs in an area of West Africa we call the Mandesphere, the historical sphere of influence for the \ili{Mande} empires which were politically dominant in West Africa for much of its recent history, as discussed further in \sectref{sec:distribution}.

There is one major difference in the clausal syntax of \ili{Kru} and \ili{Mande} languages, however. In \ili{Kru}, but not in \ili{Mande}, \isi{verb movement} occurs in sentences without an overt \isi{auxiliary}, resulting in SVO order. While this is an important syntactic difference between the languages, it seems to be inconsequential for the purposes of word order typology: both \ili{Mande} and \ili{Kru} languages are overwhelmingly head-final below the clause level, another property which is characteristic of the Mandesphere.

Outside of the Mandesphere, languages are generally head-initial \citep{heine76}. Where apparent SAuxOV orders occur, we demonstrate that these do not involve mixed clausal \isi{headedness} \citep{manfredi97,kandy03, aboh09}. We examine two such cases. First, we present a novel analysis of \ili{Gwari} (Nupoid), in which we demonstrate that some auxiliaries such as the completive trigger \isi{movement} of the object across the verb, while most others do not \REF{ex:2a:gwari}.\footnote{See \citet{kandy03} for a similar analysis of closely-related \ili{Nupe}.} The second case of apparent SAuxOV we examine involve nominalized complements, as in the \ili{Fongbe} (\ili{Kwa}) example in \REF{ex:2b:gwari}:
\ea \label{ex:2:fakesauxov}
\ea \langinfo{Gwari}{\ili{Benue-Congo}: Nigeria}{\citealt[51]{hyman1970}} \label{ex:2a:gwari}\\
\gll wó lá shnamá si \\
3\textsc{sg} \textsc{compl:sg} yam buy \\
\glt `S/he has bought a yam.'
\ex \langinfo{Fongbe}{\ili{Kwa}: Benin}{Lefebvre \& Brousseau 2002:215} \label{ex:2b:gwari} \\	
\gll Ùn {\`ɛ} {nú} {ɖù} jí  \\
1\textsc{sg} fall thing eat.\textsc{nom} on  \\
\glt `I began to eat.'
\z
\z
The structures beneath these word orders are quite different from those we saw for Guébie and \ili{Dafing} in \REF{ex:1:sauxov}. Tellingly, we show that languages with more restricted instances of OV order in \REF{ex:2:fakesauxov} are systematically head-initial at the clause level. 

Summing up, we will show that a head-final VP, which is a definitional property of SAuxOV languages, is a good predictor of head finality in West Africa. On the other hand the construction-specific presence of SAuxOV orders is not. The larger conclusion we draw from this observation is that typological correlations about \isi{headedness} should be based on abstract structural analyses of languages, after factoring out independent syntactic operations such as \isi{verb movement}, rather than on the presence or absence of surface orders in a given language. Moreover, it is the basic analytic toolkit supplied by generative syntax that allows such abstract generalizations to be stated.
 
 The outline of this paper is as follows: \sectref{sec:strictSAuxOV} lays out the structural characteristics of SAuxOV arising from mixed clausal \isi{headedness} in \ili{Dafing} (\ili{Mande}) and Guébie (\ili{Kru}). \sectref{sec:fake} demonstrates that \ili{Gwari} and \ili{Fongbe} are head-initial in their clauses, including within the VP; OV orders are shown to occur as an artifact of particular syntactic constructions and contexts. \sectref{sec:distribution} reports the results of a small typological survey showing that languages with mixed clausal \isi{headedness} are concentrated in the Mandesphere, and compares our structural typology to those relying on surface order, such as the word order properties listed in WALS. \sectref{sec:conclusion} concludes. 

\section{Mixed clausal headedness}\label{sec:strictSAuxOV}

In this section we present evidence for analyzing some instances of SAuxOV word order as a result of clausal mixed \isi{headedness}, where T, the position of inflection, is head-initial, but the verb phrase, VP, is head final. We show how these two structural properties are diagnosed in two languages, Guébie  (\ili{Kru}) and \ili{Dafing} (\ili{Mande}).

While there are many grammatical morphemes which can be called auxiliaries, we will use the term `\isi{auxiliary}' to refer to the element that surfaces in a position where TAM marking obligatorily occurs in declarative clauses, a position distinct from the position of the lexical verb. We analyze this position as T (for Tense) regardless of the semantic distinctions it encodes.\footnote{This position is equivalent to Infl or I$^0$ in the GB framework.} Many West African languages have such a position. To qualify as showing SAuxOV due to mixed \isi{headedness}, the T position must be adjacent to the \isi{subject}, and, in languages which index subjects, the T position must be the locus of \isi{subject agreement}. If a language allows multiple auxiliaries to occur, the T position will be the position of the highest (usually leftmost) \isi{auxiliary}.

Once the T position is identified in a language, the crucial test for whether it shows mixed \isi{headedness} is whether, in the presence of an overt \isi{auxiliary} in T, objects obligatorily precede the verb. We \isi{focus} on clauses where the relevant object is the single object of a transitive verb.

\subsection{SAuxOV in Kru}

In this section we show that Guébie, a \ili{Kru} language spoken in southwest Côte d'Ivoire, has mixed clausal \isi{headedness}. Word order properties in Guébie are similar to word order across Eastern \ili{Kru} languages \citep[cf.][]{Marchese1979}, so we are using Guébie data here to diagnose SAuxOV across Eastern \ili{Kru} more generally. It should be noted that in certain Western \ili{Kru} languages like \ili{Grebo} \citep{Innes:1966} some of the tense/\isi{aspect} marking is done through verbal suffixes, rather than auxiliaries. However, across the family, whenever an \isi{auxiliary} is present, the verb surfaces after a \isi{direct object}: SAuxOV \citep{Marchese1979}.

Most clauses in Guébie show SAuxOV order, where nothing can intervene between \isi{subject} and \isi{auxiliary}, and the verb is clause final. This is true of both main clauses, \REF{ex:sande:3a}, and embedded clauses, \REF{ex:sande:3b}.\footnote{The Guébie data presented here come from original work on the language. The data were collected between 2013 and 2017, in Berkeley, California and Gnagbodougnoa, Côte d'Ivoire, with six primary consultants \citep[cf.][]{Sande:2017}. Guébie is a tonal language with four distinct \isi{tone} heights. Tone is marked here with numbers 1--4, where 4 is high.}

\ea 
\ea SAuxOV word order in \langinfo{Guébie}{\ili{Kru}: Côte d'Ivoire}{Notes}\\ 
\gll {e}$^{4}$ {ji}$^{3}$ {ɟa}$^{31}$ {li}$^{3}$  \\
\textsc{1sg}.\textsc{nom} \textsc{fut} coconuts eat \\
\glt `I will eat coconuts.'  \label{ex:sande:3a}
\ex \gll {e}$^{4}$ {wa}$^{2}$ {gba}$^{1}$ {e}$^{4}$ \textbf{{ka}$^{3}$} {tɛlɛ}$^{3.3}$ \textbf{{kɔklalɛ}$^{3.2.2}$} \\
\textsc{1sg}.\textsc{nom} want.\textsc{ipfv} that \textsc{1sg}.\textsc{nom} \textsc{irr} snake touch \\
\glt `I want to touch the snake ' \label{ex:sande:3b}
\z
\z


As is well known, a number of other word order properties correlate with OV across languages \citep{Greenberg:1963, dryer07}. These include postpositions, genitive-noun order, and manner adverbs before main verbs.\footnote{We do not consider properties such as noun-adjective, which Dryer does not find to correlate with OV versus VO order across languages.} Guébie displays all of these typological characteristics, as shown in \REF{ex:4:Guebie}.

\ea \langinfo{Guébie}{\ili{Kru}: Côte d'Ivoire}{Notes} \label{ex:4:Guebie} 
\ea Postpositions\\
\gll {ɔ}$^{3}$ {ji}$^{3}$ \textbf{{su}$^{3}$} \textbf{{mɛ}$^{3}$} {gara}$^{1.1}$ \\
\textsc{3sg}.\textsc{nom} \textsc{fut} tree in perch\\
\glt  `He will perch in a tree.' 

\ex {Gen-N}\\
\gll \textbf{{touri}$^{1.1.3}$} {la}$^{2}$ \textbf{{dəre}$^{3.3}$}\\
Touri \textsc{gen} money\\
\glt `Touri's money'

\ex {AdvV}\\
\gll {e}$^{4}$ {ji}$^{3}$ \textbf{{fafa}$^{4.4}$} {ɟa}$^{31}$ \textbf{{li}$^{3}$} \\
\textsc{1sg}.\textsc{nom} \textsc{fut} quickly  coconuts eat \\
\glt `I will eat coconuts quickly' \label{advv}
\z
\z

With regards to \REF{advv}, some Western \ili{Kru} languages like \ili{Krahn} and Wobé place manner adverbs after verbs within the VP \citep[80-81]{Marchese1979}, much like the \ili{Mande} word order discussed in \sectref{sec:Mande}. It is possible that this variation is due to contact of some Western \ili{Kru} languages with \ili{Mande}. However, because most Eastern and some Western \ili{Kru} languages show the same word order as Guébie with respect to \REF{ex:4:Guebie}, it would seem that Adv-V order was present in Proto-\ili{Kru} (Lynell Zogbo, p.c.).

In addition to word order properties that correlate with OV order across languages, we see other head-final properties in Guébie, such as nominalized verbal objects, which surface before the main verb, \REF{ex:5:Guebie}.

\newpage 
\ea {SAux[OV]$_{\textsc{nom}}$V} in \langinfo{Guébie}{\ili{Kru}: Côte d'Ivoire}{Notes}\\ \label{ex:5:Guebie}
\gll {e}$^{4}$ {ji}$^{3}$ [ {ɟa}$^{31}$ {la}$^{2}$ {li-li-je}$^{3.2.2}$ ] {koci}$^{23.1}$ \\
\textsc{1sg}.\textsc{nom} \textsc{fut} {} coconuts of eat-\textsc{red}-\textsc{nmlz} {} start \\
\glt `I will start eating coconuts.'
\z

We see that word order in Guébie is overwhelmingly head final. However, when there is no \isi{auxiliary} present, the verb fails to surface clause-finally, and instead appears immediately after the \isi{subject}, resulting in SVO order, \REF{ex:6:Guebie}. SVO order only appears in two clause types: simple perfective, \REF{ex:6a:Guebie},  and simple imperfective, \REF{ex:6b:Guebie}.

\ea {Verb \isi{movement}: S-V$_i$-O-$t_i$} in \langinfo{Guébie}{\ili{Kru}: Côte d'Ivoire}{Notes}\\ \label{ex:6:Guebie}
\ea 
\gll {e}$^{4}$ \textbf{{li}$^{3}$}  {ɟa}$^{31}$ \\
\textsc{1sg}.\textsc{nom} eat.\textsc{pfv} coconuts \\
\glt `I ate coconuts.' \label{ex:6a:Guebie}
\ex 
\gll {e}$^{4}$ \textbf{{li}$^{2}$}  {ɟa}$^{31}$ \\
\textsc{1sg}.\textsc{nom} eat.\textsc{ipfv} coconuts \\
\glt `I eat coconuts.' \label{ex:6b:Guebie}
\z
\z

The difference between perfective and imperfective verbs in Guébie is tonal. Verbs are only differentiated for \isi{aspect} when they surface in the immediately-post-\isi{subject} position. That is, verbs only show inflection when there is no \isi{auxiliary}. This is a point of variation in \ili{Kru} languages, where some languages show inflection on verbs even when they are not in the inflectional position \citep{Marchese1979,Koopman:1984}.

Reviewing the word-order properties of Guébie, we see that it follows the proposed diagnostics for a mixed-headed SAuxOV structure. First, it has a syntactic \isi{auxiliary} position immediately following the \isi{subject}, where TAM is marked. Usually TAM is marked by auxiliaries, but when verbs surface in this position (see below), they are marked with inflection. Guébie also shows obligatory OV word order within the verb phrase. The following diagram shows our proposed structure for Guébie SAuxOV clauses.

\begin{figure}
% % % \jtree[xunit=2.5em,yunit=1.25em]
% % % \psset{labelgap=0}
% % % \! = {TP}
% % % : ({DP}!a ) {T'}
% % % : ({T}!b ) [scaleby=1.25 1] {VP}
% % % : ({DP}!c ) ({V}!d ) .
% % % \!a = ({e$^{4}$}{I}).
% % % ɓ = ({ji$^{3}$}{will}).
% % % \!c = <vartri>{{ɟ{}a}$^{31}$}{coconuts}.
% % % \!d = {li$^{3}$}{eat}.
% % % \endjtree
\begin{forest}
[TP
    [DP\\e\textsuperscript{4}\\I] [T'
        [T\\ji\textsuperscript{3}\\will] [VP
            [DP[ɟa\textsuperscript{31}\\coconuts,roof]] [V\\li\textsuperscript{3}\\eat]
        ]
    ]
]
\end{forest}
\caption{Guébie clause structure \citep[cf.][]{Sande:2017}} \label{fig:sande:Guebie}
\end{figure}

We see in \figref{fig:sande:Guebie} that the \isi{auxiliary} is in T, the inflectional position. We also see that objects precede verbs within the verb phrase. %\footnote{We acknowledge that in syntactic theories which assume \citet{kayne94}'s Linear Correspondence Axiom even SAuxOV structures languages would involve object \isi{movement}, although it would be obligatory. We set this issue aside for now, as it does not affect our main point. This question is particularly interesting for \ili{Mande} languages, which we discuss in the next subsection, as non-nominal objects follow the verb. This is surprising as \citet{dryer07} observes that OV languages typically have V-Adv and V-PP orders.} 
%TODO: Okay with Peter that we don't need this footnote given the other Kayne footnote later?
When there is no \isi{auxiliary} present, we propose that the clause-final verb undergoes \isi{movement} to T, the inflectional position. This is shown in \figref{fig:sande:Guebieverbmove}.

\begin{figure} 
% \jtree[xunit=2.5em,yunit=1.25em]
% \psset{labelgap=0}
% \! = {TP}
% : ({DP}!a ) {T'}
% : ({V+T}!b ) [scaleby=1.25 1] {VP}
% : ({DP}!c ) ({\rnode{1}{V}}) .
% \!a = ({e$^{4}$}{I}).
% ɓ = ({li$^{3}$}{eat\rnode{2}{.}\textsc{pfv}}).
% \!c = <vartri>{{ɟ{}a}$^{31}$}{coconuts}.
% \ncbar[angle=-90,nodesep=5pt,arm=4em,arrowscale=1]{->}{1}{2} \hspace{12pt}
% \endjtree
% % \vspace{6pt}
\begin{forest} for tree={fit=band}
[TP
    [DP\\e\textsuperscript{4}\\I] [T'
        [V+T\\li\textsuperscript{3}\\eat.\textsc{pfv},name=eat] [VP 
            [DP[ɟa\textsuperscript{31}\\coconuts,name=coconuts,roof]] [V,name=V]
        ]
    ]
]
\draw[-{Triangle[]}] (V) |- (coconuts.south) -| (eat);
\end{forest}
\caption{Verb movement in Guébie \citep[cf.][]{Koopman:1984, Sande:2017}} \label{fig:sande:Guebieverbmove}
\end{figure}

We will see that it is not only \ili{Kru} languages which show mixed-headed SAuxOV structure, but other languages in West Africa as well.

\subsection{SAuxOV in Mande}\label{sec:Mande}

Our second example with mixed clausal \isi{headedness} is \ili{Dafing}, also known as \ili{Marka}, a Western \ili{Mande} language spoken by 180,000 people in Burkina Faso \citep{prost77,diallo88}.\footnote{The \ili{Dafing} data in this paper was collected via elicitation in Berkeley, CA with a single consultant who is also a native speaker of \ili{Jula} (\ili{Mande}), and a fluent speaker of Mooré (\ili{Gur}).} \ili{Dafing} is closely related to \ili{Bambara} and \ili{Jula} (Dioula), which are both major \ili{Mande} languages in the area with millions of speakers. Word order in \ili{Dafing} is representative of \ili{Mande} languages more generally \citep[e.g.][]{Creissels2005typology, nikitina11}, and we take it as a representative language. The genetic affiliation of \ili{Mande} is uncertain; it has been claimed to be of Niger-Congo stock \citep{Greenberg:1966}, although this classification is not well established \citep{Dimmendaal:2008}.

As in Guébie, \ili{Dafing} shows SAuxOV word order. There is an \isi{auxiliary} position which must occur immediately after the \isi{subject}, and the verb surfaces after the object when auxiliaries are present. This is true for both main and embedded clauses.

\ea SAuxOV word order in \langinfo{Dafing}{\ili{Mande}:Burkina Faso}{Notes} \label{ex:7:dafing}
\ea \gll wúrú-ꜝú \textbf{{}{ná}} {ʃwó-ꜝó} \textbf{{ɲì mì}}  \\
{dog}-\textsc{def} \textsc{fut} meat-\textsc{def} eat\\
\glt `The dog will eat the meat.' \label{ex:7a:dafing}
\ex \gll {\^ɛː} ná {f{\`ɔ}} ká wúrú-ꜝú \textbf{{\ꜝ}{ná}} {ʃwó-ꜝó} \textbf{{ɲì mì}}  \\
\textsc{3sg} \textsc{pfv} say \textsc{comp} {dog}-\textsc{def} \textsc{fut} meat-\textsc{def} eat\\
\glt `She said that the dog will eat the meat.' \label{ex:7b:dafing}
\z
\z
This \isi{auxiliary} position is typically called the ``predicative marker'' in the Mandeist literature \citep[e.g.][]{idiatov2000,creissels2019}; the number and types of distinction that are marked in this position are large, as it is a composite marker of {tense}, {aspect}, modality, negation, as well as transitivity, for example, in Soninke \cite[][]{creissels2017}. 

Like in Guébie, \ili{Dafing} has obligatory OV order in the verb phrase. Thus, we take \ili{Dafing} to be a language with mixed clausal \isi{headedness}, a head initial TP and a head final VP.

\begin{figure}
% \jtree[xunit=2.5em,yunit=1.25em]
% \psset{labelgap=0}
% \! = {TP}
% : ({DP}!a ) [scaleby=1.5 1] {T'}
% : ({T}!b ) [scaleby=1.5 1] {VP}
% : ({DP}!c ) ( {V}!d ).
% \!a = <vartri>{{wúrú-ꜝú}}{the dog}.
% ɓ = ({{ná}}{\textsc{fut}}).
% \!c = <vartri>{{ʃwó-ꜝó}}{the meat}.
% \!d = {{ɲì mì}}{eat}.
% \endjtree

\begin{forest}
[TP
    [DP[wúrú-ꜝú\\the dog,roof]] [T'
        [T\\ná\\\textsc{fut}] [VP
            [DP[ʃwó-ꜝó\\the meat,roof]] [V\\ɲì mì\\eat]]
        ]
]
\end{forest}

\caption{SAuxOV in Dafing\label{fig:sande:dafingtree}}
\end{figure}

\newpage 
A similar structure is proposed by \citet{nikitina09}.\footnote{\label{koopfn}A different analysis is suggested by \citet{Koopman:1984,koopman92}, who maintains that objects move from a postverbal position to a preverbal one in \ili{Mande}. If this analysis is adopted, as it must be if one assumes that syntactic structures across languages are uniformly right-branching \citep{kayne94}, then the criterion of mixed clausal \isi{headedness} we refer to throughout would be reanalyzed as obligatory \isi{movement} of the object to a position before the verb. In general, then, the main takeaway here and below would be the systematic difference between languages where objects obligatorily move to a position before verbs (Guébie and \ili{Dafing}), resulting in general surface OV, and languages where objects only move to a position before verbs in specific syntactic contexts.}

Outside of the mixed-\isi{headedness} in the clause, \ili{Mande} languages like \ili{Dafing} have many of the head-final properties that were also found in \ili{Kru} languages like Guébie. For example, \ili{Dafing} has postpositions \REF{ex:8a:dafing} and genitive-noun word order in the \isi{noun phrase} \REF{ex:8b:dafing}. 

Another head final property that \ili{Dafing} shares with \ili{Kru} languages like Guébie \REF{ex:5:Guebie} is that nominalized complement clauses precede embedding verbs.

\ea \langinfo{SAux[OV]$_{\textsc{nom}}V$}{}{Notes}\\
\gll wúrú-ꜝú {ꜝ{ní}} [ {ʃwó-ꜝó} \textbf{{ɲì mí-í}} ] \textbf{{dàmnà}} \\
{dog}-\textsc{def} \textsc{pfv} {} meat-\textsc{def} eat-\textsc{def} {} begin\\
\glt `The dog began eating the meat.'
\z 

This is a point of variation in \ili{Mande}, as Eastern \ili{Mande} languages such as \ili{Wan} do not allow the full nominalized VP to precede the higher verb \citep{nikitina09}.

There are two significant differences which distinguish \ili{Mande} and \ili{Kru}. First, in \ili{Mande} languages, all VP constituents besides the primary object follow the verb, including adverbs, clausal complements, and oblique arguments \citep{nikitina09}. This is illustrated in \REF{ex:8c:dafing}, which shows verb-adverb order in \ili{Dafing}.

\ea Head finality in \langinfo{Dafing}{}{Notes} \label{ex:8:dafing}
\ea {Postpositions} \\
\gll {wúrú-ꜝú} \textbf{{tábàrí-{\ꜝ}í}} \textbf{{zúk{\`ɔ}}} \\
dog-\textsc{def} table-\textsc{def} under \\
\glt  `The dog is under the table.'  \label{ex:8a:dafing}
\ex {Gen-N}\\
\gll \textbf{{ʃíì}} káꜝá \textbf{{wúrú-{\ꜝ}ú}} \\
Sidiki \textsc{gen} dog-\textsc{def} \\
\glt `Sidiki's dog' \label{ex:8b:dafing} 
\ex {VAdv (*VAdv)}\\
\gll wúrú-ꜝú {ꜝ{ní}} {ʃwó-ꜝó} \textbf{{ɲì mì}} \textbf{{zònà-zònà}} \\
{dog}-\textsc{def} \textsc{pst} meat-\textsc{def} eat quickly\\
\glt `The dog ate the meat quickly.' \label{ex:8c:dafing}
\z
\z
As verb-manner adverb order is generally a property of VO languages \citep{dryer07}, Mande languages can be seen as somewhat less consistently head-final than Kru languages.\footnote{Valentin Vydrin (p.c.) reports that some Mande languages also have Adv-V word order, so this may indeed be a point of variation across Mande, though we could not identify any convincing cases in the literature. One interesting case is Soninke, which allows adverbial content to occur in the preverbal object position with intransitive verbs, but only if that content is in the form of a DP \citep{creissels2017}; in such a case the existence of Adv-V word order is a question of analysis.}

The second difference between \ili{Kru} and \ili{Mande} is that \ili{Mande} languages like \ili{Dafing} never allow \isi{verb movement} in transitive clauses, even in the absence of an overt \isi{auxiliary}. %\footnote{\citet{koopman92} proposes \isi{verb movement} only in the intransitive to account for the presence of a tense suffix found only in intransitive clauses. Yet it is hard to find evidence for intransitive \isi{verb movement} for the obvious reason that the object, the only element which could be used to identify the position of the verb in either T or V, is absent.} 
\ea No \isi{verb movement} in \langinfo{Dafing}{}{Notes}\\
\gll wúrú-ꜝú {} {ʃwó-ꜝó} \textbf{{ɲì mì}} \\
{dog}-\textsc{def} {} meat-\textsc{def} eat\\
\glt `The dog eats the meat.'
\z
In the preceding sentence, which is interpreted habitually, no overt \isi{auxiliary} element occupies the T position. Yet SOV order still occurs. We assume that in such cases there is a null \isi{auxiliary} in T, such that the structure is identical to \figref{fig:sande:dafingtree}, unlike in \ili{Kru}, where verbs move to T when T lacks segmental content.

\subsection{Summary}

We have seen that Guébie (\ili{Kru}) and \ili{Dafing} (\ili{Mande}) both have a mixed clausal headedness, a head initial TP and a head final VP. Independent differences conceal their structural similarity, such as differences in \isi{verb movement} and adverb position. We also saw that \ili{Dafing} and Guébie have head final structures elsewhere: both have Gen-N word order and postpositions. We revisit this connection in \sectref{sec:distribution}, where we will see that when we look at a broader sample of languages in West Africa, mixed clausal \isi{headedness} is indeed a good predictor of head-finality below the clause level. Verb \isi{movement}, on the other hand, has no clear correlations with head finality or SAuxOV, as would be expected if it is an independent syntactic operation.

\section{Apparent mixed clausal headedness} 
\label{sec:fake}

\subsection{Introduction}

In this section we present data from two languages, \ili{Gwari} (Nupoid) and \ili{Fongbe} (\ili{Gbe}) that exhibit apparent mixed clausal \isi{headedness} as a result of SAuxOV order in a restricted set of constructions. In these languages, SAuxOV is not a general organizing principle of \isi{clause structure}, as in \ili{Kru} and \ili{Mande}. Instead, \ili{Gwari} and \ili{Fongbe} have uniformly head-initial \isi{clause structures}. Their SAuxOV orders instead arise in the context of specific syntactic constructions. In \ili{Gwari}, \REF{ex:9a:gwari}, SAuxOV order surfaces with a restricted set of aspectual particles. In \ili{Fongbe}, \REF{ex:9b:fongbe}, putative OV order only occurs in the context of nominalized VP complements. Hence, putative SAuxOV in \ili{Fongbe} is in fact SVGenN. 

\ea  {Apparent cases of SAuxOV in \ili{Gwari} and Fongbe}
\ea  \label{ex:9a:gwari} \ili{Gwari} (\ili{Benue-Congo}: Nigeria)	\\
\gll wó kú \`{a}shnamá si \\
3\textsc{sg} \textsc{compl:pl} yams buy \\
\glt `S/he has bought yams.' \hfill \citep[][56]{hyman1970}
\ex \label{ex:9b:fongbe}
\ili{Fongbe} (\ili{Kwa}: Benin)	\\
\gll Ùn {\`ɛ} {nú} {ɖù} jí  \\
1\textsc{sg} fall thing eat.\textsc{nom} on  \\
\glt `I began to eat.' \hfill \citep[][215]{lefebvre2002}
\z
\z 
Our proposals about \ili{Gwari} and \ili{Fongbe} resemble existing syntactic analyses of closely related languages. Putative OV order in \ili{Gwari} is derived by \isi{object shift} across the aspectual particle followed by further \isi{movement} of this particle above the shifted object \citep{manfredi97,kandy03,aboh09}. In contrast, the putative OV order in \ili{Fongbe} nominalized complements are due to the fact that genitives precede nouns in \ili{Gbe} languages \citep{aboh05}.

%We will see that languages with a more restricted distribution of OV order display fewer head final properties than the languages like Guébie and \ili{Dafing}. This reinforces the point that more restricted OV orders 
%
%
% with a more restricted distribution of OV order display fewer head final properties than Strict SAuxOV languages. This reinforces the conclusion that Fake SAuxOV languages are not characterized by OV within the VP, unlike Strict SAuxOV languages. 

\subsection{Gwari}

In clauses without an \isi{auxiliary}, \ili{Gwari} (Nupoid, Nigeria) displays SVO word order, as shown in \REF{ex:10:gwari}.

\ea \label{ex:10:gwari}
SVO word order in \ili{Gwari}	\\
\ea 
\gll wo si \=obw\=\i \\
3\textsc{sg} buy groundnut \\
\glt `S/he buys groundnuts.' \hfill \citep[][51]{hyman1970}	

\ex \gll wo lá shnamá \\
3\textsc{sg} take:\textsc{sg} yam \\
\glt `S/he takes a yam.' \hfill \citep[][51]{hyman1970}	
\z
\z
Past tense is marked with an overt \isi{auxiliary} that appears after the \isi{subject}. The word order in \isi{past tense} clauses is SAuxVO, as shown in \REF{ex:11:gwari}:

\ea \label{ex:11:gwari}
\ea  Today past continuous	\\
\gll wo {ɓé\=\i} si shnamá\\
3\textsc{sg} \textsc{t.pst} buy yam \\
\glt `S/he was buying yams.' \hfill \citep[][54]{hyman1970}
\ex Yesterday past continuous	\\
\gll wò {ɓei} sii {\=obw\=\i }\\
3\textsc{sg} \textsc{y.pst} buy groundnut\\
\glt `S/he was buying groundnuts.' \hfill \citep[][54]{hyman1970}
\ex Beyond \isi{yesterday past} continuous	\\
\gll wò {ɓei} si {\=obw\=\i } \\
3\textsc{sg} \textsc{by.pst} buy groundnut \\
\glt `S/he was buying groundnuts.' \hfill \citep[][54]{hyman1970}
\z
\z
While \ili{Gwari} is like Guébie in having optional auxiliaries, the data in \REF{ex:11:gwari} distinguish the two types of languages. In Guébie, as we saw above, the presence of any overt \isi{auxiliary} forces a change from VO to OV order. In \ili{Gwari}, the presence versus absence of the past \isi{tense marker} does not result in such an alternation. Because the presence of an overt \isi{auxiliary} must block the \isi{movement} of verbs to T, the persistence of VO word order in the presence of an \isi{auxiliary} suggests that the \ili{Gwari} has a head-initial (VO) VP, unlike Guébie.

The fact that \ili{Gwari} has a head-initial VP correlates with other head-initial properties of \ili{Gwari}, including prepositions \REF{ex:12a:gwari} and verb-adverb order \REF{ex:12c:gwari}, although the presence of genitive-noun order is a head-final property \REF{ex:12b:gwari}. 

\ea
\ea  \label{ex:12a:gwari} {Prepositions/Postpositions}\\
\gll wo tú shnamá l\=o ó t\=ebùl-\`{} \\
3\textsc{sg} put yam \textsc{stat} \textsc{loc} table-\textsc{loc} \\
\glt `S/he is putting the yam on the table.'

\ex \label{ex:12b:gwari} {Genitive-Noun}\\
\gll {\=eɓí} {yàɓà}\\
child banana \\
\glt `the child's banana' \hfill \citep[][25]{hyman1970}

\ex \label{ex:12c:gwari} {V-Adverb}\\
\gll yi gô àkyàuta {c\=\i c\=\i} \\
1\textsc{pl} buy gifts always \\
\glt `We always buy gifts.' \hfill \citep[][51]{hyman1970}
\z
\z
In fact, genitive-noun word order is the most common exception to head-initiality in West African languages, which otherwise show word orders that correspond typologically with head-initial VPs, as also noted in \citet{heine76}. Genitive-noun order plays a critical role in the discussion of \ili{Fongbe} below.

While it is generally head-initial, some auxiliaries in \ili{Gwari} trigger OV order, most notably the \isi{completive aspect} marker.\footnote{This pattern is also found in closely related \ili{Nupe} in the completive, analyzed in \citet{kandy03}, whose analysis shares several elements with ours, as discussed further below.} Completive \isi{aspect} is marked with an \isi{auxiliary} that occurs between the \isi{subject} and VP. Unlike the \isi{past tense} however, where we see the surface order SAuxVO, completive-marked sentences have the surface order SAuxOV:

\ea \label{ex:13:gwari}
\ea  {\textit{lá}: singular objects}	\\
\gll wó lá shnamá si \\
3\textsc{sg} \textsc{compl:sg} yam buy \\
\glt `S/he has bought a yam.' \hfill \citep[][64]{hyman1970}

\ex {\textit{kú}: plural objects}	\\
\gll wó  kú àshnamá si \\
3\textsc{sg} \textsc{compl:pl} yams buy \\
\glt `S/he has bought yams.' \hfill \citep[][56]{hyman1970}
\z
\z
In addition to the difference in word order that these two auxiliaries enforce, they behave differently with respect to agreement. As shown in \REF{ex:13:gwari}, the completive \isi{auxiliary} agrees with the number of the object. This is not the case for the \isi{past tense} \isi{auxiliary}, which does not agree with either the \isi{subject} or the object; this agreement relationship is indicative of a closer syntactic relationship between the completive and the object than the past \isi{tense marker}.

Now the \isi{past tense} \isi{auxiliary} and completive \isi{auxiliary} may be combined, as shown in \REF{ex:14:gwari}. In such sentences the \isi{past tense} precedes the \isi{completive aspect}, indicating that \isi{past tense} is structurally higher than the completive, following the general head-initiality of \ili{Gwari} \isi{clause structure}. When both past and completive markers are present, the surface word order is SAuxAuxOV, in the today and before \isi{yesterday past}, or SAuxOV, with the completive and tense markers fusing, in the \isi{yesterday past}.

\ea \label{ex:14:gwari}
\ea \label{ex:14a:gwari} {Today past completive}	\\
\gll w-a kú àshnamá si  \\
3\textsc{sg}-\textsc{t.pst} \textsc{compl:pl} yams buy \\
\glt `S/he bought yams.' [today] \hfill \citep[][57]{hyman1970}
\ex \label{ex:14b:gwari} {Yesterday past completive}	\\
\gll wò  k\=uì àshnamá si  \\
3\textsc{sg} \textsc{y.pst}.\textsc{compl:pl} yams buy \\
\glt `S/he has bought yams.' [yesterday] \hfill \citep[][57]{hyman1970}
\ex \label{ex:14c:gwari}{Beyond \isi{yesterday past} completive}	\\
\gll wò {ɓei} kú àshnamá si  \\
3\textsc{sg} \textsc{by.pst} \textsc{compl:pl} yams buy \\
\glt `S/he has bought yams.' [before yesterday] \hfill \citep[][57]{hyman1970}
\z
\z
The fact that the \isi{past tense} and \isi{completive aspect} can be combined in this way demonstrates that they are not competing for the same structural position. While the high tense \isi{auxiliary} is like its \isi{auxiliary} counterparts in Guébie and \ili{Dafing}, for example in hosting the \isi{subject} in its specifier position, the lower completive \isi{auxiliary} has no clear counterpart in those languages. Furthermore, it is the presence of the completive which is responsible for OV order. We now demonstrate how the completive can have this effect syntactically.

The tree in \figref{fig:sande:GwariVO} illustrates an analysis of a \ili{Gwari} sentence with a \isi{past tense} \isi{auxiliary} in T. The verb originates in a VP projection where the object is base-generated and receives its theta role. The verb obligatorily moves to a distinct $v$ head which introduces the external argument (not shown), resulting in SVO order.

\begin{figure}[p]
% % {\scriptsize \jtree[xunit=2.5em,yunit=1.25em]
% % \psset{labelgap=0}
% % \! = {TP}
% % : ({DP$_{sbj}$}!a ) 
% % : ({T}!b ) [scaleby=1.25 ] {$v$P}
% % : ({\rnode{2}{V+$v$}}!c ) {VP}
% % : ({\rnode{1}{\sout{V}}}!e ) ({DP$_{obj}$}!d ) .
% % \!a = {wo}{\textsc{3sg}}.
% % ɓ = {{\texthtb{}é\=\i}}{\textsc{t.pst}}.
% % \!d = <vartri>{àshnamá}{\rnode{4}{yams}}.
% % \!c = {si}{b\rnode{2}{u}y}.
% % \ncbar[angle=-90,nodesep=3pt,arm=2.5em,arrowscale=1]{->}{1}{2}
% % %\ncbar[angle=-90,nodesep=2.5pt,arm=2.25em,arrowscale=1]{->}{6}{4}
% % \endjtree}	
\begin{forest} fairly nice empty nodes, for tree={fit=tight}
[TP
    [DP\textsubscript{sbj}\\wo\\\textsc{3sg}] [ 
        [T\\ɓéī\\\textsc{t.pst}]  
            [\textit{v}P [V+\textit{v}\\si\\buy,name=buy] [VP
                        [\sout{V},name=V] [DP\textsubscript{obj} [
                            àshnamá\\yams,roof
                            ]
                        ]
                    ]
                ]
    ]
]
\draw[-{Triangle[]}] (V.south) |- +(-.5,-.25) -| (buy.south);
\end{forest}
\caption{Structure for SAuxVO in Gwari\label{fig:sande:GwariVO}}
\end{figure}

\begin{figure}[p]
% % {\scriptsize \jtree[xunit=2.5em,yunit=1.25em]
% % \psset{labelgap=0}
% % \! = {TP}
% % : ({DP$_{sbj}$}!a ) 
% % : ({T}!b ) [scaleby=1.25 ] {$v$P}
% % : ({V$_\textsc{compl}+v$}!c ) [scaleby=1.75 ] {VP$_\textsc{compl}$}
% % : ({\rnode{5}{DP$_{obj}$}}!d ) [scaleby=1.25] 
% % "{\psset{scaleby=.75}} : {\rnode{1}{\sout{V$_\textsc{compl}$}}} {VP}
% % : ({V}!f ) {\sout{\rnode{6}{DP$_{obj}$}}}.
% % \!a = {w(o)}{\textsc{3sg}}.
% % ɓ = {-á}{\textsc{t.pst}}.
% % \!c = {kú}{\rnode{2}{\textsc{compl:pl}}}.
% % \!d = <vartri>{àshnamá}{\rnode{4}{yams}}.
% % \!f = {si}{buy}.
% % \ncbar[angle=-90,nodesep=3pt,arm=3em,arrowscale=1]{->}{1}{2}
% % \ncbar[angle=-90,nodesep=2.5pt,arm=2.25em,arrowscale=1]{->}{6}{4}
% % \endjtree}
% % \vspace{1.75em}
\begin{forest} for tree={fit=tight}, fairly nice empty nodes
[TP
    [DP\textsubscript{sbj}\\w(o)\\\textsc{3sg}] [
        [T\\-á\\\textsc{t.pst}] [\textit{v}P
            [V\textsubscript{\textsc{compl}}\,+\,\textit{v}\\kú\\\textsc{compl:pl},name=ku] [VP\textsubscript{\textsc{compl}} 
                [DP\textsubscript{obj} [àshnamá\\yams,roof,name=yams]] [
                    [\sout{V\textsubscript{\textsc{compl}}},name=compl] [VP
                        [V\\si\\buy,name=buy] [\sout{DP\textsubscript{obj}},name=dp]]
                    ]
                ]
            ]
        ]
]
\draw[-{Triangle[]},overlay] (dp.south) |- (buy.south) -| (yams.south);
\draw[-{Triangle[]},overlay] (compl.south) |- +(-1,-1.5) -| (ku.south);
\end{forest}
\caption{Structure of SAuxOV in Gwari\label{fig:sande:Gwari-OV}}
\end{figure}

%A refined analysis of \ili{Gwari} SAuxVO is presented in \figref{fig:sande:\ili{Gwari}-movement}. Adopting standard analyses of \isi{clause structure}, we propose that there are two additional projections between TP and VP -- an aspectual phrase, AspP, and an additional verb phrase layer, $v$P. We propose that the object DP always undergoes \isi{object shift} to the specifier of $v$P, the position typically assumed for \isi{object shift} in Scandanavian languages \citep{vikner06}. Crucially, in SVO and SAuxVO clauses, this shift is obscured by the verb moving from V, through $v$, to Asp.
%
%\begin{figure}
%{\scriptsize \jtree[xunit=2.5em,yunit=1.25em]
%\psset{labelgap=0}
%\! = {TP}
%: ({DP$_{sbj}$}!a ) 
%: ({T}!b ) [scaleby=1.25 ] {AspP}
%: ({$v$+Asp}!e ) [scaleby=1.65 ] {$v$P}
%: ({\rnode{5}{DP$_{obj}$}}!d ) [scaleby=1.25] 
%"{\psset{scaleby=.75}} : {\sout{\rnode{1}{$v$}}} {VP}
%: ({\sout{V}}) {\sout{\rnode{6}{DP$_{obj}$}}}.
%\!a = {wo}{\textsc{3sg}}.
%ɓ = {{\texthtb{}é\=\i}}{\textsc{t.pst}}.
%\!d = <vartri>{àshnamá}{\rnode{4}{yams}}.
%\!e = {si}{b\rnode{2}{u}y}.
%\ncbar[angle=-90,nodesep=3pt,arm=3.5em,arrowscale=1]{->}{1}{2}
%\ncbar[angle=-90,nodesep=2.5pt,arm=2.25em,arrowscale=1]{->}{6}{4}
%\endjtree}
%\vspace{1.75em}
%\caption{SAuxVO via object Shift + verb movement}
%\label{fig:sande:\ili{Gwari}-movement}
%\end{figure}

Following \citet{kandy03}, we assume that completive auxiliaries originate in a completive V head (AgrO in \citet{kandy03}), which intervenes between V and $v$, blocking \isi{movement} of the main verb to V, and moving to $v$ in its place. In addition, the completive head triggers \isi{movement} of the object to its specifier, where it agrees with the object in number \REF{ex:13:gwari}. The result is the SAuxOV word order in the \isi{completive aspect}, shown in \figref{fig:sande:Gwari-OV}.

Support for the idea that the completive is still a kind of lexical verb, rather than an \isi{auxiliary}, comes from its transparent identity to the lexical verbs \textit{lá} `take', and \textit{kú} `collect,' which occur with singular versus plural objects, respectively \citep[63]{hyman1970}:

\ea \label{ex:Gwari-takecollect}

\ea \label{ex:Gwari-take} {\ili{Gwari} `take' as a main verb}	\\
\gll wo lá shnamá l\=o \\
3\textsc{sg} take yam \textsc{stat} \\
\glt `S/he is taking a yam' \hfill \citep[][92]{hyman1970}

\ex \label{ex:Gwari-gather} {\ili{Gwari} `gather' as a main verb}	\\
\gll wo kú àshnamá l{\=o}\\
3\textsc{sg} take yam \textsc{stat}\\
\glt `S/he is taking some yams' \hfill \citep[][93]{hyman1970}
\z
\z

Compare \citet{aboh09} for a similar analysis of serial verbs involving `take' in \ili{Gbe} langauges.

Evidence for the idea that object \isi{movement} is responsible for OV orders in \ili{Gwari} comes from \isi{double object} constructions. When there is no completive \isi{auxiliary}, as in \REF{ex:Gwari-DOC-VOO}, the verb precedes both objects. In completive clauses, however, the verb occurs between the two objects. Either order of arguments is possible, as seen in (\ref{ex:Gwari-DOC}b-c). 

\ea \label{ex:Gwari-DOC}

\ea \label{ex:Gwari-DOC-VOO} {SVO$_1$O$_2$}	\\
\gll wo bma mi b\={u}sì ya lo \\
3\textsc{sg} break 1\textsc{sg} stick \textsc{part} \textsc{stat} \\
\glt `S/he is breaking my stick' \hfill \citep[][92]{hyman1970}\pagebreak

\ex \label{ex:Gwari-DOC-O1VO2} {SAuxO$_1$VO$_2$}	\\
\gll wó lá  b\={u}sì bmà mi ya \\
3\textsc{sg} \textsc{compl:sg} stick break 1\textsc{sg} \textsc{part}  \\
\glt `S/he has broken my stick' \hfill \citep[][93]{hyman1970}

\ex \label{ex:Gwari-DOC-O2VO1}{SAuxO$_2$VO$_1$}	\\
\gll wó lá mí bmà b\={u}sì ya \\
3\textsc{sg} \textsc{compl:sg} 1\textsc{sg} break stick \textsc{part} \\
\glt `S/he has broken my stick'	\hfill \citep[][93]{hyman1970}
\z

\z
Double object constructions provide evidence against a head-final VP analysis. If the \ili{Gwari} VP were head-final, then we would expect both objects to precede the verb when it does not move to Asp. The current analysis, on the other hand, accounts for this in a simple way: either object in a \isi{double object} construction is able to move to the specifier of V$_\textsc{compl}$P.

What we have seen is that \ili{Gwari} is uniformly head-initial in its clausal spine. When apparent SAuxOV word order still emerges, it is not due to mixed clausal \isi{headedness}, but instead due to a combination of \isi{verb movement} of a low \isi{aspect} head combined with object shift --- a simple schematic representation of the structure is SAuxV$_1$OV$_2$. Like Guébie, \isi{verb movement} plays a crucial role in the alternation between VO and OV orders. In \ili{Gwari}, SAuxOV order only emerges when \isi{verb movement} to Asp is blocked. However, in \ili{Gwari}, unlike in Guébie, \isi{object shift} plays a crucial role. Namely, SAuxOV order only occurs because \isi{object shift} is independent from \isi{verb movement} to Asp. This is markedly different from Guébie and \ili{Dafing}, where VP is always head final while TP is head initial.

\subsection{Fongbe}

 In this section, we will see that in \ili{Fongbe}, apparent SAuxOV order emerges from a distinct construction: nominalization.  \ili{Fongbe} is a \ili{Kwa} language spoken in Benin. \ili{Fongbe} shows SVO order in main clauses without an \isi{auxiliary}, as seen in \REF{ex:Fongbe-SVO}. Like \ili{Gwari}, \ili{Fongbe} has a set of auxiliaries that occur with SAuxVO word order, such as the habitual in \REF{ex:Fongbe-SAuxVO}.%%TODO: add example of SAuxVO in \ili{Fongbe}. Does this exist? We don't currently say whether ALL auxiliaries result in SAuxOV or if we sometimes see SAuxVO.

\ea \label{ex:Fongbe-SVO}  
{SVO}\\
\ea 
\gll K{{\`ɔ}kú} xò Àsíbá \\
Koku hit Asiba \\
\glt `Koku hit Asiba.' \hfill \citep[][247]{lefebvre2002}
\ex \label{ex:Fongbe-SAuxVO}
 {SAuxVO}\\
\gll Lili {n{\`ɔ}} {ɖù} {gbàɖé}  \\
Lili \textsc{hab} eat corn \\
\glt `Lili (habitually) eats corn.' \hfill \citep[][94]{lefebvre2002}
\z
\z 
Other auxiliaries occurring in the same position as the habitual above include the future, irrealis, and anterior markers. So SAuxVO is the general word order in clauses with auxiliaries in \ili{Fongbe}.

Like \ili{Gwari}, \ili{Fongbe} displays mixed \isi{headedness} properties, an issue which is examined in detail in \citet{aboh04} for \ili{Kwa} languages in general. Like in many \ili{Kwa} languages, \ili{Fongbe} nominal complements precede the noun that selects them, a head-final characteristic \REF{ex:Fongbe-CompN}. On the other hand, possessors follow the noun they modify, a head-initial characteristic, \REF{ex:Fongbe-NGen}.

\ea
\ea  \label{ex:Fongbe-CompN} {Comp-N}\\
\gll càkpálò sín gò {\'ɔ} \\
beer \textsc{obj} bottle \textsc{def} \\
\glt `the bottle of beer' \hfill \citep[][45]{lefebvre2002}
\ex \label{ex:Fongbe-NGen} {N-Gen}\\
\gll àwà ví {\'ɔ} {t{\`ɔ}n} \\
arm child \textsc{def} \textit{gen} \\
\glt `the child's arm' \hfill \citep[][45]{lefebvre2002}
\z
\z 

Other word order properties also give similar mixed results. \ili{Fongbe} has both prepositions and postpositions, as shown in \REF{ex:Fongbe-PP}. Verbs precede adverbial modifiers, as shown in \REF{ex:Fongbe-Adv}. 

\ea
\ea \label{ex:Fongbe-PP} {Pre- and postpositions}\\
 \gll K{\`ɔ}kú xò às{\'ɔ}n nú  Àsíbá \\
Koku buy crab for Asiba \\
\glt `Koku bought crab for Asiba' \hfill \citep[][302]{lefebvre2002}
\ex \gll K{\`ɔ}kú ɖò àxì m{\`ɛ}\\
Koku be.at market in \\
\glt `Koku is in the market' \hfill \citep[][325]{lefebvre2002}
\z
\z

\ea \label{ex:Fongbe-Adv}
{V-Adv}\\
\gll K{{\`ɔ}kú} wà {àz{\'ɔ}} gànjí \\
Koku do work well \\
\glt `Koku worked well' \hfill \citep[][381]{lefebvre2002}
\z

While it has some head-final properties, \ili{Fongbe} is largely head-initial at the level of the clause. We demonstrate below that apparent OV order is not due to mixed clausal \isi{headedness} in \ili{Fongbe} but rather due to a nominalized complement of a lexical verb.

Our main interest here is what \citet{lefebvre2002} call an ``aspectual verb construction''. Superficially, this construction has SAuxOV word order, in that the lexical verb in the clause is preceded by its object, as shown in \REF{ex:Fongbe-SVOV}. 

\ea \label{ex:Fongbe-SVOV} {SV[OV]$_{\textsc{nom}}$}\\
\ea \gll Àsíbá  {ɖò} [[ ví {\'ɔ}  {kp{\'ɔ}n} ]] {w{\`ɛ}} ]  \\
Asiba be.at {} child \textsc{def} look.at.\textsc{nom} {} \textsc{post}  \\
\glt `Asiba is looking at the child' \hfill \citep[][215]{lefebvre2002}

\ex \label{ex:Fongbe-SVOVa} \gll Ùn {\`ɛ} [[  {nú}  {ɖù}  ] jí ]  \\
1\textsc{sg} fall {} thing eat.\textsc{nom} {} on  \\
\glt `I began to eat.' \hfill \citep[][215]{lefebvre2002}
\z
\z

However, as can clearly be seen from the data in \REF{ex:Fongbe-SVOV}, the aspectual verbs \textit{{ɖò}} `be at' and \textit{{\`ɛ}} `begin' (lit. `fall') actually take a PP complement, the head of which selects a \isi{nominalized verb} phrase. 

This fact makes the \ili{Fongbe} aspectual verb construction quite different from the constructions we have examined so far. In the other languages surveyed, SAuxOV word order involves a single extended projection of a lexical verb, and the placement of that verb in relation to its object changes based on the properties of heads higher in the clausal spine. In \ili{Fongbe}, apparent OV order involves a \isi{nominalized verb}. The inflected verb here is a lexical verb that selects a PP complement; it is not an \isi{auxiliary}. Aspectual verbs in \ili{Fongbe} retain their lexical uses. For example, the verb \textit{{\`ɛ}} in \REF{ex:Fongbe-SVOVa} can be used to simply mean `fall'. Thus, these word orders are better labeled SVGenN or SVO than SAuxOV.

The data above demonstrate \ili{Fongbe} is head initial for both TP and VP. In SVO clauses, no \isi{movement} is needed to derive the word order, as shown in \figref{fig:sande:Fongbe-SAuxVO}. 

\begin{figure}
% % % {\footnotesize \jtree[xunit=2.5em,yunit=1.25em]
% % % \psset{labelgap=0}
% % % \! = {TP}
% % % : ({DP}!a ) [scaleby=1.25]
% % % : {T}!b {VP}
% % % : {V}!c {DP}{Àsíbá}.
% % % \!a = {K{{\`ɔ}kú}}.
% % % ɓ = {\O{}}.
% % % \!c = {xó}.
% % % \endjtree}
\begin{forest} nice empty nodes
[TP
    [DP\\K{\`ɔ}kú] [ 
        [T\\∅] [VP
            [V\\xó] [DP\\Àsíbá]
        ]
    ]
]
\end{forest}
\caption{SVO Structure in Fongbe}
\label{fig:sande:Fongbe-SAuxVO}
\end{figure}

In contrast apparent SAuxOV order in \ili{Fongbe} occurs when a main verb selects a PP complement.\footnote{See \citet{aboh10} for discussion of the structure of \ili{Kwa} noun phrases as well as an account of the combined pre- and postpositions typical of \ili{Kwa}. Unlike our analysis below, Aboh adopts uniform head-initial structures with righward complements moving to specifier positions to the left of the noun.} The head of the PP, in turn, selects a nominalized VP complement. The structure is shown in \figref{fig:sande:Fongbe-SVOV}.

\begin{figure}
% % % {\footnotesize \jtree[xunit=2.5em,yunit=1.25em]
% % % \psset{labelgap=0}
% % % \! = {TP}
% % % : ({DP}!a )
% % % : [scaleby=.9] ({T}) [scaleby=.9] {VP}
% % % : ({V}!b ) [scaleby=1.5] {PP}
% % % "{\psset{scaleby=.75}} : ({DP}!c ) ({P}{{w{\`ɛ}}}).
% % % \!a = {Àsíbá}.
% % % ɓ = {{ɖò}}{be.at}.
% % % \!c = : ({VP$_{\textsc{nom}}$}!d ) {D}.
% % % \!d = "{\psset{scaleby=.9 }} : {DP}!e  {V$_{\textsc{nom}}$}{{kp{\'ɔ}n}}{look.at}.
% % % \!e = <vartri>{ví {\'ɔ}}{the child}.
% % % \endjtree}
\begin{forest}fairly nice empty nodes
[TP
    [DP\\Àsíbá] [
        [T] [VP
            [V\\ɖò\\be.at] [PP
                [DP
                    [VP\textsubscript{\textsc{nom}} 
                        [DP[ví {\'ɔ}\\the child,roof]] [V\textsubscript{\textsc{nom}}\\kp{\'ɔ}n\\look.at]
                    ] [D]
                ] [P\\w{\`ɛ}]
            ]
        ]
    ]
]
\end{forest}
\caption{SVOV Structure in Fongbe; cf. \citet[ch. 6]{aboh04}}
\label{fig:sande:Fongbe-SVOV}
\end{figure}

Because nominal complements always precede the noun that selects them in \ili{Fongbe} \REF{ex:Fongbe-CompN}, apparent OV order inside the nominal VP arises simply because Gen-N is the normal order for noun phrases, including nominal complements. Because \ili{Fongbe} is head-initial in verb phrases, the aspectual verb precedes its complement, and this gives rise to apparent SAuxOV order. In fact, however, this is simply SVGenN word order, where N is a \isi{nominalized verb}.

\subsection{Summary}

We have seen that neither \ili{Gwari} (Nupoid) nor \ili{Fongbe} (\ili{Kwa}) has a head-final VP, and therefore OV order is not a general organizing characteristic of their clausal architecture. This makes them different from Guébie and \ili{Dafing} in several ways. First, surface OV order has a restricted distribution in both languages. In \ili{Gwari}, it occurs only when there is a completive verb which triggers \isi{object shift} and blocks \isi{movement} of the lexical verb. In \ili{Fongbe}, OV order only occurs in \isi{nominalized verb} phrases. Second, outside these narrow contexts, auxiliaries occur with VO word order. Under our analysis of \ili{Gwari} and \ili{Fongbe}, this is because these auxiliaries occupy the T$^0$ head of TP, and TP is head-initial.

The derivation of apparent SAuxOV word order in \ili{Gwari} differs from that in \ili{Fongbe}. In \ili{Gwari}, a combination of \isi{object shift} and lack of verb raising conspires to yield apparent SAuxOV orders, orders that we noted were in fact S(Aux)V$_1$OV$_2$. In \ili{Fongbe}, OV order emerges in nominalized complements to certain aspectual verbs, so the \ili{Fongbe} order is in fact S(Aux)VGenN. One path forward for formal typological research is to identify how much variation there is within languages with apparent SAuxOV structures. It seems certain that both phenomena (\isi{object shift}, nominalized complements) are relatively common in West Africa, the latter in particular given the frequency of GenN word order.

There are additional cases of apparent SAuxOV in West Africa that are conditioned by other factors. For example, \isi{object shift} is obligatory with pronouns in \ili{Ogoni} languages such as \ili{Kana} \citep{ikoro96}, and it is conditioned by \isi{negation} in Leggbó \citep{good07}. Yet all of these cases, occurring in languages spoken well to the east of the Mandesphere, should not be conflated with the mixed clausal \isi{headedness} which is at the root of SAuxOV in \ili{Kru} and \ili{Mande} languages.


\section{Survey results: Distribution of SAuxOV}\label{sec:distribution}

In this section we examine the distribution of SAuxOV order with mixed-head\-ed\-ness within the Macro-Sudan Belt, and specifically within West Africa. In order to carry out this structure-based typological study, we followed three steps: 1) establishing a relevant structure, 2) identifying structural diagnostics based on descriptive facts, and 3) conducting a survey on the basis of those structural diagnostics. These three steps result in a typological survey based on both hierarchical structure and descriptions of linear word order properties. 

Step one, above, is discussed in \sectref{sec:strictSAuxOV}, where we define the relevant structure for SAuxOV with clausal mixed-\isi{headedness}. This structure involves a dedicated inflectional position immediately following the \isi{subject}, and general OV word order within the verb phrase. To address steps two and three, we identified 26 syntactic variables meant to identify SAuxOV structures, and we carried out a survey of 54 languages from the Macro-Sudan belt, recording the value for each syntactic variable whenever relevant information was available. Metadata about each language, the sources used to determine the survey responses for those languages, and where each language is spoken were collected. The survey was informed by both typology and hierarchical structure, examining word order properties that have been found to be most closely associated with head finality \citep{dryer92, dryer07}, those that determine \isi{headedness} within the VP, and those that distinguish SAuxOV due to clausal mixed-head\-ed\-ness  from verb-second languages and head-initial languages with \isi{object shift}. A full list of the 26 variables examined, along with the values of those variables reported for each language, is given in the appendix.

The languages surveyed comprise a diversity sample based on genetic affiliation and geography, loosely based on the sample used by \citet{clements08}. The number of languages in each family in our survey is given in \tabref{tab:sande:surveylangs}. The remainder of this section reports on the results of our survey.

\begin{table}
\caption{Languages included in survey} \label{tab:sande:surveylangs} 
\begin{tabularx}{\textwidth}{llQl}
\lsptoprule
{Language family} & {\textit{n}} & {Languages} & {Map key}\\
\midrule
Adamawa & 3 & Mundang, Mambay, Banda-Linda & A\\
Atlantic & 2 & \ilit{Sereer}, \ilit{Fula} & Z\\
Bantoid/\ilit{Bantu} & 2 & \ilit{Noni}, Bisa & B\\
Bongo-Bagirmi & 3 & Kabba, Kenga, Mbay & P\\
Chadic & 6 & \ilit{Hausa}, Pero, Mupun, Mina, Miya, Lele & C\\
Central \ilit{Sudanic} & 1 & Ma'di & S\\
Cross River & 1 & Khana & R\\
Dogon & 2 & Jamsay, Tommo So & D \\
Edoid & 2 & Esan, Degema & E\\
Ethio-Semitic & 1 & \ilit{Amharic} & V\\
Gbaya & 1 & \ilit{Ngbaka} & Y\\
\ilit{Gbe} &  2 & Maxi, \ilit{Fongbe} & F\\
\ilit{Gur}/Senufo & 7 & \ilit{Dagbani}, Bwamu, Bariba, Koromfe, \ilit{Supyire}, \ilit{Dagaare}, Lobi & G\\
Ijoid & 1 & Kalabari & I \\
Kordofanian & 1 & Otoro & O\\
\ilit{Kru} & 5 & Guébie, Godié, \ilit{Grebo}, \ilit{Wobe}, \ilit{Krahn} & K\\
\ilit{Kwa} & 2 & Tafi, \ilit{Akan} & W \\
\ilit{Mande} & 6 & \ilit{Mano}, \ilit{Dafing}, Bamana, Boko, Bobo, \ilit{Gouro} & M\\
Mel & 1 & Mani & U\\
\ilit{Nilotic} & 1 & Lango & L\\
Nupoid & 1 & \ilit{Gwari} & N \\
Saharan & 1 & Beria & X\\
Songhay & 2 & \ilit{Koyraboro Senni}, \ilit{Tondi Songway} Kiini & H\\
\lspbottomrule
\end{tabularx}
\end{table}


The map in \figref{fig:sande:sauxov} shows the distribution of languages with mixed-head\-ed\-ness in the clause leading to structural SAuxOV based on our survey. Each language is marked on the map with a colored letter, where the letter represents language family. The letter key is given in \tabref{tab:sande:surveylangs}. Colors represent different word order relationships between auxiliaries, objects, and verbs, where red represents SAuxOV order with mixed-\isi{headedness} in the clause. Language families and latitude and longitude for each language are determined from Glottolog \citep{Glottolog}.

\begin{figure}[p]
    \centering
    \includegraphics[width =0.9\textwidth]{figures/SAOV.png}
    \caption{Distribution of SAuxOV (red). The language in black is \ili{Dagbani}, a \ili{Gur} language in which we were not able to identify auxiliaries.} \label{fig:sande:sauxov}
\end{figure}


We see that there is a strong cluster of SAuxOV with clausal mixed-\isi{headedness} in West Africa. There is a strong centralization of SAuxOV orders in the area around \ili{Mande} and \ili{Kru} languages, which we call the \textit{Mandesphere} given the historical influence of the \ili{Mande}-speaking Mali Empire in this area.

%	\textitem[Guébie (\ili{Kru}, Côte d'Ivoire)] AdvOV, pervasively head final except for TP.
%	\textitem[\ili{Dafing} (\ili{Mande}, Burkina Faso)] OVAdv, head final in sub-clausal constituents.
%	\textitem[\ili{Gwari} (Nupoid, Nigeria)] VOAdv, pervasively head initial with serial verbs.
%	\textitem[\ili{Akan} (\ili{Kwa}) (??)]

\newpage 
In order to discover whether other head-final properties are distributed in the same way as SAuxOV structures with mixed \isi{headedness}, we look first at the distribution of postpositions, which closely mirror the postposition map of Africa from the World Atlas of Language Structure \citep{wals-85} (\figref{fig:sande:WALS}).

\begin{figure}[p]
    \centering
    \includegraphics[width = .7\textwidth]{figures/Post.png}
        \includegraphics[width = .7\textwidth]{figures/WALSPost.png}
    \caption{Distribution of postpositions in our survey (top) and WALS (bottom) \citep{wals-85}} \label{fig:sande:WALS}
\end{figure}

 %This confirms that our sample is representative of the word order properties throughout the Macro-Sudan belt.



Like postpositions, Genitive-Noun word order correlates with OV across languages \citep{dryer07}, and it is well known that adposition and genitive order track each other across languages based on their relationship in grammaticalization. The distribution of Genitive-Noun order given our survey is shown in \figref{fig:sande:Gen-Noun}. The WALS map of Genitive-Noun order in Africa shows a very similar distribution.

\begin{figure}
    \centering
        \includegraphics[width = .7\textwidth]{figures/GenN.png}
    \includegraphics[width = .7\textwidth]{figures/WALSGenN.png}
    \caption{Distribution of GenN in our survey (top) and WALS (bottom) \citep{wals-86}} \label{fig:sande:Gen-Noun}
\end{figure}

\cite{dryer07} also observes that OV languages surface with manner adverbs before verbs. However, we found that Manner Adverb-Verb order has a much narrower distribution within West Africa than are other head-final properties like postpositions, Genitive-Noun order, and even mixed-headed SAuxOV.

\begin{figure}
    \centering
    \includegraphics[width = .7\textwidth]{figures/VAdv.png}
    \caption{Distribution of Adv-V (red)}
    \label{fig:sande:AdvV}
  \end{figure}

Unlike the distribution of postpositions and Genitive-Noun order, which resemble the distribution of SAuxOV, the order of manner adverbs and verbs does not seem to correlate with other head-final properties in West Africa (\figref{fig:sande:AdvV}). This is likely due to the combination of VAdv and OV word order in \ili{Mande} and some \ili{Kru} languages.

Verb \isi{movement} also shows a different distribution from SAuxOV with mixed \isi{headedness}. We saw in Guébie, a language that shows clausal mixed \isi{headedness}, that when there is no \isi{auxiliary} present, the verb surfaces immediately after the \isi{subject} in the inflectional position. We analyze this SVO order as \isi{verb movement}. In \figref{fig:sande:vmove}, the combination of two word order properties determines whether \isi{verb movement} is present in a language: 1) word order when an \isi{auxiliary} is present (say, SAuxOV), and 2) word order in clauses without an \isi{auxiliary} (say, SVO). While the Mandesphere is almost entirely characterized by clausal mixed \isi{headedness}, only a subset of these languages shows \isi{verb movement}. Verb \isi{movement} is detectable in a number of head-initial languages, with SAuxVO order, based on the requirement that the verb need not be adjacent to the object, i.e., adverbs can intervene these two elements when an \isi{auxiliary} is absent. We conclude that \isi{verb movement} is independent from \isi{headedness}.

\begin{figure}
   \centering
    \includegraphics[width = .7\textwidth]{figures/VMove}
    \caption{Distribution of verb movement (red)}\label{fig:sande:vmove}
\end{figure}

The results of our survey are summarized in \tabref{tab:sande:wosummary}. We conclude that head-final properties like postpositions and Genitive-Noun order correlate strongly with clausal mixed \isi{headedness} (SAuxOV  order) in the Macro-Sudan belt, and specifically in West Africa. As head final properties are centered around the Mandesphere, along with clausal mixed \isi{headedness}, we concur with \citet{heine76} that Proto-\ili{Mande} was likely head final, and is likely the source of this areal pattern, particularly in light of the outsized economic and cultural influence of \ili{Mande} speakers in the West African history. The results of our survey show that only languages in the Mandesphere show clausal mixed-\isi{headedness}. The appearance of conditioned SAuxOV, discussed in  \sectref{sec:fake}, does not correlate as neatly with head-final properties as mixed \isi{headedness} does in the Mandesphere.

\begin{table}
\begin{tabular}{lcc}
\lsptoprule
& {Correlates with SAuxOV} & {Independent of SAuxOV}\\
\midrule
{Postpositions} & X & \\
{Genitive-Noun} & X & \\
{Verb-Adverb order} & & X\\
{Verb movement} & & X\\	
\lspbottomrule
\end{tabular}
    \caption{Head-final properties whose distribution correlates with mixed-headed SAuxOV}\label{tab:sande:wosummary}
\end{table}

The fact that clausal mixed \isi{headedness} is a better predictor of head-final properties than the presence of apparent SAuxOV such as those in \ili{Gwari} and \ili{Fongbe} highlights a more general point about syntactic typology we would like to emphasize: while many typological discussions of word order are based on surface order, the results in this section clearly demonstrate that syntactic typologies should be based on structural analyses of languages instead. The success of this approach in the survey above indicates that cross-linguistic tendencies about word order might be more profitably framed in terms of the underlying structures that give rise to these word orders rather the existence of various surface patterns.

\section{Conclusion}\label{sec:conclusion}
It has been understood since at least \citet{heine76} that SAuxOV word order is a typologically significant property of West African languages. More recently, \citet{guld08,guld11} has suggested that S(Aux)OVX (with emphasis on X) is a property of a \isi{linguistic area} he labels the Macro-Sudan Belt, similar to the \ili{Sudanic} zone of \citet{clements08}, which stretches west to Senegal and Guinea and east to the Central African Republic.  

A potential problem for this claim is that, as we have now seen, S(Aux)OVX is almost certainly not a single syntactic phenomenon. In particular, we must be careful to distinguish between the superficial appearance of such a word order with a structure that is actually distinct, as in \ili{Gwari} and \ili{Fongbe}, from the existence of genuine mixed clausal \isi{headedness} in \ili{Mande} and \ili{Kru}.

At the same time, the more fine-grained picture we have sketched clarifies a number of interesting historical and areal questions. For example, what is the distribution in West Africa of OV due to \isi{object shift} (as in \ili{Gwari}) versus OV due to nominalization (as in \ili{Fongbe})? Are these constructions generally found, and hence reconstructable, in their narrower language families? Are these structures more common among languages directly adjacent to the Mandesphere, suggesting a contact-based origin? While these questions can only be asked in the context of a structural analysis, such an approach should provide new insights into the history of linguistic change and contact in West Africa.


\section*{Acknowledgements}
We would like to thank the group on Areal Linguistic Features in Africa (ALFA) at UC Berkeley for their support and discussion, as well as for providing data on particular languages for our SAuxOV survey. ALFA members (other than the authors) include Larry Hyman, Emily Clem, Matthew Faytak, Jevon Heath, Maria Khachaturyan, Spencer Lamoureux, Florian Lionnet, Jack Merrill, and Nicholas Rolle. Thanks also to our two reviewers for their feedback. Finally, we would like to thank the Guébie community and Rassidatou Konate, our \ili{Dafing} consultant, for providing data discussed in this paper.

\pagebreak\section*{Abbreviations}
\begin{multicols}{2}
\begin{tabbing}
\textsc{compl}\hspace{.5em} \= third person\kill
\textsc{1} \>  \isi{first person}\\
\textsc{2} \>   \isi{second person}\\
\textsc{3} \>   \isi{third person}\\
\textsc{by} \>   before yesterday\\
\textsc{compl} \>   completive\\
\textsc{def}   \>   definite\\
\textsc{fut} \>   future\\
\textsc{gen} \>   genitive\\
\textsc{hab} \>  habitual\\
\textsc{ipfv} \>  imperfective\\
\textsc{irr} \>   irrealis\\
\textsc{loc} \>   location\\
\textsc{nmlz} \>   nominalizer\\
\textsc{nom} \>   \isi{nominative}\\
\textsc{obj} \>   object\\
\textsc{part} \>   particle\\
\textsc{pfv} \>   perfective\\
\textsc{pl} \>   plural\\
\textsc{post} \>   postposition\\
\textsc{pst}   \>   past\\
\textsc{red} \>   reduplication\\
\textsc{sg} \>   singular\\
\textsc{stat} \>   stative\\
\textsc{t.pst} \>   today past\\
\textsc{y} \>   yesterday
\end{tabbing}
\end{multicols}


% \appendix 
% \setcounter{secnumdepth}{0}
\section*{Appendix}

\begin{table}[hp]
\caption{Variables examined in the SAuxOV survey} \label{tab:sande:variables} 
\small
\begin{tabular}{ll}
\lsptoprule
	 & {Variable}\\
	 \midrule
	 1. & Relative order of O and V in clauses containing auxiliaries\\
	 2. & Relative order of adpositions and their object nouns\\
	 3. & Relative order of Gen and N in a genitive construction\\
	 4. & Relative order of S, Aux, O, and V in clauses containing auxiliaries\\
	 5. & Relative order of manner adverb and V in clauses containing auxiliaries\\
	 6. & Relative order of PP adjunct and non-locative V in clauses containing auxiliaries\\
	 7. & Relative order of CP adjunct and V in clauses containing auxiliaries\\
	 8. & Relative order of object \isi{pronoun} and V in clauses containing auxiliaries\\
	 9. & Relative order of NP/PP locative object and V in clauses containing auxiliaries\\
	 10. & Relative order of CP objects and V in clauses containing auxiliaries\\
	 11. & Relative order of V and multiple NP objects in clauses containing auxiliaries\\
	 12. & Relative order of theme and goal in clauses containing auxiliaries\\
	 13. & Relative order of \isi{pronoun} and full NP objects in clauses containing auxiliaries\\
	 14. & Whether it is possible for a sentence to lack an \isi{auxiliary}\\
	 15. & Relative order of S, V, and O when no \isi{auxiliary} is present\\
	 16. & Which inflectional categories auxiliaries can mark\\
	 17. & Whether multiple auxiliaries are possible in the same clause\\
	 18. & Whether there is an overt polar question marker\\
	 19. & Relative order of polar question marker with S, Aux, O, and V\\
	 20. & Position of Wh-words within Wh-questions\\
	 21. & Whether \isi{negation} is marked with an \isi{auxiliary} or other overt marker\\
	 22. & Position of non-\isi{auxiliary} negative markers within the clause\\
	 23. & Whether \isi{negation} affects clausal word order when an \isi{auxiliary} is present\\ 
	 24. & Position of complementizers within embedded clauses\\
	 25. & Whether objects can appear before auxiliaries (OAuxSV order)\\
	 26. & Whether adverbs can occur before an \isi{auxiliary} (AdvAuxSV order)\\
\lspbottomrule
\end{tabular} 
\end{table}

A list of variables extracted for our survey from grammars and from linguists with expertise in the languages examined is given in \tabref{tab:sande:variables}. The survey was conducted primarily in multiple choice format via Google Forms, with the option of choosing multiple possible word orders per question. Space was provided after each question to leave additional comments or examples. The particular variables chosen are meant to determine the \isi{headedness} properties of each language, along with which languages display mixed-\isi{headedness} within the clause, which languages have a dedicated tense/\isi{aspect} position immediately after the \isi{subject}, and whether \isi{verb movement} to the \isi{auxiliary} position is possible.

The values of the six variables most relevant for the results presented in this paper are given in \tabref{tab:sande:results1} and \tabref{tab:sande:results2} for each language in our survey. A * after the result means that the specified word order only occurs in the case of (nominalized) V complements of aspectual verbs. For further results and survey details, please contact the authors.

\begin{table}[hp]
\caption{Survey results} \label{tab:sande:results1} 
\footnotesize
\begin{tabular}{p{0.8in}llllll}
\lsptoprule
{Language}	& 1	& 2	& 3	& 4	& 5	& 15\\
\midrule
Otoro	& \textcolor{blue}{VO}	& Pre, Post 	& NG	& SAuxVO	& VAdv	& SVOX \\
Guébie	& \textcolor{red}{OV}	& Post 	& GN	& SAuxOV	& AdvV	& SVOX, SVXO\\
\ilit{Mano}	& \textcolor{red}{OV}	& Post 	& GN	& SAuxOV	& VAdv	& SOVX\\
Bamana	& \textcolor{red}{OV}	& Pre, Post 	& GN	& SAuxOV	& AdvV, VAdv	& SOVX\\
Mani	& \textcolor{red}{OV}	& Pre, Post 	& NG	& SAuxOV	&  VAdv	& SVOX\\
Godié	& \textcolor{red}{OV}	& Post 	& GN	& SAuxOV	& AdvV	& SVOX, SVXO\\
Boko/Busa	&  \textcolor{red}{OV}	& Post 	& GN	& SAuxOV	& VAdv	& SOVX\\
\ilit{Grebo}	& \textcolor{red}{OV}	& Post 	& GN	& SAuxOV	& 	& SVOX\\
\ilit{Wobe}	& \textcolor{red}{OV}	& Post 	& GN	& SAuxOV	& VAdv	& SVOX\\
\ilit{Krahn}	& \textcolor{red}{OV}	& Post 	& GN	& SAuxOV	& VAdv	& SVOX\\
Bobo	& \textcolor{red}{OV}	& Post 	& GN	& SAuxOV	& VAdv	& SVOX\\
Bisa	& \textcolor{red}{OV}	& Post 	& GN	& SAuxOV	& VAdv	& SVOX, SOVX\\
\ilit{Dagbani}	& \textcolor{blue}{VO}	& Post 	& GN	& SVO (no Aux)	& 	& SVOX\\
Jamsay	& \textcolor{red}{OV}	& Post 	& GN	& SOVAux	& AdvV	& SOXV\\
Tafi	& \textcolor{blue}{VO}	& Pre 	& NG	& SAuxVO	& VAdv	& SVOX\\
Bwamu	& \textcolor{blue}{VO}	& Pre, Post 	& GN	& SAuxVO	& AdvV	& SVOX\\
Bariba	& \textcolor{red}{OV}	& Post 	& GN	& SAuxOV	& 	& SVOX, SVXO\\
Mundang	& \textcolor{blue}{VO}	& Pre 	& NG	& SAuxVO	& VAdv	& SVOX \\
Koromfe	& \textcolor{blue}{VO}	& Post 	& GN	& SAuxVO	& VAdv	& SVOX \\
\ilit{Gwari}	& \textcolor{red}{OV}, VO	& Pre 	& GN	& SAuxVO	& VAdv	& SVOX \\
Mambay	& \textcolor{blue}{VO}	& Pre 	& NG	& SAuxVO	& VAdv& SVOX\\
\ilit{Sereer}	& \textcolor{blue}{VO}	& Pre 	& NG	& SAuxVO	& VAdv& SVOX \\
\ilit{Supyire}	& \textcolor{red}{OV}	& Post 	& GN	& SAuxOV	& VAdv& SOVX\\
Esan	& \textcolor{blue}{VO}	& Pre 	& NG	& SAuxVO	& AdvV	& SVOX \\
\ilit{Noni}	& \textcolor{blue}{VO}	& Pre 	& NG	& SAuxVO	& VAdv	& SVOX \\
\ilit{Hausa}	& \textcolor{blue}{VO}	& Pre 	& NG	&SAuxVO	& VAdv	& SVOX \\
\lspbottomrule
\end{tabular} 
\end{table}

 
\begin{table}[hp]
\caption{Survey results (cont.)} \label{tab:sande:results2} 
\footnotesize
\begin{tabular}{p{13mm}p{10mm}lllll}
\lsptoprule
{Language} & 1 & 2 & 3 & 4 & 5 & 15\\
\midrule
\ilit{Koyraboro Senni}	& \textcolor{red}{OV}	& Post 	& GN	& SAuxOV	& VAdv	& SOVX\\
\ilit{Tondi Songway} Kiini	& \textcolor{blue}{VO}	& Post 	& GN	& SAuxOV	& VAdv	& SOVX\\
Dogon	& \textcolor{red}{OV}	& Post 	& GN	& SOVAux	& AdvV	& SXOV \\
Mupun	& \textcolor{blue}{VO}	& Pre 	& NG	& SAuxVO	& VAdv	& SVOX \\
Pero	& \textcolor{blue}{VO}	& Pre 	& NG	& SAuxVO	& VAdv	& SVOX\\
\ilit{Amharic}	& \textcolor{red}{OV}	& Pre, Post 	& GN	& SOVAux	& AdvV	& SOXV, SXOV\\
Maxi	& \textcolor{blue}{VO}	& Pre 	&  GN	& SAuxVO	& VAdv	& SVOX \\
Degema	& \textcolor{blue}{VO}	& Pre 	& NG	& SAuxVO	& VAdv& SVOX \\
Pulaar	& \textcolor{blue}{VO}	& Pre 	& NG	& SAuxVO	& VAdv	& SVOX \\
Mina	& \textcolor{blue}{VO}	& Pre 	& NG	& SAuxVO	& VAdv	& SVOX \\
Ma'di	& \textcolor{red}{OV}	& Post 	& GN	& SAuxOV	& AdvV, VAdv	& SOVX \\
\ilit{Dagaare}	& \textcolor{blue}{VO}	& Post 	& NG	& SAuxVO	& VAdv	&SVOX \\
Khana & \textcolor{blue}{VO} & Pre  & NG & SAuxVO & VAdv & SVOX \\
Lango & \textcolor{blue}{VO} & Pre  & NG & SAuxVO & VAdv & SVOX \\
Kabba & \textcolor{blue}{VO} & Pre, Post  & NG & SAuxVO & VAdv & SVOX \\
Miya & \textcolor{blue}{VO} & Pre  & GN & SAuxOV, AuxOVS & VAdv & SVOX, VOXS\\
Banda-linda & \textcolor{blue}{VO} & Pre & NG & SAuxVO & VAdv & SVOX \\
\ilit{Fongbe} & \textcolor{blue}{VO} (OV in gerunds) & Pre, Post & NG & SAspOV* & VAdv & SVOX \\
Kalabari & \textcolor{red}{OV} & Post & GN & SOVAux & AdvV & SXOV \\
\ilit{Akan} & \textcolor{blue}{VO} (OV in gerunds) & Post & GN & SAspOV* & VAdv & SVOX \\
Beria & \textcolor{red}{OV} & Post & GN & SOVAux & AdvV & SOXV, SXOV\\
Kenga & \textcolor{blue}{VO} & Pre &  NG & SAuxVO & VAdv & SVOX \\
\ilit{Ngbaka} & \textcolor{blue}{VO} & Pre & NG & SAuxVO & VAdv & SVOX \\
Mbay & \textcolor{blue}{VO} & Pre & NG & SAuxVO &  VAdv & SVOX \\
Lele & \textcolor{blue}{VO} & Post & NG & SAuxVO & VAdv & SVOX \\
\ilit{Gouro} & \textcolor{red}{OV} & Post & GN & SAuxOV, SAuxVO & VAdv & SVOX, SOVX\\
Lobiri & \textcolor{blue}{VO} & Pre & GN & SAuxVO & VAdv & SVOX\\
\lspbottomrule
\end{tabular} 
\end{table}\clearpage

{\sloppy\printbibliography[heading=subbibliography,notkeyword=this]}
\end{document}
